\documentclass[12pt, a0paper, landscape]{tikzposter}
\usepackage{anyfontsize}
\usepackage{hyperref}
\usepackage{fontspec}
\usepackage{luatexja-fontspec}

% Set font for English text.
\setmainfont{Noto Sans}

% Set font for Japanese text.
% Tested fonts, availability may be different depending on your system:
\setmainjfont{IPAexMincho}
% \setmainjfont{IPAexGothic}
% \setmainjfont{Noto Serif CJK JP} % Similar to IPAexMincho
% \setmainjfont{Noto Sans CJK JP} % Similar to IPAexGothic

\urlstyle{same}

\usetheme{Simple}

\newcommand{\Size}{2.05cm} % must match cell_size in list_kanji.py.
\tikzset{BaseStyle/.style={
  inner sep=0pt, text width=\Size, text height=\Size, minimum size=\Size}
}

\tikzset{Square/.style={BaseStyle, draw=lightgray, ultra thin, align=center}}

\tikzset{Kanji/.style={BaseStyle, font=\fontsize{36}{43}\selectfont, align=center}}

\tikzset{SmallGreyText/.style={BaseStyle, font=\fontsize{7}{8.2}\selectfont, text=gray}}
\tikzset{Onyomi/.style={SmallGreyText, align=left}}
\tikzset{Kunyomi/.style={SmallGreyText, align=right}}
\tikzset{Meaning/.style={SmallGreyText, align=center}}

\begin{document}

\node[Kanji] at (-56.000000, 40.500000) {\textcolor[HTML]{cd8268}{々}};
\node[Square] at (-56.000000, 40.000000) {};
\node[Onyomi] at (-55.950000, 40.100000) {\hbox{\tate ノマ}};
\node[Kunyomi] at (-56.050000, 40.100000) {\hbox{\tate のま}};
\node[Meaning] at (-56.000000, 41.750000) {repeater};
\node[Kanji] at (-53.950000, 40.500000) {\textcolor[HTML]{a11d25}{一}};
\node[Square] at (-53.950000, 40.000000) {};
\node[Onyomi] at (-53.900000, 40.100000) {\hbox{\tate イチ}};
\node[Kunyomi] at (-54.000000, 40.100000) {\hbox{\tate ひと.*}};
\node[Meaning] at (-53.950000, 41.750000) {one};
\node[Kanji] at (-51.900000, 40.500000) {\textcolor[HTML]{c36143}{二}};
\node[Square] at (-51.900000, 40.000000) {};
\node[Onyomi] at (-51.850000, 40.100000) {\hbox{\tate ニ}};
\node[Kunyomi] at (-51.950000, 40.100000) {\hbox{\tate ふた.*}};
\node[Meaning] at (-51.900000, 41.750000) {two};
\node[Kanji] at (-49.850000, 40.500000) {\textcolor[HTML]{c36143}{三}};
\node[Square] at (-49.850000, 40.000000) {};
\node[Onyomi] at (-49.800000, 40.100000) {\hbox{\tate サン}};
\node[Kunyomi] at (-49.900000, 40.100000) {\hbox{\tate み.*}};
\node[Meaning] at (-49.850000, 41.750000) {three};
\node[Kanji] at (-47.800000, 40.500000) {\textcolor[HTML]{d69f8d}{四}};
\node[Square] at (-47.800000, 40.000000) {};
\node[Onyomi] at (-47.750000, 40.100000) {\hbox{\tate シ}};
\node[Kunyomi] at (-47.850000, 40.100000) {\hbox{\tate よん・よ}};
\node[Meaning] at (-47.800000, 41.750000) {four};
\node[Kanji] at (-45.750000, 40.500000) {\textcolor[HTML]{d69f8d}{五}};
\node[Square] at (-45.750000, 40.000000) {};
\node[Onyomi] at (-45.700000, 40.100000) {\hbox{\tate ゴ}};
\node[Kunyomi] at (-45.800000, 40.100000) {\hbox{\tate いつ.つ}};
\node[Meaning] at (-45.750000, 41.750000) {five};
\node[Kanji] at (-43.700000, 40.500000) {\textcolor[HTML]{d2a293}{六}};
\node[Square] at (-43.700000, 40.000000) {};
\node[Onyomi] at (-43.650000, 40.100000) {\hbox{\tate ロク}};
\node[Kunyomi] at (-43.750000, 40.100000) {\hbox{\tate む.つ}};
\node[Meaning] at (-43.700000, 41.750000) {six};
\node[Kanji] at (-41.650000, 40.500000) {\textcolor[HTML]{c8a59d}{七}};
\node[Square] at (-41.650000, 40.000000) {};
\node[Onyomi] at (-41.600000, 40.100000) {\hbox{\tate シチ}};
\node[Kunyomi] at (-41.700000, 40.100000) {\hbox{\tate なな.*}};
\node[Meaning] at (-41.650000, 41.750000) {seven};
\node[Kanji] at (-39.600000, 40.500000) {\textcolor[HTML]{d69f8d}{八}};
\node[Square] at (-39.600000, 40.000000) {};
\node[Onyomi] at (-39.550000, 40.100000) {\hbox{\tate ハチ}};
\node[Kunyomi] at (-39.650000, 40.100000) {\hbox{\tate や.*}};
\node[Meaning] at (-39.600000, 41.750000) {eight};
\node[Kanji] at (-37.550000, 40.500000) {\textcolor[HTML]{d2a293}{九}};
\node[Square] at (-37.550000, 40.000000) {};
\node[Onyomi] at (-37.500000, 40.100000) {\hbox{\tate ク・キュウ}};
\node[Kunyomi] at (-37.600000, 40.100000) {\hbox{\tate ここの.*}};
\node[Meaning] at (-37.550000, 41.750000) {nine};
\node[Kanji] at (-35.500000, 40.500000) {\textcolor[HTML]{d69f8d}{十}};
\node[Square] at (-35.500000, 40.000000) {};
\node[Onyomi] at (-35.450000, 40.100000) {\hbox{\tate ジュウ}};
\node[Kunyomi] at (-35.550000, 40.100000) {\hbox{\tate とお.*}};
\node[Meaning] at (-35.500000, 41.750000) {ten};
\node[Kanji] at (-33.450000, 40.500000) {\textcolor[HTML]{cd8268}{口}};
\node[Square] at (-33.450000, 40.000000) {};
\node[Onyomi] at (-33.400000, 40.100000) {\hbox{\tate コウ・ク}};
\node[Kunyomi] at (-33.500000, 40.100000) {\hbox{\tate くち}};
\node[Meaning] at (-33.450000, 41.750000) {mouth};
\node[Kanji] at (-31.400000, 40.500000) {\textcolor[HTML]{5e0e21}{日}};
\node[Square] at (-31.400000, 40.000000) {};
\node[Onyomi] at (-31.350000, 40.100000) {\hbox{\tate ニチ・ジツ}};
\node[Kunyomi] at (-31.450000, 40.100000) {\hbox{\tate ひ・か・び}};
\node[Meaning] at (-31.400000, 41.750000) {sun};
\node[Kanji] at (-29.350000, 40.500000) {\textcolor[HTML]{830e29}{月}};
\node[Square] at (-29.350000, 40.000000) {};
\node[Onyomi] at (-29.300000, 40.100000) {\hbox{\tate ゲツ・ガツ}};
\node[Kunyomi] at (-29.400000, 40.100000) {\hbox{\tate つき}};
\node[Meaning] at (-29.350000, 41.750000) {moon};
\node[Kanji] at (-27.300000, 40.500000) {\textcolor[HTML]{b74029}{田}};
\node[Square] at (-27.300000, 40.000000) {};
\node[Onyomi] at (-27.250000, 40.100000) {\hbox{\tate デン}};
\node[Kunyomi] at (-27.350000, 40.100000) {\hbox{\tate た}};
\node[Meaning] at (-27.300000, 41.750000) {rice paddy};
\node[Kanji] at (-25.250000, 40.500000) {\textcolor[HTML]{b74029}{目}};
\node[Square] at (-25.250000, 40.000000) {};
\node[Onyomi] at (-25.200000, 40.100000) {\hbox{\tate モク}};
\node[Kunyomi] at (-25.300000, 40.100000) {\hbox{\tate め}};
\node[Meaning] at (-25.250000, 41.750000) {eye};
\node[Kanji] at (-23.200000, 40.500000) {\textcolor[HTML]{c36143}{古}};
\node[Square] at (-23.200000, 40.000000) {};
\node[Onyomi] at (-23.150000, 40.100000) {\hbox{\tate コ}};
\node[Kunyomi] at (-23.250000, 40.100000) {\hbox{\tate ふる.い}};
\node[Meaning] at (-23.200000, 41.750000) {old};
\node[Kanji] at (-21.150000, 40.500000) {\textcolor[HTML]{91b7c3}{吾}};
\node[Square] at (-21.150000, 40.000000) {};
\node[Onyomi] at (-21.100000, 40.100000) {\hbox{\tate ゴ}};
\node[Kunyomi] at (-21.200000, 40.100000) {\hbox{\tate わが}};
\node[Meaning] at (-21.150000, 41.750000) {i};
\node[Kanji] at (-19.100000, 40.500000) {\textcolor[HTML]{a3bac2}{冒}};
\node[Square] at (-19.100000, 40.000000) {};
\node[Onyomi] at (-19.050000, 40.100000) {\hbox{\tate ボウ}};
\node[Kunyomi] at (-19.150000, 40.100000) {\hbox{\tate おか.す}};
\node[Meaning] at (-19.100000, 41.750000) {dare};
\node[Kanji] at (-17.050000, 40.500000) {\textcolor[HTML]{c36143}{明}};
\node[Square] at (-17.050000, 40.000000) {};
\node[Onyomi] at (-17.000000, 40.100000) {\hbox{\tate メイ}};
\node[Kunyomi] at (-17.100000, 40.100000) {\hbox{\tate あ}};
\node[Meaning] at (-17.050000, 41.750000) {bright};
\node[Kanji] at (-15.000000, 40.500000) {\textcolor[HTML]{b0b0b5}{唱}};
\node[Square] at (-15.000000, 40.000000) {};
\node[Onyomi] at (-14.950000, 40.100000) {\hbox{\tate ショウ}};
\node[Kunyomi] at (-15.050000, 40.100000) {\hbox{\tate とな.える}};
\node[Meaning] at (-15.000000, 41.750000) {chant};
\node[Kanji] at (-12.950000, 40.500000) {\textcolor[HTML]{a3bac2}{晶}};
\node[Square] at (-12.950000, 40.000000) {};
\node[Onyomi] at (-12.900000, 40.100000) {\hbox{\tate ショウ}};
\node[Meaning] at (-12.950000, 41.750000) {crystal};
\node[Kanji] at (-10.900000, 40.500000) {\textcolor[HTML]{c36143}{品}};
\node[Square] at (-10.900000, 40.000000) {};
\node[Onyomi] at (-10.850000, 40.100000) {\hbox{\tate ヒン}};
\node[Kunyomi] at (-10.950000, 40.100000) {\hbox{\tate しな}};
\node[Meaning] at (-10.900000, 41.750000) {product};
\node[Kanji] at (-8.850000, 40.500000) {\textcolor[HTML]{a3bac2}{呂}};
\node[Square] at (-8.850000, 40.000000) {};
\node[Onyomi] at (-8.800000, 40.100000) {\hbox{\tate ロ・リョ}};
\node[Kunyomi] at (-8.900000, 40.100000) {\hbox{\tate せぼね}};
\node[Meaning] at (-8.850000, 41.750000) {bath};
\node[Kanji] at (-6.800000, 40.500000) {\textcolor[HTML]{a3bac2}{昌}};
\node[Square] at (-6.800000, 40.000000) {};
\node[Onyomi] at (-6.750000, 40.100000) {\hbox{\tate ショウ}};
\node[Kunyomi] at (-6.850000, 40.100000) {\hbox{\tate さかん}};
\node[Meaning] at (-6.800000, 41.750000) {prosperous};
\node[Kanji] at (-4.750000, 40.500000) {\textcolor[HTML]{d2a293}{早}};
\node[Square] at (-4.750000, 40.000000) {};
\node[Onyomi] at (-4.700000, 40.100000) {\hbox{\tate ソウ}};
\node[Kunyomi] at (-4.800000, 40.100000) {\hbox{\tate はや.い・さ}};
\node[Meaning] at (-4.750000, 41.750000) {early};
\node[Kanji] at (-2.700000, 40.500000) {\textcolor[HTML]{c36143}{世}};
\node[Square] at (-2.700000, 40.000000) {};
\node[Onyomi] at (-2.650000, 40.100000) {\hbox{\tate セ・セイ}};
\node[Kunyomi] at (-2.750000, 40.100000) {\hbox{\tate よ}};
\node[Meaning] at (-2.700000, 41.750000) {generation};
\node[Kanji] at (-0.650000, 40.500000) {\textcolor[HTML]{408dba}{胃}};
\node[Square] at (-0.650000, 40.000000) {};
\node[Onyomi] at (-0.600000, 40.100000) {\hbox{\tate イ}};
\node[Meaning] at (-0.650000, 41.750000) {stomach};
\node[Kanji] at (1.400000, 40.500000) {\textcolor[HTML]{91b7c3}{旦}};
\node[Square] at (1.400000, 40.000000) {};
\node[Onyomi] at (1.450000, 40.100000) {\hbox{\tate タン・ダン}};
\node[Kunyomi] at (1.350000, 40.100000) {\hbox{\tate あきら・あき}};
\node[Meaning] at (1.400000, 41.750000) {dawn};
\node[Kanji] at (3.450000, 40.500000) {\textcolor[HTML]{68a4bc}{胆}};
\node[Square] at (3.450000, 40.000000) {};
\node[Onyomi] at (3.500000, 40.100000) {\hbox{\tate タン}};
\node[Kunyomi] at (3.400000, 40.100000) {\hbox{\tate きも}};
\node[Meaning] at (3.450000, 41.750000) {guts};
\node[Kanji] at (5.500000, 40.500000) {\textcolor[HTML]{1e76bb}{凹}};
\node[Square] at (5.500000, 40.000000) {};
\node[Onyomi] at (5.550000, 40.100000) {\hbox{\tate オウ}};
\node[Kunyomi] at (5.450000, 40.100000) {\hbox{\tate くぼ.む}};
\node[Meaning] at (5.500000, 41.750000) {concave};
\node[Kanji] at (7.550000, 40.500000) {\textcolor[HTML]{1e76bb}{凸}};
\node[Square] at (7.550000, 40.000000) {};
\node[Onyomi] at (7.600000, 40.100000) {\hbox{\tate トツ}};
\node[Kunyomi] at (7.500000, 40.100000) {\hbox{\tate でこ}};
\node[Meaning] at (7.550000, 41.750000) {convex};
\node[Kanji] at (9.600000, 40.500000) {\textcolor[HTML]{d69f8d}{旧}};
\node[Square] at (9.600000, 40.000000) {};
\node[Onyomi] at (9.650000, 40.100000) {\hbox{\tate キュウ}};
\node[Meaning] at (9.600000, 41.750000) {former};
\node[Kanji] at (11.650000, 40.500000) {\textcolor[HTML]{b74029}{自}};
\node[Square] at (11.650000, 40.000000) {};
\node[Onyomi] at (11.700000, 40.100000) {\hbox{\tate ジ・シ}};
\node[Meaning] at (11.650000, 41.750000) {self};
\node[Kanji] at (13.700000, 40.500000) {\textcolor[HTML]{d69f8d}{白}};
\node[Square] at (13.700000, 40.000000) {};
\node[Onyomi] at (13.750000, 40.100000) {\hbox{\tate ハク}};
\node[Kunyomi] at (13.650000, 40.100000) {\hbox{\tate しろ・しら}};
\node[Meaning] at (13.700000, 41.750000) {white};
\node[Kanji] at (15.750000, 40.500000) {\textcolor[HTML]{d2a293}{百}};
\node[Square] at (15.750000, 40.000000) {};
\node[Onyomi] at (15.800000, 40.100000) {\hbox{\tate ヒャク}};
\node[Meaning] at (15.750000, 41.750000) {hundred};
\node[Kanji] at (17.800000, 40.500000) {\textcolor[HTML]{830e29}{中}};
\node[Square] at (17.800000, 40.000000) {};
\node[Onyomi] at (17.850000, 40.100000) {\hbox{\tate チュウ}};
\node[Kunyomi] at (17.750000, 40.100000) {\hbox{\tate なか}};
\node[Meaning] at (17.800000, 41.750000) {middle};
\node[Kanji] at (19.850000, 40.500000) {\textcolor[HTML]{d69f8d}{千}};
\node[Square] at (19.850000, 40.000000) {};
\node[Onyomi] at (19.900000, 40.100000) {\hbox{\tate セン}};
\node[Kunyomi] at (19.800000, 40.100000) {\hbox{\tate ち}};
\node[Meaning] at (19.850000, 41.750000) {thousand};
\node[Kanji] at (21.900000, 40.500000) {\textcolor[HTML]{68a4bc}{舌}};
\node[Square] at (21.900000, 40.000000) {};
\node[Onyomi] at (21.950000, 40.100000) {\hbox{\tate ゼツ}};
\node[Kunyomi] at (21.850000, 40.100000) {\hbox{\tate した}};
\node[Meaning] at (21.900000, 41.750000) {tongue};
\node[Kanji] at (23.950000, 40.500000) {\textcolor[HTML]{1e76bb}{升}};
\node[Square] at (23.950000, 40.000000) {};
\node[Onyomi] at (24.000000, 40.100000) {\hbox{\tate ショウ}};
\node[Kunyomi] at (23.900000, 40.100000) {\hbox{\tate ます}};
\node[Meaning] at (23.950000, 41.750000) {grid};
\node[Kanji] at (26.000000, 40.500000) {\textcolor[HTML]{c8a59d}{昇}};
\node[Square] at (26.000000, 40.000000) {};
\node[Onyomi] at (26.050000, 40.100000) {\hbox{\tate ショウ}};
\node[Kunyomi] at (25.950000, 40.100000) {\hbox{\tate のぼ.る}};
\node[Meaning] at (26.000000, 41.750000) {ascend};
\node[Kanji] at (28.050000, 40.500000) {\textcolor[HTML]{d2a293}{丸}};
\node[Square] at (28.050000, 40.000000) {};
\node[Onyomi] at (28.100000, 40.100000) {\hbox{\tate ガン}};
\node[Kunyomi] at (28.000000, 40.100000) {\hbox{\tate まる}};
\node[Meaning] at (28.050000, 41.750000) {circle};
\node[Kanji] at (30.100000, 40.500000) {\textcolor[HTML]{91b7c3}{寸}};
\node[Square] at (30.100000, 40.000000) {};
\node[Onyomi] at (30.150000, 40.100000) {\hbox{\tate スン}};
\node[Meaning] at (30.100000, 41.750000) {measurement};
\node[Kanji] at (32.150000, 40.500000) {\textcolor[HTML]{d69f8d}{専}};
\node[Square] at (32.150000, 40.000000) {};
\node[Onyomi] at (32.200000, 40.100000) {\hbox{\tate セン}};
\node[Kunyomi] at (32.100000, 40.100000) {\hbox{\tate もっぱ.ら}};
\node[Meaning] at (32.150000, 41.750000) {specialty};
\node[Kanji] at (34.200000, 40.500000) {\textcolor[HTML]{d2a293}{博}};
\node[Square] at (34.200000, 40.000000) {};
\node[Onyomi] at (34.250000, 40.100000) {\hbox{\tate ハク・バク}};
\node[Meaning] at (34.200000, 41.750000) {exhibition};
\node[Kanji] at (36.250000, 40.500000) {\textcolor[HTML]{c8a59d}{占}};
\node[Square] at (36.250000, 40.000000) {};
\node[Onyomi] at (36.300000, 40.100000) {\hbox{\tate セン}};
\node[Kunyomi] at (36.200000, 40.100000) {\hbox{\tate うらな.い}};
\node[Meaning] at (36.250000, 41.750000) {fortune};
\node[Kanji] at (38.300000, 40.500000) {\textcolor[HTML]{a11d25}{上}};
\node[Square] at (38.300000, 40.000000) {};
\node[Onyomi] at (38.350000, 40.100000) {\hbox{\tate ジョウ}};
\node[Kunyomi] at (38.250000, 40.100000) {\hbox{\tate うえ・あ}};
\node[Meaning] at (38.300000, 41.750000) {above};
\node[Kanji] at (40.350000, 40.500000) {\textcolor[HTML]{b74029}{下}};
\node[Square] at (40.350000, 40.000000) {};
\node[Onyomi] at (40.400000, 40.100000) {\hbox{\tate カ・ゲ}};
\node[Kunyomi] at (40.300000, 40.100000) {\hbox{\tate した・さが}};
\node[Meaning] at (40.350000, 41.750000) {below};
\node[Kanji] at (42.400000, 40.500000) {\textcolor[HTML]{91b7c3}{卓}};
\node[Square] at (42.400000, 40.000000) {};
\node[Onyomi] at (42.450000, 40.100000) {\hbox{\tate タク}};
\node[Meaning] at (42.400000, 41.750000) {table};
\node[Kanji] at (44.450000, 40.500000) {\textcolor[HTML]{cd8268}{朝}};
\node[Square] at (44.450000, 40.000000) {};
\node[Onyomi] at (44.500000, 40.100000) {\hbox{\tate チョウ}};
\node[Kunyomi] at (44.400000, 40.100000) {\hbox{\tate あさ}};
\node[Meaning] at (44.450000, 41.750000) {morning};
\node[Kanji] at (46.500000, 40.500000) {\textcolor[HTML]{a3bac2}{貝}};
\node[Square] at (46.500000, 40.000000) {};
\node[Kunyomi] at (46.450000, 40.100000) {\hbox{\tate かい}};
\node[Meaning] at (46.500000, 41.750000) {shellfish};
\node[Kanji] at (48.550000, 40.500000) {\textcolor[HTML]{b0b0b5}{貞}};
\node[Square] at (48.550000, 40.000000) {};
\node[Onyomi] at (48.600000, 40.100000) {\hbox{\tate テイ}};
\node[Kunyomi] at (48.500000, 40.100000) {\hbox{\tate さだ}};
\node[Meaning] at (48.550000, 41.750000) {chastity};
\node[Kanji] at (50.600000, 40.500000) {\textcolor[HTML]{c36143}{員}};
\node[Square] at (50.600000, 40.000000) {};
\node[Onyomi] at (50.650000, 40.100000) {\hbox{\tate イン}};
\node[Meaning] at (50.600000, 41.750000) {member};
\node[Kanji] at (52.650000, 40.500000) {\textcolor[HTML]{b74029}{見}};
\node[Square] at (52.650000, 40.000000) {};
\node[Onyomi] at (52.700000, 40.100000) {\hbox{\tate ケン}};
\node[Kunyomi] at (52.600000, 40.100000) {\hbox{\tate み}};
\node[Meaning] at (52.650000, 41.750000) {see};
\node[Kanji] at (54.700000, 40.500000) {\textcolor[HTML]{d2a293}{児}};
\node[Square] at (54.700000, 40.000000) {};
\node[Onyomi] at (54.750000, 40.100000) {\hbox{\tate ジ}};
\node[Kunyomi] at (54.650000, 40.100000) {\hbox{\tate こ}};
\node[Meaning] at (54.700000, 41.750000) {child};
\node[Kanji] at (56.750000, 40.500000) {\textcolor[HTML]{c36143}{元}};
\node[Square] at (56.750000, 40.000000) {};
\node[Onyomi] at (56.800000, 40.100000) {\hbox{\tate ゲン・ガン}};
\node[Kunyomi] at (56.700000, 40.100000) {\hbox{\tate もと}};
\node[Meaning] at (56.750000, 41.750000) {origin};
\node[Meaning] at (-58.050000, 40.600000) {7.40\%};
\node[Kanji] at (-56.000000, 38.450000) {\textcolor[HTML]{68a4bc}{頑}};
\node[Square] at (-56.000000, 37.950000) {};
\node[Onyomi] at (-55.950000, 38.050000) {\hbox{\tate ガン}};
\node[Meaning] at (-56.000000, 39.700000) {stubborn};
\node[Kanji] at (-53.950000, 38.450000) {\textcolor[HTML]{91b7c3}{凡}};
\node[Square] at (-53.950000, 37.950000) {};
\node[Onyomi] at (-53.900000, 38.050000) {\hbox{\tate ボン・ハン}};
\node[Kunyomi] at (-54.000000, 38.050000) {\hbox{\tate おうよ.そ}};
\node[Meaning] at (-53.950000, 39.700000) {mediocre};
\node[Kanji] at (-51.900000, 38.450000) {\textcolor[HTML]{d2a293}{負}};
\node[Square] at (-51.900000, 37.950000) {};
\node[Onyomi] at (-51.850000, 38.050000) {\hbox{\tate フ}};
\node[Kunyomi] at (-51.950000, 38.050000) {\hbox{\tate ま.ける}};
\node[Meaning] at (-51.900000, 39.700000) {lose};
\node[Kanji] at (-49.850000, 38.450000) {\textcolor[HTML]{d69f8d}{万}};
\node[Square] at (-49.850000, 37.950000) {};
\node[Onyomi] at (-49.800000, 38.050000) {\hbox{\tate マン・バン}};
\node[Meaning] at (-49.850000, 39.700000) {ten thousand};
\node[Kanji] at (-47.800000, 38.450000) {\textcolor[HTML]{a3bac2}{句}};
\node[Square] at (-47.800000, 37.950000) {};
\node[Onyomi] at (-47.750000, 38.050000) {\hbox{\tate ク}};
\node[Meaning] at (-47.800000, 39.700000) {paragraph};
\node[Kanji] at (-45.750000, 38.450000) {\textcolor[HTML]{408dba}{肌}};
\node[Square] at (-45.750000, 37.950000) {};
\node[Kunyomi] at (-45.800000, 38.050000) {\hbox{\tate はだ}};
\node[Meaning] at (-45.750000, 39.700000) {skin};
\node[Kanji] at (-43.700000, 38.450000) {\textcolor[HTML]{a3bac2}{旬}};
\node[Square] at (-43.700000, 37.950000) {};
\node[Onyomi] at (-43.650000, 38.050000) {\hbox{\tate シュン}};
\node[Meaning] at (-43.700000, 39.700000) {in season};
\node[Kanji] at (-41.650000, 38.450000) {\textcolor[HTML]{a11d25}{的}};
\node[Square] at (-41.650000, 37.950000) {};
\node[Onyomi] at (-41.600000, 38.050000) {\hbox{\tate テキ}};
\node[Kunyomi] at (-41.700000, 38.050000) {\hbox{\tate まと}};
\node[Meaning] at (-41.650000, 39.700000) {target};
\node[Kanji] at (-39.600000, 38.450000) {\textcolor[HTML]{d2a293}{首}};
\node[Square] at (-39.600000, 37.950000) {};
\node[Onyomi] at (-39.550000, 38.050000) {\hbox{\tate シュ}};
\node[Kunyomi] at (-39.650000, 38.050000) {\hbox{\tate くび}};
\node[Meaning] at (-39.600000, 39.700000) {neck};
\node[Kanji] at (-37.550000, 38.450000) {\textcolor[HTML]{91b7c3}{乙}};
\node[Square] at (-37.550000, 37.950000) {};
\node[Onyomi] at (-37.500000, 38.050000) {\hbox{\tate オツ・イツ}};
\node[Kunyomi] at (-37.600000, 38.050000) {\hbox{\tate おと・きのと}};
\node[Meaning] at (-37.550000, 39.700000) {latter};
\node[Kanji] at (-35.500000, 38.450000) {\textcolor[HTML]{d2a293}{乱}};
\node[Square] at (-35.500000, 37.950000) {};
\node[Onyomi] at (-35.450000, 38.050000) {\hbox{\tate ラン}};
\node[Kunyomi] at (-35.550000, 38.050000) {\hbox{\tate みだ.す}};
\node[Meaning] at (-35.500000, 39.700000) {riot};
\node[Kanji] at (-33.450000, 38.450000) {\textcolor[HTML]{cd8268}{直}};
\node[Square] at (-33.450000, 37.950000) {};
\node[Onyomi] at (-33.400000, 38.050000) {\hbox{\tate チョク・ジキ}};
\node[Kunyomi] at (-33.500000, 38.050000) {\hbox{\tate なお.す}};
\node[Meaning] at (-33.450000, 39.700000) {fix};
\node[Kanji] at (-31.400000, 38.450000) {\textcolor[HTML]{d2a293}{具}};
\node[Square] at (-31.400000, 37.950000) {};
\node[Onyomi] at (-31.350000, 38.050000) {\hbox{\tate グ}};
\node[Meaning] at (-31.400000, 39.700000) {tool};
\node[Kanji] at (-29.350000, 38.450000) {\textcolor[HTML]{d69f8d}{真}};
\node[Square] at (-29.350000, 37.950000) {};
\node[Onyomi] at (-29.300000, 38.050000) {\hbox{\tate シン}};
\node[Kunyomi] at (-29.400000, 38.050000) {\hbox{\tate ま}};
\node[Meaning] at (-29.350000, 39.700000) {reality};
\node[Kanji] at (-27.300000, 38.450000) {\textcolor[HTML]{cd8268}{工}};
\node[Square] at (-27.300000, 37.950000) {};
\node[Onyomi] at (-27.250000, 38.050000) {\hbox{\tate コウ}};
\node[Meaning] at (-27.300000, 39.700000) {industry};
\node[Kanji] at (-25.250000, 38.450000) {\textcolor[HTML]{d2a293}{左}};
\node[Square] at (-25.250000, 37.950000) {};
\node[Onyomi] at (-25.200000, 38.050000) {\hbox{\tate サ}};
\node[Kunyomi] at (-25.300000, 38.050000) {\hbox{\tate ひだり}};
\node[Meaning] at (-25.250000, 39.700000) {left};
\node[Kanji] at (-23.200000, 38.450000) {\textcolor[HTML]{d2a293}{右}};
\node[Square] at (-23.200000, 37.950000) {};
\node[Onyomi] at (-23.150000, 38.050000) {\hbox{\tate ウ・ユウ}};
\node[Kunyomi] at (-23.250000, 38.050000) {\hbox{\tate みぎ}};
\node[Meaning] at (-23.200000, 39.700000) {right};
\node[Kanji] at (-21.150000, 38.450000) {\textcolor[HTML]{c36143}{有}};
\node[Square] at (-21.150000, 37.950000) {};
\node[Onyomi] at (-21.100000, 38.050000) {\hbox{\tate ユウ・ウ}};
\node[Kunyomi] at (-21.200000, 38.050000) {\hbox{\tate あ.る}};
\node[Meaning] at (-21.150000, 39.700000) {have};
\node[Kanji] at (-19.100000, 38.450000) {\textcolor[HTML]{408dba}{賄}};
\node[Square] at (-19.100000, 37.950000) {};
\node[Onyomi] at (-19.050000, 38.050000) {\hbox{\tate ワイ}};
\node[Kunyomi] at (-19.150000, 38.050000) {\hbox{\tate まかな.う}};
\node[Meaning] at (-19.100000, 39.700000) {bribe};
\node[Kanji] at (-17.050000, 38.450000) {\textcolor[HTML]{b0b0b5}{貢}};
\node[Square] at (-17.050000, 37.950000) {};
\node[Onyomi] at (-17.000000, 38.050000) {\hbox{\tate コウ}};
\node[Kunyomi] at (-17.100000, 38.050000) {\hbox{\tate みつ.ぐ}};
\node[Meaning] at (-17.050000, 39.700000) {tribute};
\node[Kanji] at (-15.000000, 38.450000) {\textcolor[HTML]{d2a293}{項}};
\node[Square] at (-15.000000, 37.950000) {};
\node[Onyomi] at (-14.950000, 38.050000) {\hbox{\tate コウ}};
\node[Meaning] at (-15.000000, 39.700000) {paragraph};
\node[Kanji] at (-12.950000, 38.450000) {\textcolor[HTML]{b0b0b5}{刀}};
\node[Square] at (-12.950000, 37.950000) {};
\node[Onyomi] at (-12.900000, 38.050000) {\hbox{\tate トウ}};
\node[Kunyomi] at (-13.000000, 38.050000) {\hbox{\tate かたな}};
\node[Meaning] at (-12.950000, 39.700000) {sword};
\node[Kanji] at (-10.900000, 38.450000) {\textcolor[HTML]{68a4bc}{刃}};
\node[Square] at (-10.900000, 37.950000) {};
\node[Onyomi] at (-10.850000, 38.050000) {\hbox{\tate ジン}};
\node[Kunyomi] at (-10.950000, 38.050000) {\hbox{\tate は}};
\node[Meaning] at (-10.900000, 39.700000) {blade};
\node[Kanji] at (-8.850000, 38.450000) {\textcolor[HTML]{d69f8d}{切}};
\node[Square] at (-8.850000, 37.950000) {};
\node[Onyomi] at (-8.800000, 38.050000) {\hbox{\tate セツ}};
\node[Kunyomi] at (-8.900000, 38.050000) {\hbox{\tate き.る}};
\node[Meaning] at (-8.850000, 39.700000) {cut};
\node[Kanji] at (-6.800000, 38.450000) {\textcolor[HTML]{91b7c3}{召}};
\node[Square] at (-6.800000, 37.950000) {};
\node[Onyomi] at (-6.750000, 38.050000) {\hbox{\tate ショウ}};
\node[Kunyomi] at (-6.850000, 38.050000) {\hbox{\tate め.す}};
\node[Meaning] at (-6.800000, 39.700000) {call};
\node[Kanji] at (-4.750000, 38.450000) {\textcolor[HTML]{cd8268}{昭}};
\node[Square] at (-4.750000, 37.950000) {};
\node[Onyomi] at (-4.700000, 38.050000) {\hbox{\tate ショウ}};
\node[Meaning] at (-4.750000, 39.700000) {shining};
\node[Kanji] at (-2.700000, 38.450000) {\textcolor[HTML]{d2a293}{則}};
\node[Square] at (-2.700000, 37.950000) {};
\node[Onyomi] at (-2.650000, 38.050000) {\hbox{\tate ソク}};
\node[Kunyomi] at (-2.750000, 38.050000) {\hbox{\tate のっと.る}};
\node[Meaning] at (-2.700000, 39.700000) {rule};
\node[Kanji] at (-0.650000, 38.450000) {\textcolor[HTML]{c8a59d}{副}};
\node[Square] at (-0.650000, 37.950000) {};
\node[Onyomi] at (-0.600000, 38.050000) {\hbox{\tate フク}};
\node[Meaning] at (-0.650000, 39.700000) {vice};
\node[Kanji] at (1.400000, 38.450000) {\textcolor[HTML]{cd8268}{別}};
\node[Square] at (1.400000, 37.950000) {};
\node[Onyomi] at (1.450000, 38.050000) {\hbox{\tate ベツ}};
\node[Kunyomi] at (1.350000, 38.050000) {\hbox{\tate わか.*}};
\node[Meaning] at (1.400000, 39.700000) {separate};
\node[Kanji] at (3.450000, 38.450000) {\textcolor[HTML]{c8a59d}{丁}};
\node[Square] at (3.450000, 37.950000) {};
\node[Onyomi] at (3.500000, 38.050000) {\hbox{\tate チョウ}};
\node[Meaning] at (3.450000, 39.700000) {street};
\node[Kanji] at (5.500000, 38.450000) {\textcolor[HTML]{c36143}{町}};
\node[Square] at (5.500000, 37.950000) {};
\node[Onyomi] at (5.550000, 38.050000) {\hbox{\tate チョウ}};
\node[Kunyomi] at (5.450000, 38.050000) {\hbox{\tate まち}};
\node[Meaning] at (5.500000, 39.700000) {town};
\node[Kanji] at (7.550000, 38.450000) {\textcolor[HTML]{cd8268}{可}};
\node[Square] at (7.550000, 37.950000) {};
\node[Onyomi] at (7.600000, 38.050000) {\hbox{\tate カ}};
\node[Meaning] at (7.550000, 39.700000) {possible};
\node[Kanji] at (9.600000, 38.450000) {\textcolor[HTML]{b0b0b5}{頂}};
\node[Square] at (9.600000, 37.950000) {};
\node[Onyomi] at (9.650000, 38.050000) {\hbox{\tate チョウ}};
\node[Kunyomi] at (9.550000, 38.050000) {\hbox{\tate いただき}};
\node[Meaning] at (9.600000, 39.700000) {summit};
\node[Kanji] at (11.650000, 38.450000) {\textcolor[HTML]{b74029}{子}};
\node[Square] at (11.650000, 37.950000) {};
\node[Onyomi] at (11.700000, 38.050000) {\hbox{\tate シ・ス}};
\node[Kunyomi] at (11.600000, 38.050000) {\hbox{\tate こ}};
\node[Meaning] at (11.650000, 39.700000) {child};
\node[Kanji] at (13.700000, 38.450000) {\textcolor[HTML]{91b7c3}{孔}};
\node[Square] at (13.700000, 37.950000) {};
\node[Onyomi] at (13.750000, 38.050000) {\hbox{\tate コウ}};
\node[Kunyomi] at (13.650000, 38.050000) {\hbox{\tate あな}};
\node[Meaning] at (13.700000, 39.700000) {cavity};
\node[Kanji] at (15.750000, 38.450000) {\textcolor[HTML]{d2a293}{了}};
\node[Square] at (15.750000, 37.950000) {};
\node[Onyomi] at (15.800000, 38.050000) {\hbox{\tate リョウ}};
\node[Meaning] at (15.750000, 39.700000) {finish};
\node[Kanji] at (17.800000, 38.450000) {\textcolor[HTML]{c36143}{女}};
\node[Square] at (17.800000, 37.950000) {};
\node[Onyomi] at (17.850000, 38.050000) {\hbox{\tate ジョ}};
\node[Kunyomi] at (17.750000, 38.050000) {\hbox{\tate おんな・め}};
\node[Meaning] at (17.800000, 39.700000) {woman};
\node[Kanji] at (19.850000, 38.450000) {\textcolor[HTML]{d2a293}{好}};
\node[Square] at (19.850000, 37.950000) {};
\node[Onyomi] at (19.900000, 38.050000) {\hbox{\tate コウ}};
\node[Kunyomi] at (19.800000, 38.050000) {\hbox{\tate す.き}};
\node[Meaning] at (19.850000, 39.700000) {like};
\node[Kanji] at (21.900000, 38.450000) {\textcolor[HTML]{a3bac2}{如}};
\node[Square] at (21.900000, 37.950000) {};
\node[Onyomi] at (21.950000, 38.050000) {\hbox{\tate ジョ}};
\node[Kunyomi] at (21.850000, 38.050000) {\hbox{\tate ごと.し}};
\node[Meaning] at (21.900000, 39.700000) {likeness};
\node[Kanji] at (23.950000, 38.450000) {\textcolor[HTML]{d69f8d}{母}};
\node[Square] at (23.950000, 37.950000) {};
\node[Onyomi] at (24.000000, 38.050000) {\hbox{\tate ボ}};
\node[Kunyomi] at (23.900000, 38.050000) {\hbox{\tate はは}};
\node[Meaning] at (23.950000, 39.700000) {mother};
\node[Kanji] at (26.000000, 38.450000) {\textcolor[HTML]{b0b0b5}{貫}};
\node[Square] at (26.000000, 37.950000) {};
\node[Onyomi] at (26.050000, 38.050000) {\hbox{\tate カン}};
\node[Kunyomi] at (25.950000, 38.050000) {\hbox{\tate つらぬ・ぬき}};
\node[Meaning] at (26.000000, 39.700000) {pierce};
\node[Kanji] at (28.050000, 38.450000) {\textcolor[HTML]{c8a59d}{兄}};
\node[Square] at (28.050000, 37.950000) {};
\node[Onyomi] at (28.100000, 38.050000) {\hbox{\tate キョウ}};
\node[Kunyomi] at (28.000000, 38.050000) {\hbox{\tate あに}};
\node[Meaning] at (28.050000, 39.700000) {older brother};
\node[Kanji] at (30.100000, 38.450000) {\textcolor[HTML]{91b7c3}{克}};
\node[Square] at (30.100000, 37.950000) {};
\node[Onyomi] at (30.150000, 38.050000) {\hbox{\tate コク}};
\node[Meaning] at (30.100000, 39.700000) {overcome};
\node[Kanji] at (32.150000, 38.450000) {\textcolor[HTML]{b74029}{小}};
\node[Square] at (32.150000, 37.950000) {};
\node[Onyomi] at (32.200000, 38.050000) {\hbox{\tate ショウ}};
\node[Kunyomi] at (32.100000, 38.050000) {\hbox{\tate ちい.さい}};
\node[Meaning] at (32.150000, 39.700000) {small};
\node[Kanji] at (34.200000, 38.450000) {\textcolor[HTML]{cd8268}{少}};
\node[Square] at (34.200000, 37.950000) {};
\node[Onyomi] at (34.250000, 38.050000) {\hbox{\tate ショウ}};
\node[Kunyomi] at (34.150000, 38.050000) {\hbox{\tate すこ.し}};
\node[Meaning] at (34.200000, 39.700000) {few};
\node[Kanji] at (36.250000, 38.450000) {\textcolor[HTML]{830e29}{大}};
\node[Square] at (36.250000, 37.950000) {};
\node[Onyomi] at (36.300000, 38.050000) {\hbox{\tate タイ・ダイ}};
\node[Kunyomi] at (36.200000, 38.050000) {\hbox{\tate おお}};
\node[Meaning] at (36.250000, 39.700000) {big};
\node[Kanji] at (38.300000, 38.450000) {\textcolor[HTML]{b74029}{多}};
\node[Square] at (38.300000, 37.950000) {};
\node[Onyomi] at (38.350000, 38.050000) {\hbox{\tate タ}};
\node[Kunyomi] at (38.250000, 38.050000) {\hbox{\tate おお.い}};
\node[Meaning] at (38.300000, 39.700000) {many};
\node[Kanji] at (40.350000, 38.450000) {\textcolor[HTML]{b0b0b5}{夕}};
\node[Square] at (40.350000, 37.950000) {};
\node[Onyomi] at (40.400000, 38.050000) {\hbox{\tate セキ}};
\node[Kunyomi] at (40.300000, 38.050000) {\hbox{\tate ゆう}};
\node[Meaning] at (40.350000, 39.700000) {evening};
\node[Kanji] at (42.400000, 38.450000) {\textcolor[HTML]{c36143}{外}};
\node[Square] at (42.400000, 37.950000) {};
\node[Onyomi] at (42.450000, 38.050000) {\hbox{\tate ガイ}};
\node[Kunyomi] at (42.350000, 38.050000) {\hbox{\tate そと・はず}};
\node[Meaning] at (42.400000, 39.700000) {outside};
\node[Kanji] at (44.450000, 38.450000) {\textcolor[HTML]{a11d25}{名}};
\node[Square] at (44.450000, 37.950000) {};
\node[Onyomi] at (44.500000, 38.050000) {\hbox{\tate メイ・ミョウ}};
\node[Kunyomi] at (44.400000, 38.050000) {\hbox{\tate な}};
\node[Meaning] at (44.450000, 39.700000) {name};
\node[Kanji] at (46.500000, 38.450000) {\textcolor[HTML]{cd8268}{石}};
\node[Square] at (46.500000, 37.950000) {};
\node[Onyomi] at (46.550000, 38.050000) {\hbox{\tate セキ}};
\node[Kunyomi] at (46.450000, 38.050000) {\hbox{\tate いし}};
\node[Meaning] at (46.500000, 39.700000) {stone};
\node[Kanji] at (48.550000, 38.450000) {\textcolor[HTML]{68a4bc}{肖}};
\node[Square] at (48.550000, 37.950000) {};
\node[Onyomi] at (48.600000, 38.050000) {\hbox{\tate ショウ}};
\node[Kunyomi] at (48.500000, 38.050000) {\hbox{\tate あやか}};
\node[Meaning] at (48.550000, 39.700000) {resemblance};
\node[Kanji] at (50.600000, 38.450000) {\textcolor[HTML]{408dba}{硝}};
\node[Square] at (50.600000, 37.950000) {};
\node[Onyomi] at (50.650000, 38.050000) {\hbox{\tate ショウ}};
\node[Meaning] at (50.600000, 39.700000) {nitrate};
\node[Kanji] at (52.650000, 38.450000) {\textcolor[HTML]{68a4bc}{砕}};
\node[Square] at (52.650000, 37.950000) {};
\node[Onyomi] at (52.700000, 38.050000) {\hbox{\tate サイ}};
\node[Kunyomi] at (52.600000, 38.050000) {\hbox{\tate くだ.*}};
\node[Meaning] at (52.650000, 39.700000) {smash};
\node[Kanji] at (54.700000, 38.450000) {\textcolor[HTML]{b0b0b5}{砂}};
\node[Square] at (54.700000, 37.950000) {};
\node[Onyomi] at (54.750000, 38.050000) {\hbox{\tate サ}};
\node[Kunyomi] at (54.650000, 38.050000) {\hbox{\tate すな}};
\node[Meaning] at (54.700000, 39.700000) {sand};
\node[Kanji] at (56.750000, 38.450000) {\textcolor[HTML]{b74029}{削}};
\node[Square] at (56.750000, 37.950000) {};
\node[Onyomi] at (56.800000, 38.050000) {\hbox{\tate サク}};
\node[Kunyomi] at (56.700000, 38.050000) {\hbox{\tate けず.る}};
\node[Meaning] at (56.750000, 39.700000) {whittle down};
\node[Meaning] at (-58.050000, 38.550000) {12.38\%};
\node[Kanji] at (-56.000000, 36.400000) {\textcolor[HTML]{cd8268}{光}};
\node[Square] at (-56.000000, 35.900000) {};
\node[Onyomi] at (-55.950000, 36.000000) {\hbox{\tate コウ}};
\node[Kunyomi] at (-56.050000, 36.000000) {\hbox{\tate ひかり}};
\node[Meaning] at (-56.000000, 37.650000) {sunlight};
\node[Kanji] at (-53.950000, 36.400000) {\textcolor[HTML]{cd8268}{太}};
\node[Square] at (-53.950000, 35.900000) {};
\node[Onyomi] at (-53.900000, 36.000000) {\hbox{\tate タイ}};
\node[Kunyomi] at (-54.000000, 36.000000) {\hbox{\tate ふと.い}};
\node[Meaning] at (-53.950000, 37.650000) {fat};
\node[Kanji] at (-51.900000, 36.400000) {\textcolor[HTML]{d69f8d}{器}};
\node[Square] at (-51.900000, 35.900000) {};
\node[Onyomi] at (-51.850000, 36.000000) {\hbox{\tate キ}};
\node[Kunyomi] at (-51.950000, 36.000000) {\hbox{\tate うつわ}};
\node[Meaning] at (-51.900000, 37.650000) {container};
\node[Kanji] at (-49.850000, 36.400000) {\textcolor[HTML]{68a4bc}{臭}};
\node[Square] at (-49.850000, 35.900000) {};
\node[Onyomi] at (-49.800000, 36.000000) {\hbox{\tate シュウ}};
\node[Kunyomi] at (-49.900000, 36.000000) {\hbox{\tate くさ}};
\node[Meaning] at (-49.850000, 37.650000) {stinking};
\node[Kanji] at (-47.800000, 36.400000) {\textcolor[HTML]{a3bac2}{妙}};
\node[Square] at (-47.800000, 35.900000) {};
\node[Onyomi] at (-47.750000, 36.000000) {\hbox{\tate ミョウ}};
\node[Kunyomi] at (-47.850000, 36.000000) {\hbox{\tate たえ.なる}};
\node[Meaning] at (-47.800000, 37.650000) {strange};
\node[Kanji] at (-45.750000, 36.400000) {\textcolor[HTML]{d69f8d}{省}};
\node[Square] at (-45.750000, 35.900000) {};
\node[Onyomi] at (-45.700000, 36.000000) {\hbox{\tate ショウ・セイ}};
\node[Kunyomi] at (-45.800000, 36.000000) {\hbox{\tate はぶ.く}};
\node[Meaning] at (-45.750000, 37.650000) {conserve};
\node[Kanji] at (-43.700000, 36.400000) {\textcolor[HTML]{b0b0b5}{厚}};
\node[Square] at (-43.700000, 35.900000) {};
\node[Onyomi] at (-43.650000, 36.000000) {\hbox{\tate コウ}};
\node[Kunyomi] at (-43.750000, 36.000000) {\hbox{\tate あつ}};
\node[Meaning] at (-43.700000, 37.650000) {thick};
\node[Kanji] at (-41.650000, 36.400000) {\textcolor[HTML]{b0b0b5}{奇}};
\node[Square] at (-41.650000, 35.900000) {};
\node[Onyomi] at (-41.600000, 36.000000) {\hbox{\tate キ}};
\node[Meaning] at (-41.650000, 37.650000) {odd};
\node[Kanji] at (-39.600000, 36.400000) {\textcolor[HTML]{b74029}{川}};
\node[Square] at (-39.600000, 35.900000) {};
\node[Onyomi] at (-39.550000, 36.000000) {\hbox{\tate セン}};
\node[Kunyomi] at (-39.650000, 36.000000) {\hbox{\tate かわ}};
\node[Meaning] at (-39.600000, 37.650000) {river};
\node[Kanji] at (-37.550000, 36.400000) {\textcolor[HTML]{cd8268}{州}};
\node[Square] at (-37.550000, 35.900000) {};
\node[Onyomi] at (-37.500000, 36.000000) {\hbox{\tate シュウ}};
\node[Meaning] at (-37.550000, 37.650000) {state};
\node[Kanji] at (-35.500000, 36.400000) {\textcolor[HTML]{d2a293}{順}};
\node[Square] at (-35.500000, 35.900000) {};
\node[Onyomi] at (-35.450000, 36.000000) {\hbox{\tate ジュン}};
\node[Meaning] at (-35.500000, 37.650000) {order};
\node[Kanji] at (-33.450000, 36.400000) {\textcolor[HTML]{c36143}{水}};
\node[Square] at (-33.450000, 35.900000) {};
\node[Onyomi] at (-33.400000, 36.000000) {\hbox{\tate スイ}};
\node[Kunyomi] at (-33.500000, 36.000000) {\hbox{\tate みず}};
\node[Meaning] at (-33.450000, 37.650000) {water};
\node[Kanji] at (-31.400000, 36.400000) {\textcolor[HTML]{a3bac2}{氷}};
\node[Square] at (-31.400000, 35.900000) {};
\node[Onyomi] at (-31.350000, 36.000000) {\hbox{\tate ヒョウ}};
\node[Kunyomi] at (-31.450000, 36.000000) {\hbox{\tate こおり}};
\node[Meaning] at (-31.400000, 37.650000) {ice};
\node[Kanji] at (-29.350000, 36.400000) {\textcolor[HTML]{d2a293}{永}};
\node[Square] at (-29.350000, 35.900000) {};
\node[Onyomi] at (-29.300000, 36.000000) {\hbox{\tate エイ}};
\node[Meaning] at (-29.350000, 37.650000) {eternity};
\node[Kanji] at (-27.300000, 36.400000) {\textcolor[HTML]{d2a293}{泉}};
\node[Square] at (-27.300000, 35.900000) {};
\node[Onyomi] at (-27.250000, 36.000000) {\hbox{\tate セン}};
\node[Kunyomi] at (-27.350000, 36.000000) {\hbox{\tate いずみ}};
\node[Meaning] at (-27.300000, 37.650000) {spring};
\node[Kanji] at (-25.250000, 36.400000) {\textcolor[HTML]{c36143}{原}};
\node[Square] at (-25.250000, 35.900000) {};
\node[Onyomi] at (-25.200000, 36.000000) {\hbox{\tate ゲン}};
\node[Kunyomi] at (-25.300000, 36.000000) {\hbox{\tate はら}};
\node[Meaning] at (-25.250000, 37.650000) {original};
\node[Kanji] at (-23.200000, 36.400000) {\textcolor[HTML]{c8a59d}{願}};
\node[Square] at (-23.200000, 35.900000) {};
\node[Onyomi] at (-23.150000, 36.000000) {\hbox{\tate ガン}};
\node[Kunyomi] at (-23.250000, 36.000000) {\hbox{\tate ねが・ねがい}};
\node[Meaning] at (-23.200000, 37.650000) {request};
\node[Kanji] at (-21.150000, 36.400000) {\textcolor[HTML]{91b7c3}{泳}};
\node[Square] at (-21.150000, 35.900000) {};
\node[Onyomi] at (-21.100000, 36.000000) {\hbox{\tate エイ}};
\node[Kunyomi] at (-21.200000, 36.000000) {\hbox{\tate およ}};
\node[Meaning] at (-21.150000, 37.650000) {swim};
\node[Kanji] at (-19.100000, 36.400000) {\textcolor[HTML]{b0b0b5}{沼}};
\node[Square] at (-19.100000, 35.900000) {};
\node[Onyomi] at (-19.050000, 36.000000) {\hbox{\tate ショウ}};
\node[Kunyomi] at (-19.150000, 36.000000) {\hbox{\tate ぬま}};
\node[Meaning] at (-19.100000, 37.650000) {bog};
\node[Kanji] at (-17.050000, 36.400000) {\textcolor[HTML]{c8a59d}{沖}};
\node[Square] at (-17.050000, 35.900000) {};
\node[Onyomi] at (-17.000000, 36.000000) {\hbox{\tate チュウ}};
\node[Kunyomi] at (-17.100000, 36.000000) {\hbox{\tate おき}};
\node[Meaning] at (-17.050000, 37.650000) {open sea};
\node[Kanji] at (-15.000000, 36.400000) {\textcolor[HTML]{d69f8d}{江}};
\node[Square] at (-15.000000, 35.900000) {};
\node[Onyomi] at (-14.950000, 36.000000) {\hbox{\tate コウ}};
\node[Kunyomi] at (-15.050000, 36.000000) {\hbox{\tate え}};
\node[Meaning] at (-15.000000, 37.650000) {inlet};
\node[Kanji] at (-12.950000, 36.400000) {\textcolor[HTML]{68a4bc}{汁}};
\node[Square] at (-12.950000, 35.900000) {};
\node[Onyomi] at (-12.900000, 36.000000) {\hbox{\tate ジュウ}};
\node[Kunyomi] at (-13.000000, 36.000000) {\hbox{\tate しる}};
\node[Meaning] at (-12.950000, 37.650000) {soup};
\node[Kanji] at (-10.900000, 36.400000) {\textcolor[HTML]{b0b0b5}{潮}};
\node[Square] at (-10.900000, 35.900000) {};
\node[Onyomi] at (-10.850000, 36.000000) {\hbox{\tate チョウ}};
\node[Kunyomi] at (-10.950000, 36.000000) {\hbox{\tate しお}};
\node[Meaning] at (-10.900000, 37.650000) {tide};
\node[Kanji] at (-8.850000, 36.400000) {\textcolor[HTML]{d2a293}{源}};
\node[Square] at (-8.850000, 35.900000) {};
\node[Onyomi] at (-8.800000, 36.000000) {\hbox{\tate ゲン}};
\node[Kunyomi] at (-8.900000, 36.000000) {\hbox{\tate みなもと}};
\node[Meaning] at (-8.850000, 37.650000) {origin};
\node[Kanji] at (-6.800000, 36.400000) {\textcolor[HTML]{c36143}{活}};
\node[Square] at (-6.800000, 35.900000) {};
\node[Onyomi] at (-6.750000, 36.000000) {\hbox{\tate カツ}};
\node[Meaning] at (-6.800000, 37.650000) {lively};
\node[Kanji] at (-4.750000, 36.400000) {\textcolor[HTML]{d2a293}{消}};
\node[Square] at (-4.750000, 35.900000) {};
\node[Onyomi] at (-4.700000, 36.000000) {\hbox{\tate ショウ}};
\node[Kunyomi] at (-4.800000, 36.000000) {\hbox{\tate き.*}};
\node[Meaning] at (-4.750000, 37.650000) {extinguish};
\node[Kanji] at (-2.700000, 36.400000) {\textcolor[HTML]{c8a59d}{況}};
\node[Square] at (-2.700000, 35.900000) {};
\node[Onyomi] at (-2.650000, 36.000000) {\hbox{\tate キョウ}};
\node[Meaning] at (-2.700000, 37.650000) {condition};
\node[Kanji] at (-0.650000, 36.400000) {\textcolor[HTML]{d69f8d}{河}};
\node[Square] at (-0.650000, 35.900000) {};
\node[Onyomi] at (-0.600000, 36.000000) {\hbox{\tate カ}};
\node[Kunyomi] at (-0.700000, 36.000000) {\hbox{\tate かわ}};
\node[Meaning] at (-0.650000, 37.650000) {river};
\node[Kanji] at (1.400000, 36.400000) {\textcolor[HTML]{a3bac2}{泊}};
\node[Square] at (1.400000, 35.900000) {};
\node[Onyomi] at (1.450000, 36.000000) {\hbox{\tate ハク}};
\node[Kunyomi] at (1.350000, 36.000000) {\hbox{\tate と.まる}};
\node[Meaning] at (1.400000, 37.650000) {overnight};
\node[Kanji] at (3.450000, 36.400000) {\textcolor[HTML]{c8a59d}{湖}};
\node[Square] at (3.450000, 35.900000) {};
\node[Onyomi] at (3.500000, 36.000000) {\hbox{\tate コ}};
\node[Kunyomi] at (3.400000, 36.000000) {\hbox{\tate みずうみ}};
\node[Meaning] at (3.450000, 37.650000) {lake};
\node[Kanji] at (5.500000, 36.400000) {\textcolor[HTML]{d2a293}{測}};
\node[Square] at (5.500000, 35.900000) {};
\node[Onyomi] at (5.550000, 36.000000) {\hbox{\tate ソク}};
\node[Kunyomi] at (5.450000, 36.000000) {\hbox{\tate はか.る}};
\node[Meaning] at (5.500000, 37.650000) {measure};
\node[Kanji] at (7.550000, 36.400000) {\textcolor[HTML]{cd8268}{土}};
\node[Square] at (7.550000, 35.900000) {};
\node[Onyomi] at (7.600000, 36.000000) {\hbox{\tate ド・ト}};
\node[Kunyomi] at (7.500000, 36.000000) {\hbox{\tate つち}};
\node[Meaning] at (7.550000, 37.650000) {soil};
\node[Kanji] at (9.600000, 36.400000) {\textcolor[HTML]{68a4bc}{吐}};
\node[Square] at (9.600000, 35.900000) {};
\node[Onyomi] at (9.650000, 36.000000) {\hbox{\tate ト}};
\node[Kunyomi] at (9.550000, 36.000000) {\hbox{\tate は}};
\node[Meaning] at (9.600000, 37.650000) {throw up};
\node[Kanji] at (11.650000, 36.400000) {\textcolor[HTML]{d2a293}{圧}};
\node[Square] at (11.650000, 35.900000) {};
\node[Onyomi] at (11.700000, 36.000000) {\hbox{\tate アツ}};
\node[Meaning] at (11.650000, 37.650000) {pressure};
\node[Kanji] at (13.700000, 36.400000) {\textcolor[HTML]{b0b0b5}{埼}};
\node[Square] at (13.700000, 35.900000) {};
\node[Onyomi] at (13.750000, 36.000000) {\hbox{\tate キ}};
\node[Kunyomi] at (13.650000, 36.000000) {\hbox{\tate さい}};
\node[Meaning] at (13.700000, 37.650000) {cape};
\node[Kanji] at (15.750000, 36.400000) {\textcolor[HTML]{a3bac2}{垣}};
\node[Square] at (15.750000, 35.900000) {};
\node[Kunyomi] at (15.700000, 36.000000) {\hbox{\tate かき}};
\node[Meaning] at (15.750000, 37.650000) {hedge};
\node[Kanji] at (17.800000, 36.400000) {\textcolor[HTML]{b0b0b5}{封}};
\node[Square] at (17.800000, 35.900000) {};
\node[Onyomi] at (17.850000, 36.000000) {\hbox{\tate フウ・ホウ}};
\node[Meaning] at (17.800000, 37.650000) {seal};
\node[Kanji] at (19.850000, 36.400000) {\textcolor[HTML]{91b7c3}{涯}};
\node[Square] at (19.850000, 35.900000) {};
\node[Onyomi] at (19.900000, 36.000000) {\hbox{\tate ガイ}};
\node[Kunyomi] at (19.800000, 36.000000) {\hbox{\tate はて}};
\node[Meaning] at (19.850000, 37.650000) {horizon};
\node[Kanji] at (21.900000, 36.400000) {\textcolor[HTML]{cd8268}{寺}};
\node[Square] at (21.900000, 35.900000) {};
\node[Onyomi] at (21.950000, 36.000000) {\hbox{\tate ジ}};
\node[Kunyomi] at (21.850000, 36.000000) {\hbox{\tate てら}};
\node[Meaning] at (21.900000, 37.650000) {temple};
\node[Kanji] at (23.950000, 36.400000) {\textcolor[HTML]{a11d25}{時}};
\node[Square] at (23.950000, 35.900000) {};
\node[Onyomi] at (24.000000, 36.000000) {\hbox{\tate ジ}};
\node[Kunyomi] at (23.900000, 36.000000) {\hbox{\tate とき}};
\node[Meaning] at (23.950000, 37.650000) {time};
\node[Kanji] at (26.000000, 36.400000) {\textcolor[HTML]{c8a59d}{均}};
\node[Square] at (26.000000, 35.900000) {};
\node[Onyomi] at (26.050000, 36.000000) {\hbox{\tate キン}};
\node[Kunyomi] at (25.950000, 36.000000) {\hbox{\tate ひと.しい}};
\node[Meaning] at (26.000000, 37.650000) {equal};
\node[Kanji] at (28.050000, 36.400000) {\textcolor[HTML]{d69f8d}{火}};
\node[Square] at (28.050000, 35.900000) {};
\node[Onyomi] at (28.100000, 36.000000) {\hbox{\tate カ}};
\node[Kunyomi] at (28.000000, 36.000000) {\hbox{\tate ひ・ほ}};
\node[Meaning] at (28.050000, 37.650000) {fire};
\node[Kanji] at (30.100000, 36.400000) {\textcolor[HTML]{b0b0b5}{炎}};
\node[Square] at (30.100000, 35.900000) {};
\node[Onyomi] at (30.150000, 36.000000) {\hbox{\tate エン}};
\node[Kunyomi] at (30.050000, 36.000000) {\hbox{\tate ほのお}};
\node[Meaning] at (30.100000, 37.650000) {flame};
\node[Kanji] at (32.150000, 36.400000) {\textcolor[HTML]{1059be}{煩}};
\node[Square] at (32.150000, 35.900000) {};
\node[Onyomi] at (32.200000, 36.000000) {\hbox{\tate ハン・ボン}};
\node[Kunyomi] at (32.100000, 36.000000) {\hbox{\tate うるさ}};
\node[Meaning] at (32.150000, 37.650000) {annoy};
\node[Kanji] at (34.200000, 36.400000) {\textcolor[HTML]{a3bac2}{淡}};
\node[Square] at (34.200000, 35.900000) {};
\node[Onyomi] at (34.250000, 36.000000) {\hbox{\tate タン}};
\node[Kunyomi] at (34.150000, 36.000000) {\hbox{\tate あわ.い}};
\node[Meaning] at (34.200000, 37.650000) {faint};
\node[Kanji] at (36.250000, 36.400000) {\textcolor[HTML]{b0b0b5}{灯}};
\node[Square] at (36.250000, 35.900000) {};
\node[Onyomi] at (36.300000, 36.000000) {\hbox{\tate トウ}};
\node[Kunyomi] at (36.200000, 36.000000) {\hbox{\tate あかり・とも}};
\node[Meaning] at (36.250000, 37.650000) {lamp};
\node[Kanji] at (38.300000, 36.400000) {\textcolor[HTML]{a3bac2}{畑}};
\node[Square] at (38.300000, 35.900000) {};
\node[Kunyomi] at (38.250000, 36.000000) {\hbox{\tate はたけ・はた}};
\node[Meaning] at (38.300000, 37.650000) {field};
\node[Kanji] at (40.350000, 36.400000) {\textcolor[HTML]{c8a59d}{災}};
\node[Square] at (40.350000, 35.900000) {};
\node[Onyomi] at (40.400000, 36.000000) {\hbox{\tate サイ}};
\node[Kunyomi] at (40.300000, 36.000000) {\hbox{\tate わざわ.い}};
\node[Meaning] at (40.350000, 37.650000) {disaster};
\node[Kanji] at (42.400000, 36.400000) {\textcolor[HTML]{a3bac2}{灰}};
\node[Square] at (42.400000, 35.900000) {};
\node[Onyomi] at (42.450000, 36.000000) {\hbox{\tate カイ}};
\node[Kunyomi] at (42.350000, 36.000000) {\hbox{\tate はい}};
\node[Meaning] at (42.400000, 37.650000) {ashes};
\node[Kanji] at (44.450000, 36.400000) {\textcolor[HTML]{c36143}{点}};
\node[Square] at (44.450000, 35.900000) {};
\node[Onyomi] at (44.500000, 36.000000) {\hbox{\tate テン}};
\node[Kunyomi] at (44.400000, 36.000000) {\hbox{\tate つ.ける}};
\node[Meaning] at (44.450000, 37.650000) {point};
\node[Kanji] at (46.500000, 36.400000) {\textcolor[HTML]{d2a293}{照}};
\node[Square] at (46.500000, 35.900000) {};
\node[Onyomi] at (46.550000, 36.000000) {\hbox{\tate ショウ}};
\node[Kunyomi] at (46.450000, 36.000000) {\hbox{\tate て.*}};
\node[Meaning] at (46.500000, 37.650000) {illuminate};
\node[Kanji] at (48.550000, 36.400000) {\textcolor[HTML]{c8a59d}{魚}};
\node[Square] at (48.550000, 35.900000) {};
\node[Onyomi] at (48.600000, 36.000000) {\hbox{\tate ギョ}};
\node[Kunyomi] at (48.500000, 36.000000) {\hbox{\tate さかな}};
\node[Meaning] at (48.550000, 37.650000) {fish};
\node[Kanji] at (50.600000, 36.400000) {\textcolor[HTML]{b0b0b5}{漁}};
\node[Square] at (50.600000, 35.900000) {};
\node[Onyomi] at (50.650000, 36.000000) {\hbox{\tate ギョ・リョウ}};
\node[Kunyomi] at (50.550000, 36.000000) {\hbox{\tate あさ.る}};
\node[Meaning] at (50.600000, 37.650000) {fishing};
\node[Kanji] at (52.650000, 36.400000) {\textcolor[HTML]{c8a59d}{里}};
\node[Square] at (52.650000, 35.900000) {};
\node[Onyomi] at (52.700000, 36.000000) {\hbox{\tate リ}};
\node[Kunyomi] at (52.600000, 36.000000) {\hbox{\tate さと}};
\node[Meaning] at (52.650000, 37.650000) {home village};
\node[Kanji] at (54.700000, 36.400000) {\textcolor[HTML]{d2a293}{黒}};
\node[Square] at (54.700000, 35.900000) {};
\node[Onyomi] at (54.750000, 36.000000) {\hbox{\tate コク}};
\node[Kunyomi] at (54.650000, 36.000000) {\hbox{\tate くろ.い}};
\node[Meaning] at (54.700000, 37.650000) {black};
\node[Kanji] at (56.750000, 36.400000) {\textcolor[HTML]{68a4bc}{墨}};
\node[Square] at (56.750000, 35.900000) {};
\node[Kunyomi] at (56.700000, 36.000000) {\hbox{\tate すみ}};
\node[Meaning] at (56.750000, 37.650000) {black ink};
\node[Meaning] at (-58.050000, 36.500000) {15.22\%};
\node[Kanji] at (-56.000000, 34.350000) {\textcolor[HTML]{1059be}{鯉}};
\node[Square] at (-56.000000, 33.850000) {};
\node[Onyomi] at (-55.950000, 33.950000) {\hbox{\tate リ}};
\node[Kunyomi] at (-56.050000, 33.950000) {\hbox{\tate こい}};
\node[Meaning] at (-56.000000, 35.600000) {carp};
\node[Kanji] at (-53.950000, 34.350000) {\textcolor[HTML]{d69f8d}{量}};
\node[Square] at (-53.950000, 33.850000) {};
\node[Onyomi] at (-53.900000, 33.950000) {\hbox{\tate リョウ}};
\node[Kunyomi] at (-54.000000, 33.950000) {\hbox{\tate はか.る}};
\node[Meaning] at (-53.950000, 35.600000) {quantity};
\node[Kanji] at (-51.900000, 34.350000) {\textcolor[HTML]{29409e}{厘}};
\node[Square] at (-51.900000, 33.850000) {};
\node[Onyomi] at (-51.850000, 33.950000) {\hbox{\tate リン}};
\node[Meaning] at (-51.900000, 35.600000) {thousandth};
\node[Kanji] at (-49.850000, 34.350000) {\textcolor[HTML]{b0b0b5}{埋}};
\node[Square] at (-49.850000, 33.850000) {};
\node[Onyomi] at (-49.800000, 33.950000) {\hbox{\tate マイ}};
\node[Kunyomi] at (-49.900000, 33.950000) {\hbox{\tate う}};
\node[Meaning] at (-49.850000, 35.600000) {bury};
\node[Kanji] at (-47.800000, 34.350000) {\textcolor[HTML]{b74029}{同}};
\node[Square] at (-47.800000, 33.850000) {};
\node[Onyomi] at (-47.750000, 33.950000) {\hbox{\tate ドウ}};
\node[Kunyomi] at (-47.850000, 33.950000) {\hbox{\tate おな.じ}};
\node[Meaning] at (-47.800000, 35.600000) {same};
\node[Kanji] at (-45.750000, 34.350000) {\textcolor[HTML]{a3bac2}{洞}};
\node[Square] at (-45.750000, 33.850000) {};
\node[Onyomi] at (-45.700000, 33.950000) {\hbox{\tate ドウ}};
\node[Kunyomi] at (-45.800000, 33.950000) {\hbox{\tate ほら}};
\node[Meaning] at (-45.750000, 35.600000) {cave};
\node[Kanji] at (-43.700000, 34.350000) {\textcolor[HTML]{91b7c3}{胴}};
\node[Square] at (-43.700000, 33.850000) {};
\node[Onyomi] at (-43.650000, 33.950000) {\hbox{\tate ドウ}};
\node[Meaning] at (-43.700000, 35.600000) {torso};
\node[Kanji] at (-41.650000, 34.350000) {\textcolor[HTML]{cd8268}{向}};
\node[Square] at (-41.650000, 33.850000) {};
\node[Onyomi] at (-41.600000, 33.950000) {\hbox{\tate コウ}};
\node[Kunyomi] at (-41.700000, 33.950000) {\hbox{\tate む.き}};
\node[Meaning] at (-41.650000, 35.600000) {yonder};
\node[Kanji] at (-39.600000, 34.350000) {\textcolor[HTML]{b0b0b5}{尚}};
\node[Square] at (-39.600000, 33.850000) {};
\node[Onyomi] at (-39.550000, 33.950000) {\hbox{\tate ショウ}};
\node[Kunyomi] at (-39.650000, 33.950000) {\hbox{\tate なお}};
\node[Meaning] at (-39.600000, 35.600000) {furthermore};
\node[Kanji] at (-37.550000, 34.350000) {\textcolor[HTML]{d69f8d}{字}};
\node[Square] at (-37.550000, 33.850000) {};
\node[Onyomi] at (-37.500000, 33.950000) {\hbox{\tate ジ}};
\node[Meaning] at (-37.550000, 35.600000) {letter};
\node[Kanji] at (-35.500000, 34.350000) {\textcolor[HTML]{d69f8d}{守}};
\node[Square] at (-35.500000, 33.850000) {};
\node[Onyomi] at (-35.450000, 33.950000) {\hbox{\tate ス・シュ}};
\node[Kunyomi] at (-35.550000, 33.950000) {\hbox{\tate まも.る}};
\node[Meaning] at (-35.500000, 35.600000) {protect};
\node[Kanji] at (-33.450000, 34.350000) {\textcolor[HTML]{d69f8d}{完}};
\node[Square] at (-33.450000, 33.850000) {};
\node[Onyomi] at (-33.400000, 33.950000) {\hbox{\tate カン}};
\node[Meaning] at (-33.450000, 35.600000) {perfect};
\node[Kanji] at (-31.400000, 34.350000) {\textcolor[HTML]{c8a59d}{宣}};
\node[Square] at (-31.400000, 33.850000) {};
\node[Onyomi] at (-31.350000, 33.950000) {\hbox{\tate セン}};
\node[Kunyomi] at (-31.450000, 33.950000) {\hbox{\tate のたま.う}};
\node[Meaning] at (-31.400000, 35.600000) {proclaim};
\node[Kanji] at (-29.350000, 34.350000) {\textcolor[HTML]{1059be}{宵}};
\node[Square] at (-29.350000, 33.850000) {};
\node[Onyomi] at (-29.300000, 33.950000) {\hbox{\tate ショウ}};
\node[Kunyomi] at (-29.400000, 33.950000) {\hbox{\tate よい}};
\node[Meaning] at (-29.350000, 35.600000) {wee hours};
\node[Kanji] at (-27.300000, 34.350000) {\textcolor[HTML]{cd8268}{安}};
\node[Square] at (-27.300000, 33.850000) {};
\node[Onyomi] at (-27.250000, 33.950000) {\hbox{\tate アン}};
\node[Kunyomi] at (-27.350000, 33.950000) {\hbox{\tate やす.い}};
\node[Meaning] at (-27.300000, 35.600000) {relax};
\node[Kanji] at (-25.250000, 34.350000) {\textcolor[HTML]{68a4bc}{宴}};
\node[Square] at (-25.250000, 33.850000) {};
\node[Onyomi] at (-25.200000, 33.950000) {\hbox{\tate エン}};
\node[Kunyomi] at (-25.300000, 33.950000) {\hbox{\tate うたげ}};
\node[Meaning] at (-25.250000, 35.600000) {banquet};
\node[Kanji] at (-23.200000, 34.350000) {\textcolor[HTML]{d2a293}{寄}};
\node[Square] at (-23.200000, 33.850000) {};
\node[Onyomi] at (-23.150000, 33.950000) {\hbox{\tate キ}};
\node[Kunyomi] at (-23.250000, 33.950000) {\hbox{\tate よ.る}};
\node[Meaning] at (-23.200000, 35.600000) {approach};
\node[Kanji] at (-21.150000, 34.350000) {\textcolor[HTML]{d2a293}{富}};
\node[Square] at (-21.150000, 33.850000) {};
\node[Onyomi] at (-21.100000, 33.950000) {\hbox{\tate フ}};
\node[Kunyomi] at (-21.200000, 33.950000) {\hbox{\tate と・とみ}};
\node[Meaning] at (-21.150000, 35.600000) {rich};
\node[Kanji] at (-19.100000, 34.350000) {\textcolor[HTML]{91b7c3}{貯}};
\node[Square] at (-19.100000, 33.850000) {};
\node[Onyomi] at (-19.050000, 33.950000) {\hbox{\tate チョ}};
\node[Kunyomi] at (-19.150000, 33.950000) {\hbox{\tate たくわ.える}};
\node[Meaning] at (-19.100000, 35.600000) {savings};
\node[Kanji] at (-17.050000, 34.350000) {\textcolor[HTML]{cd8268}{木}};
\node[Square] at (-17.050000, 33.850000) {};
\node[Onyomi] at (-17.000000, 33.950000) {\hbox{\tate モク}};
\node[Kunyomi] at (-17.100000, 33.950000) {\hbox{\tate き・こ}};
\node[Meaning] at (-17.050000, 35.600000) {tree};
\node[Kanji] at (-15.000000, 34.350000) {\textcolor[HTML]{d2a293}{林}};
\node[Square] at (-15.000000, 33.850000) {};
\node[Onyomi] at (-14.950000, 33.950000) {\hbox{\tate リン}};
\node[Kunyomi] at (-15.050000, 33.950000) {\hbox{\tate はやし}};
\node[Meaning] at (-15.000000, 35.600000) {forest};
\node[Kanji] at (-12.950000, 34.350000) {\textcolor[HTML]{d2a293}{森}};
\node[Square] at (-12.950000, 33.850000) {};
\node[Onyomi] at (-12.900000, 33.950000) {\hbox{\tate シン}};
\node[Kunyomi] at (-13.000000, 33.950000) {\hbox{\tate もり}};
\node[Meaning] at (-12.950000, 35.600000) {forest};
\node[Kanji] at (-10.900000, 34.350000) {\textcolor[HTML]{c8a59d}{枠}};
\node[Square] at (-10.900000, 33.850000) {};
\node[Kunyomi] at (-10.950000, 33.950000) {\hbox{\tate わく}};
\node[Meaning] at (-10.900000, 35.600000) {frame};
\node[Kanji] at (-8.850000, 34.350000) {\textcolor[HTML]{68a4bc}{棚}};
\node[Square] at (-8.850000, 33.850000) {};
\node[Onyomi] at (-8.800000, 33.950000) {\hbox{\tate ホウ}};
\node[Kunyomi] at (-8.900000, 33.950000) {\hbox{\tate たな}};
\node[Meaning] at (-8.850000, 35.600000) {shelf};
\node[Kanji] at (-6.800000, 34.350000) {\textcolor[HTML]{408dba}{杏}};
\node[Square] at (-6.800000, 33.850000) {};
\node[Onyomi] at (-6.750000, 33.950000) {\hbox{\tate アン・キョウ}};
\node[Kunyomi] at (-6.850000, 33.950000) {\hbox{\tate あんず}};
\node[Meaning] at (-6.800000, 35.600000) {apricot};
\node[Kanji] at (-4.750000, 34.350000) {\textcolor[HTML]{d2a293}{植}};
\node[Square] at (-4.750000, 33.850000) {};
\node[Onyomi] at (-4.700000, 33.950000) {\hbox{\tate ショク}};
\node[Kunyomi] at (-4.800000, 33.950000) {\hbox{\tate う.*}};
\node[Meaning] at (-4.750000, 35.600000) {plant};
\node[Kanji] at (-2.700000, 34.350000) {\textcolor[HTML]{68a4bc}{枯}};
\node[Square] at (-2.700000, 33.850000) {};
\node[Onyomi] at (-2.650000, 33.950000) {\hbox{\tate コ}};
\node[Kunyomi] at (-2.750000, 33.950000) {\hbox{\tate か}};
\node[Meaning] at (-2.700000, 35.600000) {wither};
\node[Kanji] at (-0.650000, 34.350000) {\textcolor[HTML]{68a4bc}{朴}};
\node[Square] at (-0.650000, 33.850000) {};
\node[Onyomi] at (-0.600000, 33.950000) {\hbox{\tate ボク}};
\node[Kunyomi] at (-0.700000, 33.950000) {\hbox{\tate えのき・ほう}};
\node[Meaning] at (-0.650000, 35.600000) {simple};
\node[Kanji] at (1.400000, 34.350000) {\textcolor[HTML]{c36143}{村}};
\node[Square] at (1.400000, 33.850000) {};
\node[Onyomi] at (1.450000, 33.950000) {\hbox{\tate ソン}};
\node[Kunyomi] at (1.350000, 33.950000) {\hbox{\tate むら}};
\node[Meaning] at (1.400000, 35.600000) {village};
\node[Kanji] at (3.450000, 34.350000) {\textcolor[HTML]{cd8268}{相}};
\node[Square] at (3.450000, 33.850000) {};
\node[Onyomi] at (3.500000, 33.950000) {\hbox{\tate ソウ・ショウ}};
\node[Kunyomi] at (3.400000, 33.950000) {\hbox{\tate あい}};
\node[Meaning] at (3.450000, 35.600000) {mutual};
\node[Kanji] at (5.500000, 34.350000) {\textcolor[HTML]{1e76bb}{机}};
\node[Square] at (5.500000, 33.850000) {};
\node[Kunyomi] at (5.450000, 33.950000) {\hbox{\tate つくえ}};
\node[Meaning] at (5.500000, 35.600000) {desk};
\node[Kanji] at (7.550000, 34.350000) {\textcolor[HTML]{830e29}{本}};
\node[Square] at (7.550000, 33.850000) {};
\node[Onyomi] at (7.600000, 33.950000) {\hbox{\tate ホン}};
\node[Kunyomi] at (7.500000, 33.950000) {\hbox{\tate もと}};
\node[Meaning] at (7.550000, 35.600000) {book};
\node[Kanji] at (9.600000, 34.350000) {\textcolor[HTML]{c8a59d}{札}};
\node[Square] at (9.600000, 33.850000) {};
\node[Onyomi] at (9.650000, 33.950000) {\hbox{\tate サツ}};
\node[Kunyomi] at (9.550000, 33.950000) {\hbox{\tate ふだ}};
\node[Meaning] at (9.600000, 35.600000) {bill};
\node[Kanji] at (11.650000, 34.350000) {\textcolor[HTML]{c8a59d}{暦}};
\node[Square] at (11.650000, 33.850000) {};
\node[Onyomi] at (11.700000, 33.950000) {\hbox{\tate レキ}};
\node[Kunyomi] at (11.600000, 33.950000) {\hbox{\tate こよみ}};
\node[Meaning] at (11.650000, 35.600000) {calendar};
\node[Kanji] at (13.700000, 34.350000) {\textcolor[HTML]{d2a293}{案}};
\node[Square] at (13.700000, 33.850000) {};
\node[Onyomi] at (13.750000, 33.950000) {\hbox{\tate アン}};
\node[Meaning] at (13.700000, 35.600000) {plan};
\node[Kanji] at (15.750000, 34.350000) {\textcolor[HTML]{91b7c3}{燥}};
\node[Square] at (15.750000, 33.850000) {};
\node[Onyomi] at (15.800000, 33.950000) {\hbox{\tate ソウ}};
\node[Kunyomi] at (15.700000, 33.950000) {\hbox{\tate はしゃ.ぐ}};
\node[Meaning] at (15.750000, 35.600000) {dry up};
\node[Kanji] at (17.800000, 34.350000) {\textcolor[HTML]{d2a293}{未}};
\node[Square] at (17.800000, 33.850000) {};
\node[Onyomi] at (17.850000, 33.950000) {\hbox{\tate ミ}};
\node[Kunyomi] at (17.750000, 33.950000) {\hbox{\tate ま.だ}};
\node[Meaning] at (17.800000, 35.600000) {not yet};
\node[Kanji] at (19.850000, 34.350000) {\textcolor[HTML]{d69f8d}{末}};
\node[Square] at (19.850000, 33.850000) {};
\node[Onyomi] at (19.900000, 33.950000) {\hbox{\tate マツ}};
\node[Kunyomi] at (19.800000, 33.950000) {\hbox{\tate すえ}};
\node[Meaning] at (19.850000, 35.600000) {end};
\node[Kanji] at (21.900000, 34.350000) {\textcolor[HTML]{d69f8d}{味}};
\node[Square] at (21.900000, 33.850000) {};
\node[Onyomi] at (21.950000, 33.950000) {\hbox{\tate ミ}};
\node[Kunyomi] at (21.850000, 33.950000) {\hbox{\tate あじ}};
\node[Meaning] at (21.900000, 35.600000) {flavor};
\node[Kanji] at (23.950000, 34.350000) {\textcolor[HTML]{b0b0b5}{妹}};
\node[Square] at (23.950000, 33.850000) {};
\node[Onyomi] at (24.000000, 33.950000) {\hbox{\tate マイ}};
\node[Kunyomi] at (23.900000, 33.950000) {\hbox{\tate いもうと}};
\node[Meaning] at (23.950000, 35.600000) {younger sister};
\node[Kanji] at (26.000000, 34.350000) {\textcolor[HTML]{91b7c3}{朱}};
\node[Square] at (26.000000, 33.850000) {};
\node[Onyomi] at (26.050000, 33.950000) {\hbox{\tate シュ}};
\node[Kunyomi] at (25.950000, 33.950000) {\hbox{\tate あけ}};
\node[Meaning] at (26.000000, 35.600000) {vermillion};
\node[Kanji] at (28.050000, 34.350000) {\textcolor[HTML]{d2a293}{株}};
\node[Square] at (28.050000, 33.850000) {};
\node[Onyomi] at (28.100000, 33.950000) {\hbox{\tate シュ}};
\node[Kunyomi] at (28.000000, 33.950000) {\hbox{\tate かぶ}};
\node[Meaning] at (28.050000, 35.600000) {stocks};
\node[Kanji] at (30.100000, 34.350000) {\textcolor[HTML]{d2a293}{若}};
\node[Square] at (30.100000, 33.850000) {};
\node[Onyomi] at (30.150000, 33.950000) {\hbox{\tate ジャク}};
\node[Kunyomi] at (30.050000, 33.950000) {\hbox{\tate わか・も}};
\node[Meaning] at (30.100000, 35.600000) {young};
\node[Kanji] at (32.150000, 34.350000) {\textcolor[HTML]{c8a59d}{草}};
\node[Square] at (32.150000, 33.850000) {};
\node[Onyomi] at (32.200000, 33.950000) {\hbox{\tate ソウ}};
\node[Kunyomi] at (32.100000, 33.950000) {\hbox{\tate くさ}};
\node[Meaning] at (32.150000, 35.600000) {grass};
\node[Kanji] at (34.200000, 34.350000) {\textcolor[HTML]{b0b0b5}{苦}};
\node[Square] at (34.200000, 33.850000) {};
\node[Onyomi] at (34.250000, 33.950000) {\hbox{\tate ク}};
\node[Kunyomi] at (34.150000, 33.950000) {\hbox{\tate くる・にが}};
\node[Meaning] at (34.200000, 35.600000) {suffering};
\node[Kanji] at (36.250000, 34.350000) {\textcolor[HTML]{b0b0b5}{寛}};
\node[Square] at (36.250000, 33.850000) {};
\node[Onyomi] at (36.300000, 33.950000) {\hbox{\tate カン}};
\node[Kunyomi] at (36.200000, 33.950000) {\hbox{\tate くつろ.ぐ}};
\node[Meaning] at (36.250000, 35.600000) {tolerance};
\node[Kanji] at (38.300000, 34.350000) {\textcolor[HTML]{a3bac2}{薄}};
\node[Square] at (38.300000, 33.850000) {};
\node[Onyomi] at (38.350000, 33.950000) {\hbox{\tate ハク}};
\node[Kunyomi] at (38.250000, 33.950000) {\hbox{\tate うす.*}};
\node[Meaning] at (38.300000, 35.600000) {dilute};
\node[Kanji] at (40.350000, 34.350000) {\textcolor[HTML]{d69f8d}{葉}};
\node[Square] at (40.350000, 33.850000) {};
\node[Kunyomi] at (40.300000, 33.950000) {\hbox{\tate は・ば}};
\node[Meaning] at (40.350000, 35.600000) {leaf};
\node[Kanji] at (42.400000, 34.350000) {\textcolor[HTML]{d2a293}{模}};
\node[Square] at (42.400000, 33.850000) {};
\node[Onyomi] at (42.450000, 33.950000) {\hbox{\tate モ・ボ}};
\node[Meaning] at (42.400000, 35.600000) {imitation};
\node[Kanji] at (44.450000, 34.350000) {\textcolor[HTML]{68a4bc}{漠}};
\node[Square] at (44.450000, 33.850000) {};
\node[Onyomi] at (44.500000, 33.950000) {\hbox{\tate バク}};
\node[Meaning] at (44.450000, 35.600000) {desert};
\node[Kanji] at (46.500000, 34.350000) {\textcolor[HTML]{b0b0b5}{墓}};
\node[Square] at (46.500000, 33.850000) {};
\node[Onyomi] at (46.550000, 33.950000) {\hbox{\tate ボ}};
\node[Kunyomi] at (46.450000, 33.950000) {\hbox{\tate はか}};
\node[Meaning] at (46.500000, 35.600000) {grave};
\node[Kanji] at (48.550000, 34.350000) {\textcolor[HTML]{a3bac2}{暮}};
\node[Square] at (48.550000, 33.850000) {};
\node[Onyomi] at (48.600000, 33.950000) {\hbox{\tate ボ}};
\node[Kunyomi] at (48.500000, 33.950000) {\hbox{\tate く.*}};
\node[Meaning] at (48.550000, 35.600000) {livelihood};
\node[Kanji] at (50.600000, 34.350000) {\textcolor[HTML]{a3bac2}{膜}};
\node[Square] at (50.600000, 33.850000) {};
\node[Onyomi] at (50.650000, 33.950000) {\hbox{\tate マク}};
\node[Meaning] at (50.600000, 35.600000) {membrane};
\node[Kanji] at (52.650000, 34.350000) {\textcolor[HTML]{68a4bc}{苗}};
\node[Square] at (52.650000, 33.850000) {};
\node[Onyomi] at (52.700000, 33.950000) {\hbox{\tate ミョウ}};
\node[Kunyomi] at (52.600000, 33.950000) {\hbox{\tate なえ・なわ}};
\node[Meaning] at (52.650000, 35.600000) {seedling};
\node[Kanji] at (54.700000, 34.350000) {\textcolor[HTML]{68a4bc}{兆}};
\node[Square] at (54.700000, 33.850000) {};
\node[Onyomi] at (54.750000, 33.950000) {\hbox{\tate チョウ}};
\node[Meaning] at (54.700000, 35.600000) {omen};
\node[Kanji] at (56.750000, 34.350000) {\textcolor[HTML]{91b7c3}{桃}};
\node[Square] at (56.750000, 33.850000) {};
\node[Kunyomi] at (56.700000, 33.950000) {\hbox{\tate もも}};
\node[Meaning] at (56.750000, 35.600000) {peach};
\node[Meaning] at (-58.050000, 34.450000) {17.98\%};
\node[Kanji] at (-56.000000, 32.300000) {\textcolor[HTML]{408dba}{眺}};
\node[Square] at (-56.000000, 31.800000) {};
\node[Onyomi] at (-55.950000, 31.900000) {\hbox{\tate チョウ}};
\node[Kunyomi] at (-56.050000, 31.900000) {\hbox{\tate なが.める}};
\node[Meaning] at (-56.000000, 33.550000) {stare};
\node[Kanji] at (-53.950000, 32.300000) {\textcolor[HTML]{b0b0b5}{犬}};
\node[Square] at (-53.950000, 31.800000) {};
\node[Onyomi] at (-53.900000, 31.900000) {\hbox{\tate ケン}};
\node[Kunyomi] at (-54.000000, 31.900000) {\hbox{\tate いぬ}};
\node[Meaning] at (-53.950000, 33.550000) {dog};
\node[Kanji] at (-51.900000, 32.300000) {\textcolor[HTML]{cd8268}{状}};
\node[Square] at (-51.900000, 31.800000) {};
\node[Onyomi] at (-51.850000, 31.900000) {\hbox{\tate ジョウ}};
\node[Meaning] at (-51.900000, 33.550000) {condition};
\node[Kanji] at (-49.850000, 32.300000) {\textcolor[HTML]{68a4bc}{黙}};
\node[Square] at (-49.850000, 31.800000) {};
\node[Onyomi] at (-49.800000, 31.900000) {\hbox{\tate モク}};
\node[Kunyomi] at (-49.900000, 31.900000) {\hbox{\tate だま.る}};
\node[Meaning] at (-49.850000, 33.550000) {shut up};
\node[Kanji] at (-47.800000, 32.300000) {\textcolor[HTML]{d2a293}{然}};
\node[Square] at (-47.800000, 31.800000) {};
\node[Onyomi] at (-47.750000, 31.900000) {\hbox{\tate ゼン・ネン}};
\node[Kunyomi] at (-47.850000, 31.900000) {\hbox{\tate しか・さ}};
\node[Meaning] at (-47.800000, 33.550000) {nature};
\node[Kanji] at (-45.750000, 32.300000) {\textcolor[HTML]{a3bac2}{狩}};
\node[Square] at (-45.750000, 31.800000) {};
\node[Onyomi] at (-45.700000, 31.900000) {\hbox{\tate シュ}};
\node[Kunyomi] at (-45.800000, 31.900000) {\hbox{\tate か}};
\node[Meaning] at (-45.750000, 33.550000) {hunt};
\node[Kanji] at (-43.700000, 32.300000) {\textcolor[HTML]{91b7c3}{猫}};
\node[Square] at (-43.700000, 31.800000) {};
\node[Kunyomi] at (-43.750000, 31.900000) {\hbox{\tate ねこ}};
\node[Meaning] at (-43.700000, 33.550000) {cat};
\node[Kanji] at (-41.650000, 32.300000) {\textcolor[HTML]{b0b0b5}{牛}};
\node[Square] at (-41.650000, 31.800000) {};
\node[Onyomi] at (-41.600000, 31.900000) {\hbox{\tate ギュウ}};
\node[Kunyomi] at (-41.700000, 31.900000) {\hbox{\tate うし}};
\node[Meaning] at (-41.650000, 33.550000) {cow};
\node[Kanji] at (-39.600000, 32.300000) {\textcolor[HTML]{c36143}{特}};
\node[Square] at (-39.600000, 31.800000) {};
\node[Onyomi] at (-39.550000, 31.900000) {\hbox{\tate トク}};
\node[Meaning] at (-39.600000, 33.550000) {special};
\node[Kanji] at (-37.550000, 32.300000) {\textcolor[HTML]{d69f8d}{告}};
\node[Square] at (-37.550000, 31.800000) {};
\node[Onyomi] at (-37.500000, 31.900000) {\hbox{\tate コク}};
\node[Kunyomi] at (-37.600000, 31.900000) {\hbox{\tate つ.げる}};
\node[Meaning] at (-37.550000, 33.550000) {announce};
\node[Kanji] at (-35.500000, 32.300000) {\textcolor[HTML]{cd8268}{先}};
\node[Square] at (-35.500000, 31.800000) {};
\node[Onyomi] at (-35.450000, 31.900000) {\hbox{\tate セン}};
\node[Kunyomi] at (-35.550000, 31.900000) {\hbox{\tate さき・まず}};
\node[Meaning] at (-35.500000, 33.550000) {previous};
\node[Kanji] at (-33.450000, 32.300000) {\textcolor[HTML]{a3bac2}{洗}};
\node[Square] at (-33.450000, 31.800000) {};
\node[Onyomi] at (-33.400000, 31.900000) {\hbox{\tate セン}};
\node[Kunyomi] at (-33.500000, 31.900000) {\hbox{\tate あら.う}};
\node[Meaning] at (-33.450000, 33.550000) {wash};
\node[Kanji] at (-31.400000, 32.300000) {\textcolor[HTML]{d2a293}{介}};
\node[Square] at (-31.400000, 31.800000) {};
\node[Onyomi] at (-31.350000, 31.900000) {\hbox{\tate カイ}};
\node[Meaning] at (-31.400000, 33.550000) {jammed in};
\node[Kanji] at (-29.350000, 32.300000) {\textcolor[HTML]{cd8268}{界}};
\node[Square] at (-29.350000, 31.800000) {};
\node[Onyomi] at (-29.300000, 31.900000) {\hbox{\tate カイ}};
\node[Meaning] at (-29.350000, 33.550000) {world};
\node[Kanji] at (-27.300000, 32.300000) {\textcolor[HTML]{b0b0b5}{茶}};
\node[Square] at (-27.300000, 31.800000) {};
\node[Onyomi] at (-27.250000, 31.900000) {\hbox{\tate チャ・サ}};
\node[Meaning] at (-27.300000, 33.550000) {tea};
\node[Kanji] at (-25.250000, 32.300000) {\textcolor[HTML]{a11d25}{合}};
\node[Square] at (-25.250000, 31.800000) {};
\node[Onyomi] at (-25.200000, 31.900000) {\hbox{\tate ゴウ・ガッ}};
\node[Kunyomi] at (-25.300000, 31.900000) {\hbox{\tate あ・あい}};
\node[Meaning] at (-25.250000, 33.550000) {suit};
\node[Kanji] at (-23.200000, 32.300000) {\textcolor[HTML]{b0b0b5}{塔}};
\node[Square] at (-23.200000, 31.800000) {};
\node[Onyomi] at (-23.150000, 31.900000) {\hbox{\tate トウ}};
\node[Meaning] at (-23.200000, 33.550000) {tower};
\node[Kanji] at (-21.150000, 32.300000) {\textcolor[HTML]{cd8268}{王}};
\node[Square] at (-21.150000, 31.800000) {};
\node[Onyomi] at (-21.100000, 31.900000) {\hbox{\tate オウ}};
\node[Meaning] at (-21.150000, 33.550000) {king};
\node[Kanji] at (-19.100000, 32.300000) {\textcolor[HTML]{d2a293}{玉}};
\node[Square] at (-19.100000, 31.800000) {};
\node[Onyomi] at (-19.050000, 31.900000) {\hbox{\tate ギョク}};
\node[Kunyomi] at (-19.150000, 31.900000) {\hbox{\tate たま}};
\node[Meaning] at (-19.100000, 33.550000) {ball};
\node[Kanji] at (-17.050000, 32.300000) {\textcolor[HTML]{d2a293}{宝}};
\node[Square] at (-17.050000, 31.800000) {};
\node[Onyomi] at (-17.000000, 31.900000) {\hbox{\tate ホウ}};
\node[Kunyomi] at (-17.100000, 31.900000) {\hbox{\tate たから}};
\node[Meaning] at (-17.050000, 33.550000) {treasure};
\node[Kanji] at (-15.000000, 32.300000) {\textcolor[HTML]{91b7c3}{珠}};
\node[Square] at (-15.000000, 31.800000) {};
\node[Onyomi] at (-14.950000, 31.900000) {\hbox{\tate シュ}};
\node[Kunyomi] at (-15.050000, 31.900000) {\hbox{\tate たましい}};
\node[Meaning] at (-15.000000, 33.550000) {pearl};
\node[Kanji] at (-12.950000, 32.300000) {\textcolor[HTML]{b74029}{現}};
\node[Square] at (-12.950000, 31.800000) {};
\node[Onyomi] at (-12.900000, 31.900000) {\hbox{\tate ゲン}};
\node[Kunyomi] at (-13.000000, 31.900000) {\hbox{\tate あらわ.*}};
\node[Meaning] at (-12.950000, 33.550000) {present time};
\node[Kanji] at (-10.900000, 32.300000) {\textcolor[HTML]{91b7c3}{狂}};
\node[Square] at (-10.900000, 31.800000) {};
\node[Onyomi] at (-10.850000, 31.900000) {\hbox{\tate キョウ}};
\node[Kunyomi] at (-10.950000, 31.900000) {\hbox{\tate くる.*}};
\node[Meaning] at (-10.900000, 33.550000) {lunatic};
\node[Kanji] at (-8.850000, 32.300000) {\textcolor[HTML]{d69f8d}{皇}};
\node[Square] at (-8.850000, 31.800000) {};
\node[Onyomi] at (-8.800000, 31.900000) {\hbox{\tate コウ}};
\node[Meaning] at (-8.850000, 33.550000) {emperor};
\node[Kanji] at (-6.800000, 32.300000) {\textcolor[HTML]{91b7c3}{呈}};
\node[Square] at (-6.800000, 31.800000) {};
\node[Onyomi] at (-6.750000, 31.900000) {\hbox{\tate テイ}};
\node[Meaning] at (-6.800000, 33.550000) {present};
\node[Kanji] at (-4.750000, 32.300000) {\textcolor[HTML]{b74029}{全}};
\node[Square] at (-4.750000, 31.800000) {};
\node[Onyomi] at (-4.700000, 31.900000) {\hbox{\tate ゼン}};
\node[Kunyomi] at (-4.800000, 31.900000) {\hbox{\tate すべ.て}};
\node[Meaning] at (-4.750000, 33.550000) {all};
\node[Kanji] at (-2.700000, 32.300000) {\textcolor[HTML]{408dba}{栓}};
\node[Square] at (-2.700000, 31.800000) {};
\node[Onyomi] at (-2.650000, 31.900000) {\hbox{\tate セン}};
\node[Meaning] at (-2.700000, 33.550000) {cork};
\node[Kanji] at (-0.650000, 32.300000) {\textcolor[HTML]{c36143}{理}};
\node[Square] at (-0.650000, 31.800000) {};
\node[Onyomi] at (-0.600000, 31.900000) {\hbox{\tate リ}};
\node[Kunyomi] at (-0.700000, 31.900000) {\hbox{\tate ことわり}};
\node[Meaning] at (-0.650000, 33.550000) {reason};
\node[Kanji] at (1.400000, 32.300000) {\textcolor[HTML]{b74029}{主}};
\node[Square] at (1.400000, 31.800000) {};
\node[Onyomi] at (1.450000, 31.900000) {\hbox{\tate シュ}};
\node[Kunyomi] at (1.350000, 31.900000) {\hbox{\tate おも・ぬし}};
\node[Meaning] at (1.400000, 33.550000) {master};
\node[Kanji] at (3.450000, 32.300000) {\textcolor[HTML]{cd8268}{注}};
\node[Square] at (3.450000, 31.800000) {};
\node[Onyomi] at (3.500000, 31.900000) {\hbox{\tate チュウ}};
\node[Kunyomi] at (3.400000, 31.900000) {\hbox{\tate そそ.ぐ}};
\node[Meaning] at (3.450000, 33.550000) {pour};
\node[Kanji] at (5.500000, 32.300000) {\textcolor[HTML]{a3bac2}{柱}};
\node[Square] at (5.500000, 31.800000) {};
\node[Onyomi] at (5.550000, 31.900000) {\hbox{\tate チュウ}};
\node[Kunyomi] at (5.450000, 31.900000) {\hbox{\tate はしら}};
\node[Meaning] at (5.500000, 33.550000) {pillar};
\node[Kanji] at (7.550000, 32.300000) {\textcolor[HTML]{c36143}{金}};
\node[Square] at (7.550000, 31.800000) {};
\node[Onyomi] at (7.600000, 31.900000) {\hbox{\tate キン}};
\node[Kunyomi] at (7.500000, 31.900000) {\hbox{\tate かね}};
\node[Meaning] at (7.550000, 33.550000) {gold};
\node[Kanji] at (9.600000, 32.300000) {\textcolor[HTML]{408dba}{鉢}};
\node[Square] at (9.600000, 31.800000) {};
\node[Onyomi] at (9.650000, 31.900000) {\hbox{\tate ハチ}};
\node[Meaning] at (9.600000, 33.550000) {bowl};
\node[Kanji] at (11.650000, 32.300000) {\textcolor[HTML]{a3bac2}{銅}};
\node[Square] at (11.650000, 31.800000) {};
\node[Onyomi] at (11.700000, 31.900000) {\hbox{\tate ドウ}};
\node[Kunyomi] at (11.600000, 31.900000) {\hbox{\tate あかがね}};
\node[Meaning] at (11.650000, 33.550000) {copper};
\node[Kanji] at (13.700000, 32.300000) {\textcolor[HTML]{91b7c3}{釣}};
\node[Square] at (13.700000, 31.800000) {};
\node[Onyomi] at (13.750000, 31.900000) {\hbox{\tate チョウ}};
\node[Kunyomi] at (13.650000, 31.900000) {\hbox{\tate つ}};
\node[Meaning] at (13.700000, 33.550000) {fishing};
\node[Kanji] at (15.750000, 32.300000) {\textcolor[HTML]{c8a59d}{針}};
\node[Square] at (15.750000, 31.800000) {};
\node[Onyomi] at (15.800000, 31.900000) {\hbox{\tate シン}};
\node[Kunyomi] at (15.700000, 31.900000) {\hbox{\tate はり}};
\node[Meaning] at (15.750000, 33.550000) {needle};
\node[Kanji] at (17.800000, 32.300000) {\textcolor[HTML]{91b7c3}{銘}};
\node[Square] at (17.800000, 31.800000) {};
\node[Onyomi] at (17.850000, 31.900000) {\hbox{\tate メイ}};
\node[Meaning] at (17.800000, 33.550000) {inscription};
\node[Kanji] at (19.850000, 32.300000) {\textcolor[HTML]{b0b0b5}{鎮}};
\node[Square] at (19.850000, 31.800000) {};
\node[Onyomi] at (19.900000, 31.900000) {\hbox{\tate チン}};
\node[Kunyomi] at (19.800000, 31.900000) {\hbox{\tate おさえ・しず}};
\node[Meaning] at (19.850000, 33.550000) {tranquilize};
\node[Kanji] at (21.900000, 32.300000) {\textcolor[HTML]{b74029}{道}};
\node[Square] at (21.900000, 31.800000) {};
\node[Onyomi] at (21.950000, 31.900000) {\hbox{\tate ドウ}};
\node[Kunyomi] at (21.850000, 31.900000) {\hbox{\tate みち}};
\node[Meaning] at (21.900000, 33.550000) {road};
\node[Kanji] at (23.950000, 32.300000) {\textcolor[HTML]{d69f8d}{導}};
\node[Square] at (23.950000, 31.800000) {};
\node[Onyomi] at (24.000000, 31.900000) {\hbox{\tate ドウ}};
\node[Kunyomi] at (23.900000, 31.900000) {\hbox{\tate みちび.く}};
\node[Meaning] at (23.950000, 33.550000) {lead};
\node[Kanji] at (26.000000, 32.300000) {\textcolor[HTML]{408dba}{迅}};
\node[Square] at (26.000000, 31.800000) {};
\node[Onyomi] at (26.050000, 31.900000) {\hbox{\tate ジン}};
\node[Meaning] at (26.000000, 33.550000) {swift};
\node[Kanji] at (28.050000, 32.300000) {\textcolor[HTML]{cd8268}{造}};
\node[Square] at (28.050000, 31.800000) {};
\node[Onyomi] at (28.100000, 31.900000) {\hbox{\tate ゾウ}};
\node[Kunyomi] at (28.000000, 31.900000) {\hbox{\tate つく.る}};
\node[Meaning] at (28.050000, 33.550000) {create};
\node[Kanji] at (30.100000, 32.300000) {\textcolor[HTML]{b0b0b5}{迫}};
\node[Square] at (30.100000, 31.800000) {};
\node[Onyomi] at (30.150000, 31.900000) {\hbox{\tate ハク}};
\node[Kunyomi] at (30.050000, 31.900000) {\hbox{\tate せま.る}};
\node[Meaning] at (30.100000, 33.550000) {urge};
\node[Kanji] at (32.150000, 32.300000) {\textcolor[HTML]{c8a59d}{逃}};
\node[Square] at (32.150000, 31.800000) {};
\node[Onyomi] at (32.200000, 31.900000) {\hbox{\tate トウ}};
\node[Kunyomi] at (32.100000, 31.900000) {\hbox{\tate に.げる}};
\node[Meaning] at (32.150000, 33.550000) {escape};
\node[Kanji] at (34.200000, 32.300000) {\textcolor[HTML]{d2a293}{辺}};
\node[Square] at (34.200000, 31.800000) {};
\node[Onyomi] at (34.250000, 31.900000) {\hbox{\tate ヘン}};
\node[Kunyomi] at (34.150000, 31.900000) {\hbox{\tate あた.り・べ}};
\node[Meaning] at (34.200000, 33.550000) {area};
\node[Kanji] at (36.250000, 32.300000) {\textcolor[HTML]{c8a59d}{巡}};
\node[Square] at (36.250000, 31.800000) {};
\node[Onyomi] at (36.300000, 31.900000) {\hbox{\tate ジュン}};
\node[Kunyomi] at (36.200000, 31.900000) {\hbox{\tate めぐ.る}};
\node[Meaning] at (36.250000, 33.550000) {patrol};
\node[Kanji] at (38.300000, 32.300000) {\textcolor[HTML]{b74029}{車}};
\node[Square] at (38.300000, 31.800000) {};
\node[Onyomi] at (38.350000, 31.900000) {\hbox{\tate シャ}};
\node[Kunyomi] at (38.250000, 31.900000) {\hbox{\tate くるま}};
\node[Meaning] at (38.300000, 33.550000) {car};
\node[Kanji] at (40.350000, 32.300000) {\textcolor[HTML]{c36143}{連}};
\node[Square] at (40.350000, 31.800000) {};
\node[Onyomi] at (40.400000, 31.900000) {\hbox{\tate レン}};
\node[Kunyomi] at (40.300000, 31.900000) {\hbox{\tate つ・つら}};
\node[Meaning] at (40.350000, 33.550000) {take along};
\node[Kanji] at (42.400000, 32.300000) {\textcolor[HTML]{c8a59d}{軌}};
\node[Square] at (42.400000, 31.800000) {};
\node[Onyomi] at (42.450000, 31.900000) {\hbox{\tate キ}};
\node[Meaning] at (42.400000, 33.550000) {rut};
\node[Kanji] at (44.450000, 32.300000) {\textcolor[HTML]{d2a293}{輸}};
\node[Square] at (44.450000, 31.800000) {};
\node[Onyomi] at (44.500000, 31.900000) {\hbox{\tate ユ}};
\node[Meaning] at (44.450000, 33.550000) {transport};
\node[Kanji] at (46.500000, 32.300000) {\textcolor[HTML]{b74029}{前}};
\node[Square] at (46.500000, 31.800000) {};
\node[Onyomi] at (46.550000, 31.900000) {\hbox{\tate ゼン}};
\node[Kunyomi] at (46.450000, 31.900000) {\hbox{\tate まえ}};
\node[Meaning] at (46.500000, 33.550000) {front};
\node[Kanji] at (48.550000, 32.300000) {\textcolor[HTML]{cd8268}{各}};
\node[Square] at (48.550000, 31.800000) {};
\node[Onyomi] at (48.600000, 31.900000) {\hbox{\tate カク}};
\node[Kunyomi] at (48.500000, 31.900000) {\hbox{\tate おの}};
\node[Meaning] at (48.550000, 33.550000) {each};
\node[Kanji] at (50.600000, 32.300000) {\textcolor[HTML]{cd8268}{格}};
\node[Square] at (50.600000, 31.800000) {};
\node[Onyomi] at (50.650000, 31.900000) {\hbox{\tate カク}};
\node[Meaning] at (50.600000, 33.550000) {status};
\node[Kanji] at (52.650000, 32.300000) {\textcolor[HTML]{d2a293}{略}};
\node[Square] at (52.650000, 31.800000) {};
\node[Onyomi] at (52.700000, 31.900000) {\hbox{\tate リャク}};
\node[Kunyomi] at (52.600000, 31.900000) {\hbox{\tate りゃく.す}};
\node[Meaning] at (52.650000, 33.550000) {abbreviation};
\node[Kanji] at (54.700000, 32.300000) {\textcolor[HTML]{d69f8d}{客}};
\node[Square] at (54.700000, 31.800000) {};
\node[Onyomi] at (54.750000, 31.900000) {\hbox{\tate キャク}};
\node[Meaning] at (54.700000, 33.550000) {guest};
\node[Kanji] at (56.750000, 32.300000) {\textcolor[HTML]{c8a59d}{額}};
\node[Square] at (56.750000, 31.800000) {};
\node[Onyomi] at (56.800000, 31.900000) {\hbox{\tate ガク}};
\node[Kunyomi] at (56.700000, 31.900000) {\hbox{\tate ひたい}};
\node[Meaning] at (56.750000, 33.550000) {amount};
\node[Meaning] at (-58.050000, 32.400000) {22.44\%};
\node[Kanji] at (-56.000000, 30.250000) {\textcolor[HTML]{c8a59d}{夏}};
\node[Square] at (-56.000000, 29.750000) {};
\node[Onyomi] at (-55.950000, 29.850000) {\hbox{\tate ゲ・カ・ガ}};
\node[Kunyomi] at (-56.050000, 29.850000) {\hbox{\tate なつ}};
\node[Meaning] at (-56.000000, 31.500000) {summer};
\node[Kanji] at (-53.950000, 30.250000) {\textcolor[HTML]{d2a293}{処}};
\node[Square] at (-53.950000, 29.750000) {};
\node[Onyomi] at (-53.900000, 29.850000) {\hbox{\tate ショ}};
\node[Meaning] at (-53.950000, 31.500000) {deal with};
\node[Kanji] at (-51.900000, 30.250000) {\textcolor[HTML]{cd8268}{条}};
\node[Square] at (-51.900000, 29.750000) {};
\node[Onyomi] at (-51.850000, 29.850000) {\hbox{\tate ジョウ}};
\node[Meaning] at (-51.900000, 31.500000) {clause};
\node[Kanji] at (-49.850000, 30.250000) {\textcolor[HTML]{d69f8d}{落}};
\node[Square] at (-49.850000, 29.750000) {};
\node[Onyomi] at (-49.800000, 29.850000) {\hbox{\tate ラク}};
\node[Kunyomi] at (-49.900000, 29.850000) {\hbox{\tate お.ちる}};
\node[Meaning] at (-49.850000, 31.500000) {fall};
\node[Kanji] at (-47.800000, 30.250000) {\textcolor[HTML]{1e76bb}{冗}};
\node[Square] at (-47.800000, 29.750000) {};
\node[Onyomi] at (-47.750000, 29.850000) {\hbox{\tate ジョウ}};
\node[Meaning] at (-47.800000, 31.500000) {superfluous};
\node[Kanji] at (-45.750000, 30.250000) {\textcolor[HTML]{c36143}{軍}};
\node[Square] at (-45.750000, 29.750000) {};
\node[Onyomi] at (-45.700000, 29.850000) {\hbox{\tate グン}};
\node[Meaning] at (-45.750000, 31.500000) {army};
\node[Kanji] at (-43.700000, 30.250000) {\textcolor[HTML]{a3bac2}{輝}};
\node[Square] at (-43.700000, 29.750000) {};
\node[Onyomi] at (-43.650000, 29.850000) {\hbox{\tate キ}};
\node[Kunyomi] at (-43.750000, 29.850000) {\hbox{\tate かがやき}};
\node[Meaning] at (-43.700000, 31.500000) {radiance};
\node[Kanji] at (-41.650000, 30.250000) {\textcolor[HTML]{c36143}{運}};
\node[Square] at (-41.650000, 29.750000) {};
\node[Onyomi] at (-41.600000, 29.850000) {\hbox{\tate ウン}};
\node[Kunyomi] at (-41.700000, 29.850000) {\hbox{\tate はこ.ぶ}};
\node[Meaning] at (-41.650000, 31.500000) {carry};
\node[Kanji] at (-39.600000, 30.250000) {\textcolor[HTML]{b0b0b5}{冠}};
\node[Square] at (-39.600000, 29.750000) {};
\node[Onyomi] at (-39.550000, 29.850000) {\hbox{\tate カン}};
\node[Kunyomi] at (-39.650000, 29.850000) {\hbox{\tate かんむり}};
\node[Meaning] at (-39.600000, 31.500000) {crown};
\node[Kanji] at (-37.550000, 30.250000) {\textcolor[HTML]{b0b0b5}{夢}};
\node[Square] at (-37.550000, 29.750000) {};
\node[Onyomi] at (-37.500000, 29.850000) {\hbox{\tate ム}};
\node[Kunyomi] at (-37.600000, 29.850000) {\hbox{\tate ゆめ}};
\node[Meaning] at (-37.550000, 31.500000) {dream};
\node[Kanji] at (-35.500000, 30.250000) {\textcolor[HTML]{408dba}{坑}};
\node[Square] at (-35.500000, 29.750000) {};
\node[Onyomi] at (-35.450000, 29.850000) {\hbox{\tate コウ}};
\node[Meaning] at (-35.500000, 31.500000) {pit};
\node[Kanji] at (-33.450000, 30.250000) {\textcolor[HTML]{b74029}{高}};
\node[Square] at (-33.450000, 29.750000) {};
\node[Onyomi] at (-33.400000, 29.850000) {\hbox{\tate コウ}};
\node[Kunyomi] at (-33.500000, 29.850000) {\hbox{\tate たか.い}};
\node[Meaning] at (-33.450000, 31.500000) {tall};
\node[Kanji] at (-31.400000, 30.250000) {\textcolor[HTML]{b0b0b5}{享}};
\node[Square] at (-31.400000, 29.750000) {};
\node[Onyomi] at (-31.350000, 29.850000) {\hbox{\tate キョウ・コウ}};
\node[Kunyomi] at (-31.450000, 29.850000) {\hbox{\tate う}};
\node[Meaning] at (-31.400000, 31.500000) {receive};
\node[Kanji] at (-29.350000, 30.250000) {\textcolor[HTML]{a3bac2}{塾}};
\node[Square] at (-29.350000, 29.750000) {};
\node[Onyomi] at (-29.300000, 29.850000) {\hbox{\tate ジュク}};
\node[Meaning] at (-29.350000, 31.500000) {cram school};
\node[Kanji] at (-27.300000, 30.250000) {\textcolor[HTML]{a3bac2}{熟}};
\node[Square] at (-27.300000, 29.750000) {};
\node[Onyomi] at (-27.250000, 29.850000) {\hbox{\tate ジュク}};
\node[Kunyomi] at (-27.350000, 29.850000) {\hbox{\tate う.れる}};
\node[Meaning] at (-27.300000, 31.500000) {ripen};
\node[Kanji] at (-25.250000, 30.250000) {\textcolor[HTML]{a3bac2}{亭}};
\node[Square] at (-25.250000, 29.750000) {};
\node[Onyomi] at (-25.200000, 29.850000) {\hbox{\tate テイ}};
\node[Meaning] at (-25.250000, 31.500000) {restaurant};
\node[Kanji] at (-23.200000, 30.250000) {\textcolor[HTML]{c36143}{京}};
\node[Square] at (-23.200000, 29.750000) {};
\node[Onyomi] at (-23.150000, 29.850000) {\hbox{\tate キョウ}};
\node[Kunyomi] at (-23.250000, 29.850000) {\hbox{\tate みやこ}};
\node[Meaning] at (-23.200000, 31.500000) {capital};
\node[Kanji] at (-21.150000, 30.250000) {\textcolor[HTML]{68a4bc}{涼}};
\node[Square] at (-21.150000, 29.750000) {};
\node[Onyomi] at (-21.100000, 29.850000) {\hbox{\tate リョウ}};
\node[Kunyomi] at (-21.200000, 29.850000) {\hbox{\tate すず.しい}};
\node[Meaning] at (-21.150000, 31.500000) {cool};
\node[Kanji] at (-19.100000, 30.250000) {\textcolor[HTML]{d2a293}{景}};
\node[Square] at (-19.100000, 29.750000) {};
\node[Onyomi] at (-19.050000, 29.850000) {\hbox{\tate ケイ}};
\node[Meaning] at (-19.100000, 31.500000) {scene};
\node[Kanji] at (-17.050000, 30.250000) {\textcolor[HTML]{68a4bc}{鯨}};
\node[Square] at (-17.050000, 29.750000) {};
\node[Onyomi] at (-17.000000, 29.850000) {\hbox{\tate ゲイ}};
\node[Kunyomi] at (-17.100000, 29.850000) {\hbox{\tate くじら}};
\node[Meaning] at (-17.050000, 31.500000) {whale};
\node[Kanji] at (-15.000000, 30.250000) {\textcolor[HTML]{c8a59d}{舎}};
\node[Square] at (-15.000000, 29.750000) {};
\node[Onyomi] at (-14.950000, 29.850000) {\hbox{\tate シャ}};
\node[Meaning] at (-15.000000, 31.500000) {cottage};
\node[Kanji] at (-12.950000, 30.250000) {\textcolor[HTML]{d69f8d}{周}};
\node[Square] at (-12.950000, 29.750000) {};
\node[Onyomi] at (-12.900000, 29.850000) {\hbox{\tate シュウ}};
\node[Kunyomi] at (-13.000000, 29.850000) {\hbox{\tate まわ.り}};
\node[Meaning] at (-12.950000, 31.500000) {circumference};
\node[Kanji] at (-10.900000, 30.250000) {\textcolor[HTML]{d2a293}{週}};
\node[Square] at (-10.900000, 29.750000) {};
\node[Onyomi] at (-10.850000, 29.850000) {\hbox{\tate シュウ}};
\node[Meaning] at (-10.900000, 31.500000) {week};
\node[Kanji] at (-8.850000, 30.250000) {\textcolor[HTML]{cd8268}{士}};
\node[Square] at (-8.850000, 29.750000) {};
\node[Onyomi] at (-8.800000, 29.850000) {\hbox{\tate シ}};
\node[Kunyomi] at (-8.900000, 29.850000) {\hbox{\tate さむらい}};
\node[Meaning] at (-8.850000, 31.500000) {samurai};
\node[Kanji] at (-6.800000, 30.250000) {\textcolor[HTML]{d69f8d}{吉}};
\node[Square] at (-6.800000, 29.750000) {};
\node[Onyomi] at (-6.750000, 29.850000) {\hbox{\tate キツ・キチ}};
\node[Kunyomi] at (-6.850000, 29.850000) {\hbox{\tate よし}};
\node[Meaning] at (-6.800000, 31.500000) {good luck};
\node[Kanji] at (-4.750000, 30.250000) {\textcolor[HTML]{68a4bc}{壮}};
\node[Square] at (-4.750000, 29.750000) {};
\node[Onyomi] at (-4.700000, 29.850000) {\hbox{\tate ソウ}};
\node[Meaning] at (-4.750000, 31.500000) {robust};
\node[Kanji] at (-2.700000, 30.250000) {\textcolor[HTML]{a3bac2}{荘}};
\node[Square] at (-2.700000, 29.750000) {};
\node[Onyomi] at (-2.650000, 29.850000) {\hbox{\tate ソウ・ショウ}};
\node[Kunyomi] at (-2.750000, 29.850000) {\hbox{\tate あごそ}};
\node[Meaning] at (-2.700000, 31.500000) {villa};
\node[Kanji] at (-0.650000, 30.250000) {\textcolor[HTML]{cd8268}{売}};
\node[Square] at (-0.650000, 29.750000) {};
\node[Onyomi] at (-0.600000, 29.850000) {\hbox{\tate バイ}};
\node[Kunyomi] at (-0.700000, 29.850000) {\hbox{\tate う}};
\node[Meaning] at (-0.650000, 31.500000) {sell};
\node[Kanji] at (1.400000, 30.250000) {\textcolor[HTML]{830e29}{学}};
\node[Square] at (1.400000, 29.750000) {};
\node[Onyomi] at (1.450000, 29.850000) {\hbox{\tate ガク}};
\node[Kunyomi] at (1.350000, 29.850000) {\hbox{\tate まな.ぶ}};
\node[Meaning] at (1.400000, 31.500000) {study};
\node[Kanji] at (3.450000, 30.250000) {\textcolor[HTML]{c8a59d}{覚}};
\node[Square] at (3.450000, 29.750000) {};
\node[Onyomi] at (3.500000, 29.850000) {\hbox{\tate カク}};
\node[Kunyomi] at (3.400000, 29.850000) {\hbox{\tate おぼ・さ}};
\node[Meaning] at (3.450000, 31.500000) {memorize};
\node[Kanji] at (5.500000, 30.250000) {\textcolor[HTML]{c8a59d}{栄}};
\node[Square] at (5.500000, 29.750000) {};
\node[Onyomi] at (5.550000, 29.850000) {\hbox{\tate エイ}};
\node[Kunyomi] at (5.450000, 29.850000) {\hbox{\tate さか.える}};
\node[Meaning] at (5.500000, 31.500000) {prosperity};
\node[Kanji] at (7.550000, 30.250000) {\textcolor[HTML]{b74029}{書}};
\node[Square] at (7.550000, 29.750000) {};
\node[Onyomi] at (7.600000, 29.850000) {\hbox{\tate ショ}};
\node[Kunyomi] at (7.500000, 29.850000) {\hbox{\tate か.く}};
\node[Meaning] at (7.550000, 31.500000) {write};
\node[Kanji] at (9.600000, 30.250000) {\textcolor[HTML]{d69f8d}{津}};
\node[Square] at (9.600000, 29.750000) {};
\node[Onyomi] at (9.650000, 29.850000) {\hbox{\tate シン}};
\node[Kunyomi] at (9.550000, 29.850000) {\hbox{\tate つ}};
\node[Meaning] at (9.600000, 31.500000) {haven};
\node[Kanji] at (11.650000, 30.250000) {\textcolor[HTML]{b0b0b5}{牧}};
\node[Square] at (11.650000, 29.750000) {};
\node[Onyomi] at (11.700000, 29.850000) {\hbox{\tate ボク}};
\node[Kunyomi] at (11.600000, 29.850000) {\hbox{\tate まき}};
\node[Meaning] at (11.650000, 31.500000) {pasture};
\node[Kanji] at (13.700000, 30.250000) {\textcolor[HTML]{d69f8d}{攻}};
\node[Square] at (13.700000, 29.750000) {};
\node[Onyomi] at (13.750000, 29.850000) {\hbox{\tate コウ}};
\node[Kunyomi] at (13.650000, 29.850000) {\hbox{\tate せ.める}};
\node[Meaning] at (13.700000, 31.500000) {aggression};
\node[Kanji] at (15.750000, 30.250000) {\textcolor[HTML]{d2a293}{敗}};
\node[Square] at (15.750000, 29.750000) {};
\node[Onyomi] at (15.800000, 29.850000) {\hbox{\tate ハイ}};
\node[Kunyomi] at (15.700000, 29.850000) {\hbox{\tate やぶ.れる}};
\node[Meaning] at (15.750000, 31.500000) {failure};
\node[Kanji] at (17.800000, 30.250000) {\textcolor[HTML]{c8a59d}{枚}};
\node[Square] at (17.800000, 29.750000) {};
\node[Onyomi] at (17.850000, 29.850000) {\hbox{\tate マイ}};
\node[Meaning] at (17.800000, 31.500000) {counter: sheets};
\node[Kanji] at (19.850000, 30.250000) {\textcolor[HTML]{d2a293}{故}};
\node[Square] at (19.850000, 29.750000) {};
\node[Onyomi] at (19.900000, 29.850000) {\hbox{\tate コ}};
\node[Kunyomi] at (19.800000, 29.850000) {\hbox{\tate ゆえ}};
\node[Meaning] at (19.850000, 31.500000) {circumstance};
\node[Kanji] at (21.900000, 30.250000) {\textcolor[HTML]{b0b0b5}{敬}};
\node[Square] at (21.900000, 29.750000) {};
\node[Onyomi] at (21.950000, 29.850000) {\hbox{\tate ケイ}};
\node[Kunyomi] at (21.850000, 29.850000) {\hbox{\tate うやま.う}};
\node[Meaning] at (21.900000, 31.500000) {respect};
\node[Kanji] at (23.950000, 30.250000) {\textcolor[HTML]{c36143}{言}};
\node[Square] at (23.950000, 29.750000) {};
\node[Onyomi] at (24.000000, 29.850000) {\hbox{\tate ゲン・ゴン}};
\node[Kunyomi] at (23.900000, 29.850000) {\hbox{\tate い.う・こと}};
\node[Meaning] at (23.950000, 31.500000) {say};
\node[Kanji] at (26.000000, 30.250000) {\textcolor[HTML]{d2a293}{警}};
\node[Square] at (26.000000, 29.750000) {};
\node[Onyomi] at (26.050000, 29.850000) {\hbox{\tate ケイ}};
\node[Meaning] at (26.000000, 31.500000) {warn};
\node[Kanji] at (28.050000, 30.250000) {\textcolor[HTML]{cd8268}{計}};
\node[Square] at (28.050000, 29.750000) {};
\node[Onyomi] at (28.100000, 29.850000) {\hbox{\tate ケイ}};
\node[Kunyomi] at (28.000000, 29.850000) {\hbox{\tate はか.る}};
\node[Meaning] at (28.050000, 31.500000) {measure};
\node[Kanji] at (30.100000, 30.250000) {\textcolor[HTML]{a3bac2}{獄}};
\node[Square] at (30.100000, 29.750000) {};
\node[Onyomi] at (30.150000, 29.850000) {\hbox{\tate ゴク}};
\node[Meaning] at (30.100000, 31.500000) {prison};
\node[Kanji] at (32.150000, 30.250000) {\textcolor[HTML]{91b7c3}{訂}};
\node[Square] at (32.150000, 29.750000) {};
\node[Onyomi] at (32.200000, 29.850000) {\hbox{\tate テイ}};
\node[Meaning] at (32.150000, 31.500000) {revise};
\node[Kanji] at (34.200000, 30.250000) {\textcolor[HTML]{c8a59d}{討}};
\node[Square] at (34.200000, 29.750000) {};
\node[Onyomi] at (34.250000, 29.850000) {\hbox{\tate トウ}};
\node[Meaning] at (34.200000, 31.500000) {chastise};
\node[Kanji] at (36.250000, 30.250000) {\textcolor[HTML]{b0b0b5}{訓}};
\node[Square] at (36.250000, 29.750000) {};
\node[Onyomi] at (36.300000, 29.850000) {\hbox{\tate クン}};
\node[Kunyomi] at (36.200000, 29.850000) {\hbox{\tate よ.む}};
\node[Meaning] at (36.250000, 31.500000) {instruction};
\node[Kanji] at (38.300000, 30.250000) {\textcolor[HTML]{408dba}{詔}};
\node[Square] at (38.300000, 29.750000) {};
\node[Onyomi] at (38.350000, 29.850000) {\hbox{\tate ショウ}};
\node[Kunyomi] at (38.250000, 29.850000) {\hbox{\tate みことのり}};
\node[Meaning] at (38.300000, 31.500000) {imperial edict};
\node[Kanji] at (40.350000, 30.250000) {\textcolor[HTML]{a3bac2}{詰}};
\node[Square] at (40.350000, 29.750000) {};
\node[Onyomi] at (40.400000, 29.850000) {\hbox{\tate キツ・キチ}};
\node[Kunyomi] at (40.300000, 29.850000) {\hbox{\tate つ・づ}};
\node[Meaning] at (40.350000, 31.500000) {stuffed};
\node[Kanji] at (42.400000, 30.250000) {\textcolor[HTML]{cd8268}{話}};
\node[Square] at (42.400000, 29.750000) {};
\node[Onyomi] at (42.450000, 29.850000) {\hbox{\tate ワ}};
\node[Kunyomi] at (42.350000, 29.850000) {\hbox{\tate はな.す}};
\node[Meaning] at (42.400000, 31.500000) {talk};
\node[Kanji] at (44.450000, 30.250000) {\textcolor[HTML]{68a4bc}{詠}};
\node[Square] at (44.450000, 29.750000) {};
\node[Onyomi] at (44.500000, 29.850000) {\hbox{\tate エイ}};
\node[Kunyomi] at (44.400000, 29.850000) {\hbox{\tate よ・うた}};
\node[Meaning] at (44.450000, 31.500000) {compose};
\node[Kanji] at (46.500000, 30.250000) {\textcolor[HTML]{b0b0b5}{詩}};
\node[Square] at (46.500000, 29.750000) {};
\node[Onyomi] at (46.550000, 29.850000) {\hbox{\tate シ}};
\node[Kunyomi] at (46.450000, 29.850000) {\hbox{\tate し}};
\node[Meaning] at (46.500000, 31.500000) {poem};
\node[Kanji] at (48.550000, 30.250000) {\textcolor[HTML]{c36143}{語}};
\node[Square] at (48.550000, 29.750000) {};
\node[Onyomi] at (48.600000, 29.850000) {\hbox{\tate ゴ}};
\node[Kunyomi] at (48.500000, 29.850000) {\hbox{\tate かた.る}};
\node[Meaning] at (48.550000, 31.500000) {language};
\node[Kanji] at (50.600000, 30.250000) {\textcolor[HTML]{d2a293}{読}};
\node[Square] at (50.600000, 29.750000) {};
\node[Onyomi] at (50.650000, 29.850000) {\hbox{\tate トウ・ドク}};
\node[Kunyomi] at (50.550000, 29.850000) {\hbox{\tate よ}};
\node[Meaning] at (50.600000, 31.500000) {read};
\node[Kanji] at (52.650000, 30.250000) {\textcolor[HTML]{cd8268}{調}};
\node[Square] at (52.650000, 29.750000) {};
\node[Onyomi] at (52.700000, 29.850000) {\hbox{\tate チョウ}};
\node[Kunyomi] at (52.600000, 29.850000) {\hbox{\tate しら.べる}};
\node[Meaning] at (52.650000, 31.500000) {investigate};
\node[Kanji] at (54.700000, 30.250000) {\textcolor[HTML]{c8a59d}{談}};
\node[Square] at (54.700000, 29.750000) {};
\node[Onyomi] at (54.750000, 29.850000) {\hbox{\tate ダン}};
\node[Meaning] at (54.700000, 31.500000) {discuss};
\node[Kanji] at (56.750000, 30.250000) {\textcolor[HTML]{68a4bc}{諾}};
\node[Square] at (56.750000, 29.750000) {};
\node[Onyomi] at (56.800000, 29.850000) {\hbox{\tate ダク}};
\node[Meaning] at (56.750000, 31.500000) {agreement};
\node[Meaning] at (-58.050000, 30.350000) {25.93\%};
\node[Kanji] at (-56.000000, 28.200000) {\textcolor[HTML]{68a4bc}{諭}};
\node[Square] at (-56.000000, 27.700000) {};
\node[Onyomi] at (-55.950000, 27.800000) {\hbox{\tate ユ}};
\node[Kunyomi] at (-56.050000, 27.800000) {\hbox{\tate さと}};
\node[Meaning] at (-56.000000, 29.450000) {admonish};
\node[Kanji] at (-53.950000, 28.200000) {\textcolor[HTML]{c36143}{式}};
\node[Square] at (-53.950000, 27.700000) {};
\node[Onyomi] at (-53.900000, 27.800000) {\hbox{\tate シキ}};
\node[Meaning] at (-53.950000, 29.450000) {ritual};
\node[Kanji] at (-51.900000, 28.200000) {\textcolor[HTML]{d69f8d}{試}};
\node[Square] at (-51.900000, 27.700000) {};
\node[Onyomi] at (-51.850000, 27.800000) {\hbox{\tate シ}};
\node[Kunyomi] at (-51.950000, 27.800000) {\hbox{\tate こころ.みる}};
\node[Meaning] at (-51.900000, 29.450000) {try};
\node[Kanji] at (-49.850000, 28.200000) {\textcolor[HTML]{1059be}{弐}};
\node[Square] at (-49.850000, 27.700000) {};
\node[Onyomi] at (-49.800000, 27.800000) {\hbox{\tate ニ}};
\node[Meaning] at (-49.850000, 29.450000) {II, second};
\node[Kanji] at (-47.800000, 28.200000) {\textcolor[HTML]{cd8268}{域}};
\node[Square] at (-47.800000, 27.700000) {};
\node[Onyomi] at (-47.750000, 27.800000) {\hbox{\tate イキ}};
\node[Meaning] at (-47.800000, 29.450000) {region};
\node[Kanji] at (-45.750000, 28.200000) {\textcolor[HTML]{91b7c3}{賊}};
\node[Square] at (-45.750000, 27.700000) {};
\node[Onyomi] at (-45.700000, 27.800000) {\hbox{\tate ゾク}};
\node[Meaning] at (-45.750000, 29.450000) {robber};
\node[Kanji] at (-43.700000, 28.200000) {\textcolor[HTML]{a3bac2}{栽}};
\node[Square] at (-43.700000, 27.700000) {};
\node[Onyomi] at (-43.650000, 27.800000) {\hbox{\tate サイ}};
\node[Meaning] at (-43.700000, 29.450000) {planting};
\node[Kanji] at (-41.650000, 28.200000) {\textcolor[HTML]{cd8268}{載}};
\node[Square] at (-41.650000, 27.700000) {};
\node[Onyomi] at (-41.600000, 27.800000) {\hbox{\tate サイ}};
\node[Kunyomi] at (-41.700000, 27.800000) {\hbox{\tate の.せる}};
\node[Meaning] at (-41.650000, 29.450000) {publish};
\node[Kanji] at (-39.600000, 28.200000) {\textcolor[HTML]{b0b0b5}{茂}};
\node[Square] at (-39.600000, 27.700000) {};
\node[Onyomi] at (-39.550000, 27.800000) {\hbox{\tate モ}};
\node[Kunyomi] at (-39.650000, 27.800000) {\hbox{\tate しげ.る}};
\node[Meaning] at (-39.600000, 29.450000) {luxuriant};
\node[Kanji] at (-37.550000, 28.200000) {\textcolor[HTML]{a11d25}{成}};
\node[Square] at (-37.550000, 27.700000) {};
\node[Onyomi] at (-37.500000, 27.800000) {\hbox{\tate セイ}};
\node[Kunyomi] at (-37.600000, 27.800000) {\hbox{\tate な.る}};
\node[Meaning] at (-37.550000, 29.450000) {become};
\node[Kanji] at (-35.500000, 28.200000) {\textcolor[HTML]{cd8268}{城}};
\node[Square] at (-35.500000, 27.700000) {};
\node[Onyomi] at (-35.450000, 27.800000) {\hbox{\tate ジョウ}};
\node[Kunyomi] at (-35.550000, 27.800000) {\hbox{\tate しろ}};
\node[Meaning] at (-35.500000, 29.450000) {castle};
\node[Kanji] at (-33.450000, 28.200000) {\textcolor[HTML]{a3bac2}{誠}};
\node[Square] at (-33.450000, 27.700000) {};
\node[Onyomi] at (-33.400000, 27.800000) {\hbox{\tate セイ}};
\node[Kunyomi] at (-33.500000, 27.800000) {\hbox{\tate まこと}};
\node[Meaning] at (-33.450000, 29.450000) {sincerity};
\node[Kanji] at (-31.400000, 28.200000) {\textcolor[HTML]{a3bac2}{威}};
\node[Square] at (-31.400000, 27.700000) {};
\node[Onyomi] at (-31.350000, 27.800000) {\hbox{\tate イ}};
\node[Meaning] at (-31.400000, 29.450000) {majesty};
\node[Kanji] at (-29.350000, 28.200000) {\textcolor[HTML]{c8a59d}{滅}};
\node[Square] at (-29.350000, 27.700000) {};
\node[Onyomi] at (-29.300000, 27.800000) {\hbox{\tate メツ}};
\node[Kunyomi] at (-29.400000, 27.800000) {\hbox{\tate ほろ.*}};
\node[Meaning] at (-29.350000, 29.450000) {destroy};
\node[Kanji] at (-27.300000, 28.200000) {\textcolor[HTML]{d2a293}{減}};
\node[Square] at (-27.300000, 27.700000) {};
\node[Onyomi] at (-27.250000, 27.800000) {\hbox{\tate ゲン}};
\node[Kunyomi] at (-27.350000, 27.800000) {\hbox{\tate へ.る}};
\node[Meaning] at (-27.300000, 29.450000) {decrease};
\node[Kanji] at (-25.250000, 28.200000) {\textcolor[HTML]{1e76bb}{桟}};
\node[Square] at (-25.250000, 27.700000) {};
\node[Onyomi] at (-25.200000, 27.800000) {\hbox{\tate サン・セン}};
\node[Kunyomi] at (-25.300000, 27.800000) {\hbox{\tate かけはし}};
\node[Meaning] at (-25.250000, 29.450000) {jetty};
\node[Kanji] at (-23.200000, 28.200000) {\textcolor[HTML]{a3bac2}{銭}};
\node[Square] at (-23.200000, 27.700000) {};
\node[Onyomi] at (-23.150000, 27.800000) {\hbox{\tate セン}};
\node[Kunyomi] at (-23.250000, 27.800000) {\hbox{\tate ぜに}};
\node[Meaning] at (-23.200000, 29.450000) {coin};
\node[Kanji] at (-21.150000, 28.200000) {\textcolor[HTML]{b0b0b5}{浅}};
\node[Square] at (-21.150000, 27.700000) {};
\node[Onyomi] at (-21.100000, 27.800000) {\hbox{\tate セン}};
\node[Kunyomi] at (-21.200000, 27.800000) {\hbox{\tate あさ}};
\node[Meaning] at (-21.150000, 29.450000) {shallow};
\node[Kanji] at (-19.100000, 28.200000) {\textcolor[HTML]{cd8268}{止}};
\node[Square] at (-19.100000, 27.700000) {};
\node[Onyomi] at (-19.050000, 27.800000) {\hbox{\tate シ}};
\node[Kunyomi] at (-19.150000, 27.800000) {\hbox{\tate と.まる}};
\node[Meaning] at (-19.100000, 29.450000) {stop};
\node[Kanji] at (-17.050000, 28.200000) {\textcolor[HTML]{d2a293}{歩}};
\node[Square] at (-17.050000, 27.700000) {};
\node[Onyomi] at (-17.000000, 27.800000) {\hbox{\tate ホ}};
\node[Kunyomi] at (-17.100000, 27.800000) {\hbox{\tate ある.く}};
\node[Meaning] at (-17.050000, 29.450000) {walk};
\node[Kanji] at (-15.000000, 28.200000) {\textcolor[HTML]{a3bac2}{渉}};
\node[Square] at (-15.000000, 27.700000) {};
\node[Onyomi] at (-14.950000, 27.800000) {\hbox{\tate ショウ}};
\node[Kunyomi] at (-15.050000, 27.800000) {\hbox{\tate わた.る}};
\node[Meaning] at (-15.000000, 29.450000) {ford};
\node[Kanji] at (-12.950000, 28.200000) {\textcolor[HTML]{a3bac2}{頻}};
\node[Square] at (-12.950000, 27.700000) {};
\node[Onyomi] at (-12.900000, 27.800000) {\hbox{\tate ヒン}};
\node[Kunyomi] at (-13.000000, 27.800000) {\hbox{\tate しき.りに}};
\node[Meaning] at (-12.950000, 29.450000) {frequent};
\node[Kanji] at (-10.900000, 28.200000) {\textcolor[HTML]{408dba}{肯}};
\node[Square] at (-10.900000, 27.700000) {};
\node[Onyomi] at (-10.850000, 27.800000) {\hbox{\tate コウ}};
\node[Kunyomi] at (-10.950000, 27.800000) {\hbox{\tate がえんじ.る}};
\node[Meaning] at (-10.900000, 29.450000) {agreement};
\node[Kanji] at (-8.850000, 28.200000) {\textcolor[HTML]{d69f8d}{企}};
\node[Square] at (-8.850000, 27.700000) {};
\node[Onyomi] at (-8.800000, 27.800000) {\hbox{\tate キ}};
\node[Kunyomi] at (-8.900000, 27.800000) {\hbox{\tate くわだ.てる}};
\node[Meaning] at (-8.850000, 29.450000) {plan};
\node[Kanji] at (-6.800000, 28.200000) {\textcolor[HTML]{d69f8d}{歴}};
\node[Square] at (-6.800000, 27.700000) {};
\node[Onyomi] at (-6.750000, 27.800000) {\hbox{\tate レキ}};
\node[Kunyomi] at (-6.850000, 27.800000) {\hbox{\tate へ.る}};
\node[Meaning] at (-6.800000, 29.450000) {continuation};
\node[Kanji] at (-4.750000, 28.200000) {\textcolor[HTML]{cd8268}{武}};
\node[Square] at (-4.750000, 27.700000) {};
\node[Onyomi] at (-4.700000, 27.800000) {\hbox{\tate ブ}};
\node[Kunyomi] at (-4.800000, 27.800000) {\hbox{\tate たけ}};
\node[Meaning] at (-4.750000, 29.450000) {military};
\node[Kanji] at (-2.700000, 28.200000) {\textcolor[HTML]{1e76bb}{賦}};
\node[Square] at (-2.700000, 27.700000) {};
\node[Onyomi] at (-2.650000, 27.800000) {\hbox{\tate フ}};
\node[Meaning] at (-2.700000, 29.450000) {levy};
\node[Kanji] at (-0.650000, 28.200000) {\textcolor[HTML]{c36143}{正}};
\node[Square] at (-0.650000, 27.700000) {};
\node[Onyomi] at (-0.600000, 27.800000) {\hbox{\tate セイ・ショウ}};
\node[Kunyomi] at (-0.700000, 27.800000) {\hbox{\tate ただ.しい}};
\node[Meaning] at (-0.650000, 29.450000) {correct};
\node[Kanji] at (1.400000, 28.200000) {\textcolor[HTML]{d2a293}{証}};
\node[Square] at (1.400000, 27.700000) {};
\node[Onyomi] at (1.450000, 27.800000) {\hbox{\tate ショウ}};
\node[Kunyomi] at (1.350000, 27.800000) {\hbox{\tate あかし}};
\node[Meaning] at (1.400000, 29.450000) {evidence};
\node[Kanji] at (3.450000, 28.200000) {\textcolor[HTML]{c36143}{政}};
\node[Square] at (3.450000, 27.700000) {};
\node[Onyomi] at (3.500000, 27.800000) {\hbox{\tate セイ}};
\node[Meaning] at (3.450000, 29.450000) {politics};
\node[Kanji] at (5.500000, 28.200000) {\textcolor[HTML]{a11d25}{定}};
\node[Square] at (5.500000, 27.700000) {};
\node[Onyomi] at (5.550000, 27.800000) {\hbox{\tate テイ・ジョウ}};
\node[Kunyomi] at (5.450000, 27.800000) {\hbox{\tate さだ}};
\node[Meaning] at (5.500000, 29.450000) {determine};
\node[Kanji] at (7.550000, 28.200000) {\textcolor[HTML]{408dba}{錠}};
\node[Square] at (7.550000, 27.700000) {};
\node[Onyomi] at (7.600000, 27.800000) {\hbox{\tate ジョウ}};
\node[Meaning] at (7.550000, 29.450000) {lock};
\node[Kanji] at (9.600000, 28.200000) {\textcolor[HTML]{d69f8d}{走}};
\node[Square] at (9.600000, 27.700000) {};
\node[Onyomi] at (9.650000, 27.800000) {\hbox{\tate ソウ}};
\node[Kunyomi] at (9.550000, 27.800000) {\hbox{\tate はし.る}};
\node[Meaning] at (9.600000, 29.450000) {run};
\node[Kanji] at (11.650000, 28.200000) {\textcolor[HTML]{d2a293}{超}};
\node[Square] at (11.650000, 27.700000) {};
\node[Onyomi] at (11.700000, 27.800000) {\hbox{\tate チョウ}};
\node[Kunyomi] at (11.600000, 27.800000) {\hbox{\tate こ.*}};
\node[Meaning] at (11.650000, 29.450000) {ultra};
\node[Kanji] at (13.700000, 28.200000) {\textcolor[HTML]{91b7c3}{赴}};
\node[Square] at (13.700000, 27.700000) {};
\node[Onyomi] at (13.750000, 27.800000) {\hbox{\tate フ}};
\node[Kunyomi] at (13.650000, 27.800000) {\hbox{\tate おもむ}};
\node[Meaning] at (13.700000, 29.450000) {proceed};
\node[Kanji] at (15.750000, 28.200000) {\textcolor[HTML]{d2a293}{越}};
\node[Square] at (15.750000, 27.700000) {};
\node[Onyomi] at (15.800000, 27.800000) {\hbox{\tate エツ}};
\node[Kunyomi] at (15.700000, 27.800000) {\hbox{\tate こ.*}};
\node[Meaning] at (15.750000, 29.450000) {go beyond};
\node[Kanji] at (17.800000, 28.200000) {\textcolor[HTML]{91b7c3}{是}};
\node[Square] at (17.800000, 27.700000) {};
\node[Onyomi] at (17.850000, 27.800000) {\hbox{\tate ゼ}};
\node[Meaning] at (17.800000, 29.450000) {absolutely};
\node[Kanji] at (19.850000, 28.200000) {\textcolor[HTML]{cd8268}{題}};
\node[Square] at (19.850000, 27.700000) {};
\node[Onyomi] at (19.900000, 27.800000) {\hbox{\tate ダイ}};
\node[Meaning] at (19.850000, 29.450000) {topic};
\node[Kanji] at (21.900000, 28.200000) {\textcolor[HTML]{91b7c3}{堤}};
\node[Square] at (21.900000, 27.700000) {};
\node[Onyomi] at (21.950000, 27.800000) {\hbox{\tate テイ}};
\node[Kunyomi] at (21.850000, 27.800000) {\hbox{\tate つつみ}};
\node[Meaning] at (21.900000, 29.450000) {embankment};
\node[Kanji] at (23.950000, 28.200000) {\textcolor[HTML]{cd8268}{建}};
\node[Square] at (23.950000, 27.700000) {};
\node[Onyomi] at (24.000000, 27.800000) {\hbox{\tate ケン}};
\node[Kunyomi] at (23.900000, 27.800000) {\hbox{\tate た.*}};
\node[Meaning] at (23.950000, 29.450000) {build};
\node[Kanji] at (26.000000, 28.200000) {\textcolor[HTML]{d2a293}{延}};
\node[Square] at (26.000000, 27.700000) {};
\node[Onyomi] at (26.050000, 27.800000) {\hbox{\tate エン}};
\node[Kunyomi] at (25.950000, 27.800000) {\hbox{\tate のば.す}};
\node[Meaning] at (26.000000, 29.450000) {prolong};
\node[Kanji] at (28.050000, 28.200000) {\textcolor[HTML]{b0b0b5}{誕}};
\node[Square] at (28.050000, 27.700000) {};
\node[Onyomi] at (28.100000, 27.800000) {\hbox{\tate タン}};
\node[Meaning] at (28.050000, 29.450000) {birth};
\node[Kanji] at (30.100000, 28.200000) {\textcolor[HTML]{a3bac2}{礎}};
\node[Square] at (30.100000, 27.700000) {};
\node[Onyomi] at (30.150000, 27.800000) {\hbox{\tate ソ}};
\node[Kunyomi] at (30.050000, 27.800000) {\hbox{\tate いしずえ}};
\node[Meaning] at (30.100000, 29.450000) {foundation};
\node[Kanji] at (32.150000, 28.200000) {\textcolor[HTML]{408dba}{婿}};
\node[Square] at (32.150000, 27.700000) {};
\node[Onyomi] at (32.200000, 27.800000) {\hbox{\tate セイ}};
\node[Kunyomi] at (32.100000, 27.800000) {\hbox{\tate むこ}};
\node[Meaning] at (32.150000, 29.450000) {groom};
\node[Kanji] at (34.200000, 28.200000) {\textcolor[HTML]{b0b0b5}{衣}};
\node[Square] at (34.200000, 27.700000) {};
\node[Onyomi] at (34.250000, 27.800000) {\hbox{\tate イ・エ}};
\node[Kunyomi] at (34.150000, 27.800000) {\hbox{\tate ころも}};
\node[Meaning] at (34.200000, 29.450000) {clothes};
\node[Kanji] at (36.250000, 28.200000) {\textcolor[HTML]{d2a293}{裁}};
\node[Square] at (36.250000, 27.700000) {};
\node[Onyomi] at (36.300000, 27.800000) {\hbox{\tate サイ}};
\node[Kunyomi] at (36.200000, 27.800000) {\hbox{\tate さば.く}};
\node[Meaning] at (36.250000, 29.450000) {judge};
\node[Kanji] at (38.300000, 28.200000) {\textcolor[HTML]{cd8268}{装}};
\node[Square] at (38.300000, 27.700000) {};
\node[Onyomi] at (38.350000, 27.800000) {\hbox{\tate ソウ・ショウ}};
\node[Kunyomi] at (38.250000, 27.800000) {\hbox{\tate よそお.う}};
\node[Meaning] at (38.300000, 29.450000) {attire};
\node[Kanji] at (40.350000, 28.200000) {\textcolor[HTML]{c8a59d}{裏}};
\node[Square] at (40.350000, 27.700000) {};
\node[Onyomi] at (40.400000, 27.800000) {\hbox{\tate リ}};
\node[Kunyomi] at (40.300000, 27.800000) {\hbox{\tate うら}};
\node[Meaning] at (40.350000, 29.450000) {backside};
\node[Kanji] at (42.400000, 28.200000) {\textcolor[HTML]{c8a59d}{壊}};
\node[Square] at (42.400000, 27.700000) {};
\node[Onyomi] at (42.450000, 27.800000) {\hbox{\tate カイ}};
\node[Kunyomi] at (42.350000, 27.800000) {\hbox{\tate こわ.*}};
\node[Meaning] at (42.400000, 29.450000) {break};
\node[Kanji] at (44.450000, 28.200000) {\textcolor[HTML]{408dba}{哀}};
\node[Square] at (44.450000, 27.700000) {};
\node[Onyomi] at (44.500000, 27.800000) {\hbox{\tate アイ}};
\node[Kunyomi] at (44.400000, 27.800000) {\hbox{\tate あわ.れ*}};
\node[Meaning] at (44.450000, 29.450000) {pathetic};
\node[Kanji] at (46.500000, 28.200000) {\textcolor[HTML]{d2a293}{遠}};
\node[Square] at (46.500000, 27.700000) {};
\node[Onyomi] at (46.550000, 27.800000) {\hbox{\tate エン}};
\node[Kunyomi] at (46.450000, 27.800000) {\hbox{\tate とお}};
\node[Meaning] at (46.500000, 29.450000) {far};
\node[Kanji] at (48.550000, 28.200000) {\textcolor[HTML]{91b7c3}{猿}};
\node[Square] at (48.550000, 27.700000) {};
\node[Onyomi] at (48.600000, 27.800000) {\hbox{\tate エン}};
\node[Kunyomi] at (48.500000, 27.800000) {\hbox{\tate さる}};
\node[Meaning] at (48.550000, 29.450000) {monkey};
\node[Kanji] at (50.600000, 28.200000) {\textcolor[HTML]{b74029}{初}};
\node[Square] at (50.600000, 27.700000) {};
\node[Onyomi] at (50.650000, 27.800000) {\hbox{\tate ショ}};
\node[Kunyomi] at (50.550000, 27.800000) {\hbox{\tate はじ・はつ}};
\node[Meaning] at (50.600000, 29.450000) {first};
\node[Kanji] at (52.650000, 28.200000) {\textcolor[HTML]{d2a293}{布}};
\node[Square] at (52.650000, 27.700000) {};
\node[Onyomi] at (52.700000, 27.800000) {\hbox{\tate フ}};
\node[Kunyomi] at (52.600000, 27.800000) {\hbox{\tate ぬの}};
\node[Meaning] at (52.650000, 29.450000) {cloth};
\node[Kanji] at (54.700000, 28.200000) {\textcolor[HTML]{68a4bc}{帆}};
\node[Square] at (54.700000, 27.700000) {};
\node[Onyomi] at (54.750000, 27.800000) {\hbox{\tate ハン}};
\node[Kunyomi] at (54.650000, 27.800000) {\hbox{\tate ほ}};
\node[Meaning] at (54.700000, 29.450000) {sail};
\node[Kanji] at (56.750000, 28.200000) {\textcolor[HTML]{d2a293}{幅}};
\node[Square] at (56.750000, 27.700000) {};
\node[Onyomi] at (56.800000, 27.800000) {\hbox{\tate フク}};
\node[Kunyomi] at (56.700000, 27.800000) {\hbox{\tate はば}};
\node[Meaning] at (56.750000, 29.450000) {width};
\node[Meaning] at (-58.050000, 28.300000) {28.99\%};
\node[Kanji] at (-56.000000, 26.150000) {\textcolor[HTML]{68a4bc}{帽}};
\node[Square] at (-56.000000, 25.650000) {};
\node[Onyomi] at (-55.950000, 25.750000) {\hbox{\tate ボウ}};
\node[Meaning] at (-56.000000, 27.400000) {hat};
\node[Kanji] at (-53.950000, 26.150000) {\textcolor[HTML]{d2a293}{幕}};
\node[Square] at (-53.950000, 25.650000) {};
\node[Onyomi] at (-53.900000, 25.750000) {\hbox{\tate マク・バク}};
\node[Kunyomi] at (-54.000000, 25.750000) {\hbox{\tate とばり}};
\node[Meaning] at (-53.950000, 27.400000) {curtain};
\node[Kanji] at (-51.900000, 26.150000) {\textcolor[HTML]{91b7c3}{錦}};
\node[Square] at (-51.900000, 25.650000) {};
\node[Onyomi] at (-51.850000, 25.750000) {\hbox{\tate キン}};
\node[Kunyomi] at (-51.950000, 25.750000) {\hbox{\tate にしき}};
\node[Meaning] at (-51.900000, 27.400000) {brocade};
\node[Kanji] at (-49.850000, 26.150000) {\textcolor[HTML]{a11d25}{市}};
\node[Square] at (-49.850000, 25.650000) {};
\node[Onyomi] at (-49.800000, 25.750000) {\hbox{\tate シ}};
\node[Kunyomi] at (-49.900000, 25.750000) {\hbox{\tate いち}};
\node[Meaning] at (-49.850000, 27.400000) {city};
\node[Kanji] at (-47.800000, 26.150000) {\textcolor[HTML]{b0b0b5}{姉}};
\node[Square] at (-47.800000, 25.650000) {};
\node[Onyomi] at (-47.750000, 25.750000) {\hbox{\tate シ}};
\node[Kunyomi] at (-47.850000, 25.750000) {\hbox{\tate お.ねえ}};
\node[Meaning] at (-47.800000, 27.400000) {older sister};
\node[Kanji] at (-45.750000, 26.150000) {\textcolor[HTML]{68a4bc}{肺}};
\node[Square] at (-45.750000, 25.650000) {};
\node[Onyomi] at (-45.700000, 25.750000) {\hbox{\tate ハイ}};
\node[Kunyomi] at (-45.800000, 25.750000) {\hbox{\tate はい}};
\node[Meaning] at (-45.750000, 27.400000) {lung};
\node[Kanji] at (-43.700000, 26.150000) {\textcolor[HTML]{d69f8d}{帯}};
\node[Square] at (-43.700000, 25.650000) {};
\node[Onyomi] at (-43.650000, 25.750000) {\hbox{\tate タイ}};
\node[Kunyomi] at (-43.750000, 25.750000) {\hbox{\tate おび}};
\node[Meaning] at (-43.700000, 27.400000) {belt};
\node[Kanji] at (-41.650000, 26.150000) {\textcolor[HTML]{a3bac2}{滞}};
\node[Square] at (-41.650000, 25.650000) {};
\node[Onyomi] at (-41.600000, 25.750000) {\hbox{\tate タイ}};
\node[Kunyomi] at (-41.700000, 25.750000) {\hbox{\tate とどこお.る}};
\node[Meaning] at (-41.650000, 27.400000) {stagnate};
\node[Kanji] at (-39.600000, 26.150000) {\textcolor[HTML]{b0b0b5}{刺}};
\node[Square] at (-39.600000, 25.650000) {};
\node[Onyomi] at (-39.550000, 25.750000) {\hbox{\tate シ}};
\node[Kunyomi] at (-39.650000, 25.750000) {\hbox{\tate さ.*}};
\node[Meaning] at (-39.600000, 27.400000) {stab};
\node[Kanji] at (-37.550000, 26.150000) {\textcolor[HTML]{c36143}{制}};
\node[Square] at (-37.550000, 25.650000) {};
\node[Onyomi] at (-37.500000, 25.750000) {\hbox{\tate セイ}};
\node[Meaning] at (-37.550000, 27.400000) {control};
\node[Kanji] at (-35.500000, 26.150000) {\textcolor[HTML]{cd8268}{製}};
\node[Square] at (-35.500000, 25.650000) {};
\node[Onyomi] at (-35.450000, 25.750000) {\hbox{\tate セイ}};
\node[Meaning] at (-35.500000, 27.400000) {manufacture};
\node[Kanji] at (-33.450000, 26.150000) {\textcolor[HTML]{cd8268}{転}};
\node[Square] at (-33.450000, 25.650000) {};
\node[Onyomi] at (-33.400000, 25.750000) {\hbox{\tate テン}};
\node[Kunyomi] at (-33.500000, 25.750000) {\hbox{\tate ころ.ぶ}};
\node[Meaning] at (-33.450000, 27.400000) {revolve};
\node[Kanji] at (-31.400000, 26.150000) {\textcolor[HTML]{d2a293}{芸}};
\node[Square] at (-31.400000, 25.650000) {};
\node[Onyomi] at (-31.350000, 25.750000) {\hbox{\tate ゲイ}};
\node[Meaning] at (-31.400000, 27.400000) {acting};
\node[Kanji] at (-29.350000, 26.150000) {\textcolor[HTML]{b0b0b5}{雨}};
\node[Square] at (-29.350000, 25.650000) {};
\node[Onyomi] at (-29.300000, 25.750000) {\hbox{\tate ウ}};
\node[Kunyomi] at (-29.400000, 25.750000) {\hbox{\tate あめ・あま}};
\node[Meaning] at (-29.350000, 27.400000) {rain};
\node[Kanji] at (-27.300000, 26.150000) {\textcolor[HTML]{c8a59d}{雲}};
\node[Square] at (-27.300000, 25.650000) {};
\node[Onyomi] at (-27.250000, 25.750000) {\hbox{\tate ウン}};
\node[Kunyomi] at (-27.350000, 25.750000) {\hbox{\tate くも}};
\node[Meaning] at (-27.300000, 27.400000) {cloud};
\node[Kanji] at (-25.250000, 26.150000) {\textcolor[HTML]{408dba}{曇}};
\node[Square] at (-25.250000, 25.650000) {};
\node[Kunyomi] at (-25.300000, 25.750000) {\hbox{\tate くも}};
\node[Meaning] at (-25.250000, 27.400000) {cloudy};
\node[Kanji] at (-23.200000, 26.150000) {\textcolor[HTML]{b0b0b5}{雷}};
\node[Square] at (-23.200000, 25.650000) {};
\node[Onyomi] at (-23.150000, 25.750000) {\hbox{\tate ライ}};
\node[Kunyomi] at (-23.250000, 25.750000) {\hbox{\tate かみなり}};
\node[Meaning] at (-23.200000, 27.400000) {thunder};
\node[Kanji] at (-21.150000, 26.150000) {\textcolor[HTML]{408dba}{霜}};
\node[Square] at (-21.150000, 25.650000) {};
\node[Kunyomi] at (-21.200000, 25.750000) {\hbox{\tate しも}};
\node[Meaning] at (-21.150000, 27.400000) {frost};
\node[Kanji] at (-19.100000, 26.150000) {\textcolor[HTML]{b0b0b5}{冬}};
\node[Square] at (-19.100000, 25.650000) {};
\node[Onyomi] at (-19.050000, 25.750000) {\hbox{\tate トウ}};
\node[Kunyomi] at (-19.150000, 25.750000) {\hbox{\tate ふゆ}};
\node[Meaning] at (-19.100000, 27.400000) {winter};
\node[Kanji] at (-17.050000, 26.150000) {\textcolor[HTML]{cd8268}{天}};
\node[Square] at (-17.050000, 25.650000) {};
\node[Onyomi] at (-17.000000, 25.750000) {\hbox{\tate テン}};
\node[Kunyomi] at (-17.100000, 25.750000) {\hbox{\tate あま}};
\node[Meaning] at (-17.050000, 27.400000) {heaven};
\node[Kanji] at (-15.000000, 26.150000) {\textcolor[HTML]{d69f8d}{橋}};
\node[Square] at (-15.000000, 25.650000) {};
\node[Onyomi] at (-14.950000, 25.750000) {\hbox{\tate キョウ}};
\node[Kunyomi] at (-15.050000, 25.750000) {\hbox{\tate はし}};
\node[Meaning] at (-15.000000, 27.400000) {bridge};
\node[Kanji] at (-12.950000, 26.150000) {\textcolor[HTML]{b74029}{立}};
\node[Square] at (-12.950000, 25.650000) {};
\node[Onyomi] at (-12.900000, 25.750000) {\hbox{\tate リツ}};
\node[Kunyomi] at (-13.000000, 25.750000) {\hbox{\tate た.つ}};
\node[Meaning] at (-12.950000, 27.400000) {stand};
\node[Kanji] at (-10.900000, 26.150000) {\textcolor[HTML]{68a4bc}{泣}};
\node[Square] at (-10.900000, 25.650000) {};
\node[Onyomi] at (-10.850000, 25.750000) {\hbox{\tate キュウ}};
\node[Kunyomi] at (-10.950000, 25.750000) {\hbox{\tate な}};
\node[Meaning] at (-10.900000, 27.400000) {cry};
\node[Kanji] at (-8.850000, 26.150000) {\textcolor[HTML]{d2a293}{章}};
\node[Square] at (-8.850000, 25.650000) {};
\node[Onyomi] at (-8.800000, 25.750000) {\hbox{\tate ショウ}};
\node[Meaning] at (-8.850000, 27.400000) {chapter};
\node[Kanji] at (-6.800000, 26.150000) {\textcolor[HTML]{d69f8d}{競}};
\node[Square] at (-6.800000, 25.650000) {};
\node[Onyomi] at (-6.750000, 25.750000) {\hbox{\tate キョウ}};
\node[Kunyomi] at (-6.850000, 25.750000) {\hbox{\tate きそ.*}};
\node[Meaning] at (-6.800000, 27.400000) {compete};
\node[Kanji] at (-4.750000, 26.150000) {\textcolor[HTML]{d2a293}{帝}};
\node[Square] at (-4.750000, 25.650000) {};
\node[Onyomi] at (-4.700000, 25.750000) {\hbox{\tate テイ}};
\node[Kunyomi] at (-4.800000, 25.750000) {\hbox{\tate みかど}};
\node[Meaning] at (-4.750000, 27.400000) {sovereign};
\node[Kanji] at (-2.700000, 26.150000) {\textcolor[HTML]{b0b0b5}{童}};
\node[Square] at (-2.700000, 25.650000) {};
\node[Onyomi] at (-2.650000, 25.750000) {\hbox{\tate ドウ}};
\node[Meaning] at (-2.700000, 27.400000) {juvenile};
\node[Kanji] at (-0.650000, 26.150000) {\textcolor[HTML]{408dba}{瞳}};
\node[Square] at (-0.650000, 25.650000) {};
\node[Onyomi] at (-0.600000, 25.750000) {\hbox{\tate トウ・ドウ}};
\node[Kunyomi] at (-0.700000, 25.750000) {\hbox{\tate ひとみ}};
\node[Meaning] at (-0.650000, 27.400000) {pupil};
\node[Kanji] at (1.400000, 26.150000) {\textcolor[HTML]{68a4bc}{鐘}};
\node[Square] at (1.400000, 25.650000) {};
\node[Onyomi] at (1.450000, 25.750000) {\hbox{\tate ショウ}};
\node[Kunyomi] at (1.350000, 25.750000) {\hbox{\tate かね}};
\node[Meaning] at (1.400000, 27.400000) {bell};
\node[Kanji] at (3.450000, 26.150000) {\textcolor[HTML]{d69f8d}{商}};
\node[Square] at (3.450000, 25.650000) {};
\node[Onyomi] at (3.500000, 25.750000) {\hbox{\tate ショウ}};
\node[Kunyomi] at (3.400000, 25.750000) {\hbox{\tate あきな.い}};
\node[Meaning] at (3.450000, 27.400000) {merchandise};
\node[Kanji] at (5.500000, 26.150000) {\textcolor[HTML]{91b7c3}{嫡}};
\node[Square] at (5.500000, 25.650000) {};
\node[Onyomi] at (5.550000, 25.750000) {\hbox{\tate チャク}};
\node[Meaning] at (5.500000, 27.400000) {legitimate wife};
\node[Kanji] at (7.550000, 26.150000) {\textcolor[HTML]{d2a293}{適}};
\node[Square] at (7.550000, 25.650000) {};
\node[Onyomi] at (7.600000, 25.750000) {\hbox{\tate テキ}};
\node[Meaning] at (7.550000, 27.400000) {suitable};
\node[Kanji] at (9.600000, 26.150000) {\textcolor[HTML]{1e76bb}{滴}};
\node[Square] at (9.600000, 25.650000) {};
\node[Onyomi] at (9.650000, 25.750000) {\hbox{\tate テキ}};
\node[Kunyomi] at (9.550000, 25.750000) {\hbox{\tate したた.る}};
\node[Meaning] at (9.600000, 27.400000) {drip};
\node[Kanji] at (11.650000, 26.150000) {\textcolor[HTML]{c8a59d}{敵}};
\node[Square] at (11.650000, 25.650000) {};
\node[Onyomi] at (11.700000, 25.750000) {\hbox{\tate テキ}};
\node[Kunyomi] at (11.600000, 25.750000) {\hbox{\tate かな.う}};
\node[Meaning] at (11.650000, 27.400000) {enemy};
\node[Kanji] at (13.700000, 26.150000) {\textcolor[HTML]{c36143}{北}};
\node[Square] at (13.700000, 25.650000) {};
\node[Onyomi] at (13.750000, 25.750000) {\hbox{\tate ホク}};
\node[Kunyomi] at (13.650000, 25.750000) {\hbox{\tate きた}};
\node[Meaning] at (13.700000, 27.400000) {north};
\node[Kanji] at (15.750000, 26.150000) {\textcolor[HTML]{c8a59d}{背}};
\node[Square] at (15.750000, 25.650000) {};
\node[Onyomi] at (15.800000, 25.750000) {\hbox{\tate ハイ}};
\node[Kunyomi] at (15.700000, 25.750000) {\hbox{\tate せ・そむ.く}};
\node[Meaning] at (15.750000, 27.400000) {back};
\node[Kanji] at (17.800000, 26.150000) {\textcolor[HTML]{d69f8d}{比}};
\node[Square] at (17.800000, 25.650000) {};
\node[Onyomi] at (17.850000, 25.750000) {\hbox{\tate ヒ}};
\node[Kunyomi] at (17.750000, 25.750000) {\hbox{\tate くら.べる}};
\node[Meaning] at (17.800000, 27.400000) {compare};
\node[Kanji] at (19.850000, 26.150000) {\textcolor[HTML]{91b7c3}{昆}};
\node[Square] at (19.850000, 25.650000) {};
\node[Onyomi] at (19.900000, 25.750000) {\hbox{\tate コン}};
\node[Meaning] at (19.850000, 27.400000) {descendants};
\node[Kanji] at (21.900000, 26.150000) {\textcolor[HTML]{a3bac2}{皆}};
\node[Square] at (21.900000, 25.650000) {};
\node[Onyomi] at (21.950000, 25.750000) {\hbox{\tate カイ}};
\node[Kunyomi] at (21.850000, 25.750000) {\hbox{\tate みな・みんな}};
\node[Meaning] at (21.900000, 27.400000) {all};
\node[Kanji] at (23.950000, 26.150000) {\textcolor[HTML]{c8a59d}{混}};
\node[Square] at (23.950000, 25.650000) {};
\node[Onyomi] at (24.000000, 25.750000) {\hbox{\tate コン}};
\node[Kunyomi] at (23.900000, 25.750000) {\hbox{\tate ま.*}};
\node[Meaning] at (23.950000, 27.400000) {mix};
\node[Kanji] at (26.000000, 26.150000) {\textcolor[HTML]{1e76bb}{渇}};
\node[Square] at (26.000000, 25.650000) {};
\node[Onyomi] at (26.050000, 25.750000) {\hbox{\tate カツ}};
\node[Kunyomi] at (25.950000, 25.750000) {\hbox{\tate かわ}};
\node[Meaning] at (26.000000, 27.400000) {thirst};
\node[Kanji] at (28.050000, 26.150000) {\textcolor[HTML]{1e76bb}{謁}};
\node[Square] at (28.050000, 25.650000) {};
\node[Onyomi] at (28.100000, 25.750000) {\hbox{\tate エツ}};
\node[Meaning] at (28.050000, 27.400000) {audience};
\node[Kanji] at (30.100000, 26.150000) {\textcolor[HTML]{91b7c3}{褐}};
\node[Square] at (30.100000, 25.650000) {};
\node[Onyomi] at (30.150000, 25.750000) {\hbox{\tate カツ}};
\node[Meaning] at (30.100000, 27.400000) {brown};
\node[Kanji] at (32.150000, 26.150000) {\textcolor[HTML]{1e76bb}{喝}};
\node[Square] at (32.150000, 25.650000) {};
\node[Onyomi] at (32.200000, 25.750000) {\hbox{\tate カツ}};
\node[Meaning] at (32.150000, 27.400000) {scold};
\node[Kanji] at (34.200000, 26.150000) {\textcolor[HTML]{a3bac2}{旨}};
\node[Square] at (34.200000, 25.650000) {};
\node[Onyomi] at (34.250000, 25.750000) {\hbox{\tate シ}};
\node[Kunyomi] at (34.150000, 25.750000) {\hbox{\tate うま.い}};
\node[Meaning] at (34.200000, 27.400000) {point};
\node[Kanji] at (36.250000, 26.150000) {\textcolor[HTML]{91b7c3}{脂}};
\node[Square] at (36.250000, 25.650000) {};
\node[Onyomi] at (36.300000, 25.750000) {\hbox{\tate シ}};
\node[Kunyomi] at (36.200000, 25.750000) {\hbox{\tate あぶら}};
\node[Meaning] at (36.250000, 27.400000) {fat};
\node[Kanji] at (38.300000, 26.150000) {\textcolor[HTML]{408dba}{壱}};
\node[Square] at (38.300000, 25.650000) {};
\node[Onyomi] at (38.350000, 25.750000) {\hbox{\tate イチ}};
\node[Meaning] at (38.300000, 27.400000) {1 (legal)};
\node[Kanji] at (40.350000, 26.150000) {\textcolor[HTML]{d2a293}{毎}};
\node[Square] at (40.350000, 25.650000) {};
\node[Onyomi] at (40.400000, 25.750000) {\hbox{\tate マイ}};
\node[Kunyomi] at (40.300000, 25.750000) {\hbox{\tate ごと}};
\node[Meaning] at (40.350000, 27.400000) {every};
\node[Kanji] at (42.400000, 26.150000) {\textcolor[HTML]{91b7c3}{敏}};
\node[Square] at (42.400000, 25.650000) {};
\node[Onyomi] at (42.450000, 25.750000) {\hbox{\tate ビン}};
\node[Meaning] at (42.400000, 27.400000) {alert};
\node[Kanji] at (44.450000, 26.150000) {\textcolor[HTML]{b0b0b5}{梅}};
\node[Square] at (44.450000, 25.650000) {};
\node[Onyomi] at (44.500000, 25.750000) {\hbox{\tate バイ}};
\node[Kunyomi] at (44.400000, 25.750000) {\hbox{\tate うめ}};
\node[Meaning] at (44.450000, 27.400000) {ume};
\node[Kanji] at (46.500000, 26.150000) {\textcolor[HTML]{c36143}{海}};
\node[Square] at (46.500000, 25.650000) {};
\node[Onyomi] at (46.550000, 25.750000) {\hbox{\tate カイ}};
\node[Kunyomi] at (46.450000, 25.750000) {\hbox{\tate うみ}};
\node[Meaning] at (46.500000, 27.400000) {sea};
\node[Kanji] at (48.550000, 26.150000) {\textcolor[HTML]{1e76bb}{乞}};
\node[Square] at (48.550000, 25.650000) {};
\node[Kunyomi] at (48.500000, 25.750000) {\hbox{\tate こ.う}};
\node[Meaning] at (48.550000, 27.400000) {beg};
\node[Kanji] at (50.600000, 26.150000) {\textcolor[HTML]{a3bac2}{乾}};
\node[Square] at (50.600000, 25.650000) {};
\node[Onyomi] at (50.650000, 25.750000) {\hbox{\tate カン}};
\node[Kunyomi] at (50.550000, 25.750000) {\hbox{\tate かわ・ほ}};
\node[Meaning] at (50.600000, 27.400000) {dry};
\node[Kanji] at (52.650000, 26.150000) {\textcolor[HTML]{b0b0b5}{腹}};
\node[Square] at (52.650000, 25.650000) {};
\node[Onyomi] at (52.700000, 25.750000) {\hbox{\tate フク}};
\node[Kunyomi] at (52.600000, 25.750000) {\hbox{\tate はら・なか}};
\node[Meaning] at (52.650000, 27.400000) {belly};
\node[Kanji] at (54.700000, 26.150000) {\textcolor[HTML]{d2a293}{複}};
\node[Square] at (54.700000, 25.650000) {};
\node[Onyomi] at (54.750000, 25.750000) {\hbox{\tate フク}};
\node[Meaning] at (54.700000, 27.400000) {duplicate};
\node[Kanji] at (56.750000, 26.150000) {\textcolor[HTML]{b0b0b5}{欠}};
\node[Square] at (56.750000, 25.650000) {};
\node[Onyomi] at (56.800000, 25.750000) {\hbox{\tate ケツ}};
\node[Kunyomi] at (56.700000, 25.750000) {\hbox{\tate か.ける}};
\node[Meaning] at (56.750000, 27.400000) {lack};
\node[Meaning] at (-58.050000, 26.250000) {31.35\%};
\node[Kanji] at (-56.000000, 24.100000) {\textcolor[HTML]{b0b0b5}{吹}};
\node[Square] at (-56.000000, 23.600000) {};
\node[Onyomi] at (-55.950000, 23.700000) {\hbox{\tate スイ}};
\node[Kunyomi] at (-56.050000, 23.700000) {\hbox{\tate ふ}};
\node[Meaning] at (-56.000000, 25.350000) {blow};
\node[Kanji] at (-53.950000, 24.100000) {\textcolor[HTML]{408dba}{炊}};
\node[Square] at (-53.950000, 23.600000) {};
\node[Onyomi] at (-53.900000, 23.700000) {\hbox{\tate スイ}};
\node[Kunyomi] at (-54.000000, 23.700000) {\hbox{\tate た.く}};
\node[Meaning] at (-53.950000, 25.350000) {cook};
\node[Kanji] at (-51.900000, 24.100000) {\textcolor[HTML]{d69f8d}{歌}};
\node[Square] at (-51.900000, 23.600000) {};
\node[Onyomi] at (-51.850000, 23.700000) {\hbox{\tate カ}};
\node[Kunyomi] at (-51.950000, 23.700000) {\hbox{\tate うた}};
\node[Meaning] at (-51.900000, 25.350000) {song};
\node[Kanji] at (-49.850000, 24.100000) {\textcolor[HTML]{91b7c3}{軟}};
\node[Square] at (-49.850000, 23.600000) {};
\node[Onyomi] at (-49.800000, 23.700000) {\hbox{\tate ナン}};
\node[Kunyomi] at (-49.900000, 23.700000) {\hbox{\tate やわ}};
\node[Meaning] at (-49.850000, 25.350000) {soft};
\node[Kanji] at (-47.800000, 24.100000) {\textcolor[HTML]{c36143}{次}};
\node[Square] at (-47.800000, 23.600000) {};
\node[Onyomi] at (-47.750000, 23.700000) {\hbox{\tate ジ}};
\node[Kunyomi] at (-47.850000, 23.700000) {\hbox{\tate つぎ}};
\node[Meaning] at (-47.800000, 25.350000) {next};
\node[Kanji] at (-45.750000, 24.100000) {\textcolor[HTML]{a3bac2}{茨}};
\node[Square] at (-45.750000, 23.600000) {};
\node[Onyomi] at (-45.700000, 23.700000) {\hbox{\tate シ・ジ}};
\node[Kunyomi] at (-45.800000, 23.700000) {\hbox{\tate いばら}};
\node[Meaning] at (-45.750000, 25.350000) {briar};
\node[Kanji] at (-43.700000, 24.100000) {\textcolor[HTML]{d69f8d}{資}};
\node[Square] at (-43.700000, 23.600000) {};
\node[Onyomi] at (-43.650000, 23.700000) {\hbox{\tate シ}};
\node[Meaning] at (-43.700000, 25.350000) {resources};
\node[Kanji] at (-41.650000, 24.100000) {\textcolor[HTML]{c8a59d}{姿}};
\node[Square] at (-41.650000, 23.600000) {};
\node[Onyomi] at (-41.600000, 23.700000) {\hbox{\tate シ}};
\node[Kunyomi] at (-41.700000, 23.700000) {\hbox{\tate すがた}};
\node[Meaning] at (-41.650000, 25.350000) {figure};
\node[Kanji] at (-39.600000, 24.100000) {\textcolor[HTML]{1e76bb}{諮}};
\node[Square] at (-39.600000, 23.600000) {};
\node[Onyomi] at (-39.550000, 23.700000) {\hbox{\tate シ}};
\node[Kunyomi] at (-39.650000, 23.700000) {\hbox{\tate はか.る}};
\node[Meaning] at (-39.600000, 25.350000) {consult};
\node[Kanji] at (-37.550000, 24.100000) {\textcolor[HTML]{68a4bc}{賠}};
\node[Square] at (-37.550000, 23.600000) {};
\node[Onyomi] at (-37.500000, 23.700000) {\hbox{\tate バイ}};
\node[Meaning] at (-37.550000, 25.350000) {compensation};
\node[Kanji] at (-35.500000, 24.100000) {\textcolor[HTML]{a3bac2}{培}};
\node[Square] at (-35.500000, 23.600000) {};
\node[Onyomi] at (-35.450000, 23.700000) {\hbox{\tate バイ}};
\node[Kunyomi] at (-35.550000, 23.700000) {\hbox{\tate つちか.う}};
\node[Meaning] at (-35.500000, 25.350000) {cultivate};
\node[Kanji] at (-33.450000, 24.100000) {\textcolor[HTML]{408dba}{剖}};
\node[Square] at (-33.450000, 23.600000) {};
\node[Onyomi] at (-33.400000, 23.700000) {\hbox{\tate ボウ}};
\node[Meaning] at (-33.450000, 25.350000) {divide};
\node[Kanji] at (-31.400000, 24.100000) {\textcolor[HTML]{cd8268}{音}};
\node[Square] at (-31.400000, 23.600000) {};
\node[Onyomi] at (-31.350000, 23.700000) {\hbox{\tate オン}};
\node[Kunyomi] at (-31.450000, 23.700000) {\hbox{\tate おと・ね}};
\node[Meaning] at (-31.400000, 25.350000) {sound};
\node[Kanji] at (-29.350000, 24.100000) {\textcolor[HTML]{b0b0b5}{暗}};
\node[Square] at (-29.350000, 23.600000) {};
\node[Onyomi] at (-29.300000, 23.700000) {\hbox{\tate アン}};
\node[Kunyomi] at (-29.400000, 23.700000) {\hbox{\tate くら.い}};
\node[Meaning] at (-29.350000, 25.350000) {dark};
\node[Kanji] at (-27.300000, 24.100000) {\textcolor[HTML]{68a4bc}{韻}};
\node[Square] at (-27.300000, 23.600000) {};
\node[Onyomi] at (-27.250000, 23.700000) {\hbox{\tate イン}};
\node[Meaning] at (-27.300000, 25.350000) {rhyme};
\node[Kanji] at (-25.250000, 24.100000) {\textcolor[HTML]{d2a293}{識}};
\node[Square] at (-25.250000, 23.600000) {};
\node[Onyomi] at (-25.200000, 23.700000) {\hbox{\tate シキ}};
\node[Meaning] at (-25.250000, 25.350000) {discerning};
\node[Kanji] at (-23.200000, 24.100000) {\textcolor[HTML]{b0b0b5}{鏡}};
\node[Square] at (-23.200000, 23.600000) {};
\node[Onyomi] at (-23.150000, 23.700000) {\hbox{\tate キョウ}};
\node[Kunyomi] at (-23.250000, 23.700000) {\hbox{\tate かがみ}};
\node[Meaning] at (-23.200000, 25.350000) {mirror};
\node[Kanji] at (-21.150000, 24.100000) {\textcolor[HTML]{d69f8d}{境}};
\node[Square] at (-21.150000, 23.600000) {};
\node[Onyomi] at (-21.100000, 23.700000) {\hbox{\tate キョウ}};
\node[Kunyomi] at (-21.200000, 23.700000) {\hbox{\tate さかい}};
\node[Meaning] at (-21.150000, 25.350000) {boundary};
\node[Kanji] at (-19.100000, 24.100000) {\textcolor[HTML]{c8a59d}{亡}};
\node[Square] at (-19.100000, 23.600000) {};
\node[Onyomi] at (-19.050000, 23.700000) {\hbox{\tate ボウ}};
\node[Kunyomi] at (-19.150000, 23.700000) {\hbox{\tate な.く}};
\node[Meaning] at (-19.100000, 25.350000) {deceased};
\node[Kanji] at (-17.050000, 24.100000) {\textcolor[HTML]{408dba}{盲}};
\node[Square] at (-17.050000, 23.600000) {};
\node[Onyomi] at (-17.000000, 23.700000) {\hbox{\tate モウ}};
\node[Kunyomi] at (-17.100000, 23.700000) {\hbox{\tate めくら}};
\node[Meaning] at (-17.050000, 25.350000) {blind};
\node[Kanji] at (-15.000000, 24.100000) {\textcolor[HTML]{1e76bb}{妄}};
\node[Square] at (-15.000000, 23.600000) {};
\node[Onyomi] at (-14.950000, 23.700000) {\hbox{\tate モウ・ボウ}};
\node[Kunyomi] at (-15.050000, 23.700000) {\hbox{\tate みだ}};
\node[Meaning] at (-15.000000, 25.350000) {reckless};
\node[Kanji] at (-12.950000, 24.100000) {\textcolor[HTML]{b0b0b5}{荒}};
\node[Square] at (-12.950000, 23.600000) {};
\node[Onyomi] at (-12.900000, 23.700000) {\hbox{\tate コウ}};
\node[Kunyomi] at (-13.000000, 23.700000) {\hbox{\tate あ・あら}};
\node[Meaning] at (-12.950000, 25.350000) {wild};
\node[Kanji] at (-10.900000, 24.100000) {\textcolor[HTML]{d2a293}{望}};
\node[Square] at (-10.900000, 23.600000) {};
\node[Onyomi] at (-10.850000, 23.700000) {\hbox{\tate ボウ}};
\node[Kunyomi] at (-10.950000, 23.700000) {\hbox{\tate のぞ.む}};
\node[Meaning] at (-10.900000, 25.350000) {hope};
\node[Kanji] at (-8.850000, 24.100000) {\textcolor[HTML]{b74029}{方}};
\node[Square] at (-8.850000, 23.600000) {};
\node[Onyomi] at (-8.800000, 23.700000) {\hbox{\tate ホウ}};
\node[Kunyomi] at (-8.900000, 23.700000) {\hbox{\tate かた}};
\node[Meaning] at (-8.850000, 25.350000) {direction};
\node[Kanji] at (-6.800000, 24.100000) {\textcolor[HTML]{91b7c3}{妨}};
\node[Square] at (-6.800000, 23.600000) {};
\node[Onyomi] at (-6.750000, 23.700000) {\hbox{\tate ボウ}};
\node[Kunyomi] at (-6.850000, 23.700000) {\hbox{\tate さまた.げる}};
\node[Meaning] at (-6.800000, 25.350000) {obstruct};
\node[Kanji] at (-4.750000, 24.100000) {\textcolor[HTML]{91b7c3}{坊}};
\node[Square] at (-4.750000, 23.600000) {};
\node[Onyomi] at (-4.700000, 23.700000) {\hbox{\tate ボウ}};
\node[Meaning] at (-4.750000, 25.350000) {monk};
\node[Kanji] at (-2.700000, 24.100000) {\textcolor[HTML]{a3bac2}{芳}};
\node[Square] at (-2.700000, 23.600000) {};
\node[Onyomi] at (-2.650000, 23.700000) {\hbox{\tate ホウ}};
\node[Kunyomi] at (-2.750000, 23.700000) {\hbox{\tate かんば}};
\node[Meaning] at (-2.700000, 25.350000) {perfume};
\node[Kanji] at (-0.650000, 24.100000) {\textcolor[HTML]{68a4bc}{肪}};
\node[Square] at (-0.650000, 23.600000) {};
\node[Onyomi] at (-0.600000, 23.700000) {\hbox{\tate ボウ}};
\node[Meaning] at (-0.650000, 25.350000) {obese};
\node[Kanji] at (1.400000, 24.100000) {\textcolor[HTML]{c8a59d}{訪}};
\node[Square] at (1.400000, 23.600000) {};
\node[Onyomi] at (1.450000, 23.700000) {\hbox{\tate ホウ}};
\node[Kunyomi] at (1.350000, 23.700000) {\hbox{\tate たず.ねる}};
\node[Meaning] at (1.400000, 25.350000) {visit};
\node[Kanji] at (3.450000, 24.100000) {\textcolor[HTML]{c36143}{放}};
\node[Square] at (3.450000, 23.600000) {};
\node[Onyomi] at (3.500000, 23.700000) {\hbox{\tate ホウ}};
\node[Kunyomi] at (3.400000, 23.700000) {\hbox{\tate はな}};
\node[Meaning] at (3.450000, 25.350000) {release};
\node[Kanji] at (5.500000, 24.100000) {\textcolor[HTML]{c8a59d}{激}};
\node[Square] at (5.500000, 23.600000) {};
\node[Onyomi] at (5.550000, 23.700000) {\hbox{\tate ゲキ}};
\node[Kunyomi] at (5.450000, 23.700000) {\hbox{\tate はげ.しい}};
\node[Meaning] at (5.500000, 25.350000) {fierce};
\node[Kanji] at (7.550000, 24.100000) {\textcolor[HTML]{c8a59d}{脱}};
\node[Square] at (7.550000, 23.600000) {};
\node[Onyomi] at (7.600000, 23.700000) {\hbox{\tate ダツ}};
\node[Kunyomi] at (7.500000, 23.700000) {\hbox{\tate ぬ.ぐ}};
\node[Meaning] at (7.550000, 25.350000) {undress};
\node[Kanji] at (9.600000, 24.100000) {\textcolor[HTML]{cd8268}{説}};
\node[Square] at (9.600000, 23.600000) {};
\node[Onyomi] at (9.650000, 23.700000) {\hbox{\tate セツ}};
\node[Kunyomi] at (9.550000, 23.700000) {\hbox{\tate と.く}};
\node[Meaning] at (9.600000, 25.350000) {theory};
\node[Kanji] at (11.650000, 24.100000) {\textcolor[HTML]{91b7c3}{鋭}};
\node[Square] at (11.650000, 23.600000) {};
\node[Onyomi] at (11.700000, 23.700000) {\hbox{\tate エイ}};
\node[Kunyomi] at (11.600000, 23.700000) {\hbox{\tate するど.い}};
\node[Meaning] at (11.650000, 25.350000) {sharp};
\node[Kanji] at (13.700000, 24.100000) {\textcolor[HTML]{91b7c3}{曽}};
\node[Square] at (13.700000, 23.600000) {};
\node[Onyomi] at (13.750000, 23.700000) {\hbox{\tate ソウ・(ゾ)}};
\node[Meaning] at (13.700000, 25.350000) {formerly};
\node[Kanji] at (15.750000, 24.100000) {\textcolor[HTML]{d69f8d}{増}};
\node[Square] at (15.750000, 23.600000) {};
\node[Onyomi] at (15.800000, 23.700000) {\hbox{\tate ゾウ}};
\node[Kunyomi] at (15.700000, 23.700000) {\hbox{\tate ふ.える}};
\node[Meaning] at (15.750000, 25.350000) {increase};
\node[Kanji] at (17.800000, 24.100000) {\textcolor[HTML]{a3bac2}{贈}};
\node[Square] at (17.800000, 23.600000) {};
\node[Onyomi] at (17.850000, 23.700000) {\hbox{\tate ゾウ}};
\node[Kunyomi] at (17.750000, 23.700000) {\hbox{\tate おく.*}};
\node[Meaning] at (17.800000, 25.350000) {presents};
\node[Kanji] at (19.850000, 24.100000) {\textcolor[HTML]{b74029}{東}};
\node[Square] at (19.850000, 23.600000) {};
\node[Onyomi] at (19.900000, 23.700000) {\hbox{\tate トウ}};
\node[Kunyomi] at (19.800000, 23.700000) {\hbox{\tate ひがし}};
\node[Meaning] at (19.850000, 25.350000) {east};
\node[Kanji] at (21.900000, 24.100000) {\textcolor[HTML]{a3bac2}{棟}};
\node[Square] at (21.900000, 23.600000) {};
\node[Onyomi] at (21.950000, 23.700000) {\hbox{\tate トウ}};
\node[Meaning] at (21.900000, 25.350000) {pillar};
\node[Kanji] at (23.950000, 24.100000) {\textcolor[HTML]{91b7c3}{凍}};
\node[Square] at (23.950000, 23.600000) {};
\node[Onyomi] at (24.000000, 23.700000) {\hbox{\tate トウ}};
\node[Kunyomi] at (23.900000, 23.700000) {\hbox{\tate こお.る}};
\node[Meaning] at (23.950000, 25.350000) {frozen};
\node[Kanji] at (26.000000, 24.100000) {\textcolor[HTML]{91b7c3}{妊}};
\node[Square] at (26.000000, 23.600000) {};
\node[Onyomi] at (26.050000, 23.700000) {\hbox{\tate ニン}};
\node[Meaning] at (26.000000, 25.350000) {pregnant};
\node[Kanji] at (28.050000, 24.100000) {\textcolor[HTML]{a3bac2}{廷}};
\node[Square] at (28.050000, 23.600000) {};
\node[Onyomi] at (28.100000, 23.700000) {\hbox{\tate テイ}};
\node[Meaning] at (28.050000, 25.350000) {courts};
\node[Kanji] at (30.100000, 24.100000) {\textcolor[HTML]{c8a59d}{染}};
\node[Square] at (30.100000, 23.600000) {};
\node[Onyomi] at (30.150000, 23.700000) {\hbox{\tate セン}};
\node[Kunyomi] at (30.050000, 23.700000) {\hbox{\tate しみ・そ.*}};
\node[Meaning] at (30.100000, 25.350000) {dye};
\node[Kanji] at (32.150000, 24.100000) {\textcolor[HTML]{b0b0b5}{燃}};
\node[Square] at (32.150000, 23.600000) {};
\node[Onyomi] at (32.200000, 23.700000) {\hbox{\tate ネン}};
\node[Kunyomi] at (32.100000, 23.700000) {\hbox{\tate も.*}};
\node[Meaning] at (32.150000, 25.350000) {burn};
\node[Kanji] at (34.200000, 24.100000) {\textcolor[HTML]{1e76bb}{賓}};
\node[Square] at (34.200000, 23.600000) {};
\node[Onyomi] at (34.250000, 23.700000) {\hbox{\tate ヒン}};
\node[Meaning] at (34.200000, 25.350000) {vip};
\node[Kanji] at (36.250000, 24.100000) {\textcolor[HTML]{d2a293}{歳}};
\node[Square] at (36.250000, 23.600000) {};
\node[Onyomi] at (36.300000, 23.700000) {\hbox{\tate サイ・セイ}};
\node[Meaning] at (36.250000, 25.350000) {years old};
\node[Kanji] at (38.300000, 24.100000) {\textcolor[HTML]{c36143}{県}};
\node[Square] at (38.300000, 23.600000) {};
\node[Onyomi] at (38.350000, 23.700000) {\hbox{\tate ケン}};
\node[Meaning] at (38.300000, 25.350000) {prefecture};
\node[Kanji] at (40.350000, 24.100000) {\textcolor[HTML]{91b7c3}{栃}};
\node[Square] at (40.350000, 23.600000) {};
\node[Kunyomi] at (40.300000, 23.700000) {\hbox{\tate とち}};
\node[Meaning] at (40.350000, 25.350000) {horse chestnut};
\node[Kanji] at (42.400000, 24.100000) {\textcolor[HTML]{a11d25}{地}};
\node[Square] at (42.400000, 23.600000) {};
\node[Onyomi] at (42.450000, 23.700000) {\hbox{\tate チ・ジ}};
\node[Meaning] at (42.400000, 25.350000) {earth};
\node[Kanji] at (44.450000, 24.100000) {\textcolor[HTML]{d2a293}{池}};
\node[Square] at (44.450000, 23.600000) {};
\node[Onyomi] at (44.500000, 23.700000) {\hbox{\tate チ}};
\node[Kunyomi] at (44.400000, 23.700000) {\hbox{\tate いけ}};
\node[Meaning] at (44.450000, 25.350000) {pond};
\node[Kanji] at (46.500000, 24.100000) {\textcolor[HTML]{b0b0b5}{虫}};
\node[Square] at (46.500000, 23.600000) {};
\node[Onyomi] at (46.550000, 23.700000) {\hbox{\tate チュウ・キ}};
\node[Kunyomi] at (46.450000, 23.700000) {\hbox{\tate むし}};
\node[Meaning] at (46.500000, 25.350000) {insect};
\node[Kanji] at (48.550000, 24.100000) {\textcolor[HTML]{408dba}{蛍}};
\node[Square] at (48.550000, 23.600000) {};
\node[Onyomi] at (48.600000, 23.700000) {\hbox{\tate ケイ}};
\node[Kunyomi] at (48.500000, 23.700000) {\hbox{\tate ほたる}};
\node[Meaning] at (48.550000, 25.350000) {firefly};
\node[Kanji] at (50.600000, 24.100000) {\textcolor[HTML]{68a4bc}{蛇}};
\node[Square] at (50.600000, 23.600000) {};
\node[Onyomi] at (50.650000, 23.700000) {\hbox{\tate ジャ}};
\node[Kunyomi] at (50.550000, 23.700000) {\hbox{\tate へび}};
\node[Meaning] at (50.600000, 25.350000) {snake};
\node[Kanji] at (52.650000, 24.100000) {\textcolor[HTML]{408dba}{虹}};
\node[Square] at (52.650000, 23.600000) {};
\node[Onyomi] at (52.700000, 23.700000) {\hbox{\tate コウ}};
\node[Kunyomi] at (52.600000, 23.700000) {\hbox{\tate にじ}};
\node[Meaning] at (52.650000, 25.350000) {rainbow};
\node[Kanji] at (54.700000, 24.100000) {\textcolor[HTML]{408dba}{蝶}};
\node[Square] at (54.700000, 23.600000) {};
\node[Onyomi] at (54.750000, 23.700000) {\hbox{\tate チョウ}};
\node[Meaning] at (54.700000, 25.350000) {butterfly};
\node[Kanji] at (56.750000, 24.100000) {\textcolor[HTML]{d69f8d}{独}};
\node[Square] at (56.750000, 23.600000) {};
\node[Onyomi] at (56.800000, 23.700000) {\hbox{\tate ドク}};
\node[Kunyomi] at (56.700000, 23.700000) {\hbox{\tate ひと.り}};
\node[Meaning] at (56.750000, 25.350000) {alone};
\node[Meaning] at (-58.050000, 24.200000) {33.95\%};
\node[Kanji] at (-56.000000, 22.050000) {\textcolor[HTML]{1e76bb}{蚕}};
\node[Square] at (-56.000000, 21.550000) {};
\node[Onyomi] at (-55.950000, 21.650000) {\hbox{\tate サン}};
\node[Kunyomi] at (-56.050000, 21.650000) {\hbox{\tate かいこ}};
\node[Meaning] at (-56.000000, 23.300000) {silkworm};
\node[Kanji] at (-53.950000, 22.050000) {\textcolor[HTML]{d69f8d}{風}};
\node[Square] at (-53.950000, 21.550000) {};
\node[Onyomi] at (-53.900000, 21.650000) {\hbox{\tate フウ}};
\node[Kunyomi] at (-54.000000, 21.650000) {\hbox{\tate かぜ}};
\node[Meaning] at (-53.950000, 23.300000) {wind};
\node[Kanji] at (-51.900000, 22.050000) {\textcolor[HTML]{b0b0b5}{己}};
\node[Square] at (-51.900000, 21.550000) {};
\node[Onyomi] at (-51.850000, 21.650000) {\hbox{\tate コ・キ}};
\node[Kunyomi] at (-51.950000, 21.650000) {\hbox{\tate おのれ}};
\node[Meaning] at (-51.900000, 23.300000) {oneself};
\node[Kanji] at (-49.850000, 22.050000) {\textcolor[HTML]{cd8268}{起}};
\node[Square] at (-49.850000, 21.550000) {};
\node[Onyomi] at (-49.800000, 21.650000) {\hbox{\tate キ}};
\node[Kunyomi] at (-49.900000, 21.650000) {\hbox{\tate お}};
\node[Meaning] at (-49.850000, 23.300000) {wake up};
\node[Kanji] at (-47.800000, 22.050000) {\textcolor[HTML]{91b7c3}{妃}};
\node[Square] at (-47.800000, 21.550000) {};
\node[Onyomi] at (-47.750000, 21.650000) {\hbox{\tate ヒ}};
\node[Meaning] at (-47.800000, 23.300000) {princess};
\node[Kanji] at (-45.750000, 22.050000) {\textcolor[HTML]{cd8268}{改}};
\node[Square] at (-45.750000, 21.550000) {};
\node[Onyomi] at (-45.700000, 21.650000) {\hbox{\tate カイ}};
\node[Kunyomi] at (-45.800000, 21.650000) {\hbox{\tate あらた.*}};
\node[Meaning] at (-45.750000, 23.300000) {renew};
\node[Kanji] at (-43.700000, 22.050000) {\textcolor[HTML]{b74029}{記}};
\node[Square] at (-43.700000, 21.550000) {};
\node[Onyomi] at (-43.650000, 21.650000) {\hbox{\tate キ}};
\node[Kunyomi] at (-43.750000, 21.650000) {\hbox{\tate しる.す}};
\node[Meaning] at (-43.700000, 23.300000) {write down};
\node[Kanji] at (-41.650000, 22.050000) {\textcolor[HTML]{b0b0b5}{包}};
\node[Square] at (-41.650000, 21.550000) {};
\node[Onyomi] at (-41.600000, 21.650000) {\hbox{\tate ホウ}};
\node[Kunyomi] at (-41.700000, 21.650000) {\hbox{\tate つつ.み}};
\node[Meaning] at (-41.650000, 23.300000) {wrap};
\node[Kanji] at (-39.600000, 22.050000) {\textcolor[HTML]{b0b0b5}{胞}};
\node[Square] at (-39.600000, 21.550000) {};
\node[Onyomi] at (-39.550000, 21.650000) {\hbox{\tate ホウ}};
\node[Meaning] at (-39.600000, 23.300000) {cell};
\node[Kanji] at (-37.550000, 22.050000) {\textcolor[HTML]{d2a293}{砲}};
\node[Square] at (-37.550000, 21.550000) {};
\node[Onyomi] at (-37.500000, 21.650000) {\hbox{\tate ホウ}};
\node[Meaning] at (-37.550000, 23.300000) {cannon};
\node[Kanji] at (-35.500000, 22.050000) {\textcolor[HTML]{408dba}{泡}};
\node[Square] at (-35.500000, 21.550000) {};
\node[Onyomi] at (-35.450000, 21.650000) {\hbox{\tate ホウ}};
\node[Kunyomi] at (-35.550000, 21.650000) {\hbox{\tate あわ}};
\node[Meaning] at (-35.500000, 23.300000) {bubbles};
\node[Kanji] at (-33.450000, 22.050000) {\textcolor[HTML]{b0b0b5}{亀}};
\node[Square] at (-33.450000, 21.550000) {};
\node[Kunyomi] at (-33.500000, 21.650000) {\hbox{\tate かめ}};
\node[Meaning] at (-33.450000, 23.300000) {turtle};
\node[Kanji] at (-31.400000, 22.050000) {\textcolor[HTML]{c36143}{電}};
\node[Square] at (-31.400000, 21.550000) {};
\node[Onyomi] at (-31.350000, 21.650000) {\hbox{\tate デン}};
\node[Meaning] at (-31.400000, 23.300000) {electricity};
\node[Kanji] at (-29.350000, 22.050000) {\textcolor[HTML]{b0b0b5}{竜}};
\node[Square] at (-29.350000, 21.550000) {};
\node[Onyomi] at (-29.300000, 21.650000) {\hbox{\tate リュウ}};
\node[Kunyomi] at (-29.400000, 21.650000) {\hbox{\tate たつ}};
\node[Meaning] at (-29.350000, 23.300000) {dragon};
\node[Kanji] at (-27.300000, 22.050000) {\textcolor[HTML]{a3bac2}{滝}};
\node[Square] at (-27.300000, 21.550000) {};
\node[Kunyomi] at (-27.350000, 21.650000) {\hbox{\tate たき}};
\node[Meaning] at (-27.300000, 23.300000) {waterfall};
\node[Kanji] at (-25.250000, 22.050000) {\textcolor[HTML]{68a4bc}{豚}};
\node[Square] at (-25.250000, 21.550000) {};
\node[Onyomi] at (-25.200000, 21.650000) {\hbox{\tate トン}};
\node[Kunyomi] at (-25.300000, 21.650000) {\hbox{\tate ぶた}};
\node[Meaning] at (-25.250000, 23.300000) {pork};
\node[Kanji] at (-23.200000, 22.050000) {\textcolor[HTML]{b0b0b5}{逐}};
\node[Square] at (-23.200000, 21.550000) {};
\node[Onyomi] at (-23.150000, 21.650000) {\hbox{\tate チク}};
\node[Meaning] at (-23.200000, 23.300000) {pursue};
\node[Kanji] at (-21.150000, 22.050000) {\textcolor[HTML]{a3bac2}{遂}};
\node[Square] at (-21.150000, 21.550000) {};
\node[Onyomi] at (-21.100000, 21.650000) {\hbox{\tate スイ}};
\node[Kunyomi] at (-21.200000, 21.650000) {\hbox{\tate と.げる}};
\node[Meaning] at (-21.150000, 23.300000) {accomplish};
\node[Kanji] at (-19.100000, 22.050000) {\textcolor[HTML]{b74029}{家}};
\node[Square] at (-19.100000, 21.550000) {};
\node[Onyomi] at (-19.050000, 21.650000) {\hbox{\tate カ・ケ}};
\node[Kunyomi] at (-19.150000, 21.650000) {\hbox{\tate いえ・や}};
\node[Meaning] at (-19.100000, 23.300000) {house};
\node[Kanji] at (-17.050000, 22.050000) {\textcolor[HTML]{68a4bc}{嫁}};
\node[Square] at (-17.050000, 21.550000) {};
\node[Onyomi] at (-17.000000, 21.650000) {\hbox{\tate カ}};
\node[Kunyomi] at (-17.100000, 21.650000) {\hbox{\tate よめ・とつ}};
\node[Meaning] at (-17.050000, 23.300000) {bride};
\node[Kanji] at (-15.000000, 22.050000) {\textcolor[HTML]{b0b0b5}{豪}};
\node[Square] at (-15.000000, 21.550000) {};
\node[Onyomi] at (-14.950000, 21.650000) {\hbox{\tate ゴウ}};
\node[Meaning] at (-15.000000, 23.300000) {luxurious};
\node[Kanji] at (-12.950000, 22.050000) {\textcolor[HTML]{91b7c3}{腸}};
\node[Square] at (-12.950000, 21.550000) {};
\node[Onyomi] at (-12.900000, 21.650000) {\hbox{\tate チョウ}};
\node[Kunyomi] at (-13.000000, 21.650000) {\hbox{\tate はらわた}};
\node[Meaning] at (-12.950000, 23.300000) {intestines};
\node[Kanji] at (-10.900000, 22.050000) {\textcolor[HTML]{a11d25}{場}};
\node[Square] at (-10.900000, 21.550000) {};
\node[Onyomi] at (-10.850000, 21.650000) {\hbox{\tate ジョウ}};
\node[Kunyomi] at (-10.950000, 21.650000) {\hbox{\tate ば}};
\node[Meaning] at (-10.900000, 23.300000) {location};
\node[Kanji] at (-8.850000, 22.050000) {\textcolor[HTML]{b0b0b5}{湯}};
\node[Square] at (-8.850000, 21.550000) {};
\node[Onyomi] at (-8.800000, 21.650000) {\hbox{\tate トウ}};
\node[Kunyomi] at (-8.900000, 21.650000) {\hbox{\tate ゆ}};
\node[Meaning] at (-8.850000, 23.300000) {hot water};
\node[Kanji] at (-6.800000, 22.050000) {\textcolor[HTML]{68a4bc}{羊}};
\node[Square] at (-6.800000, 21.550000) {};
\node[Onyomi] at (-6.750000, 21.650000) {\hbox{\tate ヨウ}};
\node[Kunyomi] at (-6.850000, 21.650000) {\hbox{\tate ひつじ}};
\node[Meaning] at (-6.800000, 23.300000) {sheep};
\node[Kanji] at (-4.750000, 22.050000) {\textcolor[HTML]{d69f8d}{美}};
\node[Square] at (-4.750000, 21.550000) {};
\node[Onyomi] at (-4.700000, 21.650000) {\hbox{\tate ビ}};
\node[Kunyomi] at (-4.800000, 21.650000) {\hbox{\tate うつく.しい}};
\node[Meaning] at (-4.750000, 23.300000) {beauty};
\node[Kanji] at (-2.700000, 22.050000) {\textcolor[HTML]{d69f8d}{洋}};
\node[Square] at (-2.700000, 21.550000) {};
\node[Onyomi] at (-2.650000, 21.650000) {\hbox{\tate ヨウ}};
\node[Meaning] at (-2.700000, 23.300000) {western style};
\node[Kanji] at (-0.650000, 22.050000) {\textcolor[HTML]{c8a59d}{詳}};
\node[Square] at (-0.650000, 21.550000) {};
\node[Onyomi] at (-0.600000, 21.650000) {\hbox{\tate ショウ}};
\node[Kunyomi] at (-0.700000, 21.650000) {\hbox{\tate くわ.しい}};
\node[Meaning] at (-0.650000, 23.300000) {detailed};
\node[Kanji] at (1.400000, 22.050000) {\textcolor[HTML]{c8a59d}{鮮}};
\node[Square] at (1.400000, 21.550000) {};
\node[Onyomi] at (1.450000, 21.650000) {\hbox{\tate セン}};
\node[Kunyomi] at (1.350000, 21.650000) {\hbox{\tate あざ.やか}};
\node[Meaning] at (1.400000, 23.300000) {fresh};
\node[Kanji] at (3.450000, 22.050000) {\textcolor[HTML]{d69f8d}{達}};
\node[Square] at (3.450000, 21.550000) {};
\node[Onyomi] at (3.500000, 21.650000) {\hbox{\tate タツ}};
\node[Kunyomi] at (3.400000, 21.650000) {\hbox{\tate たち}};
\node[Meaning] at (3.450000, 23.300000) {attain};
\node[Kanji] at (5.500000, 22.050000) {\textcolor[HTML]{1059be}{羨}};
\node[Square] at (5.500000, 21.550000) {};
\node[Onyomi] at (5.550000, 21.650000) {\hbox{\tate セン}};
\node[Kunyomi] at (5.450000, 21.650000) {\hbox{\tate うらや.む}};
\node[Meaning] at (5.500000, 23.300000) {envy};
\node[Kanji] at (7.550000, 22.050000) {\textcolor[HTML]{d69f8d}{差}};
\node[Square] at (7.550000, 21.550000) {};
\node[Onyomi] at (7.600000, 21.650000) {\hbox{\tate サ}};
\node[Kunyomi] at (7.500000, 21.650000) {\hbox{\tate さ}};
\node[Meaning] at (7.550000, 23.300000) {distinction};
\node[Kanji] at (9.600000, 22.050000) {\textcolor[HTML]{cd8268}{着}};
\node[Square] at (9.600000, 21.550000) {};
\node[Onyomi] at (9.650000, 21.650000) {\hbox{\tate チャク}};
\node[Kunyomi] at (9.550000, 21.650000) {\hbox{\tate き・つ}};
\node[Meaning] at (9.600000, 23.300000) {wear};
\node[Kanji] at (11.650000, 22.050000) {\textcolor[HTML]{b0b0b5}{唯}};
\node[Square] at (11.650000, 21.550000) {};
\node[Onyomi] at (11.700000, 21.650000) {\hbox{\tate ユイ}};
\node[Kunyomi] at (11.600000, 21.650000) {\hbox{\tate ただ}};
\node[Meaning] at (11.650000, 23.300000) {solely};
\node[Kanji] at (13.700000, 22.050000) {\textcolor[HTML]{68a4bc}{焦}};
\node[Square] at (13.700000, 21.550000) {};
\node[Onyomi] at (13.750000, 21.650000) {\hbox{\tate ショウ}};
\node[Kunyomi] at (13.650000, 21.650000) {\hbox{\tate こ.*}};
\node[Meaning] at (13.700000, 23.300000) {char};
\node[Kanji] at (15.750000, 22.050000) {\textcolor[HTML]{68a4bc}{礁}};
\node[Square] at (15.750000, 21.550000) {};
\node[Onyomi] at (15.800000, 21.650000) {\hbox{\tate ショウ}};
\node[Meaning] at (15.750000, 23.300000) {reef};
\node[Kanji] at (17.800000, 22.050000) {\textcolor[HTML]{c36143}{集}};
\node[Square] at (17.800000, 21.550000) {};
\node[Onyomi] at (17.850000, 21.650000) {\hbox{\tate シュウ}};
\node[Kunyomi] at (17.750000, 21.650000) {\hbox{\tate あつ.まる}};
\node[Meaning] at (17.800000, 23.300000) {collect};
\node[Kanji] at (19.850000, 22.050000) {\textcolor[HTML]{68a4bc}{准}};
\node[Square] at (19.850000, 21.550000) {};
\node[Onyomi] at (19.900000, 21.650000) {\hbox{\tate ジュン}};
\node[Meaning] at (19.850000, 23.300000) {semi};
\node[Kanji] at (21.900000, 22.050000) {\textcolor[HTML]{cd8268}{進}};
\node[Square] at (21.900000, 21.550000) {};
\node[Onyomi] at (21.950000, 21.650000) {\hbox{\tate シン}};
\node[Kunyomi] at (21.850000, 21.650000) {\hbox{\tate すす.む}};
\node[Meaning] at (21.900000, 23.300000) {advance};
\node[Kanji] at (23.950000, 22.050000) {\textcolor[HTML]{d2a293}{雑}};
\node[Square] at (23.950000, 21.550000) {};
\node[Onyomi] at (24.000000, 21.650000) {\hbox{\tate ザツ・ゾウ}};
\node[Meaning] at (23.950000, 23.300000) {random};
\node[Kanji] at (26.000000, 22.050000) {\textcolor[HTML]{91b7c3}{雌}};
\node[Square] at (26.000000, 21.550000) {};
\node[Onyomi] at (26.050000, 21.650000) {\hbox{\tate シ}};
\node[Kunyomi] at (25.950000, 21.650000) {\hbox{\tate めす・め}};
\node[Meaning] at (26.000000, 23.300000) {female};
\node[Kanji] at (28.050000, 22.050000) {\textcolor[HTML]{d69f8d}{準}};
\node[Square] at (28.050000, 21.550000) {};
\node[Onyomi] at (28.100000, 21.650000) {\hbox{\tate ジュン}};
\node[Meaning] at (28.050000, 23.300000) {standard};
\node[Kanji] at (30.100000, 22.050000) {\textcolor[HTML]{68a4bc}{奮}};
\node[Square] at (30.100000, 21.550000) {};
\node[Onyomi] at (30.150000, 21.650000) {\hbox{\tate フン}};
\node[Kunyomi] at (30.050000, 21.650000) {\hbox{\tate ふる.*}};
\node[Meaning] at (30.100000, 23.300000) {stirred up};
\node[Kanji] at (32.150000, 22.050000) {\textcolor[HTML]{b0b0b5}{奪}};
\node[Square] at (32.150000, 21.550000) {};
\node[Onyomi] at (32.200000, 21.650000) {\hbox{\tate ダツ}};
\node[Kunyomi] at (32.100000, 21.650000) {\hbox{\tate うば}};
\node[Meaning] at (32.150000, 23.300000) {rob};
\node[Kanji] at (34.200000, 22.050000) {\textcolor[HTML]{d69f8d}{確}};
\node[Square] at (34.200000, 21.550000) {};
\node[Onyomi] at (34.250000, 21.650000) {\hbox{\tate カク}};
\node[Kunyomi] at (34.150000, 21.650000) {\hbox{\tate たし.か}};
\node[Meaning] at (34.200000, 23.300000) {certain};
\node[Kanji] at (36.250000, 22.050000) {\textcolor[HTML]{b0b0b5}{午}};
\node[Square] at (36.250000, 21.550000) {};
\node[Onyomi] at (36.300000, 21.650000) {\hbox{\tate ゴ}};
\node[Meaning] at (36.250000, 23.300000) {noon};
\node[Kanji] at (38.300000, 22.050000) {\textcolor[HTML]{d2a293}{許}};
\node[Square] at (38.300000, 21.550000) {};
\node[Onyomi] at (38.350000, 21.650000) {\hbox{\tate キョ}};
\node[Kunyomi] at (38.250000, 21.650000) {\hbox{\tate ゆる.す}};
\node[Meaning] at (38.300000, 23.300000) {permit};
\node[Kanji] at (40.350000, 22.050000) {\textcolor[HTML]{68a4bc}{歓}};
\node[Square] at (40.350000, 21.550000) {};
\node[Onyomi] at (40.400000, 21.650000) {\hbox{\tate カン}};
\node[Meaning] at (40.350000, 23.300000) {delight};
\node[Kanji] at (42.400000, 22.050000) {\textcolor[HTML]{cd8268}{権}};
\node[Square] at (42.400000, 21.550000) {};
\node[Onyomi] at (42.450000, 21.650000) {\hbox{\tate ケン}};
\node[Meaning] at (42.400000, 23.300000) {rights};
\node[Kanji] at (44.450000, 22.050000) {\textcolor[HTML]{d69f8d}{観}};
\node[Square] at (44.450000, 21.550000) {};
\node[Onyomi] at (44.500000, 21.650000) {\hbox{\tate カン}};
\node[Kunyomi] at (44.400000, 21.650000) {\hbox{\tate み.る}};
\node[Meaning] at (44.450000, 23.300000) {view};
\node[Kanji] at (46.500000, 22.050000) {\textcolor[HTML]{c8a59d}{羽}};
\node[Square] at (46.500000, 21.550000) {};
\node[Kunyomi] at (46.450000, 21.650000) {\hbox{\tate はね・は}};
\node[Meaning] at (46.500000, 23.300000) {feather};
\node[Kanji] at (48.550000, 22.050000) {\textcolor[HTML]{d2a293}{習}};
\node[Square] at (48.550000, 21.550000) {};
\node[Onyomi] at (48.600000, 21.650000) {\hbox{\tate シュウ}};
\node[Kunyomi] at (48.500000, 21.650000) {\hbox{\tate なら.う}};
\node[Meaning] at (48.550000, 23.300000) {learn};
\node[Kanji] at (50.600000, 22.050000) {\textcolor[HTML]{d2a293}{翌}};
\node[Square] at (50.600000, 21.550000) {};
\node[Onyomi] at (50.650000, 21.650000) {\hbox{\tate ヨク}};
\node[Meaning] at (50.600000, 23.300000) {the following};
\node[Kanji] at (52.650000, 22.050000) {\textcolor[HTML]{d2a293}{曜}};
\node[Square] at (52.650000, 21.550000) {};
\node[Onyomi] at (52.700000, 21.650000) {\hbox{\tate ヨウ}};
\node[Meaning] at (52.650000, 23.300000) {weekday};
\node[Kanji] at (54.700000, 22.050000) {\textcolor[HTML]{1059be}{濯}};
\node[Square] at (54.700000, 21.550000) {};
\node[Onyomi] at (54.750000, 21.650000) {\hbox{\tate タク}};
\node[Kunyomi] at (54.650000, 21.650000) {\hbox{\tate すす.ぐ}};
\node[Meaning] at (54.700000, 23.300000) {wash};
\node[Kanji] at (56.750000, 22.050000) {\textcolor[HTML]{b0b0b5}{困}};
\node[Square] at (56.750000, 21.550000) {};
\node[Onyomi] at (56.800000, 21.650000) {\hbox{\tate コン}};
\node[Kunyomi] at (56.700000, 21.650000) {\hbox{\tate こま}};
\node[Meaning] at (56.750000, 23.300000) {distressed};
\node[Meaning] at (-58.050000, 22.150000) {36.70\%};
\node[Kanji] at (-56.000000, 20.000000) {\textcolor[HTML]{c8a59d}{固}};
\node[Square] at (-56.000000, 19.500000) {};
\node[Onyomi] at (-55.950000, 19.600000) {\hbox{\tate コ}};
\node[Kunyomi] at (-56.050000, 19.600000) {\hbox{\tate かた.い}};
\node[Meaning] at (-56.000000, 21.250000) {hard};
\node[Kanji] at (-53.950000, 20.000000) {\textcolor[HTML]{a11d25}{国}};
\node[Square] at (-53.950000, 19.500000) {};
\node[Onyomi] at (-53.900000, 19.600000) {\hbox{\tate コク}};
\node[Kunyomi] at (-54.000000, 19.600000) {\hbox{\tate くに}};
\node[Meaning] at (-53.950000, 21.250000) {country};
\node[Kanji] at (-51.900000, 20.000000) {\textcolor[HTML]{cd8268}{団}};
\node[Square] at (-51.900000, 19.500000) {};
\node[Onyomi] at (-51.850000, 19.600000) {\hbox{\tate ダン・トン}};
\node[Meaning] at (-51.900000, 21.250000) {group};
\node[Kanji] at (-49.850000, 20.000000) {\textcolor[HTML]{d2a293}{因}};
\node[Square] at (-49.850000, 19.500000) {};
\node[Onyomi] at (-49.800000, 19.600000) {\hbox{\tate イン}};
\node[Kunyomi] at (-49.900000, 19.600000) {\hbox{\tate よ}};
\node[Meaning] at (-49.850000, 21.250000) {cause};
\node[Kanji] at (-47.800000, 20.000000) {\textcolor[HTML]{408dba}{姻}};
\node[Square] at (-47.800000, 19.500000) {};
\node[Onyomi] at (-47.750000, 19.600000) {\hbox{\tate イン}};
\node[Meaning] at (-47.800000, 21.250000) {marry};
\node[Kanji] at (-45.750000, 20.000000) {\textcolor[HTML]{d69f8d}{園}};
\node[Square] at (-45.750000, 19.500000) {};
\node[Onyomi] at (-45.700000, 19.600000) {\hbox{\tate エン}};
\node[Meaning] at (-45.750000, 21.250000) {garden};
\node[Kanji] at (-43.700000, 20.000000) {\textcolor[HTML]{c36143}{回}};
\node[Square] at (-43.700000, 19.500000) {};
\node[Onyomi] at (-43.650000, 19.600000) {\hbox{\tate カイ}};
\node[Kunyomi] at (-43.750000, 19.600000) {\hbox{\tate まわ.*}};
\node[Meaning] at (-43.700000, 21.250000) {times};
\node[Kanji] at (-41.650000, 20.000000) {\textcolor[HTML]{91b7c3}{壇}};
\node[Square] at (-41.650000, 19.500000) {};
\node[Onyomi] at (-41.600000, 19.600000) {\hbox{\tate ダン}};
\node[Meaning] at (-41.650000, 21.250000) {podium};
\node[Kanji] at (-39.600000, 20.000000) {\textcolor[HTML]{d69f8d}{店}};
\node[Square] at (-39.600000, 19.500000) {};
\node[Onyomi] at (-39.550000, 19.600000) {\hbox{\tate テン}};
\node[Kunyomi] at (-39.650000, 19.600000) {\hbox{\tate みせ}};
\node[Meaning] at (-39.600000, 21.250000) {shop};
\node[Kanji] at (-37.550000, 20.000000) {\textcolor[HTML]{d2a293}{庫}};
\node[Square] at (-37.550000, 19.500000) {};
\node[Onyomi] at (-37.500000, 19.600000) {\hbox{\tate コ}};
\node[Kunyomi] at (-37.600000, 19.600000) {\hbox{\tate くら}};
\node[Meaning] at (-37.550000, 21.250000) {storage};
\node[Kanji] at (-35.500000, 20.000000) {\textcolor[HTML]{c8a59d}{庭}};
\node[Square] at (-35.500000, 19.500000) {};
\node[Onyomi] at (-35.450000, 19.600000) {\hbox{\tate テイ}};
\node[Kunyomi] at (-35.550000, 19.600000) {\hbox{\tate にわ}};
\node[Meaning] at (-35.500000, 21.250000) {garden};
\node[Kanji] at (-33.450000, 20.000000) {\textcolor[HTML]{d2a293}{庁}};
\node[Square] at (-33.450000, 19.500000) {};
\node[Onyomi] at (-33.400000, 19.600000) {\hbox{\tate チョウ}};
\node[Meaning] at (-33.450000, 21.250000) {agency};
\node[Kanji] at (-31.400000, 20.000000) {\textcolor[HTML]{b0b0b5}{床}};
\node[Square] at (-31.400000, 19.500000) {};
\node[Onyomi] at (-31.350000, 19.600000) {\hbox{\tate ショウ}};
\node[Kunyomi] at (-31.450000, 19.600000) {\hbox{\tate ゆか・とこ}};
\node[Meaning] at (-31.400000, 21.250000) {floor};
\node[Kanji] at (-29.350000, 20.000000) {\textcolor[HTML]{b0b0b5}{麻}};
\node[Square] at (-29.350000, 19.500000) {};
\node[Onyomi] at (-29.300000, 19.600000) {\hbox{\tate マ}};
\node[Kunyomi] at (-29.400000, 19.600000) {\hbox{\tate あさ}};
\node[Meaning] at (-29.350000, 21.250000) {hemp};
\node[Kanji] at (-27.300000, 20.000000) {\textcolor[HTML]{a3bac2}{磨}};
\node[Square] at (-27.300000, 19.500000) {};
\node[Onyomi] at (-27.250000, 19.600000) {\hbox{\tate マ}};
\node[Kunyomi] at (-27.350000, 19.600000) {\hbox{\tate みが}};
\node[Meaning] at (-27.300000, 21.250000) {polish};
\node[Kanji] at (-25.250000, 20.000000) {\textcolor[HTML]{cd8268}{心}};
\node[Square] at (-25.250000, 19.500000) {};
\node[Onyomi] at (-25.200000, 19.600000) {\hbox{\tate シン}};
\node[Kunyomi] at (-25.300000, 19.600000) {\hbox{\tate こころ}};
\node[Meaning] at (-25.250000, 21.250000) {heart};
\node[Kanji] at (-23.200000, 20.000000) {\textcolor[HTML]{91b7c3}{忘}};
\node[Square] at (-23.200000, 19.500000) {};
\node[Onyomi] at (-23.150000, 19.600000) {\hbox{\tate ボウ}};
\node[Kunyomi] at (-23.250000, 19.600000) {\hbox{\tate わす.れる}};
\node[Meaning] at (-23.200000, 21.250000) {forget};
\node[Kanji] at (-21.150000, 20.000000) {\textcolor[HTML]{91b7c3}{忍}};
\node[Square] at (-21.150000, 19.500000) {};
\node[Onyomi] at (-21.100000, 19.600000) {\hbox{\tate ニン}};
\node[Kunyomi] at (-21.200000, 19.600000) {\hbox{\tate しの.ぶ}};
\node[Meaning] at (-21.150000, 21.250000) {endure};
\node[Kanji] at (-19.100000, 20.000000) {\textcolor[HTML]{cd8268}{認}};
\node[Square] at (-19.100000, 19.500000) {};
\node[Onyomi] at (-19.050000, 19.600000) {\hbox{\tate ニン}};
\node[Kunyomi] at (-19.150000, 19.600000) {\hbox{\tate みと.める}};
\node[Meaning] at (-19.100000, 21.250000) {recognize};
\node[Kanji] at (-17.050000, 20.000000) {\textcolor[HTML]{68a4bc}{忌}};
\node[Square] at (-17.050000, 19.500000) {};
\node[Onyomi] at (-17.000000, 19.600000) {\hbox{\tate キ}};
\node[Kunyomi] at (-17.100000, 19.600000) {\hbox{\tate い}};
\node[Meaning] at (-17.050000, 21.250000) {mourning};
\node[Kanji] at (-15.000000, 20.000000) {\textcolor[HTML]{d2a293}{志}};
\node[Square] at (-15.000000, 19.500000) {};
\node[Onyomi] at (-14.950000, 19.600000) {\hbox{\tate シ}};
\node[Kunyomi] at (-15.050000, 19.600000) {\hbox{\tate こころざし}};
\node[Meaning] at (-15.000000, 21.250000) {intention};
\node[Kanji] at (-12.950000, 20.000000) {\textcolor[HTML]{d2a293}{誌}};
\node[Square] at (-12.950000, 19.500000) {};
\node[Onyomi] at (-12.900000, 19.600000) {\hbox{\tate シ}};
\node[Meaning] at (-12.950000, 21.250000) {magazine};
\node[Kanji] at (-10.900000, 20.000000) {\textcolor[HTML]{c8a59d}{忠}};
\node[Square] at (-10.900000, 19.500000) {};
\node[Onyomi] at (-10.850000, 19.600000) {\hbox{\tate チュウ}};
\node[Meaning] at (-10.900000, 21.250000) {loyalty};
\node[Kanji] at (-8.850000, 20.000000) {\textcolor[HTML]{408dba}{串}};
\node[Square] at (-8.850000, 19.500000) {};
\node[Kunyomi] at (-8.900000, 19.600000) {\hbox{\tate くし}};
\node[Meaning] at (-8.850000, 21.250000) {skewer};
\node[Kanji] at (-6.800000, 20.000000) {\textcolor[HTML]{b0b0b5}{患}};
\node[Square] at (-6.800000, 19.500000) {};
\node[Onyomi] at (-6.750000, 19.600000) {\hbox{\tate カン}};
\node[Kunyomi] at (-6.850000, 19.600000) {\hbox{\tate わずら.う}};
\node[Meaning] at (-6.800000, 21.250000) {afflicted};
\node[Kanji] at (-4.750000, 20.000000) {\textcolor[HTML]{d69f8d}{思}};
\node[Square] at (-4.750000, 19.500000) {};
\node[Onyomi] at (-4.700000, 19.600000) {\hbox{\tate シ}};
\node[Kunyomi] at (-4.800000, 19.600000) {\hbox{\tate おも.う}};
\node[Meaning] at (-4.750000, 21.250000) {think};
\node[Kanji] at (-2.700000, 20.000000) {\textcolor[HTML]{a3bac2}{恩}};
\node[Square] at (-2.700000, 19.500000) {};
\node[Onyomi] at (-2.650000, 19.600000) {\hbox{\tate オン}};
\node[Kunyomi] at (-2.750000, 19.600000) {\hbox{\tate おん}};
\node[Meaning] at (-2.700000, 21.250000) {kindness};
\node[Kanji] at (-0.650000, 20.000000) {\textcolor[HTML]{d69f8d}{応}};
\node[Square] at (-0.650000, 19.500000) {};
\node[Onyomi] at (-0.600000, 19.600000) {\hbox{\tate オウ}};
\node[Meaning] at (-0.650000, 21.250000) {respond};
\node[Kanji] at (1.400000, 20.000000) {\textcolor[HTML]{c36143}{意}};
\node[Square] at (1.400000, 19.500000) {};
\node[Onyomi] at (1.450000, 19.600000) {\hbox{\tate イ}};
\node[Meaning] at (1.400000, 21.250000) {idea};
\node[Kanji] at (3.450000, 20.000000) {\textcolor[HTML]{d2a293}{想}};
\node[Square] at (3.450000, 19.500000) {};
\node[Onyomi] at (3.500000, 19.600000) {\hbox{\tate ソウ}};
\node[Meaning] at (3.450000, 21.250000) {concept};
\node[Kanji] at (5.500000, 20.000000) {\textcolor[HTML]{c8a59d}{息}};
\node[Square] at (5.500000, 19.500000) {};
\node[Onyomi] at (5.550000, 19.600000) {\hbox{\tate ソク}};
\node[Kunyomi] at (5.450000, 19.600000) {\hbox{\tate いき}};
\node[Meaning] at (5.500000, 21.250000) {breath};
\node[Kanji] at (7.550000, 20.000000) {\textcolor[HTML]{408dba}{憩}};
\node[Square] at (7.550000, 19.500000) {};
\node[Onyomi] at (7.600000, 19.600000) {\hbox{\tate ケイ}};
\node[Kunyomi] at (7.500000, 19.600000) {\hbox{\tate いこ.い}};
\node[Meaning] at (7.550000, 21.250000) {rest};
\node[Kanji] at (9.600000, 20.000000) {\textcolor[HTML]{b0b0b5}{恵}};
\node[Square] at (9.600000, 19.500000) {};
\node[Onyomi] at (9.650000, 19.600000) {\hbox{\tate エ}};
\node[Kunyomi] at (9.550000, 19.600000) {\hbox{\tate めぐ.*}};
\node[Meaning] at (9.600000, 21.250000) {favor};
\node[Kanji] at (11.650000, 20.000000) {\textcolor[HTML]{b0b0b5}{恐}};
\node[Square] at (11.650000, 19.500000) {};
\node[Onyomi] at (11.700000, 19.600000) {\hbox{\tate キョウ}};
\node[Kunyomi] at (11.600000, 19.600000) {\hbox{\tate おそ.*}};
\node[Meaning] at (11.650000, 21.250000) {fear};
\node[Kanji] at (13.700000, 20.000000) {\textcolor[HTML]{b0b0b5}{惑}};
\node[Square] at (13.700000, 19.500000) {};
\node[Onyomi] at (13.750000, 19.600000) {\hbox{\tate ワク}};
\node[Kunyomi] at (13.650000, 19.600000) {\hbox{\tate まど.う}};
\node[Meaning] at (13.700000, 21.250000) {misguided};
\node[Kanji] at (15.750000, 20.000000) {\textcolor[HTML]{d69f8d}{感}};
\node[Square] at (15.750000, 19.500000) {};
\node[Onyomi] at (15.800000, 19.600000) {\hbox{\tate カン}};
\node[Meaning] at (15.750000, 21.250000) {feeling};
\node[Kanji] at (17.800000, 20.000000) {\textcolor[HTML]{408dba}{憂}};
\node[Square] at (17.800000, 19.500000) {};
\node[Onyomi] at (17.850000, 19.600000) {\hbox{\tate ユウ}};
\node[Kunyomi] at (17.750000, 19.600000) {\hbox{\tate う・うれ}};
\node[Meaning] at (17.800000, 21.250000) {grief};
\node[Kanji] at (19.850000, 20.000000) {\textcolor[HTML]{1059be}{寡}};
\node[Square] at (19.850000, 19.500000) {};
\node[Onyomi] at (19.900000, 19.600000) {\hbox{\tate カ}};
\node[Meaning] at (19.850000, 21.250000) {widow};
\node[Kanji] at (21.900000, 20.000000) {\textcolor[HTML]{408dba}{忙}};
\node[Square] at (21.900000, 19.500000) {};
\node[Onyomi] at (21.950000, 19.600000) {\hbox{\tate ボウ}};
\node[Kunyomi] at (21.850000, 19.600000) {\hbox{\tate いそが}};
\node[Meaning] at (21.900000, 21.250000) {busy};
\node[Kanji] at (23.950000, 20.000000) {\textcolor[HTML]{408dba}{悦}};
\node[Square] at (23.950000, 19.500000) {};
\node[Onyomi] at (24.000000, 19.600000) {\hbox{\tate エツ}};
\node[Kunyomi] at (23.900000, 19.600000) {\hbox{\tate よろこ}};
\node[Meaning] at (23.950000, 21.250000) {delight};
\node[Kanji] at (26.000000, 20.000000) {\textcolor[HTML]{a3bac2}{恒}};
\node[Square] at (26.000000, 19.500000) {};
\node[Onyomi] at (26.050000, 19.600000) {\hbox{\tate コウ}};
\node[Kunyomi] at (25.950000, 19.600000) {\hbox{\tate つね・つねに}};
\node[Meaning] at (26.000000, 21.250000) {constant};
\node[Kanji] at (28.050000, 20.000000) {\textcolor[HTML]{68a4bc}{悼}};
\node[Square] at (28.050000, 19.500000) {};
\node[Onyomi] at (28.100000, 19.600000) {\hbox{\tate トウ}};
\node[Kunyomi] at (28.000000, 19.600000) {\hbox{\tate いた}};
\node[Meaning] at (28.050000, 21.250000) {grieve};
\node[Kanji] at (30.100000, 20.000000) {\textcolor[HTML]{91b7c3}{悟}};
\node[Square] at (30.100000, 19.500000) {};
\node[Onyomi] at (30.150000, 19.600000) {\hbox{\tate ゴ}};
\node[Kunyomi] at (30.050000, 19.600000) {\hbox{\tate さと.る}};
\node[Meaning] at (30.100000, 21.250000) {comprehension};
\node[Kanji] at (32.150000, 20.000000) {\textcolor[HTML]{91b7c3}{怖}};
\node[Square] at (32.150000, 19.500000) {};
\node[Onyomi] at (32.200000, 19.600000) {\hbox{\tate フ}};
\node[Kunyomi] at (32.100000, 19.600000) {\hbox{\tate こわ.*}};
\node[Meaning] at (32.150000, 21.250000) {scary};
\node[Kanji] at (34.200000, 20.000000) {\textcolor[HTML]{408dba}{慌}};
\node[Square] at (34.200000, 19.500000) {};
\node[Onyomi] at (34.250000, 19.600000) {\hbox{\tate コウ}};
\node[Kunyomi] at (34.150000, 19.600000) {\hbox{\tate あわ.てる}};
\node[Meaning] at (34.200000, 21.250000) {disconcerted};
\node[Kanji] at (36.250000, 20.000000) {\textcolor[HTML]{408dba}{悔}};
\node[Square] at (36.250000, 19.500000) {};
\node[Onyomi] at (36.300000, 19.600000) {\hbox{\tate カイ}};
\node[Kunyomi] at (36.200000, 19.600000) {\hbox{\tate くや.しい}};
\node[Meaning] at (36.250000, 21.250000) {regret};
\node[Kanji] at (38.300000, 20.000000) {\textcolor[HTML]{408dba}{憎}};
\node[Square] at (38.300000, 19.500000) {};
\node[Onyomi] at (38.350000, 19.600000) {\hbox{\tate ゾウ}};
\node[Kunyomi] at (38.250000, 19.600000) {\hbox{\tate にく.*}};
\node[Meaning] at (38.300000, 21.250000) {hate};
\node[Kanji] at (40.350000, 20.000000) {\textcolor[HTML]{a3bac2}{慣}};
\node[Square] at (40.350000, 19.500000) {};
\node[Onyomi] at (40.400000, 19.600000) {\hbox{\tate カン}};
\node[Kunyomi] at (40.300000, 19.600000) {\hbox{\tate な.れる}};
\node[Meaning] at (40.350000, 21.250000) {accustomed};
\node[Kanji] at (42.400000, 20.000000) {\textcolor[HTML]{1059be}{愉}};
\node[Square] at (42.400000, 19.500000) {};
\node[Onyomi] at (42.450000, 19.600000) {\hbox{\tate ユ}};
\node[Kunyomi] at (42.350000, 19.600000) {\hbox{\tate たの}};
\node[Meaning] at (42.400000, 21.250000) {pleasant};
\node[Kanji] at (44.450000, 20.000000) {\textcolor[HTML]{242e6c}{惰}};
\node[Square] at (44.450000, 19.500000) {};
\node[Onyomi] at (44.500000, 19.600000) {\hbox{\tate ダ}};
\node[Meaning] at (44.450000, 21.250000) {lazy};
\node[Kanji] at (46.500000, 20.000000) {\textcolor[HTML]{91b7c3}{慎}};
\node[Square] at (46.500000, 19.500000) {};
\node[Onyomi] at (46.550000, 19.600000) {\hbox{\tate シン}};
\node[Kunyomi] at (46.450000, 19.600000) {\hbox{\tate つつし.む}};
\node[Meaning] at (46.500000, 21.250000) {humility};
\node[Kanji] at (48.550000, 20.000000) {\textcolor[HTML]{29409e}{憾}};
\node[Square] at (48.550000, 19.500000) {};
\node[Onyomi] at (48.600000, 19.600000) {\hbox{\tate カン}};
\node[Kunyomi] at (48.500000, 19.600000) {\hbox{\tate うら}};
\node[Meaning] at (48.550000, 21.250000) {remorse};
\node[Kanji] at (50.600000, 20.000000) {\textcolor[HTML]{a3bac2}{憶}};
\node[Square] at (50.600000, 19.500000) {};
\node[Onyomi] at (50.650000, 19.600000) {\hbox{\tate オク}};
\node[Meaning] at (50.600000, 21.250000) {recollection};
\node[Kanji] at (52.650000, 20.000000) {\textcolor[HTML]{68a4bc}{慕}};
\node[Square] at (52.650000, 19.500000) {};
\node[Onyomi] at (52.700000, 19.600000) {\hbox{\tate ボ}};
\node[Kunyomi] at (52.600000, 19.600000) {\hbox{\tate した}};
\node[Meaning] at (52.650000, 21.250000) {yearn for};
\node[Kanji] at (54.700000, 20.000000) {\textcolor[HTML]{a3bac2}{添}};
\node[Square] at (54.700000, 19.500000) {};
\node[Onyomi] at (54.750000, 19.600000) {\hbox{\tate テン}};
\node[Kunyomi] at (54.650000, 19.600000) {\hbox{\tate そ.える}};
\node[Meaning] at (54.700000, 21.250000) {append};
\node[Kanji] at (56.750000, 20.000000) {\textcolor[HTML]{d69f8d}{必}};
\node[Square] at (56.750000, 19.500000) {};
\node[Onyomi] at (56.800000, 19.600000) {\hbox{\tate ヒツ}};
\node[Kunyomi] at (56.700000, 19.600000) {\hbox{\tate かなら.ず}};
\node[Meaning] at (56.750000, 21.250000) {certain};
\node[Meaning] at (-58.050000, 20.100000) {38.87\%};
\node[Kanji] at (-56.000000, 17.950000) {\textcolor[HTML]{408dba}{泌}};
\node[Square] at (-56.000000, 17.450000) {};
\node[Onyomi] at (-55.950000, 17.550000) {\hbox{\tate ヒ・ヒツ}};
\node[Meaning] at (-56.000000, 19.200000) {secrete};
\node[Kanji] at (-53.950000, 17.950000) {\textcolor[HTML]{b74029}{手}};
\node[Square] at (-53.950000, 17.450000) {};
\node[Onyomi] at (-53.900000, 17.550000) {\hbox{\tate シュ}};
\node[Kunyomi] at (-54.000000, 17.550000) {\hbox{\tate て}};
\node[Meaning] at (-53.950000, 19.200000) {hand};
\node[Kanji] at (-51.900000, 17.950000) {\textcolor[HTML]{a3bac2}{看}};
\node[Square] at (-51.900000, 17.450000) {};
\node[Onyomi] at (-51.850000, 17.550000) {\hbox{\tate カン}};
\node[Meaning] at (-51.900000, 19.200000) {watch over};
\node[Kanji] at (-49.850000, 17.950000) {\textcolor[HTML]{b0b0b5}{摩}};
\node[Square] at (-49.850000, 17.450000) {};
\node[Onyomi] at (-49.800000, 17.550000) {\hbox{\tate マ}};
\node[Kunyomi] at (-49.900000, 17.550000) {\hbox{\tate さす.る}};
\node[Meaning] at (-49.850000, 19.200000) {chafe};
\node[Kanji] at (-47.800000, 17.950000) {\textcolor[HTML]{b0b0b5}{我}};
\node[Square] at (-47.800000, 17.450000) {};
\node[Onyomi] at (-47.750000, 17.550000) {\hbox{\tate ガ}};
\node[Kunyomi] at (-47.850000, 17.550000) {\hbox{\tate われ}};
\node[Meaning] at (-47.800000, 19.200000) {i};
\node[Kanji] at (-45.750000, 17.950000) {\textcolor[HTML]{cd8268}{義}};
\node[Square] at (-45.750000, 17.450000) {};
\node[Onyomi] at (-45.700000, 17.550000) {\hbox{\tate ギ}};
\node[Meaning] at (-45.750000, 19.200000) {righteousness};
\node[Kanji] at (-43.700000, 17.950000) {\textcolor[HTML]{b74029}{議}};
\node[Square] at (-43.700000, 17.450000) {};
\node[Onyomi] at (-43.650000, 17.550000) {\hbox{\tate ギ}};
\node[Meaning] at (-43.700000, 19.200000) {deliberation};
\node[Kanji] at (-41.650000, 17.950000) {\textcolor[HTML]{68a4bc}{犠}};
\node[Square] at (-41.650000, 17.450000) {};
\node[Onyomi] at (-41.600000, 17.550000) {\hbox{\tate ギ}};
\node[Meaning] at (-41.650000, 19.200000) {sacrifice};
\node[Kanji] at (-39.600000, 17.950000) {\textcolor[HTML]{408dba}{抹}};
\node[Square] at (-39.600000, 17.450000) {};
\node[Onyomi] at (-39.550000, 17.550000) {\hbox{\tate マツ}};
\node[Meaning] at (-39.600000, 19.200000) {erase};
\node[Kanji] at (-37.550000, 17.950000) {\textcolor[HTML]{b0b0b5}{抱}};
\node[Square] at (-37.550000, 17.450000) {};
\node[Onyomi] at (-37.500000, 17.550000) {\hbox{\tate ホウ}};
\node[Kunyomi] at (-37.600000, 17.550000) {\hbox{\tate だ・かか}};
\node[Meaning] at (-37.550000, 19.200000) {hug};
\node[Kanji] at (-35.500000, 17.950000) {\textcolor[HTML]{c8a59d}{搭}};
\node[Square] at (-35.500000, 17.450000) {};
\node[Onyomi] at (-35.450000, 17.550000) {\hbox{\tate トウ}};
\node[Meaning] at (-35.500000, 19.200000) {board};
\node[Kanji] at (-33.450000, 17.950000) {\textcolor[HTML]{68a4bc}{抄}};
\node[Square] at (-33.450000, 17.450000) {};
\node[Onyomi] at (-33.400000, 17.550000) {\hbox{\tate ショウ}};
\node[Meaning] at (-33.450000, 19.200000) {extract};
\node[Kanji] at (-31.400000, 17.950000) {\textcolor[HTML]{c8a59d}{抗}};
\node[Square] at (-31.400000, 17.450000) {};
\node[Onyomi] at (-31.350000, 17.550000) {\hbox{\tate コウ}};
\node[Kunyomi] at (-31.450000, 17.550000) {\hbox{\tate あらが.う}};
\node[Meaning] at (-31.400000, 19.200000) {confront};
\node[Kanji] at (-29.350000, 17.950000) {\textcolor[HTML]{c8a59d}{批}};
\node[Square] at (-29.350000, 17.450000) {};
\node[Onyomi] at (-29.300000, 17.550000) {\hbox{\tate ヒ}};
\node[Meaning] at (-29.350000, 19.200000) {criticism};
\node[Kanji] at (-27.300000, 17.950000) {\textcolor[HTML]{b0b0b5}{招}};
\node[Square] at (-27.300000, 17.450000) {};
\node[Onyomi] at (-27.250000, 17.550000) {\hbox{\tate ショウ}};
\node[Kunyomi] at (-27.350000, 17.550000) {\hbox{\tate まね.く}};
\node[Meaning] at (-27.300000, 19.200000) {beckon};
\node[Kanji] at (-25.250000, 17.950000) {\textcolor[HTML]{a3bac2}{拓}};
\node[Square] at (-25.250000, 17.450000) {};
\node[Onyomi] at (-25.200000, 17.550000) {\hbox{\tate タク}};
\node[Meaning] at (-25.250000, 19.200000) {cultivation};
\node[Kanji] at (-23.200000, 17.950000) {\textcolor[HTML]{68a4bc}{拍}};
\node[Square] at (-23.200000, 17.450000) {};
\node[Onyomi] at (-23.150000, 17.550000) {\hbox{\tate ハク・ヒョウ}};
\node[Meaning] at (-23.200000, 19.200000) {beat};
\node[Kanji] at (-21.150000, 17.950000) {\textcolor[HTML]{d69f8d}{打}};
\node[Square] at (-21.150000, 17.450000) {};
\node[Onyomi] at (-21.100000, 17.550000) {\hbox{\tate ダ}};
\node[Kunyomi] at (-21.200000, 17.550000) {\hbox{\tate う・ぶ}};
\node[Meaning] at (-21.150000, 19.200000) {hit};
\node[Kanji] at (-19.100000, 17.950000) {\textcolor[HTML]{91b7c3}{拘}};
\node[Square] at (-19.100000, 17.450000) {};
\node[Onyomi] at (-19.050000, 17.550000) {\hbox{\tate コウ}};
\node[Kunyomi] at (-19.150000, 17.550000) {\hbox{\tate かか.わる}};
\node[Meaning] at (-19.100000, 19.200000) {arrest};
\node[Kanji] at (-17.050000, 17.950000) {\textcolor[HTML]{91b7c3}{捨}};
\node[Square] at (-17.050000, 17.450000) {};
\node[Onyomi] at (-17.000000, 17.550000) {\hbox{\tate シャ}};
\node[Kunyomi] at (-17.100000, 17.550000) {\hbox{\tate す}};
\node[Meaning] at (-17.050000, 19.200000) {throw away};
\node[Kanji] at (-15.000000, 17.950000) {\textcolor[HTML]{408dba}{拐}};
\node[Square] at (-15.000000, 17.450000) {};
\node[Onyomi] at (-14.950000, 17.550000) {\hbox{\tate カイ}};
\node[Meaning] at (-15.000000, 19.200000) {kidnap};
\node[Kanji] at (-12.950000, 17.950000) {\textcolor[HTML]{b0b0b5}{摘}};
\node[Square] at (-12.950000, 17.450000) {};
\node[Onyomi] at (-12.900000, 17.550000) {\hbox{\tate テキ}};
\node[Kunyomi] at (-13.000000, 17.550000) {\hbox{\tate つ.む}};
\node[Meaning] at (-12.950000, 19.200000) {pluck};
\node[Kanji] at (-10.900000, 17.950000) {\textcolor[HTML]{a3bac2}{挑}};
\node[Square] at (-10.900000, 17.450000) {};
\node[Onyomi] at (-10.850000, 17.550000) {\hbox{\tate チョウ}};
\node[Kunyomi] at (-10.950000, 17.550000) {\hbox{\tate いど.む}};
\node[Meaning] at (-10.900000, 19.200000) {challenge};
\node[Kanji] at (-8.850000, 17.950000) {\textcolor[HTML]{c36143}{指}};
\node[Square] at (-8.850000, 17.450000) {};
\node[Onyomi] at (-8.800000, 17.550000) {\hbox{\tate シ}};
\node[Kunyomi] at (-8.900000, 17.550000) {\hbox{\tate ゆび・さ}};
\node[Meaning] at (-8.850000, 19.200000) {finger};
\node[Kanji] at (-6.800000, 17.950000) {\textcolor[HTML]{c36143}{持}};
\node[Square] at (-6.800000, 17.450000) {};
\node[Onyomi] at (-6.750000, 17.550000) {\hbox{\tate ジ}};
\node[Kunyomi] at (-6.850000, 17.550000) {\hbox{\tate も}};
\node[Meaning] at (-6.800000, 19.200000) {hold};
\node[Kanji] at (-4.750000, 17.950000) {\textcolor[HTML]{b0b0b5}{括}};
\node[Square] at (-4.750000, 17.450000) {};
\node[Onyomi] at (-4.700000, 17.550000) {\hbox{\tate カツ}};
\node[Kunyomi] at (-4.800000, 17.550000) {\hbox{\tate くく.る}};
\node[Meaning] at (-4.750000, 19.200000) {fasten};
\node[Kanji] at (-2.700000, 17.950000) {\textcolor[HTML]{c8a59d}{揮}};
\node[Square] at (-2.700000, 17.450000) {};
\node[Onyomi] at (-2.650000, 17.550000) {\hbox{\tate キ}};
\node[Meaning] at (-2.700000, 19.200000) {brandish};
\node[Kanji] at (-0.650000, 17.950000) {\textcolor[HTML]{d2a293}{推}};
\node[Square] at (-0.650000, 17.450000) {};
\node[Onyomi] at (-0.600000, 17.550000) {\hbox{\tate スイ}};
\node[Kunyomi] at (-0.700000, 17.550000) {\hbox{\tate お.す}};
\node[Meaning] at (-0.650000, 19.200000) {infer};
\node[Kanji] at (1.400000, 17.950000) {\textcolor[HTML]{a3bac2}{揚}};
\node[Square] at (1.400000, 17.450000) {};
\node[Onyomi] at (1.450000, 17.550000) {\hbox{\tate ヨウ}};
\node[Kunyomi] at (1.350000, 17.550000) {\hbox{\tate あげ}};
\node[Meaning] at (1.400000, 19.200000) {hoist};
\node[Kanji] at (3.450000, 17.950000) {\textcolor[HTML]{cd8268}{提}};
\node[Square] at (3.450000, 17.450000) {};
\node[Onyomi] at (3.500000, 17.550000) {\hbox{\tate テイ}};
\node[Meaning] at (3.450000, 19.200000) {present};
\node[Kanji] at (5.500000, 17.950000) {\textcolor[HTML]{b0b0b5}{損}};
\node[Square] at (5.500000, 17.450000) {};
\node[Onyomi] at (5.550000, 17.550000) {\hbox{\tate ソン}};
\node[Kunyomi] at (5.450000, 17.550000) {\hbox{\tate そこ.なう}};
\node[Meaning] at (5.500000, 19.200000) {loss};
\node[Kanji] at (7.550000, 17.950000) {\textcolor[HTML]{68a4bc}{拾}};
\node[Square] at (7.550000, 17.450000) {};
\node[Kunyomi] at (7.500000, 17.550000) {\hbox{\tate ひろ}};
\node[Meaning] at (7.550000, 19.200000) {pick up};
\node[Kanji] at (9.600000, 17.950000) {\textcolor[HTML]{d69f8d}{担}};
\node[Square] at (9.600000, 17.450000) {};
\node[Onyomi] at (9.650000, 17.550000) {\hbox{\tate タン}};
\node[Kunyomi] at (9.550000, 17.550000) {\hbox{\tate にな.う}};
\node[Meaning] at (9.600000, 19.200000) {carry};
\node[Kanji] at (11.650000, 17.950000) {\textcolor[HTML]{d2a293}{拠}};
\node[Square] at (11.650000, 17.450000) {};
\node[Onyomi] at (11.700000, 17.550000) {\hbox{\tate キョ}};
\node[Kunyomi] at (11.600000, 17.550000) {\hbox{\tate よ.る}};
\node[Meaning] at (11.650000, 19.200000) {based on};
\node[Kanji] at (13.700000, 17.950000) {\textcolor[HTML]{d2a293}{描}};
\node[Square] at (13.700000, 17.450000) {};
\node[Onyomi] at (13.750000, 17.550000) {\hbox{\tate ビョウ}};
\node[Kunyomi] at (13.650000, 17.550000) {\hbox{\tate か.く}};
\node[Meaning] at (13.700000, 19.200000) {draw};
\node[Kanji] at (15.750000, 17.950000) {\textcolor[HTML]{c8a59d}{操}};
\node[Square] at (15.750000, 17.450000) {};
\node[Onyomi] at (15.800000, 17.550000) {\hbox{\tate ソウ}};
\node[Kunyomi] at (15.700000, 17.550000) {\hbox{\tate あやつ.る}};
\node[Meaning] at (15.750000, 19.200000) {manipulate};
\node[Kanji] at (17.800000, 17.950000) {\textcolor[HTML]{cd8268}{接}};
\node[Square] at (17.800000, 17.450000) {};
\node[Onyomi] at (17.850000, 17.550000) {\hbox{\tate セツ}};
\node[Kunyomi] at (17.750000, 17.550000) {\hbox{\tate つ.ぐ}};
\node[Meaning] at (17.800000, 19.200000) {adjoin};
\node[Kanji] at (19.850000, 17.950000) {\textcolor[HTML]{c8a59d}{掲}};
\node[Square] at (19.850000, 17.450000) {};
\node[Onyomi] at (19.900000, 17.550000) {\hbox{\tate ケイ}};
\node[Kunyomi] at (19.800000, 17.550000) {\hbox{\tate かか.げる}};
\node[Meaning] at (19.850000, 19.200000) {display};
\node[Kanji] at (21.900000, 17.950000) {\textcolor[HTML]{c8a59d}{掛}};
\node[Square] at (21.900000, 17.450000) {};
\node[Onyomi] at (21.950000, 17.550000) {\hbox{\tate ガイ}};
\node[Kunyomi] at (21.850000, 17.550000) {\hbox{\tate か}};
\node[Meaning] at (21.900000, 19.200000) {hang};
\node[Kanji] at (23.950000, 17.950000) {\textcolor[HTML]{cd8268}{研}};
\node[Square] at (23.950000, 17.450000) {};
\node[Onyomi] at (24.000000, 17.550000) {\hbox{\tate ケン}};
\node[Kunyomi] at (23.900000, 17.550000) {\hbox{\tate と}};
\node[Meaning] at (23.950000, 19.200000) {sharpen};
\node[Kanji] at (26.000000, 17.950000) {\textcolor[HTML]{a3bac2}{戒}};
\node[Square] at (26.000000, 17.450000) {};
\node[Onyomi] at (26.050000, 17.550000) {\hbox{\tate カイ}};
\node[Kunyomi] at (25.950000, 17.550000) {\hbox{\tate いまし.める}};
\node[Meaning] at (26.000000, 19.200000) {commandment};
\node[Kanji] at (28.050000, 17.950000) {\textcolor[HTML]{b0b0b5}{械}};
\node[Square] at (28.050000, 17.450000) {};
\node[Onyomi] at (28.100000, 17.550000) {\hbox{\tate カイ}};
\node[Kunyomi] at (28.000000, 17.550000) {\hbox{\tate かせ}};
\node[Meaning] at (28.050000, 19.200000) {contraption};
\node[Kanji] at (30.100000, 17.950000) {\textcolor[HTML]{91b7c3}{鼻}};
\node[Square] at (30.100000, 17.450000) {};
\node[Onyomi] at (30.150000, 17.550000) {\hbox{\tate ビ}};
\node[Kunyomi] at (30.050000, 17.550000) {\hbox{\tate はな}};
\node[Meaning] at (30.100000, 19.200000) {nose};
\node[Kanji] at (32.150000, 17.950000) {\textcolor[HTML]{c8a59d}{刑}};
\node[Square] at (32.150000, 17.450000) {};
\node[Onyomi] at (32.200000, 17.550000) {\hbox{\tate ケイ}};
\node[Meaning] at (32.150000, 19.200000) {punish};
\node[Kanji] at (34.200000, 17.950000) {\textcolor[HTML]{cd8268}{型}};
\node[Square] at (34.200000, 17.450000) {};
\node[Onyomi] at (34.250000, 17.550000) {\hbox{\tate ケイ}};
\node[Kunyomi] at (34.150000, 17.550000) {\hbox{\tate かた}};
\node[Meaning] at (34.200000, 19.200000) {model};
\node[Kanji] at (36.250000, 17.950000) {\textcolor[HTML]{a3bac2}{才}};
\node[Square] at (36.250000, 17.450000) {};
\node[Onyomi] at (36.300000, 17.550000) {\hbox{\tate サイ}};
\node[Meaning] at (36.250000, 19.200000) {genius};
\node[Kanji] at (38.300000, 17.950000) {\textcolor[HTML]{d2a293}{財}};
\node[Square] at (38.300000, 17.450000) {};
\node[Onyomi] at (38.350000, 17.550000) {\hbox{\tate サイ・ザイ}};
\node[Meaning] at (38.300000, 19.200000) {wealth};
\node[Kanji] at (40.350000, 17.950000) {\textcolor[HTML]{d2a293}{材}};
\node[Square] at (40.350000, 17.450000) {};
\node[Onyomi] at (40.400000, 17.550000) {\hbox{\tate ザイ}};
\node[Meaning] at (40.350000, 19.200000) {lumber};
\node[Kanji] at (42.400000, 17.950000) {\textcolor[HTML]{cd8268}{存}};
\node[Square] at (42.400000, 17.450000) {};
\node[Onyomi] at (42.450000, 17.550000) {\hbox{\tate ソン・ゾン}};
\node[Meaning] at (42.400000, 19.200000) {suppose};
\node[Kanji] at (44.450000, 17.950000) {\textcolor[HTML]{b74029}{在}};
\node[Square] at (44.450000, 17.450000) {};
\node[Onyomi] at (44.500000, 17.550000) {\hbox{\tate ザイ}};
\node[Meaning] at (44.450000, 19.200000) {exist};
\node[Kanji] at (46.500000, 17.950000) {\textcolor[HTML]{a3bac2}{乃}};
\node[Square] at (46.500000, 17.450000) {};
\node[Onyomi] at (46.550000, 17.550000) {\hbox{\tate ナイ}};
\node[Kunyomi] at (46.450000, 17.550000) {\hbox{\tate すなわ}};
\node[Meaning] at (46.500000, 19.200000) {from};
\node[Kanji] at (48.550000, 17.950000) {\textcolor[HTML]{c8a59d}{携}};
\node[Square] at (48.550000, 17.450000) {};
\node[Onyomi] at (48.600000, 17.550000) {\hbox{\tate ケイ}};
\node[Kunyomi] at (48.500000, 17.550000) {\hbox{\tate たずさ.わる}};
\node[Meaning] at (48.550000, 19.200000) {portable};
\node[Kanji] at (50.600000, 17.950000) {\textcolor[HTML]{d69f8d}{及}};
\node[Square] at (50.600000, 17.450000) {};
\node[Onyomi] at (50.650000, 17.550000) {\hbox{\tate キュウ}};
\node[Kunyomi] at (50.550000, 17.550000) {\hbox{\tate およ.*}};
\node[Meaning] at (50.600000, 19.200000) {reach};
\node[Kanji] at (52.650000, 17.950000) {\textcolor[HTML]{b0b0b5}{吸}};
\node[Square] at (52.650000, 17.450000) {};
\node[Onyomi] at (52.700000, 17.550000) {\hbox{\tate キュウ}};
\node[Kunyomi] at (52.600000, 17.550000) {\hbox{\tate す.う}};
\node[Meaning] at (52.650000, 19.200000) {suck};
\node[Kanji] at (54.700000, 17.950000) {\textcolor[HTML]{d2a293}{扱}};
\node[Square] at (54.700000, 17.450000) {};
\node[Onyomi] at (54.750000, 17.550000) {\hbox{\tate キュウ}};
\node[Kunyomi] at (54.650000, 17.550000) {\hbox{\tate あつか}};
\node[Meaning] at (54.700000, 19.200000) {handle};
\node[Kanji] at (56.750000, 17.950000) {\textcolor[HTML]{91b7c3}{丈}};
\node[Square] at (56.750000, 17.450000) {};
\node[Onyomi] at (56.800000, 17.550000) {\hbox{\tate ジョウ}};
\node[Kunyomi] at (56.700000, 17.550000) {\hbox{\tate たけ}};
\node[Meaning] at (56.750000, 19.200000) {height};
\node[Meaning] at (-58.050000, 18.050000) {41.42\%};
\node[Kanji] at (-56.000000, 15.900000) {\textcolor[HTML]{cd8268}{史}};
\node[Square] at (-56.000000, 15.400000) {};
\node[Onyomi] at (-55.950000, 15.500000) {\hbox{\tate シ}};
\node[Meaning] at (-56.000000, 17.150000) {history};
\node[Kanji] at (-53.950000, 15.900000) {\textcolor[HTML]{408dba}{吏}};
\node[Square] at (-53.950000, 15.400000) {};
\node[Onyomi] at (-53.900000, 15.500000) {\hbox{\tate リ}};
\node[Meaning] at (-53.950000, 17.150000) {officer};
\node[Kanji] at (-51.900000, 15.900000) {\textcolor[HTML]{d69f8d}{更}};
\node[Square] at (-51.900000, 15.400000) {};
\node[Onyomi] at (-51.850000, 15.500000) {\hbox{\tate コウ}};
\node[Kunyomi] at (-51.950000, 15.500000) {\hbox{\tate さら・ふ}};
\node[Meaning] at (-51.900000, 17.150000) {again};
\node[Kanji] at (-49.850000, 15.900000) {\textcolor[HTML]{a3bac2}{硬}};
\node[Square] at (-49.850000, 15.400000) {};
\node[Onyomi] at (-49.800000, 15.500000) {\hbox{\tate コウ}};
\node[Kunyomi] at (-49.900000, 15.500000) {\hbox{\tate かた.い}};
\node[Meaning] at (-49.850000, 17.150000) {stiff};
\node[Kanji] at (-47.800000, 15.900000) {\textcolor[HTML]{a3bac2}{又}};
\node[Square] at (-47.800000, 15.400000) {};
\node[Kunyomi] at (-47.850000, 15.500000) {\hbox{\tate また}};
\node[Meaning] at (-47.800000, 17.150000) {again};
\node[Kanji] at (-45.750000, 15.900000) {\textcolor[HTML]{b0b0b5}{双}};
\node[Square] at (-45.750000, 15.400000) {};
\node[Onyomi] at (-45.700000, 15.500000) {\hbox{\tate ソウ}};
\node[Kunyomi] at (-45.800000, 15.500000) {\hbox{\tate ふた}};
\node[Meaning] at (-45.750000, 17.150000) {pair};
\node[Kanji] at (-43.700000, 15.900000) {\textcolor[HTML]{91b7c3}{桑}};
\node[Square] at (-43.700000, 15.400000) {};
\node[Onyomi] at (-43.650000, 15.500000) {\hbox{\tate ソウ}};
\node[Kunyomi] at (-43.750000, 15.500000) {\hbox{\tate くわ}};
\node[Meaning] at (-43.700000, 17.150000) {mulberry};
\node[Kanji] at (-41.650000, 15.900000) {\textcolor[HTML]{a3bac2}{隻}};
\node[Square] at (-41.650000, 15.400000) {};
\node[Onyomi] at (-41.600000, 15.500000) {\hbox{\tate セキ}};
\node[Meaning] at (-41.650000, 17.150000) {ship counter};
\node[Kanji] at (-39.600000, 15.900000) {\textcolor[HTML]{d69f8d}{護}};
\node[Square] at (-39.600000, 15.400000) {};
\node[Onyomi] at (-39.550000, 15.500000) {\hbox{\tate ゴ}};
\node[Meaning] at (-39.600000, 17.150000) {defend};
\node[Kanji] at (-37.550000, 15.900000) {\textcolor[HTML]{c8a59d}{獲}};
\node[Square] at (-37.550000, 15.400000) {};
\node[Onyomi] at (-37.500000, 15.500000) {\hbox{\tate カク}};
\node[Kunyomi] at (-37.600000, 15.500000) {\hbox{\tate え.る}};
\node[Meaning] at (-37.550000, 17.150000) {seize};
\node[Kanji] at (-35.500000, 15.900000) {\textcolor[HTML]{91b7c3}{奴}};
\node[Square] at (-35.500000, 15.400000) {};
\node[Onyomi] at (-35.450000, 15.500000) {\hbox{\tate ド}};
\node[Kunyomi] at (-35.550000, 15.500000) {\hbox{\tate やつ}};
\node[Meaning] at (-35.500000, 17.150000) {dude};
\node[Kanji] at (-33.450000, 15.900000) {\textcolor[HTML]{a3bac2}{怒}};
\node[Square] at (-33.450000, 15.400000) {};
\node[Onyomi] at (-33.400000, 15.500000) {\hbox{\tate ド}};
\node[Kunyomi] at (-33.500000, 15.500000) {\hbox{\tate おこ.る}};
\node[Meaning] at (-33.450000, 17.150000) {angry};
\node[Kanji] at (-31.400000, 15.900000) {\textcolor[HTML]{d2a293}{友}};
\node[Square] at (-31.400000, 15.400000) {};
\node[Onyomi] at (-31.350000, 15.500000) {\hbox{\tate ユウ}};
\node[Kunyomi] at (-31.450000, 15.500000) {\hbox{\tate とも}};
\node[Meaning] at (-31.400000, 17.150000) {friend};
\node[Kanji] at (-29.350000, 15.900000) {\textcolor[HTML]{c8a59d}{抜}};
\node[Square] at (-29.350000, 15.400000) {};
\node[Onyomi] at (-29.300000, 15.500000) {\hbox{\tate バツ・ハツ}};
\node[Kunyomi] at (-29.400000, 15.500000) {\hbox{\tate ぬ}};
\node[Meaning] at (-29.350000, 17.150000) {extract};
\node[Kanji] at (-27.300000, 15.900000) {\textcolor[HTML]{d69f8d}{投}};
\node[Square] at (-27.300000, 15.400000) {};
\node[Onyomi] at (-27.250000, 15.500000) {\hbox{\tate トウ}};
\node[Kunyomi] at (-27.350000, 15.500000) {\hbox{\tate な}};
\node[Meaning] at (-27.300000, 17.150000) {throw};
\node[Kanji] at (-25.250000, 15.900000) {\textcolor[HTML]{c8a59d}{没}};
\node[Square] at (-25.250000, 15.400000) {};
\node[Onyomi] at (-25.200000, 15.500000) {\hbox{\tate ボツ・モツ}};
\node[Kunyomi] at (-25.300000, 15.500000) {\hbox{\tate おぼ・しず}};
\node[Meaning] at (-25.250000, 17.150000) {die};
\node[Kanji] at (-23.200000, 15.900000) {\textcolor[HTML]{b74029}{設}};
\node[Square] at (-23.200000, 15.400000) {};
\node[Onyomi] at (-23.150000, 15.500000) {\hbox{\tate セツ}};
\node[Kunyomi] at (-23.250000, 15.500000) {\hbox{\tate もう.ける}};
\node[Meaning] at (-23.200000, 17.150000) {establish};
\node[Kanji] at (-21.150000, 15.900000) {\textcolor[HTML]{cd8268}{撃}};
\node[Square] at (-21.150000, 15.400000) {};
\node[Onyomi] at (-21.100000, 15.500000) {\hbox{\tate ゲキ}};
\node[Kunyomi] at (-21.200000, 15.500000) {\hbox{\tate う.つ}};
\node[Meaning] at (-21.150000, 17.150000) {attack};
\node[Kanji] at (-19.100000, 15.900000) {\textcolor[HTML]{91b7c3}{殻}};
\node[Square] at (-19.100000, 15.400000) {};
\node[Onyomi] at (-19.050000, 15.500000) {\hbox{\tate カク・コク}};
\node[Kunyomi] at (-19.150000, 15.500000) {\hbox{\tate から・がら}};
\node[Meaning] at (-19.100000, 17.150000) {husk};
\node[Kanji] at (-17.050000, 15.900000) {\textcolor[HTML]{cd8268}{支}};
\node[Square] at (-17.050000, 15.400000) {};
\node[Onyomi] at (-17.000000, 15.500000) {\hbox{\tate シ}};
\node[Kunyomi] at (-17.100000, 15.500000) {\hbox{\tate ささ.える}};
\node[Meaning] at (-17.050000, 17.150000) {support};
\node[Kanji] at (-15.000000, 15.900000) {\textcolor[HTML]{cd8268}{技}};
\node[Square] at (-15.000000, 15.400000) {};
\node[Onyomi] at (-14.950000, 15.500000) {\hbox{\tate ギ}};
\node[Kunyomi] at (-15.050000, 15.500000) {\hbox{\tate わざ}};
\node[Meaning] at (-15.000000, 17.150000) {skill};
\node[Kanji] at (-12.950000, 15.900000) {\textcolor[HTML]{b0b0b5}{枝}};
\node[Square] at (-12.950000, 15.400000) {};
\node[Onyomi] at (-12.900000, 15.500000) {\hbox{\tate シ}};
\node[Kunyomi] at (-13.000000, 15.500000) {\hbox{\tate えだ}};
\node[Meaning] at (-12.950000, 17.150000) {branch};
\node[Kanji] at (-10.900000, 15.900000) {\textcolor[HTML]{91b7c3}{肢}};
\node[Square] at (-10.900000, 15.400000) {};
\node[Onyomi] at (-10.850000, 15.500000) {\hbox{\tate シ}};
\node[Meaning] at (-10.900000, 17.150000) {limb};
\node[Kanji] at (-8.850000, 15.900000) {\textcolor[HTML]{91b7c3}{茎}};
\node[Square] at (-8.850000, 15.400000) {};
\node[Onyomi] at (-8.800000, 15.500000) {\hbox{\tate キョウ・ケイ}};
\node[Kunyomi] at (-8.900000, 15.500000) {\hbox{\tate くき}};
\node[Meaning] at (-8.850000, 17.150000) {stem};
\node[Kanji] at (-6.800000, 15.900000) {\textcolor[HTML]{b0b0b5}{怪}};
\node[Square] at (-6.800000, 15.400000) {};
\node[Onyomi] at (-6.750000, 15.500000) {\hbox{\tate カイ・ケ}};
\node[Kunyomi] at (-6.850000, 15.500000) {\hbox{\tate あや.しい}};
\node[Meaning] at (-6.800000, 17.150000) {suspicious};
\node[Kanji] at (-4.750000, 15.900000) {\textcolor[HTML]{c8a59d}{軽}};
\node[Square] at (-4.750000, 15.400000) {};
\node[Onyomi] at (-4.700000, 15.500000) {\hbox{\tate ケイ}};
\node[Kunyomi] at (-4.800000, 15.500000) {\hbox{\tate かる・かろ}};
\node[Meaning] at (-4.750000, 17.150000) {lightweight};
\node[Kanji] at (-2.700000, 15.900000) {\textcolor[HTML]{91b7c3}{叔}};
\node[Square] at (-2.700000, 15.400000) {};
\node[Onyomi] at (-2.650000, 15.500000) {\hbox{\tate シュク}};
\node[Meaning] at (-2.700000, 17.150000) {uncle};
\node[Kanji] at (-0.650000, 15.900000) {\textcolor[HTML]{d2a293}{督}};
\node[Square] at (-0.650000, 15.400000) {};
\node[Onyomi] at (-0.600000, 15.500000) {\hbox{\tate トク}};
\node[Meaning] at (-0.650000, 17.150000) {coach};
\node[Kanji] at (1.400000, 15.900000) {\textcolor[HTML]{408dba}{寂}};
\node[Square] at (1.400000, 15.400000) {};
\node[Onyomi] at (1.450000, 15.500000) {\hbox{\tate ジャク・セキ}};
\node[Kunyomi] at (1.350000, 15.500000) {\hbox{\tate さび・さみ}};
\node[Meaning] at (1.400000, 17.150000) {lonely};
\node[Kanji] at (3.450000, 15.900000) {\textcolor[HTML]{1e76bb}{淑}};
\node[Square] at (3.450000, 15.400000) {};
\node[Onyomi] at (3.500000, 15.500000) {\hbox{\tate シュク}};
\node[Kunyomi] at (3.400000, 15.500000) {\hbox{\tate しと}};
\node[Meaning] at (3.450000, 17.150000) {graceful};
\node[Kanji] at (5.500000, 15.900000) {\textcolor[HTML]{d69f8d}{反}};
\node[Square] at (5.500000, 15.400000) {};
\node[Onyomi] at (5.550000, 15.500000) {\hbox{\tate ハン}};
\node[Meaning] at (5.500000, 17.150000) {anti};
\node[Kanji] at (7.550000, 15.900000) {\textcolor[HTML]{d2a293}{坂}};
\node[Square] at (7.550000, 15.400000) {};
\node[Onyomi] at (7.600000, 15.500000) {\hbox{\tate ハン}};
\node[Kunyomi] at (7.500000, 15.500000) {\hbox{\tate さか}};
\node[Meaning] at (7.550000, 17.150000) {slope};
\node[Kanji] at (9.600000, 15.900000) {\textcolor[HTML]{d2a293}{板}};
\node[Square] at (9.600000, 15.400000) {};
\node[Onyomi] at (9.650000, 15.500000) {\hbox{\tate ハン}};
\node[Kunyomi] at (9.550000, 15.500000) {\hbox{\tate いた}};
\node[Meaning] at (9.600000, 17.150000) {board};
\node[Kanji] at (11.650000, 15.900000) {\textcolor[HTML]{d2a293}{返}};
\node[Square] at (11.650000, 15.400000) {};
\node[Onyomi] at (11.700000, 15.500000) {\hbox{\tate ヘン}};
\node[Kunyomi] at (11.600000, 15.500000) {\hbox{\tate かえ.る}};
\node[Meaning] at (11.650000, 17.150000) {return};
\node[Kanji] at (13.700000, 15.900000) {\textcolor[HTML]{d2a293}{販}};
\node[Square] at (13.700000, 15.400000) {};
\node[Onyomi] at (13.750000, 15.500000) {\hbox{\tate ハン}};
\node[Meaning] at (13.700000, 17.150000) {sell};
\node[Kanji] at (15.750000, 15.900000) {\textcolor[HTML]{68a4bc}{爪}};
\node[Square] at (15.750000, 15.400000) {};
\node[Onyomi] at (15.800000, 15.500000) {\hbox{\tate ソウ}};
\node[Kunyomi] at (15.700000, 15.500000) {\hbox{\tate つま・つめ}};
\node[Meaning] at (15.750000, 17.150000) {claw};
\node[Kanji] at (17.800000, 15.900000) {\textcolor[HTML]{68a4bc}{妥}};
\node[Square] at (17.800000, 15.400000) {};
\node[Onyomi] at (17.850000, 15.500000) {\hbox{\tate ダ}};
\node[Meaning] at (17.800000, 17.150000) {gentle};
\node[Kanji] at (19.850000, 15.900000) {\textcolor[HTML]{b0b0b5}{乳}};
\node[Square] at (19.850000, 15.400000) {};
\node[Onyomi] at (19.900000, 15.500000) {\hbox{\tate ニュウ}};
\node[Meaning] at (19.850000, 17.150000) {milk};
\node[Kanji] at (21.900000, 15.900000) {\textcolor[HTML]{b0b0b5}{浮}};
\node[Square] at (21.900000, 15.400000) {};
\node[Onyomi] at (21.950000, 15.500000) {\hbox{\tate フ}};
\node[Kunyomi] at (21.850000, 15.500000) {\hbox{\tate う・うわ}};
\node[Meaning] at (21.900000, 17.150000) {float};
\node[Kanji] at (23.950000, 15.900000) {\textcolor[HTML]{d69f8d}{将}};
\node[Square] at (23.950000, 15.400000) {};
\node[Onyomi] at (24.000000, 15.500000) {\hbox{\tate ショウ}};
\node[Meaning] at (23.950000, 17.150000) {commander};
\node[Kanji] at (26.000000, 15.900000) {\textcolor[HTML]{a3bac2}{奨}};
\node[Square] at (26.000000, 15.400000) {};
\node[Onyomi] at (26.050000, 15.500000) {\hbox{\tate ショウ}};
\node[Meaning] at (26.000000, 17.150000) {encourage};
\node[Kanji] at (28.050000, 15.900000) {\textcolor[HTML]{d2a293}{採}};
\node[Square] at (28.050000, 15.400000) {};
\node[Onyomi] at (28.100000, 15.500000) {\hbox{\tate サイ}};
\node[Kunyomi] at (28.000000, 15.500000) {\hbox{\tate と.る}};
\node[Meaning] at (28.050000, 17.150000) {gather};
\node[Kanji] at (30.100000, 15.900000) {\textcolor[HTML]{a3bac2}{菜}};
\node[Square] at (30.100000, 15.400000) {};
\node[Onyomi] at (30.150000, 15.500000) {\hbox{\tate サイ}};
\node[Meaning] at (30.100000, 17.150000) {vegetable};
\node[Kanji] at (32.150000, 15.900000) {\textcolor[HTML]{c36143}{受}};
\node[Square] at (32.150000, 15.400000) {};
\node[Onyomi] at (32.200000, 15.500000) {\hbox{\tate ジュ}};
\node[Kunyomi] at (32.100000, 15.500000) {\hbox{\tate う}};
\node[Meaning] at (32.150000, 17.150000) {accept};
\node[Kanji] at (34.200000, 15.900000) {\textcolor[HTML]{d2a293}{授}};
\node[Square] at (34.200000, 15.400000) {};
\node[Onyomi] at (34.250000, 15.500000) {\hbox{\tate ジュ}};
\node[Kunyomi] at (34.150000, 15.500000) {\hbox{\tate さず.ける}};
\node[Meaning] at (34.200000, 17.150000) {instruct};
\node[Kanji] at (36.250000, 15.900000) {\textcolor[HTML]{d69f8d}{愛}};
\node[Square] at (36.250000, 15.400000) {};
\node[Onyomi] at (36.300000, 15.500000) {\hbox{\tate アイ}};
\node[Kunyomi] at (36.200000, 15.500000) {\hbox{\tate まな}};
\node[Meaning] at (36.250000, 17.150000) {love};
\node[Kanji] at (38.300000, 15.900000) {\textcolor[HTML]{b0b0b5}{払}};
\node[Square] at (38.300000, 15.400000) {};
\node[Kunyomi] at (38.250000, 15.500000) {\hbox{\tate はら}};
\node[Meaning] at (38.300000, 17.150000) {pay};
\node[Kanji] at (40.350000, 15.900000) {\textcolor[HTML]{cd8268}{広}};
\node[Square] at (40.350000, 15.400000) {};
\node[Onyomi] at (40.400000, 15.500000) {\hbox{\tate コウ}};
\node[Kunyomi] at (40.300000, 15.500000) {\hbox{\tate ひろ}};
\node[Meaning] at (40.350000, 17.150000) {wide};
\node[Kanji] at (42.400000, 15.900000) {\textcolor[HTML]{c8a59d}{拡}};
\node[Square] at (42.400000, 15.400000) {};
\node[Onyomi] at (42.450000, 15.500000) {\hbox{\tate カク}};
\node[Kunyomi] at (42.350000, 15.500000) {\hbox{\tate ひろ.がる}};
\node[Meaning] at (42.400000, 17.150000) {extend};
\node[Kanji] at (44.450000, 15.900000) {\textcolor[HTML]{b0b0b5}{鉱}};
\node[Square] at (44.450000, 15.400000) {};
\node[Onyomi] at (44.500000, 15.500000) {\hbox{\tate コウ}};
\node[Kunyomi] at (44.400000, 15.500000) {\hbox{\tate あらがね}};
\node[Meaning] at (44.450000, 17.150000) {mineral};
\node[Kanji] at (46.500000, 15.900000) {\textcolor[HTML]{c8a59d}{弁}};
\node[Square] at (46.500000, 15.400000) {};
\node[Onyomi] at (46.550000, 15.500000) {\hbox{\tate ベン}};
\node[Meaning] at (46.500000, 17.150000) {dialect};
\node[Kanji] at (48.550000, 15.900000) {\textcolor[HTML]{d2a293}{雄}};
\node[Square] at (48.550000, 15.400000) {};
\node[Onyomi] at (48.600000, 15.500000) {\hbox{\tate ユウ}};
\node[Kunyomi] at (48.500000, 15.500000) {\hbox{\tate おす}};
\node[Meaning] at (48.550000, 17.150000) {male};
\node[Kanji] at (50.600000, 15.900000) {\textcolor[HTML]{cd8268}{台}};
\node[Square] at (50.600000, 15.400000) {};
\node[Onyomi] at (50.650000, 15.500000) {\hbox{\tate ダイ・タイ}};
\node[Meaning] at (50.600000, 17.150000) {machine};
\node[Kanji] at (52.650000, 15.900000) {\textcolor[HTML]{1e76bb}{怠}};
\node[Square] at (52.650000, 15.400000) {};
\node[Onyomi] at (52.700000, 15.500000) {\hbox{\tate タイ}};
\node[Kunyomi] at (52.600000, 15.500000) {\hbox{\tate おこた・なま}};
\node[Meaning] at (52.650000, 17.150000) {lazy};
\node[Kanji] at (54.700000, 15.900000) {\textcolor[HTML]{c36143}{治}};
\node[Square] at (54.700000, 15.400000) {};
\node[Onyomi] at (54.750000, 15.500000) {\hbox{\tate ジ・チ}};
\node[Kunyomi] at (54.650000, 15.500000) {\hbox{\tate なお.す}};
\node[Meaning] at (54.700000, 17.150000) {cure};
\node[Kanji] at (56.750000, 15.900000) {\textcolor[HTML]{cd8268}{始}};
\node[Square] at (56.750000, 15.400000) {};
\node[Onyomi] at (56.800000, 15.500000) {\hbox{\tate シ}};
\node[Kunyomi] at (56.700000, 15.500000) {\hbox{\tate はじ}};
\node[Meaning] at (56.750000, 17.150000) {begin};
\node[Meaning] at (-58.050000, 16.000000) {43.64\%};
\node[Kanji] at (-56.000000, 13.850000) {\textcolor[HTML]{68a4bc}{胎}};
\node[Square] at (-56.000000, 13.350000) {};
\node[Onyomi] at (-55.950000, 13.450000) {\hbox{\tate タイ}};
\node[Meaning] at (-56.000000, 15.100000) {womb};
\node[Kanji] at (-53.950000, 13.850000) {\textcolor[HTML]{b0b0b5}{窓}};
\node[Square] at (-53.950000, 13.350000) {};
\node[Onyomi] at (-53.900000, 13.450000) {\hbox{\tate ソウ}};
\node[Kunyomi] at (-54.000000, 13.450000) {\hbox{\tate まど}};
\node[Meaning] at (-53.950000, 15.100000) {window};
\node[Kanji] at (-51.900000, 13.850000) {\textcolor[HTML]{cd8268}{去}};
\node[Square] at (-51.900000, 13.350000) {};
\node[Onyomi] at (-51.850000, 13.450000) {\hbox{\tate キョ・コ}};
\node[Kunyomi] at (-51.950000, 13.450000) {\hbox{\tate さ.る}};
\node[Meaning] at (-51.900000, 15.100000) {past};
\node[Kanji] at (-49.850000, 13.850000) {\textcolor[HTML]{b74029}{法}};
\node[Square] at (-49.850000, 13.350000) {};
\node[Onyomi] at (-49.800000, 13.450000) {\hbox{\tate ホウ}};
\node[Meaning] at (-49.850000, 15.100000) {method};
\node[Kanji] at (-47.800000, 13.850000) {\textcolor[HTML]{a11d25}{会}};
\node[Square] at (-47.800000, 13.350000) {};
\node[Onyomi] at (-47.750000, 13.450000) {\hbox{\tate カイ}};
\node[Kunyomi] at (-47.850000, 13.450000) {\hbox{\tate あ.う}};
\node[Meaning] at (-47.800000, 15.100000) {meet};
\node[Kanji] at (-45.750000, 13.850000) {\textcolor[HTML]{c8a59d}{至}};
\node[Square] at (-45.750000, 13.350000) {};
\node[Onyomi] at (-45.700000, 13.450000) {\hbox{\tate シ}};
\node[Kunyomi] at (-45.800000, 13.450000) {\hbox{\tate いた.る}};
\node[Meaning] at (-45.750000, 15.100000) {attain};
\node[Kanji] at (-43.700000, 13.850000) {\textcolor[HTML]{d69f8d}{室}};
\node[Square] at (-43.700000, 13.350000) {};
\node[Onyomi] at (-43.650000, 13.450000) {\hbox{\tate シツ}};
\node[Meaning] at (-43.700000, 15.100000) {room};
\node[Kanji] at (-41.650000, 13.850000) {\textcolor[HTML]{b0b0b5}{到}};
\node[Square] at (-41.650000, 13.350000) {};
\node[Onyomi] at (-41.600000, 13.450000) {\hbox{\tate トウ}};
\node[Meaning] at (-41.650000, 15.100000) {arrival};
\node[Kanji] at (-39.600000, 13.850000) {\textcolor[HTML]{b0b0b5}{致}};
\node[Square] at (-39.600000, 13.350000) {};
\node[Onyomi] at (-39.550000, 13.450000) {\hbox{\tate チ}};
\node[Kunyomi] at (-39.650000, 13.450000) {\hbox{\tate いた.す}};
\node[Meaning] at (-39.600000, 15.100000) {do};
\node[Kanji] at (-37.550000, 13.850000) {\textcolor[HTML]{c8a59d}{互}};
\node[Square] at (-37.550000, 13.350000) {};
\node[Onyomi] at (-37.500000, 13.450000) {\hbox{\tate ゴ}};
\node[Kunyomi] at (-37.600000, 13.450000) {\hbox{\tate たが.い}};
\node[Meaning] at (-37.550000, 15.100000) {mutual};
\node[Kanji] at (-35.500000, 13.850000) {\textcolor[HTML]{a3bac2}{棄}};
\node[Square] at (-35.500000, 13.350000) {};
\node[Onyomi] at (-35.450000, 13.450000) {\hbox{\tate キ}};
\node[Meaning] at (-35.500000, 15.100000) {abandon};
\node[Kanji] at (-33.450000, 13.850000) {\textcolor[HTML]{cd8268}{育}};
\node[Square] at (-33.450000, 13.350000) {};
\node[Onyomi] at (-33.400000, 13.450000) {\hbox{\tate イク}};
\node[Kunyomi] at (-33.500000, 13.450000) {\hbox{\tate そだ}};
\node[Meaning] at (-33.450000, 15.100000) {nurture};
\node[Kanji] at (-31.400000, 13.850000) {\textcolor[HTML]{b0b0b5}{撤}};
\node[Square] at (-31.400000, 13.350000) {};
\node[Onyomi] at (-31.350000, 13.450000) {\hbox{\tate テツ}};
\node[Meaning] at (-31.400000, 15.100000) {withdrawal};
\node[Kanji] at (-29.350000, 13.850000) {\textcolor[HTML]{b0b0b5}{充}};
\node[Square] at (-29.350000, 13.350000) {};
\node[Onyomi] at (-29.300000, 13.450000) {\hbox{\tate ジュウ}};
\node[Kunyomi] at (-29.400000, 13.450000) {\hbox{\tate あ.てる}};
\node[Meaning] at (-29.350000, 15.100000) {allocate};
\node[Kanji] at (-27.300000, 13.850000) {\textcolor[HTML]{c8a59d}{銃}};
\node[Square] at (-27.300000, 13.350000) {};
\node[Onyomi] at (-27.250000, 13.450000) {\hbox{\tate ジュウ}};
\node[Meaning] at (-27.300000, 15.100000) {gun};
\node[Kanji] at (-25.250000, 13.850000) {\textcolor[HTML]{91b7c3}{硫}};
\node[Square] at (-25.250000, 13.350000) {};
\node[Onyomi] at (-25.200000, 13.450000) {\hbox{\tate リュウ}};
\node[Meaning] at (-25.250000, 15.100000) {sulfur};
\node[Kanji] at (-23.200000, 13.850000) {\textcolor[HTML]{c36143}{流}};
\node[Square] at (-23.200000, 13.350000) {};
\node[Onyomi] at (-23.150000, 13.450000) {\hbox{\tate リュウ・ル}};
\node[Kunyomi] at (-23.250000, 13.450000) {\hbox{\tate なが.*}};
\node[Meaning] at (-23.200000, 15.100000) {stream};
\node[Kanji] at (-21.150000, 13.850000) {\textcolor[HTML]{68a4bc}{唆}};
\node[Square] at (-21.150000, 13.350000) {};
\node[Onyomi] at (-21.100000, 13.450000) {\hbox{\tate サ}};
\node[Kunyomi] at (-21.200000, 13.450000) {\hbox{\tate そそのか}};
\node[Meaning] at (-21.150000, 15.100000) {instigate};
\node[Kanji] at (-19.100000, 13.850000) {\textcolor[HTML]{a11d25}{出}};
\node[Square] at (-19.100000, 13.350000) {};
\node[Onyomi] at (-19.050000, 13.450000) {\hbox{\tate シュツ}};
\node[Kunyomi] at (-19.150000, 13.450000) {\hbox{\tate で.る}};
\node[Meaning] at (-19.100000, 15.100000) {exit};
\node[Kanji] at (-17.050000, 13.850000) {\textcolor[HTML]{b74029}{山}};
\node[Square] at (-17.050000, 13.350000) {};
\node[Onyomi] at (-17.000000, 13.450000) {\hbox{\tate サン}};
\node[Kunyomi] at (-17.100000, 13.450000) {\hbox{\tate やま}};
\node[Meaning] at (-17.050000, 15.100000) {mountain};
\node[Kanji] at (-15.000000, 13.850000) {\textcolor[HTML]{1059be}{拙}};
\node[Square] at (-15.000000, 13.350000) {};
\node[Onyomi] at (-14.950000, 13.450000) {\hbox{\tate セツ}};
\node[Kunyomi] at (-15.050000, 13.450000) {\hbox{\tate つたな}};
\node[Meaning] at (-15.000000, 15.100000) {clumsy};
\node[Kanji] at (-12.950000, 13.850000) {\textcolor[HTML]{d2a293}{岩}};
\node[Square] at (-12.950000, 13.350000) {};
\node[Onyomi] at (-12.900000, 13.450000) {\hbox{\tate ガン}};
\node[Kunyomi] at (-13.000000, 13.450000) {\hbox{\tate いわ}};
\node[Meaning] at (-12.950000, 15.100000) {boulder};
\node[Kanji] at (-10.900000, 13.850000) {\textcolor[HTML]{b0b0b5}{炭}};
\node[Square] at (-10.900000, 13.350000) {};
\node[Onyomi] at (-10.850000, 13.450000) {\hbox{\tate タン}};
\node[Kunyomi] at (-10.950000, 13.450000) {\hbox{\tate すみ}};
\node[Meaning] at (-10.900000, 15.100000) {charcoal};
\node[Kanji] at (-8.850000, 13.850000) {\textcolor[HTML]{c8a59d}{岐}};
\node[Square] at (-8.850000, 13.350000) {};
\node[Onyomi] at (-8.800000, 13.450000) {\hbox{\tate キ・ギ}};
\node[Meaning] at (-8.850000, 15.100000) {branch off};
\node[Kanji] at (-6.800000, 13.850000) {\textcolor[HTML]{a3bac2}{峠}};
\node[Square] at (-6.800000, 13.350000) {};
\node[Kunyomi] at (-6.850000, 13.450000) {\hbox{\tate とうげ}};
\node[Meaning] at (-6.800000, 15.100000) {ridge};
\node[Kanji] at (-4.750000, 13.850000) {\textcolor[HTML]{b0b0b5}{崩}};
\node[Square] at (-4.750000, 13.350000) {};
\node[Onyomi] at (-4.700000, 13.450000) {\hbox{\tate ホウ}};
\node[Kunyomi] at (-4.800000, 13.450000) {\hbox{\tate くず.*}};
\node[Meaning] at (-4.750000, 15.100000) {crumble};
\node[Kanji] at (-2.700000, 13.850000) {\textcolor[HTML]{d2a293}{密}};
\node[Square] at (-2.700000, 13.350000) {};
\node[Onyomi] at (-2.650000, 13.450000) {\hbox{\tate ミツ}};
\node[Kunyomi] at (-2.750000, 13.450000) {\hbox{\tate ひそ.か}};
\node[Meaning] at (-2.700000, 15.100000) {secrecy};
\node[Kanji] at (-0.650000, 13.850000) {\textcolor[HTML]{408dba}{蜜}};
\node[Square] at (-0.650000, 13.350000) {};
\node[Onyomi] at (-0.600000, 13.450000) {\hbox{\tate ミツ}};
\node[Meaning] at (-0.650000, 15.100000) {honey};
\node[Kanji] at (1.400000, 13.850000) {\textcolor[HTML]{91b7c3}{嵐}};
\node[Square] at (1.400000, 13.350000) {};
\node[Kunyomi] at (1.350000, 13.450000) {\hbox{\tate あらし}};
\node[Meaning] at (1.400000, 15.100000) {storm};
\node[Kanji] at (3.450000, 13.850000) {\textcolor[HTML]{d69f8d}{崎}};
\node[Square] at (3.450000, 13.350000) {};
\node[Onyomi] at (3.500000, 13.450000) {\hbox{\tate キ}};
\node[Kunyomi] at (3.400000, 13.450000) {\hbox{\tate さき}};
\node[Meaning] at (3.450000, 15.100000) {cape};
\node[Kanji] at (5.500000, 13.850000) {\textcolor[HTML]{b74029}{入}};
\node[Square] at (5.500000, 13.350000) {};
\node[Onyomi] at (5.550000, 13.450000) {\hbox{\tate ニュウ}};
\node[Kunyomi] at (5.450000, 13.450000) {\hbox{\tate はい.る}};
\node[Meaning] at (5.500000, 15.100000) {enter};
\node[Kanji] at (7.550000, 13.850000) {\textcolor[HTML]{d69f8d}{込}};
\node[Square] at (7.550000, 13.350000) {};
\node[Kunyomi] at (7.500000, 13.450000) {\hbox{\tate こ}};
\node[Meaning] at (7.550000, 15.100000) {crowded};
\node[Kanji] at (9.600000, 13.850000) {\textcolor[HTML]{b74029}{分}};
\node[Square] at (9.600000, 13.350000) {};
\node[Onyomi] at (9.650000, 13.450000) {\hbox{\tate ブン・フン}};
\node[Kunyomi] at (9.550000, 13.450000) {\hbox{\tate わ.かる}};
\node[Meaning] at (9.600000, 15.100000) {part};
\node[Kanji] at (11.650000, 13.850000) {\textcolor[HTML]{91b7c3}{貧}};
\node[Square] at (11.650000, 13.350000) {};
\node[Onyomi] at (11.700000, 13.450000) {\hbox{\tate ビン・ヒン}};
\node[Kunyomi] at (11.600000, 13.450000) {\hbox{\tate まず.しい}};
\node[Meaning] at (11.650000, 15.100000) {poor};
\node[Kanji] at (13.700000, 13.850000) {\textcolor[HTML]{1059be}{頒}};
\node[Square] at (13.700000, 13.350000) {};
\node[Onyomi] at (13.750000, 13.450000) {\hbox{\tate ハン}};
\node[Meaning] at (13.700000, 15.100000) {partition};
\node[Kanji] at (15.750000, 13.850000) {\textcolor[HTML]{c36143}{公}};
\node[Square] at (15.750000, 13.350000) {};
\node[Onyomi] at (15.800000, 13.450000) {\hbox{\tate コウ}};
\node[Meaning] at (15.750000, 15.100000) {public};
\node[Kanji] at (17.800000, 13.850000) {\textcolor[HTML]{cd8268}{松}};
\node[Square] at (17.800000, 13.350000) {};
\node[Onyomi] at (17.850000, 13.450000) {\hbox{\tate ショウ}};
\node[Kunyomi] at (17.750000, 13.450000) {\hbox{\tate まつ}};
\node[Meaning] at (17.800000, 15.100000) {pine};
\node[Kanji] at (19.850000, 13.850000) {\textcolor[HTML]{408dba}{翁}};
\node[Square] at (19.850000, 13.350000) {};
\node[Onyomi] at (19.900000, 13.450000) {\hbox{\tate オウ}};
\node[Meaning] at (19.850000, 15.100000) {old man};
\node[Kanji] at (21.900000, 13.850000) {\textcolor[HTML]{91b7c3}{訟}};
\node[Square] at (21.900000, 13.350000) {};
\node[Onyomi] at (21.950000, 13.450000) {\hbox{\tate ショウ}};
\node[Meaning] at (21.900000, 15.100000) {lawsuit};
\node[Kanji] at (23.950000, 13.850000) {\textcolor[HTML]{d69f8d}{谷}};
\node[Square] at (23.950000, 13.350000) {};
\node[Onyomi] at (24.000000, 13.450000) {\hbox{\tate コク}};
\node[Kunyomi] at (23.900000, 13.450000) {\hbox{\tate たに}};
\node[Meaning] at (23.950000, 15.100000) {valley};
\node[Kanji] at (26.000000, 13.850000) {\textcolor[HTML]{a3bac2}{浴}};
\node[Square] at (26.000000, 13.350000) {};
\node[Onyomi] at (26.050000, 13.450000) {\hbox{\tate ヨク}};
\node[Kunyomi] at (25.950000, 13.450000) {\hbox{\tate あ}};
\node[Meaning] at (26.000000, 15.100000) {bathe};
\node[Kanji] at (28.050000, 13.850000) {\textcolor[HTML]{d69f8d}{容}};
\node[Square] at (28.050000, 13.350000) {};
\node[Onyomi] at (28.100000, 13.450000) {\hbox{\tate ヨウ}};
\node[Meaning] at (28.050000, 15.100000) {form};
\node[Kanji] at (30.100000, 13.850000) {\textcolor[HTML]{b0b0b5}{溶}};
\node[Square] at (30.100000, 13.350000) {};
\node[Onyomi] at (30.150000, 13.450000) {\hbox{\tate ヨウ}};
\node[Kunyomi] at (30.050000, 13.450000) {\hbox{\tate と.ける}};
\node[Meaning] at (30.100000, 15.100000) {melt};
\node[Kanji] at (32.150000, 13.850000) {\textcolor[HTML]{a3bac2}{欲}};
\node[Square] at (32.150000, 13.350000) {};
\node[Onyomi] at (32.200000, 13.450000) {\hbox{\tate ヨク}};
\node[Kunyomi] at (32.100000, 13.450000) {\hbox{\tate ほ.しい}};
\node[Meaning] at (32.150000, 15.100000) {want};
\node[Kanji] at (34.200000, 13.850000) {\textcolor[HTML]{a3bac2}{裕}};
\node[Square] at (34.200000, 13.350000) {};
\node[Onyomi] at (34.250000, 13.450000) {\hbox{\tate ユウ}};
\node[Meaning] at (34.200000, 15.100000) {abundant};
\node[Kanji] at (36.250000, 13.850000) {\textcolor[HTML]{68a4bc}{鉛}};
\node[Square] at (36.250000, 13.350000) {};
\node[Onyomi] at (36.300000, 13.450000) {\hbox{\tate エン}};
\node[Kunyomi] at (36.200000, 13.450000) {\hbox{\tate なまり}};
\node[Meaning] at (36.250000, 15.100000) {lead};
\node[Kanji] at (38.300000, 13.850000) {\textcolor[HTML]{c8a59d}{沿}};
\node[Square] at (38.300000, 13.350000) {};
\node[Onyomi] at (38.350000, 13.450000) {\hbox{\tate エン}};
\node[Kunyomi] at (38.250000, 13.450000) {\hbox{\tate そ.う}};
\node[Meaning] at (38.300000, 15.100000) {run alongside};
\node[Kanji] at (40.350000, 13.850000) {\textcolor[HTML]{d69f8d}{賞}};
\node[Square] at (40.350000, 13.350000) {};
\node[Onyomi] at (40.400000, 13.450000) {\hbox{\tate ショウ}};
\node[Meaning] at (40.350000, 15.100000) {prize};
\node[Kanji] at (42.400000, 13.850000) {\textcolor[HTML]{d69f8d}{党}};
\node[Square] at (42.400000, 13.350000) {};
\node[Onyomi] at (42.450000, 13.450000) {\hbox{\tate トウ}};
\node[Meaning] at (42.400000, 15.100000) {group};
\node[Kanji] at (44.450000, 13.850000) {\textcolor[HTML]{d2a293}{堂}};
\node[Square] at (44.450000, 13.350000) {};
\node[Onyomi] at (44.500000, 13.450000) {\hbox{\tate ドウ}};
\node[Meaning] at (44.450000, 15.100000) {hall};
\node[Kanji] at (46.500000, 13.850000) {\textcolor[HTML]{cd8268}{常}};
\node[Square] at (46.500000, 13.350000) {};
\node[Onyomi] at (46.550000, 13.450000) {\hbox{\tate ジョウ}};
\node[Kunyomi] at (46.450000, 13.450000) {\hbox{\tate つね}};
\node[Meaning] at (46.500000, 15.100000) {normal};
\node[Kanji] at (48.550000, 13.850000) {\textcolor[HTML]{91b7c3}{掌}};
\node[Square] at (48.550000, 13.350000) {};
\node[Onyomi] at (48.600000, 13.450000) {\hbox{\tate ショウ}};
\node[Kunyomi] at (48.500000, 13.450000) {\hbox{\tate てのひら}};
\node[Meaning] at (48.550000, 15.100000) {manipulate};
\node[Kanji] at (50.600000, 13.850000) {\textcolor[HTML]{b0b0b5}{皮}};
\node[Square] at (50.600000, 13.350000) {};
\node[Onyomi] at (50.650000, 13.450000) {\hbox{\tate ヒ}};
\node[Kunyomi] at (50.550000, 13.450000) {\hbox{\tate かわ}};
\node[Meaning] at (50.600000, 15.100000) {skin};
\node[Kanji] at (52.650000, 13.850000) {\textcolor[HTML]{d69f8d}{波}};
\node[Square] at (52.650000, 13.350000) {};
\node[Onyomi] at (52.700000, 13.450000) {\hbox{\tate ハ}};
\node[Kunyomi] at (52.600000, 13.450000) {\hbox{\tate なみ}};
\node[Meaning] at (52.650000, 15.100000) {wave};
\node[Kanji] at (54.700000, 13.850000) {\textcolor[HTML]{408dba}{婆}};
\node[Square] at (54.700000, 13.350000) {};
\node[Onyomi] at (54.750000, 13.450000) {\hbox{\tate バ}};
\node[Kunyomi] at (54.650000, 13.450000) {\hbox{\tate ばあ}};
\node[Meaning] at (54.700000, 15.100000) {old woman};
\node[Kanji] at (56.750000, 13.850000) {\textcolor[HTML]{a3bac2}{披}};
\node[Square] at (56.750000, 13.350000) {};
\node[Onyomi] at (56.800000, 13.450000) {\hbox{\tate ヒ}};
\node[Meaning] at (56.750000, 15.100000) {expose};
\node[Meaning] at (-58.050000, 13.950000) {47.42\%};
\node[Kanji] at (-56.000000, 11.800000) {\textcolor[HTML]{d2a293}{破}};
\node[Square] at (-56.000000, 11.300000) {};
\node[Onyomi] at (-55.950000, 11.400000) {\hbox{\tate ハ}};
\node[Kunyomi] at (-56.050000, 11.400000) {\hbox{\tate やぶ.*}};
\node[Meaning] at (-56.000000, 13.050000) {tear};
\node[Kanji] at (-53.950000, 11.800000) {\textcolor[HTML]{d2a293}{被}};
\node[Square] at (-53.950000, 11.300000) {};
\node[Onyomi] at (-53.900000, 11.400000) {\hbox{\tate ヒ}};
\node[Kunyomi] at (-54.000000, 11.400000) {\hbox{\tate かぶ.る}};
\node[Meaning] at (-53.950000, 13.050000) {incur};
\node[Kanji] at (-51.900000, 11.800000) {\textcolor[HTML]{d69f8d}{残}};
\node[Square] at (-51.900000, 11.300000) {};
\node[Onyomi] at (-51.850000, 11.400000) {\hbox{\tate ザン}};
\node[Kunyomi] at (-51.950000, 11.400000) {\hbox{\tate のこ.*}};
\node[Meaning] at (-51.900000, 13.050000) {remainder};
\node[Kanji] at (-49.850000, 11.800000) {\textcolor[HTML]{408dba}{殉}};
\node[Square] at (-49.850000, 11.300000) {};
\node[Onyomi] at (-49.800000, 11.400000) {\hbox{\tate ジュン}};
\node[Meaning] at (-49.850000, 13.050000) {martyr};
\node[Kanji] at (-47.800000, 11.800000) {\textcolor[HTML]{b0b0b5}{殊}};
\node[Square] at (-47.800000, 11.300000) {};
\node[Onyomi] at (-47.750000, 11.400000) {\hbox{\tate シュ}};
\node[Kunyomi] at (-47.850000, 11.400000) {\hbox{\tate こと}};
\node[Meaning] at (-47.800000, 13.050000) {especially};
\node[Kanji] at (-45.750000, 11.800000) {\textcolor[HTML]{b0b0b5}{殖}};
\node[Square] at (-45.750000, 11.300000) {};
\node[Onyomi] at (-45.700000, 11.400000) {\hbox{\tate ショク}};
\node[Kunyomi] at (-45.800000, 11.400000) {\hbox{\tate ふ.える}};
\node[Meaning] at (-45.750000, 13.050000) {multiply};
\node[Kanji] at (-43.700000, 11.800000) {\textcolor[HTML]{d69f8d}{列}};
\node[Square] at (-43.700000, 11.300000) {};
\node[Onyomi] at (-43.650000, 11.400000) {\hbox{\tate レツ・レ}};
\node[Meaning] at (-43.700000, 13.050000) {row};
\node[Kanji] at (-41.650000, 11.800000) {\textcolor[HTML]{a3bac2}{裂}};
\node[Square] at (-41.650000, 11.300000) {};
\node[Onyomi] at (-41.600000, 11.400000) {\hbox{\tate レツ}};
\node[Kunyomi] at (-41.700000, 11.400000) {\hbox{\tate さ.*}};
\node[Meaning] at (-41.650000, 13.050000) {split};
\node[Kanji] at (-39.600000, 11.800000) {\textcolor[HTML]{91b7c3}{烈}};
\node[Square] at (-39.600000, 11.300000) {};
\node[Onyomi] at (-39.550000, 11.400000) {\hbox{\tate レツ}};
\node[Kunyomi] at (-39.650000, 11.400000) {\hbox{\tate はげ.しい}};
\node[Meaning] at (-39.600000, 13.050000) {violent};
\node[Kanji] at (-37.550000, 11.800000) {\textcolor[HTML]{cd8268}{死}};
\node[Square] at (-37.550000, 11.300000) {};
\node[Onyomi] at (-37.500000, 11.400000) {\hbox{\tate シ}};
\node[Kunyomi] at (-37.600000, 11.400000) {\hbox{\tate し.ぬ}};
\node[Meaning] at (-37.550000, 13.050000) {death};
\node[Kanji] at (-35.500000, 11.800000) {\textcolor[HTML]{b0b0b5}{葬}};
\node[Square] at (-35.500000, 11.300000) {};
\node[Onyomi] at (-35.450000, 11.400000) {\hbox{\tate ソウ}};
\node[Kunyomi] at (-35.550000, 11.400000) {\hbox{\tate ほうむ.る}};
\node[Meaning] at (-35.500000, 13.050000) {burial};
\node[Kanji] at (-33.450000, 11.800000) {\textcolor[HTML]{91b7c3}{瞬}};
\node[Square] at (-33.450000, 11.300000) {};
\node[Onyomi] at (-33.400000, 11.400000) {\hbox{\tate シュン}};
\node[Kunyomi] at (-33.500000, 11.400000) {\hbox{\tate またた.く}};
\node[Meaning] at (-33.450000, 13.050000) {blink};
\node[Kanji] at (-31.400000, 11.800000) {\textcolor[HTML]{a3bac2}{耳}};
\node[Square] at (-31.400000, 11.300000) {};
\node[Onyomi] at (-31.350000, 11.400000) {\hbox{\tate ジ}};
\node[Kunyomi] at (-31.450000, 11.400000) {\hbox{\tate みみ}};
\node[Meaning] at (-31.400000, 13.050000) {ear};
\node[Kanji] at (-29.350000, 11.800000) {\textcolor[HTML]{c36143}{取}};
\node[Square] at (-29.350000, 11.300000) {};
\node[Onyomi] at (-29.300000, 11.400000) {\hbox{\tate シュ}};
\node[Kunyomi] at (-29.400000, 11.400000) {\hbox{\tate と}};
\node[Meaning] at (-29.350000, 13.050000) {take};
\node[Kanji] at (-27.300000, 11.800000) {\textcolor[HTML]{b0b0b5}{趣}};
\node[Square] at (-27.300000, 11.300000) {};
\node[Onyomi] at (-27.250000, 11.400000) {\hbox{\tate シュ}};
\node[Kunyomi] at (-27.350000, 11.400000) {\hbox{\tate おもむき}};
\node[Meaning] at (-27.300000, 13.050000) {gist};
\node[Kanji] at (-25.250000, 11.800000) {\textcolor[HTML]{b74029}{最}};
\node[Square] at (-25.250000, 11.300000) {};
\node[Onyomi] at (-25.200000, 11.400000) {\hbox{\tate サイ}};
\node[Kunyomi] at (-25.300000, 11.400000) {\hbox{\tate もっと}};
\node[Meaning] at (-25.250000, 13.050000) {most};
\node[Kanji] at (-23.200000, 11.800000) {\textcolor[HTML]{d2a293}{撮}};
\node[Square] at (-23.200000, 11.300000) {};
\node[Onyomi] at (-23.150000, 11.400000) {\hbox{\tate サツ}};
\node[Kunyomi] at (-23.250000, 11.400000) {\hbox{\tate と.る}};
\node[Meaning] at (-23.200000, 13.050000) {photograph};
\node[Kanji] at (-21.150000, 11.800000) {\textcolor[HTML]{408dba}{恥}};
\node[Square] at (-21.150000, 11.300000) {};
\node[Onyomi] at (-21.100000, 11.400000) {\hbox{\tate チ}};
\node[Kunyomi] at (-21.200000, 11.400000) {\hbox{\tate は・はじ}};
\node[Meaning] at (-21.150000, 13.050000) {shame};
\node[Kanji] at (-19.100000, 11.800000) {\textcolor[HTML]{d69f8d}{職}};
\node[Square] at (-19.100000, 11.300000) {};
\node[Onyomi] at (-19.050000, 11.400000) {\hbox{\tate ショク}};
\node[Meaning] at (-19.100000, 13.050000) {employment};
\node[Kanji] at (-17.050000, 11.800000) {\textcolor[HTML]{d2a293}{聖}};
\node[Square] at (-17.050000, 11.300000) {};
\node[Onyomi] at (-17.000000, 11.400000) {\hbox{\tate セイ}};
\node[Meaning] at (-17.050000, 13.050000) {holy};
\node[Kanji] at (-15.000000, 11.800000) {\textcolor[HTML]{68a4bc}{敢}};
\node[Square] at (-15.000000, 11.300000) {};
\node[Onyomi] at (-14.950000, 11.400000) {\hbox{\tate カン}};
\node[Kunyomi] at (-15.050000, 11.400000) {\hbox{\tate あ}};
\node[Meaning] at (-15.000000, 13.050000) {daring};
\node[Kanji] at (-12.950000, 11.800000) {\textcolor[HTML]{c8a59d}{聴}};
\node[Square] at (-12.950000, 11.300000) {};
\node[Onyomi] at (-12.900000, 11.400000) {\hbox{\tate チョウ}};
\node[Kunyomi] at (-13.000000, 11.400000) {\hbox{\tate き.く}};
\node[Meaning] at (-12.950000, 13.050000) {listen};
\node[Kanji] at (-10.900000, 11.800000) {\textcolor[HTML]{91b7c3}{懐}};
\node[Square] at (-10.900000, 11.300000) {};
\node[Onyomi] at (-10.850000, 11.400000) {\hbox{\tate カイ}};
\node[Kunyomi] at (-10.950000, 11.400000) {\hbox{\tate なつ}};
\node[Meaning] at (-10.900000, 13.050000) {nostalgia};
\node[Kanji] at (-8.850000, 11.800000) {\textcolor[HTML]{91b7c3}{慢}};
\node[Square] at (-8.850000, 11.300000) {};
\node[Onyomi] at (-8.800000, 11.400000) {\hbox{\tate マン}};
\node[Meaning] at (-8.850000, 13.050000) {ridicule};
\node[Kanji] at (-6.800000, 11.800000) {\textcolor[HTML]{c8a59d}{漫}};
\node[Square] at (-6.800000, 11.300000) {};
\node[Onyomi] at (-6.750000, 11.400000) {\hbox{\tate マン}};
\node[Meaning] at (-6.800000, 13.050000) {manga};
\node[Kanji] at (-4.750000, 11.800000) {\textcolor[HTML]{c8a59d}{買}};
\node[Square] at (-4.750000, 11.300000) {};
\node[Onyomi] at (-4.700000, 11.400000) {\hbox{\tate バイ}};
\node[Kunyomi] at (-4.800000, 11.400000) {\hbox{\tate か}};
\node[Meaning] at (-4.750000, 13.050000) {buy};
\node[Kanji] at (-2.700000, 11.800000) {\textcolor[HTML]{c36143}{置}};
\node[Square] at (-2.700000, 11.300000) {};
\node[Onyomi] at (-2.650000, 11.400000) {\hbox{\tate チ}};
\node[Kunyomi] at (-2.750000, 11.400000) {\hbox{\tate お.く}};
\node[Meaning] at (-2.700000, 13.050000) {put};
\node[Kanji] at (-0.650000, 11.800000) {\textcolor[HTML]{a3bac2}{罰}};
\node[Square] at (-0.650000, 11.300000) {};
\node[Onyomi] at (-0.600000, 11.400000) {\hbox{\tate バツ}};
\node[Kunyomi] at (-0.700000, 11.400000) {\hbox{\tate ばつ}};
\node[Meaning] at (-0.650000, 13.050000) {penalty};
\node[Kanji] at (1.400000, 11.800000) {\textcolor[HTML]{91b7c3}{寧}};
\node[Square] at (1.400000, 11.300000) {};
\node[Onyomi] at (1.450000, 11.400000) {\hbox{\tate ネイ}};
\node[Kunyomi] at (1.350000, 11.400000) {\hbox{\tate むし.ろ}};
\node[Meaning] at (1.400000, 13.050000) {rather};
\node[Kanji] at (3.450000, 11.800000) {\textcolor[HTML]{408dba}{濁}};
\node[Square] at (3.450000, 11.300000) {};
\node[Onyomi] at (3.500000, 11.400000) {\hbox{\tate ダク・ジョク}};
\node[Kunyomi] at (3.400000, 11.400000) {\hbox{\tate にご}};
\node[Meaning] at (3.450000, 13.050000) {muddy};
\node[Kanji] at (5.500000, 11.800000) {\textcolor[HTML]{d2a293}{環}};
\node[Square] at (5.500000, 11.300000) {};
\node[Onyomi] at (5.550000, 11.400000) {\hbox{\tate カン}};
\node[Meaning] at (5.500000, 13.050000) {loop};
\node[Kanji] at (7.550000, 11.800000) {\textcolor[HTML]{b0b0b5}{還}};
\node[Square] at (7.550000, 11.300000) {};
\node[Onyomi] at (7.600000, 11.400000) {\hbox{\tate カン}};
\node[Kunyomi] at (7.500000, 11.400000) {\hbox{\tate かえ.る}};
\node[Meaning] at (7.550000, 13.050000) {send back};
\node[Kanji] at (9.600000, 11.800000) {\textcolor[HTML]{d2a293}{夫}};
\node[Square] at (9.600000, 11.300000) {};
\node[Onyomi] at (9.650000, 11.400000) {\hbox{\tate フウ・フ}};
\node[Kunyomi] at (9.550000, 11.400000) {\hbox{\tate おっと}};
\node[Meaning] at (9.600000, 13.050000) {husband};
\node[Kanji] at (11.650000, 11.800000) {\textcolor[HTML]{68a4bc}{扶}};
\node[Square] at (11.650000, 11.300000) {};
\node[Onyomi] at (11.700000, 11.400000) {\hbox{\tate フ}};
\node[Kunyomi] at (11.600000, 11.400000) {\hbox{\tate たす}};
\node[Meaning] at (11.650000, 13.050000) {aid};
\node[Kanji] at (13.700000, 11.800000) {\textcolor[HTML]{91b7c3}{渓}};
\node[Square] at (13.700000, 11.300000) {};
\node[Onyomi] at (13.750000, 11.400000) {\hbox{\tate ケイ}};
\node[Kunyomi] at (13.650000, 11.400000) {\hbox{\tate たに}};
\node[Meaning] at (13.700000, 13.050000) {valley};
\node[Kanji] at (15.750000, 11.800000) {\textcolor[HTML]{d69f8d}{規}};
\node[Square] at (15.750000, 11.300000) {};
\node[Onyomi] at (15.800000, 11.400000) {\hbox{\tate キ}};
\node[Meaning] at (15.750000, 13.050000) {standard};
\node[Kanji] at (17.800000, 11.800000) {\textcolor[HTML]{c8a59d}{替}};
\node[Square] at (17.800000, 11.300000) {};
\node[Onyomi] at (17.850000, 11.400000) {\hbox{\tate タイ}};
\node[Kunyomi] at (17.750000, 11.400000) {\hbox{\tate か}};
\node[Meaning] at (17.800000, 13.050000) {replace};
\node[Kanji] at (19.850000, 11.800000) {\textcolor[HTML]{b0b0b5}{賛}};
\node[Square] at (19.850000, 11.300000) {};
\node[Onyomi] at (19.900000, 11.400000) {\hbox{\tate サン}};
\node[Meaning] at (19.850000, 13.050000) {agree};
\node[Kanji] at (21.900000, 11.800000) {\textcolor[HTML]{b0b0b5}{潜}};
\node[Square] at (21.900000, 11.300000) {};
\node[Onyomi] at (21.950000, 11.400000) {\hbox{\tate セン}};
\node[Kunyomi] at (21.850000, 11.400000) {\hbox{\tate くぐ.る}};
\node[Meaning] at (21.900000, 13.050000) {conceal};
\node[Kanji] at (23.950000, 11.800000) {\textcolor[HTML]{d2a293}{失}};
\node[Square] at (23.950000, 11.300000) {};
\node[Onyomi] at (24.000000, 11.400000) {\hbox{\tate シツ}};
\node[Kunyomi] at (23.900000, 11.400000) {\hbox{\tate うしな.う}};
\node[Meaning] at (23.950000, 13.050000) {fault};
\node[Kanji] at (26.000000, 11.800000) {\textcolor[HTML]{c36143}{鉄}};
\node[Square] at (26.000000, 11.300000) {};
\node[Onyomi] at (26.050000, 11.400000) {\hbox{\tate テツ}};
\node[Meaning] at (26.000000, 13.050000) {iron};
\node[Kanji] at (28.050000, 11.800000) {\textcolor[HTML]{1059be}{迭}};
\node[Square] at (28.050000, 11.300000) {};
\node[Onyomi] at (28.100000, 11.400000) {\hbox{\tate テツ}};
\node[Meaning] at (28.050000, 13.050000) {alternate};
\node[Kanji] at (30.100000, 11.800000) {\textcolor[HTML]{d2a293}{臣}};
\node[Square] at (30.100000, 11.300000) {};
\node[Onyomi] at (30.150000, 11.400000) {\hbox{\tate シン・ジン}};
\node[Meaning] at (30.100000, 13.050000) {servant};
\node[Kanji] at (32.150000, 11.800000) {\textcolor[HTML]{b0b0b5}{姫}};
\node[Square] at (32.150000, 11.300000) {};
\node[Kunyomi] at (32.100000, 11.400000) {\hbox{\tate ひめ}};
\node[Meaning] at (32.150000, 13.050000) {princess};
\node[Kanji] at (34.200000, 11.800000) {\textcolor[HTML]{d2a293}{蔵}};
\node[Square] at (34.200000, 11.300000) {};
\node[Onyomi] at (34.250000, 11.400000) {\hbox{\tate ゾウ}};
\node[Kunyomi] at (34.150000, 11.400000) {\hbox{\tate くら}};
\node[Meaning] at (34.200000, 13.050000) {storehouse};
\node[Kanji] at (36.250000, 11.800000) {\textcolor[HTML]{91b7c3}{臓}};
\node[Square] at (36.250000, 11.300000) {};
\node[Onyomi] at (36.300000, 11.400000) {\hbox{\tate ゾウ}};
\node[Meaning] at (36.250000, 13.050000) {internal organs};
\node[Kanji] at (38.300000, 11.800000) {\textcolor[HTML]{a3bac2}{賢}};
\node[Square] at (38.300000, 11.300000) {};
\node[Onyomi] at (38.350000, 11.400000) {\hbox{\tate ケン}};
\node[Kunyomi] at (38.250000, 11.400000) {\hbox{\tate かしこ.い}};
\node[Meaning] at (38.300000, 13.050000) {clever};
\node[Kanji] at (40.350000, 11.800000) {\textcolor[HTML]{91b7c3}{堅}};
\node[Square] at (40.350000, 11.300000) {};
\node[Onyomi] at (40.400000, 11.400000) {\hbox{\tate ケン}};
\node[Kunyomi] at (40.300000, 11.400000) {\hbox{\tate かた}};
\node[Meaning] at (40.350000, 13.050000) {solid};
\node[Kanji] at (42.400000, 11.800000) {\textcolor[HTML]{c8a59d}{臨}};
\node[Square] at (42.400000, 11.300000) {};
\node[Onyomi] at (42.450000, 11.400000) {\hbox{\tate リン}};
\node[Kunyomi] at (42.350000, 11.400000) {\hbox{\tate のぞ.む}};
\node[Meaning] at (42.400000, 13.050000) {look to};
\node[Kanji] at (44.450000, 11.800000) {\textcolor[HTML]{d69f8d}{覧}};
\node[Square] at (44.450000, 11.300000) {};
\node[Onyomi] at (44.500000, 11.400000) {\hbox{\tate ラン}};
\node[Meaning] at (44.450000, 13.050000) {look at};
\node[Kanji] at (46.500000, 11.800000) {\textcolor[HTML]{b0b0b5}{巨}};
\node[Square] at (46.500000, 11.300000) {};
\node[Onyomi] at (46.550000, 11.400000) {\hbox{\tate キョ}};
\node[Meaning] at (46.500000, 13.050000) {giant};
\node[Kanji] at (48.550000, 11.800000) {\textcolor[HTML]{a3bac2}{拒}};
\node[Square] at (48.550000, 11.300000) {};
\node[Onyomi] at (48.600000, 11.400000) {\hbox{\tate キョ}};
\node[Kunyomi] at (48.500000, 11.400000) {\hbox{\tate こば.む}};
\node[Meaning] at (48.550000, 13.050000) {refusal};
\node[Kanji] at (50.600000, 11.800000) {\textcolor[HTML]{c36143}{力}};
\node[Square] at (50.600000, 11.300000) {};
\node[Onyomi] at (50.650000, 11.400000) {\hbox{\tate リョク・リキ}};
\node[Kunyomi] at (50.550000, 11.400000) {\hbox{\tate ちから}};
\node[Meaning] at (50.600000, 13.050000) {power};
\node[Kanji] at (52.650000, 11.800000) {\textcolor[HTML]{cd8268}{男}};
\node[Square] at (52.650000, 11.300000) {};
\node[Onyomi] at (52.700000, 11.400000) {\hbox{\tate ダン}};
\node[Kunyomi] at (52.600000, 11.400000) {\hbox{\tate おとこ}};
\node[Meaning] at (52.650000, 13.050000) {man};
\node[Kanji] at (54.700000, 11.800000) {\textcolor[HTML]{d2a293}{労}};
\node[Square] at (54.700000, 11.300000) {};
\node[Onyomi] at (54.750000, 11.400000) {\hbox{\tate ロウ}};
\node[Kunyomi] at (54.650000, 11.400000) {\hbox{\tate いたわ.る}};
\node[Meaning] at (54.700000, 13.050000) {labor};
\node[Kanji] at (56.750000, 11.800000) {\textcolor[HTML]{b0b0b5}{募}};
\node[Square] at (56.750000, 11.300000) {};
\node[Onyomi] at (56.800000, 11.400000) {\hbox{\tate ボ}};
\node[Kunyomi] at (56.700000, 11.400000) {\hbox{\tate つの.る}};
\node[Meaning] at (56.750000, 13.050000) {recruit};
\node[Meaning] at (-58.050000, 11.900000) {49.45\%};
\node[Kanji] at (-56.000000, 9.750000) {\textcolor[HTML]{91b7c3}{劣}};
\node[Square] at (-56.000000, 9.250000) {};
\node[Onyomi] at (-55.950000, 9.350000) {\hbox{\tate レツ}};
\node[Kunyomi] at (-56.050000, 9.350000) {\hbox{\tate おと.る}};
\node[Meaning] at (-56.000000, 11.000000) {inferiority};
\node[Kanji] at (-53.950000, 9.750000) {\textcolor[HTML]{d2a293}{功}};
\node[Square] at (-53.950000, 9.250000) {};
\node[Onyomi] at (-53.900000, 9.350000) {\hbox{\tate コウ}};
\node[Meaning] at (-53.950000, 11.000000) {achievement};
\node[Kanji] at (-51.900000, 9.750000) {\textcolor[HTML]{a3bac2}{勧}};
\node[Square] at (-51.900000, 9.250000) {};
\node[Onyomi] at (-51.850000, 9.350000) {\hbox{\tate カン}};
\node[Kunyomi] at (-51.950000, 9.350000) {\hbox{\tate すす.める}};
\node[Meaning] at (-51.900000, 11.000000) {recommend};
\node[Kanji] at (-49.850000, 9.750000) {\textcolor[HTML]{a3bac2}{努}};
\node[Square] at (-49.850000, 9.250000) {};
\node[Onyomi] at (-49.800000, 9.350000) {\hbox{\tate ド}};
\node[Kunyomi] at (-49.900000, 9.350000) {\hbox{\tate つと.める}};
\node[Meaning] at (-49.850000, 11.000000) {toil};
\node[Kanji] at (-47.800000, 9.750000) {\textcolor[HTML]{91b7c3}{励}};
\node[Square] at (-47.800000, 9.250000) {};
\node[Onyomi] at (-47.750000, 9.350000) {\hbox{\tate レイ}};
\node[Kunyomi] at (-47.850000, 9.350000) {\hbox{\tate はげ.*}};
\node[Meaning] at (-47.800000, 11.000000) {encourage};
\node[Kanji] at (-45.750000, 9.750000) {\textcolor[HTML]{c36143}{加}};
\node[Square] at (-45.750000, 9.250000) {};
\node[Onyomi] at (-45.700000, 9.350000) {\hbox{\tate カ}};
\node[Kunyomi] at (-45.800000, 9.350000) {\hbox{\tate くわ.える}};
\node[Meaning] at (-45.750000, 11.000000) {add};
\node[Kanji] at (-43.700000, 9.750000) {\textcolor[HTML]{d2a293}{賀}};
\node[Square] at (-43.700000, 9.250000) {};
\node[Onyomi] at (-43.650000, 9.350000) {\hbox{\tate ガ}};
\node[Meaning] at (-43.700000, 11.000000) {congratulations};
\node[Kanji] at (-41.650000, 9.750000) {\textcolor[HTML]{b0b0b5}{架}};
\node[Square] at (-41.650000, 9.250000) {};
\node[Onyomi] at (-41.600000, 9.350000) {\hbox{\tate カ}};
\node[Kunyomi] at (-41.700000, 9.350000) {\hbox{\tate か.*}};
\node[Meaning] at (-41.650000, 11.000000) {shelf};
\node[Kanji] at (-39.600000, 9.750000) {\textcolor[HTML]{a3bac2}{脇}};
\node[Square] at (-39.600000, 9.250000) {};
\node[Onyomi] at (-39.550000, 9.350000) {\hbox{\tate キョウ}};
\node[Kunyomi] at (-39.650000, 9.350000) {\hbox{\tate わき}};
\node[Meaning] at (-39.600000, 11.000000) {armpit};
\node[Kanji] at (-37.550000, 9.750000) {\textcolor[HTML]{91b7c3}{脅}};
\node[Square] at (-37.550000, 9.250000) {};
\node[Onyomi] at (-37.500000, 9.350000) {\hbox{\tate キョウ}};
\node[Kunyomi] at (-37.600000, 9.350000) {\hbox{\tate おど}};
\node[Meaning] at (-37.550000, 11.000000) {threaten};
\node[Kanji] at (-35.500000, 9.750000) {\textcolor[HTML]{d69f8d}{協}};
\node[Square] at (-35.500000, 9.250000) {};
\node[Onyomi] at (-35.450000, 9.350000) {\hbox{\tate キョウ}};
\node[Meaning] at (-35.500000, 11.000000) {cooperation};
\node[Kanji] at (-33.450000, 9.750000) {\textcolor[HTML]{a11d25}{行}};
\node[Square] at (-33.450000, 9.250000) {};
\node[Onyomi] at (-33.400000, 9.350000) {\hbox{\tate コウ・ギョウ}};
\node[Kunyomi] at (-33.500000, 9.350000) {\hbox{\tate い.く}};
\node[Meaning] at (-33.450000, 11.000000) {go};
\node[Kanji] at (-31.400000, 9.750000) {\textcolor[HTML]{c8a59d}{律}};
\node[Square] at (-31.400000, 9.250000) {};
\node[Onyomi] at (-31.350000, 9.350000) {\hbox{\tate リツ}};
\node[Meaning] at (-31.400000, 11.000000) {law};
\node[Kanji] at (-29.350000, 9.750000) {\textcolor[HTML]{d69f8d}{復}};
\node[Square] at (-29.350000, 9.250000) {};
\node[Onyomi] at (-29.300000, 9.350000) {\hbox{\tate フク}};
\node[Meaning] at (-29.350000, 11.000000) {restore};
\node[Kanji] at (-27.300000, 9.750000) {\textcolor[HTML]{cd8268}{得}};
\node[Square] at (-27.300000, 9.250000) {};
\node[Onyomi] at (-27.250000, 9.350000) {\hbox{\tate トク}};
\node[Kunyomi] at (-27.350000, 9.350000) {\hbox{\tate え.る}};
\node[Meaning] at (-27.300000, 11.000000) {acquire};
\node[Kanji] at (-25.250000, 9.750000) {\textcolor[HTML]{d69f8d}{従}};
\node[Square] at (-25.250000, 9.250000) {};
\node[Onyomi] at (-25.200000, 9.350000) {\hbox{\tate ジュウ}};
\node[Kunyomi] at (-25.300000, 9.350000) {\hbox{\tate したが.う}};
\node[Meaning] at (-25.250000, 11.000000) {obey};
\node[Kanji] at (-23.200000, 9.750000) {\textcolor[HTML]{c8a59d}{徒}};
\node[Square] at (-23.200000, 9.250000) {};
\node[Onyomi] at (-23.150000, 9.350000) {\hbox{\tate ト}};
\node[Meaning] at (-23.200000, 11.000000) {junior};
\node[Kanji] at (-21.150000, 9.750000) {\textcolor[HTML]{c8a59d}{待}};
\node[Square] at (-21.150000, 9.250000) {};
\node[Onyomi] at (-21.100000, 9.350000) {\hbox{\tate タイ}};
\node[Kunyomi] at (-21.200000, 9.350000) {\hbox{\tate ま}};
\node[Meaning] at (-21.150000, 11.000000) {wait};
\node[Kanji] at (-19.100000, 9.750000) {\textcolor[HTML]{a3bac2}{往}};
\node[Square] at (-19.100000, 9.250000) {};
\node[Onyomi] at (-19.050000, 9.350000) {\hbox{\tate オウ}};
\node[Meaning] at (-19.100000, 11.000000) {depart};
\node[Kanji] at (-17.050000, 9.750000) {\textcolor[HTML]{b0b0b5}{征}};
\node[Square] at (-17.050000, 9.250000) {};
\node[Onyomi] at (-17.000000, 9.350000) {\hbox{\tate セイ}};
\node[Meaning] at (-17.050000, 11.000000) {subjugate};
\node[Kanji] at (-15.000000, 9.750000) {\textcolor[HTML]{b0b0b5}{径}};
\node[Square] at (-15.000000, 9.250000) {};
\node[Onyomi] at (-14.950000, 9.350000) {\hbox{\tate ケイ}};
\node[Meaning] at (-15.000000, 11.000000) {diameter};
\node[Kanji] at (-12.950000, 9.750000) {\textcolor[HTML]{cd8268}{彼}};
\node[Square] at (-12.950000, 9.250000) {};
\node[Onyomi] at (-12.900000, 9.350000) {\hbox{\tate ヒ}};
\node[Kunyomi] at (-13.000000, 9.350000) {\hbox{\tate かれ・かの}};
\node[Meaning] at (-12.950000, 11.000000) {he};
\node[Kanji] at (-10.900000, 9.750000) {\textcolor[HTML]{cd8268}{役}};
\node[Square] at (-10.900000, 9.250000) {};
\node[Onyomi] at (-10.850000, 9.350000) {\hbox{\tate ヤク}};
\node[Meaning] at (-10.900000, 11.000000) {service};
\node[Kanji] at (-8.850000, 9.750000) {\textcolor[HTML]{d2a293}{徳}};
\node[Square] at (-8.850000, 9.250000) {};
\node[Onyomi] at (-8.800000, 9.350000) {\hbox{\tate トク}};
\node[Meaning] at (-8.850000, 11.000000) {virtue};
\node[Kanji] at (-6.800000, 9.750000) {\textcolor[HTML]{a3bac2}{徹}};
\node[Square] at (-6.800000, 9.250000) {};
\node[Onyomi] at (-6.750000, 9.350000) {\hbox{\tate テツ}};
\node[Kunyomi] at (-6.850000, 9.350000) {\hbox{\tate てっ.する}};
\node[Meaning] at (-6.800000, 11.000000) {penetrate};
\node[Kanji] at (-4.750000, 9.750000) {\textcolor[HTML]{d2a293}{徴}};
\node[Square] at (-4.750000, 9.250000) {};
\node[Onyomi] at (-4.700000, 9.350000) {\hbox{\tate チョウ}};
\node[Meaning] at (-4.750000, 11.000000) {indication};
\node[Kanji] at (-2.700000, 9.750000) {\textcolor[HTML]{68a4bc}{懲}};
\node[Square] at (-2.700000, 9.250000) {};
\node[Onyomi] at (-2.650000, 9.350000) {\hbox{\tate チョウ}};
\node[Kunyomi] at (-2.750000, 9.350000) {\hbox{\tate こ.りる}};
\node[Meaning] at (-2.700000, 11.000000) {chastise};
\node[Kanji] at (-0.650000, 9.750000) {\textcolor[HTML]{b0b0b5}{微}};
\node[Square] at (-0.650000, 9.250000) {};
\node[Onyomi] at (-0.600000, 9.350000) {\hbox{\tate ビ}};
\node[Kunyomi] at (-0.700000, 9.350000) {\hbox{\tate かす.か}};
\node[Meaning] at (-0.650000, 11.000000) {delicate};
\node[Kanji] at (1.400000, 9.750000) {\textcolor[HTML]{d2a293}{街}};
\node[Square] at (1.400000, 9.250000) {};
\node[Onyomi] at (1.450000, 9.350000) {\hbox{\tate ガイ・カイ}};
\node[Kunyomi] at (1.350000, 9.350000) {\hbox{\tate まち}};
\node[Meaning] at (1.400000, 11.000000) {street};
\node[Kanji] at (3.450000, 9.750000) {\textcolor[HTML]{91b7c3}{衡}};
\node[Square] at (3.450000, 9.250000) {};
\node[Onyomi] at (3.500000, 9.350000) {\hbox{\tate コウ}};
\node[Meaning] at (3.450000, 11.000000) {equilibrium};
\node[Kanji] at (5.500000, 9.750000) {\textcolor[HTML]{c8a59d}{稿}};
\node[Square] at (5.500000, 9.250000) {};
\node[Onyomi] at (5.550000, 9.350000) {\hbox{\tate コウ}};
\node[Kunyomi] at (5.450000, 9.350000) {\hbox{\tate したがき}};
\node[Meaning] at (5.500000, 11.000000) {draft};
\node[Kanji] at (7.550000, 9.750000) {\textcolor[HTML]{91b7c3}{稼}};
\node[Square] at (7.550000, 9.250000) {};
\node[Onyomi] at (7.600000, 9.350000) {\hbox{\tate カ}};
\node[Kunyomi] at (7.500000, 9.350000) {\hbox{\tate かせ.ぐ}};
\node[Meaning] at (7.550000, 11.000000) {earnings};
\node[Kanji] at (9.600000, 9.750000) {\textcolor[HTML]{d69f8d}{程}};
\node[Square] at (9.600000, 9.250000) {};
\node[Onyomi] at (9.650000, 9.350000) {\hbox{\tate テイ}};
\node[Kunyomi] at (9.550000, 9.350000) {\hbox{\tate ほど}};
\node[Meaning] at (9.600000, 11.000000) {extent};
\node[Kanji] at (11.650000, 9.750000) {\textcolor[HTML]{c8a59d}{税}};
\node[Square] at (11.650000, 9.250000) {};
\node[Onyomi] at (11.700000, 9.350000) {\hbox{\tate ゼイ}};
\node[Meaning] at (11.650000, 11.000000) {tax};
\node[Kanji] at (13.700000, 9.750000) {\textcolor[HTML]{91b7c3}{稚}};
\node[Square] at (13.700000, 9.250000) {};
\node[Onyomi] at (13.750000, 9.350000) {\hbox{\tate チ}};
\node[Meaning] at (13.700000, 11.000000) {immature};
\node[Kanji] at (15.750000, 9.750000) {\textcolor[HTML]{c36143}{和}};
\node[Square] at (15.750000, 9.250000) {};
\node[Onyomi] at (15.800000, 9.350000) {\hbox{\tate ワ・オ}};
\node[Kunyomi] at (15.700000, 9.350000) {\hbox{\tate なご}};
\node[Meaning] at (15.750000, 11.000000) {peace};
\node[Kanji] at (17.800000, 9.750000) {\textcolor[HTML]{cd8268}{移}};
\node[Square] at (17.800000, 9.250000) {};
\node[Onyomi] at (17.850000, 9.350000) {\hbox{\tate イ}};
\node[Kunyomi] at (17.750000, 9.350000) {\hbox{\tate うつ.*}};
\node[Meaning] at (17.800000, 11.000000) {shift};
\node[Kanji] at (19.850000, 9.750000) {\textcolor[HTML]{b0b0b5}{秒}};
\node[Square] at (19.850000, 9.250000) {};
\node[Onyomi] at (19.900000, 9.350000) {\hbox{\tate ビョウ}};
\node[Meaning] at (19.850000, 11.000000) {second};
\node[Kanji] at (21.900000, 9.750000) {\textcolor[HTML]{d2a293}{秋}};
\node[Square] at (21.900000, 9.250000) {};
\node[Kunyomi] at (21.850000, 9.350000) {\hbox{\tate あき}};
\node[Meaning] at (21.900000, 11.000000) {autumn};
\node[Kanji] at (23.950000, 9.750000) {\textcolor[HTML]{1059be}{愁}};
\node[Square] at (23.950000, 9.250000) {};
\node[Onyomi] at (24.000000, 9.350000) {\hbox{\tate シュウ}};
\node[Kunyomi] at (23.900000, 9.350000) {\hbox{\tate うれ.える}};
\node[Meaning] at (23.950000, 11.000000) {distress};
\node[Kanji] at (26.000000, 9.750000) {\textcolor[HTML]{d2a293}{私}};
\node[Square] at (26.000000, 9.250000) {};
\node[Onyomi] at (26.050000, 9.350000) {\hbox{\tate シ}};
\node[Kunyomi] at (25.950000, 9.350000) {\hbox{\tate わたし}};
\node[Meaning] at (26.000000, 11.000000) {i};
\node[Kanji] at (28.050000, 9.750000) {\textcolor[HTML]{91b7c3}{秩}};
\node[Square] at (28.050000, 9.250000) {};
\node[Onyomi] at (28.100000, 9.350000) {\hbox{\tate チツ}};
\node[Meaning] at (28.050000, 11.000000) {order};
\node[Kanji] at (30.100000, 9.750000) {\textcolor[HTML]{b0b0b5}{秘}};
\node[Square] at (30.100000, 9.250000) {};
\node[Onyomi] at (30.150000, 9.350000) {\hbox{\tate ヒ}};
\node[Kunyomi] at (30.050000, 9.350000) {\hbox{\tate ひ.める}};
\node[Meaning] at (30.100000, 11.000000) {secret};
\node[Kanji] at (32.150000, 9.750000) {\textcolor[HTML]{cd8268}{称}};
\node[Square] at (32.150000, 9.250000) {};
\node[Onyomi] at (32.200000, 9.350000) {\hbox{\tate ショウ}};
\node[Kunyomi] at (32.100000, 9.350000) {\hbox{\tate とな.える}};
\node[Meaning] at (32.150000, 11.000000) {title};
\node[Kanji] at (34.200000, 9.750000) {\textcolor[HTML]{c36143}{利}};
\node[Square] at (34.200000, 9.250000) {};
\node[Onyomi] at (34.250000, 9.350000) {\hbox{\tate リ}};
\node[Kunyomi] at (34.150000, 9.350000) {\hbox{\tate き.く}};
\node[Meaning] at (34.200000, 11.000000) {profit};
\node[Kanji] at (36.250000, 9.750000) {\textcolor[HTML]{a3bac2}{梨}};
\node[Square] at (36.250000, 9.250000) {};
\node[Kunyomi] at (36.200000, 9.350000) {\hbox{\tate なし}};
\node[Meaning] at (36.250000, 11.000000) {pear};
\node[Kanji] at (38.300000, 9.750000) {\textcolor[HTML]{408dba}{穫}};
\node[Square] at (38.300000, 9.250000) {};
\node[Onyomi] at (38.350000, 9.350000) {\hbox{\tate カク}};
\node[Meaning] at (38.300000, 11.000000) {harvest};
\node[Kanji] at (40.350000, 9.750000) {\textcolor[HTML]{a3bac2}{穂}};
\node[Square] at (40.350000, 9.250000) {};
\node[Onyomi] at (40.400000, 9.350000) {\hbox{\tate スイ}};
\node[Kunyomi] at (40.300000, 9.350000) {\hbox{\tate ほ}};
\node[Meaning] at (40.350000, 11.000000) {head of plant};
\node[Kanji] at (42.400000, 9.750000) {\textcolor[HTML]{b0b0b5}{稲}};
\node[Square] at (42.400000, 9.250000) {};
\node[Kunyomi] at (42.350000, 9.350000) {\hbox{\tate いね・いな}};
\node[Meaning] at (42.400000, 11.000000) {rice plant};
\node[Kanji] at (44.450000, 9.750000) {\textcolor[HTML]{c8a59d}{香}};
\node[Square] at (44.450000, 9.250000) {};
\node[Onyomi] at (44.500000, 9.350000) {\hbox{\tate コウ・キョウ}};
\node[Kunyomi] at (44.400000, 9.350000) {\hbox{\tate かお・か}};
\node[Meaning] at (44.450000, 11.000000) {fragrance};
\node[Kanji] at (46.500000, 9.750000) {\textcolor[HTML]{b0b0b5}{季}};
\node[Square] at (46.500000, 9.250000) {};
\node[Onyomi] at (46.550000, 9.350000) {\hbox{\tate キ}};
\node[Meaning] at (46.500000, 11.000000) {seasons};
\node[Kanji] at (48.550000, 9.750000) {\textcolor[HTML]{d69f8d}{委}};
\node[Square] at (48.550000, 9.250000) {};
\node[Onyomi] at (48.600000, 9.350000) {\hbox{\tate イ}};
\node[Meaning] at (48.550000, 11.000000) {committee};
\node[Kanji] at (50.600000, 9.750000) {\textcolor[HTML]{d2a293}{秀}};
\node[Square] at (50.600000, 9.250000) {};
\node[Onyomi] at (50.650000, 9.350000) {\hbox{\tate シュウ}};
\node[Kunyomi] at (50.550000, 9.350000) {\hbox{\tate ひい.でる}};
\node[Meaning] at (50.600000, 11.000000) {excel};
\node[Kanji] at (52.650000, 9.750000) {\textcolor[HTML]{a3bac2}{透}};
\node[Square] at (52.650000, 9.250000) {};
\node[Onyomi] at (52.700000, 9.350000) {\hbox{\tate トウ}};
\node[Kunyomi] at (52.600000, 9.350000) {\hbox{\tate す.ける}};
\node[Meaning] at (52.650000, 11.000000) {transparent};
\node[Kanji] at (54.700000, 9.750000) {\textcolor[HTML]{b0b0b5}{誘}};
\node[Square] at (54.700000, 9.250000) {};
\node[Onyomi] at (54.750000, 9.350000) {\hbox{\tate ユウ}};
\node[Kunyomi] at (54.650000, 9.350000) {\hbox{\tate さそ.う}};
\node[Meaning] at (54.700000, 11.000000) {invite};
\node[Kanji] at (56.750000, 9.750000) {\textcolor[HTML]{68a4bc}{穀}};
\node[Square] at (56.750000, 9.250000) {};
\node[Onyomi] at (56.800000, 9.350000) {\hbox{\tate コク}};
\node[Meaning] at (56.750000, 11.000000) {grain};
\node[Meaning] at (-58.050000, 9.850000) {51.89\%};
\node[Kanji] at (-56.000000, 7.700000) {\textcolor[HTML]{a3bac2}{菌}};
\node[Square] at (-56.000000, 7.200000) {};
\node[Onyomi] at (-55.950000, 7.300000) {\hbox{\tate キン}};
\node[Meaning] at (-56.000000, 8.950000) {bacteria};
\node[Kanji] at (-53.950000, 7.700000) {\textcolor[HTML]{d69f8d}{米}};
\node[Square] at (-53.950000, 7.200000) {};
\node[Onyomi] at (-53.900000, 7.300000) {\hbox{\tate ベイ・マイ}};
\node[Kunyomi] at (-54.000000, 7.300000) {\hbox{\tate こめ}};
\node[Meaning] at (-53.950000, 8.950000) {rice};
\node[Kanji] at (-51.900000, 7.700000) {\textcolor[HTML]{a3bac2}{粉}};
\node[Square] at (-51.900000, 7.200000) {};
\node[Onyomi] at (-51.850000, 7.300000) {\hbox{\tate フン}};
\node[Kunyomi] at (-51.950000, 7.300000) {\hbox{\tate こな・こ}};
\node[Meaning] at (-51.900000, 8.950000) {powder};
\node[Kanji] at (-49.850000, 7.700000) {\textcolor[HTML]{91b7c3}{粘}};
\node[Square] at (-49.850000, 7.200000) {};
\node[Onyomi] at (-49.800000, 7.300000) {\hbox{\tate ネン}};
\node[Kunyomi] at (-49.900000, 7.300000) {\hbox{\tate ねば.る}};
\node[Meaning] at (-49.850000, 8.950000) {sticky};
\node[Kanji] at (-47.800000, 7.700000) {\textcolor[HTML]{a3bac2}{粒}};
\node[Square] at (-47.800000, 7.200000) {};
\node[Onyomi] at (-47.750000, 7.300000) {\hbox{\tate リュウ}};
\node[Kunyomi] at (-47.850000, 7.300000) {\hbox{\tate つぶ}};
\node[Meaning] at (-47.800000, 8.950000) {grains};
\node[Kanji] at (-45.750000, 7.700000) {\textcolor[HTML]{68a4bc}{粧}};
\node[Square] at (-45.750000, 7.200000) {};
\node[Onyomi] at (-45.700000, 7.300000) {\hbox{\tate ショウ}};
\node[Meaning] at (-45.750000, 8.950000) {cosmetics};
\node[Kanji] at (-43.700000, 7.700000) {\textcolor[HTML]{a3bac2}{迷}};
\node[Square] at (-43.700000, 7.200000) {};
\node[Onyomi] at (-43.650000, 7.300000) {\hbox{\tate メイ}};
\node[Kunyomi] at (-43.750000, 7.300000) {\hbox{\tate まよ.う}};
\node[Meaning] at (-43.700000, 8.950000) {astray};
\node[Kanji] at (-41.650000, 7.700000) {\textcolor[HTML]{91b7c3}{粋}};
\node[Square] at (-41.650000, 7.200000) {};
\node[Onyomi] at (-41.600000, 7.300000) {\hbox{\tate スイ}};
\node[Kunyomi] at (-41.700000, 7.300000) {\hbox{\tate いき}};
\node[Meaning] at (-41.650000, 8.950000) {stylish};
\node[Kanji] at (-39.600000, 7.700000) {\textcolor[HTML]{91b7c3}{糧}};
\node[Square] at (-39.600000, 7.200000) {};
\node[Onyomi] at (-39.550000, 7.300000) {\hbox{\tate リョウ・ロウ}};
\node[Kunyomi] at (-39.650000, 7.300000) {\hbox{\tate かて}};
\node[Meaning] at (-39.600000, 8.950000) {provisions};
\node[Kanji] at (-37.550000, 7.700000) {\textcolor[HTML]{a3bac2}{菊}};
\node[Square] at (-37.550000, 7.200000) {};
\node[Onyomi] at (-37.500000, 7.300000) {\hbox{\tate キク}};
\node[Meaning] at (-37.550000, 8.950000) {chrysanthemum};
\node[Kanji] at (-35.500000, 7.700000) {\textcolor[HTML]{c8a59d}{奥}};
\node[Square] at (-35.500000, 7.200000) {};
\node[Onyomi] at (-35.450000, 7.300000) {\hbox{\tate オウ}};
\node[Kunyomi] at (-35.550000, 7.300000) {\hbox{\tate おく}};
\node[Meaning] at (-35.500000, 8.950000) {interior};
\node[Kanji] at (-33.450000, 7.700000) {\textcolor[HTML]{b74029}{数}};
\node[Square] at (-33.450000, 7.200000) {};
\node[Onyomi] at (-33.400000, 7.300000) {\hbox{\tate スウ}};
\node[Kunyomi] at (-33.500000, 7.300000) {\hbox{\tate かぞ.える}};
\node[Meaning] at (-33.450000, 8.950000) {count};
\node[Kanji] at (-31.400000, 7.700000) {\textcolor[HTML]{68a4bc}{楼}};
\node[Square] at (-31.400000, 7.200000) {};
\node[Onyomi] at (-31.350000, 7.300000) {\hbox{\tate ロウ}};
\node[Meaning] at (-31.400000, 8.950000) {watchtower};
\node[Kanji] at (-29.350000, 7.700000) {\textcolor[HTML]{cd8268}{類}};
\node[Square] at (-29.350000, 7.200000) {};
\node[Onyomi] at (-29.300000, 7.300000) {\hbox{\tate ルイ}};
\node[Kunyomi] at (-29.400000, 7.300000) {\hbox{\tate たぐ.い}};
\node[Meaning] at (-29.350000, 8.950000) {type};
\node[Kanji] at (-27.300000, 7.700000) {\textcolor[HTML]{408dba}{漆}};
\node[Square] at (-27.300000, 7.200000) {};
\node[Onyomi] at (-27.250000, 7.300000) {\hbox{\tate シツ}};
\node[Kunyomi] at (-27.350000, 7.300000) {\hbox{\tate うるし}};
\node[Meaning] at (-27.300000, 8.950000) {lacquer};
\node[Kanji] at (-25.250000, 7.700000) {\textcolor[HTML]{cd8268}{様}};
\node[Square] at (-25.250000, 7.200000) {};
\node[Onyomi] at (-25.200000, 7.300000) {\hbox{\tate ヨウ}};
\node[Kunyomi] at (-25.300000, 7.300000) {\hbox{\tate さま}};
\node[Meaning] at (-25.250000, 8.950000) {Mr., Mrs.};
\node[Kanji] at (-23.200000, 7.700000) {\textcolor[HTML]{d2a293}{求}};
\node[Square] at (-23.200000, 7.200000) {};
\node[Onyomi] at (-23.150000, 7.300000) {\hbox{\tate キュウ}};
\node[Kunyomi] at (-23.250000, 7.300000) {\hbox{\tate もと.める}};
\node[Meaning] at (-23.200000, 8.950000) {request};
\node[Kanji] at (-21.150000, 7.700000) {\textcolor[HTML]{cd8268}{球}};
\node[Square] at (-21.150000, 7.200000) {};
\node[Onyomi] at (-21.100000, 7.300000) {\hbox{\tate キュウ}};
\node[Kunyomi] at (-21.200000, 7.300000) {\hbox{\tate たま}};
\node[Meaning] at (-21.150000, 8.950000) {sphere};
\node[Kanji] at (-19.100000, 7.700000) {\textcolor[HTML]{c8a59d}{救}};
\node[Square] at (-19.100000, 7.200000) {};
\node[Onyomi] at (-19.050000, 7.300000) {\hbox{\tate キュウ}};
\node[Kunyomi] at (-19.150000, 7.300000) {\hbox{\tate すく.う}};
\node[Meaning] at (-19.100000, 8.950000) {rescue};
\node[Kanji] at (-17.050000, 7.700000) {\textcolor[HTML]{c8a59d}{竹}};
\node[Square] at (-17.050000, 7.200000) {};
\node[Kunyomi] at (-17.100000, 7.300000) {\hbox{\tate たけ}};
\node[Meaning] at (-17.050000, 8.950000) {bamboo};
\node[Kanji] at (-15.000000, 7.700000) {\textcolor[HTML]{b0b0b5}{笑}};
\node[Square] at (-15.000000, 7.200000) {};
\node[Onyomi] at (-14.950000, 7.300000) {\hbox{\tate ショウ}};
\node[Kunyomi] at (-15.050000, 7.300000) {\hbox{\tate わら・え}};
\node[Meaning] at (-15.000000, 8.950000) {laugh};
\node[Kanji] at (-12.950000, 7.700000) {\textcolor[HTML]{a3bac2}{笠}};
\node[Square] at (-12.950000, 7.200000) {};
\node[Kunyomi] at (-13.000000, 7.300000) {\hbox{\tate かさ}};
\node[Meaning] at (-12.950000, 8.950000) {conical hat};
\node[Kanji] at (-10.900000, 7.700000) {\textcolor[HTML]{b0b0b5}{筋}};
\node[Square] at (-10.900000, 7.200000) {};
\node[Onyomi] at (-10.850000, 7.300000) {\hbox{\tate キン}};
\node[Kunyomi] at (-10.950000, 7.300000) {\hbox{\tate すじ}};
\node[Meaning] at (-10.900000, 8.950000) {muscle};
\node[Kanji] at (-8.850000, 7.700000) {\textcolor[HTML]{a3bac2}{箱}};
\node[Square] at (-8.850000, 7.200000) {};
\node[Kunyomi] at (-8.900000, 7.300000) {\hbox{\tate はこ}};
\node[Meaning] at (-8.850000, 8.950000) {box};
\node[Kanji] at (-6.800000, 7.700000) {\textcolor[HTML]{d2a293}{筆}};
\node[Square] at (-6.800000, 7.200000) {};
\node[Onyomi] at (-6.750000, 7.300000) {\hbox{\tate ヒツ}};
\node[Kunyomi] at (-6.850000, 7.300000) {\hbox{\tate ふで}};
\node[Meaning] at (-6.800000, 8.950000) {writing brush};
\node[Kanji] at (-4.750000, 7.700000) {\textcolor[HTML]{a3bac2}{筒}};
\node[Square] at (-4.750000, 7.200000) {};
\node[Onyomi] at (-4.700000, 7.300000) {\hbox{\tate トウ}};
\node[Kunyomi] at (-4.800000, 7.300000) {\hbox{\tate つつ}};
\node[Meaning] at (-4.750000, 8.950000) {cylinder};
\node[Kanji] at (-2.700000, 7.700000) {\textcolor[HTML]{c36143}{等}};
\node[Square] at (-2.700000, 7.200000) {};
\node[Onyomi] at (-2.650000, 7.300000) {\hbox{\tate トウ}};
\node[Kunyomi] at (-2.750000, 7.300000) {\hbox{\tate ひと.しい}};
\node[Meaning] at (-2.700000, 8.950000) {equal};
\node[Kanji] at (-0.650000, 7.700000) {\textcolor[HTML]{d2a293}{算}};
\node[Square] at (-0.650000, 7.200000) {};
\node[Onyomi] at (-0.600000, 7.300000) {\hbox{\tate サン}};
\node[Kunyomi] at (-0.700000, 7.300000) {\hbox{\tate そろ}};
\node[Meaning] at (-0.650000, 8.950000) {calculate};
\node[Kanji] at (1.400000, 7.700000) {\textcolor[HTML]{b0b0b5}{答}};
\node[Square] at (1.400000, 7.200000) {};
\node[Onyomi] at (1.450000, 7.300000) {\hbox{\tate トウ}};
\node[Kunyomi] at (1.350000, 7.300000) {\hbox{\tate こた}};
\node[Meaning] at (1.400000, 8.950000) {answer};
\node[Kanji] at (3.450000, 7.700000) {\textcolor[HTML]{d2a293}{策}};
\node[Square] at (3.450000, 7.200000) {};
\node[Onyomi] at (3.500000, 7.300000) {\hbox{\tate サク}};
\node[Kunyomi] at (3.400000, 7.300000) {\hbox{\tate さく}};
\node[Meaning] at (3.450000, 8.950000) {plan};
\node[Kanji] at (5.500000, 7.700000) {\textcolor[HTML]{68a4bc}{簿}};
\node[Square] at (5.500000, 7.200000) {};
\node[Onyomi] at (5.550000, 7.300000) {\hbox{\tate ボ}};
\node[Meaning] at (5.500000, 8.950000) {record book};
\node[Kanji] at (7.550000, 7.700000) {\textcolor[HTML]{d2a293}{築}};
\node[Square] at (7.550000, 7.200000) {};
\node[Onyomi] at (7.600000, 7.300000) {\hbox{\tate チク}};
\node[Kunyomi] at (7.500000, 7.300000) {\hbox{\tate きず.く}};
\node[Meaning] at (7.550000, 8.950000) {construct};
\node[Kanji] at (9.600000, 7.700000) {\textcolor[HTML]{830e29}{人}};
\node[Square] at (9.600000, 7.200000) {};
\node[Onyomi] at (9.650000, 7.300000) {\hbox{\tate ニン・ジン}};
\node[Kunyomi] at (9.550000, 7.300000) {\hbox{\tate ひと}};
\node[Meaning] at (9.600000, 8.950000) {person};
\node[Kanji] at (11.650000, 7.700000) {\textcolor[HTML]{d69f8d}{佐}};
\node[Square] at (11.650000, 7.200000) {};
\node[Onyomi] at (11.700000, 7.300000) {\hbox{\tate サ}};
\node[Meaning] at (11.650000, 8.950000) {help};
\node[Kanji] at (13.700000, 7.700000) {\textcolor[HTML]{91b7c3}{但}};
\node[Square] at (13.700000, 7.200000) {};
\node[Kunyomi] at (13.650000, 7.300000) {\hbox{\tate ただ.し}};
\node[Meaning] at (13.700000, 8.950000) {however};
\node[Kanji] at (15.750000, 7.700000) {\textcolor[HTML]{cd8268}{住}};
\node[Square] at (15.750000, 7.200000) {};
\node[Onyomi] at (15.800000, 7.300000) {\hbox{\tate ジュウ}};
\node[Kunyomi] at (15.700000, 7.300000) {\hbox{\tate す.む}};
\node[Meaning] at (15.750000, 8.950000) {dwelling};
\node[Kanji] at (17.800000, 7.700000) {\textcolor[HTML]{c36143}{位}};
\node[Square] at (17.800000, 7.200000) {};
\node[Onyomi] at (17.850000, 7.300000) {\hbox{\tate イ}};
\node[Kunyomi] at (17.750000, 7.300000) {\hbox{\tate くらい}};
\node[Meaning] at (17.800000, 8.950000) {rank};
\node[Kanji] at (19.850000, 7.700000) {\textcolor[HTML]{c8a59d}{仲}};
\node[Square] at (19.850000, 7.200000) {};
\node[Onyomi] at (19.900000, 7.300000) {\hbox{\tate チュウ}};
\node[Kunyomi] at (19.800000, 7.300000) {\hbox{\tate なか}};
\node[Meaning] at (19.850000, 8.950000) {relationship};
\node[Kanji] at (21.900000, 7.700000) {\textcolor[HTML]{b74029}{体}};
\node[Square] at (21.900000, 7.200000) {};
\node[Onyomi] at (21.950000, 7.300000) {\hbox{\tate タイ}};
\node[Kunyomi] at (21.850000, 7.300000) {\hbox{\tate からだ}};
\node[Meaning] at (21.900000, 8.950000) {body};
\node[Kanji] at (23.950000, 7.700000) {\textcolor[HTML]{408dba}{悠}};
\node[Square] at (23.950000, 7.200000) {};
\node[Onyomi] at (24.000000, 7.300000) {\hbox{\tate ユウ}};
\node[Meaning] at (23.950000, 8.950000) {leisure};
\node[Kanji] at (26.000000, 7.700000) {\textcolor[HTML]{d69f8d}{件}};
\node[Square] at (26.000000, 7.200000) {};
\node[Onyomi] at (26.050000, 7.300000) {\hbox{\tate ケン}};
\node[Meaning] at (26.000000, 8.950000) {matter};
\node[Kanji] at (28.050000, 7.700000) {\textcolor[HTML]{d69f8d}{仕}};
\node[Square] at (28.050000, 7.200000) {};
\node[Onyomi] at (28.100000, 7.300000) {\hbox{\tate シ}};
\node[Kunyomi] at (28.000000, 7.300000) {\hbox{\tate つか.える}};
\node[Meaning] at (28.050000, 8.950000) {doing};
\node[Kanji] at (30.100000, 7.700000) {\textcolor[HTML]{cd8268}{他}};
\node[Square] at (30.100000, 7.200000) {};
\node[Onyomi] at (30.150000, 7.300000) {\hbox{\tate タ}};
\node[Kunyomi] at (30.050000, 7.300000) {\hbox{\tate ほか}};
\node[Meaning] at (30.100000, 8.950000) {other};
\node[Kanji] at (32.150000, 7.700000) {\textcolor[HTML]{b0b0b5}{伏}};
\node[Square] at (32.150000, 7.200000) {};
\node[Onyomi] at (32.200000, 7.300000) {\hbox{\tate フク}};
\node[Kunyomi] at (32.100000, 7.300000) {\hbox{\tate ふ}};
\node[Meaning] at (32.150000, 8.950000) {bow};
\node[Kanji] at (34.200000, 7.700000) {\textcolor[HTML]{cd8268}{伝}};
\node[Square] at (34.200000, 7.200000) {};
\node[Onyomi] at (34.250000, 7.300000) {\hbox{\tate デン}};
\node[Kunyomi] at (34.150000, 7.300000) {\hbox{\tate つた・つて}};
\node[Meaning] at (34.200000, 8.950000) {transmit};
\node[Kanji] at (36.250000, 7.700000) {\textcolor[HTML]{c8a59d}{仏}};
\node[Square] at (36.250000, 7.200000) {};
\node[Onyomi] at (36.300000, 7.300000) {\hbox{\tate ブツ}};
\node[Kunyomi] at (36.200000, 7.300000) {\hbox{\tate ほとけ}};
\node[Meaning] at (36.250000, 8.950000) {buddha};
\node[Kanji] at (38.300000, 7.700000) {\textcolor[HTML]{c8a59d}{休}};
\node[Square] at (38.300000, 7.200000) {};
\node[Onyomi] at (38.350000, 7.300000) {\hbox{\tate キュウ}};
\node[Kunyomi] at (38.250000, 7.300000) {\hbox{\tate やす.み}};
\node[Meaning] at (38.300000, 8.950000) {rest};
\node[Kanji] at (40.350000, 7.700000) {\textcolor[HTML]{d69f8d}{仮}};
\node[Square] at (40.350000, 7.200000) {};
\node[Onyomi] at (40.400000, 7.300000) {\hbox{\tate カ}};
\node[Kunyomi] at (40.300000, 7.300000) {\hbox{\tate かり}};
\node[Meaning] at (40.350000, 8.950000) {temporary};
\node[Kanji] at (42.400000, 7.700000) {\textcolor[HTML]{b0b0b5}{伯}};
\node[Square] at (42.400000, 7.200000) {};
\node[Onyomi] at (42.450000, 7.300000) {\hbox{\tate ハク・オ}};
\node[Meaning] at (42.400000, 8.950000) {chief};
\node[Kanji] at (44.450000, 7.700000) {\textcolor[HTML]{a3bac2}{俗}};
\node[Square] at (44.450000, 7.200000) {};
\node[Onyomi] at (44.500000, 7.300000) {\hbox{\tate ゾク}};
\node[Meaning] at (44.450000, 8.950000) {vulgar};
\node[Kanji] at (46.500000, 7.700000) {\textcolor[HTML]{cd8268}{信}};
\node[Square] at (46.500000, 7.200000) {};
\node[Onyomi] at (46.550000, 7.300000) {\hbox{\tate シン}};
\node[Kunyomi] at (46.450000, 7.300000) {\hbox{\tate しん}};
\node[Meaning] at (46.500000, 8.950000) {believe};
\node[Kanji] at (48.550000, 7.700000) {\textcolor[HTML]{68a4bc}{佳}};
\node[Square] at (48.550000, 7.200000) {};
\node[Onyomi] at (48.600000, 7.300000) {\hbox{\tate カ}};
\node[Meaning] at (48.550000, 8.950000) {excellent};
\node[Kanji] at (50.600000, 7.700000) {\textcolor[HTML]{a11d25}{依}};
\node[Square] at (50.600000, 7.200000) {};
\node[Onyomi] at (50.650000, 7.300000) {\hbox{\tate イ}};
\node[Kunyomi] at (50.550000, 7.300000) {\hbox{\tate よ.る}};
\node[Meaning] at (50.600000, 8.950000) {reliant};
\node[Kanji] at (52.650000, 7.700000) {\textcolor[HTML]{cd8268}{例}};
\node[Square] at (52.650000, 7.200000) {};
\node[Onyomi] at (52.700000, 7.300000) {\hbox{\tate レイ}};
\node[Kunyomi] at (52.600000, 7.300000) {\hbox{\tate たと}};
\node[Meaning] at (52.650000, 8.950000) {example};
\node[Kanji] at (54.700000, 7.700000) {\textcolor[HTML]{d69f8d}{個}};
\node[Square] at (54.700000, 7.200000) {};
\node[Onyomi] at (54.750000, 7.300000) {\hbox{\tate コ}};
\node[Meaning] at (54.700000, 8.950000) {individual};
\node[Kanji] at (56.750000, 7.700000) {\textcolor[HTML]{c8a59d}{健}};
\node[Square] at (56.750000, 7.200000) {};
\node[Onyomi] at (56.800000, 7.300000) {\hbox{\tate ケン}};
\node[Meaning] at (56.750000, 8.950000) {healthy};
\node[Meaning] at (-58.050000, 7.800000) {55.45\%};
\node[Kanji] at (-56.000000, 5.650000) {\textcolor[HTML]{cd8268}{側}};
\node[Square] at (-56.000000, 5.150000) {};
\node[Onyomi] at (-55.950000, 5.250000) {\hbox{\tate ソク}};
\node[Kunyomi] at (-56.050000, 5.250000) {\hbox{\tate がわ・そば}};
\node[Meaning] at (-56.000000, 6.900000) {side};
\node[Kanji] at (-53.950000, 5.650000) {\textcolor[HTML]{91b7c3}{侍}};
\node[Square] at (-53.950000, 5.150000) {};
\node[Kunyomi] at (-54.000000, 5.250000) {\hbox{\tate さむらい}};
\node[Meaning] at (-53.950000, 6.900000) {samurai};
\node[Kanji] at (-51.900000, 5.650000) {\textcolor[HTML]{d2a293}{停}};
\node[Square] at (-51.900000, 5.150000) {};
\node[Onyomi] at (-51.850000, 5.250000) {\hbox{\tate テイ}};
\node[Meaning] at (-51.900000, 6.900000) {halt};
\node[Kanji] at (-49.850000, 5.650000) {\textcolor[HTML]{d2a293}{値}};
\node[Square] at (-49.850000, 5.150000) {};
\node[Onyomi] at (-49.800000, 5.250000) {\hbox{\tate チ}};
\node[Kunyomi] at (-49.900000, 5.250000) {\hbox{\tate ね・あたい}};
\node[Meaning] at (-49.850000, 6.900000) {value};
\node[Kanji] at (-47.800000, 5.650000) {\textcolor[HTML]{68a4bc}{倣}};
\node[Square] at (-47.800000, 5.150000) {};
\node[Onyomi] at (-47.750000, 5.250000) {\hbox{\tate ホウ}};
\node[Kunyomi] at (-47.850000, 5.250000) {\hbox{\tate なら.う}};
\node[Meaning] at (-47.800000, 6.900000) {emulate};
\node[Kanji] at (-45.750000, 5.650000) {\textcolor[HTML]{c8a59d}{倒}};
\node[Square] at (-45.750000, 5.150000) {};
\node[Onyomi] at (-45.700000, 5.250000) {\hbox{\tate トウ}};
\node[Kunyomi] at (-45.800000, 5.250000) {\hbox{\tate たお.す}};
\node[Meaning] at (-45.750000, 6.900000) {overthrow};
\node[Kanji] at (-43.700000, 5.650000) {\textcolor[HTML]{a3bac2}{偵}};
\node[Square] at (-43.700000, 5.150000) {};
\node[Onyomi] at (-43.650000, 5.250000) {\hbox{\tate テイ}};
\node[Meaning] at (-43.700000, 6.900000) {spy};
\node[Kanji] at (-41.650000, 5.650000) {\textcolor[HTML]{a3bac2}{僧}};
\node[Square] at (-41.650000, 5.150000) {};
\node[Onyomi] at (-41.600000, 5.250000) {\hbox{\tate ソウ}};
\node[Meaning] at (-41.650000, 6.900000) {priest};
\node[Kanji] at (-39.600000, 5.650000) {\textcolor[HTML]{b0b0b5}{億}};
\node[Square] at (-39.600000, 5.150000) {};
\node[Onyomi] at (-39.550000, 5.250000) {\hbox{\tate オク}};
\node[Meaning] at (-39.600000, 6.900000) {100 million};
\node[Kanji] at (-37.550000, 5.650000) {\textcolor[HTML]{b0b0b5}{儀}};
\node[Square] at (-37.550000, 5.150000) {};
\node[Onyomi] at (-37.500000, 5.250000) {\hbox{\tate ギ}};
\node[Meaning] at (-37.550000, 6.900000) {ceremony};
\node[Kanji] at (-35.500000, 5.650000) {\textcolor[HTML]{a3bac2}{償}};
\node[Square] at (-35.500000, 5.150000) {};
\node[Onyomi] at (-35.450000, 5.250000) {\hbox{\tate ショウ}};
\node[Kunyomi] at (-35.550000, 5.250000) {\hbox{\tate つぐな.う}};
\node[Meaning] at (-35.500000, 6.900000) {reparation};
\node[Kanji] at (-33.450000, 5.650000) {\textcolor[HTML]{c8a59d}{仙}};
\node[Square] at (-33.450000, 5.150000) {};
\node[Onyomi] at (-33.400000, 5.250000) {\hbox{\tate セン}};
\node[Meaning] at (-33.450000, 6.900000) {hermit};
\node[Kanji] at (-31.400000, 5.650000) {\textcolor[HTML]{d2a293}{催}};
\node[Square] at (-31.400000, 5.150000) {};
\node[Onyomi] at (-31.350000, 5.250000) {\hbox{\tate サイ}};
\node[Kunyomi] at (-31.450000, 5.250000) {\hbox{\tate もよお.す}};
\node[Meaning] at (-31.400000, 6.900000) {sponsor};
\node[Kanji] at (-29.350000, 5.650000) {\textcolor[HTML]{b0b0b5}{仁}};
\node[Square] at (-29.350000, 5.150000) {};
\node[Onyomi] at (-29.300000, 5.250000) {\hbox{\tate ジン}};
\node[Meaning] at (-29.350000, 6.900000) {humanity};
\node[Kanji] at (-27.300000, 5.650000) {\textcolor[HTML]{1e76bb}{侮}};
\node[Square] at (-27.300000, 5.150000) {};
\node[Onyomi] at (-27.250000, 5.250000) {\hbox{\tate ブ}};
\node[Kunyomi] at (-27.350000, 5.250000) {\hbox{\tate あなず}};
\node[Meaning] at (-27.300000, 6.900000) {despise};
\node[Kanji] at (-25.250000, 5.650000) {\textcolor[HTML]{c36143}{使}};
\node[Square] at (-25.250000, 5.150000) {};
\node[Onyomi] at (-25.200000, 5.250000) {\hbox{\tate シ}};
\node[Kunyomi] at (-25.300000, 5.250000) {\hbox{\tate つか.う}};
\node[Meaning] at (-25.250000, 6.900000) {use};
\node[Kanji] at (-23.200000, 5.650000) {\textcolor[HTML]{d2a293}{便}};
\node[Square] at (-23.200000, 5.150000) {};
\node[Onyomi] at (-23.150000, 5.250000) {\hbox{\tate ベン・ビン}};
\node[Kunyomi] at (-23.250000, 5.250000) {\hbox{\tate たよ.*}};
\node[Meaning] at (-23.200000, 6.900000) {convenience};
\node[Kanji] at (-21.150000, 5.650000) {\textcolor[HTML]{b0b0b5}{倍}};
\node[Square] at (-21.150000, 5.150000) {};
\node[Onyomi] at (-21.100000, 5.250000) {\hbox{\tate バイ}};
\node[Meaning] at (-21.150000, 6.900000) {double};
\node[Kanji] at (-19.100000, 5.650000) {\textcolor[HTML]{cd8268}{優}};
\node[Square] at (-19.100000, 5.150000) {};
\node[Onyomi] at (-19.050000, 5.250000) {\hbox{\tate ユウ}};
\node[Kunyomi] at (-19.150000, 5.250000) {\hbox{\tate やさ.しい}};
\node[Meaning] at (-19.100000, 6.900000) {superior};
\node[Kanji] at (-17.050000, 5.650000) {\textcolor[HTML]{91b7c3}{伐}};
\node[Square] at (-17.050000, 5.150000) {};
\node[Onyomi] at (-17.000000, 5.250000) {\hbox{\tate バツ}};
\node[Kunyomi] at (-17.100000, 5.250000) {\hbox{\tate う・き・そむ}};
\node[Meaning] at (-17.050000, 6.900000) {fell};
\node[Kanji] at (-15.000000, 5.650000) {\textcolor[HTML]{c8a59d}{宿}};
\node[Square] at (-15.000000, 5.150000) {};
\node[Onyomi] at (-14.950000, 5.250000) {\hbox{\tate シュク}};
\node[Kunyomi] at (-15.050000, 5.250000) {\hbox{\tate やど}};
\node[Meaning] at (-15.000000, 6.900000) {lodge};
\node[Kanji] at (-12.950000, 5.650000) {\textcolor[HTML]{c8a59d}{傷}};
\node[Square] at (-12.950000, 5.150000) {};
\node[Onyomi] at (-12.900000, 5.250000) {\hbox{\tate ショウ}};
\node[Kunyomi] at (-13.000000, 5.250000) {\hbox{\tate きず}};
\node[Meaning] at (-12.950000, 6.900000) {wound};
\node[Kanji] at (-10.900000, 5.650000) {\textcolor[HTML]{cd8268}{保}};
\node[Square] at (-10.900000, 5.150000) {};
\node[Onyomi] at (-10.850000, 5.250000) {\hbox{\tate ホ}};
\node[Kunyomi] at (-10.950000, 5.250000) {\hbox{\tate たも.つ}};
\node[Meaning] at (-10.900000, 6.900000) {preserve};
\node[Kanji] at (-8.850000, 5.650000) {\textcolor[HTML]{408dba}{褒}};
\node[Square] at (-8.850000, 5.150000) {};
\node[Onyomi] at (-8.800000, 5.250000) {\hbox{\tate ホウ}};
\node[Kunyomi] at (-8.900000, 5.250000) {\hbox{\tate ほ.める}};
\node[Meaning] at (-8.850000, 6.900000) {praise};
\node[Kanji] at (-6.800000, 5.650000) {\textcolor[HTML]{68a4bc}{傑}};
\node[Square] at (-6.800000, 5.150000) {};
\node[Onyomi] at (-6.750000, 5.250000) {\hbox{\tate ケツ}};
\node[Kunyomi] at (-6.850000, 5.250000) {\hbox{\tate すぐ}};
\node[Meaning] at (-6.800000, 6.900000) {greatness};
\node[Kanji] at (-4.750000, 5.650000) {\textcolor[HTML]{c36143}{付}};
\node[Square] at (-4.750000, 5.150000) {};
\node[Onyomi] at (-4.700000, 5.250000) {\hbox{\tate フ}};
\node[Kunyomi] at (-4.800000, 5.250000) {\hbox{\tate つ}};
\node[Meaning] at (-4.750000, 6.900000) {attach};
\node[Kanji] at (-2.700000, 5.650000) {\textcolor[HTML]{91b7c3}{符}};
\node[Square] at (-2.700000, 5.150000) {};
\node[Onyomi] at (-2.650000, 5.250000) {\hbox{\tate フ}};
\node[Meaning] at (-2.700000, 6.900000) {token};
\node[Kanji] at (-0.650000, 5.650000) {\textcolor[HTML]{cd8268}{府}};
\node[Square] at (-0.650000, 5.150000) {};
\node[Onyomi] at (-0.600000, 5.250000) {\hbox{\tate フ}};
\node[Meaning] at (-0.650000, 6.900000) {government};
\node[Kanji] at (1.400000, 5.650000) {\textcolor[HTML]{cd8268}{任}};
\node[Square] at (1.400000, 5.150000) {};
\node[Onyomi] at (1.450000, 5.250000) {\hbox{\tate ニン}};
\node[Kunyomi] at (1.350000, 5.250000) {\hbox{\tate まか.せる}};
\node[Meaning] at (1.400000, 6.900000) {duty};
\node[Kanji] at (3.450000, 5.650000) {\textcolor[HTML]{a3bac2}{賃}};
\node[Square] at (3.450000, 5.150000) {};
\node[Onyomi] at (3.500000, 5.250000) {\hbox{\tate チン}};
\node[Meaning] at (3.450000, 6.900000) {rent};
\node[Kanji] at (5.500000, 5.650000) {\textcolor[HTML]{a11d25}{代}};
\node[Square] at (5.500000, 5.150000) {};
\node[Onyomi] at (5.550000, 5.250000) {\hbox{\tate ダイ}};
\node[Kunyomi] at (5.450000, 5.250000) {\hbox{\tate か・かわ.る}};
\node[Meaning] at (5.500000, 6.900000) {substitute};
\node[Kanji] at (7.550000, 5.650000) {\textcolor[HTML]{a3bac2}{袋}};
\node[Square] at (7.550000, 5.150000) {};
\node[Onyomi] at (7.600000, 5.250000) {\hbox{\tate タイ}};
\node[Kunyomi] at (7.500000, 5.250000) {\hbox{\tate ふくろ}};
\node[Meaning] at (7.550000, 6.900000) {sack};
\node[Kanji] at (9.600000, 5.650000) {\textcolor[HTML]{a3bac2}{貸}};
\node[Square] at (9.600000, 5.150000) {};
\node[Onyomi] at (9.650000, 5.250000) {\hbox{\tate タイ}};
\node[Kunyomi] at (9.550000, 5.250000) {\hbox{\tate か}};
\node[Meaning] at (9.600000, 6.900000) {lend};
\node[Kanji] at (11.650000, 5.650000) {\textcolor[HTML]{b74029}{化}};
\node[Square] at (11.650000, 5.150000) {};
\node[Onyomi] at (11.700000, 5.250000) {\hbox{\tate カ}};
\node[Kunyomi] at (11.600000, 5.250000) {\hbox{\tate ば.ける}};
\node[Meaning] at (11.650000, 6.900000) {change};
\node[Kanji] at (13.700000, 5.650000) {\textcolor[HTML]{d69f8d}{花}};
\node[Square] at (13.700000, 5.150000) {};
\node[Onyomi] at (13.750000, 5.250000) {\hbox{\tate カ・ケ}};
\node[Kunyomi] at (13.650000, 5.250000) {\hbox{\tate はな}};
\node[Meaning] at (13.700000, 6.900000) {flower};
\node[Kanji] at (15.750000, 5.650000) {\textcolor[HTML]{c8a59d}{貨}};
\node[Square] at (15.750000, 5.150000) {};
\node[Onyomi] at (15.800000, 5.250000) {\hbox{\tate カ}};
\node[Meaning] at (15.750000, 6.900000) {freight};
\node[Kanji] at (17.800000, 5.650000) {\textcolor[HTML]{c8a59d}{傾}};
\node[Square] at (17.800000, 5.150000) {};
\node[Onyomi] at (17.850000, 5.250000) {\hbox{\tate ケイ}};
\node[Kunyomi] at (17.750000, 5.250000) {\hbox{\tate かたむ.*}};
\node[Meaning] at (17.800000, 6.900000) {lean};
\node[Kanji] at (19.850000, 5.650000) {\textcolor[HTML]{d2a293}{何}};
\node[Square] at (19.850000, 5.150000) {};
\node[Onyomi] at (19.900000, 5.250000) {\hbox{\tate カ}};
\node[Kunyomi] at (19.800000, 5.250000) {\hbox{\tate なに・なん}};
\node[Meaning] at (19.850000, 6.900000) {what};
\node[Kanji] at (21.900000, 5.650000) {\textcolor[HTML]{b0b0b5}{荷}};
\node[Square] at (21.900000, 5.150000) {};
\node[Onyomi] at (21.950000, 5.250000) {\hbox{\tate カ}};
\node[Kunyomi] at (21.850000, 5.250000) {\hbox{\tate に}};
\node[Meaning] at (21.900000, 6.900000) {luggage};
\node[Kanji] at (23.950000, 5.650000) {\textcolor[HTML]{b0b0b5}{俊}};
\node[Square] at (23.950000, 5.150000) {};
\node[Onyomi] at (24.000000, 5.250000) {\hbox{\tate シュン}};
\node[Meaning] at (23.950000, 6.900000) {genius};
\node[Kanji] at (26.000000, 5.650000) {\textcolor[HTML]{91b7c3}{傍}};
\node[Square] at (26.000000, 5.150000) {};
\node[Onyomi] at (26.050000, 5.250000) {\hbox{\tate ボウ}};
\node[Kunyomi] at (25.950000, 5.250000) {\hbox{\tate かたわ・わき}};
\node[Meaning] at (26.000000, 6.900000) {nearby};
\node[Kanji] at (28.050000, 5.650000) {\textcolor[HTML]{d2a293}{久}};
\node[Square] at (28.050000, 5.150000) {};
\node[Onyomi] at (28.100000, 5.250000) {\hbox{\tate キュウ}};
\node[Kunyomi] at (28.000000, 5.250000) {\hbox{\tate ひさ}};
\node[Meaning] at (28.050000, 6.900000) {long time};
\node[Kanji] at (30.100000, 5.650000) {\textcolor[HTML]{1e76bb}{畝}};
\node[Square] at (30.100000, 5.150000) {};
\node[Kunyomi] at (30.050000, 5.250000) {\hbox{\tate うね}};
\node[Meaning] at (30.100000, 6.900000) {furrow};
\node[Kanji] at (32.150000, 5.650000) {\textcolor[HTML]{68a4bc}{囚}};
\node[Square] at (32.150000, 5.150000) {};
\node[Onyomi] at (32.200000, 5.250000) {\hbox{\tate シュウ}};
\node[Kunyomi] at (32.100000, 5.250000) {\hbox{\tate とら}};
\node[Meaning] at (32.150000, 6.900000) {criminal};
\node[Kanji] at (34.200000, 5.650000) {\textcolor[HTML]{b74029}{内}};
\node[Square] at (34.200000, 5.150000) {};
\node[Onyomi] at (34.250000, 5.250000) {\hbox{\tate ナイ}};
\node[Kunyomi] at (34.150000, 5.250000) {\hbox{\tate うち}};
\node[Meaning] at (34.200000, 6.900000) {inside};
\node[Kanji] at (36.250000, 5.650000) {\textcolor[HTML]{1e76bb}{丙}};
\node[Square] at (36.250000, 5.150000) {};
\node[Onyomi] at (36.300000, 5.250000) {\hbox{\tate ヘイ}};
\node[Meaning] at (36.250000, 6.900000) {third class};
\node[Kanji] at (38.300000, 5.650000) {\textcolor[HTML]{b0b0b5}{柄}};
\node[Square] at (38.300000, 5.150000) {};
\node[Onyomi] at (38.350000, 5.250000) {\hbox{\tate ヘイ}};
\node[Kunyomi] at (38.250000, 5.250000) {\hbox{\tate がら}};
\node[Meaning] at (38.300000, 6.900000) {pattern};
\node[Kanji] at (40.350000, 5.650000) {\textcolor[HTML]{c8a59d}{肉}};
\node[Square] at (40.350000, 5.150000) {};
\node[Onyomi] at (40.400000, 5.250000) {\hbox{\tate ニク}};
\node[Meaning] at (40.350000, 6.900000) {meat};
\node[Kanji] at (42.400000, 5.650000) {\textcolor[HTML]{91b7c3}{腐}};
\node[Square] at (42.400000, 5.150000) {};
\node[Onyomi] at (42.450000, 5.250000) {\hbox{\tate フ}};
\node[Kunyomi] at (42.350000, 5.250000) {\hbox{\tate くさ.る}};
\node[Meaning] at (42.400000, 6.900000) {rot};
\node[Kanji] at (44.450000, 5.650000) {\textcolor[HTML]{d69f8d}{座}};
\node[Square] at (44.450000, 5.150000) {};
\node[Onyomi] at (44.500000, 5.250000) {\hbox{\tate ザ}};
\node[Kunyomi] at (44.400000, 5.250000) {\hbox{\tate すわ.る}};
\node[Meaning] at (44.450000, 6.900000) {sit};
\node[Kanji] at (46.500000, 5.650000) {\textcolor[HTML]{d2a293}{卒}};
\node[Square] at (46.500000, 5.150000) {};
\node[Onyomi] at (46.550000, 5.250000) {\hbox{\tate ソツ}};
\node[Meaning] at (46.500000, 6.900000) {graduate};
\node[Kanji] at (48.550000, 5.650000) {\textcolor[HTML]{91b7c3}{傘}};
\node[Square] at (48.550000, 5.150000) {};
\node[Onyomi] at (48.600000, 5.250000) {\hbox{\tate サン}};
\node[Kunyomi] at (48.500000, 5.250000) {\hbox{\tate かさ}};
\node[Meaning] at (48.550000, 6.900000) {umbrella};
\node[Kanji] at (50.600000, 5.650000) {\textcolor[HTML]{b74029}{以}};
\node[Square] at (50.600000, 5.150000) {};
\node[Onyomi] at (50.650000, 5.250000) {\hbox{\tate イ}};
\node[Meaning] at (50.600000, 6.900000) {by means of};
\node[Kanji] at (52.650000, 5.650000) {\textcolor[HTML]{c8a59d}{似}};
\node[Square] at (52.650000, 5.150000) {};
\node[Onyomi] at (52.700000, 5.250000) {\hbox{\tate ネ・ジ}};
\node[Kunyomi] at (52.600000, 5.250000) {\hbox{\tate に.る}};
\node[Meaning] at (52.650000, 6.900000) {resemble};
\node[Kanji] at (54.700000, 5.650000) {\textcolor[HTML]{d2a293}{併}};
\node[Square] at (54.700000, 5.150000) {};
\node[Onyomi] at (54.750000, 5.250000) {\hbox{\tate ヘイ}};
\node[Kunyomi] at (54.650000, 5.250000) {\hbox{\tate あわ.せる}};
\node[Meaning] at (54.700000, 6.900000) {join};
\node[Kanji] at (56.750000, 5.650000) {\textcolor[HTML]{91b7c3}{瓦}};
\node[Square] at (56.750000, 5.150000) {};
\node[Onyomi] at (56.800000, 5.250000) {\hbox{\tate ガ}};
\node[Kunyomi] at (56.700000, 5.250000) {\hbox{\tate かわら}};
\node[Meaning] at (56.750000, 6.900000) {tile};
\node[Meaning] at (-58.050000, 5.750000) {58.30\%};
\node[Kanji] at (-56.000000, 3.600000) {\textcolor[HTML]{68a4bc}{瓶}};
\node[Square] at (-56.000000, 3.100000) {};
\node[Onyomi] at (-55.950000, 3.200000) {\hbox{\tate ビン}};
\node[Kunyomi] at (-56.050000, 3.200000) {\hbox{\tate かめ}};
\node[Meaning] at (-56.000000, 4.850000) {bottle};
\node[Kanji] at (-53.950000, 3.600000) {\textcolor[HTML]{cd8268}{宮}};
\node[Square] at (-53.950000, 3.100000) {};
\node[Onyomi] at (-53.900000, 3.200000) {\hbox{\tate キュウ}};
\node[Kunyomi] at (-54.000000, 3.200000) {\hbox{\tate みや}};
\node[Meaning] at (-53.950000, 4.850000) {shinto shrine};
\node[Kanji] at (-51.900000, 3.600000) {\textcolor[HTML]{cd8268}{営}};
\node[Square] at (-51.900000, 3.100000) {};
\node[Onyomi] at (-51.850000, 3.200000) {\hbox{\tate エイ}};
\node[Kunyomi] at (-51.950000, 3.200000) {\hbox{\tate いとな.む}};
\node[Meaning] at (-51.900000, 4.850000) {manage};
\node[Kanji] at (-49.850000, 3.600000) {\textcolor[HTML]{c8a59d}{善}};
\node[Square] at (-49.850000, 3.100000) {};
\node[Onyomi] at (-49.800000, 3.200000) {\hbox{\tate ゼン}};
\node[Kunyomi] at (-49.900000, 3.200000) {\hbox{\tate ぜん}};
\node[Meaning] at (-49.850000, 4.850000) {morally good};
\node[Kanji] at (-47.800000, 3.600000) {\textcolor[HTML]{3c0912}{年}};
\node[Square] at (-47.800000, 3.100000) {};
\node[Onyomi] at (-47.750000, 3.200000) {\hbox{\tate ネン}};
\node[Kunyomi] at (-47.850000, 3.200000) {\hbox{\tate とし}};
\node[Meaning] at (-47.800000, 4.850000) {year};
\node[Kanji] at (-45.750000, 3.600000) {\textcolor[HTML]{d2a293}{夜}};
\node[Square] at (-45.750000, 3.100000) {};
\node[Onyomi] at (-45.700000, 3.200000) {\hbox{\tate ヤ}};
\node[Kunyomi] at (-45.800000, 3.200000) {\hbox{\tate よ・よる}};
\node[Meaning] at (-45.750000, 4.850000) {night};
\node[Kanji] at (-43.700000, 3.600000) {\textcolor[HTML]{b0b0b5}{液}};
\node[Square] at (-43.700000, 3.100000) {};
\node[Onyomi] at (-43.650000, 3.200000) {\hbox{\tate エキ}};
\node[Meaning] at (-43.700000, 4.850000) {fluid};
\node[Kanji] at (-41.650000, 3.600000) {\textcolor[HTML]{c8a59d}{塚}};
\node[Square] at (-41.650000, 3.100000) {};
\node[Onyomi] at (-41.600000, 3.200000) {\hbox{\tate チョウ}};
\node[Kunyomi] at (-41.700000, 3.200000) {\hbox{\tate つか}};
\node[Meaning] at (-41.650000, 4.850000) {mound};
\node[Kanji] at (-39.600000, 3.600000) {\textcolor[HTML]{a3bac2}{幣}};
\node[Square] at (-39.600000, 3.100000) {};
\node[Onyomi] at (-39.550000, 3.200000) {\hbox{\tate ヘイ}};
\node[Meaning] at (-39.600000, 4.850000) {cash};
\node[Kanji] at (-37.550000, 3.600000) {\textcolor[HTML]{408dba}{弊}};
\node[Square] at (-37.550000, 3.100000) {};
\node[Onyomi] at (-37.500000, 3.200000) {\hbox{\tate ヘイ}};
\node[Meaning] at (-37.550000, 4.850000) {evil};
\node[Kanji] at (-35.500000, 3.600000) {\textcolor[HTML]{408dba}{喚}};
\node[Square] at (-35.500000, 3.100000) {};
\node[Onyomi] at (-35.450000, 3.200000) {\hbox{\tate カン}};
\node[Kunyomi] at (-35.550000, 3.200000) {\hbox{\tate わめ}};
\node[Meaning] at (-35.500000, 4.850000) {scream};
\node[Kanji] at (-33.450000, 3.600000) {\textcolor[HTML]{cd8268}{換}};
\node[Square] at (-33.450000, 3.100000) {};
\node[Onyomi] at (-33.400000, 3.200000) {\hbox{\tate カン}};
\node[Kunyomi] at (-33.500000, 3.200000) {\hbox{\tate か.える}};
\node[Meaning] at (-33.450000, 4.850000) {exchange};
\node[Kanji] at (-31.400000, 3.600000) {\textcolor[HTML]{b0b0b5}{融}};
\node[Square] at (-31.400000, 3.100000) {};
\node[Onyomi] at (-31.350000, 3.200000) {\hbox{\tate ユウ}};
\node[Meaning] at (-31.400000, 4.850000) {dissolve};
\node[Kanji] at (-29.350000, 3.600000) {\textcolor[HTML]{cd8268}{施}};
\node[Square] at (-29.350000, 3.100000) {};
\node[Onyomi] at (-29.300000, 3.200000) {\hbox{\tate シ}};
\node[Kunyomi] at (-29.400000, 3.200000) {\hbox{\tate ほどこ.す}};
\node[Meaning] at (-29.350000, 4.850000) {carry out};
\node[Kanji] at (-27.300000, 3.600000) {\textcolor[HTML]{a3bac2}{旋}};
\node[Square] at (-27.300000, 3.100000) {};
\node[Onyomi] at (-27.250000, 3.200000) {\hbox{\tate セン}};
\node[Meaning] at (-27.300000, 4.850000) {rotation};
\node[Kanji] at (-25.250000, 3.600000) {\textcolor[HTML]{c8a59d}{遊}};
\node[Square] at (-25.250000, 3.100000) {};
\node[Onyomi] at (-25.200000, 3.200000) {\hbox{\tate ユウ}};
\node[Kunyomi] at (-25.300000, 3.200000) {\hbox{\tate あそ}};
\node[Meaning] at (-25.250000, 4.850000) {play};
\node[Kanji] at (-23.200000, 3.600000) {\textcolor[HTML]{d2a293}{旅}};
\node[Square] at (-23.200000, 3.100000) {};
\node[Onyomi] at (-23.150000, 3.200000) {\hbox{\tate リョ}};
\node[Kunyomi] at (-23.250000, 3.200000) {\hbox{\tate たび}};
\node[Meaning] at (-23.200000, 4.850000) {trip};
\node[Kanji] at (-21.150000, 3.600000) {\textcolor[HTML]{b74029}{物}};
\node[Square] at (-21.150000, 3.100000) {};
\node[Onyomi] at (-21.100000, 3.200000) {\hbox{\tate ブツ・モツ}};
\node[Kunyomi] at (-21.200000, 3.200000) {\hbox{\tate もの}};
\node[Meaning] at (-21.150000, 4.850000) {thing};
\node[Kanji] at (-19.100000, 3.600000) {\textcolor[HTML]{c8a59d}{易}};
\node[Square] at (-19.100000, 3.100000) {};
\node[Onyomi] at (-19.050000, 3.200000) {\hbox{\tate イ・エキ}};
\node[Kunyomi] at (-19.150000, 3.200000) {\hbox{\tate やさ.しい}};
\node[Meaning] at (-19.100000, 4.850000) {easy};
\node[Kanji] at (-17.050000, 3.600000) {\textcolor[HTML]{91b7c3}{賜}};
\node[Square] at (-17.050000, 3.100000) {};
\node[Onyomi] at (-17.000000, 3.200000) {\hbox{\tate シ}};
\node[Kunyomi] at (-17.100000, 3.200000) {\hbox{\tate たまわ.る}};
\node[Meaning] at (-17.050000, 4.850000) {grant};
\node[Kanji] at (-15.000000, 3.600000) {\textcolor[HTML]{91b7c3}{尿}};
\node[Square] at (-15.000000, 3.100000) {};
\node[Onyomi] at (-14.950000, 3.200000) {\hbox{\tate ニョウ}};
\node[Meaning] at (-15.000000, 4.850000) {urine};
\node[Kanji] at (-12.950000, 3.600000) {\textcolor[HTML]{91b7c3}{尼}};
\node[Square] at (-12.950000, 3.100000) {};
\node[Onyomi] at (-12.900000, 3.200000) {\hbox{\tate ニ}};
\node[Kunyomi] at (-13.000000, 3.200000) {\hbox{\tate あま}};
\node[Meaning] at (-12.950000, 4.850000) {nun};
\node[Kanji] at (-10.900000, 3.600000) {\textcolor[HTML]{91b7c3}{泥}};
\node[Square] at (-10.900000, 3.100000) {};
\node[Onyomi] at (-10.850000, 3.200000) {\hbox{\tate デイ}};
\node[Kunyomi] at (-10.950000, 3.200000) {\hbox{\tate どろ}};
\node[Meaning] at (-10.900000, 4.850000) {mud};
\node[Kanji] at (-8.850000, 3.600000) {\textcolor[HTML]{1e76bb}{塀}};
\node[Square] at (-8.850000, 3.100000) {};
\node[Onyomi] at (-8.800000, 3.200000) {\hbox{\tate ヘイ}};
\node[Meaning] at (-8.850000, 4.850000) {fence};
\node[Kanji] at (-6.800000, 3.600000) {\textcolor[HTML]{a3bac2}{履}};
\node[Square] at (-6.800000, 3.100000) {};
\node[Onyomi] at (-6.750000, 3.200000) {\hbox{\tate リ}};
\node[Kunyomi] at (-6.850000, 3.200000) {\hbox{\tate は.く}};
\node[Meaning] at (-6.800000, 4.850000) {boots};
\node[Kanji] at (-4.750000, 3.600000) {\textcolor[HTML]{cd8268}{屋}};
\node[Square] at (-4.750000, 3.100000) {};
\node[Onyomi] at (-4.700000, 3.200000) {\hbox{\tate オク}};
\node[Kunyomi] at (-4.800000, 3.200000) {\hbox{\tate や}};
\node[Meaning] at (-4.750000, 4.850000) {roof};
\node[Kanji] at (-2.700000, 3.600000) {\textcolor[HTML]{a3bac2}{握}};
\node[Square] at (-2.700000, 3.100000) {};
\node[Onyomi] at (-2.650000, 3.200000) {\hbox{\tate アク}};
\node[Kunyomi] at (-2.750000, 3.200000) {\hbox{\tate にぎ.る}};
\node[Meaning] at (-2.700000, 4.850000) {grip};
\node[Kanji] at (-0.650000, 3.600000) {\textcolor[HTML]{91b7c3}{屈}};
\node[Square] at (-0.650000, 3.100000) {};
\node[Onyomi] at (-0.600000, 3.200000) {\hbox{\tate クツ}};
\node[Kunyomi] at (-0.700000, 3.200000) {\hbox{\tate かが}};
\node[Meaning] at (-0.650000, 4.850000) {yield};
\node[Kanji] at (1.400000, 3.600000) {\textcolor[HTML]{b0b0b5}{掘}};
\node[Square] at (1.400000, 3.100000) {};
\node[Onyomi] at (1.450000, 3.200000) {\hbox{\tate クツ}};
\node[Kunyomi] at (1.350000, 3.200000) {\hbox{\tate ほ.る}};
\node[Meaning] at (1.400000, 4.850000) {dig};
\node[Kanji] at (3.450000, 3.600000) {\textcolor[HTML]{b0b0b5}{堀}};
\node[Square] at (3.450000, 3.100000) {};
\node[Onyomi] at (3.500000, 3.200000) {\hbox{\tate クツ}};
\node[Kunyomi] at (3.400000, 3.200000) {\hbox{\tate ほり}};
\node[Meaning] at (3.450000, 4.850000) {ditch};
\node[Kanji] at (5.500000, 3.600000) {\textcolor[HTML]{d2a293}{居}};
\node[Square] at (5.500000, 3.100000) {};
\node[Onyomi] at (5.550000, 3.200000) {\hbox{\tate キョ}};
\node[Kunyomi] at (5.450000, 3.200000) {\hbox{\tate い}};
\node[Meaning] at (5.500000, 4.850000) {alive};
\node[Kanji] at (7.550000, 3.600000) {\textcolor[HTML]{68a4bc}{据}};
\node[Square] at (7.550000, 3.100000) {};
\node[Onyomi] at (7.600000, 3.200000) {\hbox{\tate キョ}};
\node[Kunyomi] at (7.500000, 3.200000) {\hbox{\tate す}};
\node[Meaning] at (7.550000, 4.850000) {install};
\node[Kanji] at (9.600000, 3.600000) {\textcolor[HTML]{d2a293}{層}};
\node[Square] at (9.600000, 3.100000) {};
\node[Onyomi] at (9.650000, 3.200000) {\hbox{\tate ソウ}};
\node[Meaning] at (9.600000, 4.850000) {layer};
\node[Kanji] at (11.650000, 3.600000) {\textcolor[HTML]{cd8268}{局}};
\node[Square] at (11.650000, 3.100000) {};
\node[Onyomi] at (11.700000, 3.200000) {\hbox{\tate キョク}};
\node[Meaning] at (11.650000, 4.850000) {bureau};
\node[Kanji] at (13.700000, 3.600000) {\textcolor[HTML]{b0b0b5}{遅}};
\node[Square] at (13.700000, 3.100000) {};
\node[Onyomi] at (13.750000, 3.200000) {\hbox{\tate チ}};
\node[Kunyomi] at (13.650000, 3.200000) {\hbox{\tate おそ.い}};
\node[Meaning] at (13.700000, 4.850000) {slow};
\node[Kanji] at (15.750000, 3.600000) {\textcolor[HTML]{68a4bc}{漏}};
\node[Square] at (15.750000, 3.100000) {};
\node[Onyomi] at (15.800000, 3.200000) {\hbox{\tate ロウ}};
\node[Kunyomi] at (15.700000, 3.200000) {\hbox{\tate も.らす}};
\node[Meaning] at (15.750000, 4.850000) {leak};
\node[Kanji] at (17.800000, 3.600000) {\textcolor[HTML]{a3bac2}{刷}};
\node[Square] at (17.800000, 3.100000) {};
\node[Onyomi] at (17.850000, 3.200000) {\hbox{\tate サツ}};
\node[Kunyomi] at (17.750000, 3.200000) {\hbox{\tate す.る}};
\node[Meaning] at (17.800000, 4.850000) {printing};
\node[Kanji] at (19.850000, 3.600000) {\textcolor[HTML]{91b7c3}{尺}};
\node[Square] at (19.850000, 3.100000) {};
\node[Onyomi] at (19.900000, 3.200000) {\hbox{\tate シャク}};
\node[Meaning] at (19.850000, 4.850000) {shaku};
\node[Kanji] at (21.900000, 3.600000) {\textcolor[HTML]{a3bac2}{尽}};
\node[Square] at (21.900000, 3.100000) {};
\node[Onyomi] at (21.950000, 3.200000) {\hbox{\tate ジン}};
\node[Kunyomi] at (21.850000, 3.200000) {\hbox{\tate つ.くす}};
\node[Meaning] at (21.900000, 4.850000) {exhaust};
\node[Kanji] at (23.950000, 3.600000) {\textcolor[HTML]{d2a293}{沢}};
\node[Square] at (23.950000, 3.100000) {};
\node[Onyomi] at (24.000000, 3.200000) {\hbox{\tate タク}};
\node[Kunyomi] at (23.900000, 3.200000) {\hbox{\tate さわ}};
\node[Meaning] at (23.950000, 4.850000) {swamp};
\node[Kanji] at (26.000000, 3.600000) {\textcolor[HTML]{d2a293}{訳}};
\node[Square] at (26.000000, 3.100000) {};
\node[Onyomi] at (26.050000, 3.200000) {\hbox{\tate ヤク}};
\node[Kunyomi] at (25.950000, 3.200000) {\hbox{\tate わけ}};
\node[Meaning] at (26.000000, 4.850000) {translation};
\node[Kanji] at (28.050000, 3.600000) {\textcolor[HTML]{b0b0b5}{択}};
\node[Square] at (28.050000, 3.100000) {};
\node[Onyomi] at (28.100000, 3.200000) {\hbox{\tate タク}};
\node[Kunyomi] at (28.000000, 3.200000) {\hbox{\tate えら.ぶ}};
\node[Meaning] at (28.050000, 4.850000) {select};
\node[Kanji] at (30.100000, 3.600000) {\textcolor[HTML]{91b7c3}{昼}};
\node[Square] at (30.100000, 3.100000) {};
\node[Kunyomi] at (30.050000, 3.200000) {\hbox{\tate ひる}};
\node[Meaning] at (30.100000, 4.850000) {noon};
\node[Kanji] at (32.150000, 3.600000) {\textcolor[HTML]{cd8268}{戸}};
\node[Square] at (32.150000, 3.100000) {};
\node[Onyomi] at (32.200000, 3.200000) {\hbox{\tate コ}};
\node[Kunyomi] at (32.100000, 3.200000) {\hbox{\tate と}};
\node[Meaning] at (32.150000, 4.850000) {door};
\node[Kanji] at (34.200000, 3.600000) {\textcolor[HTML]{91b7c3}{肩}};
\node[Square] at (34.200000, 3.100000) {};
\node[Onyomi] at (34.250000, 3.200000) {\hbox{\tate ケン}};
\node[Kunyomi] at (34.150000, 3.200000) {\hbox{\tate かた}};
\node[Meaning] at (34.200000, 4.850000) {shoulder};
\node[Kanji] at (36.250000, 3.600000) {\textcolor[HTML]{c8a59d}{房}};
\node[Square] at (36.250000, 3.100000) {};
\node[Onyomi] at (36.300000, 3.200000) {\hbox{\tate ボウ}};
\node[Kunyomi] at (36.200000, 3.200000) {\hbox{\tate ふさ}};
\node[Meaning] at (36.250000, 4.850000) {cluster};
\node[Kanji] at (38.300000, 3.600000) {\textcolor[HTML]{91b7c3}{扇}};
\node[Square] at (38.300000, 3.100000) {};
\node[Onyomi] at (38.350000, 3.200000) {\hbox{\tate セン}};
\node[Kunyomi] at (38.250000, 3.200000) {\hbox{\tate おうぎ}};
\node[Meaning] at (38.300000, 4.850000) {folding fan};
\node[Kanji] at (40.350000, 3.600000) {\textcolor[HTML]{91b7c3}{炉}};
\node[Square] at (40.350000, 3.100000) {};
\node[Onyomi] at (40.400000, 3.200000) {\hbox{\tate ロ}};
\node[Kunyomi] at (40.300000, 3.200000) {\hbox{\tate いろり}};
\node[Meaning] at (40.350000, 4.850000) {furnace};
\node[Kanji] at (42.400000, 3.600000) {\textcolor[HTML]{c8a59d}{戻}};
\node[Square] at (42.400000, 3.100000) {};
\node[Onyomi] at (42.450000, 3.200000) {\hbox{\tate レイ}};
\node[Kunyomi] at (42.350000, 3.200000) {\hbox{\tate もど}};
\node[Meaning] at (42.400000, 4.850000) {return};
\node[Kanji] at (44.450000, 3.600000) {\textcolor[HTML]{68a4bc}{涙}};
\node[Square] at (44.450000, 3.100000) {};
\node[Onyomi] at (44.500000, 3.200000) {\hbox{\tate ルイ}};
\node[Kunyomi] at (44.400000, 3.200000) {\hbox{\tate なみだ}};
\node[Meaning] at (44.450000, 4.850000) {teardrop};
\node[Kanji] at (46.500000, 3.600000) {\textcolor[HTML]{a3bac2}{雇}};
\node[Square] at (46.500000, 3.100000) {};
\node[Onyomi] at (46.550000, 3.200000) {\hbox{\tate コ}};
\node[Kunyomi] at (46.450000, 3.200000) {\hbox{\tate やと.う}};
\node[Meaning] at (46.500000, 4.850000) {employ};
\node[Kanji] at (48.550000, 3.600000) {\textcolor[HTML]{a3bac2}{顧}};
\node[Square] at (48.550000, 3.100000) {};
\node[Onyomi] at (48.600000, 3.200000) {\hbox{\tate コ}};
\node[Kunyomi] at (48.500000, 3.200000) {\hbox{\tate かえり.みる}};
\node[Meaning] at (48.550000, 4.850000) {review};
\node[Kanji] at (50.600000, 3.600000) {\textcolor[HTML]{91b7c3}{啓}};
\node[Square] at (50.600000, 3.100000) {};
\node[Onyomi] at (50.650000, 3.200000) {\hbox{\tate ケイ}};
\node[Kunyomi] at (50.550000, 3.200000) {\hbox{\tate さと・ひら}};
\node[Meaning] at (50.600000, 4.850000) {enlighten};
\node[Kanji] at (52.650000, 3.600000) {\textcolor[HTML]{cd8268}{示}};
\node[Square] at (52.650000, 3.100000) {};
\node[Onyomi] at (52.700000, 3.200000) {\hbox{\tate ジ・シ}};
\node[Kunyomi] at (52.600000, 3.200000) {\hbox{\tate しめ.す}};
\node[Meaning] at (52.650000, 4.850000) {indicate};
\node[Kanji] at (54.700000, 3.600000) {\textcolor[HTML]{b0b0b5}{礼}};
\node[Square] at (54.700000, 3.100000) {};
\node[Onyomi] at (54.750000, 3.200000) {\hbox{\tate レイ}};
\node[Meaning] at (54.700000, 4.850000) {thanks};
\node[Kanji] at (56.750000, 3.600000) {\textcolor[HTML]{a3bac2}{祥}};
\node[Square] at (56.750000, 3.100000) {};
\node[Onyomi] at (56.800000, 3.200000) {\hbox{\tate ショウ}};
\node[Kunyomi] at (56.700000, 3.200000) {\hbox{\tate きざ・さいわ}};
\node[Meaning] at (56.750000, 4.850000) {auspicious};
\node[Meaning] at (-58.050000, 3.700000) {61.49\%};
\node[Kanji] at (-56.000000, 1.550000) {\textcolor[HTML]{a3bac2}{祝}};
\node[Square] at (-56.000000, 1.050000) {};
\node[Onyomi] at (-55.950000, 1.150000) {\hbox{\tate シュク}};
\node[Kunyomi] at (-56.050000, 1.150000) {\hbox{\tate いわ.う}};
\node[Meaning] at (-56.000000, 2.800000) {celebrate};
\node[Kanji] at (-53.950000, 1.550000) {\textcolor[HTML]{d69f8d}{福}};
\node[Square] at (-53.950000, 1.050000) {};
\node[Onyomi] at (-53.900000, 1.150000) {\hbox{\tate フク}};
\node[Meaning] at (-53.950000, 2.800000) {luck};
\node[Kanji] at (-51.900000, 1.550000) {\textcolor[HTML]{91b7c3}{祉}};
\node[Square] at (-51.900000, 1.050000) {};
\node[Onyomi] at (-51.850000, 1.150000) {\hbox{\tate シ}};
\node[Meaning] at (-51.900000, 2.800000) {welfare};
\node[Kanji] at (-49.850000, 1.550000) {\textcolor[HTML]{b74029}{社}};
\node[Square] at (-49.850000, 1.050000) {};
\node[Onyomi] at (-49.800000, 1.150000) {\hbox{\tate シャ}};
\node[Kunyomi] at (-49.900000, 1.150000) {\hbox{\tate やしろ}};
\node[Meaning] at (-49.850000, 2.800000) {company};
\node[Kanji] at (-47.800000, 1.550000) {\textcolor[HTML]{d69f8d}{視}};
\node[Square] at (-47.800000, 1.050000) {};
\node[Onyomi] at (-47.750000, 1.150000) {\hbox{\tate シ}};
\node[Meaning] at (-47.800000, 2.800000) {look at};
\node[Kanji] at (-45.750000, 1.550000) {\textcolor[HTML]{c8a59d}{奈}};
\node[Square] at (-45.750000, 1.050000) {};
\node[Onyomi] at (-45.700000, 1.150000) {\hbox{\tate ナ}};
\node[Kunyomi] at (-45.800000, 1.150000) {\hbox{\tate な}};
\node[Meaning] at (-45.750000, 2.800000) {nara};
\node[Kanji] at (-43.700000, 1.550000) {\textcolor[HTML]{a3bac2}{尉}};
\node[Square] at (-43.700000, 1.050000) {};
\node[Onyomi] at (-43.650000, 1.150000) {\hbox{\tate イ・ジョウ}};
\node[Meaning] at (-43.700000, 2.800000) {military officer};
\node[Kanji] at (-41.650000, 1.550000) {\textcolor[HTML]{91b7c3}{慰}};
\node[Square] at (-41.650000, 1.050000) {};
\node[Onyomi] at (-41.600000, 1.150000) {\hbox{\tate イ}};
\node[Kunyomi] at (-41.700000, 1.150000) {\hbox{\tate なぐさ.*}};
\node[Meaning] at (-41.650000, 2.800000) {consolation};
\node[Kanji] at (-39.600000, 1.550000) {\textcolor[HTML]{1e76bb}{款}};
\node[Square] at (-39.600000, 1.050000) {};
\node[Onyomi] at (-39.550000, 1.150000) {\hbox{\tate カン}};
\node[Meaning] at (-39.600000, 2.800000) {article};
\node[Kanji] at (-37.550000, 1.550000) {\textcolor[HTML]{c8a59d}{禁}};
\node[Square] at (-37.550000, 1.050000) {};
\node[Onyomi] at (-37.500000, 1.150000) {\hbox{\tate キン}};
\node[Meaning] at (-37.550000, 2.800000) {prohibition};
\node[Kanji] at (-35.500000, 1.550000) {\textcolor[HTML]{408dba}{襟}};
\node[Square] at (-35.500000, 1.050000) {};
\node[Onyomi] at (-35.450000, 1.150000) {\hbox{\tate キン}};
\node[Kunyomi] at (-35.550000, 1.150000) {\hbox{\tate えり}};
\node[Meaning] at (-35.500000, 2.800000) {collar};
\node[Kanji] at (-33.450000, 1.550000) {\textcolor[HTML]{d2a293}{宗}};
\node[Square] at (-33.450000, 1.050000) {};
\node[Onyomi] at (-33.400000, 1.150000) {\hbox{\tate シュウ}};
\node[Meaning] at (-33.450000, 2.800000) {religion};
\node[Kanji] at (-31.400000, 1.550000) {\textcolor[HTML]{91b7c3}{崇}};
\node[Square] at (-31.400000, 1.050000) {};
\node[Onyomi] at (-31.350000, 1.150000) {\hbox{\tate スウ}};
\node[Kunyomi] at (-31.450000, 1.150000) {\hbox{\tate あが}};
\node[Meaning] at (-31.400000, 2.800000) {worship};
\node[Kanji] at (-29.350000, 1.550000) {\textcolor[HTML]{d2a293}{祭}};
\node[Square] at (-29.350000, 1.050000) {};
\node[Onyomi] at (-29.300000, 1.150000) {\hbox{\tate サイ}};
\node[Kunyomi] at (-29.400000, 1.150000) {\hbox{\tate まつり・まつ}};
\node[Meaning] at (-29.350000, 2.800000) {festival};
\node[Kanji] at (-27.300000, 1.550000) {\textcolor[HTML]{d2a293}{察}};
\node[Square] at (-27.300000, 1.050000) {};
\node[Onyomi] at (-27.250000, 1.150000) {\hbox{\tate サツ}};
\node[Kunyomi] at (-27.350000, 1.150000) {\hbox{\tate さっ.する}};
\node[Meaning] at (-27.300000, 2.800000) {guess};
\node[Kanji] at (-25.250000, 1.550000) {\textcolor[HTML]{68a4bc}{擦}};
\node[Square] at (-25.250000, 1.050000) {};
\node[Onyomi] at (-25.200000, 1.150000) {\hbox{\tate サツ}};
\node[Kunyomi] at (-25.300000, 1.150000) {\hbox{\tate こす・す}};
\node[Meaning] at (-25.250000, 2.800000) {grate};
\node[Kanji] at (-23.200000, 1.550000) {\textcolor[HTML]{cd8268}{由}};
\node[Square] at (-23.200000, 1.050000) {};
\node[Onyomi] at (-23.150000, 1.150000) {\hbox{\tate ユウ}};
\node[Kunyomi] at (-23.250000, 1.150000) {\hbox{\tate よし・よ.る}};
\node[Meaning] at (-23.200000, 2.800000) {reason};
\node[Kanji] at (-21.150000, 1.550000) {\textcolor[HTML]{91b7c3}{抽}};
\node[Square] at (-21.150000, 1.050000) {};
\node[Onyomi] at (-21.100000, 1.150000) {\hbox{\tate チュウ}};
\node[Meaning] at (-21.150000, 2.800000) {pluck};
\node[Kanji] at (-19.100000, 1.550000) {\textcolor[HTML]{c8a59d}{油}};
\node[Square] at (-19.100000, 1.050000) {};
\node[Onyomi] at (-19.050000, 1.150000) {\hbox{\tate ユ}};
\node[Kunyomi] at (-19.150000, 1.150000) {\hbox{\tate あぶら}};
\node[Meaning] at (-19.100000, 2.800000) {oil};
\node[Kanji] at (-17.050000, 1.550000) {\textcolor[HTML]{68a4bc}{袖}};
\node[Square] at (-17.050000, 1.050000) {};
\node[Onyomi] at (-17.000000, 1.150000) {\hbox{\tate シュウ}};
\node[Kunyomi] at (-17.100000, 1.150000) {\hbox{\tate そで}};
\node[Meaning] at (-17.050000, 2.800000) {sleeve};
\node[Kanji] at (-15.000000, 1.550000) {\textcolor[HTML]{c8a59d}{宙}};
\node[Square] at (-15.000000, 1.050000) {};
\node[Onyomi] at (-14.950000, 1.150000) {\hbox{\tate チュウ}};
\node[Meaning] at (-15.000000, 2.800000) {mid air};
\node[Kanji] at (-12.950000, 1.550000) {\textcolor[HTML]{a3bac2}{届}};
\node[Square] at (-12.950000, 1.050000) {};
\node[Kunyomi] at (-13.000000, 1.150000) {\hbox{\tate とど}};
\node[Meaning] at (-12.950000, 2.800000) {deliver};
\node[Kanji] at (-10.900000, 1.550000) {\textcolor[HTML]{68a4bc}{笛}};
\node[Square] at (-10.900000, 1.050000) {};
\node[Onyomi] at (-10.850000, 1.150000) {\hbox{\tate テキ}};
\node[Kunyomi] at (-10.950000, 1.150000) {\hbox{\tate ふえ}};
\node[Meaning] at (-10.900000, 2.800000) {flute};
\node[Kanji] at (-8.850000, 1.550000) {\textcolor[HTML]{b0b0b5}{軸}};
\node[Square] at (-8.850000, 1.050000) {};
\node[Onyomi] at (-8.800000, 1.150000) {\hbox{\tate ジク}};
\node[Meaning] at (-8.850000, 2.800000) {axis};
\node[Kanji] at (-6.800000, 1.550000) {\textcolor[HTML]{d2a293}{甲}};
\node[Square] at (-6.800000, 1.050000) {};
\node[Onyomi] at (-6.750000, 1.150000) {\hbox{\tate コウ・カン}};
\node[Kunyomi] at (-6.850000, 1.150000) {\hbox{\tate か}};
\node[Meaning] at (-6.800000, 2.800000) {turtle shell};
\node[Kanji] at (-4.750000, 1.550000) {\textcolor[HTML]{b0b0b5}{押}};
\node[Square] at (-4.750000, 1.050000) {};
\node[Onyomi] at (-4.700000, 1.150000) {\hbox{\tate オウ}};
\node[Kunyomi] at (-4.800000, 1.150000) {\hbox{\tate お}};
\node[Meaning] at (-4.750000, 2.800000) {push};
\node[Kanji] at (-2.700000, 1.550000) {\textcolor[HTML]{68a4bc}{岬}};
\node[Square] at (-2.700000, 1.050000) {};
\node[Onyomi] at (-2.650000, 1.150000) {\hbox{\tate コウ}};
\node[Kunyomi] at (-2.750000, 1.150000) {\hbox{\tate みさき}};
\node[Meaning] at (-2.700000, 2.800000) {cape};
\node[Kanji] at (-0.650000, 1.550000) {\textcolor[HTML]{91b7c3}{挿}};
\node[Square] at (-0.650000, 1.050000) {};
\node[Onyomi] at (-0.600000, 1.150000) {\hbox{\tate ソウ}};
\node[Kunyomi] at (-0.700000, 1.150000) {\hbox{\tate さ.す}};
\node[Meaning] at (-0.650000, 2.800000) {insert};
\node[Kanji] at (1.400000, 1.550000) {\textcolor[HTML]{c8a59d}{申}};
\node[Square] at (1.400000, 1.050000) {};
\node[Onyomi] at (1.450000, 1.150000) {\hbox{\tate シン}};
\node[Kunyomi] at (1.350000, 1.150000) {\hbox{\tate もう}};
\node[Meaning] at (1.400000, 2.800000) {say humbly};
\node[Kanji] at (3.450000, 1.550000) {\textcolor[HTML]{c8a59d}{伸}};
\node[Square] at (3.450000, 1.050000) {};
\node[Onyomi] at (3.500000, 1.150000) {\hbox{\tate シン}};
\node[Kunyomi] at (3.400000, 1.150000) {\hbox{\tate の}};
\node[Meaning] at (3.450000, 2.800000) {stretch};
\node[Kanji] at (5.500000, 1.550000) {\textcolor[HTML]{c36143}{神}};
\node[Square] at (5.500000, 1.050000) {};
\node[Onyomi] at (5.550000, 1.150000) {\hbox{\tate シン}};
\node[Kunyomi] at (5.450000, 1.150000) {\hbox{\tate かみ}};
\node[Meaning] at (5.500000, 2.800000) {god};
\node[Kanji] at (7.550000, 1.550000) {\textcolor[HTML]{a3bac2}{捜}};
\node[Square] at (7.550000, 1.050000) {};
\node[Onyomi] at (7.600000, 1.150000) {\hbox{\tate ソウ}};
\node[Kunyomi] at (7.500000, 1.150000) {\hbox{\tate さが.す}};
\node[Meaning] at (7.550000, 2.800000) {search};
\node[Kanji] at (9.600000, 1.550000) {\textcolor[HTML]{c36143}{果}};
\node[Square] at (9.600000, 1.050000) {};
\node[Onyomi] at (9.650000, 1.150000) {\hbox{\tate カ}};
\node[Kunyomi] at (9.550000, 1.150000) {\hbox{\tate くだ}};
\node[Meaning] at (9.600000, 2.800000) {fruit};
\node[Kanji] at (11.650000, 1.550000) {\textcolor[HTML]{91b7c3}{菓}};
\node[Square] at (11.650000, 1.050000) {};
\node[Onyomi] at (11.700000, 1.150000) {\hbox{\tate カ}};
\node[Meaning] at (11.650000, 2.800000) {cake};
\node[Kanji] at (13.700000, 1.550000) {\textcolor[HTML]{c8a59d}{課}};
\node[Square] at (13.700000, 1.050000) {};
\node[Onyomi] at (13.750000, 1.150000) {\hbox{\tate カ}};
\node[Meaning] at (13.700000, 2.800000) {section};
\node[Kanji] at (15.750000, 1.550000) {\textcolor[HTML]{68a4bc}{裸}};
\node[Square] at (15.750000, 1.050000) {};
\node[Onyomi] at (15.800000, 1.150000) {\hbox{\tate ラ}};
\node[Kunyomi] at (15.700000, 1.150000) {\hbox{\tate はだか}};
\node[Meaning] at (15.750000, 2.800000) {naked};
\node[Kanji] at (17.800000, 1.550000) {\textcolor[HTML]{29409e}{斤}};
\node[Square] at (17.800000, 1.050000) {};
\node[Onyomi] at (17.850000, 1.150000) {\hbox{\tate キン}};
\node[Meaning] at (17.800000, 2.800000) {axe};
\node[Kanji] at (19.850000, 1.550000) {\textcolor[HTML]{b0b0b5}{析}};
\node[Square] at (19.850000, 1.050000) {};
\node[Onyomi] at (19.900000, 1.150000) {\hbox{\tate セキ}};
\node[Meaning] at (19.850000, 2.800000) {analysis};
\node[Kanji] at (21.900000, 1.550000) {\textcolor[HTML]{b74029}{所}};
\node[Square] at (21.900000, 1.050000) {};
\node[Onyomi] at (21.950000, 1.150000) {\hbox{\tate ショ}};
\node[Kunyomi] at (21.850000, 1.150000) {\hbox{\tate ところ}};
\node[Meaning] at (21.900000, 2.800000) {place};
\node[Kanji] at (23.950000, 1.550000) {\textcolor[HTML]{91b7c3}{祈}};
\node[Square] at (23.950000, 1.050000) {};
\node[Onyomi] at (24.000000, 1.150000) {\hbox{\tate キ}};
\node[Kunyomi] at (23.900000, 1.150000) {\hbox{\tate いの.る}};
\node[Meaning] at (23.950000, 2.800000) {pray};
\node[Kanji] at (26.000000, 1.550000) {\textcolor[HTML]{c36143}{近}};
\node[Square] at (26.000000, 1.050000) {};
\node[Onyomi] at (26.050000, 1.150000) {\hbox{\tate キン}};
\node[Kunyomi] at (25.950000, 1.150000) {\hbox{\tate ちか.い}};
\node[Meaning] at (26.000000, 2.800000) {near};
\node[Kanji] at (28.050000, 1.550000) {\textcolor[HTML]{c8a59d}{折}};
\node[Square] at (28.050000, 1.050000) {};
\node[Onyomi] at (28.100000, 1.150000) {\hbox{\tate セツ}};
\node[Kunyomi] at (28.000000, 1.150000) {\hbox{\tate お.る}};
\node[Meaning] at (28.050000, 2.800000) {fold};
\node[Kanji] at (30.100000, 1.550000) {\textcolor[HTML]{b0b0b5}{哲}};
\node[Square] at (30.100000, 1.050000) {};
\node[Onyomi] at (30.150000, 1.150000) {\hbox{\tate テツ}};
\node[Meaning] at (30.100000, 2.800000) {philosophy};
\node[Kanji] at (32.150000, 1.550000) {\textcolor[HTML]{408dba}{逝}};
\node[Square] at (32.150000, 1.050000) {};
\node[Onyomi] at (32.200000, 1.150000) {\hbox{\tate セイ}};
\node[Kunyomi] at (32.100000, 1.150000) {\hbox{\tate い.く}};
\node[Meaning] at (32.150000, 2.800000) {die};
\node[Kanji] at (34.200000, 1.550000) {\textcolor[HTML]{68a4bc}{誓}};
\node[Square] at (34.200000, 1.050000) {};
\node[Onyomi] at (34.250000, 1.150000) {\hbox{\tate セイ}};
\node[Kunyomi] at (34.150000, 1.150000) {\hbox{\tate ちか.う}};
\node[Meaning] at (34.200000, 2.800000) {vow};
\node[Kanji] at (36.250000, 1.550000) {\textcolor[HTML]{68a4bc}{暫}};
\node[Square] at (36.250000, 1.050000) {};
\node[Onyomi] at (36.300000, 1.150000) {\hbox{\tate ザン}};
\node[Kunyomi] at (36.200000, 1.150000) {\hbox{\tate しばら.く}};
\node[Meaning] at (36.250000, 2.800000) {temporarily};
\node[Kanji] at (38.300000, 1.550000) {\textcolor[HTML]{1e76bb}{漸}};
\node[Square] at (38.300000, 1.050000) {};
\node[Onyomi] at (38.350000, 1.150000) {\hbox{\tate ゼン}};
\node[Kunyomi] at (38.250000, 1.150000) {\hbox{\tate ようや・やや}};
\node[Meaning] at (38.300000, 2.800000) {gradually};
\node[Kanji] at (40.350000, 1.550000) {\textcolor[HTML]{d69f8d}{断}};
\node[Square] at (40.350000, 1.050000) {};
\node[Onyomi] at (40.400000, 1.150000) {\hbox{\tate ダン}};
\node[Kunyomi] at (40.300000, 1.150000) {\hbox{\tate ことわ.る}};
\node[Meaning] at (40.350000, 2.800000) {cut off};
\node[Kanji] at (42.400000, 1.550000) {\textcolor[HTML]{d69f8d}{質}};
\node[Square] at (42.400000, 1.050000) {};
\node[Onyomi] at (42.450000, 1.150000) {\hbox{\tate シツ・シチ}};
\node[Meaning] at (42.400000, 2.800000) {quality};
\node[Kanji] at (44.450000, 1.550000) {\textcolor[HTML]{1059be}{斥}};
\node[Square] at (44.450000, 1.050000) {};
\node[Onyomi] at (44.500000, 1.150000) {\hbox{\tate セキ}};
\node[Meaning] at (44.450000, 2.800000) {reject};
\node[Kanji] at (46.500000, 1.550000) {\textcolor[HTML]{c8a59d}{訴}};
\node[Square] at (46.500000, 1.050000) {};
\node[Onyomi] at (46.550000, 1.150000) {\hbox{\tate ソ}};
\node[Kunyomi] at (46.450000, 1.150000) {\hbox{\tate うった.える}};
\node[Meaning] at (46.500000, 2.800000) {sue};
\node[Kanji] at (48.550000, 1.550000) {\textcolor[HTML]{68a4bc}{昨}};
\node[Square] at (48.550000, 1.050000) {};
\node[Onyomi] at (48.600000, 1.150000) {\hbox{\tate サク}};
\node[Meaning] at (48.550000, 2.800000) {previous};
\node[Kanji] at (50.600000, 1.550000) {\textcolor[HTML]{408dba}{詐}};
\node[Square] at (50.600000, 1.050000) {};
\node[Onyomi] at (50.650000, 1.150000) {\hbox{\tate サ}};
\node[Kunyomi] at (50.550000, 1.150000) {\hbox{\tate いつわ.る}};
\node[Meaning] at (50.600000, 2.800000) {lie};
\node[Kanji] at (52.650000, 1.550000) {\textcolor[HTML]{a11d25}{作}};
\node[Square] at (52.650000, 1.050000) {};
\node[Onyomi] at (52.700000, 1.150000) {\hbox{\tate サク・サ}};
\node[Kunyomi] at (52.600000, 1.150000) {\hbox{\tate つく.る}};
\node[Meaning] at (52.650000, 2.800000) {make};
\node[Kanji] at (54.700000, 1.550000) {\textcolor[HTML]{b0b0b5}{雪}};
\node[Square] at (54.700000, 1.050000) {};
\node[Onyomi] at (54.750000, 1.150000) {\hbox{\tate セツ}};
\node[Kunyomi] at (54.650000, 1.150000) {\hbox{\tate ゆき}};
\node[Meaning] at (54.700000, 2.800000) {snow};
\node[Kanji] at (56.750000, 1.550000) {\textcolor[HTML]{cd8268}{録}};
\node[Square] at (56.750000, 1.050000) {};
\node[Onyomi] at (56.800000, 1.150000) {\hbox{\tate ロク}};
\node[Meaning] at (56.750000, 2.800000) {record};
\node[Meaning] at (-58.050000, 1.650000) {63.99\%};
\node[Kanji] at (-56.000000, -0.500000) {\textcolor[HTML]{91b7c3}{尋}};
\node[Square] at (-56.000000, -1.000000) {};
\node[Onyomi] at (-55.950000, -0.900000) {\hbox{\tate ジン}};
\node[Kunyomi] at (-56.050000, -0.900000) {\hbox{\tate たず.ねる}};
\node[Meaning] at (-56.000000, 0.750000) {inquire};
\node[Kanji] at (-53.950000, -0.500000) {\textcolor[HTML]{d69f8d}{急}};
\node[Square] at (-53.950000, -1.000000) {};
\node[Onyomi] at (-53.900000, -0.900000) {\hbox{\tate キュウ}};
\node[Kunyomi] at (-54.000000, -0.900000) {\hbox{\tate いそ.ぐ}};
\node[Meaning] at (-53.950000, 0.750000) {hurry};
\node[Kanji] at (-51.900000, -0.500000) {\textcolor[HTML]{68a4bc}{穏}};
\node[Square] at (-51.900000, -1.000000) {};
\node[Onyomi] at (-51.850000, -0.900000) {\hbox{\tate オン}};
\node[Kunyomi] at (-51.950000, -0.900000) {\hbox{\tate おだ.やか}};
\node[Meaning] at (-51.900000, 0.750000) {calm};
\node[Kanji] at (-49.850000, -0.500000) {\textcolor[HTML]{c8a59d}{侵}};
\node[Square] at (-49.850000, -1.000000) {};
\node[Onyomi] at (-49.800000, -0.900000) {\hbox{\tate シン}};
\node[Kunyomi] at (-49.900000, -0.900000) {\hbox{\tate おか.す}};
\node[Meaning] at (-49.850000, 0.750000) {invade};
\node[Kanji] at (-47.800000, -0.500000) {\textcolor[HTML]{91b7c3}{浸}};
\node[Square] at (-47.800000, -1.000000) {};
\node[Onyomi] at (-47.750000, -0.900000) {\hbox{\tate シン}};
\node[Kunyomi] at (-47.850000, -0.900000) {\hbox{\tate ひた.*}};
\node[Meaning] at (-47.800000, 0.750000) {immersed};
\node[Kanji] at (-45.750000, -0.500000) {\textcolor[HTML]{a3bac2}{寝}};
\node[Square] at (-45.750000, -1.000000) {};
\node[Onyomi] at (-45.700000, -0.900000) {\hbox{\tate シン}};
\node[Kunyomi] at (-45.800000, -0.900000) {\hbox{\tate ね}};
\node[Meaning] at (-45.750000, 0.750000) {lie down};
\node[Kanji] at (-43.700000, -0.500000) {\textcolor[HTML]{b0b0b5}{婦}};
\node[Square] at (-43.700000, -1.000000) {};
\node[Onyomi] at (-43.650000, -0.900000) {\hbox{\tate フ}};
\node[Meaning] at (-43.700000, 0.750000) {wife};
\node[Kanji] at (-41.650000, -0.500000) {\textcolor[HTML]{91b7c3}{掃}};
\node[Square] at (-41.650000, -1.000000) {};
\node[Onyomi] at (-41.600000, -0.900000) {\hbox{\tate ソウ}};
\node[Kunyomi] at (-41.700000, -0.900000) {\hbox{\tate は.く}};
\node[Meaning] at (-41.650000, 0.750000) {sweep};
\node[Kanji] at (-39.600000, -0.500000) {\textcolor[HTML]{b74029}{当}};
\node[Square] at (-39.600000, -1.000000) {};
\node[Onyomi] at (-39.550000, -0.900000) {\hbox{\tate トウ}};
\node[Kunyomi] at (-39.650000, -0.900000) {\hbox{\tate あ.たる}};
\node[Meaning] at (-39.600000, 0.750000) {right};
\node[Kanji] at (-37.550000, -0.500000) {\textcolor[HTML]{d69f8d}{争}};
\node[Square] at (-37.550000, -1.000000) {};
\node[Onyomi] at (-37.500000, -0.900000) {\hbox{\tate ソウ}};
\node[Kunyomi] at (-37.600000, -0.900000) {\hbox{\tate あらそ.う}};
\node[Meaning] at (-37.550000, 0.750000) {conflict};
\node[Kanji] at (-35.500000, -0.500000) {\textcolor[HTML]{a3bac2}{浄}};
\node[Square] at (-35.500000, -1.000000) {};
\node[Onyomi] at (-35.450000, -0.900000) {\hbox{\tate ジョウ・セイ}};
\node[Kunyomi] at (-35.550000, -0.900000) {\hbox{\tate きよ.い}};
\node[Meaning] at (-35.500000, 0.750000) {cleanse};
\node[Kanji] at (-33.450000, -0.500000) {\textcolor[HTML]{a11d25}{事}};
\node[Square] at (-33.450000, -1.000000) {};
\node[Onyomi] at (-33.400000, -0.900000) {\hbox{\tate ジ}};
\node[Kunyomi] at (-33.500000, -0.900000) {\hbox{\tate こと・つか}};
\node[Meaning] at (-33.450000, 0.750000) {action};
\node[Kanji] at (-31.400000, -0.500000) {\textcolor[HTML]{b0b0b5}{唐}};
\node[Square] at (-31.400000, -1.000000) {};
\node[Onyomi] at (-31.350000, -0.900000) {\hbox{\tate トウ}};
\node[Meaning] at (-31.400000, 0.750000) {china};
\node[Kanji] at (-29.350000, -0.500000) {\textcolor[HTML]{a3bac2}{糖}};
\node[Square] at (-29.350000, -1.000000) {};
\node[Onyomi] at (-29.300000, -0.900000) {\hbox{\tate トウ}};
\node[Meaning] at (-29.350000, 0.750000) {sugar};
\node[Kanji] at (-27.300000, -0.500000) {\textcolor[HTML]{c8a59d}{康}};
\node[Square] at (-27.300000, -1.000000) {};
\node[Onyomi] at (-27.250000, -0.900000) {\hbox{\tate コウ}};
\node[Meaning] at (-27.300000, 0.750000) {health};
\node[Kanji] at (-25.250000, -0.500000) {\textcolor[HTML]{a3bac2}{逮}};
\node[Square] at (-25.250000, -1.000000) {};
\node[Onyomi] at (-25.200000, -0.900000) {\hbox{\tate タイ}};
\node[Meaning] at (-25.250000, 0.750000) {apprehend};
\node[Kanji] at (-23.200000, -0.500000) {\textcolor[HTML]{d69f8d}{伊}};
\node[Square] at (-23.200000, -1.000000) {};
\node[Onyomi] at (-23.150000, -0.900000) {\hbox{\tate イ}};
\node[Kunyomi] at (-23.250000, -0.900000) {\hbox{\tate だ}};
\node[Meaning] at (-23.200000, 0.750000) {italy};
\node[Kanji] at (-21.150000, -0.500000) {\textcolor[HTML]{b0b0b5}{君}};
\node[Square] at (-21.150000, -1.000000) {};
\node[Onyomi] at (-21.100000, -0.900000) {\hbox{\tate クン}};
\node[Kunyomi] at (-21.200000, -0.900000) {\hbox{\tate きみ}};
\node[Meaning] at (-21.150000, 0.750000) {buddy};
\node[Kanji] at (-19.100000, -0.500000) {\textcolor[HTML]{d2a293}{群}};
\node[Square] at (-19.100000, -1.000000) {};
\node[Onyomi] at (-19.050000, -0.900000) {\hbox{\tate グン}};
\node[Kunyomi] at (-19.150000, -0.900000) {\hbox{\tate む・むら}};
\node[Meaning] at (-19.100000, 0.750000) {flock};
\node[Kanji] at (-17.050000, -0.500000) {\textcolor[HTML]{a3bac2}{耐}};
\node[Square] at (-17.050000, -1.000000) {};
\node[Onyomi] at (-17.000000, -0.900000) {\hbox{\tate タイ}};
\node[Kunyomi] at (-17.100000, -0.900000) {\hbox{\tate た.える}};
\node[Meaning] at (-17.050000, 0.750000) {resistant};
\node[Kanji] at (-15.000000, -0.500000) {\textcolor[HTML]{a3bac2}{需}};
\node[Square] at (-15.000000, -1.000000) {};
\node[Onyomi] at (-14.950000, -0.900000) {\hbox{\tate ジュ}};
\node[Meaning] at (-15.000000, 0.750000) {demand};
\node[Kanji] at (-12.950000, -0.500000) {\textcolor[HTML]{68a4bc}{儒}};
\node[Square] at (-12.950000, -1.000000) {};
\node[Onyomi] at (-12.900000, -0.900000) {\hbox{\tate ジュ}};
\node[Meaning] at (-12.950000, 0.750000) {Confucian};
\node[Kanji] at (-10.900000, -0.500000) {\textcolor[HTML]{d2a293}{端}};
\node[Square] at (-10.900000, -1.000000) {};
\node[Onyomi] at (-10.850000, -0.900000) {\hbox{\tate タン}};
\node[Kunyomi] at (-10.950000, -0.900000) {\hbox{\tate はし}};
\node[Meaning] at (-10.900000, 0.750000) {edge};
\node[Kanji] at (-8.850000, -0.500000) {\textcolor[HTML]{cd8268}{両}};
\node[Square] at (-8.850000, -1.000000) {};
\node[Onyomi] at (-8.800000, -0.900000) {\hbox{\tate リョウ}};
\node[Meaning] at (-8.850000, 0.750000) {both};
\node[Kanji] at (-6.800000, -0.500000) {\textcolor[HTML]{d2a293}{満}};
\node[Square] at (-6.800000, -1.000000) {};
\node[Onyomi] at (-6.750000, -0.900000) {\hbox{\tate マン}};
\node[Kunyomi] at (-6.850000, -0.900000) {\hbox{\tate み}};
\node[Meaning] at (-6.800000, 0.750000) {full};
\node[Kanji] at (-4.750000, -0.500000) {\textcolor[HTML]{b74029}{画}};
\node[Square] at (-4.750000, -1.000000) {};
\node[Onyomi] at (-4.700000, -0.900000) {\hbox{\tate ガ・カク}};
\node[Meaning] at (-4.750000, 0.750000) {drawing};
\node[Kanji] at (-2.700000, -0.500000) {\textcolor[HTML]{b0b0b5}{歯}};
\node[Square] at (-2.700000, -1.000000) {};
\node[Kunyomi] at (-2.750000, -0.900000) {\hbox{\tate は}};
\node[Meaning] at (-2.700000, 0.750000) {tooth};
\node[Kanji] at (-0.650000, -0.500000) {\textcolor[HTML]{cd8268}{曲}};
\node[Square] at (-0.650000, -1.000000) {};
\node[Onyomi] at (-0.600000, -0.900000) {\hbox{\tate キョク}};
\node[Kunyomi] at (-0.700000, -0.900000) {\hbox{\tate ま.げる}};
\node[Meaning] at (-0.650000, 0.750000) {music};
\node[Kanji] at (1.400000, -0.500000) {\textcolor[HTML]{a3bac2}{曹}};
\node[Square] at (1.400000, -1.000000) {};
\node[Onyomi] at (1.450000, -0.900000) {\hbox{\tate ソウ・ゾウ}};
\node[Kunyomi] at (1.350000, -0.900000) {\hbox{\tate つかさ}};
\node[Meaning] at (1.400000, 0.750000) {official};
\node[Kanji] at (3.450000, -0.500000) {\textcolor[HTML]{91b7c3}{遭}};
\node[Square] at (3.450000, -1.000000) {};
\node[Onyomi] at (3.500000, -0.900000) {\hbox{\tate ソウ}};
\node[Kunyomi] at (3.400000, -0.900000) {\hbox{\tate あ}};
\node[Meaning] at (3.450000, 0.750000) {encounter};
\node[Kanji] at (5.500000, -0.500000) {\textcolor[HTML]{408dba}{槽}};
\node[Square] at (5.500000, -1.000000) {};
\node[Onyomi] at (5.550000, -0.900000) {\hbox{\tate ソウ}};
\node[Kunyomi] at (5.450000, -0.900000) {\hbox{\tate ふね}};
\node[Meaning] at (5.500000, 0.750000) {tank};
\node[Kanji] at (7.550000, -0.500000) {\textcolor[HTML]{91b7c3}{斗}};
\node[Square] at (7.550000, -1.000000) {};
\node[Onyomi] at (7.600000, -0.900000) {\hbox{\tate ト}};
\node[Meaning] at (7.550000, 0.750000) {ladle};
\node[Kanji] at (9.600000, -0.500000) {\textcolor[HTML]{cd8268}{料}};
\node[Square] at (9.600000, -1.000000) {};
\node[Onyomi] at (9.650000, -0.900000) {\hbox{\tate リョウ}};
\node[Meaning] at (9.600000, 0.750000) {fee};
\node[Kanji] at (11.650000, -0.500000) {\textcolor[HTML]{cd8268}{科}};
\node[Square] at (11.650000, -1.000000) {};
\node[Onyomi] at (11.700000, -0.900000) {\hbox{\tate カ}};
\node[Meaning] at (11.650000, 0.750000) {science};
\node[Kanji] at (13.700000, -0.500000) {\textcolor[HTML]{d69f8d}{図}};
\node[Square] at (13.700000, -1.000000) {};
\node[Onyomi] at (13.750000, -0.900000) {\hbox{\tate ズ・ト}};
\node[Kunyomi] at (13.650000, -0.900000) {\hbox{\tate え・はか.る}};
\node[Meaning] at (13.700000, 0.750000) {diagram};
\node[Kanji] at (15.750000, -0.500000) {\textcolor[HTML]{a11d25}{用}};
\node[Square] at (15.750000, -1.000000) {};
\node[Onyomi] at (15.800000, -0.900000) {\hbox{\tate ヨウ}};
\node[Kunyomi] at (15.700000, -0.900000) {\hbox{\tate もち.いる}};
\node[Meaning] at (15.750000, 0.750000) {task};
\node[Kanji] at (17.800000, -0.500000) {\textcolor[HTML]{408dba}{庸}};
\node[Square] at (17.800000, -1.000000) {};
\node[Onyomi] at (17.850000, -0.900000) {\hbox{\tate ヨウ}};
\node[Meaning] at (17.800000, 0.750000) {common};
\node[Kanji] at (19.850000, -0.500000) {\textcolor[HTML]{cd8268}{備}};
\node[Square] at (19.850000, -1.000000) {};
\node[Onyomi] at (19.900000, -0.900000) {\hbox{\tate ビ}};
\node[Kunyomi] at (19.800000, -0.900000) {\hbox{\tate そな.える}};
\node[Meaning] at (19.850000, 0.750000) {provide};
\node[Kanji] at (21.900000, -0.500000) {\textcolor[HTML]{91b7c3}{昔}};
\node[Square] at (21.900000, -1.000000) {};
\node[Kunyomi] at (21.850000, -0.900000) {\hbox{\tate むかし}};
\node[Meaning] at (21.900000, 0.750000) {long ago};
\node[Kanji] at (23.950000, -0.500000) {\textcolor[HTML]{68a4bc}{錯}};
\node[Square] at (23.950000, -1.000000) {};
\node[Onyomi] at (24.000000, -0.900000) {\hbox{\tate サク・シャク}};
\node[Meaning] at (23.950000, 0.750000) {confused};
\node[Kanji] at (26.000000, -0.500000) {\textcolor[HTML]{a3bac2}{借}};
\node[Square] at (26.000000, -1.000000) {};
\node[Onyomi] at (26.050000, -0.900000) {\hbox{\tate シャク}};
\node[Kunyomi] at (25.950000, -0.900000) {\hbox{\tate か.りる}};
\node[Meaning] at (26.000000, 0.750000) {borrow};
\node[Kanji] at (28.050000, -0.500000) {\textcolor[HTML]{408dba}{惜}};
\node[Square] at (28.050000, -1.000000) {};
\node[Onyomi] at (28.100000, -0.900000) {\hbox{\tate セキ}};
\node[Kunyomi] at (28.000000, -0.900000) {\hbox{\tate お}};
\node[Meaning] at (28.050000, 0.750000) {frugal};
\node[Kanji] at (30.100000, -0.500000) {\textcolor[HTML]{91b7c3}{措}};
\node[Square] at (30.100000, -1.000000) {};
\node[Onyomi] at (30.150000, -0.900000) {\hbox{\tate ソ}};
\node[Meaning] at (30.100000, 0.750000) {set aside};
\node[Kanji] at (32.150000, -0.500000) {\textcolor[HTML]{c8a59d}{散}};
\node[Square] at (32.150000, -1.000000) {};
\node[Onyomi] at (32.200000, -0.900000) {\hbox{\tate サン}};
\node[Kunyomi] at (32.100000, -0.900000) {\hbox{\tate ち.*}};
\node[Meaning] at (32.150000, 0.750000) {scatter};
\node[Kanji] at (34.200000, -0.500000) {\textcolor[HTML]{91b7c3}{庶}};
\node[Square] at (34.200000, -1.000000) {};
\node[Onyomi] at (34.250000, -0.900000) {\hbox{\tate ショ}};
\node[Meaning] at (34.200000, 0.750000) {all};
\node[Kanji] at (36.250000, -0.500000) {\textcolor[HTML]{68a4bc}{遮}};
\node[Square] at (36.250000, -1.000000) {};
\node[Onyomi] at (36.300000, -0.900000) {\hbox{\tate シャ}};
\node[Kunyomi] at (36.200000, -0.900000) {\hbox{\tate さえぎ}};
\node[Meaning] at (36.250000, 0.750000) {intercept};
\node[Kanji] at (38.300000, -0.500000) {\textcolor[HTML]{d2a293}{席}};
\node[Square] at (38.300000, -1.000000) {};
\node[Onyomi] at (38.350000, -0.900000) {\hbox{\tate セキ}};
\node[Meaning] at (38.300000, 0.750000) {seat};
\node[Kanji] at (40.350000, -0.500000) {\textcolor[HTML]{c36143}{度}};
\node[Square] at (40.350000, -1.000000) {};
\node[Onyomi] at (40.400000, -0.900000) {\hbox{\tate ド・タク}};
\node[Kunyomi] at (40.300000, -0.900000) {\hbox{\tate たび}};
\node[Meaning] at (40.350000, 0.750000) {degrees};
\node[Kanji] at (42.400000, -0.500000) {\textcolor[HTML]{d69f8d}{渡}};
\node[Square] at (42.400000, -1.000000) {};
\node[Onyomi] at (42.450000, -0.900000) {\hbox{\tate ト}};
\node[Kunyomi] at (42.350000, -0.900000) {\hbox{\tate わた}};
\node[Meaning] at (42.400000, 0.750000) {transit};
\node[Kanji] at (44.450000, -0.500000) {\textcolor[HTML]{408dba}{奔}};
\node[Square] at (44.450000, -1.000000) {};
\node[Onyomi] at (44.500000, -0.900000) {\hbox{\tate ホン}};
\node[Kunyomi] at (44.400000, -0.900000) {\hbox{\tate はし.る}};
\node[Meaning] at (44.450000, 0.750000) {run};
\node[Kanji] at (46.500000, -0.500000) {\textcolor[HTML]{a3bac2}{噴}};
\node[Square] at (46.500000, -1.000000) {};
\node[Onyomi] at (46.550000, -0.900000) {\hbox{\tate フン}};
\node[Kunyomi] at (46.450000, -0.900000) {\hbox{\tate ふ}};
\node[Meaning] at (46.500000, 0.750000) {erupt};
\node[Kanji] at (48.550000, -0.500000) {\textcolor[HTML]{c8a59d}{墳}};
\node[Square] at (48.550000, -1.000000) {};
\node[Onyomi] at (48.600000, -0.900000) {\hbox{\tate フン}};
\node[Meaning] at (48.550000, 0.750000) {tomb};
\node[Kanji] at (50.600000, -0.500000) {\textcolor[HTML]{1e76bb}{憤}};
\node[Square] at (50.600000, -1.000000) {};
\node[Onyomi] at (50.650000, -0.900000) {\hbox{\tate フン}};
\node[Kunyomi] at (50.550000, -0.900000) {\hbox{\tate いきどお}};
\node[Meaning] at (50.600000, 0.750000) {resent};
\node[Kanji] at (52.650000, -0.500000) {\textcolor[HTML]{c8a59d}{焼}};
\node[Square] at (52.650000, -1.000000) {};
\node[Onyomi] at (52.700000, -0.900000) {\hbox{\tate ショウ}};
\node[Kunyomi] at (52.600000, -0.900000) {\hbox{\tate や}};
\node[Meaning] at (52.650000, 0.750000) {bake};
\node[Kanji] at (54.700000, -0.500000) {\textcolor[HTML]{408dba}{暁}};
\node[Square] at (54.700000, -1.000000) {};
\node[Onyomi] at (54.750000, -0.900000) {\hbox{\tate キョウ}};
\node[Kunyomi] at (54.650000, -0.900000) {\hbox{\tate あかつき}};
\node[Meaning] at (54.700000, 0.750000) {dawn};
\node[Kanji] at (56.750000, -0.500000) {\textcolor[HTML]{d69f8d}{半}};
\node[Square] at (56.750000, -1.000000) {};
\node[Onyomi] at (56.800000, -0.900000) {\hbox{\tate ハン}};
\node[Kunyomi] at (56.700000, -0.900000) {\hbox{\tate なか.ば}};
\node[Meaning] at (56.750000, 0.750000) {half};
\node[Meaning] at (-58.050000, -0.400000) {66.83\%};
\node[Kanji] at (-56.000000, -2.550000) {\textcolor[HTML]{d2a293}{伴}};
\node[Square] at (-56.000000, -3.050000) {};
\node[Onyomi] at (-55.950000, -2.950000) {\hbox{\tate ハン}};
\node[Kunyomi] at (-56.050000, -2.950000) {\hbox{\tate ともな.う}};
\node[Meaning] at (-56.000000, -1.300000) {accompany};
\node[Kanji] at (-53.950000, -2.550000) {\textcolor[HTML]{408dba}{畔}};
\node[Square] at (-53.950000, -3.050000) {};
\node[Onyomi] at (-53.900000, -2.950000) {\hbox{\tate ハン}};
\node[Kunyomi] at (-54.000000, -2.950000) {\hbox{\tate あぜ・くろ}};
\node[Meaning] at (-53.950000, -1.300000) {shore};
\node[Kanji] at (-51.900000, -2.550000) {\textcolor[HTML]{d69f8d}{判}};
\node[Square] at (-51.900000, -3.050000) {};
\node[Onyomi] at (-51.850000, -2.950000) {\hbox{\tate ハン}};
\node[Meaning] at (-51.900000, -1.300000) {judge};
\node[Kanji] at (-49.850000, -2.550000) {\textcolor[HTML]{b0b0b5}{券}};
\node[Square] at (-49.850000, -3.050000) {};
\node[Onyomi] at (-49.800000, -2.950000) {\hbox{\tate ケン}};
\node[Meaning] at (-49.850000, -1.300000) {ticket};
\node[Kanji] at (-47.800000, -2.550000) {\textcolor[HTML]{d69f8d}{巻}};
\node[Square] at (-47.800000, -3.050000) {};
\node[Onyomi] at (-47.750000, -2.950000) {\hbox{\tate カン}};
\node[Kunyomi] at (-47.850000, -2.950000) {\hbox{\tate ま.く}};
\node[Meaning] at (-47.800000, -1.300000) {scroll};
\node[Kanji] at (-45.750000, -2.550000) {\textcolor[HTML]{b0b0b5}{圏}};
\node[Square] at (-45.750000, -3.050000) {};
\node[Onyomi] at (-45.700000, -2.950000) {\hbox{\tate ケン}};
\node[Meaning] at (-45.750000, -1.300000) {range};
\node[Kanji] at (-43.700000, -2.550000) {\textcolor[HTML]{cd8268}{勝}};
\node[Square] at (-43.700000, -3.050000) {};
\node[Onyomi] at (-43.650000, -2.950000) {\hbox{\tate ショウ}};
\node[Kunyomi] at (-43.750000, -2.950000) {\hbox{\tate か.つ}};
\node[Meaning] at (-43.700000, -1.300000) {win};
\node[Kanji] at (-41.650000, -2.550000) {\textcolor[HTML]{d69f8d}{藤}};
\node[Square] at (-41.650000, -3.050000) {};
\node[Onyomi] at (-41.600000, -2.950000) {\hbox{\tate トウ・ドウ}};
\node[Kunyomi] at (-41.700000, -2.950000) {\hbox{\tate ふじ}};
\node[Meaning] at (-41.650000, -1.300000) {wisteria};
\node[Kanji] at (-39.600000, -2.550000) {\textcolor[HTML]{181c43}{謄}};
\node[Square] at (-39.600000, -3.050000) {};
\node[Onyomi] at (-39.550000, -2.950000) {\hbox{\tate トウ}};
\node[Meaning] at (-39.600000, -1.300000) {mimeograph};
\node[Kanji] at (-37.550000, -2.550000) {\textcolor[HTML]{c8a59d}{片}};
\node[Square] at (-37.550000, -3.050000) {};
\node[Onyomi] at (-37.500000, -2.950000) {\hbox{\tate ヘン}};
\node[Kunyomi] at (-37.600000, -2.950000) {\hbox{\tate かた}};
\node[Meaning] at (-37.550000, -1.300000) {one sided};
\node[Kanji] at (-35.500000, -2.550000) {\textcolor[HTML]{cd8268}{版}};
\node[Square] at (-35.500000, -3.050000) {};
\node[Onyomi] at (-35.450000, -2.950000) {\hbox{\tate ハン}};
\node[Meaning] at (-35.500000, -1.300000) {edition};
\node[Kanji] at (-33.450000, -2.550000) {\textcolor[HTML]{d2a293}{之}};
\node[Square] at (-33.450000, -3.050000) {};
\node[Onyomi] at (-33.400000, -2.950000) {\hbox{\tate シ}};
\node[Kunyomi] at (-33.500000, -2.950000) {\hbox{\tate これ}};
\node[Meaning] at (-33.450000, -1.300000) {this};
\node[Kanji] at (-31.400000, -2.550000) {\textcolor[HTML]{68a4bc}{乏}};
\node[Square] at (-31.400000, -3.050000) {};
\node[Onyomi] at (-31.350000, -2.950000) {\hbox{\tate ボウ}};
\node[Kunyomi] at (-31.450000, -2.950000) {\hbox{\tate とぼ.しい}};
\node[Meaning] at (-31.400000, -1.300000) {scarce};
\node[Kanji] at (-29.350000, -2.550000) {\textcolor[HTML]{b0b0b5}{芝}};
\node[Square] at (-29.350000, -3.050000) {};
\node[Onyomi] at (-29.300000, -2.950000) {\hbox{\tate シ}};
\node[Kunyomi] at (-29.400000, -2.950000) {\hbox{\tate しば}};
\node[Meaning] at (-29.350000, -1.300000) {lawn};
\node[Kanji] at (-27.300000, -2.550000) {\textcolor[HTML]{c36143}{不}};
\node[Square] at (-27.300000, -3.050000) {};
\node[Onyomi] at (-27.250000, -2.950000) {\hbox{\tate フ}};
\node[Meaning] at (-27.300000, -1.300000) {not};
\node[Kanji] at (-25.250000, -2.550000) {\textcolor[HTML]{c8a59d}{否}};
\node[Square] at (-25.250000, -3.050000) {};
\node[Onyomi] at (-25.200000, -2.950000) {\hbox{\tate ヒ}};
\node[Kunyomi] at (-25.300000, -2.950000) {\hbox{\tate いな・いや}};
\node[Meaning] at (-25.250000, -1.300000) {no};
\node[Kanji] at (-23.200000, -2.550000) {\textcolor[HTML]{a3bac2}{杯}};
\node[Square] at (-23.200000, -3.050000) {};
\node[Onyomi] at (-23.150000, -2.950000) {\hbox{\tate ハイ}};
\node[Meaning] at (-23.200000, -1.300000) {cup of liquid};
\node[Kanji] at (-21.150000, -2.550000) {\textcolor[HTML]{b0b0b5}{矢}};
\node[Square] at (-21.150000, -3.050000) {};
\node[Onyomi] at (-21.100000, -2.950000) {\hbox{\tate シ}};
\node[Kunyomi] at (-21.200000, -2.950000) {\hbox{\tate や}};
\node[Meaning] at (-21.150000, -1.300000) {arrow};
\node[Kanji] at (-19.100000, -2.550000) {\textcolor[HTML]{1059be}{矯}};
\node[Square] at (-19.100000, -3.050000) {};
\node[Onyomi] at (-19.050000, -2.950000) {\hbox{\tate キョウ}};
\node[Kunyomi] at (-19.150000, -2.950000) {\hbox{\tate た}};
\node[Meaning] at (-19.100000, -1.300000) {correct};
\node[Kanji] at (-17.050000, -2.550000) {\textcolor[HTML]{d69f8d}{族}};
\node[Square] at (-17.050000, -3.050000) {};
\node[Onyomi] at (-17.000000, -2.950000) {\hbox{\tate ゾク}};
\node[Meaning] at (-17.050000, -1.300000) {tribe};
\node[Kanji] at (-15.000000, -2.550000) {\textcolor[HTML]{c36143}{知}};
\node[Square] at (-15.000000, -3.050000) {};
\node[Onyomi] at (-14.950000, -2.950000) {\hbox{\tate チ}};
\node[Kunyomi] at (-15.050000, -2.950000) {\hbox{\tate し.る}};
\node[Meaning] at (-15.000000, -1.300000) {know};
\node[Kanji] at (-12.950000, -2.550000) {\textcolor[HTML]{b0b0b5}{智}};
\node[Square] at (-12.950000, -3.050000) {};
\node[Onyomi] at (-12.900000, -2.950000) {\hbox{\tate チ}};
\node[Meaning] at (-12.950000, -1.300000) {wisdom};
\node[Kanji] at (-10.900000, -2.550000) {\textcolor[HTML]{68a4bc}{矛}};
\node[Square] at (-10.900000, -3.050000) {};
\node[Onyomi] at (-10.850000, -2.950000) {\hbox{\tate ム}};
\node[Kunyomi] at (-10.950000, -2.950000) {\hbox{\tate ほこ}};
\node[Meaning] at (-10.900000, -1.300000) {halberd};
\node[Kanji] at (-8.850000, -2.550000) {\textcolor[HTML]{b0b0b5}{柔}};
\node[Square] at (-8.850000, -3.050000) {};
\node[Onyomi] at (-8.800000, -2.950000) {\hbox{\tate ジュウ}};
\node[Kunyomi] at (-8.900000, -2.950000) {\hbox{\tate やわ.*}};
\node[Meaning] at (-8.850000, -1.300000) {gentle};
\node[Kanji] at (-6.800000, -2.550000) {\textcolor[HTML]{c36143}{務}};
\node[Square] at (-6.800000, -3.050000) {};
\node[Onyomi] at (-6.750000, -2.950000) {\hbox{\tate ム}};
\node[Kunyomi] at (-6.850000, -2.950000) {\hbox{\tate つと.める}};
\node[Meaning] at (-6.800000, -1.300000) {task};
\node[Kanji] at (-4.750000, -2.550000) {\textcolor[HTML]{91b7c3}{霧}};
\node[Square] at (-4.750000, -3.050000) {};
\node[Onyomi] at (-4.700000, -2.950000) {\hbox{\tate ム}};
\node[Kunyomi] at (-4.800000, -2.950000) {\hbox{\tate きり}};
\node[Meaning] at (-4.750000, -1.300000) {fog};
\node[Kanji] at (-2.700000, -2.550000) {\textcolor[HTML]{91b7c3}{班}};
\node[Square] at (-2.700000, -3.050000) {};
\node[Onyomi] at (-2.650000, -2.950000) {\hbox{\tate ハン}};
\node[Meaning] at (-2.700000, -1.300000) {squad};
\node[Kanji] at (-0.650000, -2.550000) {\textcolor[HTML]{d2a293}{帰}};
\node[Square] at (-0.650000, -3.050000) {};
\node[Onyomi] at (-0.600000, -2.950000) {\hbox{\tate キ}};
\node[Kunyomi] at (-0.700000, -2.950000) {\hbox{\tate かえ}};
\node[Meaning] at (-0.650000, -1.300000) {return};
\node[Kanji] at (1.400000, -2.550000) {\textcolor[HTML]{91b7c3}{弓}};
\node[Square] at (1.400000, -3.050000) {};
\node[Onyomi] at (1.450000, -2.950000) {\hbox{\tate キュウ}};
\node[Kunyomi] at (1.350000, -2.950000) {\hbox{\tate ゆみ}};
\node[Meaning] at (1.400000, -1.300000) {bow};
\node[Kanji] at (3.450000, -2.550000) {\textcolor[HTML]{cd8268}{引}};
\node[Square] at (3.450000, -3.050000) {};
\node[Onyomi] at (3.500000, -2.950000) {\hbox{\tate イン}};
\node[Kunyomi] at (3.400000, -2.950000) {\hbox{\tate ひ}};
\node[Meaning] at (3.450000, -1.300000) {pull};
\node[Kanji] at (5.500000, -2.550000) {\textcolor[HTML]{1e76bb}{弔}};
\node[Square] at (5.500000, -3.050000) {};
\node[Onyomi] at (5.550000, -2.950000) {\hbox{\tate チョウ}};
\node[Kunyomi] at (5.450000, -2.950000) {\hbox{\tate とぶら.う}};
\node[Meaning] at (5.500000, -1.300000) {condolence};
\node[Kanji] at (7.550000, -2.550000) {\textcolor[HTML]{cd8268}{強}};
\node[Square] at (7.550000, -3.050000) {};
\node[Onyomi] at (7.600000, -2.950000) {\hbox{\tate キョウ}};
\node[Kunyomi] at (7.500000, -2.950000) {\hbox{\tate つよ.い}};
\node[Meaning] at (7.550000, -1.300000) {strong};
\node[Kanji] at (9.600000, -2.550000) {\textcolor[HTML]{b0b0b5}{弱}};
\node[Square] at (9.600000, -3.050000) {};
\node[Onyomi] at (9.650000, -2.950000) {\hbox{\tate ジャク}};
\node[Kunyomi] at (9.550000, -2.950000) {\hbox{\tate よわ.い}};
\node[Meaning] at (9.600000, -1.300000) {weak};
\node[Kanji] at (11.650000, -2.550000) {\textcolor[HTML]{408dba}{沸}};
\node[Square] at (11.650000, -3.050000) {};
\node[Onyomi] at (11.700000, -2.950000) {\hbox{\tate フツ}};
\node[Kunyomi] at (11.600000, -2.950000) {\hbox{\tate わ}};
\node[Meaning] at (11.650000, -1.300000) {boil};
\node[Kanji] at (13.700000, -2.550000) {\textcolor[HTML]{d2a293}{費}};
\node[Square] at (13.700000, -3.050000) {};
\node[Onyomi] at (13.750000, -2.950000) {\hbox{\tate ヒ}};
\node[Kunyomi] at (13.650000, -2.950000) {\hbox{\tate つい.やす}};
\node[Meaning] at (13.700000, -1.300000) {expense};
\node[Kanji] at (15.750000, -2.550000) {\textcolor[HTML]{b74029}{第}};
\node[Square] at (15.750000, -3.050000) {};
\node[Onyomi] at (15.800000, -2.950000) {\hbox{\tate ダイ}};
\node[Meaning] at (15.750000, -1.300000) {No.};
\node[Kanji] at (17.800000, -2.550000) {\textcolor[HTML]{d2a293}{弟}};
\node[Square] at (17.800000, -3.050000) {};
\node[Onyomi] at (17.850000, -2.950000) {\hbox{\tate ダイ・デ}};
\node[Kunyomi] at (17.750000, -2.950000) {\hbox{\tate おとうと}};
\node[Meaning] at (17.800000, -1.300000) {younger brother};
\node[Kanji] at (19.850000, -2.550000) {\textcolor[HTML]{68a4bc}{巧}};
\node[Square] at (19.850000, -3.050000) {};
\node[Onyomi] at (19.900000, -2.950000) {\hbox{\tate コウ}};
\node[Kunyomi] at (19.800000, -2.950000) {\hbox{\tate うま.い}};
\node[Meaning] at (19.850000, -1.300000) {adept};
\node[Kanji] at (21.900000, -2.550000) {\textcolor[HTML]{c36143}{号}};
\node[Square] at (21.900000, -3.050000) {};
\node[Onyomi] at (21.950000, -2.950000) {\hbox{\tate ゴウ}};
\node[Meaning] at (21.900000, -1.300000) {number};
\node[Kanji] at (23.950000, -2.550000) {\textcolor[HTML]{68a4bc}{朽}};
\node[Square] at (23.950000, -3.050000) {};
\node[Onyomi] at (24.000000, -2.950000) {\hbox{\tate キュウ}};
\node[Kunyomi] at (23.900000, -2.950000) {\hbox{\tate く}};
\node[Meaning] at (23.950000, -1.300000) {rot};
\node[Kanji] at (26.000000, -2.550000) {\textcolor[HTML]{91b7c3}{誇}};
\node[Square] at (26.000000, -3.050000) {};
\node[Onyomi] at (26.050000, -2.950000) {\hbox{\tate コ}};
\node[Kunyomi] at (25.950000, -2.950000) {\hbox{\tate ほこ.る}};
\node[Meaning] at (26.000000, -1.300000) {pride};
\node[Kanji] at (28.050000, -2.550000) {\textcolor[HTML]{91b7c3}{汚}};
\node[Square] at (28.050000, -3.050000) {};
\node[Onyomi] at (28.100000, -2.950000) {\hbox{\tate オ}};
\node[Kunyomi] at (28.000000, -2.950000) {\hbox{\tate よご・きたな}};
\node[Meaning] at (28.050000, -1.300000) {dirty};
\node[Kanji] at (30.100000, -2.550000) {\textcolor[HTML]{d69f8d}{与}};
\node[Square] at (30.100000, -3.050000) {};
\node[Onyomi] at (30.150000, -2.950000) {\hbox{\tate ヨ}};
\node[Kunyomi] at (30.050000, -2.950000) {\hbox{\tate あた.える}};
\node[Meaning] at (30.100000, -1.300000) {give};
\node[Kanji] at (32.150000, -2.550000) {\textcolor[HTML]{d2a293}{写}};
\node[Square] at (32.150000, -3.050000) {};
\node[Onyomi] at (32.200000, -2.950000) {\hbox{\tate シャ}};
\node[Kunyomi] at (32.100000, -2.950000) {\hbox{\tate うつ.す}};
\node[Meaning] at (32.150000, -1.300000) {copy};
\node[Kanji] at (34.200000, -2.550000) {\textcolor[HTML]{cd8268}{身}};
\node[Square] at (34.200000, -3.050000) {};
\node[Onyomi] at (34.250000, -2.950000) {\hbox{\tate シン}};
\node[Kunyomi] at (34.150000, -2.950000) {\hbox{\tate み}};
\node[Meaning] at (34.200000, -1.300000) {somebody};
\node[Kanji] at (36.250000, -2.550000) {\textcolor[HTML]{d2a293}{射}};
\node[Square] at (36.250000, -3.050000) {};
\node[Onyomi] at (36.300000, -2.950000) {\hbox{\tate シャ}};
\node[Kunyomi] at (36.200000, -2.950000) {\hbox{\tate い.る}};
\node[Meaning] at (36.250000, -1.300000) {shoot};
\node[Kanji] at (38.300000, -2.550000) {\textcolor[HTML]{a3bac2}{謝}};
\node[Square] at (38.300000, -3.050000) {};
\node[Onyomi] at (38.350000, -2.950000) {\hbox{\tate シャ}};
\node[Kunyomi] at (38.250000, -2.950000) {\hbox{\tate あやま.る}};
\node[Meaning] at (38.300000, -1.300000) {apologize};
\node[Kanji] at (40.350000, -2.550000) {\textcolor[HTML]{c8a59d}{老}};
\node[Square] at (40.350000, -3.050000) {};
\node[Onyomi] at (40.400000, -2.950000) {\hbox{\tate ロウ}};
\node[Meaning] at (40.350000, -1.300000) {elderly};
\node[Kanji] at (42.400000, -2.550000) {\textcolor[HTML]{c36143}{考}};
\node[Square] at (42.400000, -3.050000) {};
\node[Onyomi] at (42.450000, -2.950000) {\hbox{\tate コウ}};
\node[Kunyomi] at (42.350000, -2.950000) {\hbox{\tate かんが}};
\node[Meaning] at (42.400000, -1.300000) {think};
\node[Kanji] at (44.450000, -2.550000) {\textcolor[HTML]{a3bac2}{孝}};
\node[Square] at (44.450000, -3.050000) {};
\node[Onyomi] at (44.500000, -2.950000) {\hbox{\tate コウ}};
\node[Meaning] at (44.450000, -1.300000) {filial piety};
\node[Kanji] at (46.500000, -2.550000) {\textcolor[HTML]{c36143}{教}};
\node[Square] at (46.500000, -3.050000) {};
\node[Onyomi] at (46.550000, -2.950000) {\hbox{\tate キョウ}};
\node[Kunyomi] at (46.450000, -2.950000) {\hbox{\tate おし.える}};
\node[Meaning] at (46.500000, -1.300000) {teach};
\node[Kanji] at (48.550000, -2.550000) {\textcolor[HTML]{1e76bb}{拷}};
\node[Square] at (48.550000, -3.050000) {};
\node[Onyomi] at (48.600000, -2.950000) {\hbox{\tate ゴウ}};
\node[Meaning] at (48.550000, -1.300000) {torture};
\node[Kanji] at (50.600000, -2.550000) {\textcolor[HTML]{a11d25}{者}};
\node[Square] at (50.600000, -3.050000) {};
\node[Onyomi] at (50.650000, -2.950000) {\hbox{\tate シャ}};
\node[Kunyomi] at (50.550000, -2.950000) {\hbox{\tate もの}};
\node[Meaning] at (50.600000, -1.300000) {someone};
\node[Kanji] at (52.650000, -2.550000) {\textcolor[HTML]{68a4bc}{煮}};
\node[Square] at (52.650000, -3.050000) {};
\node[Onyomi] at (52.700000, -2.950000) {\hbox{\tate シャ}};
\node[Kunyomi] at (52.600000, -2.950000) {\hbox{\tate に}};
\node[Meaning] at (52.650000, -1.300000) {boil};
\node[Kanji] at (54.700000, -2.550000) {\textcolor[HTML]{d69f8d}{著}};
\node[Square] at (54.700000, -3.050000) {};
\node[Onyomi] at (54.750000, -2.950000) {\hbox{\tate チョ}};
\node[Kunyomi] at (54.650000, -2.950000) {\hbox{\tate }};
\node[Meaning] at (54.700000, -1.300000) {author};
\node[Kanji] at (56.750000, -2.550000) {\textcolor[HTML]{b0b0b5}{署}};
\node[Square] at (56.750000, -3.050000) {};
\node[Onyomi] at (56.800000, -2.950000) {\hbox{\tate ショ}};
\node[Meaning] at (56.750000, -1.300000) {govt. office};
\node[Meaning] at (-58.050000, -2.450000) {69.85\%};
\node[Kanji] at (-56.000000, -4.600000) {\textcolor[HTML]{408dba}{暑}};
\node[Square] at (-56.000000, -5.100000) {};
\node[Onyomi] at (-55.950000, -5.000000) {\hbox{\tate ショ}};
\node[Kunyomi] at (-56.050000, -5.000000) {\hbox{\tate あつ}};
\node[Meaning] at (-56.000000, -3.350000) {hot};
\node[Kanji] at (-53.950000, -4.600000) {\textcolor[HTML]{d2a293}{諸}};
\node[Square] at (-53.950000, -5.100000) {};
\node[Onyomi] at (-53.900000, -5.000000) {\hbox{\tate ショ}};
\node[Kunyomi] at (-54.000000, -5.000000) {\hbox{\tate もろ}};
\node[Meaning] at (-53.950000, -3.350000) {various};
\node[Kanji] at (-51.900000, -4.600000) {\textcolor[HTML]{408dba}{賭}};
\node[Square] at (-51.900000, -5.100000) {};
\node[Kunyomi] at (-51.950000, -5.000000) {\hbox{\tate か}};
\node[Meaning] at (-51.900000, -3.350000) {gamble};
\node[Kanji] at (-49.850000, -4.600000) {\textcolor[HTML]{91b7c3}{峡}};
\node[Square] at (-49.850000, -5.100000) {};
\node[Onyomi] at (-49.800000, -5.000000) {\hbox{\tate キョウ・コウ}};
\node[Kunyomi] at (-49.900000, -5.000000) {\hbox{\tate はざま}};
\node[Meaning] at (-49.850000, -3.350000) {ravine};
\node[Kanji] at (-47.800000, -4.600000) {\textcolor[HTML]{b0b0b5}{狭}};
\node[Square] at (-47.800000, -5.100000) {};
\node[Onyomi] at (-47.750000, -5.000000) {\hbox{\tate キョウ}};
\node[Kunyomi] at (-47.850000, -5.000000) {\hbox{\tate せま・せば}};
\node[Meaning] at (-47.800000, -3.350000) {narrow};
\node[Kanji] at (-45.750000, -4.600000) {\textcolor[HTML]{a3bac2}{挟}};
\node[Square] at (-45.750000, -5.100000) {};
\node[Onyomi] at (-45.700000, -5.000000) {\hbox{\tate キョウ}};
\node[Kunyomi] at (-45.800000, -5.000000) {\hbox{\tate はさ}};
\node[Meaning] at (-45.750000, -3.350000) {between};
\node[Kanji] at (-43.700000, -4.600000) {\textcolor[HTML]{d69f8d}{追}};
\node[Square] at (-43.700000, -5.100000) {};
\node[Onyomi] at (-43.650000, -5.000000) {\hbox{\tate ツイ}};
\node[Kunyomi] at (-43.750000, -5.000000) {\hbox{\tate お}};
\node[Meaning] at (-43.700000, -3.350000) {follow};
\node[Kanji] at (-41.650000, -4.600000) {\textcolor[HTML]{d69f8d}{師}};
\node[Square] at (-41.650000, -5.100000) {};
\node[Onyomi] at (-41.600000, -5.000000) {\hbox{\tate シ}};
\node[Meaning] at (-41.650000, -3.350000) {teacher};
\node[Kanji] at (-39.600000, -4.600000) {\textcolor[HTML]{68a4bc}{帥}};
\node[Square] at (-39.600000, -5.100000) {};
\node[Onyomi] at (-39.550000, -5.000000) {\hbox{\tate スイ}};
\node[Meaning] at (-39.600000, -3.350000) {commander};
\node[Kanji] at (-37.550000, -4.600000) {\textcolor[HTML]{cd8268}{官}};
\node[Square] at (-37.550000, -5.100000) {};
\node[Onyomi] at (-37.500000, -5.000000) {\hbox{\tate カン}};
\node[Meaning] at (-37.550000, -3.350000) {government};
\node[Kanji] at (-35.500000, -4.600000) {\textcolor[HTML]{68a4bc}{棺}};
\node[Square] at (-35.500000, -5.100000) {};
\node[Onyomi] at (-35.450000, -5.000000) {\hbox{\tate カン}};
\node[Meaning] at (-35.500000, -3.350000) {coffin};
\node[Kanji] at (-33.450000, -4.600000) {\textcolor[HTML]{d69f8d}{管}};
\node[Square] at (-33.450000, -5.100000) {};
\node[Onyomi] at (-33.400000, -5.000000) {\hbox{\tate カン}};
\node[Kunyomi] at (-33.500000, -5.000000) {\hbox{\tate くだ}};
\node[Meaning] at (-33.450000, -3.350000) {pipe};
\node[Kanji] at (-31.400000, -4.600000) {\textcolor[HTML]{d69f8d}{父}};
\node[Square] at (-31.400000, -5.100000) {};
\node[Onyomi] at (-31.350000, -5.000000) {\hbox{\tate フ}};
\node[Kunyomi] at (-31.450000, -5.000000) {\hbox{\tate ちち}};
\node[Meaning] at (-31.400000, -3.350000) {father};
\node[Kanji] at (-29.350000, -4.600000) {\textcolor[HTML]{cd8268}{交}};
\node[Square] at (-29.350000, -5.100000) {};
\node[Onyomi] at (-29.300000, -5.000000) {\hbox{\tate コウ}};
\node[Kunyomi] at (-29.400000, -5.000000) {\hbox{\tate まじ・ま・か}};
\node[Meaning] at (-29.350000, -3.350000) {mix};
\node[Kanji] at (-27.300000, -4.600000) {\textcolor[HTML]{d2a293}{効}};
\node[Square] at (-27.300000, -5.100000) {};
\node[Onyomi] at (-27.250000, -5.000000) {\hbox{\tate コウ}};
\node[Kunyomi] at (-27.350000, -5.000000) {\hbox{\tate き.く}};
\node[Meaning] at (-27.300000, -3.350000) {effective};
\node[Kanji] at (-25.250000, -4.600000) {\textcolor[HTML]{c8a59d}{較}};
\node[Square] at (-25.250000, -5.100000) {};
\node[Onyomi] at (-25.200000, -5.000000) {\hbox{\tate カク}};
\node[Meaning] at (-25.250000, -3.350000) {contrast};
\node[Kanji] at (-23.200000, -4.600000) {\textcolor[HTML]{b74029}{校}};
\node[Square] at (-23.200000, -5.100000) {};
\node[Onyomi] at (-23.150000, -5.000000) {\hbox{\tate コウ}};
\node[Meaning] at (-23.200000, -3.350000) {school};
\node[Kanji] at (-21.150000, -4.600000) {\textcolor[HTML]{d69f8d}{足}};
\node[Square] at (-21.150000, -5.100000) {};
\node[Onyomi] at (-21.100000, -5.000000) {\hbox{\tate ソク}};
\node[Kunyomi] at (-21.200000, -5.000000) {\hbox{\tate あし}};
\node[Meaning] at (-21.150000, -3.350000) {foot};
\node[Kanji] at (-19.100000, -4.600000) {\textcolor[HTML]{a3bac2}{促}};
\node[Square] at (-19.100000, -5.100000) {};
\node[Onyomi] at (-19.050000, -5.000000) {\hbox{\tate ソク}};
\node[Kunyomi] at (-19.150000, -5.000000) {\hbox{\tate うなが.す}};
\node[Meaning] at (-19.100000, -3.350000) {urge};
\node[Kanji] at (-17.050000, -4.600000) {\textcolor[HTML]{c8a59d}{距}};
\node[Square] at (-17.050000, -5.100000) {};
\node[Onyomi] at (-17.000000, -5.000000) {\hbox{\tate キョ}};
\node[Meaning] at (-17.050000, -3.350000) {distance};
\node[Kanji] at (-15.000000, -4.600000) {\textcolor[HTML]{cd8268}{路}};
\node[Square] at (-15.000000, -5.100000) {};
\node[Onyomi] at (-14.950000, -5.000000) {\hbox{\tate ロ}};
\node[Kunyomi] at (-15.050000, -5.000000) {\hbox{\tate じ・みち}};
\node[Meaning] at (-15.000000, -3.350000) {road};
\node[Kanji] at (-12.950000, -4.600000) {\textcolor[HTML]{c8a59d}{露}};
\node[Square] at (-12.950000, -5.100000) {};
\node[Onyomi] at (-12.900000, -5.000000) {\hbox{\tate ロ・ロウ}};
\node[Kunyomi] at (-13.000000, -5.000000) {\hbox{\tate つゆ}};
\node[Meaning] at (-12.950000, -3.350000) {expose};
\node[Kanji] at (-10.900000, -4.600000) {\textcolor[HTML]{68a4bc}{跳}};
\node[Square] at (-10.900000, -5.100000) {};
\node[Onyomi] at (-10.850000, -5.000000) {\hbox{\tate チョウ}};
\node[Kunyomi] at (-10.950000, -5.000000) {\hbox{\tate と.ぶ}};
\node[Meaning] at (-10.900000, -3.350000) {hop};
\node[Kanji] at (-8.850000, -4.600000) {\textcolor[HTML]{c8a59d}{躍}};
\node[Square] at (-8.850000, -5.100000) {};
\node[Onyomi] at (-8.800000, -5.000000) {\hbox{\tate ヤク}};
\node[Kunyomi] at (-8.900000, -5.000000) {\hbox{\tate おど.る}};
\node[Meaning] at (-8.850000, -3.350000) {leap};
\node[Kanji] at (-6.800000, -4.600000) {\textcolor[HTML]{68a4bc}{践}};
\node[Square] at (-6.800000, -5.100000) {};
\node[Onyomi] at (-6.750000, -5.000000) {\hbox{\tate セン}};
\node[Kunyomi] at (-6.850000, -5.000000) {\hbox{\tate ふ}};
\node[Meaning] at (-6.800000, -3.350000) {practice};
\node[Kanji] at (-4.750000, -4.600000) {\textcolor[HTML]{b0b0b5}{踏}};
\node[Square] at (-4.750000, -5.100000) {};
\node[Onyomi] at (-4.700000, -5.000000) {\hbox{\tate トウ}};
\node[Kunyomi] at (-4.800000, -5.000000) {\hbox{\tate ふ.む}};
\node[Meaning] at (-4.750000, -3.350000) {step};
\node[Kanji] at (-2.700000, -4.600000) {\textcolor[HTML]{c8a59d}{骨}};
\node[Square] at (-2.700000, -5.100000) {};
\node[Onyomi] at (-2.650000, -5.000000) {\hbox{\tate コツ}};
\node[Kunyomi] at (-2.750000, -5.000000) {\hbox{\tate ほね}};
\node[Meaning] at (-2.700000, -3.350000) {bone};
\node[Kanji] at (-0.650000, -4.600000) {\textcolor[HTML]{a3bac2}{滑}};
\node[Square] at (-0.650000, -5.100000) {};
\node[Onyomi] at (-0.600000, -5.000000) {\hbox{\tate カツ}};
\node[Kunyomi] at (-0.700000, -5.000000) {\hbox{\tate すべ.る}};
\node[Meaning] at (-0.650000, -3.350000) {slippery};
\node[Kanji] at (1.400000, -4.600000) {\textcolor[HTML]{68a4bc}{髄}};
\node[Square] at (1.400000, -5.100000) {};
\node[Onyomi] at (1.450000, -5.000000) {\hbox{\tate ズイ}};
\node[Meaning] at (1.400000, -3.350000) {marrow};
\node[Kanji] at (3.450000, -4.600000) {\textcolor[HTML]{1059be}{禍}};
\node[Square] at (3.450000, -5.100000) {};
\node[Onyomi] at (3.500000, -5.000000) {\hbox{\tate カ}};
\node[Kunyomi] at (3.400000, -5.000000) {\hbox{\tate わざわい}};
\node[Meaning] at (3.450000, -3.350000) {evil};
\node[Kanji] at (5.500000, -4.600000) {\textcolor[HTML]{408dba}{渦}};
\node[Square] at (5.500000, -5.100000) {};
\node[Onyomi] at (5.550000, -5.000000) {\hbox{\tate カ}};
\node[Kunyomi] at (5.450000, -5.000000) {\hbox{\tate うず}};
\node[Meaning] at (5.500000, -3.350000) {whirlpool};
\node[Kanji] at (7.550000, -4.600000) {\textcolor[HTML]{c36143}{過}};
\node[Square] at (7.550000, -5.100000) {};
\node[Onyomi] at (7.600000, -5.000000) {\hbox{\tate カ}};
\node[Kunyomi] at (7.500000, -5.000000) {\hbox{\tate す.ぎ}};
\node[Meaning] at (7.550000, -3.350000) {surpass};
\node[Kanji] at (9.600000, -4.600000) {\textcolor[HTML]{d69f8d}{阪}};
\node[Square] at (9.600000, -5.100000) {};
\node[Onyomi] at (9.650000, -5.000000) {\hbox{\tate ハン}};
\node[Kunyomi] at (9.550000, -5.000000) {\hbox{\tate さか}};
\node[Meaning] at (9.600000, -3.350000) {heights};
\node[Kanji] at (11.650000, -4.600000) {\textcolor[HTML]{c8a59d}{阿}};
\node[Square] at (11.650000, -5.100000) {};
\node[Onyomi] at (11.700000, -5.000000) {\hbox{\tate ア・オ}};
\node[Kunyomi] at (11.600000, -5.000000) {\hbox{\tate おもね・くま}};
\node[Meaning] at (11.650000, -3.350000) {flatter};
\node[Kanji] at (13.700000, -4.600000) {\textcolor[HTML]{cd8268}{際}};
\node[Square] at (13.700000, -5.100000) {};
\node[Onyomi] at (13.750000, -5.000000) {\hbox{\tate サイ}};
\node[Kunyomi] at (13.650000, -5.000000) {\hbox{\tate きわ}};
\node[Meaning] at (13.700000, -3.350000) {occasion};
\node[Kanji] at (15.750000, -4.600000) {\textcolor[HTML]{c8a59d}{障}};
\node[Square] at (15.750000, -5.100000) {};
\node[Onyomi] at (15.800000, -5.000000) {\hbox{\tate ショウ}};
\node[Kunyomi] at (15.700000, -5.000000) {\hbox{\tate さわ.る}};
\node[Meaning] at (15.750000, -3.350000) {hinder};
\node[Kanji] at (17.800000, -4.600000) {\textcolor[HTML]{a3bac2}{随}};
\node[Square] at (17.800000, -5.100000) {};
\node[Onyomi] at (17.850000, -5.000000) {\hbox{\tate ズイ}};
\node[Kunyomi] at (17.750000, -5.000000) {\hbox{\tate したが.う}};
\node[Meaning] at (17.800000, -3.350000) {all};
\node[Kanji] at (19.850000, -4.600000) {\textcolor[HTML]{1e76bb}{陪}};
\node[Square] at (19.850000, -5.100000) {};
\node[Onyomi] at (19.900000, -5.000000) {\hbox{\tate バイ}};
\node[Meaning] at (19.850000, -3.350000) {accompany};
\node[Kanji] at (21.900000, -4.600000) {\textcolor[HTML]{d2a293}{陽}};
\node[Square] at (21.900000, -5.100000) {};
\node[Onyomi] at (21.950000, -5.000000) {\hbox{\tate ヨウ}};
\node[Kunyomi] at (21.850000, -5.000000) {\hbox{\tate ひ}};
\node[Meaning] at (21.900000, -3.350000) {sunshine};
\node[Kanji] at (23.950000, -4.600000) {\textcolor[HTML]{a3bac2}{陳}};
\node[Square] at (23.950000, -5.100000) {};
\node[Onyomi] at (24.000000, -5.000000) {\hbox{\tate チン}};
\node[Kunyomi] at (23.900000, -5.000000) {\hbox{\tate ひ.ねる}};
\node[Meaning] at (23.950000, -3.350000) {exhibit};
\node[Kanji] at (26.000000, -4.600000) {\textcolor[HTML]{d69f8d}{防}};
\node[Square] at (26.000000, -5.100000) {};
\node[Onyomi] at (26.050000, -5.000000) {\hbox{\tate ボウ}};
\node[Kunyomi] at (25.950000, -5.000000) {\hbox{\tate ふせ.ぐ}};
\node[Meaning] at (26.000000, -3.350000) {prevent};
\node[Kanji] at (28.050000, -4.600000) {\textcolor[HTML]{a3bac2}{附}};
\node[Square] at (28.050000, -5.100000) {};
\node[Onyomi] at (28.100000, -5.000000) {\hbox{\tate フ}};
\node[Meaning] at (28.050000, -3.350000) {affixed};
\node[Kanji] at (30.100000, -4.600000) {\textcolor[HTML]{cd8268}{院}};
\node[Square] at (30.100000, -5.100000) {};
\node[Onyomi] at (30.150000, -5.000000) {\hbox{\tate イン}};
\node[Meaning] at (30.100000, -3.350000) {institution};
\node[Kanji] at (32.150000, -4.600000) {\textcolor[HTML]{c8a59d}{陣}};
\node[Square] at (32.150000, -5.100000) {};
\node[Onyomi] at (32.200000, -5.000000) {\hbox{\tate ジン}};
\node[Meaning] at (32.150000, -3.350000) {army base};
\node[Kanji] at (34.200000, -4.600000) {\textcolor[HTML]{cd8268}{隊}};
\node[Square] at (34.200000, -5.100000) {};
\node[Onyomi] at (34.250000, -5.000000) {\hbox{\tate タイ}};
\node[Meaning] at (34.200000, -3.350000) {squad};
\node[Kanji] at (36.250000, -4.600000) {\textcolor[HTML]{68a4bc}{墜}};
\node[Square] at (36.250000, -5.100000) {};
\node[Onyomi] at (36.300000, -5.000000) {\hbox{\tate ツイ}};
\node[Meaning] at (36.250000, -3.350000) {crash};
\node[Kanji] at (38.300000, -4.600000) {\textcolor[HTML]{d69f8d}{降}};
\node[Square] at (38.300000, -5.100000) {};
\node[Onyomi] at (38.350000, -5.000000) {\hbox{\tate コウ}};
\node[Kunyomi] at (38.250000, -5.000000) {\hbox{\tate お.りる}};
\node[Meaning] at (38.300000, -3.350000) {descend};
\node[Kanji] at (40.350000, -4.600000) {\textcolor[HTML]{d2a293}{階}};
\node[Square] at (40.350000, -5.100000) {};
\node[Onyomi] at (40.400000, -5.000000) {\hbox{\tate カイ}};
\node[Meaning] at (40.350000, -3.350000) {floor};
\node[Kanji] at (42.400000, -4.600000) {\textcolor[HTML]{1059be}{陛}};
\node[Square] at (42.400000, -5.100000) {};
\node[Onyomi] at (42.450000, -5.000000) {\hbox{\tate ヘイ}};
\node[Meaning] at (42.400000, -3.350000) {highness};
\node[Kanji] at (44.450000, -4.600000) {\textcolor[HTML]{d2a293}{隣}};
\node[Square] at (44.450000, -5.100000) {};
\node[Onyomi] at (44.500000, -5.000000) {\hbox{\tate リン}};
\node[Kunyomi] at (44.400000, -5.000000) {\hbox{\tate となり}};
\node[Meaning] at (44.450000, -3.350000) {neighbor};
\node[Kanji] at (46.500000, -4.600000) {\textcolor[HTML]{a3bac2}{隔}};
\node[Square] at (46.500000, -5.100000) {};
\node[Onyomi] at (46.550000, -5.000000) {\hbox{\tate カク}};
\node[Kunyomi] at (46.450000, -5.000000) {\hbox{\tate へだ.*}};
\node[Meaning] at (46.500000, -3.350000) {isolate};
\node[Kanji] at (48.550000, -4.600000) {\textcolor[HTML]{b0b0b5}{隠}};
\node[Square] at (48.550000, -5.100000) {};
\node[Onyomi] at (48.600000, -5.000000) {\hbox{\tate イン}};
\node[Kunyomi] at (48.500000, -5.000000) {\hbox{\tate かく.*}};
\node[Meaning] at (48.550000, -3.350000) {hide};
\node[Kanji] at (50.600000, -4.600000) {\textcolor[HTML]{1e76bb}{堕}};
\node[Square] at (50.600000, -5.100000) {};
\node[Onyomi] at (50.650000, -5.000000) {\hbox{\tate ダ}};
\node[Kunyomi] at (50.550000, -5.000000) {\hbox{\tate お・くず}};
\node[Meaning] at (50.600000, -3.350000) {degenerate};
\node[Kanji] at (52.650000, -4.600000) {\textcolor[HTML]{a3bac2}{陥}};
\node[Square] at (52.650000, -5.100000) {};
\node[Onyomi] at (52.700000, -5.000000) {\hbox{\tate カン}};
\node[Kunyomi] at (52.600000, -5.000000) {\hbox{\tate おちい}};
\node[Meaning] at (52.650000, -3.350000) {cave in};
\node[Kanji] at (54.700000, -4.600000) {\textcolor[HTML]{b0b0b5}{穴}};
\node[Square] at (54.700000, -5.100000) {};
\node[Onyomi] at (54.750000, -5.000000) {\hbox{\tate ケツ}};
\node[Kunyomi] at (54.650000, -5.000000) {\hbox{\tate あな}};
\node[Meaning] at (54.700000, -3.350000) {hole};
\node[Kanji] at (56.750000, -4.600000) {\textcolor[HTML]{cd8268}{空}};
\node[Square] at (56.750000, -5.100000) {};
\node[Onyomi] at (56.800000, -5.000000) {\hbox{\tate クウ}};
\node[Kunyomi] at (56.700000, -5.000000) {\hbox{\tate そら・あ}};
\node[Meaning] at (56.750000, -3.350000) {sky};
\node[Meaning] at (-58.050000, -4.500000) {71.94\%};
\node[Kanji] at (-56.000000, -6.650000) {\textcolor[HTML]{91b7c3}{控}};
\node[Square] at (-56.000000, -7.150000) {};
\node[Onyomi] at (-55.950000, -7.050000) {\hbox{\tate コウ}};
\node[Kunyomi] at (-56.050000, -7.050000) {\hbox{\tate ひか}};
\node[Meaning] at (-56.000000, -5.400000) {abstain};
\node[Kanji] at (-53.950000, -6.650000) {\textcolor[HTML]{d2a293}{突}};
\node[Square] at (-53.950000, -7.150000) {};
\node[Onyomi] at (-53.900000, -7.050000) {\hbox{\tate トツ}};
\node[Kunyomi] at (-54.000000, -7.050000) {\hbox{\tate つ.く}};
\node[Meaning] at (-53.950000, -5.400000) {stab};
\node[Kanji] at (-51.900000, -6.650000) {\textcolor[HTML]{cd8268}{究}};
\node[Square] at (-51.900000, -7.150000) {};
\node[Onyomi] at (-51.850000, -7.050000) {\hbox{\tate キュウ}};
\node[Kunyomi] at (-51.950000, -7.050000) {\hbox{\tate きわ.める}};
\node[Meaning] at (-51.900000, -5.400000) {research};
\node[Kanji] at (-49.850000, -6.650000) {\textcolor[HTML]{408dba}{窒}};
\node[Square] at (-49.850000, -7.150000) {};
\node[Onyomi] at (-49.800000, -7.050000) {\hbox{\tate チツ}};
\node[Meaning] at (-49.850000, -5.400000) {suffocate};
\node[Kanji] at (-47.800000, -6.650000) {\textcolor[HTML]{1e76bb}{窃}};
\node[Square] at (-47.800000, -7.150000) {};
\node[Onyomi] at (-47.750000, -7.050000) {\hbox{\tate セツ}};
\node[Kunyomi] at (-47.850000, -7.050000) {\hbox{\tate ぬす・ひそ}};
\node[Meaning] at (-47.800000, -5.400000) {steal};
\node[Kanji] at (-45.750000, -6.650000) {\textcolor[HTML]{1e76bb}{搾}};
\node[Square] at (-45.750000, -7.150000) {};
\node[Onyomi] at (-45.700000, -7.050000) {\hbox{\tate サク}};
\node[Kunyomi] at (-45.800000, -7.050000) {\hbox{\tate しぼ}};
\node[Meaning] at (-45.750000, -5.400000) {squeeze};
\node[Kanji] at (-43.700000, -6.650000) {\textcolor[HTML]{68a4bc}{窯}};
\node[Square] at (-43.700000, -7.150000) {};
\node[Onyomi] at (-43.650000, -7.050000) {\hbox{\tate ヨウ}};
\node[Kunyomi] at (-43.750000, -7.050000) {\hbox{\tate かま}};
\node[Meaning] at (-43.700000, -5.400000) {kiln};
\node[Kanji] at (-41.650000, -6.650000) {\textcolor[HTML]{68a4bc}{窮}};
\node[Square] at (-41.650000, -7.150000) {};
\node[Onyomi] at (-41.600000, -7.050000) {\hbox{\tate キュウ}};
\node[Kunyomi] at (-41.700000, -7.050000) {\hbox{\tate きわ}};
\node[Meaning] at (-41.650000, -5.400000) {destitute};
\node[Kanji] at (-39.600000, -6.650000) {\textcolor[HTML]{c8a59d}{探}};
\node[Square] at (-39.600000, -7.150000) {};
\node[Onyomi] at (-39.550000, -7.050000) {\hbox{\tate タン}};
\node[Kunyomi] at (-39.650000, -7.050000) {\hbox{\tate さが.す}};
\node[Meaning] at (-39.600000, -5.400000) {look for};
\node[Kanji] at (-37.550000, -6.650000) {\textcolor[HTML]{d2a293}{深}};
\node[Square] at (-37.550000, -7.150000) {};
\node[Onyomi] at (-37.500000, -7.050000) {\hbox{\tate シン}};
\node[Kunyomi] at (-37.600000, -7.050000) {\hbox{\tate ふか.*}};
\node[Meaning] at (-37.550000, -5.400000) {deep};
\node[Kanji] at (-35.500000, -6.650000) {\textcolor[HTML]{c8a59d}{丘}};
\node[Square] at (-35.500000, -7.150000) {};
\node[Onyomi] at (-35.450000, -7.050000) {\hbox{\tate キュウ}};
\node[Kunyomi] at (-35.550000, -7.050000) {\hbox{\tate おか}};
\node[Meaning] at (-35.500000, -5.400000) {hill};
\node[Kanji] at (-33.450000, -6.650000) {\textcolor[HTML]{b0b0b5}{岳}};
\node[Square] at (-33.450000, -7.150000) {};
\node[Onyomi] at (-33.400000, -7.050000) {\hbox{\tate ガク}};
\node[Kunyomi] at (-33.500000, -7.050000) {\hbox{\tate たけ}};
\node[Meaning] at (-33.450000, -5.400000) {peak};
\node[Kanji] at (-31.400000, -6.650000) {\textcolor[HTML]{cd8268}{兵}};
\node[Square] at (-31.400000, -7.150000) {};
\node[Onyomi] at (-31.350000, -7.050000) {\hbox{\tate ヘイ・ヒョウ}};
\node[Meaning] at (-31.400000, -5.400000) {soldier};
\node[Kanji] at (-29.350000, -6.650000) {\textcolor[HTML]{d2a293}{浜}};
\node[Square] at (-29.350000, -7.150000) {};
\node[Onyomi] at (-29.300000, -7.050000) {\hbox{\tate ヒン}};
\node[Kunyomi] at (-29.400000, -7.050000) {\hbox{\tate はま}};
\node[Meaning] at (-29.350000, -5.400000) {beach};
\node[Kanji] at (-27.300000, -6.650000) {\textcolor[HTML]{a3bac2}{糸}};
\node[Square] at (-27.300000, -7.150000) {};
\node[Onyomi] at (-27.250000, -7.050000) {\hbox{\tate シ}};
\node[Kunyomi] at (-27.350000, -7.050000) {\hbox{\tate いと}};
\node[Meaning] at (-27.300000, -5.400000) {thread};
\node[Kanji] at (-25.250000, -6.650000) {\textcolor[HTML]{d2a293}{織}};
\node[Square] at (-25.250000, -7.150000) {};
\node[Onyomi] at (-25.200000, -7.050000) {\hbox{\tate シキ・ショク}};
\node[Kunyomi] at (-25.300000, -7.050000) {\hbox{\tate お.る}};
\node[Meaning] at (-25.250000, -5.400000) {weave};
\node[Kanji] at (-23.200000, -6.650000) {\textcolor[HTML]{1e76bb}{繕}};
\node[Square] at (-23.200000, -7.150000) {};
\node[Onyomi] at (-23.150000, -7.050000) {\hbox{\tate ゼン}};
\node[Kunyomi] at (-23.250000, -7.050000) {\hbox{\tate つくろ.う}};
\node[Meaning] at (-23.200000, -5.400000) {darning};
\node[Kanji] at (-21.150000, -6.650000) {\textcolor[HTML]{b0b0b5}{縮}};
\node[Square] at (-21.150000, -7.150000) {};
\node[Onyomi] at (-21.100000, -7.050000) {\hbox{\tate シュク}};
\node[Kunyomi] at (-21.200000, -7.050000) {\hbox{\tate ちぢ・ちじ}};
\node[Meaning] at (-21.150000, -5.400000) {shrink};
\node[Kanji] at (-19.100000, -6.650000) {\textcolor[HTML]{b0b0b5}{繁}};
\node[Square] at (-19.100000, -7.150000) {};
\node[Onyomi] at (-19.050000, -7.050000) {\hbox{\tate ハン}};
\node[Kunyomi] at (-19.150000, -7.050000) {\hbox{\tate しげ.*}};
\node[Meaning] at (-19.100000, -5.400000) {overgrown};
\node[Kanji] at (-17.050000, -6.650000) {\textcolor[HTML]{b0b0b5}{縦}};
\node[Square] at (-17.050000, -7.150000) {};
\node[Onyomi] at (-17.000000, -7.050000) {\hbox{\tate ジュウ}};
\node[Kunyomi] at (-17.100000, -7.050000) {\hbox{\tate たて}};
\node[Meaning] at (-17.050000, -5.400000) {vertical};
\node[Kanji] at (-15.000000, -6.650000) {\textcolor[HTML]{b74029}{線}};
\node[Square] at (-15.000000, -7.150000) {};
\node[Onyomi] at (-14.950000, -7.050000) {\hbox{\tate セン}};
\node[Meaning] at (-15.000000, -5.400000) {line};
\node[Kanji] at (-12.950000, -6.650000) {\textcolor[HTML]{b0b0b5}{締}};
\node[Square] at (-12.950000, -7.150000) {};
\node[Onyomi] at (-12.900000, -7.050000) {\hbox{\tate テイ}};
\node[Kunyomi] at (-13.000000, -7.050000) {\hbox{\tate し}};
\node[Meaning] at (-12.950000, -5.400000) {tighten};
\node[Kanji] at (-10.900000, -6.650000) {\textcolor[HTML]{c8a59d}{維}};
\node[Square] at (-10.900000, -7.150000) {};
\node[Onyomi] at (-10.850000, -7.050000) {\hbox{\tate イ}};
\node[Meaning] at (-10.900000, -5.400000) {maintain};
\node[Kanji] at (-8.850000, -6.650000) {\textcolor[HTML]{b0b0b5}{羅}};
\node[Square] at (-8.850000, -7.150000) {};
\node[Onyomi] at (-8.800000, -7.050000) {\hbox{\tate ラ}};
\node[Kunyomi] at (-8.900000, -7.050000) {\hbox{\tate うすもの}};
\node[Meaning] at (-8.850000, -5.400000) {spread out};
\node[Kanji] at (-6.800000, -6.650000) {\textcolor[HTML]{c8a59d}{練}};
\node[Square] at (-6.800000, -7.150000) {};
\node[Onyomi] at (-6.750000, -7.050000) {\hbox{\tate レン}};
\node[Kunyomi] at (-6.850000, -7.050000) {\hbox{\tate ね}};
\node[Meaning] at (-6.800000, -5.400000) {practice};
\node[Kanji] at (-4.750000, -6.650000) {\textcolor[HTML]{b0b0b5}{緒}};
\node[Square] at (-4.750000, -7.150000) {};
\node[Onyomi] at (-4.700000, -7.050000) {\hbox{\tate ショ}};
\node[Meaning] at (-4.750000, -5.400000) {together};
\node[Kanji] at (-2.700000, -6.650000) {\textcolor[HTML]{c36143}{続}};
\node[Square] at (-2.700000, -7.150000) {};
\node[Onyomi] at (-2.650000, -7.050000) {\hbox{\tate ゾク}};
\node[Kunyomi] at (-2.750000, -7.050000) {\hbox{\tate つづ.く}};
\node[Meaning] at (-2.700000, -5.400000) {continue};
\node[Kanji] at (-0.650000, -6.650000) {\textcolor[HTML]{c8a59d}{絵}};
\node[Square] at (-0.650000, -7.150000) {};
\node[Onyomi] at (-0.600000, -7.050000) {\hbox{\tate エ}};
\node[Meaning] at (-0.650000, -5.400000) {drawing};
\node[Kanji] at (1.400000, -6.650000) {\textcolor[HTML]{cd8268}{統}};
\node[Square] at (1.400000, -7.150000) {};
\node[Onyomi] at (1.450000, -7.050000) {\hbox{\tate トウ}};
\node[Kunyomi] at (1.350000, -7.050000) {\hbox{\tate す.べる}};
\node[Meaning] at (1.400000, -5.400000) {unite};
\node[Kanji] at (3.450000, -6.650000) {\textcolor[HTML]{68a4bc}{絞}};
\node[Square] at (3.450000, -7.150000) {};
\node[Onyomi] at (3.500000, -7.050000) {\hbox{\tate コウ}};
\node[Kunyomi] at (3.400000, -7.050000) {\hbox{\tate し・しぼ}};
\node[Meaning] at (3.450000, -5.400000) {strangle};
\node[Kanji] at (5.500000, -6.650000) {\textcolor[HTML]{d2a293}{給}};
\node[Square] at (5.500000, -7.150000) {};
\node[Onyomi] at (5.550000, -7.050000) {\hbox{\tate キュウ}};
\node[Kunyomi] at (5.450000, -7.050000) {\hbox{\tate たま.う}};
\node[Meaning] at (5.500000, -5.400000) {salary};
\node[Kanji] at (7.550000, -6.650000) {\textcolor[HTML]{b0b0b5}{絡}};
\node[Square] at (7.550000, -7.150000) {};
\node[Onyomi] at (7.600000, -7.050000) {\hbox{\tate ラク}};
\node[Kunyomi] at (7.500000, -7.050000) {\hbox{\tate から.む}};
\node[Meaning] at (7.550000, -5.400000) {entangle};
\node[Kanji] at (9.600000, -6.650000) {\textcolor[HTML]{c36143}{結}};
\node[Square] at (9.600000, -7.150000) {};
\node[Onyomi] at (9.650000, -7.050000) {\hbox{\tate ケツ}};
\node[Kunyomi] at (9.550000, -7.050000) {\hbox{\tate むす.ぶ}};
\node[Meaning] at (9.600000, -5.400000) {bind};
\node[Kanji] at (11.650000, -6.650000) {\textcolor[HTML]{cd8268}{終}};
\node[Square] at (11.650000, -7.150000) {};
\node[Onyomi] at (11.700000, -7.050000) {\hbox{\tate シュウ}};
\node[Kunyomi] at (11.600000, -7.050000) {\hbox{\tate おわ.り・お}};
\node[Meaning] at (11.650000, -5.400000) {end};
\node[Kanji] at (13.700000, -6.650000) {\textcolor[HTML]{d69f8d}{級}};
\node[Square] at (13.700000, -7.150000) {};
\node[Onyomi] at (13.750000, -7.050000) {\hbox{\tate キュウ}};
\node[Meaning] at (13.700000, -5.400000) {rank};
\node[Kanji] at (15.750000, -6.650000) {\textcolor[HTML]{d69f8d}{紀}};
\node[Square] at (15.750000, -7.150000) {};
\node[Onyomi] at (15.800000, -7.050000) {\hbox{\tate キ}};
\node[Meaning] at (15.750000, -5.400000) {account};
\node[Kanji] at (17.800000, -6.650000) {\textcolor[HTML]{a3bac2}{紅}};
\node[Square] at (17.800000, -7.150000) {};
\node[Onyomi] at (17.850000, -7.050000) {\hbox{\tate コウ}};
\node[Kunyomi] at (17.750000, -7.050000) {\hbox{\tate べに}};
\node[Meaning] at (17.800000, -5.400000) {deep red};
\node[Kanji] at (19.850000, -6.650000) {\textcolor[HTML]{c8a59d}{納}};
\node[Square] at (19.850000, -7.150000) {};
\node[Onyomi] at (19.900000, -7.050000) {\hbox{\tate ノウ}};
\node[Kunyomi] at (19.800000, -7.050000) {\hbox{\tate おさ・なっ}};
\node[Meaning] at (19.850000, -5.400000) {supply};
\node[Kanji] at (21.900000, -6.650000) {\textcolor[HTML]{68a4bc}{紡}};
\node[Square] at (21.900000, -7.150000) {};
\node[Onyomi] at (21.950000, -7.050000) {\hbox{\tate ボウ}};
\node[Kunyomi] at (21.850000, -7.050000) {\hbox{\tate つむ}};
\node[Meaning] at (21.900000, -5.400000) {spinning};
\node[Kanji] at (23.950000, -6.650000) {\textcolor[HTML]{91b7c3}{紛}};
\node[Square] at (23.950000, -7.150000) {};
\node[Onyomi] at (24.000000, -7.050000) {\hbox{\tate フン}};
\node[Kunyomi] at (23.900000, -7.050000) {\hbox{\tate まぎ.*}};
\node[Meaning] at (23.950000, -5.400000) {distract};
\node[Kanji] at (26.000000, -6.650000) {\textcolor[HTML]{c8a59d}{紹}};
\node[Square] at (26.000000, -7.150000) {};
\node[Onyomi] at (26.050000, -7.050000) {\hbox{\tate ショウ}};
\node[Meaning] at (26.000000, -5.400000) {introduce};
\node[Kanji] at (28.050000, -6.650000) {\textcolor[HTML]{c36143}{経}};
\node[Square] at (28.050000, -7.150000) {};
\node[Onyomi] at (28.100000, -7.050000) {\hbox{\tate ケイ}};
\node[Kunyomi] at (28.000000, -7.050000) {\hbox{\tate た.つ}};
\node[Meaning] at (28.050000, -5.400000) {passage of time};
\node[Kanji] at (30.100000, -6.650000) {\textcolor[HTML]{408dba}{紳}};
\node[Square] at (30.100000, -7.150000) {};
\node[Onyomi] at (30.150000, -7.050000) {\hbox{\tate シン}};
\node[Meaning] at (30.100000, -5.400000) {gentleman};
\node[Kanji] at (32.150000, -6.650000) {\textcolor[HTML]{cd8268}{約}};
\node[Square] at (32.150000, -7.150000) {};
\node[Onyomi] at (32.200000, -7.050000) {\hbox{\tate ヤク}};
\node[Meaning] at (32.150000, -5.400000) {promise};
\node[Kanji] at (34.200000, -6.650000) {\textcolor[HTML]{d69f8d}{細}};
\node[Square] at (34.200000, -7.150000) {};
\node[Onyomi] at (34.250000, -7.050000) {\hbox{\tate サイ}};
\node[Kunyomi] at (34.150000, -7.050000) {\hbox{\tate ほそ・こま}};
\node[Meaning] at (34.200000, -5.400000) {thin};
\node[Kanji] at (36.250000, -6.650000) {\textcolor[HTML]{68a4bc}{累}};
\node[Square] at (36.250000, -7.150000) {};
\node[Onyomi] at (36.300000, -7.050000) {\hbox{\tate ルイ}};
\node[Meaning] at (36.250000, -5.400000) {accumulate};
\node[Kanji] at (38.300000, -6.650000) {\textcolor[HTML]{b0b0b5}{索}};
\node[Square] at (38.300000, -7.150000) {};
\node[Onyomi] at (38.350000, -7.050000) {\hbox{\tate サク}};
\node[Meaning] at (38.300000, -5.400000) {search};
\node[Kanji] at (40.350000, -6.650000) {\textcolor[HTML]{cd8268}{総}};
\node[Square] at (40.350000, -7.150000) {};
\node[Onyomi] at (40.400000, -7.050000) {\hbox{\tate ソウ}};
\node[Meaning] at (40.350000, -5.400000) {whole};
\node[Kanji] at (42.400000, -6.650000) {\textcolor[HTML]{91b7c3}{綿}};
\node[Square] at (42.400000, -7.150000) {};
\node[Onyomi] at (42.450000, -7.050000) {\hbox{\tate メン}};
\node[Kunyomi] at (42.350000, -7.050000) {\hbox{\tate わた}};
\node[Meaning] at (42.400000, -5.400000) {cotton};
\node[Kanji] at (44.450000, -6.650000) {\textcolor[HTML]{408dba}{絹}};
\node[Square] at (44.450000, -7.150000) {};
\node[Onyomi] at (44.500000, -7.050000) {\hbox{\tate ケン}};
\node[Kunyomi] at (44.400000, -7.050000) {\hbox{\tate きぬ}};
\node[Meaning] at (44.450000, -5.400000) {silk};
\node[Kanji] at (46.500000, -6.650000) {\textcolor[HTML]{b0b0b5}{繰}};
\node[Square] at (46.500000, -7.150000) {};
\node[Onyomi] at (46.550000, -7.050000) {\hbox{\tate ソウ}};
\node[Kunyomi] at (46.450000, -7.050000) {\hbox{\tate く}};
\node[Meaning] at (46.500000, -5.400000) {spin};
\node[Kanji] at (48.550000, -6.650000) {\textcolor[HTML]{d69f8d}{継}};
\node[Square] at (48.550000, -7.150000) {};
\node[Onyomi] at (48.600000, -7.050000) {\hbox{\tate ケイ}};
\node[Kunyomi] at (48.500000, -7.050000) {\hbox{\tate つ.ぐ}};
\node[Meaning] at (48.550000, -5.400000) {inherit};
\node[Kanji] at (50.600000, -6.650000) {\textcolor[HTML]{b0b0b5}{緑}};
\node[Square] at (50.600000, -7.150000) {};
\node[Onyomi] at (50.650000, -7.050000) {\hbox{\tate リョク}};
\node[Kunyomi] at (50.550000, -7.050000) {\hbox{\tate みどり}};
\node[Meaning] at (50.600000, -5.400000) {green};
\node[Kanji] at (52.650000, -6.650000) {\textcolor[HTML]{c8a59d}{縁}};
\node[Square] at (52.650000, -7.150000) {};
\node[Onyomi] at (52.700000, -7.050000) {\hbox{\tate エン・ネン}};
\node[Kunyomi] at (52.600000, -7.050000) {\hbox{\tate ふち}};
\node[Meaning] at (52.650000, -5.400000) {edge};
\node[Kanji] at (54.700000, -6.650000) {\textcolor[HTML]{b0b0b5}{網}};
\node[Square] at (54.700000, -7.150000) {};
\node[Onyomi] at (54.750000, -7.050000) {\hbox{\tate モウ}};
\node[Kunyomi] at (54.650000, -7.050000) {\hbox{\tate あみ}};
\node[Meaning] at (54.700000, -5.400000) {netting};
\node[Kanji] at (56.750000, -6.650000) {\textcolor[HTML]{a3bac2}{緊}};
\node[Square] at (56.750000, -7.150000) {};
\node[Onyomi] at (56.800000, -7.050000) {\hbox{\tate キン}};
\node[Meaning] at (56.750000, -5.400000) {tense};
\node[Meaning] at (-58.050000, -6.550000) {74.04\%};
\node[Kanji] at (-56.000000, -8.700000) {\textcolor[HTML]{a3bac2}{紫}};
\node[Square] at (-56.000000, -9.200000) {};
\node[Onyomi] at (-55.950000, -9.100000) {\hbox{\tate シ}};
\node[Kunyomi] at (-56.050000, -9.100000) {\hbox{\tate むらさき}};
\node[Meaning] at (-56.000000, -7.450000) {purple};
\node[Kanji] at (-53.950000, -8.700000) {\textcolor[HTML]{68a4bc}{縛}};
\node[Square] at (-53.950000, -9.200000) {};
\node[Onyomi] at (-53.900000, -9.100000) {\hbox{\tate バク}};
\node[Kunyomi] at (-54.000000, -9.100000) {\hbox{\tate しば}};
\node[Meaning] at (-53.950000, -7.450000) {bind};
\node[Kanji] at (-51.900000, -8.700000) {\textcolor[HTML]{c8a59d}{縄}};
\node[Square] at (-51.900000, -9.200000) {};
\node[Onyomi] at (-51.850000, -9.100000) {\hbox{\tate ジョウ}};
\node[Kunyomi] at (-51.950000, -9.100000) {\hbox{\tate なわ}};
\node[Meaning] at (-51.900000, -7.450000) {rope};
\node[Kanji] at (-49.850000, -8.700000) {\textcolor[HTML]{c8a59d}{幼}};
\node[Square] at (-49.850000, -9.200000) {};
\node[Onyomi] at (-49.800000, -9.100000) {\hbox{\tate ヨウ}};
\node[Kunyomi] at (-49.900000, -9.100000) {\hbox{\tate おさな.い}};
\node[Meaning] at (-49.850000, -7.450000) {infancy};
\node[Kanji] at (-47.800000, -8.700000) {\textcolor[HTML]{a11d25}{後}};
\node[Square] at (-47.800000, -9.200000) {};
\node[Onyomi] at (-47.750000, -9.100000) {\hbox{\tate ゴ・コウ}};
\node[Kunyomi] at (-47.850000, -9.100000) {\hbox{\tate うし.ろ}};
\node[Meaning] at (-47.800000, -7.450000) {behind};
\node[Kanji] at (-45.750000, -8.700000) {\textcolor[HTML]{68a4bc}{幽}};
\node[Square] at (-45.750000, -9.200000) {};
\node[Onyomi] at (-45.700000, -9.100000) {\hbox{\tate ユウ}};
\node[Meaning] at (-45.750000, -7.450000) {secluded};
\node[Kanji] at (-43.700000, -8.700000) {\textcolor[HTML]{a3bac2}{幾}};
\node[Square] at (-43.700000, -9.200000) {};
\node[Onyomi] at (-43.650000, -9.100000) {\hbox{\tate キ}};
\node[Kunyomi] at (-43.750000, -9.100000) {\hbox{\tate いく}};
\node[Meaning] at (-43.700000, -7.450000) {how many};
\node[Kanji] at (-41.650000, -8.700000) {\textcolor[HTML]{b74029}{機}};
\node[Square] at (-41.650000, -9.200000) {};
\node[Onyomi] at (-41.600000, -9.100000) {\hbox{\tate キ}};
\node[Kunyomi] at (-41.700000, -9.100000) {\hbox{\tate はた}};
\node[Meaning] at (-41.650000, -7.450000) {machine};
\node[Kanji] at (-39.600000, -8.700000) {\textcolor[HTML]{a3bac2}{玄}};
\node[Square] at (-39.600000, -9.200000) {};
\node[Onyomi] at (-39.550000, -9.100000) {\hbox{\tate ゲン}};
\node[Kunyomi] at (-39.650000, -9.100000) {\hbox{\tate くろ}};
\node[Meaning] at (-39.600000, -7.450000) {mysterious};
\node[Kanji] at (-37.550000, -8.700000) {\textcolor[HTML]{91b7c3}{畜}};
\node[Square] at (-37.550000, -9.200000) {};
\node[Onyomi] at (-37.500000, -9.100000) {\hbox{\tate チク}};
\node[Meaning] at (-37.550000, -7.450000) {livestock};
\node[Kanji] at (-35.500000, -8.700000) {\textcolor[HTML]{91b7c3}{蓄}};
\node[Square] at (-35.500000, -9.200000) {};
\node[Onyomi] at (-35.450000, -9.100000) {\hbox{\tate チク}};
\node[Kunyomi] at (-35.550000, -9.100000) {\hbox{\tate たくわ.える}};
\node[Meaning] at (-35.500000, -7.450000) {amass};
\node[Kanji] at (-33.450000, -8.700000) {\textcolor[HTML]{a3bac2}{弦}};
\node[Square] at (-33.450000, -9.200000) {};
\node[Onyomi] at (-33.400000, -9.100000) {\hbox{\tate ゲン}};
\node[Kunyomi] at (-33.500000, -9.100000) {\hbox{\tate つる}};
\node[Meaning] at (-33.450000, -7.450000) {chord};
\node[Kanji] at (-31.400000, -8.700000) {\textcolor[HTML]{91b7c3}{擁}};
\node[Square] at (-31.400000, -9.200000) {};
\node[Onyomi] at (-31.350000, -9.100000) {\hbox{\tate ヨウ}};
\node[Meaning] at (-31.400000, -7.450000) {embrace};
\node[Kanji] at (-29.350000, -8.700000) {\textcolor[HTML]{91b7c3}{滋}};
\node[Square] at (-29.350000, -9.200000) {};
\node[Onyomi] at (-29.300000, -9.100000) {\hbox{\tate ジ}};
\node[Meaning] at (-29.350000, -7.450000) {nourishing};
\node[Kanji] at (-27.300000, -8.700000) {\textcolor[HTML]{68a4bc}{慈}};
\node[Square] at (-27.300000, -9.200000) {};
\node[Onyomi] at (-27.250000, -9.100000) {\hbox{\tate ジ}};
\node[Kunyomi] at (-27.350000, -9.100000) {\hbox{\tate いつく}};
\node[Meaning] at (-27.300000, -7.450000) {mercy};
\node[Kanji] at (-25.250000, -8.700000) {\textcolor[HTML]{b0b0b5}{磁}};
\node[Square] at (-25.250000, -9.200000) {};
\node[Onyomi] at (-25.200000, -9.100000) {\hbox{\tate ジ}};
\node[Meaning] at (-25.250000, -7.450000) {magnet};
\node[Kanji] at (-23.200000, -8.700000) {\textcolor[HTML]{cd8268}{系}};
\node[Square] at (-23.200000, -9.200000) {};
\node[Onyomi] at (-23.150000, -9.100000) {\hbox{\tate ケイ}};
\node[Meaning] at (-23.200000, -7.450000) {lineage};
\node[Kanji] at (-21.150000, -8.700000) {\textcolor[HTML]{d69f8d}{係}};
\node[Square] at (-21.150000, -9.200000) {};
\node[Onyomi] at (-21.100000, -9.100000) {\hbox{\tate ケイ}};
\node[Kunyomi] at (-21.200000, -9.100000) {\hbox{\tate かか・かかり}};
\node[Meaning] at (-21.150000, -7.450000) {connection};
\node[Kanji] at (-19.100000, -8.700000) {\textcolor[HTML]{c8a59d}{孫}};
\node[Square] at (-19.100000, -9.200000) {};
\node[Onyomi] at (-19.050000, -9.100000) {\hbox{\tate ソン}};
\node[Kunyomi] at (-19.150000, -9.100000) {\hbox{\tate まご}};
\node[Meaning] at (-19.100000, -7.450000) {grandchild};
\node[Kanji] at (-17.050000, -8.700000) {\textcolor[HTML]{a3bac2}{懸}};
\node[Square] at (-17.050000, -9.200000) {};
\node[Onyomi] at (-17.000000, -9.100000) {\hbox{\tate ケン}};
\node[Kunyomi] at (-17.100000, -9.100000) {\hbox{\tate か.*}};
\node[Meaning] at (-17.050000, -7.450000) {suspend};
\node[Kanji] at (-15.000000, -8.700000) {\textcolor[HTML]{b0b0b5}{却}};
\node[Square] at (-15.000000, -9.200000) {};
\node[Onyomi] at (-14.950000, -9.100000) {\hbox{\tate キャク}};
\node[Kunyomi] at (-15.050000, -9.100000) {\hbox{\tate かえって}};
\node[Meaning] at (-15.000000, -7.450000) {contrary};
\node[Kanji] at (-12.950000, -8.700000) {\textcolor[HTML]{c8a59d}{脚}};
\node[Square] at (-12.950000, -9.200000) {};
\node[Onyomi] at (-12.900000, -9.100000) {\hbox{\tate キャク}};
\node[Kunyomi] at (-13.000000, -9.100000) {\hbox{\tate あし}};
\node[Meaning] at (-12.950000, -7.450000) {leg};
\node[Kanji] at (-10.900000, -8.700000) {\textcolor[HTML]{408dba}{卸}};
\node[Square] at (-10.900000, -9.200000) {};
\node[Onyomi] at (-10.850000, -9.100000) {\hbox{\tate シャ}};
\node[Kunyomi] at (-10.950000, -9.100000) {\hbox{\tate おろし・おろ}};
\node[Meaning] at (-10.900000, -7.450000) {wholesale};
\node[Kanji] at (-8.850000, -8.700000) {\textcolor[HTML]{d69f8d}{御}};
\node[Square] at (-8.850000, -9.200000) {};
\node[Onyomi] at (-8.800000, -9.100000) {\hbox{\tate ゴ}};
\node[Kunyomi] at (-8.900000, -9.100000) {\hbox{\tate お}};
\node[Meaning] at (-8.850000, -7.450000) {honorable};
\node[Kanji] at (-6.800000, -8.700000) {\textcolor[HTML]{c8a59d}{服}};
\node[Square] at (-6.800000, -9.200000) {};
\node[Onyomi] at (-6.750000, -9.100000) {\hbox{\tate フク}};
\node[Meaning] at (-6.800000, -7.450000) {clothes};
\node[Kanji] at (-4.750000, -8.700000) {\textcolor[HTML]{d69f8d}{命}};
\node[Square] at (-4.750000, -9.200000) {};
\node[Onyomi] at (-4.700000, -9.100000) {\hbox{\tate メイ・ミョウ}};
\node[Kunyomi] at (-4.800000, -9.100000) {\hbox{\tate いのち}};
\node[Meaning] at (-4.750000, -7.450000) {fate};
\node[Kanji] at (-2.700000, -8.700000) {\textcolor[HTML]{d69f8d}{令}};
\node[Square] at (-2.700000, -9.200000) {};
\node[Onyomi] at (-2.650000, -9.100000) {\hbox{\tate レイ}};
\node[Meaning] at (-2.700000, -7.450000) {orders};
\node[Kanji] at (-0.650000, -8.700000) {\textcolor[HTML]{68a4bc}{零}};
\node[Square] at (-0.650000, -9.200000) {};
\node[Onyomi] at (-0.600000, -9.100000) {\hbox{\tate レイ}};
\node[Kunyomi] at (-0.700000, -9.100000) {\hbox{\tate こぼ.す}};
\node[Meaning] at (-0.650000, -7.450000) {zero};
\node[Kanji] at (1.400000, -8.700000) {\textcolor[HTML]{b0b0b5}{齢}};
\node[Square] at (1.400000, -9.200000) {};
\node[Onyomi] at (1.450000, -9.100000) {\hbox{\tate レイ}};
\node[Kunyomi] at (1.350000, -9.100000) {\hbox{\tate よわい}};
\node[Meaning] at (1.400000, -7.450000) {age};
\node[Kanji] at (3.450000, -8.700000) {\textcolor[HTML]{b0b0b5}{冷}};
\node[Square] at (3.450000, -9.200000) {};
\node[Onyomi] at (3.500000, -9.100000) {\hbox{\tate レイ}};
\node[Kunyomi] at (3.400000, -9.100000) {\hbox{\tate つめ.たい}};
\node[Meaning] at (3.450000, -7.450000) {cool};
\node[Kanji] at (5.500000, -8.700000) {\textcolor[HTML]{d69f8d}{領}};
\node[Square] at (5.500000, -9.200000) {};
\node[Onyomi] at (5.550000, -9.100000) {\hbox{\tate リョウ}};
\node[Meaning] at (5.500000, -7.450000) {territory};
\node[Kanji] at (7.550000, -8.700000) {\textcolor[HTML]{b0b0b5}{鈴}};
\node[Square] at (7.550000, -9.200000) {};
\node[Onyomi] at (7.600000, -9.100000) {\hbox{\tate リン}};
\node[Kunyomi] at (7.500000, -9.100000) {\hbox{\tate すず}};
\node[Meaning] at (7.550000, -7.450000) {buzzer};
\node[Kanji] at (9.600000, -8.700000) {\textcolor[HTML]{a3bac2}{勇}};
\node[Square] at (9.600000, -9.200000) {};
\node[Onyomi] at (9.650000, -9.100000) {\hbox{\tate ユウ}};
\node[Kunyomi] at (9.550000, -9.100000) {\hbox{\tate いさ}};
\node[Meaning] at (9.600000, -7.450000) {courage};
\node[Kanji] at (11.650000, -8.700000) {\textcolor[HTML]{b74029}{通}};
\node[Square] at (11.650000, -9.200000) {};
\node[Onyomi] at (11.700000, -9.100000) {\hbox{\tate ツウ}};
\node[Kunyomi] at (11.600000, -9.100000) {\hbox{\tate とお.る}};
\node[Meaning] at (11.650000, -7.450000) {pass through};
\node[Kanji] at (13.700000, -8.700000) {\textcolor[HTML]{a3bac2}{踊}};
\node[Square] at (13.700000, -9.200000) {};
\node[Onyomi] at (13.750000, -9.100000) {\hbox{\tate ヨウ}};
\node[Kunyomi] at (13.650000, -9.100000) {\hbox{\tate おど}};
\node[Meaning] at (13.700000, -7.450000) {dance};
\node[Kanji] at (15.750000, -8.700000) {\textcolor[HTML]{c8a59d}{疑}};
\node[Square] at (15.750000, -9.200000) {};
\node[Onyomi] at (15.800000, -9.100000) {\hbox{\tate ギ}};
\node[Kunyomi] at (15.700000, -9.100000) {\hbox{\tate うたが.う}};
\node[Meaning] at (15.750000, -7.450000) {doubt};
\node[Kanji] at (17.800000, -8.700000) {\textcolor[HTML]{68a4bc}{擬}};
\node[Square] at (17.800000, -9.200000) {};
\node[Onyomi] at (17.850000, -9.100000) {\hbox{\tate ギ}};
\node[Kunyomi] at (17.750000, -9.100000) {\hbox{\tate まが・もど}};
\node[Meaning] at (17.800000, -7.450000) {imitate};
\node[Kanji] at (19.850000, -8.700000) {\textcolor[HTML]{68a4bc}{凝}};
\node[Square] at (19.850000, -9.200000) {};
\node[Onyomi] at (19.900000, -9.100000) {\hbox{\tate ギョウ}};
\node[Kunyomi] at (19.800000, -9.100000) {\hbox{\tate こ・こご}};
\node[Meaning] at (19.850000, -7.450000) {congeal};
\node[Kanji] at (21.900000, -8.700000) {\textcolor[HTML]{c8a59d}{範}};
\node[Square] at (21.900000, -9.200000) {};
\node[Onyomi] at (21.950000, -9.100000) {\hbox{\tate ハン}};
\node[Meaning] at (21.900000, -7.450000) {example};
\node[Kanji] at (23.950000, -8.700000) {\textcolor[HTML]{c8a59d}{犯}};
\node[Square] at (23.950000, -9.200000) {};
\node[Onyomi] at (24.000000, -9.100000) {\hbox{\tate ハン}};
\node[Kunyomi] at (23.900000, -9.100000) {\hbox{\tate おか.す}};
\node[Meaning] at (23.950000, -7.450000) {crime};
\node[Kanji] at (26.000000, -8.700000) {\textcolor[HTML]{1e76bb}{厄}};
\node[Square] at (26.000000, -9.200000) {};
\node[Onyomi] at (26.050000, -9.100000) {\hbox{\tate ヤク}};
\node[Meaning] at (26.000000, -7.450000) {unlucky};
\node[Kanji] at (28.050000, -8.700000) {\textcolor[HTML]{c8a59d}{危}};
\node[Square] at (28.050000, -9.200000) {};
\node[Onyomi] at (28.100000, -9.100000) {\hbox{\tate キ}};
\node[Kunyomi] at (28.000000, -9.100000) {\hbox{\tate あぶ.ない}};
\node[Meaning] at (28.050000, -7.450000) {dangerous};
\node[Kanji] at (30.100000, -8.700000) {\textcolor[HTML]{68a4bc}{宛}};
\node[Square] at (30.100000, -9.200000) {};
\node[Kunyomi] at (30.050000, -9.100000) {\hbox{\tate あ.てる}};
\node[Meaning] at (30.100000, -7.450000) {allocate};
\node[Kanji] at (32.150000, -8.700000) {\textcolor[HTML]{a3bac2}{腕}};
\node[Square] at (32.150000, -9.200000) {};
\node[Onyomi] at (32.200000, -9.100000) {\hbox{\tate ワン}};
\node[Kunyomi] at (32.100000, -9.100000) {\hbox{\tate うで}};
\node[Meaning] at (32.150000, -7.450000) {arm};
\node[Kanji] at (34.200000, -8.700000) {\textcolor[HTML]{1e76bb}{怨}};
\node[Square] at (34.200000, -9.200000) {};
\node[Onyomi] at (34.250000, -9.100000) {\hbox{\tate エン・オン}};
\node[Meaning] at (34.200000, -7.450000) {grudge};
\node[Kanji] at (36.250000, -8.700000) {\textcolor[HTML]{b0b0b5}{柳}};
\node[Square] at (36.250000, -9.200000) {};
\node[Onyomi] at (36.300000, -9.100000) {\hbox{\tate リュウ}};
\node[Kunyomi] at (36.200000, -9.100000) {\hbox{\tate やなぎ}};
\node[Meaning] at (36.250000, -7.450000) {willow};
\node[Kanji] at (38.300000, -8.700000) {\textcolor[HTML]{b0b0b5}{卵}};
\node[Square] at (38.300000, -9.200000) {};
\node[Onyomi] at (38.350000, -9.100000) {\hbox{\tate ラン}};
\node[Kunyomi] at (38.250000, -9.100000) {\hbox{\tate たまご}};
\node[Meaning] at (38.300000, -7.450000) {egg};
\node[Kanji] at (40.350000, -8.700000) {\textcolor[HTML]{d2a293}{留}};
\node[Square] at (40.350000, -9.200000) {};
\node[Onyomi] at (40.400000, -9.100000) {\hbox{\tate ル・リュウ}};
\node[Kunyomi] at (40.300000, -9.100000) {\hbox{\tate と}};
\node[Meaning] at (40.350000, -7.450000) {detain};
\node[Kanji] at (42.400000, -8.700000) {\textcolor[HTML]{91b7c3}{貿}};
\node[Square] at (42.400000, -9.200000) {};
\node[Onyomi] at (42.450000, -9.100000) {\hbox{\tate ボウ}};
\node[Meaning] at (42.400000, -7.450000) {trade};
\node[Kanji] at (44.450000, -8.700000) {\textcolor[HTML]{c8a59d}{印}};
\node[Square] at (44.450000, -9.200000) {};
\node[Onyomi] at (44.500000, -9.100000) {\hbox{\tate イン}};
\node[Kunyomi] at (44.400000, -9.100000) {\hbox{\tate しるし}};
\node[Meaning] at (44.450000, -7.450000) {seal};
\node[Kanji] at (46.500000, -8.700000) {\textcolor[HTML]{d2a293}{興}};
\node[Square] at (46.500000, -9.200000) {};
\node[Onyomi] at (46.550000, -9.100000) {\hbox{\tate キョウ・コウ}};
\node[Meaning] at (46.500000, -7.450000) {interest};
\node[Kanji] at (48.550000, -8.700000) {\textcolor[HTML]{c8a59d}{酒}};
\node[Square] at (48.550000, -9.200000) {};
\node[Onyomi] at (48.600000, -9.100000) {\hbox{\tate シュ}};
\node[Kunyomi] at (48.500000, -9.100000) {\hbox{\tate さけ・さか}};
\node[Meaning] at (48.550000, -7.450000) {alcohol};
\node[Kanji] at (50.600000, -8.700000) {\textcolor[HTML]{29409e}{酌}};
\node[Square] at (50.600000, -9.200000) {};
\node[Onyomi] at (50.650000, -9.100000) {\hbox{\tate シャク}};
\node[Kunyomi] at (50.550000, -9.100000) {\hbox{\tate く}};
\node[Meaning] at (50.600000, -7.450000) {serve};
\node[Kanji] at (52.650000, -8.700000) {\textcolor[HTML]{91b7c3}{酵}};
\node[Square] at (52.650000, -9.200000) {};
\node[Onyomi] at (52.700000, -9.100000) {\hbox{\tate コウ}};
\node[Meaning] at (52.650000, -7.450000) {ferment};
\node[Kanji] at (54.700000, -8.700000) {\textcolor[HTML]{91b7c3}{酷}};
\node[Square] at (54.700000, -9.200000) {};
\node[Onyomi] at (54.750000, -9.100000) {\hbox{\tate コク}};
\node[Kunyomi] at (54.650000, -9.100000) {\hbox{\tate ひど}};
\node[Meaning] at (54.700000, -7.450000) {cruel};
\node[Kanji] at (56.750000, -8.700000) {\textcolor[HTML]{68a4bc}{酬}};
\node[Square] at (56.750000, -9.200000) {};
\node[Onyomi] at (56.800000, -9.100000) {\hbox{\tate シュウ・シュ}};
\node[Kunyomi] at (56.700000, -9.100000) {\hbox{\tate むく}};
\node[Meaning] at (56.750000, -7.450000) {repay};
\node[Meaning] at (-58.050000, -8.600000) {75.98\%};
\node[Kanji] at (-56.000000, -10.750000) {\textcolor[HTML]{1059be}{酪}};
\node[Square] at (-56.000000, -11.250000) {};
\node[Onyomi] at (-55.950000, -11.150000) {\hbox{\tate ラク}};
\node[Meaning] at (-56.000000, -9.500000) {dairy};
\node[Kanji] at (-53.950000, -10.750000) {\textcolor[HTML]{408dba}{酢}};
\node[Square] at (-53.950000, -11.250000) {};
\node[Kunyomi] at (-54.000000, -11.150000) {\hbox{\tate す}};
\node[Meaning] at (-53.950000, -9.500000) {vinegar};
\node[Kanji] at (-51.900000, -10.750000) {\textcolor[HTML]{68a4bc}{酔}};
\node[Square] at (-51.900000, -11.250000) {};
\node[Onyomi] at (-51.850000, -11.150000) {\hbox{\tate スイ}};
\node[Kunyomi] at (-51.950000, -11.150000) {\hbox{\tate よ.う}};
\node[Meaning] at (-51.900000, -9.500000) {drunk};
\node[Kanji] at (-49.850000, -10.750000) {\textcolor[HTML]{cd8268}{配}};
\node[Square] at (-49.850000, -11.250000) {};
\node[Onyomi] at (-49.800000, -11.150000) {\hbox{\tate ハイ}};
\node[Kunyomi] at (-49.900000, -11.150000) {\hbox{\tate くば.る}};
\node[Meaning] at (-49.850000, -9.500000) {distribute};
\node[Kanji] at (-47.800000, -10.750000) {\textcolor[HTML]{d2a293}{酸}};
\node[Square] at (-47.800000, -11.250000) {};
\node[Onyomi] at (-47.750000, -11.150000) {\hbox{\tate サン}};
\node[Kunyomi] at (-47.850000, -11.150000) {\hbox{\tate す}};
\node[Meaning] at (-47.800000, -9.500000) {acid};
\node[Kanji] at (-45.750000, -10.750000) {\textcolor[HTML]{408dba}{猶}};
\node[Square] at (-45.750000, -11.250000) {};
\node[Onyomi] at (-45.700000, -11.150000) {\hbox{\tate ユウ・ユ}};
\node[Kunyomi] at (-45.800000, -11.150000) {\hbox{\tate なお}};
\node[Meaning] at (-45.750000, -9.500000) {still};
\node[Kanji] at (-43.700000, -10.750000) {\textcolor[HTML]{b0b0b5}{尊}};
\node[Square] at (-43.700000, -11.250000) {};
\node[Onyomi] at (-43.650000, -11.150000) {\hbox{\tate ソン}};
\node[Kunyomi] at (-43.750000, -11.150000) {\hbox{\tate とうと.い}};
\node[Meaning] at (-43.700000, -9.500000) {revered};
\node[Kanji] at (-41.650000, -10.750000) {\textcolor[HTML]{a3bac2}{豆}};
\node[Square] at (-41.650000, -11.250000) {};
\node[Onyomi] at (-41.600000, -11.150000) {\hbox{\tate トウ}};
\node[Kunyomi] at (-41.700000, -11.150000) {\hbox{\tate まめ}};
\node[Meaning] at (-41.650000, -9.500000) {beans};
\node[Kanji] at (-39.600000, -10.750000) {\textcolor[HTML]{d69f8d}{頭}};
\node[Square] at (-39.600000, -11.250000) {};
\node[Onyomi] at (-39.550000, -11.150000) {\hbox{\tate ズ・トウ}};
\node[Kunyomi] at (-39.650000, -11.150000) {\hbox{\tate あたま}};
\node[Meaning] at (-39.600000, -9.500000) {head};
\node[Kanji] at (-37.550000, -10.750000) {\textcolor[HTML]{d2a293}{短}};
\node[Square] at (-37.550000, -11.250000) {};
\node[Onyomi] at (-37.500000, -11.150000) {\hbox{\tate タン}};
\node[Kunyomi] at (-37.600000, -11.150000) {\hbox{\tate みじか.い}};
\node[Meaning] at (-37.550000, -9.500000) {short};
\node[Kanji] at (-35.500000, -10.750000) {\textcolor[HTML]{d2a293}{豊}};
\node[Square] at (-35.500000, -11.250000) {};
\node[Onyomi] at (-35.450000, -11.150000) {\hbox{\tate ホウ}};
\node[Kunyomi] at (-35.550000, -11.150000) {\hbox{\tate ゆた.か}};
\node[Meaning] at (-35.500000, -9.500000) {plentiful};
\node[Kanji] at (-33.450000, -10.750000) {\textcolor[HTML]{91b7c3}{鼓}};
\node[Square] at (-33.450000, -11.250000) {};
\node[Onyomi] at (-33.400000, -11.150000) {\hbox{\tate コ}};
\node[Kunyomi] at (-33.500000, -11.150000) {\hbox{\tate つづみ}};
\node[Meaning] at (-33.450000, -9.500000) {beat};
\node[Kanji] at (-31.400000, -10.750000) {\textcolor[HTML]{c8a59d}{喜}};
\node[Square] at (-31.400000, -11.250000) {};
\node[Onyomi] at (-31.350000, -11.150000) {\hbox{\tate キ}};
\node[Kunyomi] at (-31.450000, -11.150000) {\hbox{\tate よろこ}};
\node[Meaning] at (-31.400000, -9.500000) {rejoice};
\node[Kanji] at (-29.350000, -10.750000) {\textcolor[HTML]{c8a59d}{樹}};
\node[Square] at (-29.350000, -11.250000) {};
\node[Onyomi] at (-29.300000, -11.150000) {\hbox{\tate ジュ}};
\node[Kunyomi] at (-29.400000, -11.150000) {\hbox{\tate き}};
\node[Meaning] at (-29.350000, -9.500000) {wood};
\node[Kanji] at (-27.300000, -10.750000) {\textcolor[HTML]{408dba}{皿}};
\node[Square] at (-27.300000, -11.250000) {};
\node[Kunyomi] at (-27.350000, -11.150000) {\hbox{\tate さら}};
\node[Meaning] at (-27.300000, -9.500000) {plate};
\node[Kanji] at (-25.250000, -10.750000) {\textcolor[HTML]{d2a293}{血}};
\node[Square] at (-25.250000, -11.250000) {};
\node[Onyomi] at (-25.200000, -11.150000) {\hbox{\tate ケツ}};
\node[Kunyomi] at (-25.300000, -11.150000) {\hbox{\tate ち}};
\node[Meaning] at (-25.250000, -9.500000) {blood};
\node[Kanji] at (-23.200000, -10.750000) {\textcolor[HTML]{91b7c3}{盆}};
\node[Square] at (-23.200000, -11.250000) {};
\node[Onyomi] at (-23.150000, -11.150000) {\hbox{\tate ボン}};
\node[Meaning] at (-23.200000, -9.500000) {lantern festival};
\node[Kanji] at (-21.150000, -10.750000) {\textcolor[HTML]{c8a59d}{盟}};
\node[Square] at (-21.150000, -11.250000) {};
\node[Onyomi] at (-21.100000, -11.150000) {\hbox{\tate メイ}};
\node[Meaning] at (-21.150000, -9.500000) {alliance};
\node[Kanji] at (-19.100000, -10.750000) {\textcolor[HTML]{a3bac2}{盗}};
\node[Square] at (-19.100000, -11.250000) {};
\node[Onyomi] at (-19.050000, -11.150000) {\hbox{\tate トウ}};
\node[Kunyomi] at (-19.150000, -11.150000) {\hbox{\tate ぬす.む}};
\node[Meaning] at (-19.100000, -9.500000) {steal};
\node[Kanji] at (-17.050000, -10.750000) {\textcolor[HTML]{d2a293}{温}};
\node[Square] at (-17.050000, -11.250000) {};
\node[Onyomi] at (-17.000000, -11.150000) {\hbox{\tate オン}};
\node[Kunyomi] at (-17.100000, -11.150000) {\hbox{\tate あたた.*}};
\node[Meaning] at (-17.050000, -9.500000) {warm};
\node[Kanji] at (-15.000000, -10.750000) {\textcolor[HTML]{d69f8d}{監}};
\node[Square] at (-15.000000, -11.250000) {};
\node[Onyomi] at (-14.950000, -11.150000) {\hbox{\tate カン}};
\node[Meaning] at (-15.000000, -9.500000) {oversee};
\node[Kanji] at (-12.950000, -10.750000) {\textcolor[HTML]{408dba}{濫}};
\node[Square] at (-12.950000, -11.250000) {};
\node[Onyomi] at (-12.900000, -11.150000) {\hbox{\tate ラン}};
\node[Meaning] at (-12.950000, -9.500000) {excessive};
\node[Kanji] at (-10.900000, -10.750000) {\textcolor[HTML]{b0b0b5}{鑑}};
\node[Square] at (-10.900000, -11.250000) {};
\node[Onyomi] at (-10.850000, -11.150000) {\hbox{\tate カン}};
\node[Meaning] at (-10.900000, -9.500000) {model};
\node[Kanji] at (-8.850000, -10.750000) {\textcolor[HTML]{91b7c3}{猛}};
\node[Square] at (-8.850000, -11.250000) {};
\node[Onyomi] at (-8.800000, -11.150000) {\hbox{\tate モウ}};
\node[Meaning] at (-8.850000, -9.500000) {fierce};
\node[Kanji] at (-6.800000, -10.750000) {\textcolor[HTML]{c8a59d}{盛}};
\node[Square] at (-6.800000, -11.250000) {};
\node[Onyomi] at (-6.750000, -11.150000) {\hbox{\tate セイ・ジョウ}};
\node[Kunyomi] at (-6.850000, -11.150000) {\hbox{\tate も.る}};
\node[Meaning] at (-6.800000, -9.500000) {pile};
\node[Kanji] at (-4.750000, -10.750000) {\textcolor[HTML]{c8a59d}{塩}};
\node[Square] at (-4.750000, -11.250000) {};
\node[Onyomi] at (-4.700000, -11.150000) {\hbox{\tate エン}};
\node[Kunyomi] at (-4.800000, -11.150000) {\hbox{\tate しお}};
\node[Meaning] at (-4.750000, -9.500000) {salt};
\node[Kanji] at (-2.700000, -10.750000) {\textcolor[HTML]{d2a293}{銀}};
\node[Square] at (-2.700000, -11.250000) {};
\node[Onyomi] at (-2.650000, -11.150000) {\hbox{\tate ギン}};
\node[Meaning] at (-2.700000, -9.500000) {silver};
\node[Kanji] at (-0.650000, -10.750000) {\textcolor[HTML]{408dba}{恨}};
\node[Square] at (-0.650000, -11.250000) {};
\node[Onyomi] at (-0.600000, -11.150000) {\hbox{\tate コン}};
\node[Kunyomi] at (-0.700000, -11.150000) {\hbox{\tate うら.む}};
\node[Meaning] at (-0.650000, -9.500000) {grudge};
\node[Kanji] at (1.400000, -10.750000) {\textcolor[HTML]{d2a293}{根}};
\node[Square] at (1.400000, -11.250000) {};
\node[Onyomi] at (1.450000, -11.150000) {\hbox{\tate コン}};
\node[Kunyomi] at (1.350000, -11.150000) {\hbox{\tate ね}};
\node[Meaning] at (1.400000, -9.500000) {root};
\node[Kanji] at (3.450000, -10.750000) {\textcolor[HTML]{c8a59d}{即}};
\node[Square] at (3.450000, -11.250000) {};
\node[Onyomi] at (3.500000, -11.150000) {\hbox{\tate ソク}};
\node[Kunyomi] at (3.400000, -11.150000) {\hbox{\tate すなわ.ち}};
\node[Meaning] at (3.450000, -9.500000) {instant};
\node[Kanji] at (5.500000, -10.750000) {\textcolor[HTML]{b0b0b5}{爵}};
\node[Square] at (5.500000, -11.250000) {};
\node[Onyomi] at (5.550000, -11.150000) {\hbox{\tate シャク}};
\node[Meaning] at (5.500000, -9.500000) {baron};
\node[Kanji] at (7.550000, -10.750000) {\textcolor[HTML]{d2a293}{節}};
\node[Square] at (7.550000, -11.250000) {};
\node[Onyomi] at (7.600000, -11.150000) {\hbox{\tate セツ}};
\node[Kunyomi] at (7.500000, -11.150000) {\hbox{\tate ふし}};
\node[Meaning] at (7.550000, -9.500000) {season};
\node[Kanji] at (9.600000, -10.750000) {\textcolor[HTML]{d69f8d}{退}};
\node[Square] at (9.600000, -11.250000) {};
\node[Onyomi] at (9.650000, -11.150000) {\hbox{\tate タイ}};
\node[Kunyomi] at (9.550000, -11.150000) {\hbox{\tate しりぞ.く}};
\node[Meaning] at (9.600000, -9.500000) {retreat};
\node[Kanji] at (11.650000, -10.750000) {\textcolor[HTML]{d69f8d}{限}};
\node[Square] at (11.650000, -11.250000) {};
\node[Onyomi] at (11.700000, -11.150000) {\hbox{\tate ゲン}};
\node[Kunyomi] at (11.600000, -11.150000) {\hbox{\tate かぎ.る}};
\node[Meaning] at (11.650000, -9.500000) {limit};
\node[Kanji] at (13.700000, -10.750000) {\textcolor[HTML]{b0b0b5}{眼}};
\node[Square] at (13.700000, -11.250000) {};
\node[Onyomi] at (13.750000, -11.150000) {\hbox{\tate ガン}};
\node[Kunyomi] at (13.650000, -11.150000) {\hbox{\tate め}};
\node[Meaning] at (13.700000, -9.500000) {eyeball};
\node[Kanji] at (15.750000, -10.750000) {\textcolor[HTML]{d69f8d}{良}};
\node[Square] at (15.750000, -11.250000) {};
\node[Onyomi] at (15.800000, -11.150000) {\hbox{\tate リョウ}};
\node[Kunyomi] at (15.700000, -11.150000) {\hbox{\tate よ・い}};
\node[Meaning] at (15.750000, -9.500000) {good};
\node[Kanji] at (17.800000, -10.750000) {\textcolor[HTML]{91b7c3}{朗}};
\node[Square] at (17.800000, -11.250000) {};
\node[Onyomi] at (17.850000, -11.150000) {\hbox{\tate ロウ}};
\node[Kunyomi] at (17.750000, -11.150000) {\hbox{\tate ほが.らか}};
\node[Meaning] at (17.800000, -9.500000) {bright};
\node[Kanji] at (19.850000, -10.750000) {\textcolor[HTML]{a3bac2}{浪}};
\node[Square] at (19.850000, -11.250000) {};
\node[Onyomi] at (19.900000, -11.150000) {\hbox{\tate ロウ}};
\node[Meaning] at (19.850000, -9.500000) {wander};
\node[Kanji] at (21.900000, -10.750000) {\textcolor[HTML]{c8a59d}{娘}};
\node[Square] at (21.900000, -11.250000) {};
\node[Kunyomi] at (21.850000, -11.150000) {\hbox{\tate むすめ}};
\node[Meaning] at (21.900000, -9.500000) {daughter};
\node[Kanji] at (23.950000, -10.750000) {\textcolor[HTML]{d69f8d}{食}};
\node[Square] at (23.950000, -11.250000) {};
\node[Onyomi] at (24.000000, -11.150000) {\hbox{\tate ショク}};
\node[Kunyomi] at (23.900000, -11.150000) {\hbox{\tate た.べる・く}};
\node[Meaning] at (23.950000, -9.500000) {eat};
\node[Kanji] at (26.000000, -10.750000) {\textcolor[HTML]{b0b0b5}{飯}};
\node[Square] at (26.000000, -11.250000) {};
\node[Onyomi] at (26.050000, -11.150000) {\hbox{\tate ハン}};
\node[Kunyomi] at (25.950000, -11.150000) {\hbox{\tate めし}};
\node[Meaning] at (26.000000, -9.500000) {meal};
\node[Kanji] at (28.050000, -10.750000) {\textcolor[HTML]{b0b0b5}{飲}};
\node[Square] at (28.050000, -11.250000) {};
\node[Onyomi] at (28.100000, -11.150000) {\hbox{\tate イン}};
\node[Kunyomi] at (28.000000, -11.150000) {\hbox{\tate の}};
\node[Meaning] at (28.050000, -9.500000) {drink};
\node[Kanji] at (30.100000, -10.750000) {\textcolor[HTML]{408dba}{飢}};
\node[Square] at (30.100000, -11.250000) {};
\node[Onyomi] at (30.150000, -11.150000) {\hbox{\tate キ}};
\node[Kunyomi] at (30.050000, -11.150000) {\hbox{\tate う.える}};
\node[Meaning] at (30.100000, -9.500000) {starve};
\node[Kanji] at (32.150000, -10.750000) {\textcolor[HTML]{1e76bb}{餓}};
\node[Square] at (32.150000, -11.250000) {};
\node[Onyomi] at (32.200000, -11.150000) {\hbox{\tate ガ}};
\node[Kunyomi] at (32.100000, -11.150000) {\hbox{\tate う.える}};
\node[Meaning] at (32.150000, -9.500000) {starve};
\node[Kanji] at (34.200000, -10.750000) {\textcolor[HTML]{b0b0b5}{飾}};
\node[Square] at (34.200000, -11.250000) {};
\node[Onyomi] at (34.250000, -11.150000) {\hbox{\tate ショク}};
\node[Kunyomi] at (34.150000, -11.150000) {\hbox{\tate かざ.る}};
\node[Meaning] at (34.200000, -9.500000) {decorate};
\node[Kanji] at (36.250000, -10.750000) {\textcolor[HTML]{cd8268}{館}};
\node[Square] at (36.250000, -11.250000) {};
\node[Onyomi] at (36.300000, -11.150000) {\hbox{\tate カン}};
\node[Meaning] at (36.250000, -9.500000) {public building};
\node[Kanji] at (38.300000, -10.750000) {\textcolor[HTML]{d2a293}{養}};
\node[Square] at (38.300000, -11.250000) {};
\node[Onyomi] at (38.350000, -11.150000) {\hbox{\tate ヨウ}};
\node[Kunyomi] at (38.250000, -11.150000) {\hbox{\tate やしな.う}};
\node[Meaning] at (38.300000, -9.500000) {foster};
\node[Kanji] at (40.350000, -10.750000) {\textcolor[HTML]{408dba}{飽}};
\node[Square] at (40.350000, -11.250000) {};
\node[Onyomi] at (40.400000, -11.150000) {\hbox{\tate ホウ}};
\node[Kunyomi] at (40.300000, -11.150000) {\hbox{\tate あ}};
\node[Meaning] at (40.350000, -9.500000) {bored};
\node[Kanji] at (42.400000, -10.750000) {\textcolor[HTML]{c8a59d}{既}};
\node[Square] at (42.400000, -11.250000) {};
\node[Onyomi] at (42.450000, -11.150000) {\hbox{\tate キ}};
\node[Kunyomi] at (42.350000, -11.150000) {\hbox{\tate すで}};
\node[Meaning] at (42.400000, -9.500000) {previously};
\node[Kanji] at (44.450000, -10.750000) {\textcolor[HTML]{c8a59d}{概}};
\node[Square] at (44.450000, -11.250000) {};
\node[Onyomi] at (44.500000, -11.150000) {\hbox{\tate ガイ}};
\node[Kunyomi] at (44.400000, -11.150000) {\hbox{\tate おおむ.ね}};
\node[Meaning] at (44.450000, -9.500000) {approximation};
\node[Kanji] at (46.500000, -10.750000) {\textcolor[HTML]{1059be}{慨}};
\node[Square] at (46.500000, -11.250000) {};
\node[Onyomi] at (46.550000, -11.150000) {\hbox{\tate ガイ}};
\node[Meaning] at (46.500000, -9.500000) {sigh};
\node[Kanji] at (48.550000, -10.750000) {\textcolor[HTML]{c36143}{平}};
\node[Square] at (48.550000, -11.250000) {};
\node[Onyomi] at (48.600000, -11.150000) {\hbox{\tate ヘイ・ヒョウ}};
\node[Kunyomi] at (48.500000, -11.150000) {\hbox{\tate たいら・ひら}};
\node[Meaning] at (48.550000, -9.500000) {flat};
\node[Kanji] at (50.600000, -10.750000) {\textcolor[HTML]{cd8268}{呼}};
\node[Square] at (50.600000, -11.250000) {};
\node[Onyomi] at (50.650000, -11.150000) {\hbox{\tate コ}};
\node[Kunyomi] at (50.550000, -11.150000) {\hbox{\tate よ}};
\node[Meaning] at (50.600000, -9.500000) {call};
\node[Kanji] at (52.650000, -10.750000) {\textcolor[HTML]{68a4bc}{坪}};
\node[Square] at (52.650000, -11.250000) {};
\node[Onyomi] at (52.700000, -11.150000) {\hbox{\tate ヘイ}};
\node[Kunyomi] at (52.600000, -11.150000) {\hbox{\tate つぼ}};
\node[Meaning] at (52.650000, -9.500000) {two mat area};
\node[Kanji] at (54.700000, -10.750000) {\textcolor[HTML]{d69f8d}{評}};
\node[Square] at (54.700000, -11.250000) {};
\node[Onyomi] at (54.750000, -11.150000) {\hbox{\tate ヒョウ}};
\node[Meaning] at (54.700000, -9.500000) {evaluate};
\node[Kanji] at (56.750000, -10.750000) {\textcolor[HTML]{68a4bc}{刈}};
\node[Square] at (56.750000, -11.250000) {};
\node[Kunyomi] at (56.700000, -11.150000) {\hbox{\tate か}};
\node[Meaning] at (56.750000, -9.500000) {prune};
\node[Meaning] at (-58.050000, -10.650000) {77.54\%};
\node[Kanji] at (-56.000000, -12.800000) {\textcolor[HTML]{b0b0b5}{希}};
\node[Square] at (-56.000000, -13.300000) {};
\node[Onyomi] at (-55.950000, -13.200000) {\hbox{\tate キ}};
\node[Kunyomi] at (-56.050000, -13.200000) {\hbox{\tate まれ}};
\node[Meaning] at (-56.000000, -11.550000) {wish};
\node[Kanji] at (-53.950000, -12.800000) {\textcolor[HTML]{408dba}{凶}};
\node[Square] at (-53.950000, -13.300000) {};
\node[Onyomi] at (-53.900000, -13.200000) {\hbox{\tate キョウ}};
\node[Meaning] at (-53.950000, -11.550000) {villain};
\node[Kanji] at (-51.900000, -12.800000) {\textcolor[HTML]{a3bac2}{胸}};
\node[Square] at (-51.900000, -13.300000) {};
\node[Onyomi] at (-51.850000, -13.200000) {\hbox{\tate キョウ}};
\node[Kunyomi] at (-51.950000, -13.200000) {\hbox{\tate むね}};
\node[Meaning] at (-51.900000, -11.550000) {chest};
\node[Kanji] at (-49.850000, -12.800000) {\textcolor[HTML]{d69f8d}{離}};
\node[Square] at (-49.850000, -13.300000) {};
\node[Onyomi] at (-49.800000, -13.200000) {\hbox{\tate リ}};
\node[Kunyomi] at (-49.900000, -13.200000) {\hbox{\tate はな.*}};
\node[Meaning] at (-49.850000, -11.550000) {detach};
\node[Kanji] at (-47.800000, -12.800000) {\textcolor[HTML]{d2a293}{殺}};
\node[Square] at (-47.800000, -13.300000) {};
\node[Onyomi] at (-47.750000, -13.200000) {\hbox{\tate サツ}};
\node[Kunyomi] at (-47.850000, -13.200000) {\hbox{\tate ころ.す}};
\node[Meaning] at (-47.800000, -11.550000) {kill};
\node[Kanji] at (-45.750000, -12.800000) {\textcolor[HTML]{c8a59d}{純}};
\node[Square] at (-45.750000, -13.300000) {};
\node[Onyomi] at (-45.700000, -13.200000) {\hbox{\tate ジュン}};
\node[Meaning] at (-45.750000, -11.550000) {pure};
\node[Kanji] at (-43.700000, -12.800000) {\textcolor[HTML]{1e76bb}{鈍}};
\node[Square] at (-43.700000, -13.300000) {};
\node[Onyomi] at (-43.650000, -13.200000) {\hbox{\tate ドン}};
\node[Kunyomi] at (-43.750000, -13.200000) {\hbox{\tate にぶ.い}};
\node[Meaning] at (-43.700000, -11.550000) {dull};
\node[Kanji] at (-41.650000, -12.800000) {\textcolor[HTML]{91b7c3}{辛}};
\node[Square] at (-41.650000, -13.300000) {};
\node[Onyomi] at (-41.600000, -13.200000) {\hbox{\tate シン}};
\node[Kunyomi] at (-41.700000, -13.200000) {\hbox{\tate から.い}};
\node[Meaning] at (-41.650000, -11.550000) {spicy};
\node[Kanji] at (-39.600000, -12.800000) {\textcolor[HTML]{c8a59d}{辞}};
\node[Square] at (-39.600000, -13.300000) {};
\node[Onyomi] at (-39.550000, -13.200000) {\hbox{\tate ジ}};
\node[Kunyomi] at (-39.650000, -13.200000) {\hbox{\tate や.める}};
\node[Meaning] at (-39.600000, -11.550000) {quit};
\node[Kanji] at (-37.550000, -12.800000) {\textcolor[HTML]{1e76bb}{梓}};
\node[Square] at (-37.550000, -13.300000) {};
\node[Onyomi] at (-37.500000, -13.200000) {\hbox{\tate シ}};
\node[Kunyomi] at (-37.600000, -13.200000) {\hbox{\tate あずさ}};
\node[Meaning] at (-37.550000, -11.550000) {wood block};
\node[Kanji] at (-35.500000, -12.800000) {\textcolor[HTML]{91b7c3}{宰}};
\node[Square] at (-35.500000, -13.300000) {};
\node[Onyomi] at (-35.450000, -13.200000) {\hbox{\tate サイ}};
\node[Meaning] at (-35.500000, -11.550000) {manager};
\node[Kanji] at (-33.450000, -12.800000) {\textcolor[HTML]{b0b0b5}{壁}};
\node[Square] at (-33.450000, -13.300000) {};
\node[Onyomi] at (-33.400000, -13.200000) {\hbox{\tate ヘキ}};
\node[Kunyomi] at (-33.500000, -13.200000) {\hbox{\tate かべ}};
\node[Meaning] at (-33.450000, -11.550000) {wall};
\node[Kanji] at (-31.400000, -12.800000) {\textcolor[HTML]{c8a59d}{避}};
\node[Square] at (-31.400000, -13.300000) {};
\node[Onyomi] at (-31.350000, -13.200000) {\hbox{\tate ヒ}};
\node[Kunyomi] at (-31.450000, -13.200000) {\hbox{\tate さ.ける}};
\node[Meaning] at (-31.400000, -11.550000) {dodge};
\node[Kanji] at (-29.350000, -12.800000) {\textcolor[HTML]{a11d25}{新}};
\node[Square] at (-29.350000, -13.300000) {};
\node[Onyomi] at (-29.300000, -13.200000) {\hbox{\tate シン}};
\node[Kunyomi] at (-29.400000, -13.200000) {\hbox{\tate あたら.しい}};
\node[Meaning] at (-29.350000, -11.550000) {new};
\node[Kanji] at (-27.300000, -12.800000) {\textcolor[HTML]{1e76bb}{薪}};
\node[Square] at (-27.300000, -13.300000) {};
\node[Onyomi] at (-27.250000, -13.200000) {\hbox{\tate シン}};
\node[Kunyomi] at (-27.350000, -13.200000) {\hbox{\tate たきぎ}};
\node[Meaning] at (-27.300000, -11.550000) {fuel};
\node[Kanji] at (-25.250000, -12.800000) {\textcolor[HTML]{d69f8d}{親}};
\node[Square] at (-25.250000, -13.300000) {};
\node[Onyomi] at (-25.200000, -13.200000) {\hbox{\tate シン}};
\node[Kunyomi] at (-25.300000, -13.200000) {\hbox{\tate おや}};
\node[Meaning] at (-25.250000, -11.550000) {parent};
\node[Kanji] at (-23.200000, -12.800000) {\textcolor[HTML]{c8a59d}{幸}};
\node[Square] at (-23.200000, -13.300000) {};
\node[Onyomi] at (-23.150000, -13.200000) {\hbox{\tate コウ}};
\node[Kunyomi] at (-23.250000, -13.200000) {\hbox{\tate しあわ.せ}};
\node[Meaning] at (-23.200000, -11.550000) {happiness};
\node[Kanji] at (-21.150000, -12.800000) {\textcolor[HTML]{c8a59d}{執}};
\node[Square] at (-21.150000, -13.300000) {};
\node[Onyomi] at (-21.100000, -13.200000) {\hbox{\tate シュウ・シツ}};
\node[Kunyomi] at (-21.200000, -13.200000) {\hbox{\tate と.る}};
\node[Meaning] at (-21.150000, -11.550000) {tenacious};
\node[Kanji] at (-19.100000, -12.800000) {\textcolor[HTML]{cd8268}{報}};
\node[Square] at (-19.100000, -13.300000) {};
\node[Onyomi] at (-19.050000, -13.200000) {\hbox{\tate ホウ}};
\node[Kunyomi] at (-19.150000, -13.200000) {\hbox{\tate むく.いる}};
\node[Meaning] at (-19.100000, -11.550000) {news};
\node[Kanji] at (-17.050000, -12.800000) {\textcolor[HTML]{68a4bc}{叫}};
\node[Square] at (-17.050000, -13.300000) {};
\node[Onyomi] at (-17.000000, -13.200000) {\hbox{\tate キョウ}};
\node[Kunyomi] at (-17.100000, -13.200000) {\hbox{\tate さけ.ぶ}};
\node[Meaning] at (-17.050000, -11.550000) {shout};
\node[Kanji] at (-15.000000, -12.800000) {\textcolor[HTML]{1e76bb}{糾}};
\node[Square] at (-15.000000, -13.300000) {};
\node[Onyomi] at (-14.950000, -13.200000) {\hbox{\tate キュウ}};
\node[Meaning] at (-15.000000, -11.550000) {twist};
\node[Kanji] at (-12.950000, -12.800000) {\textcolor[HTML]{cd8268}{収}};
\node[Square] at (-12.950000, -13.300000) {};
\node[Onyomi] at (-12.900000, -13.200000) {\hbox{\tate シュウ}};
\node[Kunyomi] at (-13.000000, -13.200000) {\hbox{\tate おさ.める}};
\node[Meaning] at (-12.950000, -11.550000) {obtain};
\node[Kanji] at (-10.900000, -12.800000) {\textcolor[HTML]{408dba}{卑}};
\node[Square] at (-10.900000, -13.300000) {};
\node[Onyomi] at (-10.850000, -13.200000) {\hbox{\tate ヒ}};
\node[Kunyomi] at (-10.950000, -13.200000) {\hbox{\tate いや}};
\node[Meaning] at (-10.900000, -11.550000) {lowly};
\node[Kanji] at (-8.850000, -12.800000) {\textcolor[HTML]{a3bac2}{碑}};
\node[Square] at (-8.850000, -13.300000) {};
\node[Onyomi] at (-8.800000, -13.200000) {\hbox{\tate ヒ}};
\node[Kunyomi] at (-8.900000, -13.200000) {\hbox{\tate いしぶみ}};
\node[Meaning] at (-8.850000, -11.550000) {tombstone};
\node[Kanji] at (-6.800000, -12.800000) {\textcolor[HTML]{d69f8d}{陸}};
\node[Square] at (-6.800000, -13.300000) {};
\node[Onyomi] at (-6.750000, -13.200000) {\hbox{\tate リク}};
\node[Meaning] at (-6.800000, -11.550000) {land};
\node[Kanji] at (-4.750000, -12.800000) {\textcolor[HTML]{68a4bc}{睦}};
\node[Square] at (-4.750000, -13.300000) {};
\node[Onyomi] at (-4.700000, -13.200000) {\hbox{\tate ボク・モク}};
\node[Kunyomi] at (-4.800000, -13.200000) {\hbox{\tate むつ}};
\node[Meaning] at (-4.750000, -11.550000) {friendly};
\node[Kanji] at (-2.700000, -12.800000) {\textcolor[HTML]{d69f8d}{勢}};
\node[Square] at (-2.700000, -13.300000) {};
\node[Onyomi] at (-2.650000, -13.200000) {\hbox{\tate セイ}};
\node[Kunyomi] at (-2.750000, -13.200000) {\hbox{\tate いきお.い}};
\node[Meaning] at (-2.700000, -11.550000) {force};
\node[Kanji] at (-0.650000, -12.800000) {\textcolor[HTML]{d2a293}{熱}};
\node[Square] at (-0.650000, -13.300000) {};
\node[Onyomi] at (-0.600000, -13.200000) {\hbox{\tate ネツ}};
\node[Kunyomi] at (-0.700000, -13.200000) {\hbox{\tate あつ.い}};
\node[Meaning] at (-0.650000, -11.550000) {heat};
\node[Kanji] at (1.400000, -12.800000) {\textcolor[HTML]{a3bac2}{陵}};
\node[Square] at (1.400000, -13.300000) {};
\node[Onyomi] at (1.450000, -13.200000) {\hbox{\tate リョウ}};
\node[Kunyomi] at (1.350000, -13.200000) {\hbox{\tate みささぎ}};
\node[Meaning] at (1.400000, -11.550000) {mausoleum};
\node[Kanji] at (3.450000, -12.800000) {\textcolor[HTML]{c8a59d}{核}};
\node[Square] at (3.450000, -13.300000) {};
\node[Onyomi] at (3.500000, -13.200000) {\hbox{\tate カク}};
\node[Kunyomi] at (3.400000, -13.200000) {\hbox{\tate かく}};
\node[Meaning] at (3.450000, -11.550000) {nucleus};
\node[Kanji] at (5.500000, -12.800000) {\textcolor[HTML]{c8a59d}{刻}};
\node[Square] at (5.500000, -13.300000) {};
\node[Onyomi] at (5.550000, -13.200000) {\hbox{\tate コク}};
\node[Kunyomi] at (5.450000, -13.200000) {\hbox{\tate きざ.む}};
\node[Meaning] at (5.500000, -11.550000) {carve};
\node[Kanji] at (7.550000, -12.800000) {\textcolor[HTML]{b0b0b5}{該}};
\node[Square] at (7.550000, -13.300000) {};
\node[Onyomi] at (7.600000, -13.200000) {\hbox{\tate ガイ}};
\node[Meaning] at (7.550000, -11.550000) {the above};
\node[Kanji] at (9.600000, -12.800000) {\textcolor[HTML]{1059be}{劾}};
\node[Square] at (9.600000, -13.300000) {};
\node[Onyomi] at (9.650000, -13.200000) {\hbox{\tate ガイ}};
\node[Meaning] at (9.600000, -11.550000) {censure};
\node[Kanji] at (11.650000, -12.800000) {\textcolor[HTML]{d69f8d}{述}};
\node[Square] at (11.650000, -13.300000) {};
\node[Onyomi] at (11.700000, -13.200000) {\hbox{\tate ジュツ}};
\node[Kunyomi] at (11.600000, -13.200000) {\hbox{\tate の.べる}};
\node[Meaning] at (11.650000, -11.550000) {mention};
\node[Kanji] at (13.700000, -12.800000) {\textcolor[HTML]{cd8268}{術}};
\node[Square] at (13.700000, -13.300000) {};
\node[Onyomi] at (13.750000, -13.200000) {\hbox{\tate ジュツ}};
\node[Meaning] at (13.700000, -11.550000) {art};
\node[Kanji] at (15.750000, -12.800000) {\textcolor[HTML]{a3bac2}{寒}};
\node[Square] at (15.750000, -13.300000) {};
\node[Onyomi] at (15.800000, -13.200000) {\hbox{\tate カン}};
\node[Kunyomi] at (15.700000, -13.200000) {\hbox{\tate さむ}};
\node[Meaning] at (15.750000, -11.550000) {cold};
\node[Kanji] at (17.800000, -12.800000) {\textcolor[HTML]{68a4bc}{醸}};
\node[Square] at (17.800000, -13.300000) {};
\node[Onyomi] at (17.850000, -13.200000) {\hbox{\tate ジョウ}};
\node[Kunyomi] at (17.750000, -13.200000) {\hbox{\tate かも}};
\node[Meaning] at (17.800000, -11.550000) {brew};
\node[Kanji] at (19.850000, -12.800000) {\textcolor[HTML]{b0b0b5}{譲}};
\node[Square] at (19.850000, -13.300000) {};
\node[Onyomi] at (19.900000, -13.200000) {\hbox{\tate ジョウ}};
\node[Kunyomi] at (19.800000, -13.200000) {\hbox{\tate ゆず.る}};
\node[Meaning] at (19.850000, -11.550000) {defer};
\node[Kanji] at (21.900000, -12.800000) {\textcolor[HTML]{68a4bc}{壌}};
\node[Square] at (21.900000, -13.300000) {};
\node[Onyomi] at (21.950000, -13.200000) {\hbox{\tate ジョウ}};
\node[Kunyomi] at (21.850000, -13.200000) {\hbox{\tate つち}};
\node[Meaning] at (21.900000, -11.550000) {soil};
\node[Kanji] at (23.950000, -12.800000) {\textcolor[HTML]{1e76bb}{嬢}};
\node[Square] at (23.950000, -13.300000) {};
\node[Onyomi] at (24.000000, -13.200000) {\hbox{\tate ジョウ}};
\node[Kunyomi] at (23.900000, -13.200000) {\hbox{\tate むすめ}};
\node[Meaning] at (23.950000, -11.550000) {miss};
\node[Kanji] at (26.000000, -12.800000) {\textcolor[HTML]{a3bac2}{毒}};
\node[Square] at (26.000000, -13.300000) {};
\node[Onyomi] at (26.050000, -13.200000) {\hbox{\tate ドク}};
\node[Meaning] at (26.000000, -11.550000) {poison};
\node[Kanji] at (28.050000, -12.800000) {\textcolor[HTML]{d69f8d}{素}};
\node[Square] at (28.050000, -13.300000) {};
\node[Onyomi] at (28.100000, -13.200000) {\hbox{\tate ス・ソ}};
\node[Meaning] at (28.050000, -11.550000) {element};
\node[Kanji] at (30.100000, -12.800000) {\textcolor[HTML]{91b7c3}{麦}};
\node[Square] at (30.100000, -13.300000) {};
\node[Onyomi] at (30.150000, -13.200000) {\hbox{\tate バク}};
\node[Kunyomi] at (30.050000, -13.200000) {\hbox{\tate むぎ}};
\node[Meaning] at (30.100000, -11.550000) {wheat};
\node[Kanji] at (32.150000, -12.800000) {\textcolor[HTML]{d2a293}{青}};
\node[Square] at (32.150000, -13.300000) {};
\node[Onyomi] at (32.200000, -13.200000) {\hbox{\tate ショウ・セイ}};
\node[Kunyomi] at (32.100000, -13.200000) {\hbox{\tate あお}};
\node[Meaning] at (32.150000, -11.550000) {blue};
\node[Kanji] at (34.200000, -12.800000) {\textcolor[HTML]{d2a293}{精}};
\node[Square] at (34.200000, -13.300000) {};
\node[Onyomi] at (34.250000, -13.200000) {\hbox{\tate セイ}};
\node[Meaning] at (34.200000, -11.550000) {spirit};
\node[Kanji] at (36.250000, -12.800000) {\textcolor[HTML]{c8a59d}{請}};
\node[Square] at (36.250000, -13.300000) {};
\node[Onyomi] at (36.300000, -13.200000) {\hbox{\tate セイ・シン}};
\node[Kunyomi] at (36.200000, -13.200000) {\hbox{\tate う.ける}};
\node[Meaning] at (36.250000, -11.550000) {request};
\node[Kanji] at (38.300000, -12.800000) {\textcolor[HTML]{d69f8d}{情}};
\node[Square] at (38.300000, -13.300000) {};
\node[Onyomi] at (38.350000, -13.200000) {\hbox{\tate ジョウ}};
\node[Kunyomi] at (38.250000, -13.200000) {\hbox{\tate なさけ}};
\node[Meaning] at (38.300000, -11.550000) {feeling};
\node[Kanji] at (40.350000, -12.800000) {\textcolor[HTML]{b0b0b5}{晴}};
\node[Square] at (40.350000, -13.300000) {};
\node[Onyomi] at (40.400000, -13.200000) {\hbox{\tate セイ}};
\node[Kunyomi] at (40.300000, -13.200000) {\hbox{\tate は}};
\node[Meaning] at (40.350000, -11.550000) {clear up};
\node[Kanji] at (42.400000, -12.800000) {\textcolor[HTML]{d2a293}{清}};
\node[Square] at (42.400000, -13.300000) {};
\node[Onyomi] at (42.450000, -13.200000) {\hbox{\tate セイ・ショウ}};
\node[Kunyomi] at (42.350000, -13.200000) {\hbox{\tate きよ.い}};
\node[Meaning] at (42.400000, -11.550000) {pure};
\node[Kanji] at (44.450000, -12.800000) {\textcolor[HTML]{c8a59d}{静}};
\node[Square] at (44.450000, -13.300000) {};
\node[Onyomi] at (44.500000, -13.200000) {\hbox{\tate セイ}};
\node[Kunyomi] at (44.400000, -13.200000) {\hbox{\tate しず.か}};
\node[Meaning] at (44.450000, -11.550000) {quiet};
\node[Kanji] at (46.500000, -12.800000) {\textcolor[HTML]{b0b0b5}{責}};
\node[Square] at (46.500000, -13.300000) {};
\node[Onyomi] at (46.550000, -13.200000) {\hbox{\tate セキ}};
\node[Kunyomi] at (46.450000, -13.200000) {\hbox{\tate せ.める}};
\node[Meaning] at (46.500000, -11.550000) {blame};
\node[Kanji] at (48.550000, -12.800000) {\textcolor[HTML]{c8a59d}{績}};
\node[Square] at (48.550000, -13.300000) {};
\node[Onyomi] at (48.600000, -13.200000) {\hbox{\tate セキ}};
\node[Meaning] at (48.550000, -11.550000) {exploits};
\node[Kanji] at (50.600000, -12.800000) {\textcolor[HTML]{d69f8d}{積}};
\node[Square] at (50.600000, -13.300000) {};
\node[Onyomi] at (50.650000, -13.200000) {\hbox{\tate セキ}};
\node[Kunyomi] at (50.550000, -13.200000) {\hbox{\tate つ.む}};
\node[Meaning] at (50.600000, -11.550000) {accumulate};
\node[Kanji] at (52.650000, -12.800000) {\textcolor[HTML]{a3bac2}{債}};
\node[Square] at (52.650000, -13.300000) {};
\node[Onyomi] at (52.700000, -13.200000) {\hbox{\tate サイ}};
\node[Meaning] at (52.650000, -11.550000) {debt};
\node[Kanji] at (54.700000, -12.800000) {\textcolor[HTML]{408dba}{漬}};
\node[Square] at (54.700000, -13.300000) {};
\node[Onyomi] at (54.750000, -13.200000) {\hbox{\tate シ}};
\node[Kunyomi] at (54.650000, -13.200000) {\hbox{\tate つ}};
\node[Meaning] at (54.700000, -11.550000) {pickle};
\node[Kanji] at (56.750000, -12.800000) {\textcolor[HTML]{c36143}{表}};
\node[Square] at (56.750000, -13.300000) {};
\node[Onyomi] at (56.800000, -13.200000) {\hbox{\tate ヒョウ}};
\node[Kunyomi] at (56.700000, -13.200000) {\hbox{\tate あらわ.す}};
\node[Meaning] at (56.750000, -11.550000) {express};
\node[Meaning] at (-58.050000, -12.700000) {79.51\%};
\node[Kanji] at (-56.000000, -14.850000) {\textcolor[HTML]{68a4bc}{俵}};
\node[Square] at (-56.000000, -15.350000) {};
\node[Onyomi] at (-55.950000, -15.250000) {\hbox{\tate ヒョウ}};
\node[Kunyomi] at (-56.050000, -15.250000) {\hbox{\tate たわら}};
\node[Meaning] at (-56.000000, -13.600000) {sack};
\node[Kanji] at (-53.950000, -14.850000) {\textcolor[HTML]{68a4bc}{潔}};
\node[Square] at (-53.950000, -15.350000) {};
\node[Onyomi] at (-53.900000, -15.250000) {\hbox{\tate ケツ}};
\node[Kunyomi] at (-54.000000, -15.250000) {\hbox{\tate いさぎよ.い}};
\node[Meaning] at (-53.950000, -13.600000) {pure};
\node[Kanji] at (-51.900000, -14.850000) {\textcolor[HTML]{c8a59d}{契}};
\node[Square] at (-51.900000, -15.350000) {};
\node[Onyomi] at (-51.850000, -15.250000) {\hbox{\tate ケイ}};
\node[Meaning] at (-51.900000, -13.600000) {pledge};
\node[Kanji] at (-49.850000, -14.850000) {\textcolor[HTML]{91b7c3}{喫}};
\node[Square] at (-49.850000, -15.350000) {};
\node[Onyomi] at (-49.800000, -15.250000) {\hbox{\tate キツ}};
\node[Kunyomi] at (-49.900000, -15.250000) {\hbox{\tate の.む}};
\node[Meaning] at (-49.850000, -13.600000) {consume};
\node[Kanji] at (-47.800000, -14.850000) {\textcolor[HTML]{d69f8d}{害}};
\node[Square] at (-47.800000, -15.350000) {};
\node[Onyomi] at (-47.750000, -15.250000) {\hbox{\tate ガイ}};
\node[Meaning] at (-47.800000, -13.600000) {damage};
\node[Kanji] at (-45.750000, -14.850000) {\textcolor[HTML]{a3bac2}{轄}};
\node[Square] at (-45.750000, -15.350000) {};
\node[Onyomi] at (-45.700000, -15.250000) {\hbox{\tate カツ}};
\node[Kunyomi] at (-45.800000, -15.250000) {\hbox{\tate くさび}};
\node[Meaning] at (-45.750000, -13.600000) {control};
\node[Kanji] at (-43.700000, -14.850000) {\textcolor[HTML]{d2a293}{割}};
\node[Square] at (-43.700000, -15.350000) {};
\node[Onyomi] at (-43.650000, -15.250000) {\hbox{\tate カツ}};
\node[Kunyomi] at (-43.750000, -15.250000) {\hbox{\tate わり・わ}};
\node[Meaning] at (-43.700000, -13.600000) {divide};
\node[Kanji] at (-41.650000, -14.850000) {\textcolor[HTML]{c8a59d}{憲}};
\node[Square] at (-41.650000, -15.350000) {};
\node[Onyomi] at (-41.600000, -15.250000) {\hbox{\tate ケン}};
\node[Meaning] at (-41.650000, -13.600000) {constitution};
\node[Kanji] at (-39.600000, -14.850000) {\textcolor[HTML]{a11d25}{生}};
\node[Square] at (-39.600000, -15.350000) {};
\node[Onyomi] at (-39.550000, -15.250000) {\hbox{\tate セイ・ショウ}};
\node[Kunyomi] at (-39.650000, -15.250000) {\hbox{\tate い.きる}};
\node[Meaning] at (-39.600000, -13.600000) {life};
\node[Kanji] at (-37.550000, -14.850000) {\textcolor[HTML]{d69f8d}{星}};
\node[Square] at (-37.550000, -15.350000) {};
\node[Onyomi] at (-37.500000, -15.250000) {\hbox{\tate セイ}};
\node[Kunyomi] at (-37.600000, -15.250000) {\hbox{\tate ほし}};
\node[Meaning] at (-37.550000, -13.600000) {star};
\node[Kanji] at (-35.500000, -14.850000) {\textcolor[HTML]{b0b0b5}{姓}};
\node[Square] at (-35.500000, -15.350000) {};
\node[Onyomi] at (-35.450000, -15.250000) {\hbox{\tate セイ・ショウ}};
\node[Meaning] at (-35.500000, -13.600000) {surname};
\node[Kanji] at (-33.450000, -14.850000) {\textcolor[HTML]{b74029}{性}};
\node[Square] at (-33.450000, -15.350000) {};
\node[Onyomi] at (-33.400000, -15.250000) {\hbox{\tate セイ・ショウ}};
\node[Meaning] at (-33.450000, -13.600000) {gender};
\node[Kanji] at (-31.400000, -14.850000) {\textcolor[HTML]{68a4bc}{牲}};
\node[Square] at (-31.400000, -15.350000) {};
\node[Onyomi] at (-31.350000, -15.250000) {\hbox{\tate セイ}};
\node[Meaning] at (-31.400000, -13.600000) {offering};
\node[Kanji] at (-29.350000, -14.850000) {\textcolor[HTML]{cd8268}{産}};
\node[Square] at (-29.350000, -15.350000) {};
\node[Onyomi] at (-29.300000, -15.250000) {\hbox{\tate サン}};
\node[Kunyomi] at (-29.400000, -15.250000) {\hbox{\tate う.む}};
\node[Meaning] at (-29.350000, -13.600000) {give birth};
\node[Kanji] at (-27.300000, -14.850000) {\textcolor[HTML]{b0b0b5}{隆}};
\node[Square] at (-27.300000, -15.350000) {};
\node[Onyomi] at (-27.250000, -15.250000) {\hbox{\tate リュウ}};
\node[Meaning] at (-27.300000, -13.600000) {prosperity};
\node[Kanji] at (-25.250000, -14.850000) {\textcolor[HTML]{a3bac2}{峰}};
\node[Square] at (-25.250000, -15.350000) {};
\node[Onyomi] at (-25.200000, -15.250000) {\hbox{\tate ホウ}};
\node[Kunyomi] at (-25.300000, -15.250000) {\hbox{\tate みね}};
\node[Meaning] at (-25.250000, -13.600000) {summit};
\node[Kanji] at (-23.200000, -14.850000) {\textcolor[HTML]{408dba}{縫}};
\node[Square] at (-23.200000, -15.350000) {};
\node[Onyomi] at (-23.150000, -15.250000) {\hbox{\tate ホウ}};
\node[Kunyomi] at (-23.250000, -15.250000) {\hbox{\tate ぬ}};
\node[Meaning] at (-23.200000, -13.600000) {sew};
\node[Kanji] at (-21.150000, -14.850000) {\textcolor[HTML]{a3bac2}{拝}};
\node[Square] at (-21.150000, -15.350000) {};
\node[Onyomi] at (-21.100000, -15.250000) {\hbox{\tate ハイ}};
\node[Kunyomi] at (-21.200000, -15.250000) {\hbox{\tate おが.む}};
\node[Meaning] at (-21.150000, -13.600000) {worship};
\node[Kanji] at (-19.100000, -14.850000) {\textcolor[HTML]{b0b0b5}{寿}};
\node[Square] at (-19.100000, -15.350000) {};
\node[Onyomi] at (-19.050000, -15.250000) {\hbox{\tate ジュ・ス}};
\node[Kunyomi] at (-19.150000, -15.250000) {\hbox{\tate ことぶき}};
\node[Meaning] at (-19.100000, -13.600000) {lifespan};
\node[Kanji] at (-17.050000, -14.850000) {\textcolor[HTML]{68a4bc}{鋳}};
\node[Square] at (-17.050000, -15.350000) {};
\node[Onyomi] at (-17.000000, -15.250000) {\hbox{\tate チュウ}};
\node[Kunyomi] at (-17.100000, -15.250000) {\hbox{\tate い}};
\node[Meaning] at (-17.050000, -13.600000) {cast};
\node[Kanji] at (-15.000000, -14.850000) {\textcolor[HTML]{d2a293}{籍}};
\node[Square] at (-15.000000, -15.350000) {};
\node[Onyomi] at (-14.950000, -15.250000) {\hbox{\tate セキ}};
\node[Meaning] at (-15.000000, -13.600000) {enroll};
\node[Kanji] at (-12.950000, -14.850000) {\textcolor[HTML]{d2a293}{春}};
\node[Square] at (-12.950000, -15.350000) {};
\node[Onyomi] at (-12.900000, -15.250000) {\hbox{\tate シュン}};
\node[Kunyomi] at (-13.000000, -15.250000) {\hbox{\tate はる}};
\node[Meaning] at (-12.950000, -13.600000) {spring};
\node[Kanji] at (-10.900000, -14.850000) {\textcolor[HTML]{a3bac2}{泰}};
\node[Square] at (-10.900000, -15.350000) {};
\node[Onyomi] at (-10.850000, -15.250000) {\hbox{\tate タイ}};
\node[Meaning] at (-10.900000, -13.600000) {peace};
\node[Kanji] at (-8.850000, -14.850000) {\textcolor[HTML]{c8a59d}{奏}};
\node[Square] at (-8.850000, -15.350000) {};
\node[Onyomi] at (-8.800000, -15.250000) {\hbox{\tate ソウ}};
\node[Kunyomi] at (-8.900000, -15.250000) {\hbox{\tate かな.でる}};
\node[Meaning] at (-8.850000, -13.600000) {play music};
\node[Kanji] at (-6.800000, -14.850000) {\textcolor[HTML]{c36143}{実}};
\node[Square] at (-6.800000, -15.350000) {};
\node[Onyomi] at (-6.750000, -15.250000) {\hbox{\tate ジツ}};
\node[Kunyomi] at (-6.850000, -15.250000) {\hbox{\tate み}};
\node[Meaning] at (-6.800000, -13.600000) {truth};
\node[Kanji] at (-4.750000, -14.850000) {\textcolor[HTML]{b0b0b5}{奉}};
\node[Square] at (-4.750000, -15.350000) {};
\node[Onyomi] at (-4.700000, -15.250000) {\hbox{\tate ホウ・ブ}};
\node[Kunyomi] at (-4.800000, -15.250000) {\hbox{\tate たてまつ}};
\node[Meaning] at (-4.750000, -13.600000) {dedicate};
\node[Kanji] at (-2.700000, -14.850000) {\textcolor[HTML]{1e76bb}{俸}};
\node[Square] at (-2.700000, -15.350000) {};
\node[Onyomi] at (-2.650000, -15.250000) {\hbox{\tate ホウ}};
\node[Meaning] at (-2.700000, -13.600000) {salary};
\node[Kanji] at (-0.650000, -14.850000) {\textcolor[HTML]{a3bac2}{棒}};
\node[Square] at (-0.650000, -15.350000) {};
\node[Onyomi] at (-0.600000, -15.250000) {\hbox{\tate ボウ}};
\node[Kunyomi] at (-0.700000, -15.250000) {\hbox{\tate ぼう}};
\node[Meaning] at (-0.650000, -13.600000) {pole};
\node[Kanji] at (1.400000, -14.850000) {\textcolor[HTML]{1e76bb}{謹}};
\node[Square] at (1.400000, -15.350000) {};
\node[Onyomi] at (1.450000, -15.250000) {\hbox{\tate キン}};
\node[Kunyomi] at (1.350000, -15.250000) {\hbox{\tate つつし}};
\node[Meaning] at (1.400000, -13.600000) {humble};
\node[Kanji] at (3.450000, -14.850000) {\textcolor[HTML]{c8a59d}{勤}};
\node[Square] at (3.450000, -15.350000) {};
\node[Onyomi] at (3.500000, -15.250000) {\hbox{\tate キン}};
\node[Kunyomi] at (3.400000, -15.250000) {\hbox{\tate つと.*}};
\node[Meaning] at (3.450000, -13.600000) {work};
\node[Kanji] at (5.500000, -14.850000) {\textcolor[HTML]{c8a59d}{漢}};
\node[Square] at (5.500000, -15.350000) {};
\node[Onyomi] at (5.550000, -15.250000) {\hbox{\tate カン}};
\node[Meaning] at (5.500000, -13.600000) {chinese};
\node[Kanji] at (7.550000, -14.850000) {\textcolor[HTML]{68a4bc}{嘆}};
\node[Square] at (7.550000, -15.350000) {};
\node[Onyomi] at (7.600000, -15.250000) {\hbox{\tate タン}};
\node[Kunyomi] at (7.500000, -15.250000) {\hbox{\tate なげ.く}};
\node[Meaning] at (7.550000, -13.600000) {sigh};
\node[Kanji] at (9.600000, -14.850000) {\textcolor[HTML]{d2a293}{難}};
\node[Square] at (9.600000, -15.350000) {};
\node[Onyomi] at (9.650000, -15.250000) {\hbox{\tate ナン}};
\node[Kunyomi] at (9.550000, -15.250000) {\hbox{\tate むずか.しい}};
\node[Meaning] at (9.600000, -13.600000) {difficult};
\node[Kanji] at (11.650000, -14.850000) {\textcolor[HTML]{c8a59d}{華}};
\node[Square] at (11.650000, -15.350000) {};
\node[Onyomi] at (11.700000, -15.250000) {\hbox{\tate カ}};
\node[Kunyomi] at (11.600000, -15.250000) {\hbox{\tate はな}};
\node[Meaning] at (11.650000, -13.600000) {showy};
\node[Kanji] at (13.700000, -14.850000) {\textcolor[HTML]{a3bac2}{垂}};
\node[Square] at (13.700000, -15.350000) {};
\node[Onyomi] at (13.750000, -15.250000) {\hbox{\tate スイ}};
\node[Kunyomi] at (13.650000, -15.250000) {\hbox{\tate た.*}};
\node[Meaning] at (13.700000, -13.600000) {dangle};
\node[Kanji] at (15.750000, -14.850000) {\textcolor[HTML]{408dba}{睡}};
\node[Square] at (15.750000, -15.350000) {};
\node[Onyomi] at (15.800000, -15.250000) {\hbox{\tate スイ}};
\node[Meaning] at (15.750000, -13.600000) {drowsy};
\node[Kanji] at (17.800000, -14.850000) {\textcolor[HTML]{d69f8d}{乗}};
\node[Square] at (17.800000, -15.350000) {};
\node[Onyomi] at (17.850000, -15.250000) {\hbox{\tate ジョウ}};
\node[Kunyomi] at (17.750000, -15.250000) {\hbox{\tate の}};
\node[Meaning] at (17.800000, -13.600000) {ride};
\node[Kanji] at (19.850000, -14.850000) {\textcolor[HTML]{91b7c3}{剰}};
\node[Square] at (19.850000, -15.350000) {};
\node[Onyomi] at (19.900000, -15.250000) {\hbox{\tate ジョウ}};
\node[Kunyomi] at (19.800000, -15.250000) {\hbox{\tate あまつさえ}};
\node[Meaning] at (19.850000, -13.600000) {surplus};
\node[Kanji] at (21.900000, -14.850000) {\textcolor[HTML]{d69f8d}{今}};
\node[Square] at (21.900000, -15.350000) {};
\node[Onyomi] at (21.950000, -15.250000) {\hbox{\tate コン}};
\node[Kunyomi] at (21.850000, -15.250000) {\hbox{\tate いま}};
\node[Meaning] at (21.900000, -13.600000) {now};
\node[Kanji] at (23.950000, -14.850000) {\textcolor[HTML]{d69f8d}{含}};
\node[Square] at (23.950000, -15.350000) {};
\node[Onyomi] at (24.000000, -15.250000) {\hbox{\tate ガン}};
\node[Kunyomi] at (23.900000, -15.250000) {\hbox{\tate ふく.む}};
\node[Meaning] at (23.950000, -13.600000) {include};
\node[Kanji] at (26.000000, -14.850000) {\textcolor[HTML]{1e76bb}{吟}};
\node[Square] at (26.000000, -15.350000) {};
\node[Onyomi] at (26.050000, -15.250000) {\hbox{\tate ギン}};
\node[Meaning] at (26.000000, -13.600000) {recital};
\node[Kanji] at (28.050000, -14.850000) {\textcolor[HTML]{d69f8d}{念}};
\node[Square] at (28.050000, -15.350000) {};
\node[Onyomi] at (28.100000, -15.250000) {\hbox{\tate ネン}};
\node[Meaning] at (28.050000, -13.600000) {thought};
\node[Kanji] at (30.100000, -14.850000) {\textcolor[HTML]{91b7c3}{琴}};
\node[Square] at (30.100000, -15.350000) {};
\node[Kunyomi] at (30.050000, -15.250000) {\hbox{\tate こと}};
\node[Meaning] at (30.100000, -13.600000) {harp};
\node[Kanji] at (32.150000, -14.850000) {\textcolor[HTML]{b0b0b5}{陰}};
\node[Square] at (32.150000, -15.350000) {};
\node[Onyomi] at (32.200000, -15.250000) {\hbox{\tate イン}};
\node[Kunyomi] at (32.100000, -15.250000) {\hbox{\tate かげ}};
\node[Meaning] at (32.150000, -13.600000) {shade};
\node[Kanji] at (34.200000, -14.850000) {\textcolor[HTML]{d69f8d}{予}};
\node[Square] at (34.200000, -15.350000) {};
\node[Onyomi] at (34.250000, -15.250000) {\hbox{\tate ヨ}};
\node[Kunyomi] at (34.150000, -15.250000) {\hbox{\tate あらかじ}};
\node[Meaning] at (34.200000, -13.600000) {beforehand};
\node[Kanji] at (36.250000, -14.850000) {\textcolor[HTML]{b0b0b5}{序}};
\node[Square] at (36.250000, -15.350000) {};
\node[Onyomi] at (36.300000, -15.250000) {\hbox{\tate ジョ}};
\node[Kunyomi] at (36.200000, -15.250000) {\hbox{\tate つい・ついで}};
\node[Meaning] at (36.250000, -13.600000) {preface};
\node[Kanji] at (38.300000, -14.850000) {\textcolor[HTML]{91b7c3}{預}};
\node[Square] at (38.300000, -15.350000) {};
\node[Onyomi] at (38.350000, -15.250000) {\hbox{\tate ヨ}};
\node[Kunyomi] at (38.250000, -15.250000) {\hbox{\tate あず.ける}};
\node[Meaning] at (38.300000, -13.600000) {deposit};
\node[Kanji] at (40.350000, -14.850000) {\textcolor[HTML]{c36143}{野}};
\node[Square] at (40.350000, -15.350000) {};
\node[Onyomi] at (40.400000, -15.250000) {\hbox{\tate ヤ}};
\node[Kunyomi] at (40.300000, -15.250000) {\hbox{\tate の}};
\node[Meaning] at (40.350000, -13.600000) {field};
\node[Kanji] at (42.400000, -14.850000) {\textcolor[HTML]{c8a59d}{兼}};
\node[Square] at (42.400000, -15.350000) {};
\node[Onyomi] at (42.450000, -15.250000) {\hbox{\tate ケン}};
\node[Kunyomi] at (42.350000, -15.250000) {\hbox{\tate か.ねる}};
\node[Meaning] at (42.400000, -13.600000) {concurrently};
\node[Kanji] at (44.450000, -14.850000) {\textcolor[HTML]{a3bac2}{嫌}};
\node[Square] at (44.450000, -15.350000) {};
\node[Onyomi] at (44.500000, -15.250000) {\hbox{\tate ケン}};
\node[Kunyomi] at (44.400000, -15.250000) {\hbox{\tate いや}};
\node[Meaning] at (44.450000, -13.600000) {dislike};
\node[Kanji] at (46.500000, -14.850000) {\textcolor[HTML]{b0b0b5}{鎌}};
\node[Square] at (46.500000, -15.350000) {};
\node[Onyomi] at (46.550000, -15.250000) {\hbox{\tate ケン・レン}};
\node[Kunyomi] at (46.450000, -15.250000) {\hbox{\tate かま}};
\node[Meaning] at (46.500000, -13.600000) {sickle};
\node[Kanji] at (48.550000, -14.850000) {\textcolor[HTML]{91b7c3}{謙}};
\node[Square] at (48.550000, -15.350000) {};
\node[Onyomi] at (48.600000, -15.250000) {\hbox{\tate ケン}};
\node[Meaning] at (48.550000, -13.600000) {modesty};
\node[Kanji] at (50.600000, -14.850000) {\textcolor[HTML]{68a4bc}{廉}};
\node[Square] at (50.600000, -15.350000) {};
\node[Onyomi] at (50.650000, -15.250000) {\hbox{\tate レン}};
\node[Meaning] at (50.600000, -13.600000) {bargain};
\node[Kanji] at (52.650000, -14.850000) {\textcolor[HTML]{c36143}{西}};
\node[Square] at (52.650000, -15.350000) {};
\node[Onyomi] at (52.700000, -15.250000) {\hbox{\tate セイ}};
\node[Kunyomi] at (52.600000, -15.250000) {\hbox{\tate にし}};
\node[Meaning] at (52.650000, -13.600000) {west};
\node[Kanji] at (54.700000, -14.850000) {\textcolor[HTML]{d69f8d}{価}};
\node[Square] at (54.700000, -15.350000) {};
\node[Onyomi] at (54.750000, -15.250000) {\hbox{\tate カ}};
\node[Kunyomi] at (54.650000, -15.250000) {\hbox{\tate あたい}};
\node[Meaning] at (54.700000, -13.600000) {value};
\node[Kanji] at (56.750000, -14.850000) {\textcolor[HTML]{c36143}{要}};
\node[Square] at (56.750000, -15.350000) {};
\node[Onyomi] at (56.800000, -15.250000) {\hbox{\tate ヨウ}};
\node[Kunyomi] at (56.700000, -15.250000) {\hbox{\tate い・かなめ}};
\node[Meaning] at (56.750000, -13.600000) {need};
\node[Meaning] at (-58.050000, -14.750000) {81.96\%};
\node[Kanji] at (-56.000000, -16.900000) {\textcolor[HTML]{91b7c3}{腰}};
\node[Square] at (-56.000000, -17.400000) {};
\node[Onyomi] at (-55.950000, -17.300000) {\hbox{\tate ヨウ}};
\node[Kunyomi] at (-56.050000, -17.300000) {\hbox{\tate こし}};
\node[Meaning] at (-56.000000, -15.650000) {waist};
\node[Kanji] at (-53.950000, -16.900000) {\textcolor[HTML]{c8a59d}{票}};
\node[Square] at (-53.950000, -17.400000) {};
\node[Onyomi] at (-53.900000, -17.300000) {\hbox{\tate ヒョウ}};
\node[Meaning] at (-53.950000, -15.650000) {ballot};
\node[Kanji] at (-51.900000, -16.900000) {\textcolor[HTML]{68a4bc}{漂}};
\node[Square] at (-51.900000, -17.400000) {};
\node[Onyomi] at (-51.850000, -17.300000) {\hbox{\tate ヒョウ}};
\node[Kunyomi] at (-51.950000, -17.300000) {\hbox{\tate ただよ.う}};
\node[Meaning] at (-51.900000, -15.650000) {drift};
\node[Kanji] at (-49.850000, -16.900000) {\textcolor[HTML]{d2a293}{標}};
\node[Square] at (-49.850000, -17.400000) {};
\node[Onyomi] at (-49.800000, -17.300000) {\hbox{\tate ヒョウ}};
\node[Kunyomi] at (-49.900000, -17.300000) {\hbox{\tate しるし}};
\node[Meaning] at (-49.850000, -15.650000) {signpost};
\node[Kanji] at (-47.800000, -16.900000) {\textcolor[HTML]{a3bac2}{遷}};
\node[Square] at (-47.800000, -17.400000) {};
\node[Onyomi] at (-47.750000, -17.300000) {\hbox{\tate セン}};
\node[Kunyomi] at (-47.850000, -17.300000) {\hbox{\tate うつ}};
\node[Meaning] at (-47.800000, -15.650000) {transition};
\node[Kanji] at (-45.750000, -16.900000) {\textcolor[HTML]{a3bac2}{覆}};
\node[Square] at (-45.750000, -17.400000) {};
\node[Onyomi] at (-45.700000, -17.300000) {\hbox{\tate フク}};
\node[Kunyomi] at (-45.800000, -17.300000) {\hbox{\tate おお.う}};
\node[Meaning] at (-45.750000, -15.650000) {capsize};
\node[Kanji] at (-43.700000, -16.900000) {\textcolor[HTML]{a3bac2}{煙}};
\node[Square] at (-43.700000, -17.400000) {};
\node[Onyomi] at (-43.650000, -17.300000) {\hbox{\tate エン}};
\node[Kunyomi] at (-43.750000, -17.300000) {\hbox{\tate けむ.り}};
\node[Meaning] at (-43.700000, -15.650000) {smoke};
\node[Kanji] at (-41.650000, -16.900000) {\textcolor[HTML]{c36143}{南}};
\node[Square] at (-41.650000, -17.400000) {};
\node[Onyomi] at (-41.600000, -17.300000) {\hbox{\tate ナン}};
\node[Kunyomi] at (-41.700000, -17.300000) {\hbox{\tate みなみ}};
\node[Meaning] at (-41.650000, -15.650000) {south};
\node[Kanji] at (-39.600000, -16.900000) {\textcolor[HTML]{c8a59d}{献}};
\node[Square] at (-39.600000, -17.400000) {};
\node[Onyomi] at (-39.550000, -17.300000) {\hbox{\tate ケン・コン}};
\node[Kunyomi] at (-39.650000, -17.300000) {\hbox{\tate たてまつ.る}};
\node[Meaning] at (-39.600000, -15.650000) {offer};
\node[Kanji] at (-37.550000, -16.900000) {\textcolor[HTML]{cd8268}{門}};
\node[Square] at (-37.550000, -17.400000) {};
\node[Onyomi] at (-37.500000, -17.300000) {\hbox{\tate モン}};
\node[Meaning] at (-37.550000, -15.650000) {gates};
\node[Kanji] at (-35.500000, -16.900000) {\textcolor[HTML]{cd8268}{問}};
\node[Square] at (-35.500000, -17.400000) {};
\node[Onyomi] at (-35.450000, -17.300000) {\hbox{\tate モン}};
\node[Kunyomi] at (-35.550000, -17.300000) {\hbox{\tate と・とん}};
\node[Meaning] at (-35.500000, -15.650000) {problem};
\node[Kanji] at (-33.450000, -16.900000) {\textcolor[HTML]{c8a59d}{閲}};
\node[Square] at (-33.450000, -17.400000) {};
\node[Onyomi] at (-33.400000, -17.300000) {\hbox{\tate エツ}};
\node[Kunyomi] at (-33.500000, -17.300000) {\hbox{\tate けみ}};
\node[Meaning] at (-33.450000, -15.650000) {inspection};
\node[Kanji] at (-31.400000, -16.900000) {\textcolor[HTML]{91b7c3}{閥}};
\node[Square] at (-31.400000, -17.400000) {};
\node[Onyomi] at (-31.350000, -17.300000) {\hbox{\tate バツ}};
\node[Meaning] at (-31.400000, -15.650000) {clique};
\node[Kanji] at (-29.350000, -16.900000) {\textcolor[HTML]{b74029}{間}};
\node[Square] at (-29.350000, -17.400000) {};
\node[Onyomi] at (-29.300000, -17.300000) {\hbox{\tate カン・ケン}};
\node[Kunyomi] at (-29.400000, -17.300000) {\hbox{\tate あいだ・ま}};
\node[Meaning] at (-29.350000, -15.650000) {interval};
\node[Kanji] at (-27.300000, -16.900000) {\textcolor[HTML]{b0b0b5}{簡}};
\node[Square] at (-27.300000, -17.400000) {};
\node[Onyomi] at (-27.250000, -17.300000) {\hbox{\tate カン}};
\node[Meaning] at (-27.300000, -15.650000) {simplicity};
\node[Kanji] at (-25.250000, -16.900000) {\textcolor[HTML]{b74029}{開}};
\node[Square] at (-25.250000, -17.400000) {};
\node[Onyomi] at (-25.200000, -17.300000) {\hbox{\tate カイ}};
\node[Kunyomi] at (-25.300000, -17.300000) {\hbox{\tate あ.ける}};
\node[Meaning] at (-25.250000, -15.650000) {open};
\node[Kanji] at (-23.200000, -16.900000) {\textcolor[HTML]{c8a59d}{閉}};
\node[Square] at (-23.200000, -17.400000) {};
\node[Onyomi] at (-23.150000, -17.300000) {\hbox{\tate ヘイ}};
\node[Kunyomi] at (-23.250000, -17.300000) {\hbox{\tate し・と}};
\node[Meaning] at (-23.200000, -15.650000) {closed};
\node[Kanji] at (-21.150000, -16.900000) {\textcolor[HTML]{c8a59d}{閣}};
\node[Square] at (-21.150000, -17.400000) {};
\node[Onyomi] at (-21.100000, -17.300000) {\hbox{\tate カク}};
\node[Meaning] at (-21.150000, -15.650000) {the cabinet};
\node[Kanji] at (-19.100000, -16.900000) {\textcolor[HTML]{408dba}{閑}};
\node[Square] at (-19.100000, -17.400000) {};
\node[Onyomi] at (-19.050000, -17.300000) {\hbox{\tate カン}};
\node[Meaning] at (-19.100000, -15.650000) {leisure};
\node[Kanji] at (-17.050000, -16.900000) {\textcolor[HTML]{cd8268}{聞}};
\node[Square] at (-17.050000, -17.400000) {};
\node[Onyomi] at (-17.000000, -17.300000) {\hbox{\tate ブン・モン}};
\node[Kunyomi] at (-17.100000, -17.300000) {\hbox{\tate き.く}};
\node[Meaning] at (-17.050000, -15.650000) {hear};
\node[Kanji] at (-15.000000, -16.900000) {\textcolor[HTML]{68a4bc}{潤}};
\node[Square] at (-15.000000, -17.400000) {};
\node[Onyomi] at (-14.950000, -17.300000) {\hbox{\tate ジュン}};
\node[Kunyomi] at (-15.050000, -17.300000) {\hbox{\tate うるお.*}};
\node[Meaning] at (-15.000000, -15.650000) {watered};
\node[Kanji] at (-12.950000, -16.900000) {\textcolor[HTML]{91b7c3}{欄}};
\node[Square] at (-12.950000, -17.400000) {};
\node[Onyomi] at (-12.900000, -17.300000) {\hbox{\tate ラン}};
\node[Kunyomi] at (-13.000000, -17.300000) {\hbox{\tate てすり}};
\node[Meaning] at (-12.950000, -15.650000) {column};
\node[Kanji] at (-10.900000, -16.900000) {\textcolor[HTML]{d2a293}{闘}};
\node[Square] at (-10.900000, -17.400000) {};
\node[Onyomi] at (-10.850000, -17.300000) {\hbox{\tate トウ}};
\node[Kunyomi] at (-10.950000, -17.300000) {\hbox{\tate たたか.う}};
\node[Meaning] at (-10.900000, -15.650000) {struggle};
\node[Kanji] at (-8.850000, -16.900000) {\textcolor[HTML]{d2a293}{倉}};
\node[Square] at (-8.850000, -17.400000) {};
\node[Onyomi] at (-8.800000, -17.300000) {\hbox{\tate ソウ}};
\node[Kunyomi] at (-8.900000, -17.300000) {\hbox{\tate くら}};
\node[Meaning] at (-8.850000, -15.650000) {warehouse};
\node[Kanji] at (-6.800000, -16.900000) {\textcolor[HTML]{d2a293}{創}};
\node[Square] at (-6.800000, -17.400000) {};
\node[Onyomi] at (-6.750000, -17.300000) {\hbox{\tate ソウ}};
\node[Meaning] at (-6.800000, -15.650000) {create};
\node[Kanji] at (-4.750000, -16.900000) {\textcolor[HTML]{d69f8d}{非}};
\node[Square] at (-4.750000, -17.400000) {};
\node[Onyomi] at (-4.700000, -17.300000) {\hbox{\tate ヒ}};
\node[Meaning] at (-4.750000, -15.650000) {injustice};
\node[Kanji] at (-2.700000, -16.900000) {\textcolor[HTML]{b0b0b5}{俳}};
\node[Square] at (-2.700000, -17.400000) {};
\node[Onyomi] at (-2.650000, -17.300000) {\hbox{\tate ハイ}};
\node[Meaning] at (-2.700000, -15.650000) {haiku};
\node[Kanji] at (-0.650000, -16.900000) {\textcolor[HTML]{b0b0b5}{排}};
\node[Square] at (-0.650000, -17.400000) {};
\node[Onyomi] at (-0.600000, -17.300000) {\hbox{\tate ハイ}};
\node[Meaning] at (-0.650000, -15.650000) {reject};
\node[Kanji] at (1.400000, -16.900000) {\textcolor[HTML]{a3bac2}{悲}};
\node[Square] at (1.400000, -17.400000) {};
\node[Onyomi] at (1.450000, -17.300000) {\hbox{\tate ヒ}};
\node[Kunyomi] at (1.350000, -17.300000) {\hbox{\tate かな}};
\node[Meaning] at (1.400000, -15.650000) {sad};
\node[Kanji] at (3.450000, -16.900000) {\textcolor[HTML]{c8a59d}{罪}};
\node[Square] at (3.450000, -17.400000) {};
\node[Onyomi] at (3.500000, -17.300000) {\hbox{\tate ザイ}};
\node[Kunyomi] at (3.400000, -17.300000) {\hbox{\tate つみ}};
\node[Meaning] at (3.450000, -15.650000) {guilt};
\node[Kanji] at (5.500000, -16.900000) {\textcolor[HTML]{a3bac2}{輩}};
\node[Square] at (5.500000, -17.400000) {};
\node[Onyomi] at (5.550000, -17.300000) {\hbox{\tate ハイ}};
\node[Meaning] at (5.500000, -15.650000) {comrade};
\node[Kanji] at (7.550000, -16.900000) {\textcolor[HTML]{a3bac2}{扉}};
\node[Square] at (7.550000, -17.400000) {};
\node[Onyomi] at (7.600000, -17.300000) {\hbox{\tate ヒ}};
\node[Kunyomi] at (7.500000, -17.300000) {\hbox{\tate とびら}};
\node[Meaning] at (7.550000, -15.650000) {front door};
\node[Kanji] at (9.600000, -16.900000) {\textcolor[HTML]{a3bac2}{侯}};
\node[Square] at (9.600000, -17.400000) {};
\node[Onyomi] at (9.650000, -17.300000) {\hbox{\tate コウ}};
\node[Meaning] at (9.600000, -15.650000) {marquis};
\node[Kanji] at (11.650000, -16.900000) {\textcolor[HTML]{c8a59d}{候}};
\node[Square] at (11.650000, -17.400000) {};
\node[Onyomi] at (11.700000, -17.300000) {\hbox{\tate コウ}};
\node[Meaning] at (11.650000, -15.650000) {climate};
\node[Kanji] at (13.700000, -16.900000) {\textcolor[HTML]{c36143}{決}};
\node[Square] at (13.700000, -17.400000) {};
\node[Onyomi] at (13.750000, -17.300000) {\hbox{\tate ケツ}};
\node[Kunyomi] at (13.650000, -17.300000) {\hbox{\tate き.める}};
\node[Meaning] at (13.700000, -15.650000) {decide};
\node[Kanji] at (15.750000, -16.900000) {\textcolor[HTML]{b0b0b5}{快}};
\node[Square] at (15.750000, -17.400000) {};
\node[Onyomi] at (15.800000, -17.300000) {\hbox{\tate カイ}};
\node[Kunyomi] at (15.700000, -17.300000) {\hbox{\tate こころよ.い}};
\node[Meaning] at (15.750000, -15.650000) {pleasant};
\node[Kanji] at (17.800000, -16.900000) {\textcolor[HTML]{68a4bc}{偉}};
\node[Square] at (17.800000, -17.400000) {};
\node[Onyomi] at (17.850000, -17.300000) {\hbox{\tate イ}};
\node[Kunyomi] at (17.750000, -17.300000) {\hbox{\tate えら}};
\node[Meaning] at (17.800000, -15.650000) {greatness};
\node[Kanji] at (19.850000, -16.900000) {\textcolor[HTML]{d2a293}{違}};
\node[Square] at (19.850000, -17.400000) {};
\node[Onyomi] at (19.900000, -17.300000) {\hbox{\tate イ}};
\node[Kunyomi] at (19.800000, -17.300000) {\hbox{\tate ちが}};
\node[Meaning] at (19.850000, -15.650000) {different};
\node[Kanji] at (21.900000, -16.900000) {\textcolor[HTML]{b0b0b5}{緯}};
\node[Square] at (21.900000, -17.400000) {};
\node[Onyomi] at (21.950000, -17.300000) {\hbox{\tate イ}};
\node[Kunyomi] at (21.850000, -17.300000) {\hbox{\tate ぬき}};
\node[Meaning] at (21.900000, -15.650000) {latitude};
\node[Kanji] at (23.950000, -16.900000) {\textcolor[HTML]{d69f8d}{衛}};
\node[Square] at (23.950000, -17.400000) {};
\node[Onyomi] at (24.000000, -17.300000) {\hbox{\tate エイ}};
\node[Meaning] at (23.950000, -15.650000) {defense};
\node[Kanji] at (26.000000, -16.900000) {\textcolor[HTML]{c8a59d}{韓}};
\node[Square] at (26.000000, -17.400000) {};
\node[Onyomi] at (26.050000, -17.300000) {\hbox{\tate カン}};
\node[Meaning] at (26.000000, -15.650000) {korea};
\node[Kanji] at (28.050000, -16.900000) {\textcolor[HTML]{b0b0b5}{干}};
\node[Square] at (28.050000, -17.400000) {};
\node[Onyomi] at (28.100000, -17.300000) {\hbox{\tate カン}};
\node[Kunyomi] at (28.000000, -17.300000) {\hbox{\tate ほ.す・ひ}};
\node[Meaning] at (28.050000, -15.650000) {dry};
\node[Kanji] at (30.100000, -16.900000) {\textcolor[HTML]{91b7c3}{肝}};
\node[Square] at (30.100000, -17.400000) {};
\node[Onyomi] at (30.150000, -17.300000) {\hbox{\tate カン}};
\node[Kunyomi] at (30.050000, -17.300000) {\hbox{\tate きも}};
\node[Meaning] at (30.100000, -15.650000) {liver};
\node[Kanji] at (32.150000, -16.900000) {\textcolor[HTML]{d69f8d}{刊}};
\node[Square] at (32.150000, -17.400000) {};
\node[Onyomi] at (32.200000, -17.300000) {\hbox{\tate カン}};
\node[Meaning] at (32.150000, -15.650000) {edition};
\node[Kanji] at (34.200000, -16.900000) {\textcolor[HTML]{408dba}{汗}};
\node[Square] at (34.200000, -17.400000) {};
\node[Onyomi] at (34.250000, -17.300000) {\hbox{\tate カン}};
\node[Kunyomi] at (34.150000, -17.300000) {\hbox{\tate あせ}};
\node[Meaning] at (34.200000, -15.650000) {sweat};
\node[Kanji] at (36.250000, -16.900000) {\textcolor[HTML]{91b7c3}{軒}};
\node[Square] at (36.250000, -17.400000) {};
\node[Onyomi] at (36.300000, -17.300000) {\hbox{\tate ケン}};
\node[Kunyomi] at (36.200000, -17.300000) {\hbox{\tate のき}};
\node[Meaning] at (36.250000, -15.650000) {house counter};
\node[Kanji] at (38.300000, -16.900000) {\textcolor[HTML]{d2a293}{岸}};
\node[Square] at (38.300000, -17.400000) {};
\node[Onyomi] at (38.350000, -17.300000) {\hbox{\tate ガン}};
\node[Kunyomi] at (38.250000, -17.300000) {\hbox{\tate きし}};
\node[Meaning] at (38.300000, -15.650000) {coast};
\node[Kanji] at (40.350000, -16.900000) {\textcolor[HTML]{c8a59d}{幹}};
\node[Square] at (40.350000, -17.400000) {};
\node[Onyomi] at (40.400000, -17.300000) {\hbox{\tate カン}};
\node[Kunyomi] at (40.300000, -17.300000) {\hbox{\tate みき}};
\node[Meaning] at (40.350000, -15.650000) {tree trunk};
\node[Kanji] at (42.400000, -16.900000) {\textcolor[HTML]{1e76bb}{芋}};
\node[Square] at (42.400000, -17.400000) {};
\node[Kunyomi] at (42.350000, -17.300000) {\hbox{\tate いも}};
\node[Meaning] at (42.400000, -15.650000) {potato};
\node[Kanji] at (44.450000, -16.900000) {\textcolor[HTML]{d2a293}{宇}};
\node[Square] at (44.450000, -17.400000) {};
\node[Onyomi] at (44.500000, -17.300000) {\hbox{\tate ウ}};
\node[Meaning] at (44.450000, -15.650000) {outer space};
\node[Kanji] at (46.500000, -16.900000) {\textcolor[HTML]{c8a59d}{余}};
\node[Square] at (46.500000, -17.400000) {};
\node[Onyomi] at (46.550000, -17.300000) {\hbox{\tate ヨ}};
\node[Kunyomi] at (46.450000, -17.300000) {\hbox{\tate あま.る}};
\node[Meaning] at (46.500000, -15.650000) {surplus};
\node[Kanji] at (48.550000, -16.900000) {\textcolor[HTML]{b74029}{除}};
\node[Square] at (48.550000, -17.400000) {};
\node[Onyomi] at (48.600000, -17.300000) {\hbox{\tate ジョ}};
\node[Kunyomi] at (48.500000, -17.300000) {\hbox{\tate のぞ.く}};
\node[Meaning] at (48.550000, -15.650000) {exclude};
\node[Kanji] at (50.600000, -16.900000) {\textcolor[HTML]{a3bac2}{徐}};
\node[Square] at (50.600000, -17.400000) {};
\node[Onyomi] at (50.650000, -17.300000) {\hbox{\tate ジョ}};
\node[Kunyomi] at (50.550000, -17.300000) {\hbox{\tate おもむ}};
\node[Meaning] at (50.600000, -15.650000) {gently};
\node[Kanji] at (52.650000, -16.900000) {\textcolor[HTML]{91b7c3}{叙}};
\node[Square] at (52.650000, -17.400000) {};
\node[Onyomi] at (52.700000, -17.300000) {\hbox{\tate ジョ}};
\node[Kunyomi] at (52.600000, -17.300000) {\hbox{\tate つい・ついで}};
\node[Meaning] at (52.650000, -15.650000) {describe};
\node[Kanji] at (54.700000, -16.900000) {\textcolor[HTML]{c8a59d}{途}};
\node[Square] at (54.700000, -17.400000) {};
\node[Onyomi] at (54.750000, -17.300000) {\hbox{\tate ト}};
\node[Meaning] at (54.700000, -15.650000) {route};
\node[Kanji] at (56.750000, -16.900000) {\textcolor[HTML]{a3bac2}{斜}};
\node[Square] at (56.750000, -17.400000) {};
\node[Onyomi] at (56.800000, -17.300000) {\hbox{\tate シャ}};
\node[Kunyomi] at (56.700000, -17.300000) {\hbox{\tate なな.め}};
\node[Meaning] at (56.750000, -15.650000) {diagonal};
\node[Meaning] at (-58.050000, -16.800000) {84.37\%};
\node[Kanji] at (-56.000000, -18.950000) {\textcolor[HTML]{a3bac2}{塗}};
\node[Square] at (-56.000000, -19.450000) {};
\node[Onyomi] at (-55.950000, -19.350000) {\hbox{\tate ト}};
\node[Kunyomi] at (-56.050000, -19.350000) {\hbox{\tate ぬる}};
\node[Meaning] at (-56.000000, -17.700000) {paint};
\node[Kanji] at (-53.950000, -18.950000) {\textcolor[HTML]{b0b0b5}{束}};
\node[Square] at (-53.950000, -19.450000) {};
\node[Onyomi] at (-53.900000, -19.350000) {\hbox{\tate ソク}};
\node[Kunyomi] at (-54.000000, -19.350000) {\hbox{\tate たば}};
\node[Meaning] at (-53.950000, -17.700000) {bundle};
\node[Kanji] at (-51.900000, -18.950000) {\textcolor[HTML]{a11d25}{頼}};
\node[Square] at (-51.900000, -19.450000) {};
\node[Onyomi] at (-51.850000, -19.350000) {\hbox{\tate ライ}};
\node[Kunyomi] at (-51.950000, -19.350000) {\hbox{\tate たの・たよ}};
\node[Meaning] at (-51.900000, -17.700000) {trust};
\node[Kanji] at (-49.850000, -18.950000) {\textcolor[HTML]{c8a59d}{瀬}};
\node[Square] at (-49.850000, -19.450000) {};
\node[Onyomi] at (-49.800000, -19.350000) {\hbox{\tate ライ}};
\node[Kunyomi] at (-49.900000, -19.350000) {\hbox{\tate せ}};
\node[Meaning] at (-49.850000, -17.700000) {rapids};
\node[Kanji] at (-47.800000, -18.950000) {\textcolor[HTML]{91b7c3}{勅}};
\node[Square] at (-47.800000, -19.450000) {};
\node[Onyomi] at (-47.750000, -19.350000) {\hbox{\tate チョク}};
\node[Meaning] at (-47.800000, -17.700000) {imperial order};
\node[Kanji] at (-45.750000, -18.950000) {\textcolor[HTML]{68a4bc}{疎}};
\node[Square] at (-45.750000, -19.450000) {};
\node[Onyomi] at (-45.700000, -19.350000) {\hbox{\tate ソ・ショ}};
\node[Kunyomi] at (-45.800000, -19.350000) {\hbox{\tate うと・まば}};
\node[Meaning] at (-45.750000, -17.700000) {neglect};
\node[Kanji] at (-43.700000, -18.950000) {\textcolor[HTML]{d69f8d}{速}};
\node[Square] at (-43.700000, -19.450000) {};
\node[Onyomi] at (-43.650000, -19.350000) {\hbox{\tate ソク}};
\node[Kunyomi] at (-43.750000, -19.350000) {\hbox{\tate はや.い}};
\node[Meaning] at (-43.700000, -17.700000) {fast};
\node[Kanji] at (-41.650000, -18.950000) {\textcolor[HTML]{d2a293}{整}};
\node[Square] at (-41.650000, -19.450000) {};
\node[Onyomi] at (-41.600000, -19.350000) {\hbox{\tate セイ}};
\node[Kunyomi] at (-41.700000, -19.350000) {\hbox{\tate ととの.*}};
\node[Meaning] at (-41.650000, -17.700000) {arrange};
\node[Kanji] at (-39.600000, -18.950000) {\textcolor[HTML]{b0b0b5}{剣}};
\node[Square] at (-39.600000, -19.450000) {};
\node[Onyomi] at (-39.550000, -19.350000) {\hbox{\tate ケン}};
\node[Kunyomi] at (-39.650000, -19.350000) {\hbox{\tate つるぎ}};
\node[Meaning] at (-39.600000, -17.700000) {sword};
\node[Kanji] at (-37.550000, -18.950000) {\textcolor[HTML]{c8a59d}{険}};
\node[Square] at (-37.550000, -19.450000) {};
\node[Onyomi] at (-37.500000, -19.350000) {\hbox{\tate ケン}};
\node[Kunyomi] at (-37.600000, -19.350000) {\hbox{\tate けわ.しい}};
\node[Meaning] at (-37.550000, -17.700000) {risky};
\node[Kanji] at (-35.500000, -18.950000) {\textcolor[HTML]{d69f8d}{検}};
\node[Square] at (-35.500000, -19.450000) {};
\node[Onyomi] at (-35.450000, -19.350000) {\hbox{\tate ケン}};
\node[Meaning] at (-35.500000, -17.700000) {examine};
\node[Kanji] at (-33.450000, -18.950000) {\textcolor[HTML]{29409e}{倹}};
\node[Square] at (-33.450000, -19.450000) {};
\node[Onyomi] at (-33.400000, -19.350000) {\hbox{\tate ケン}};
\node[Kunyomi] at (-33.500000, -19.350000) {\hbox{\tate つづまやか}};
\node[Meaning] at (-33.450000, -17.700000) {thrifty};
\node[Kanji] at (-31.400000, -18.950000) {\textcolor[HTML]{c36143}{重}};
\node[Square] at (-31.400000, -19.450000) {};
\node[Onyomi] at (-31.350000, -19.350000) {\hbox{\tate ジュウ}};
\node[Kunyomi] at (-31.450000, -19.350000) {\hbox{\tate おも.い}};
\node[Meaning] at (-31.400000, -17.700000) {heavy};
\node[Kanji] at (-29.350000, -18.950000) {\textcolor[HTML]{a11d25}{動}};
\node[Square] at (-29.350000, -19.450000) {};
\node[Onyomi] at (-29.300000, -19.350000) {\hbox{\tate ドウ}};
\node[Kunyomi] at (-29.400000, -19.350000) {\hbox{\tate うご.*}};
\node[Meaning] at (-29.350000, -17.700000) {move};
\node[Kanji] at (-27.300000, -18.950000) {\textcolor[HTML]{a3bac2}{勲}};
\node[Square] at (-27.300000, -19.450000) {};
\node[Onyomi] at (-27.250000, -19.350000) {\hbox{\tate クン}};
\node[Kunyomi] at (-27.350000, -19.350000) {\hbox{\tate いさお}};
\node[Meaning] at (-27.300000, -17.700000) {merit};
\node[Kanji] at (-25.250000, -18.950000) {\textcolor[HTML]{d2a293}{働}};
\node[Square] at (-25.250000, -19.450000) {};
\node[Onyomi] at (-25.200000, -19.350000) {\hbox{\tate ドウ}};
\node[Kunyomi] at (-25.300000, -19.350000) {\hbox{\tate はたら.*}};
\node[Meaning] at (-25.250000, -17.700000) {work};
\node[Kanji] at (-23.200000, -18.950000) {\textcolor[HTML]{cd8268}{種}};
\node[Square] at (-23.200000, -19.450000) {};
\node[Onyomi] at (-23.150000, -19.350000) {\hbox{\tate シュ}};
\node[Kunyomi] at (-23.250000, -19.350000) {\hbox{\tate たね・ぐさ}};
\node[Meaning] at (-23.200000, -17.700000) {kind};
\node[Kanji] at (-21.150000, -18.950000) {\textcolor[HTML]{b0b0b5}{衝}};
\node[Square] at (-21.150000, -19.450000) {};
\node[Onyomi] at (-21.100000, -19.350000) {\hbox{\tate ショウ}};
\node[Kunyomi] at (-21.200000, -19.350000) {\hbox{\tate つ.く}};
\node[Meaning] at (-21.150000, -17.700000) {collide};
\node[Kanji] at (-19.100000, -18.950000) {\textcolor[HTML]{408dba}{薫}};
\node[Square] at (-19.100000, -19.450000) {};
\node[Onyomi] at (-19.050000, -19.350000) {\hbox{\tate クン}};
\node[Kunyomi] at (-19.150000, -19.350000) {\hbox{\tate かお.る}};
\node[Meaning] at (-19.100000, -17.700000) {fragrant};
\node[Kanji] at (-17.050000, -18.950000) {\textcolor[HTML]{d2a293}{病}};
\node[Square] at (-17.050000, -19.450000) {};
\node[Onyomi] at (-17.000000, -19.350000) {\hbox{\tate ビョウ}};
\node[Kunyomi] at (-17.100000, -19.350000) {\hbox{\tate や・やまい}};
\node[Meaning] at (-17.050000, -17.700000) {sick};
\node[Kanji] at (-15.000000, -18.950000) {\textcolor[HTML]{1e76bb}{痴}};
\node[Square] at (-15.000000, -19.450000) {};
\node[Onyomi] at (-14.950000, -19.350000) {\hbox{\tate チ}};
\node[Kunyomi] at (-15.050000, -19.350000) {\hbox{\tate おろか・し}};
\node[Meaning] at (-15.000000, -17.700000) {stupid};
\node[Kanji] at (-12.950000, -18.950000) {\textcolor[HTML]{1059be}{痘}};
\node[Square] at (-12.950000, -19.450000) {};
\node[Onyomi] at (-12.900000, -19.350000) {\hbox{\tate トウ}};
\node[Meaning] at (-12.950000, -17.700000) {pox};
\node[Kanji] at (-10.900000, -18.950000) {\textcolor[HTML]{c8a59d}{症}};
\node[Square] at (-10.900000, -19.450000) {};
\node[Onyomi] at (-10.850000, -19.350000) {\hbox{\tate ショウ}};
\node[Meaning] at (-10.900000, -17.700000) {symptom};
\node[Kanji] at (-8.850000, -18.950000) {\textcolor[HTML]{91b7c3}{疾}};
\node[Square] at (-8.850000, -19.450000) {};
\node[Onyomi] at (-8.800000, -19.350000) {\hbox{\tate シツ}};
\node[Kunyomi] at (-8.900000, -19.350000) {\hbox{\tate はや}};
\node[Meaning] at (-8.850000, -17.700000) {rapid};
\node[Kanji] at (-6.800000, -18.950000) {\textcolor[HTML]{1059be}{痢}};
\node[Square] at (-6.800000, -19.450000) {};
\node[Onyomi] at (-6.750000, -19.350000) {\hbox{\tate リ}};
\node[Meaning] at (-6.800000, -17.700000) {diarrhea};
\node[Kanji] at (-4.750000, -18.950000) {\textcolor[HTML]{68a4bc}{疲}};
\node[Square] at (-4.750000, -19.450000) {};
\node[Onyomi] at (-4.700000, -19.350000) {\hbox{\tate ヒ}};
\node[Kunyomi] at (-4.800000, -19.350000) {\hbox{\tate つか.れる}};
\node[Meaning] at (-4.750000, -17.700000) {exhausted};
\node[Kanji] at (-2.700000, -18.950000) {\textcolor[HTML]{68a4bc}{疫}};
\node[Square] at (-2.700000, -19.450000) {};
\node[Onyomi] at (-2.650000, -19.350000) {\hbox{\tate エキ}};
\node[Meaning] at (-2.700000, -17.700000) {epidemic};
\node[Kanji] at (-0.650000, -18.950000) {\textcolor[HTML]{a3bac2}{痛}};
\node[Square] at (-0.650000, -19.450000) {};
\node[Onyomi] at (-0.600000, -19.350000) {\hbox{\tate ツウ}};
\node[Kunyomi] at (-0.700000, -19.350000) {\hbox{\tate いた.い}};
\node[Meaning] at (-0.650000, -17.700000) {pain};
\node[Kanji] at (1.400000, -18.950000) {\textcolor[HTML]{408dba}{癖}};
\node[Square] at (1.400000, -19.450000) {};
\node[Onyomi] at (1.450000, -19.350000) {\hbox{\tate ヘキ}};
\node[Kunyomi] at (1.350000, -19.350000) {\hbox{\tate くせ}};
\node[Meaning] at (1.400000, -17.700000) {habit};
\node[Kanji] at (3.450000, -18.950000) {\textcolor[HTML]{68a4bc}{匿}};
\node[Square] at (3.450000, -19.450000) {};
\node[Onyomi] at (3.500000, -19.350000) {\hbox{\tate トク}};
\node[Kunyomi] at (3.400000, -19.350000) {\hbox{\tate かくま.う}};
\node[Meaning] at (3.450000, -17.700000) {hide};
\node[Kanji] at (5.500000, -18.950000) {\textcolor[HTML]{91b7c3}{匠}};
\node[Square] at (5.500000, -19.450000) {};
\node[Onyomi] at (5.550000, -19.350000) {\hbox{\tate ショウ}};
\node[Kunyomi] at (5.450000, -19.350000) {\hbox{\tate たくみ}};
\node[Meaning] at (5.500000, -17.700000) {artisan};
\node[Kanji] at (7.550000, -18.950000) {\textcolor[HTML]{d2a293}{医}};
\node[Square] at (7.550000, -19.450000) {};
\node[Onyomi] at (7.600000, -19.350000) {\hbox{\tate イ}};
\node[Meaning] at (7.550000, -17.700000) {medicine};
\node[Kanji] at (9.600000, -18.950000) {\textcolor[HTML]{91b7c3}{匹}};
\node[Square] at (9.600000, -19.450000) {};
\node[Kunyomi] at (9.550000, -19.350000) {\hbox{\tate ひき}};
\node[Meaning] at (9.600000, -17.700000) {small animal};
\node[Kanji] at (11.650000, -18.950000) {\textcolor[HTML]{c36143}{区}};
\node[Square] at (11.650000, -19.450000) {};
\node[Onyomi] at (11.700000, -19.350000) {\hbox{\tate ク}};
\node[Meaning] at (11.650000, -17.700000) {district};
\node[Kanji] at (13.700000, -18.950000) {\textcolor[HTML]{68a4bc}{枢}};
\node[Square] at (13.700000, -19.450000) {};
\node[Onyomi] at (13.750000, -19.350000) {\hbox{\tate スウ}};
\node[Kunyomi] at (13.650000, -19.350000) {\hbox{\tate からくり}};
\node[Meaning] at (13.700000, -17.700000) {hinge};
\node[Kanji] at (15.750000, -18.950000) {\textcolor[HTML]{408dba}{殴}};
\node[Square] at (15.750000, -19.450000) {};
\node[Onyomi] at (15.800000, -19.350000) {\hbox{\tate オウ}};
\node[Kunyomi] at (15.700000, -19.350000) {\hbox{\tate なぐ.る}};
\node[Meaning] at (15.750000, -17.700000) {assault};
\node[Kanji] at (17.800000, -18.950000) {\textcolor[HTML]{c8a59d}{欧}};
\node[Square] at (17.800000, -19.450000) {};
\node[Onyomi] at (17.850000, -19.350000) {\hbox{\tate オウ}};
\node[Meaning] at (17.800000, -17.700000) {europe};
\node[Kanji] at (19.850000, -18.950000) {\textcolor[HTML]{b0b0b5}{抑}};
\node[Square] at (19.850000, -19.450000) {};
\node[Onyomi] at (19.900000, -19.350000) {\hbox{\tate ヨク}};
\node[Kunyomi] at (19.800000, -19.350000) {\hbox{\tate おさ.える}};
\node[Meaning] at (19.850000, -17.700000) {suppress};
\node[Kanji] at (21.900000, -18.950000) {\textcolor[HTML]{b0b0b5}{仰}};
\node[Square] at (21.900000, -19.450000) {};
\node[Onyomi] at (21.950000, -19.350000) {\hbox{\tate ギョウ}};
\node[Kunyomi] at (21.850000, -19.350000) {\hbox{\tate あお.ぐ}};
\node[Meaning] at (21.900000, -17.700000) {look up to};
\node[Kanji] at (23.950000, -18.950000) {\textcolor[HTML]{c8a59d}{迎}};
\node[Square] at (23.950000, -19.450000) {};
\node[Onyomi] at (24.000000, -19.350000) {\hbox{\tate ゲイ}};
\node[Kunyomi] at (23.900000, -19.350000) {\hbox{\tate むか.える}};
\node[Meaning] at (23.950000, -17.700000) {welcome};
\node[Kanji] at (26.000000, -18.950000) {\textcolor[HTML]{cd8268}{登}};
\node[Square] at (26.000000, -19.450000) {};
\node[Onyomi] at (26.050000, -19.350000) {\hbox{\tate トウ・ト}};
\node[Kunyomi] at (25.950000, -19.350000) {\hbox{\tate のぼ.る}};
\node[Meaning] at (26.000000, -17.700000) {climb};
\node[Kanji] at (28.050000, -18.950000) {\textcolor[HTML]{91b7c3}{澄}};
\node[Square] at (28.050000, -19.450000) {};
\node[Onyomi] at (28.100000, -19.350000) {\hbox{\tate チョウ}};
\node[Kunyomi] at (28.000000, -19.350000) {\hbox{\tate す.*}};
\node[Meaning] at (28.050000, -17.700000) {lucidity};
\node[Kanji] at (30.100000, -18.950000) {\textcolor[HTML]{a11d25}{発}};
\node[Square] at (30.100000, -19.450000) {};
\node[Onyomi] at (30.150000, -19.350000) {\hbox{\tate ハツ}};
\node[Meaning] at (30.100000, -17.700000) {departure};
\node[Kanji] at (32.150000, -18.950000) {\textcolor[HTML]{d69f8d}{廃}};
\node[Square] at (32.150000, -19.450000) {};
\node[Onyomi] at (32.200000, -19.350000) {\hbox{\tate ハイ}};
\node[Kunyomi] at (32.100000, -19.350000) {\hbox{\tate すた}};
\node[Meaning] at (32.150000, -17.700000) {obsolete};
\node[Kanji] at (34.200000, -18.950000) {\textcolor[HTML]{a3bac2}{僚}};
\node[Square] at (34.200000, -19.450000) {};
\node[Onyomi] at (34.250000, -19.350000) {\hbox{\tate リョウ}};
\node[Meaning] at (34.200000, -17.700000) {colleague};
\node[Kanji] at (36.250000, -18.950000) {\textcolor[HTML]{91b7c3}{寮}};
\node[Square] at (36.250000, -19.450000) {};
\node[Onyomi] at (36.300000, -19.350000) {\hbox{\tate リョウ}};
\node[Meaning] at (36.250000, -17.700000) {dormitory};
\node[Kanji] at (38.300000, -18.950000) {\textcolor[HTML]{d2a293}{療}};
\node[Square] at (38.300000, -19.450000) {};
\node[Onyomi] at (38.350000, -19.350000) {\hbox{\tate リョウ}};
\node[Meaning] at (38.300000, -17.700000) {heal};
\node[Kanji] at (40.350000, -18.950000) {\textcolor[HTML]{91b7c3}{彫}};
\node[Square] at (40.350000, -19.450000) {};
\node[Onyomi] at (40.400000, -19.350000) {\hbox{\tate チョウ}};
\node[Kunyomi] at (40.300000, -19.350000) {\hbox{\tate ほ.る}};
\node[Meaning] at (40.350000, -17.700000) {carve};
\node[Kanji] at (42.400000, -18.950000) {\textcolor[HTML]{c36143}{形}};
\node[Square] at (42.400000, -19.450000) {};
\node[Onyomi] at (42.450000, -19.350000) {\hbox{\tate ケイ}};
\node[Kunyomi] at (42.350000, -19.350000) {\hbox{\tate かた・かたち}};
\node[Meaning] at (42.400000, -17.700000) {shape};
\node[Kanji] at (44.450000, -18.950000) {\textcolor[HTML]{d69f8d}{影}};
\node[Square] at (44.450000, -19.450000) {};
\node[Onyomi] at (44.500000, -19.350000) {\hbox{\tate エイ}};
\node[Kunyomi] at (44.400000, -19.350000) {\hbox{\tate かげ}};
\node[Meaning] at (44.450000, -17.700000) {shadow};
\node[Kanji] at (46.500000, -18.950000) {\textcolor[HTML]{b0b0b5}{杉}};
\node[Square] at (46.500000, -19.450000) {};
\node[Kunyomi] at (46.450000, -19.350000) {\hbox{\tate すぎ}};
\node[Meaning] at (46.500000, -17.700000) {cedar};
\node[Kanji] at (48.550000, -18.950000) {\textcolor[HTML]{a3bac2}{彩}};
\node[Square] at (48.550000, -19.450000) {};
\node[Onyomi] at (48.600000, -19.350000) {\hbox{\tate サイ}};
\node[Kunyomi] at (48.500000, -19.350000) {\hbox{\tate いろど.る}};
\node[Meaning] at (48.550000, -17.700000) {coloring};
\node[Kanji] at (50.600000, -18.950000) {\textcolor[HTML]{91b7c3}{彰}};
\node[Square] at (50.600000, -19.450000) {};
\node[Onyomi] at (50.650000, -19.350000) {\hbox{\tate ショウ}};
\node[Meaning] at (50.600000, -17.700000) {patent};
\node[Kanji] at (52.650000, -18.950000) {\textcolor[HTML]{b0b0b5}{顔}};
\node[Square] at (52.650000, -19.450000) {};
\node[Onyomi] at (52.700000, -19.350000) {\hbox{\tate ガン}};
\node[Kunyomi] at (52.600000, -19.350000) {\hbox{\tate かお}};
\node[Meaning] at (52.650000, -17.700000) {face};
\node[Kanji] at (54.700000, -18.950000) {\textcolor[HTML]{b0b0b5}{須}};
\node[Square] at (54.700000, -19.450000) {};
\node[Onyomi] at (54.750000, -19.350000) {\hbox{\tate ス・シュ}};
\node[Kunyomi] at (54.650000, -19.350000) {\hbox{\tate すべから}};
\node[Meaning] at (54.700000, -17.700000) {necessary};
\node[Kanji] at (56.750000, -18.950000) {\textcolor[HTML]{91b7c3}{膨}};
\node[Square] at (56.750000, -19.450000) {};
\node[Onyomi] at (56.800000, -19.350000) {\hbox{\tate ボウ}};
\node[Kunyomi] at (56.700000, -19.350000) {\hbox{\tate ふく}};
\node[Meaning] at (56.750000, -17.700000) {swell};
\node[Meaning] at (-58.050000, -18.850000) {86.99\%};
\node[Kanji] at (-56.000000, -21.000000) {\textcolor[HTML]{c36143}{参}};
\node[Square] at (-56.000000, -21.500000) {};
\node[Onyomi] at (-55.950000, -21.400000) {\hbox{\tate サン}};
\node[Kunyomi] at (-56.050000, -21.400000) {\hbox{\tate まい.る}};
\node[Meaning] at (-56.000000, -19.750000) {participate};
\node[Kanji] at (-53.950000, -21.000000) {\textcolor[HTML]{68a4bc}{惨}};
\node[Square] at (-53.950000, -21.500000) {};
\node[Onyomi] at (-53.900000, -21.400000) {\hbox{\tate サン・ザン}};
\node[Kunyomi] at (-54.000000, -21.400000) {\hbox{\tate みじ・いた}};
\node[Meaning] at (-53.950000, -19.750000) {disaster};
\node[Kanji] at (-51.900000, -21.000000) {\textcolor[HTML]{d69f8d}{修}};
\node[Square] at (-51.900000, -21.500000) {};
\node[Onyomi] at (-51.850000, -21.400000) {\hbox{\tate シュウ}};
\node[Kunyomi] at (-51.950000, -21.400000) {\hbox{\tate おさ.まる}};
\node[Meaning] at (-51.900000, -19.750000) {discipline};
\node[Kanji] at (-49.850000, -21.000000) {\textcolor[HTML]{a3bac2}{珍}};
\node[Square] at (-49.850000, -21.500000) {};
\node[Onyomi] at (-49.800000, -21.400000) {\hbox{\tate チン}};
\node[Kunyomi] at (-49.900000, -21.400000) {\hbox{\tate めずら.しい}};
\node[Meaning] at (-49.850000, -19.750000) {rare};
\node[Kanji] at (-47.800000, -21.000000) {\textcolor[HTML]{a3bac2}{診}};
\node[Square] at (-47.800000, -21.500000) {};
\node[Onyomi] at (-47.750000, -21.400000) {\hbox{\tate シン}};
\node[Kunyomi] at (-47.850000, -21.400000) {\hbox{\tate み.る}};
\node[Meaning] at (-47.800000, -19.750000) {diagnose};
\node[Kanji] at (-45.750000, -21.000000) {\textcolor[HTML]{b74029}{文}};
\node[Square] at (-45.750000, -21.500000) {};
\node[Onyomi] at (-45.700000, -21.400000) {\hbox{\tate ブン・モン}};
\node[Meaning] at (-45.750000, -19.750000) {writing};
\node[Kanji] at (-43.700000, -21.000000) {\textcolor[HTML]{b74029}{対}};
\node[Square] at (-43.700000, -21.500000) {};
\node[Onyomi] at (-43.650000, -21.400000) {\hbox{\tate タイ}};
\node[Meaning] at (-43.700000, -19.750000) {versus};
\node[Kanji] at (-41.650000, -21.000000) {\textcolor[HTML]{a3bac2}{紋}};
\node[Square] at (-41.650000, -21.500000) {};
\node[Onyomi] at (-41.600000, -21.400000) {\hbox{\tate モン}};
\node[Meaning] at (-41.650000, -19.750000) {family crest};
\node[Kanji] at (-39.600000, -21.000000) {\textcolor[HTML]{1059be}{蚊}};
\node[Square] at (-39.600000, -21.500000) {};
\node[Kunyomi] at (-39.650000, -21.400000) {\hbox{\tate か}};
\node[Meaning] at (-39.600000, -19.750000) {mosquito};
\node[Kanji] at (-37.550000, -21.000000) {\textcolor[HTML]{a3bac2}{斉}};
\node[Square] at (-37.550000, -21.500000) {};
\node[Onyomi] at (-37.500000, -21.400000) {\hbox{\tate セイ}};
\node[Meaning] at (-37.550000, -19.750000) {simultaneous};
\node[Kanji] at (-35.500000, -21.000000) {\textcolor[HTML]{b0b0b5}{剤}};
\node[Square] at (-35.500000, -21.500000) {};
\node[Onyomi] at (-35.450000, -21.400000) {\hbox{\tate ザイ}};
\node[Meaning] at (-35.500000, -19.750000) {dose};
\node[Kanji] at (-33.450000, -21.000000) {\textcolor[HTML]{d69f8d}{済}};
\node[Square] at (-33.450000, -21.500000) {};
\node[Onyomi] at (-33.400000, -21.400000) {\hbox{\tate サイ}};
\node[Kunyomi] at (-33.500000, -21.400000) {\hbox{\tate す.ます}};
\node[Meaning] at (-33.450000, -19.750000) {come to an end};
\node[Kanji] at (-31.400000, -21.000000) {\textcolor[HTML]{a3bac2}{斎}};
\node[Square] at (-31.400000, -21.500000) {};
\node[Onyomi] at (-31.350000, -21.400000) {\hbox{\tate サイ}};
\node[Kunyomi] at (-31.450000, -21.400000) {\hbox{\tate いつ.く}};
\node[Meaning] at (-31.400000, -19.750000) {purification};
\node[Kanji] at (-29.350000, -21.000000) {\textcolor[HTML]{68a4bc}{粛}};
\node[Square] at (-29.350000, -21.500000) {};
\node[Onyomi] at (-29.300000, -21.400000) {\hbox{\tate シュク・スク}};
\node[Kunyomi] at (-29.400000, -21.400000) {\hbox{\tate つつし}};
\node[Meaning] at (-29.350000, -19.750000) {solemn};
\node[Kanji] at (-27.300000, -21.000000) {\textcolor[HTML]{b0b0b5}{塁}};
\node[Square] at (-27.300000, -21.500000) {};
\node[Onyomi] at (-27.250000, -21.400000) {\hbox{\tate ルイ}};
\node[Meaning] at (-27.300000, -19.750000) {base};
\node[Kanji] at (-25.250000, -21.000000) {\textcolor[HTML]{cd8268}{楽}};
\node[Square] at (-25.250000, -21.500000) {};
\node[Onyomi] at (-25.200000, -21.400000) {\hbox{\tate ガク・ラク}};
\node[Kunyomi] at (-25.300000, -21.400000) {\hbox{\tate たの.しい}};
\node[Meaning] at (-25.250000, -19.750000) {comfort};
\node[Kanji] at (-23.200000, -21.000000) {\textcolor[HTML]{d2a293}{薬}};
\node[Square] at (-23.200000, -21.500000) {};
\node[Onyomi] at (-23.150000, -21.400000) {\hbox{\tate ヤク}};
\node[Kunyomi] at (-23.250000, -21.400000) {\hbox{\tate くすり}};
\node[Meaning] at (-23.200000, -19.750000) {medicine};
\node[Kanji] at (-21.150000, -21.000000) {\textcolor[HTML]{d69f8d}{率}};
\node[Square] at (-21.150000, -21.500000) {};
\node[Onyomi] at (-21.100000, -21.400000) {\hbox{\tate リツ・ソツ}};
\node[Kunyomi] at (-21.200000, -21.400000) {\hbox{\tate ひき.いる}};
\node[Meaning] at (-21.150000, -19.750000) {percent};
\node[Kanji] at (-19.100000, -21.000000) {\textcolor[HTML]{a3bac2}{渋}};
\node[Square] at (-19.100000, -21.500000) {};
\node[Onyomi] at (-19.050000, -21.400000) {\hbox{\tate ジュウ}};
\node[Kunyomi] at (-19.150000, -21.400000) {\hbox{\tate しぶ.い}};
\node[Meaning] at (-19.100000, -19.750000) {bitter};
\node[Kanji] at (-17.050000, -21.000000) {\textcolor[HTML]{a3bac2}{摂}};
\node[Square] at (-17.050000, -21.500000) {};
\node[Onyomi] at (-17.000000, -21.400000) {\hbox{\tate セツ・ショウ}};
\node[Kunyomi] at (-17.100000, -21.400000) {\hbox{\tate おさ・かね}};
\node[Meaning] at (-17.050000, -19.750000) {in addition};
\node[Kanji] at (-15.000000, -21.000000) {\textcolor[HTML]{d69f8d}{央}};
\node[Square] at (-15.000000, -21.500000) {};
\node[Onyomi] at (-14.950000, -21.400000) {\hbox{\tate オウ}};
\node[Meaning] at (-15.000000, -19.750000) {central};
\node[Kanji] at (-12.950000, -21.000000) {\textcolor[HTML]{d69f8d}{英}};
\node[Square] at (-12.950000, -21.500000) {};
\node[Onyomi] at (-12.900000, -21.400000) {\hbox{\tate エイ}};
\node[Meaning] at (-12.950000, -19.750000) {england};
\node[Kanji] at (-10.900000, -21.000000) {\textcolor[HTML]{cd8268}{映}};
\node[Square] at (-10.900000, -21.500000) {};
\node[Onyomi] at (-10.850000, -21.400000) {\hbox{\tate エイ}};
\node[Kunyomi] at (-10.950000, -21.400000) {\hbox{\tate うつ・は}};
\node[Meaning] at (-10.900000, -19.750000) {reflect};
\node[Kanji] at (-8.850000, -21.000000) {\textcolor[HTML]{d2a293}{赤}};
\node[Square] at (-8.850000, -21.500000) {};
\node[Onyomi] at (-8.800000, -21.400000) {\hbox{\tate セキ}};
\node[Kunyomi] at (-8.900000, -21.400000) {\hbox{\tate あか}};
\node[Meaning] at (-8.850000, -19.750000) {red};
\node[Kanji] at (-6.800000, -21.000000) {\textcolor[HTML]{408dba}{赦}};
\node[Square] at (-6.800000, -21.500000) {};
\node[Onyomi] at (-6.750000, -21.400000) {\hbox{\tate シャ}};
\node[Meaning] at (-6.800000, -19.750000) {pardon};
\node[Kanji] at (-4.750000, -21.000000) {\textcolor[HTML]{cd8268}{変}};
\node[Square] at (-4.750000, -21.500000) {};
\node[Onyomi] at (-4.700000, -21.400000) {\hbox{\tate ヘン}};
\node[Kunyomi] at (-4.800000, -21.400000) {\hbox{\tate か.*}};
\node[Meaning] at (-4.750000, -19.750000) {change};
\node[Kanji] at (-2.700000, -21.000000) {\textcolor[HTML]{d2a293}{跡}};
\node[Square] at (-2.700000, -21.500000) {};
\node[Onyomi] at (-2.650000, -21.400000) {\hbox{\tate セキ}};
\node[Kunyomi] at (-2.750000, -21.400000) {\hbox{\tate あと}};
\node[Meaning] at (-2.700000, -19.750000) {traces};
\node[Kanji] at (-0.650000, -21.000000) {\textcolor[HTML]{1e76bb}{蛮}};
\node[Square] at (-0.650000, -21.500000) {};
\node[Onyomi] at (-0.600000, -21.400000) {\hbox{\tate バン}};
\node[Kunyomi] at (-0.700000, -21.400000) {\hbox{\tate えびす}};
\node[Meaning] at (-0.650000, -19.750000) {barbarian};
\node[Kanji] at (1.400000, -21.000000) {\textcolor[HTML]{b0b0b5}{恋}};
\node[Square] at (1.400000, -21.500000) {};
\node[Onyomi] at (1.450000, -21.400000) {\hbox{\tate レン}};
\node[Kunyomi] at (1.350000, -21.400000) {\hbox{\tate こい}};
\node[Meaning] at (1.400000, -19.750000) {romance};
\node[Kanji] at (3.450000, -21.000000) {\textcolor[HTML]{c8a59d}{湾}};
\node[Square] at (3.450000, -21.500000) {};
\node[Onyomi] at (3.500000, -21.400000) {\hbox{\tate ワン}};
\node[Meaning] at (3.450000, -19.750000) {gulf};
\node[Kanji] at (5.500000, -21.000000) {\textcolor[HTML]{c8a59d}{黄}};
\node[Square] at (5.500000, -21.500000) {};
\node[Onyomi] at (5.550000, -21.400000) {\hbox{\tate オウ}};
\node[Kunyomi] at (5.450000, -21.400000) {\hbox{\tate き}};
\node[Meaning] at (5.500000, -19.750000) {yellow};
\node[Kanji] at (7.550000, -21.000000) {\textcolor[HTML]{d69f8d}{横}};
\node[Square] at (7.550000, -21.500000) {};
\node[Onyomi] at (7.600000, -21.400000) {\hbox{\tate オウ}};
\node[Kunyomi] at (7.500000, -21.400000) {\hbox{\tate よこ}};
\node[Meaning] at (7.550000, -19.750000) {side};
\node[Kanji] at (9.600000, -21.000000) {\textcolor[HTML]{68a4bc}{把}};
\node[Square] at (9.600000, -21.500000) {};
\node[Onyomi] at (9.650000, -21.400000) {\hbox{\tate ワ・ハ}};
\node[Meaning] at (9.600000, -19.750000) {bundle};
\node[Kanji] at (11.650000, -21.000000) {\textcolor[HTML]{cd8268}{色}};
\node[Square] at (11.650000, -21.500000) {};
\node[Onyomi] at (11.700000, -21.400000) {\hbox{\tate シキ・ショク}};
\node[Kunyomi] at (11.600000, -21.400000) {\hbox{\tate いろ}};
\node[Meaning] at (11.650000, -19.750000) {color};
\node[Kanji] at (13.700000, -21.000000) {\textcolor[HTML]{c8a59d}{絶}};
\node[Square] at (13.700000, -21.500000) {};
\node[Onyomi] at (13.750000, -21.400000) {\hbox{\tate ゼツ}};
\node[Kunyomi] at (13.650000, -21.400000) {\hbox{\tate た.*}};
\node[Meaning] at (13.700000, -19.750000) {extinction};
\node[Kanji] at (15.750000, -21.000000) {\textcolor[HTML]{1059be}{艶}};
\node[Square] at (15.750000, -21.500000) {};
\node[Onyomi] at (15.800000, -21.400000) {\hbox{\tate エン}};
\node[Kunyomi] at (15.700000, -21.400000) {\hbox{\tate つや}};
\node[Meaning] at (15.750000, -19.750000) {glossy};
\node[Kanji] at (17.800000, -21.000000) {\textcolor[HTML]{a3bac2}{肥}};
\node[Square] at (17.800000, -21.500000) {};
\node[Onyomi] at (17.850000, -21.400000) {\hbox{\tate ヒ}};
\node[Kunyomi] at (17.750000, -21.400000) {\hbox{\tate こ.える}};
\node[Meaning] at (17.800000, -19.750000) {obese};
\node[Kanji] at (19.850000, -21.000000) {\textcolor[HTML]{91b7c3}{甘}};
\node[Square] at (19.850000, -21.500000) {};
\node[Onyomi] at (19.900000, -21.400000) {\hbox{\tate カン}};
\node[Kunyomi] at (19.800000, -21.400000) {\hbox{\tate あま}};
\node[Meaning] at (19.850000, -19.750000) {sweet};
\node[Kanji] at (21.900000, -21.000000) {\textcolor[HTML]{408dba}{紺}};
\node[Square] at (21.900000, -21.500000) {};
\node[Onyomi] at (21.950000, -21.400000) {\hbox{\tate コン}};
\node[Meaning] at (21.900000, -19.750000) {navy};
\node[Kanji] at (23.950000, -21.000000) {\textcolor[HTML]{1e76bb}{某}};
\node[Square] at (23.950000, -21.500000) {};
\node[Onyomi] at (24.000000, -21.400000) {\hbox{\tate ボウ}};
\node[Kunyomi] at (23.900000, -21.400000) {\hbox{\tate それがし}};
\node[Meaning] at (23.950000, -19.750000) {certain};
\node[Kanji] at (26.000000, -21.000000) {\textcolor[HTML]{a3bac2}{謀}};
\node[Square] at (26.000000, -21.500000) {};
\node[Onyomi] at (26.050000, -21.400000) {\hbox{\tate ボウ}};
\node[Kunyomi] at (25.950000, -21.400000) {\hbox{\tate はか.る}};
\node[Meaning] at (26.000000, -19.750000) {conspire};
\node[Kanji] at (28.050000, -21.000000) {\textcolor[HTML]{91b7c3}{媒}};
\node[Square] at (28.050000, -21.500000) {};
\node[Onyomi] at (28.100000, -21.400000) {\hbox{\tate バイ}};
\node[Kunyomi] at (28.000000, -21.400000) {\hbox{\tate なこうど}};
\node[Meaning] at (28.050000, -19.750000) {mediator};
\node[Kanji] at (30.100000, -21.000000) {\textcolor[HTML]{408dba}{欺}};
\node[Square] at (30.100000, -21.500000) {};
\node[Onyomi] at (30.150000, -21.400000) {\hbox{\tate ギ}};
\node[Kunyomi] at (30.050000, -21.400000) {\hbox{\tate あざむ.く}};
\node[Meaning] at (30.100000, -19.750000) {deceit};
\node[Kanji] at (32.150000, -21.000000) {\textcolor[HTML]{a3bac2}{棋}};
\node[Square] at (32.150000, -21.500000) {};
\node[Onyomi] at (32.200000, -21.400000) {\hbox{\tate キ}};
\node[Kunyomi] at (32.100000, -21.400000) {\hbox{\tate ご}};
\node[Meaning] at (32.150000, -19.750000) {chess piece};
\node[Kanji] at (34.200000, -21.000000) {\textcolor[HTML]{b0b0b5}{旗}};
\node[Square] at (34.200000, -21.500000) {};
\node[Onyomi] at (34.250000, -21.400000) {\hbox{\tate キ}};
\node[Kunyomi] at (34.150000, -21.400000) {\hbox{\tate はた}};
\node[Meaning] at (34.200000, -19.750000) {flag};
\node[Kanji] at (36.250000, -21.000000) {\textcolor[HTML]{c36143}{期}};
\node[Square] at (36.250000, -21.500000) {};
\node[Onyomi] at (36.300000, -21.400000) {\hbox{\tate キ}};
\node[Meaning] at (36.250000, -19.750000) {period of time};
\node[Kanji] at (38.300000, -21.000000) {\textcolor[HTML]{68a4bc}{碁}};
\node[Square] at (38.300000, -21.500000) {};
\node[Onyomi] at (38.350000, -21.400000) {\hbox{\tate ゴ}};
\node[Meaning] at (38.300000, -19.750000) {go};
\node[Kanji] at (40.350000, -21.000000) {\textcolor[HTML]{cd8268}{基}};
\node[Square] at (40.350000, -21.500000) {};
\node[Onyomi] at (40.400000, -21.400000) {\hbox{\tate キ}};
\node[Kunyomi] at (40.300000, -21.400000) {\hbox{\tate もと}};
\node[Meaning] at (40.350000, -19.750000) {foundation};
\node[Kanji] at (42.400000, -21.000000) {\textcolor[HTML]{408dba}{甚}};
\node[Square] at (42.400000, -21.500000) {};
\node[Onyomi] at (42.450000, -21.400000) {\hbox{\tate ジン}};
\node[Kunyomi] at (42.350000, -21.400000) {\hbox{\tate はなは}};
\node[Meaning] at (42.400000, -19.750000) {very};
\node[Kanji] at (44.450000, -21.000000) {\textcolor[HTML]{91b7c3}{勘}};
\node[Square] at (44.450000, -21.500000) {};
\node[Onyomi] at (44.500000, -21.400000) {\hbox{\tate カン}};
\node[Meaning] at (44.450000, -19.750000) {intuition};
\node[Kanji] at (46.500000, -21.000000) {\textcolor[HTML]{1e76bb}{堪}};
\node[Square] at (46.500000, -21.500000) {};
\node[Onyomi] at (46.550000, -21.400000) {\hbox{\tate カン・タン}};
\node[Kunyomi] at (46.450000, -21.400000) {\hbox{\tate た・こた・こ}};
\node[Meaning] at (46.500000, -19.750000) {endure};
\node[Kanji] at (48.550000, -21.000000) {\textcolor[HTML]{c8a59d}{貴}};
\node[Square] at (48.550000, -21.500000) {};
\node[Onyomi] at (48.600000, -21.400000) {\hbox{\tate キ}};
\node[Kunyomi] at (48.500000, -21.400000) {\hbox{\tate とうと.い}};
\node[Meaning] at (48.550000, -19.750000) {valuable};
\node[Kanji] at (50.600000, -21.000000) {\textcolor[HTML]{d2a293}{遺}};
\node[Square] at (50.600000, -21.500000) {};
\node[Onyomi] at (50.650000, -21.400000) {\hbox{\tate イ}};
\node[Kunyomi] at (50.550000, -21.400000) {\hbox{\tate のこ.す}};
\node[Meaning] at (50.600000, -19.750000) {leave behind};
\node[Kanji] at (52.650000, -21.000000) {\textcolor[HTML]{b0b0b5}{遣}};
\node[Square] at (52.650000, -21.500000) {};
\node[Onyomi] at (52.700000, -21.400000) {\hbox{\tate ケン}};
\node[Kunyomi] at (52.600000, -21.400000) {\hbox{\tate つか.う}};
\node[Meaning] at (52.650000, -19.750000) {dispatch};
\node[Kanji] at (54.700000, -21.000000) {\textcolor[HTML]{d2a293}{舞}};
\node[Square] at (54.700000, -21.500000) {};
\node[Onyomi] at (54.750000, -21.400000) {\hbox{\tate ブ}};
\node[Kunyomi] at (54.650000, -21.400000) {\hbox{\tate まい・ま.う}};
\node[Meaning] at (54.700000, -19.750000) {dance};
\node[Kanji] at (56.750000, -21.000000) {\textcolor[HTML]{cd8268}{無}};
\node[Square] at (56.750000, -21.500000) {};
\node[Onyomi] at (56.800000, -21.400000) {\hbox{\tate ム・ブ}};
\node[Kunyomi] at (56.700000, -21.400000) {\hbox{\tate な.い}};
\node[Meaning] at (56.750000, -19.750000) {nothing};
\node[Meaning] at (-58.050000, -20.900000) {89.24\%};
\node[Kanji] at (-56.000000, -23.050000) {\textcolor[HTML]{c36143}{組}};
\node[Square] at (-56.000000, -23.550000) {};
\node[Onyomi] at (-55.950000, -23.450000) {\hbox{\tate ソ}};
\node[Kunyomi] at (-56.050000, -23.450000) {\hbox{\tate くみ}};
\node[Meaning] at (-56.000000, -21.800000) {group};
\node[Kanji] at (-53.950000, -23.050000) {\textcolor[HTML]{68a4bc}{粗}};
\node[Square] at (-53.950000, -23.550000) {};
\node[Onyomi] at (-53.900000, -23.450000) {\hbox{\tate ソ}};
\node[Kunyomi] at (-54.000000, -23.450000) {\hbox{\tate あら}};
\node[Meaning] at (-53.950000, -21.800000) {coarse};
\node[Kanji] at (-51.900000, -23.050000) {\textcolor[HTML]{408dba}{租}};
\node[Square] at (-51.900000, -23.550000) {};
\node[Onyomi] at (-51.850000, -23.450000) {\hbox{\tate ソ}};
\node[Meaning] at (-51.900000, -21.800000) {tariff};
\node[Kanji] at (-49.850000, -23.050000) {\textcolor[HTML]{c8a59d}{祖}};
\node[Square] at (-49.850000, -23.550000) {};
\node[Onyomi] at (-49.800000, -23.450000) {\hbox{\tate ソ}};
\node[Meaning] at (-49.850000, -21.800000) {ancestor};
\node[Kanji] at (-47.800000, -23.050000) {\textcolor[HTML]{91b7c3}{阻}};
\node[Square] at (-47.800000, -23.550000) {};
\node[Onyomi] at (-47.750000, -23.450000) {\hbox{\tate ソ}};
\node[Kunyomi] at (-47.850000, -23.450000) {\hbox{\tate はば.む}};
\node[Meaning] at (-47.800000, -21.800000) {thwart};
\node[Kanji] at (-45.750000, -23.050000) {\textcolor[HTML]{d69f8d}{査}};
\node[Square] at (-45.750000, -23.550000) {};
\node[Onyomi] at (-45.700000, -23.450000) {\hbox{\tate サ}};
\node[Meaning] at (-45.750000, -21.800000) {inspect};
\node[Kanji] at (-43.700000, -23.050000) {\textcolor[HTML]{d69f8d}{助}};
\node[Square] at (-43.700000, -23.550000) {};
\node[Onyomi] at (-43.650000, -23.450000) {\hbox{\tate ジョ}};
\node[Kunyomi] at (-43.750000, -23.450000) {\hbox{\tate たす・すけ}};
\node[Meaning] at (-43.700000, -21.800000) {help};
\node[Kanji] at (-41.650000, -23.050000) {\textcolor[HTML]{91b7c3}{宜}};
\node[Square] at (-41.650000, -23.550000) {};
\node[Onyomi] at (-41.600000, -23.450000) {\hbox{\tate ギ}};
\node[Kunyomi] at (-41.700000, -23.450000) {\hbox{\tate よろ}};
\node[Meaning] at (-41.650000, -21.800000) {best regards};
\node[Kanji] at (-39.600000, -23.050000) {\textcolor[HTML]{68a4bc}{畳}};
\node[Square] at (-39.600000, -23.550000) {};
\node[Onyomi] at (-39.550000, -23.450000) {\hbox{\tate ジョウ}};
\node[Kunyomi] at (-39.650000, -23.450000) {\hbox{\tate たたみ}};
\node[Meaning] at (-39.600000, -21.800000) {tatami mat};
\node[Kanji] at (-37.550000, -23.050000) {\textcolor[HTML]{d2a293}{並}};
\node[Square] at (-37.550000, -23.550000) {};
\node[Onyomi] at (-37.500000, -23.450000) {\hbox{\tate ヘイ}};
\node[Kunyomi] at (-37.600000, -23.450000) {\hbox{\tate なら.*}};
\node[Meaning] at (-37.550000, -21.800000) {line up};
\node[Kanji] at (-35.500000, -23.050000) {\textcolor[HTML]{d2a293}{普}};
\node[Square] at (-35.500000, -23.550000) {};
\node[Onyomi] at (-35.450000, -23.450000) {\hbox{\tate フ}};
\node[Meaning] at (-35.500000, -21.800000) {normal};
\node[Kanji] at (-33.450000, -23.050000) {\textcolor[HTML]{a3bac2}{譜}};
\node[Square] at (-33.450000, -23.550000) {};
\node[Onyomi] at (-33.400000, -23.450000) {\hbox{\tate フ}};
\node[Meaning] at (-33.450000, -21.800000) {genealogy};
\node[Kanji] at (-31.400000, -23.050000) {\textcolor[HTML]{a3bac2}{湿}};
\node[Square] at (-31.400000, -23.550000) {};
\node[Onyomi] at (-31.350000, -23.450000) {\hbox{\tate シツ}};
\node[Kunyomi] at (-31.450000, -23.450000) {\hbox{\tate しめ.らせる}};
\node[Meaning] at (-31.400000, -21.800000) {damp};
\node[Kanji] at (-29.350000, -23.050000) {\textcolor[HTML]{a3bac2}{顕}};
\node[Square] at (-29.350000, -23.550000) {};
\node[Onyomi] at (-29.300000, -23.450000) {\hbox{\tate ケン}};
\node[Kunyomi] at (-29.400000, -23.450000) {\hbox{\tate あきらか}};
\node[Meaning] at (-29.350000, -21.800000) {appear};
\node[Kanji] at (-27.300000, -23.050000) {\textcolor[HTML]{91b7c3}{繊}};
\node[Square] at (-27.300000, -23.550000) {};
\node[Onyomi] at (-27.250000, -23.450000) {\hbox{\tate セン}};
\node[Meaning] at (-27.300000, -21.800000) {fiber};
\node[Kanji] at (-25.250000, -23.050000) {\textcolor[HTML]{b0b0b5}{霊}};
\node[Square] at (-25.250000, -23.550000) {};
\node[Onyomi] at (-25.200000, -23.450000) {\hbox{\tate レイ・リョウ}};
\node[Meaning] at (-25.250000, -21.800000) {ghost};
\node[Kanji] at (-23.200000, -23.050000) {\textcolor[HTML]{b74029}{業}};
\node[Square] at (-23.200000, -23.550000) {};
\node[Onyomi] at (-23.150000, -23.450000) {\hbox{\tate ギョウ}};
\node[Meaning] at (-23.200000, -21.800000) {business};
\node[Kanji] at (-21.150000, -23.050000) {\textcolor[HTML]{a3bac2}{撲}};
\node[Square] at (-21.150000, -23.550000) {};
\node[Onyomi] at (-21.100000, -23.450000) {\hbox{\tate ボク}};
\node[Meaning] at (-21.150000, -21.800000) {slap};
\node[Kanji] at (-19.100000, -23.050000) {\textcolor[HTML]{91b7c3}{僕}};
\node[Square] at (-19.100000, -23.550000) {};
\node[Onyomi] at (-19.050000, -23.450000) {\hbox{\tate ボク}};
\node[Meaning] at (-19.100000, -21.800000) {i};
\node[Kanji] at (-17.050000, -23.050000) {\textcolor[HTML]{cd8268}{共}};
\node[Square] at (-17.050000, -23.550000) {};
\node[Onyomi] at (-17.000000, -23.450000) {\hbox{\tate キョウ}};
\node[Kunyomi] at (-17.100000, -23.450000) {\hbox{\tate とも}};
\node[Meaning] at (-17.050000, -21.800000) {together};
\node[Kanji] at (-15.000000, -23.050000) {\textcolor[HTML]{d69f8d}{供}};
\node[Square] at (-15.000000, -23.550000) {};
\node[Onyomi] at (-14.950000, -23.450000) {\hbox{\tate キョウ}};
\node[Kunyomi] at (-15.050000, -23.450000) {\hbox{\tate とも}};
\node[Meaning] at (-15.000000, -21.800000) {servant};
\node[Kanji] at (-12.950000, -23.050000) {\textcolor[HTML]{d69f8d}{異}};
\node[Square] at (-12.950000, -23.550000) {};
\node[Onyomi] at (-12.900000, -23.450000) {\hbox{\tate イ}};
\node[Kunyomi] at (-13.000000, -23.450000) {\hbox{\tate こと.*}};
\node[Meaning] at (-12.950000, -21.800000) {differ};
\node[Kanji] at (-10.900000, -23.050000) {\textcolor[HTML]{b0b0b5}{翼}};
\node[Square] at (-10.900000, -23.550000) {};
\node[Onyomi] at (-10.850000, -23.450000) {\hbox{\tate ヨク}};
\node[Kunyomi] at (-10.950000, -23.450000) {\hbox{\tate つばさ}};
\node[Meaning] at (-10.900000, -21.800000) {wing};
\node[Kanji] at (-8.850000, -23.050000) {\textcolor[HTML]{91b7c3}{洪}};
\node[Square] at (-8.850000, -23.550000) {};
\node[Onyomi] at (-8.800000, -23.450000) {\hbox{\tate コウ}};
\node[Meaning] at (-8.850000, -21.800000) {flood};
\node[Kanji] at (-6.800000, -23.050000) {\textcolor[HTML]{d69f8d}{港}};
\node[Square] at (-6.800000, -23.550000) {};
\node[Onyomi] at (-6.750000, -23.450000) {\hbox{\tate コウ}};
\node[Kunyomi] at (-6.850000, -23.450000) {\hbox{\tate みなと}};
\node[Meaning] at (-6.800000, -21.800000) {harbor};
\node[Kanji] at (-4.750000, -23.050000) {\textcolor[HTML]{b0b0b5}{暴}};
\node[Square] at (-4.750000, -23.550000) {};
\node[Onyomi] at (-4.700000, -23.450000) {\hbox{\tate ボウ}};
\node[Kunyomi] at (-4.800000, -23.450000) {\hbox{\tate あば.れる}};
\node[Meaning] at (-4.750000, -21.800000) {violence};
\node[Kanji] at (-2.700000, -23.050000) {\textcolor[HTML]{c8a59d}{爆}};
\node[Square] at (-2.700000, -23.550000) {};
\node[Onyomi] at (-2.650000, -23.450000) {\hbox{\tate バク}};
\node[Kunyomi] at (-2.750000, -23.450000) {\hbox{\tate は.ぜる}};
\node[Meaning] at (-2.700000, -21.800000) {explode};
\node[Kanji] at (-0.650000, -23.050000) {\textcolor[HTML]{68a4bc}{恭}};
\node[Square] at (-0.650000, -23.550000) {};
\node[Onyomi] at (-0.600000, -23.450000) {\hbox{\tate キョウ}};
\node[Kunyomi] at (-0.700000, -23.450000) {\hbox{\tate うやうや}};
\node[Meaning] at (-0.650000, -21.800000) {respect};
\node[Kanji] at (1.400000, -23.050000) {\textcolor[HTML]{c36143}{選}};
\node[Square] at (1.400000, -23.550000) {};
\node[Onyomi] at (1.450000, -23.450000) {\hbox{\tate セン}};
\node[Kunyomi] at (1.350000, -23.450000) {\hbox{\tate えら.ぶ}};
\node[Meaning] at (1.400000, -21.800000) {choose};
\node[Kanji] at (3.450000, -23.050000) {\textcolor[HTML]{c8a59d}{殿}};
\node[Square] at (3.450000, -23.550000) {};
\node[Onyomi] at (3.500000, -23.450000) {\hbox{\tate デン}};
\node[Kunyomi] at (3.400000, -23.450000) {\hbox{\tate との・どの}};
\node[Meaning] at (3.450000, -21.800000) {milord};
\node[Kanji] at (5.500000, -23.050000) {\textcolor[HTML]{cd8268}{井}};
\node[Square] at (5.500000, -23.550000) {};
\node[Onyomi] at (5.550000, -23.450000) {\hbox{\tate ショウ}};
\node[Kunyomi] at (5.450000, -23.450000) {\hbox{\tate い}};
\node[Meaning] at (5.500000, -21.800000) {well};
\node[Kanji] at (7.550000, -23.050000) {\textcolor[HTML]{d2a293}{囲}};
\node[Square] at (7.550000, -23.550000) {};
\node[Onyomi] at (7.600000, -23.450000) {\hbox{\tate イ}};
\node[Kunyomi] at (7.500000, -23.450000) {\hbox{\tate かこ.む}};
\node[Meaning] at (7.550000, -21.800000) {surround};
\node[Kanji] at (9.600000, -23.050000) {\textcolor[HTML]{91b7c3}{耕}};
\node[Square] at (9.600000, -23.550000) {};
\node[Onyomi] at (9.650000, -23.450000) {\hbox{\tate コウ}};
\node[Kunyomi] at (9.550000, -23.450000) {\hbox{\tate たがや.す}};
\node[Meaning] at (9.600000, -21.800000) {plow};
\node[Kanji] at (11.650000, -23.050000) {\textcolor[HTML]{b0b0b5}{亜}};
\node[Square] at (11.650000, -23.550000) {};
\node[Onyomi] at (11.700000, -23.450000) {\hbox{\tate ア}};
\node[Kunyomi] at (11.600000, -23.450000) {\hbox{\tate つ}};
\node[Meaning] at (11.650000, -21.800000) {asia};
\node[Kanji] at (13.700000, -23.050000) {\textcolor[HTML]{d2a293}{悪}};
\node[Square] at (13.700000, -23.550000) {};
\node[Onyomi] at (13.750000, -23.450000) {\hbox{\tate アク・オ}};
\node[Kunyomi] at (13.650000, -23.450000) {\hbox{\tate わる.い}};
\node[Meaning] at (13.700000, -21.800000) {bad};
\node[Kanji] at (15.750000, -23.050000) {\textcolor[HTML]{d69f8d}{円}};
\node[Square] at (15.750000, -23.550000) {};
\node[Onyomi] at (15.800000, -23.450000) {\hbox{\tate エン}};
\node[Kunyomi] at (15.700000, -23.450000) {\hbox{\tate まる.い}};
\node[Meaning] at (15.750000, -21.800000) {yen};
\node[Kanji] at (17.800000, -23.050000) {\textcolor[HTML]{d2a293}{角}};
\node[Square] at (17.800000, -23.550000) {};
\node[Onyomi] at (17.850000, -23.450000) {\hbox{\tate カク}};
\node[Kunyomi] at (17.750000, -23.450000) {\hbox{\tate かど}};
\node[Meaning] at (17.800000, -21.800000) {angle};
\node[Kanji] at (19.850000, -23.050000) {\textcolor[HTML]{b0b0b5}{触}};
\node[Square] at (19.850000, -23.550000) {};
\node[Onyomi] at (19.900000, -23.450000) {\hbox{\tate ショク}};
\node[Kunyomi] at (19.800000, -23.450000) {\hbox{\tate さわ.る}};
\node[Meaning] at (19.850000, -21.800000) {touch};
\node[Kanji] at (21.900000, -23.050000) {\textcolor[HTML]{cd8268}{解}};
\node[Square] at (21.900000, -23.550000) {};
\node[Onyomi] at (21.950000, -23.450000) {\hbox{\tate カイ}};
\node[Kunyomi] at (21.850000, -23.450000) {\hbox{\tate と.く}};
\node[Meaning] at (21.900000, -21.800000) {untie};
\node[Kanji] at (23.950000, -23.050000) {\textcolor[HTML]{cd8268}{再}};
\node[Square] at (23.950000, -23.550000) {};
\node[Onyomi] at (24.000000, -23.450000) {\hbox{\tate サ・サイ}};
\node[Kunyomi] at (23.900000, -23.450000) {\hbox{\tate ふたた.び}};
\node[Meaning] at (23.950000, -21.800000) {again};
\node[Kanji] at (26.000000, -23.050000) {\textcolor[HTML]{d2a293}{講}};
\node[Square] at (26.000000, -23.550000) {};
\node[Onyomi] at (26.050000, -23.450000) {\hbox{\tate コウ}};
\node[Meaning] at (26.000000, -21.800000) {lecture};
\node[Kanji] at (28.050000, -23.050000) {\textcolor[HTML]{b0b0b5}{購}};
\node[Square] at (28.050000, -23.550000) {};
\node[Onyomi] at (28.100000, -23.450000) {\hbox{\tate コウ}};
\node[Meaning] at (28.050000, -21.800000) {subscription};
\node[Kanji] at (30.100000, -23.050000) {\textcolor[HTML]{cd8268}{構}};
\node[Square] at (30.100000, -23.550000) {};
\node[Onyomi] at (30.150000, -23.450000) {\hbox{\tate コウ}};
\node[Kunyomi] at (30.050000, -23.450000) {\hbox{\tate かま.*}};
\node[Meaning] at (30.100000, -21.800000) {set up};
\node[Kanji] at (32.150000, -23.050000) {\textcolor[HTML]{91b7c3}{溝}};
\node[Square] at (32.150000, -23.550000) {};
\node[Onyomi] at (32.200000, -23.450000) {\hbox{\tate コウ}};
\node[Kunyomi] at (32.100000, -23.450000) {\hbox{\tate みぞ}};
\node[Meaning] at (32.150000, -21.800000) {gutter};
\node[Kanji] at (34.200000, -23.050000) {\textcolor[HTML]{b74029}{論}};
\node[Square] at (34.200000, -23.550000) {};
\node[Onyomi] at (34.250000, -23.450000) {\hbox{\tate ロン}};
\node[Meaning] at (34.200000, -21.800000) {theory};
\node[Kanji] at (36.250000, -23.050000) {\textcolor[HTML]{91b7c3}{倫}};
\node[Square] at (36.250000, -23.550000) {};
\node[Onyomi] at (36.300000, -23.450000) {\hbox{\tate リン}};
\node[Meaning] at (36.250000, -21.800000) {ethics};
\node[Kanji] at (38.300000, -23.050000) {\textcolor[HTML]{c8a59d}{輪}};
\node[Square] at (38.300000, -23.550000) {};
\node[Onyomi] at (38.350000, -23.450000) {\hbox{\tate リン}};
\node[Kunyomi] at (38.250000, -23.450000) {\hbox{\tate わ}};
\node[Meaning] at (38.300000, -21.800000) {wheel};
\node[Kanji] at (40.350000, -23.050000) {\textcolor[HTML]{91b7c3}{偏}};
\node[Square] at (40.350000, -23.550000) {};
\node[Onyomi] at (40.400000, -23.450000) {\hbox{\tate ヘン}};
\node[Kunyomi] at (40.300000, -23.450000) {\hbox{\tate かたよ}};
\node[Meaning] at (40.350000, -21.800000) {biased};
\node[Kanji] at (42.400000, -23.050000) {\textcolor[HTML]{68a4bc}{遍}};
\node[Square] at (42.400000, -23.550000) {};
\node[Onyomi] at (42.450000, -23.450000) {\hbox{\tate ヘン}};
\node[Kunyomi] at (42.350000, -23.450000) {\hbox{\tate あまね}};
\node[Meaning] at (42.400000, -21.800000) {universal};
\node[Kanji] at (44.450000, -23.050000) {\textcolor[HTML]{cd8268}{編}};
\node[Square] at (44.450000, -23.550000) {};
\node[Onyomi] at (44.500000, -23.450000) {\hbox{\tate ヘン}};
\node[Kunyomi] at (44.400000, -23.450000) {\hbox{\tate あ.む}};
\node[Meaning] at (44.450000, -21.800000) {knit};
\node[Kanji] at (46.500000, -23.050000) {\textcolor[HTML]{a3bac2}{冊}};
\node[Square] at (46.500000, -23.550000) {};
\node[Onyomi] at (46.550000, -23.450000) {\hbox{\tate サツ}};
\node[Meaning] at (46.500000, -21.800000) {books counter};
\node[Kanji] at (48.550000, -23.050000) {\textcolor[HTML]{d69f8d}{典}};
\node[Square] at (48.550000, -23.550000) {};
\node[Onyomi] at (48.600000, -23.450000) {\hbox{\tate テン}};
\node[Meaning] at (48.550000, -21.800000) {rule};
\node[Kanji] at (50.600000, -23.050000) {\textcolor[HTML]{cd8268}{氏}};
\node[Square] at (50.600000, -23.550000) {};
\node[Onyomi] at (50.650000, -23.450000) {\hbox{\tate シ}};
\node[Kunyomi] at (50.550000, -23.450000) {\hbox{\tate うじ}};
\node[Meaning] at (50.600000, -21.800000) {family name};
\node[Kanji] at (52.650000, -23.050000) {\textcolor[HTML]{d2a293}{紙}};
\node[Square] at (52.650000, -23.550000) {};
\node[Onyomi] at (52.700000, -23.450000) {\hbox{\tate シ}};
\node[Kunyomi] at (52.600000, -23.450000) {\hbox{\tate かみ}};
\node[Meaning] at (52.650000, -21.800000) {paper};
\node[Kanji] at (54.700000, -23.050000) {\textcolor[HTML]{c8a59d}{婚}};
\node[Square] at (54.700000, -23.550000) {};
\node[Onyomi] at (54.750000, -23.450000) {\hbox{\tate コン}};
\node[Meaning] at (54.700000, -21.800000) {marriage};
\node[Kanji] at (56.750000, -23.050000) {\textcolor[HTML]{d69f8d}{低}};
\node[Square] at (56.750000, -23.550000) {};
\node[Onyomi] at (56.800000, -23.450000) {\hbox{\tate テイ}};
\node[Kunyomi] at (56.700000, -23.450000) {\hbox{\tate ひく.い}};
\node[Meaning] at (56.750000, -21.800000) {low};
\node[Meaning] at (-58.050000, -22.950000) {91.77\%};
\node[Kanji] at (-56.000000, -25.100000) {\textcolor[HTML]{a3bac2}{抵}};
\node[Square] at (-56.000000, -25.600000) {};
\node[Onyomi] at (-55.950000, -25.500000) {\hbox{\tate テイ}};
\node[Meaning] at (-56.000000, -23.850000) {resist};
\node[Kanji] at (-53.950000, -25.100000) {\textcolor[HTML]{b0b0b5}{底}};
\node[Square] at (-53.950000, -25.600000) {};
\node[Onyomi] at (-53.900000, -25.500000) {\hbox{\tate テイ}};
\node[Kunyomi] at (-54.000000, -25.500000) {\hbox{\tate そこ}};
\node[Meaning] at (-53.950000, -23.850000) {bottom};
\node[Kanji] at (-51.900000, -25.100000) {\textcolor[HTML]{c36143}{民}};
\node[Square] at (-51.900000, -25.600000) {};
\node[Onyomi] at (-51.850000, -25.500000) {\hbox{\tate ミン}};
\node[Kunyomi] at (-51.950000, -25.500000) {\hbox{\tate たみ}};
\node[Meaning] at (-51.900000, -23.850000) {peoples};
\node[Kanji] at (-49.850000, -25.100000) {\textcolor[HTML]{91b7c3}{眠}};
\node[Square] at (-49.850000, -25.600000) {};
\node[Onyomi] at (-49.800000, -25.500000) {\hbox{\tate ミン}};
\node[Kunyomi] at (-49.900000, -25.500000) {\hbox{\tate ねむ.*}};
\node[Meaning] at (-49.850000, -23.850000) {sleep};
\node[Kanji] at (-47.800000, -25.100000) {\textcolor[HTML]{c8a59d}{捕}};
\node[Square] at (-47.800000, -25.600000) {};
\node[Onyomi] at (-47.750000, -25.500000) {\hbox{\tate ホ}};
\node[Kunyomi] at (-47.850000, -25.500000) {\hbox{\tate つか.まる}};
\node[Meaning] at (-47.800000, -23.850000) {catch};
\node[Kanji] at (-45.750000, -25.100000) {\textcolor[HTML]{c8a59d}{浦}};
\node[Square] at (-45.750000, -25.600000) {};
\node[Onyomi] at (-45.700000, -25.500000) {\hbox{\tate ホ}};
\node[Kunyomi] at (-45.800000, -25.500000) {\hbox{\tate うら}};
\node[Meaning] at (-45.750000, -23.850000) {bay};
\node[Kanji] at (-43.700000, -25.100000) {\textcolor[HTML]{b0b0b5}{舗}};
\node[Square] at (-43.700000, -25.600000) {};
\node[Onyomi] at (-43.650000, -25.500000) {\hbox{\tate ホ}};
\node[Meaning] at (-43.700000, -23.850000) {shop};
\node[Kanji] at (-41.650000, -25.100000) {\textcolor[HTML]{d2a293}{補}};
\node[Square] at (-41.650000, -25.600000) {};
\node[Onyomi] at (-41.600000, -25.500000) {\hbox{\tate ホ}};
\node[Kunyomi] at (-41.700000, -25.500000) {\hbox{\tate おぎな.う}};
\node[Meaning] at (-41.650000, -23.850000) {supplement};
\node[Kanji] at (-39.600000, -25.100000) {\textcolor[HTML]{a3bac2}{邸}};
\node[Square] at (-39.600000, -25.600000) {};
\node[Onyomi] at (-39.550000, -25.500000) {\hbox{\tate テイ}};
\node[Kunyomi] at (-39.650000, -25.500000) {\hbox{\tate やしき}};
\node[Meaning] at (-39.600000, -23.850000) {residence};
\node[Kanji] at (-37.550000, -25.100000) {\textcolor[HTML]{91b7c3}{郭}};
\node[Square] at (-37.550000, -25.600000) {};
\node[Onyomi] at (-37.500000, -25.500000) {\hbox{\tate カク}};
\node[Kunyomi] at (-37.600000, -25.500000) {\hbox{\tate くるわ}};
\node[Meaning] at (-37.550000, -23.850000) {enclosure};
\node[Kanji] at (-35.500000, -25.100000) {\textcolor[HTML]{d69f8d}{郡}};
\node[Square] at (-35.500000, -25.600000) {};
\node[Onyomi] at (-35.450000, -25.500000) {\hbox{\tate グン}};
\node[Kunyomi] at (-35.550000, -25.500000) {\hbox{\tate こおり}};
\node[Meaning] at (-35.500000, -23.850000) {county};
\node[Kanji] at (-33.450000, -25.100000) {\textcolor[HTML]{a3bac2}{郊}};
\node[Square] at (-33.450000, -25.600000) {};
\node[Onyomi] at (-33.400000, -25.500000) {\hbox{\tate コウ}};
\node[Meaning] at (-33.450000, -23.850000) {suburbs};
\node[Kanji] at (-31.400000, -25.100000) {\textcolor[HTML]{a11d25}{部}};
\node[Square] at (-31.400000, -25.600000) {};
\node[Onyomi] at (-31.350000, -25.500000) {\hbox{\tate ブ}};
\node[Kunyomi] at (-31.450000, -25.500000) {\hbox{\tate へ}};
\node[Meaning] at (-31.400000, -23.850000) {part};
\node[Kanji] at (-29.350000, -25.100000) {\textcolor[HTML]{c36143}{都}};
\node[Square] at (-29.350000, -25.600000) {};
\node[Onyomi] at (-29.300000, -25.500000) {\hbox{\tate ト・ツ}};
\node[Kunyomi] at (-29.400000, -25.500000) {\hbox{\tate みやこ}};
\node[Meaning] at (-29.350000, -23.850000) {metropolis};
\node[Kanji] at (-27.300000, -25.100000) {\textcolor[HTML]{b0b0b5}{郵}};
\node[Square] at (-27.300000, -25.600000) {};
\node[Onyomi] at (-27.250000, -25.500000) {\hbox{\tate ユウ}};
\node[Meaning] at (-27.300000, -23.850000) {mail};
\node[Kanji] at (-25.250000, -25.100000) {\textcolor[HTML]{c8a59d}{邦}};
\node[Square] at (-25.250000, -25.600000) {};
\node[Onyomi] at (-25.200000, -25.500000) {\hbox{\tate ホウ}};
\node[Kunyomi] at (-25.300000, -25.500000) {\hbox{\tate くに}};
\node[Meaning] at (-25.250000, -23.850000) {home country};
\node[Kanji] at (-23.200000, -25.100000) {\textcolor[HTML]{d2a293}{郷}};
\node[Square] at (-23.200000, -25.600000) {};
\node[Onyomi] at (-23.150000, -25.500000) {\hbox{\tate キョウ}};
\node[Meaning] at (-23.200000, -23.850000) {hometown};
\node[Kanji] at (-21.150000, -25.100000) {\textcolor[HTML]{d2a293}{響}};
\node[Square] at (-21.150000, -25.600000) {};
\node[Onyomi] at (-21.100000, -25.500000) {\hbox{\tate キョウ}};
\node[Kunyomi] at (-21.200000, -25.500000) {\hbox{\tate ひび.く}};
\node[Meaning] at (-21.150000, -23.850000) {echo};
\node[Kanji] at (-19.100000, -25.100000) {\textcolor[HTML]{d69f8d}{郎}};
\node[Square] at (-19.100000, -25.600000) {};
\node[Onyomi] at (-19.050000, -25.500000) {\hbox{\tate ロウ}};
\node[Meaning] at (-19.100000, -23.850000) {guy};
\node[Kanji] at (-17.050000, -25.100000) {\textcolor[HTML]{68a4bc}{廊}};
\node[Square] at (-17.050000, -25.600000) {};
\node[Onyomi] at (-17.000000, -25.500000) {\hbox{\tate ロウ}};
\node[Meaning] at (-17.050000, -23.850000) {corridor};
\node[Kanji] at (-15.000000, -25.100000) {\textcolor[HTML]{91b7c3}{盾}};
\node[Square] at (-15.000000, -25.600000) {};
\node[Onyomi] at (-14.950000, -25.500000) {\hbox{\tate ジュン}};
\node[Kunyomi] at (-15.050000, -25.500000) {\hbox{\tate たて}};
\node[Meaning] at (-15.000000, -23.850000) {shield};
\node[Kanji] at (-12.950000, -25.100000) {\textcolor[HTML]{91b7c3}{循}};
\node[Square] at (-12.950000, -25.600000) {};
\node[Onyomi] at (-12.900000, -25.500000) {\hbox{\tate ジュン}};
\node[Meaning] at (-12.950000, -23.850000) {circulation};
\node[Kanji] at (-10.900000, -25.100000) {\textcolor[HTML]{d69f8d}{派}};
\node[Square] at (-10.900000, -25.600000) {};
\node[Onyomi] at (-10.850000, -25.500000) {\hbox{\tate ハ}};
\node[Meaning] at (-10.900000, -23.850000) {sect};
\node[Kanji] at (-8.850000, -25.100000) {\textcolor[HTML]{b0b0b5}{脈}};
\node[Square] at (-8.850000, -25.600000) {};
\node[Onyomi] at (-8.800000, -25.500000) {\hbox{\tate ミャク}};
\node[Meaning] at (-8.850000, -23.850000) {vein};
\node[Kanji] at (-6.800000, -25.100000) {\textcolor[HTML]{d2a293}{衆}};
\node[Square] at (-6.800000, -25.600000) {};
\node[Onyomi] at (-6.750000, -25.500000) {\hbox{\tate シュウ}};
\node[Meaning] at (-6.800000, -23.850000) {populace};
\node[Kanji] at (-4.750000, -25.100000) {\textcolor[HTML]{1e76bb}{逓}};
\node[Square] at (-4.750000, -25.600000) {};
\node[Onyomi] at (-4.700000, -25.500000) {\hbox{\tate テイ}};
\node[Meaning] at (-4.750000, -23.850000) {relay};
\node[Kanji] at (-2.700000, -25.100000) {\textcolor[HTML]{d69f8d}{段}};
\node[Square] at (-2.700000, -25.600000) {};
\node[Onyomi] at (-2.650000, -25.500000) {\hbox{\tate ダン}};
\node[Meaning] at (-2.700000, -23.850000) {steps};
\node[Kanji] at (-0.650000, -25.100000) {\textcolor[HTML]{68a4bc}{鍛}};
\node[Square] at (-0.650000, -25.600000) {};
\node[Onyomi] at (-0.600000, -25.500000) {\hbox{\tate タン}};
\node[Kunyomi] at (-0.700000, -25.500000) {\hbox{\tate きた.える}};
\node[Meaning] at (-0.650000, -23.850000) {forge};
\node[Kanji] at (1.400000, -25.100000) {\textcolor[HTML]{91b7c3}{后}};
\node[Square] at (1.400000, -25.600000) {};
\node[Onyomi] at (1.450000, -25.500000) {\hbox{\tate コウ・ゴ}};
\node[Kunyomi] at (1.350000, -25.500000) {\hbox{\tate きさき}};
\node[Meaning] at (1.400000, -23.850000) {empress};
\node[Kanji] at (3.450000, -25.100000) {\textcolor[HTML]{91b7c3}{幻}};
\node[Square] at (3.450000, -25.600000) {};
\node[Onyomi] at (3.500000, -25.500000) {\hbox{\tate ゲン}};
\node[Kunyomi] at (3.400000, -25.500000) {\hbox{\tate まぼろし}};
\node[Meaning] at (3.450000, -23.850000) {illusion};
\node[Kanji] at (5.500000, -25.100000) {\textcolor[HTML]{d69f8d}{司}};
\node[Square] at (5.500000, -25.600000) {};
\node[Onyomi] at (5.550000, -25.500000) {\hbox{\tate シ}};
\node[Kunyomi] at (5.450000, -25.500000) {\hbox{\tate つかさど.る}};
\node[Meaning] at (5.500000, -23.850000) {director};
\node[Kanji] at (7.550000, -25.100000) {\textcolor[HTML]{408dba}{伺}};
\node[Square] at (7.550000, -25.600000) {};
\node[Onyomi] at (7.600000, -25.500000) {\hbox{\tate シ}};
\node[Kunyomi] at (7.500000, -25.500000) {\hbox{\tate うかが}};
\node[Meaning] at (7.550000, -23.850000) {pay respects};
\node[Kanji] at (9.600000, -25.100000) {\textcolor[HTML]{c8a59d}{詞}};
\node[Square] at (9.600000, -25.600000) {};
\node[Onyomi] at (9.650000, -25.500000) {\hbox{\tate シ}};
\node[Meaning] at (9.600000, -23.850000) {part of speech};
\node[Kanji] at (11.650000, -25.100000) {\textcolor[HTML]{a3bac2}{飼}};
\node[Square] at (11.650000, -25.600000) {};
\node[Onyomi] at (11.700000, -25.500000) {\hbox{\tate シ}};
\node[Kunyomi] at (11.600000, -25.500000) {\hbox{\tate か}};
\node[Meaning] at (11.650000, -23.850000) {domesticate};
\node[Kanji] at (13.700000, -25.100000) {\textcolor[HTML]{68a4bc}{嗣}};
\node[Square] at (13.700000, -25.600000) {};
\node[Onyomi] at (13.750000, -25.500000) {\hbox{\tate シ}};
\node[Meaning] at (13.700000, -23.850000) {heir};
\node[Kanji] at (15.750000, -25.100000) {\textcolor[HTML]{91b7c3}{舟}};
\node[Square] at (15.750000, -25.600000) {};
\node[Onyomi] at (15.800000, -25.500000) {\hbox{\tate シュウ}};
\node[Kunyomi] at (15.700000, -25.500000) {\hbox{\tate ふね・ふな}};
\node[Meaning] at (15.750000, -23.850000) {boat};
\node[Kanji] at (17.800000, -25.100000) {\textcolor[HTML]{91b7c3}{舶}};
\node[Square] at (17.800000, -25.600000) {};
\node[Onyomi] at (17.850000, -25.500000) {\hbox{\tate ハク}};
\node[Meaning] at (17.800000, -23.850000) {ship};
\node[Kanji] at (19.850000, -25.100000) {\textcolor[HTML]{d69f8d}{航}};
\node[Square] at (19.850000, -25.600000) {};
\node[Onyomi] at (19.900000, -25.500000) {\hbox{\tate コウ}};
\node[Meaning] at (19.850000, -23.850000) {navigation};
\node[Kanji] at (21.900000, -25.100000) {\textcolor[HTML]{d69f8d}{般}};
\node[Square] at (21.900000, -25.600000) {};
\node[Onyomi] at (21.950000, -25.500000) {\hbox{\tate ハン}};
\node[Meaning] at (21.900000, -23.850000) {generally};
\node[Kanji] at (23.950000, -25.100000) {\textcolor[HTML]{d2a293}{盤}};
\node[Square] at (23.950000, -25.600000) {};
\node[Onyomi] at (24.000000, -25.500000) {\hbox{\tate バン}};
\node[Kunyomi] at (23.900000, -25.500000) {\hbox{\tate ばん}};
\node[Meaning] at (23.950000, -23.850000) {tray};
\node[Kanji] at (26.000000, -25.100000) {\textcolor[HTML]{91b7c3}{搬}};
\node[Square] at (26.000000, -25.600000) {};
\node[Onyomi] at (26.050000, -25.500000) {\hbox{\tate ハン}};
\node[Meaning] at (26.000000, -23.850000) {transport};
\node[Kanji] at (28.050000, -25.100000) {\textcolor[HTML]{d69f8d}{船}};
\node[Square] at (28.050000, -25.600000) {};
\node[Onyomi] at (28.100000, -25.500000) {\hbox{\tate セン}};
\node[Kunyomi] at (28.000000, -25.500000) {\hbox{\tate ふね}};
\node[Meaning] at (28.050000, -23.850000) {boat};
\node[Kanji] at (30.100000, -25.100000) {\textcolor[HTML]{cd8268}{艦}};
\node[Square] at (30.100000, -25.600000) {};
\node[Onyomi] at (30.150000, -25.500000) {\hbox{\tate カン}};
\node[Meaning] at (30.100000, -23.850000) {warship};
\node[Kanji] at (32.150000, -25.100000) {\textcolor[HTML]{b0b0b5}{艇}};
\node[Square] at (32.150000, -25.600000) {};
\node[Onyomi] at (32.200000, -25.500000) {\hbox{\tate テイ}};
\node[Meaning] at (32.150000, -23.850000) {rowboat};
\node[Kanji] at (34.200000, -25.100000) {\textcolor[HTML]{1e76bb}{瓜}};
\node[Square] at (34.200000, -25.600000) {};
\node[Onyomi] at (34.250000, -25.500000) {\hbox{\tate カ・ケ}};
\node[Kunyomi] at (34.150000, -25.500000) {\hbox{\tate うり}};
\node[Meaning] at (34.200000, -23.850000) {melon};
\node[Kanji] at (36.250000, -25.100000) {\textcolor[HTML]{91b7c3}{弧}};
\node[Square] at (36.250000, -25.600000) {};
\node[Onyomi] at (36.300000, -25.500000) {\hbox{\tate コ}};
\node[Meaning] at (36.250000, -23.850000) {arc};
\node[Kanji] at (38.300000, -25.100000) {\textcolor[HTML]{91b7c3}{孤}};
\node[Square] at (38.300000, -25.600000) {};
\node[Onyomi] at (38.350000, -25.500000) {\hbox{\tate コ}};
\node[Meaning] at (38.300000, -23.850000) {orphan};
\node[Kanji] at (40.350000, -25.100000) {\textcolor[HTML]{29409e}{繭}};
\node[Square] at (40.350000, -25.600000) {};
\node[Onyomi] at (40.400000, -25.500000) {\hbox{\tate ケン}};
\node[Kunyomi] at (40.300000, -25.500000) {\hbox{\tate まゆ}};
\node[Meaning] at (40.350000, -23.850000) {cocoon};
\node[Kanji] at (42.400000, -25.100000) {\textcolor[HTML]{b0b0b5}{益}};
\node[Square] at (42.400000, -25.600000) {};
\node[Onyomi] at (42.450000, -25.500000) {\hbox{\tate エキ}};
\node[Meaning] at (42.400000, -23.850000) {benefit};
\node[Kanji] at (44.450000, -25.100000) {\textcolor[HTML]{408dba}{暇}};
\node[Square] at (44.450000, -25.600000) {};
\node[Onyomi] at (44.500000, -25.500000) {\hbox{\tate カ}};
\node[Kunyomi] at (44.400000, -25.500000) {\hbox{\tate ひま}};
\node[Meaning] at (44.450000, -23.850000) {spare time};
\node[Kanji] at (46.500000, -25.100000) {\textcolor[HTML]{c8a59d}{敷}};
\node[Square] at (46.500000, -25.600000) {};
\node[Onyomi] at (46.550000, -25.500000) {\hbox{\tate フ}};
\node[Kunyomi] at (46.450000, -25.500000) {\hbox{\tate しき・し}};
\node[Meaning] at (46.500000, -23.850000) {spread};
\node[Kanji] at (48.550000, -25.100000) {\textcolor[HTML]{c36143}{来}};
\node[Square] at (48.550000, -25.600000) {};
\node[Onyomi] at (48.600000, -25.500000) {\hbox{\tate ライ}};
\node[Kunyomi] at (48.500000, -25.500000) {\hbox{\tate く.る}};
\node[Meaning] at (48.550000, -23.850000) {come};
\node[Kanji] at (50.600000, -25.100000) {\textcolor[HTML]{c36143}{気}};
\node[Square] at (50.600000, -25.600000) {};
\node[Onyomi] at (50.650000, -25.500000) {\hbox{\tate キ・ケ}};
\node[Kunyomi] at (50.550000, -25.500000) {\hbox{\tate いき}};
\node[Meaning] at (50.600000, -23.850000) {energy};
\node[Kanji] at (52.650000, -25.100000) {\textcolor[HTML]{68a4bc}{汽}};
\node[Square] at (52.650000, -25.600000) {};
\node[Onyomi] at (52.700000, -25.500000) {\hbox{\tate キ}};
\node[Meaning] at (52.650000, -23.850000) {steam};
\node[Kanji] at (54.700000, -25.100000) {\textcolor[HTML]{d2a293}{飛}};
\node[Square] at (54.700000, -25.600000) {};
\node[Onyomi] at (54.750000, -25.500000) {\hbox{\tate ヒ}};
\node[Kunyomi] at (54.650000, -25.500000) {\hbox{\tate と}};
\node[Meaning] at (54.700000, -23.850000) {fly};
\node[Kanji] at (56.750000, -25.100000) {\textcolor[HTML]{b0b0b5}{沈}};
\node[Square] at (56.750000, -25.600000) {};
\node[Onyomi] at (56.800000, -25.500000) {\hbox{\tate チン}};
\node[Kunyomi] at (56.700000, -25.500000) {\hbox{\tate しず.*}};
\node[Meaning] at (56.750000, -23.850000) {sink};
\node[Meaning] at (-58.050000, -25.000000) {93.93\%};
\node[Kanji] at (-56.000000, -27.150000) {\textcolor[HTML]{c8a59d}{妻}};
\node[Square] at (-56.000000, -27.650000) {};
\node[Onyomi] at (-55.950000, -27.550000) {\hbox{\tate サイ}};
\node[Kunyomi] at (-56.050000, -27.550000) {\hbox{\tate つま}};
\node[Meaning] at (-56.000000, -25.900000) {wife};
\node[Kanji] at (-53.950000, -27.150000) {\textcolor[HTML]{91b7c3}{衰}};
\node[Square] at (-53.950000, -27.650000) {};
\node[Onyomi] at (-53.900000, -27.550000) {\hbox{\tate スイ}};
\node[Kunyomi] at (-54.000000, -27.550000) {\hbox{\tate おとろ.える}};
\node[Meaning] at (-53.950000, -25.900000) {decline};
\node[Kanji] at (-51.900000, -27.150000) {\textcolor[HTML]{1059be}{衷}};
\node[Square] at (-51.900000, -27.650000) {};
\node[Onyomi] at (-51.850000, -27.550000) {\hbox{\tate チュウ}};
\node[Meaning] at (-51.900000, -25.900000) {inmost};
\node[Kanji] at (-49.850000, -27.150000) {\textcolor[HTML]{c36143}{面}};
\node[Square] at (-49.850000, -27.650000) {};
\node[Onyomi] at (-49.800000, -27.550000) {\hbox{\tate メン}};
\node[Kunyomi] at (-49.900000, -27.550000) {\hbox{\tate おも・おもて}};
\node[Meaning] at (-49.850000, -25.900000) {surface};
\node[Kanji] at (-47.800000, -27.150000) {\textcolor[HTML]{c8a59d}{革}};
\node[Square] at (-47.800000, -27.650000) {};
\node[Onyomi] at (-47.750000, -27.550000) {\hbox{\tate カク}};
\node[Kunyomi] at (-47.850000, -27.550000) {\hbox{\tate かわ}};
\node[Meaning] at (-47.800000, -25.900000) {leather};
\node[Kanji] at (-45.750000, -27.150000) {\textcolor[HTML]{68a4bc}{靴}};
\node[Square] at (-45.750000, -27.650000) {};
\node[Onyomi] at (-45.700000, -27.550000) {\hbox{\tate カ}};
\node[Kunyomi] at (-45.800000, -27.550000) {\hbox{\tate くつ}};
\node[Meaning] at (-45.750000, -25.900000) {shoes};
\node[Kanji] at (-43.700000, -27.150000) {\textcolor[HTML]{a3bac2}{覇}};
\node[Square] at (-43.700000, -27.650000) {};
\node[Onyomi] at (-43.650000, -27.550000) {\hbox{\tate ハ・ハク}};
\node[Kunyomi] at (-43.750000, -27.550000) {\hbox{\tate はたがしら}};
\node[Meaning] at (-43.700000, -25.900000) {leadership};
\node[Kanji] at (-41.650000, -27.150000) {\textcolor[HTML]{d2a293}{声}};
\node[Square] at (-41.650000, -27.650000) {};
\node[Onyomi] at (-41.600000, -27.550000) {\hbox{\tate セイ}};
\node[Kunyomi] at (-41.700000, -27.550000) {\hbox{\tate こえ}};
\node[Meaning] at (-41.650000, -25.900000) {voice};
\node[Kanji] at (-39.600000, -27.150000) {\textcolor[HTML]{a3bac2}{呉}};
\node[Square] at (-39.600000, -27.650000) {};
\node[Onyomi] at (-39.550000, -27.550000) {\hbox{\tate ゴ}};
\node[Kunyomi] at (-39.650000, -27.550000) {\hbox{\tate くれ・く}};
\node[Meaning] at (-39.600000, -25.900000) {give};
\node[Kanji] at (-37.550000, -27.150000) {\textcolor[HTML]{408dba}{娯}};
\node[Square] at (-37.550000, -27.650000) {};
\node[Onyomi] at (-37.500000, -27.550000) {\hbox{\tate ゴ}};
\node[Meaning] at (-37.550000, -25.900000) {recreation};
\node[Kanji] at (-35.500000, -27.150000) {\textcolor[HTML]{b0b0b5}{誤}};
\node[Square] at (-35.500000, -27.650000) {};
\node[Onyomi] at (-35.450000, -27.550000) {\hbox{\tate ゴ}};
\node[Kunyomi] at (-35.550000, -27.550000) {\hbox{\tate あやま.*}};
\node[Meaning] at (-35.500000, -25.900000) {mistake};
\node[Kanji] at (-33.450000, -27.150000) {\textcolor[HTML]{a3bac2}{蒸}};
\node[Square] at (-33.450000, -27.650000) {};
\node[Onyomi] at (-33.400000, -27.550000) {\hbox{\tate ジョウ}};
\node[Kunyomi] at (-33.500000, -27.550000) {\hbox{\tate む.れる}};
\node[Meaning] at (-33.450000, -25.900000) {steam};
\node[Kanji] at (-31.400000, -27.150000) {\textcolor[HTML]{c8a59d}{承}};
\node[Square] at (-31.400000, -27.650000) {};
\node[Onyomi] at (-31.350000, -27.550000) {\hbox{\tate ショウ}};
\node[Kunyomi] at (-31.450000, -27.550000) {\hbox{\tate }};
\node[Meaning] at (-31.400000, -25.900000) {consent};
\node[Kanji] at (-29.350000, -27.150000) {\textcolor[HTML]{d2a293}{極}};
\node[Square] at (-29.350000, -27.650000) {};
\node[Onyomi] at (-29.300000, -27.550000) {\hbox{\tate キョク・ゴク}};
\node[Kunyomi] at (-29.400000, -27.550000) {\hbox{\tate きわ.める}};
\node[Meaning] at (-29.350000, -25.900000) {extreme};
\node[Kanji] at (-27.300000, -27.150000) {\textcolor[HTML]{408dba}{牙}};
\node[Square] at (-27.300000, -27.650000) {};
\node[Onyomi] at (-27.250000, -27.550000) {\hbox{\tate ゲ}};
\node[Kunyomi] at (-27.350000, -27.550000) {\hbox{\tate きば}};
\node[Meaning] at (-27.300000, -25.900000) {fang};
\node[Kanji] at (-25.250000, -27.150000) {\textcolor[HTML]{91b7c3}{芽}};
\node[Square] at (-25.250000, -27.650000) {};
\node[Onyomi] at (-25.200000, -27.550000) {\hbox{\tate ガ}};
\node[Kunyomi] at (-25.300000, -27.550000) {\hbox{\tate め}};
\node[Meaning] at (-25.250000, -25.900000) {sprout};
\node[Kanji] at (-23.200000, -27.150000) {\textcolor[HTML]{68a4bc}{邪}};
\node[Square] at (-23.200000, -27.650000) {};
\node[Onyomi] at (-23.150000, -27.550000) {\hbox{\tate ジャ}};
\node[Kunyomi] at (-23.250000, -27.550000) {\hbox{\tate よこし.ま}};
\node[Meaning] at (-23.200000, -25.900000) {wicked};
\node[Kanji] at (-21.150000, -27.150000) {\textcolor[HTML]{a3bac2}{雅}};
\node[Square] at (-21.150000, -27.650000) {};
\node[Onyomi] at (-21.100000, -27.550000) {\hbox{\tate ガ}};
\node[Kunyomi] at (-21.200000, -27.550000) {\hbox{\tate みや.び}};
\node[Meaning] at (-21.150000, -25.900000) {elegant};
\node[Kanji] at (-19.100000, -27.150000) {\textcolor[HTML]{c8a59d}{釈}};
\node[Square] at (-19.100000, -27.650000) {};
\node[Onyomi] at (-19.050000, -27.550000) {\hbox{\tate シャク・セキ}};
\node[Kunyomi] at (-19.150000, -27.550000) {\hbox{\tate す・とく}};
\node[Meaning] at (-19.100000, -25.900000) {explanation};
\node[Kanji] at (-17.050000, -27.150000) {\textcolor[HTML]{c36143}{番}};
\node[Square] at (-17.050000, -27.650000) {};
\node[Onyomi] at (-17.000000, -27.550000) {\hbox{\tate バン}};
\node[Meaning] at (-17.050000, -25.900000) {number (series)};
\node[Kanji] at (-15.000000, -27.150000) {\textcolor[HTML]{c8a59d}{審}};
\node[Square] at (-15.000000, -27.650000) {};
\node[Onyomi] at (-14.950000, -27.550000) {\hbox{\tate シン}};
\node[Meaning] at (-15.000000, -25.900000) {judge};
\node[Kanji] at (-12.950000, -27.150000) {\textcolor[HTML]{b0b0b5}{翻}};
\node[Square] at (-12.950000, -27.650000) {};
\node[Onyomi] at (-12.900000, -27.550000) {\hbox{\tate ホン}};
\node[Kunyomi] at (-13.000000, -27.550000) {\hbox{\tate ひるがえ.*}};
\node[Meaning] at (-12.950000, -25.900000) {flip};
\node[Kanji] at (-10.900000, -27.150000) {\textcolor[HTML]{d2a293}{藩}};
\node[Square] at (-10.900000, -27.650000) {};
\node[Onyomi] at (-10.850000, -27.550000) {\hbox{\tate ハン}};
\node[Meaning] at (-10.900000, -25.900000) {fiefdom};
\node[Kanji] at (-8.850000, -27.150000) {\textcolor[HTML]{c8a59d}{毛}};
\node[Square] at (-8.850000, -27.650000) {};
\node[Onyomi] at (-8.800000, -27.550000) {\hbox{\tate モウ}};
\node[Kunyomi] at (-8.900000, -27.550000) {\hbox{\tate け}};
\node[Meaning] at (-8.850000, -25.900000) {fur};
\node[Kanji] at (-6.800000, -27.150000) {\textcolor[HTML]{408dba}{耗}};
\node[Square] at (-6.800000, -27.650000) {};
\node[Onyomi] at (-6.750000, -27.550000) {\hbox{\tate モウ}};
\node[Meaning] at (-6.800000, -25.900000) {decrease};
\node[Kanji] at (-4.750000, -27.150000) {\textcolor[HTML]{d2a293}{尾}};
\node[Square] at (-4.750000, -27.650000) {};
\node[Onyomi] at (-4.700000, -27.550000) {\hbox{\tate ビ}};
\node[Kunyomi] at (-4.800000, -27.550000) {\hbox{\tate お・ぽ}};
\node[Meaning] at (-4.750000, -25.900000) {tail};
\node[Kanji] at (-2.700000, -27.150000) {\textcolor[HTML]{d2a293}{宅}};
\node[Square] at (-2.700000, -27.650000) {};
\node[Onyomi] at (-2.650000, -27.550000) {\hbox{\tate タク}};
\node[Meaning] at (-2.700000, -25.900000) {house};
\node[Kanji] at (-0.650000, -27.150000) {\textcolor[HTML]{b0b0b5}{託}};
\node[Square] at (-0.650000, -27.650000) {};
\node[Onyomi] at (-0.600000, -27.550000) {\hbox{\tate タク}};
\node[Kunyomi] at (-0.700000, -27.550000) {\hbox{\tate かこ.*}};
\node[Meaning] at (-0.650000, -25.900000) {consign};
\node[Kanji] at (1.400000, -27.150000) {\textcolor[HTML]{d2a293}{為}};
\node[Square] at (1.400000, -27.650000) {};
\node[Onyomi] at (1.450000, -27.550000) {\hbox{\tate イ}};
\node[Kunyomi] at (1.350000, -27.550000) {\hbox{\tate ため・な.す}};
\node[Meaning] at (1.400000, -25.900000) {sake};
\node[Kanji] at (3.450000, -27.150000) {\textcolor[HTML]{a3bac2}{偽}};
\node[Square] at (3.450000, -27.650000) {};
\node[Onyomi] at (3.500000, -27.550000) {\hbox{\tate ギ}};
\node[Kunyomi] at (3.400000, -27.550000) {\hbox{\tate にせ}};
\node[Meaning] at (3.450000, -25.900000) {fake};
\node[Kanji] at (5.500000, -27.150000) {\textcolor[HTML]{b74029}{長}};
\node[Square] at (5.500000, -27.650000) {};
\node[Onyomi] at (5.550000, -27.550000) {\hbox{\tate チョウ}};
\node[Kunyomi] at (5.450000, -27.550000) {\hbox{\tate なが.い}};
\node[Meaning] at (5.500000, -25.900000) {long};
\node[Kanji] at (7.550000, -27.150000) {\textcolor[HTML]{d69f8d}{張}};
\node[Square] at (7.550000, -27.650000) {};
\node[Onyomi] at (7.600000, -27.550000) {\hbox{\tate チョウ}};
\node[Kunyomi] at (7.500000, -27.550000) {\hbox{\tate は.る}};
\node[Meaning] at (7.550000, -25.900000) {stretch};
\node[Kanji] at (9.600000, -27.150000) {\textcolor[HTML]{a3bac2}{帳}};
\node[Square] at (9.600000, -27.650000) {};
\node[Onyomi] at (9.650000, -27.550000) {\hbox{\tate チョウ}};
\node[Kunyomi] at (9.550000, -27.550000) {\hbox{\tate とばり}};
\node[Meaning] at (9.600000, -25.900000) {notebook};
\node[Kanji] at (11.650000, -27.150000) {\textcolor[HTML]{a3bac2}{髪}};
\node[Square] at (11.650000, -27.650000) {};
\node[Onyomi] at (11.700000, -27.550000) {\hbox{\tate ハツ}};
\node[Kunyomi] at (11.600000, -27.550000) {\hbox{\tate かみ}};
\node[Meaning] at (11.650000, -25.900000) {hair};
\node[Kanji] at (13.700000, -27.150000) {\textcolor[HTML]{d69f8d}{展}};
\node[Square] at (13.700000, -27.650000) {};
\node[Onyomi] at (13.750000, -27.550000) {\hbox{\tate テン}};
\node[Kunyomi] at (13.650000, -27.550000) {\hbox{\tate のぶ・のび}};
\node[Meaning] at (13.700000, -25.900000) {expand};
\node[Kanji] at (15.750000, -27.150000) {\textcolor[HTML]{68a4bc}{喪}};
\node[Square] at (15.750000, -27.650000) {};
\node[Onyomi] at (15.800000, -27.550000) {\hbox{\tate ソウ}};
\node[Kunyomi] at (15.700000, -27.550000) {\hbox{\tate も}};
\node[Meaning] at (15.750000, -25.900000) {mourning};
\node[Kanji] at (17.800000, -27.150000) {\textcolor[HTML]{a3bac2}{巣}};
\node[Square] at (17.800000, -27.650000) {};
\node[Onyomi] at (17.850000, -27.550000) {\hbox{\tate ソウ}};
\node[Kunyomi] at (17.750000, -27.550000) {\hbox{\tate す}};
\node[Meaning] at (17.800000, -25.900000) {nest};
\node[Kanji] at (19.850000, -27.150000) {\textcolor[HTML]{d69f8d}{単}};
\node[Square] at (19.850000, -27.650000) {};
\node[Onyomi] at (19.900000, -27.550000) {\hbox{\tate タン}};
\node[Meaning] at (19.850000, -25.900000) {simple};
\node[Kanji] at (21.900000, -27.150000) {\textcolor[HTML]{b74029}{戦}};
\node[Square] at (21.900000, -27.650000) {};
\node[Onyomi] at (21.950000, -27.550000) {\hbox{\tate セン}};
\node[Kunyomi] at (21.850000, -27.550000) {\hbox{\tate たたか.*}};
\node[Meaning] at (21.900000, -25.900000) {war};
\node[Kanji] at (23.950000, -27.150000) {\textcolor[HTML]{91b7c3}{禅}};
\node[Square] at (23.950000, -27.650000) {};
\node[Onyomi] at (24.000000, -27.550000) {\hbox{\tate ゼン}};
\node[Meaning] at (23.950000, -25.900000) {zen};
\node[Kanji] at (26.000000, -27.150000) {\textcolor[HTML]{d2a293}{弾}};
\node[Square] at (26.000000, -27.650000) {};
\node[Onyomi] at (26.050000, -27.550000) {\hbox{\tate ダン}};
\node[Kunyomi] at (25.950000, -27.550000) {\hbox{\tate ひ.く}};
\node[Meaning] at (26.000000, -25.900000) {bullet};
\node[Kanji] at (28.050000, -27.150000) {\textcolor[HTML]{b0b0b5}{桜}};
\node[Square] at (28.050000, -27.650000) {};
\node[Kunyomi] at (28.000000, -27.550000) {\hbox{\tate さくら}};
\node[Meaning] at (28.050000, -25.900000) {sakura};
\node[Kanji] at (30.100000, -27.150000) {\textcolor[HTML]{a3bac2}{獣}};
\node[Square] at (30.100000, -27.650000) {};
\node[Onyomi] at (30.150000, -27.550000) {\hbox{\tate ジュウ}};
\node[Kunyomi] at (30.050000, -27.550000) {\hbox{\tate けもの}};
\node[Meaning] at (30.100000, -25.900000) {beast};
\node[Kanji] at (32.150000, -27.150000) {\textcolor[HTML]{b0b0b5}{脳}};
\node[Square] at (32.150000, -27.650000) {};
\node[Onyomi] at (32.200000, -27.550000) {\hbox{\tate ノウ}};
\node[Meaning] at (32.150000, -25.900000) {brain};
\node[Kanji] at (34.200000, -27.150000) {\textcolor[HTML]{91b7c3}{悩}};
\node[Square] at (34.200000, -27.650000) {};
\node[Onyomi] at (34.250000, -27.550000) {\hbox{\tate ノウ}};
\node[Kunyomi] at (34.150000, -27.550000) {\hbox{\tate なや}};
\node[Meaning] at (34.200000, -25.900000) {worry};
\node[Kanji] at (36.250000, -27.150000) {\textcolor[HTML]{b0b0b5}{厳}};
\node[Square] at (36.250000, -27.650000) {};
\node[Onyomi] at (36.300000, -27.550000) {\hbox{\tate ゲン・ゴン}};
\node[Kunyomi] at (36.200000, -27.550000) {\hbox{\tate きび.しい}};
\node[Meaning] at (36.250000, -25.900000) {strict};
\node[Kanji] at (38.300000, -27.150000) {\textcolor[HTML]{b0b0b5}{鎖}};
\node[Square] at (38.300000, -27.650000) {};
\node[Onyomi] at (38.350000, -27.550000) {\hbox{\tate サ}};
\node[Kunyomi] at (38.250000, -27.550000) {\hbox{\tate くさり・とざ}};
\node[Meaning] at (38.300000, -25.900000) {chain};
\node[Kanji] at (40.350000, -27.150000) {\textcolor[HTML]{d69f8d}{挙}};
\node[Square] at (40.350000, -27.650000) {};
\node[Onyomi] at (40.400000, -27.550000) {\hbox{\tate キョ}};
\node[Kunyomi] at (40.300000, -27.550000) {\hbox{\tate あ.がる}};
\node[Meaning] at (40.350000, -25.900000) {raise};
\node[Kanji] at (42.400000, -27.150000) {\textcolor[HTML]{a3bac2}{誉}};
\node[Square] at (42.400000, -27.650000) {};
\node[Onyomi] at (42.450000, -27.550000) {\hbox{\tate ヨ}};
\node[Kunyomi] at (42.350000, -27.550000) {\hbox{\tate ほ.める}};
\node[Meaning] at (42.400000, -25.900000) {honor};
\node[Kanji] at (44.450000, -27.150000) {\textcolor[HTML]{91b7c3}{猟}};
\node[Square] at (44.450000, -27.650000) {};
\node[Onyomi] at (44.500000, -27.550000) {\hbox{\tate リョウ・レフ}};
\node[Kunyomi] at (44.400000, -27.550000) {\hbox{\tate かり・か}};
\node[Meaning] at (44.450000, -25.900000) {hunting};
\node[Kanji] at (46.500000, -27.150000) {\textcolor[HTML]{d2a293}{鳥}};
\node[Square] at (46.500000, -27.650000) {};
\node[Onyomi] at (46.550000, -27.550000) {\hbox{\tate チョウ}};
\node[Kunyomi] at (46.450000, -27.550000) {\hbox{\tate とり}};
\node[Meaning] at (46.500000, -25.900000) {bird};
\node[Kanji] at (48.550000, -27.150000) {\textcolor[HTML]{a3bac2}{鳴}};
\node[Square] at (48.550000, -27.650000) {};
\node[Onyomi] at (48.600000, -27.550000) {\hbox{\tate メイ}};
\node[Kunyomi] at (48.500000, -27.550000) {\hbox{\tate な}};
\node[Meaning] at (48.550000, -25.900000) {chirp};
\node[Kanji] at (50.600000, -27.150000) {\textcolor[HTML]{b0b0b5}{鶴}};
\node[Square] at (50.600000, -27.650000) {};
\node[Onyomi] at (50.650000, -27.550000) {\hbox{\tate カク}};
\node[Kunyomi] at (50.550000, -27.550000) {\hbox{\tate つる}};
\node[Meaning] at (50.600000, -25.900000) {crane};
\node[Kanji] at (52.650000, -27.150000) {\textcolor[HTML]{68a4bc}{烏}};
\node[Square] at (52.650000, -27.650000) {};
\node[Onyomi] at (52.700000, -27.550000) {\hbox{\tate ウオ}};
\node[Kunyomi] at (52.600000, -27.550000) {\hbox{\tate からす}};
\node[Meaning] at (52.650000, -25.900000) {crow};
\node[Kanji] at (54.700000, -27.150000) {\textcolor[HTML]{408dba}{鳩}};
\node[Square] at (54.700000, -27.650000) {};
\node[Onyomi] at (54.750000, -27.550000) {\hbox{\tate ク}};
\node[Kunyomi] at (54.650000, -27.550000) {\hbox{\tate はと}};
\node[Meaning] at (54.700000, -25.900000) {dove};
\node[Kanji] at (56.750000, -27.150000) {\textcolor[HTML]{68a4bc}{鶏}};
\node[Square] at (56.750000, -27.650000) {};
\node[Onyomi] at (56.800000, -27.550000) {\hbox{\tate ケイ}};
\node[Kunyomi] at (56.700000, -27.550000) {\hbox{\tate とり}};
\node[Meaning] at (56.750000, -25.900000) {chicken};
\node[Meaning] at (-58.050000, -27.050000) {95.80\%};
\node[Kanji] at (-56.000000, -29.200000) {\textcolor[HTML]{c36143}{島}};
\node[Square] at (-56.000000, -29.700000) {};
\node[Onyomi] at (-55.950000, -29.600000) {\hbox{\tate トウ}};
\node[Kunyomi] at (-56.050000, -29.600000) {\hbox{\tate しま}};
\node[Meaning] at (-56.000000, -27.950000) {island};
\node[Kanji] at (-53.950000, -29.200000) {\textcolor[HTML]{91b7c3}{暖}};
\node[Square] at (-53.950000, -29.700000) {};
\node[Onyomi] at (-53.900000, -29.600000) {\hbox{\tate ダン}};
\node[Kunyomi] at (-54.000000, -29.600000) {\hbox{\tate あたた.かい}};
\node[Meaning] at (-53.950000, -27.950000) {warm};
\node[Kanji] at (-51.900000, -29.200000) {\textcolor[HTML]{91b7c3}{媛}};
\node[Square] at (-51.900000, -29.700000) {};
\node[Onyomi] at (-51.850000, -29.600000) {\hbox{\tate エン}};
\node[Kunyomi] at (-51.950000, -29.600000) {\hbox{\tate ひめ}};
\node[Meaning] at (-51.900000, -27.950000) {princess};
\node[Kanji] at (-49.850000, -29.200000) {\textcolor[HTML]{d2a293}{援}};
\node[Square] at (-49.850000, -29.700000) {};
\node[Onyomi] at (-49.800000, -29.600000) {\hbox{\tate エン}};
\node[Meaning] at (-49.850000, -27.950000) {aid};
\node[Kanji] at (-47.800000, -29.200000) {\textcolor[HTML]{a3bac2}{緩}};
\node[Square] at (-47.800000, -29.700000) {};
\node[Onyomi] at (-47.750000, -29.600000) {\hbox{\tate カン}};
\node[Kunyomi] at (-47.850000, -29.600000) {\hbox{\tate ゆる}};
\node[Meaning] at (-47.800000, -27.950000) {loose};
\node[Kanji] at (-45.750000, -29.200000) {\textcolor[HTML]{cd8268}{属}};
\node[Square] at (-45.750000, -29.700000) {};
\node[Onyomi] at (-45.700000, -29.600000) {\hbox{\tate ゾク}};
\node[Meaning] at (-45.750000, -27.950000) {belong};
\node[Kanji] at (-43.700000, -29.200000) {\textcolor[HTML]{408dba}{嘱}};
\node[Square] at (-43.700000, -29.700000) {};
\node[Onyomi] at (-43.650000, -29.600000) {\hbox{\tate ショク}};
\node[Kunyomi] at (-43.750000, -29.600000) {\hbox{\tate しょく.する}};
\node[Meaning] at (-43.700000, -27.950000) {request};
\node[Kanji] at (-41.650000, -29.200000) {\textcolor[HTML]{91b7c3}{偶}};
\node[Square] at (-41.650000, -29.700000) {};
\node[Onyomi] at (-41.600000, -29.600000) {\hbox{\tate グウ}};
\node[Kunyomi] at (-41.700000, -29.600000) {\hbox{\tate たま}};
\node[Meaning] at (-41.650000, -27.950000) {accidentally};
\node[Kanji] at (-39.600000, -29.200000) {\textcolor[HTML]{a3bac2}{遇}};
\node[Square] at (-39.600000, -29.700000) {};
\node[Onyomi] at (-39.550000, -29.600000) {\hbox{\tate グウ}};
\node[Kunyomi] at (-39.650000, -29.600000) {\hbox{\tate あ}};
\node[Meaning] at (-39.600000, -27.950000) {treatment};
\node[Kanji] at (-37.550000, -29.200000) {\textcolor[HTML]{408dba}{愚}};
\node[Square] at (-37.550000, -29.700000) {};
\node[Onyomi] at (-37.500000, -29.600000) {\hbox{\tate グ}};
\node[Kunyomi] at (-37.600000, -29.600000) {\hbox{\tate おろ}};
\node[Meaning] at (-37.550000, -27.950000) {foolish};
\node[Kanji] at (-35.500000, -29.200000) {\textcolor[HTML]{68a4bc}{隅}};
\node[Square] at (-35.500000, -29.700000) {};
\node[Onyomi] at (-35.450000, -29.600000) {\hbox{\tate グウ}};
\node[Kunyomi] at (-35.550000, -29.600000) {\hbox{\tate すみ}};
\node[Meaning] at (-35.500000, -27.950000) {corner};
\node[Kanji] at (-33.450000, -29.200000) {\textcolor[HTML]{c8a59d}{逆}};
\node[Square] at (-33.450000, -29.700000) {};
\node[Onyomi] at (-33.400000, -29.600000) {\hbox{\tate ギャク}};
\node[Kunyomi] at (-33.500000, -29.600000) {\hbox{\tate さか.らう}};
\node[Meaning] at (-33.450000, -27.950000) {reverse};
\node[Kanji] at (-31.400000, -29.200000) {\textcolor[HTML]{29409e}{塑}};
\node[Square] at (-31.400000, -29.700000) {};
\node[Onyomi] at (-31.350000, -29.600000) {\hbox{\tate ソ}};
\node[Meaning] at (-31.400000, -27.950000) {model};
\node[Kanji] at (-29.350000, -29.200000) {\textcolor[HTML]{d69f8d}{岡}};
\node[Square] at (-29.350000, -29.700000) {};
\node[Kunyomi] at (-29.400000, -29.600000) {\hbox{\tate おか}};
\node[Meaning] at (-29.350000, -27.950000) {hill};
\node[Kanji] at (-27.300000, -29.200000) {\textcolor[HTML]{a3bac2}{鋼}};
\node[Square] at (-27.300000, -29.700000) {};
\node[Onyomi] at (-27.250000, -29.600000) {\hbox{\tate コウ}};
\node[Kunyomi] at (-27.350000, -29.600000) {\hbox{\tate はがね}};
\node[Meaning] at (-27.300000, -27.950000) {steel};
\node[Kanji] at (-25.250000, -29.200000) {\textcolor[HTML]{b0b0b5}{綱}};
\node[Square] at (-25.250000, -29.700000) {};
\node[Onyomi] at (-25.200000, -29.600000) {\hbox{\tate コウ}};
\node[Kunyomi] at (-25.300000, -29.600000) {\hbox{\tate つな}};
\node[Meaning] at (-25.250000, -27.950000) {cable};
\node[Kanji] at (-23.200000, -29.200000) {\textcolor[HTML]{a3bac2}{剛}};
\node[Square] at (-23.200000, -29.700000) {};
\node[Onyomi] at (-23.150000, -29.600000) {\hbox{\tate ゴウ}};
\node[Meaning] at (-23.200000, -27.950000) {sturdy};
\node[Kanji] at (-21.150000, -29.200000) {\textcolor[HTML]{68a4bc}{缶}};
\node[Square] at (-21.150000, -29.700000) {};
\node[Onyomi] at (-21.100000, -29.600000) {\hbox{\tate カン}};
\node[Meaning] at (-21.150000, -27.950000) {tin can};
\node[Kanji] at (-19.100000, -29.200000) {\textcolor[HTML]{91b7c3}{陶}};
\node[Square] at (-19.100000, -29.700000) {};
\node[Onyomi] at (-19.050000, -29.600000) {\hbox{\tate トウ}};
\node[Meaning] at (-19.100000, -27.950000) {pottery};
\node[Kanji] at (-17.050000, -29.200000) {\textcolor[HTML]{91b7c3}{揺}};
\node[Square] at (-17.050000, -29.700000) {};
\node[Onyomi] at (-17.000000, -29.600000) {\hbox{\tate ヨウ}};
\node[Kunyomi] at (-17.100000, -29.600000) {\hbox{\tate ゆ.*}};
\node[Meaning] at (-17.050000, -27.950000) {shake};
\node[Kanji] at (-15.000000, -29.200000) {\textcolor[HTML]{91b7c3}{謡}};
\node[Square] at (-15.000000, -29.700000) {};
\node[Onyomi] at (-14.950000, -29.600000) {\hbox{\tate ヨウ}};
\node[Kunyomi] at (-15.050000, -29.600000) {\hbox{\tate うた}};
\node[Meaning] at (-15.000000, -27.950000) {noh chanting};
\node[Kanji] at (-12.950000, -29.200000) {\textcolor[HTML]{d2a293}{就}};
\node[Square] at (-12.950000, -29.700000) {};
\node[Onyomi] at (-12.900000, -29.600000) {\hbox{\tate シュウ}};
\node[Kunyomi] at (-13.000000, -29.600000) {\hbox{\tate つ.く}};
\node[Meaning] at (-12.950000, -27.950000) {settle in};
\node[Kanji] at (-10.900000, -29.200000) {\textcolor[HTML]{408dba}{懇}};
\node[Square] at (-10.900000, -29.700000) {};
\node[Onyomi] at (-10.850000, -29.600000) {\hbox{\tate コン}};
\node[Meaning] at (-10.900000, -27.950000) {courteous};
\node[Kanji] at (-8.850000, -29.200000) {\textcolor[HTML]{1e76bb}{墾}};
\node[Square] at (-8.850000, -29.700000) {};
\node[Onyomi] at (-8.800000, -29.600000) {\hbox{\tate コン}};
\node[Meaning] at (-8.850000, -27.950000) {break ground};
\node[Kanji] at (-6.800000, -29.200000) {\textcolor[HTML]{b0b0b5}{免}};
\node[Square] at (-6.800000, -29.700000) {};
\node[Onyomi] at (-6.750000, -29.600000) {\hbox{\tate メン}};
\node[Kunyomi] at (-6.850000, -29.600000) {\hbox{\tate まぬか.れる}};
\node[Meaning] at (-6.800000, -27.950000) {excuse};
\node[Kanji] at (-4.750000, -29.200000) {\textcolor[HTML]{91b7c3}{逸}};
\node[Square] at (-4.750000, -29.700000) {};
\node[Onyomi] at (-4.700000, -29.600000) {\hbox{\tate イツ}};
\node[Kunyomi] at (-4.800000, -29.600000) {\hbox{\tate そ・はぐ}};
\node[Meaning] at (-4.750000, -27.950000) {deviate};
\node[Kanji] at (-2.700000, -29.200000) {\textcolor[HTML]{a3bac2}{晩}};
\node[Square] at (-2.700000, -29.700000) {};
\node[Onyomi] at (-2.650000, -29.600000) {\hbox{\tate バン}};
\node[Meaning] at (-2.700000, -27.950000) {night};
\node[Kanji] at (-0.650000, -29.200000) {\textcolor[HTML]{91b7c3}{勉}};
\node[Square] at (-0.650000, -29.700000) {};
\node[Onyomi] at (-0.600000, -29.600000) {\hbox{\tate ベン}};
\node[Meaning] at (-0.650000, -27.950000) {exertion};
\node[Kanji] at (1.400000, -29.200000) {\textcolor[HTML]{d69f8d}{象}};
\node[Square] at (1.400000, -29.700000) {};
\node[Onyomi] at (1.450000, -29.600000) {\hbox{\tate ショウ・ゾウ}};
\node[Meaning] at (1.400000, -27.950000) {elephant};
\node[Kanji] at (3.450000, -29.200000) {\textcolor[HTML]{d69f8d}{像}};
\node[Square] at (3.450000, -29.700000) {};
\node[Onyomi] at (3.500000, -29.600000) {\hbox{\tate ゾウ}};
\node[Meaning] at (3.450000, -27.950000) {statue};
\node[Kanji] at (5.500000, -29.200000) {\textcolor[HTML]{cd8268}{馬}};
\node[Square] at (5.500000, -29.700000) {};
\node[Onyomi] at (5.550000, -29.600000) {\hbox{\tate バ}};
\node[Kunyomi] at (5.450000, -29.600000) {\hbox{\tate うま}};
\node[Meaning] at (5.500000, -27.950000) {horse};
\node[Kanji] at (7.550000, -29.200000) {\textcolor[HTML]{a3bac2}{駒}};
\node[Square] at (7.550000, -29.700000) {};
\node[Onyomi] at (7.600000, -29.600000) {\hbox{\tate ク}};
\node[Kunyomi] at (7.500000, -29.600000) {\hbox{\tate こま}};
\node[Meaning] at (7.550000, -27.950000) {chess piece};
\node[Kanji] at (9.600000, -29.200000) {\textcolor[HTML]{d69f8d}{験}};
\node[Square] at (9.600000, -29.700000) {};
\node[Onyomi] at (9.650000, -29.600000) {\hbox{\tate ケン}};
\node[Kunyomi] at (9.550000, -29.600000) {\hbox{\tate ため・ためし}};
\node[Meaning] at (9.600000, -27.950000) {test};
\node[Kanji] at (11.650000, -29.200000) {\textcolor[HTML]{b0b0b5}{騎}};
\node[Square] at (11.650000, -29.700000) {};
\node[Onyomi] at (11.700000, -29.600000) {\hbox{\tate キ}};
\node[Meaning] at (11.650000, -27.950000) {horse};
\node[Kanji] at (13.700000, -29.200000) {\textcolor[HTML]{b0b0b5}{駐}};
\node[Square] at (13.700000, -29.700000) {};
\node[Onyomi] at (13.750000, -29.600000) {\hbox{\tate チュウ}};
\node[Meaning] at (13.700000, -27.950000) {resident};
\node[Kanji] at (15.750000, -29.200000) {\textcolor[HTML]{c8a59d}{駆}};
\node[Square] at (15.750000, -29.700000) {};
\node[Onyomi] at (15.800000, -29.600000) {\hbox{\tate ク}};
\node[Kunyomi] at (15.700000, -29.600000) {\hbox{\tate か}};
\node[Meaning] at (15.750000, -27.950000) {gallop};
\node[Kanji] at (17.800000, -29.200000) {\textcolor[HTML]{c36143}{駅}};
\node[Square] at (17.800000, -29.700000) {};
\node[Onyomi] at (17.850000, -29.600000) {\hbox{\tate エキ}};
\node[Meaning] at (17.800000, -27.950000) {station};
\node[Kanji] at (19.850000, -29.200000) {\textcolor[HTML]{a3bac2}{騒}};
\node[Square] at (19.850000, -29.700000) {};
\node[Onyomi] at (19.900000, -29.600000) {\hbox{\tate ソウ}};
\node[Kunyomi] at (19.800000, -29.600000) {\hbox{\tate さわ.ぐ}};
\node[Meaning] at (19.850000, -27.950000) {boisterous};
\node[Kanji] at (21.900000, -29.200000) {\textcolor[HTML]{68a4bc}{駄}};
\node[Square] at (21.900000, -29.700000) {};
\node[Onyomi] at (21.950000, -29.600000) {\hbox{\tate ダ・タ}};
\node[Meaning] at (21.900000, -27.950000) {burdensome};
\node[Kanji] at (23.950000, -29.200000) {\textcolor[HTML]{91b7c3}{驚}};
\node[Square] at (23.950000, -29.700000) {};
\node[Onyomi] at (24.000000, -29.600000) {\hbox{\tate キョウ}};
\node[Kunyomi] at (23.900000, -29.600000) {\hbox{\tate おどろ.*}};
\node[Meaning] at (23.950000, -27.950000) {surprised};
\node[Kanji] at (26.000000, -29.200000) {\textcolor[HTML]{68a4bc}{篤}};
\node[Square] at (26.000000, -29.700000) {};
\node[Onyomi] at (26.050000, -29.600000) {\hbox{\tate トク}};
\node[Kunyomi] at (25.950000, -29.600000) {\hbox{\tate あつ}};
\node[Meaning] at (26.000000, -27.950000) {deliberate};
\node[Kanji] at (28.050000, -29.200000) {\textcolor[HTML]{408dba}{騰}};
\node[Square] at (28.050000, -29.700000) {};
\node[Onyomi] at (28.100000, -29.600000) {\hbox{\tate トウ}};
\node[Kunyomi] at (28.000000, -29.600000) {\hbox{\tate あが・のぼ}};
\node[Meaning] at (28.050000, -27.950000) {inflation};
\node[Kanji] at (30.100000, -29.200000) {\textcolor[HTML]{a3bac2}{虎}};
\node[Square] at (30.100000, -29.700000) {};
\node[Onyomi] at (30.150000, -29.600000) {\hbox{\tate コ}};
\node[Kunyomi] at (30.050000, -29.600000) {\hbox{\tate とら}};
\node[Meaning] at (30.100000, -27.950000) {tiger};
\node[Kanji] at (32.150000, -29.200000) {\textcolor[HTML]{91b7c3}{虜}};
\node[Square] at (32.150000, -29.700000) {};
\node[Onyomi] at (32.200000, -29.600000) {\hbox{\tate リョ・ロ}};
\node[Kunyomi] at (32.100000, -29.600000) {\hbox{\tate とりく}};
\node[Meaning] at (32.150000, -27.950000) {captive};
\node[Kanji] at (34.200000, -29.200000) {\textcolor[HTML]{68a4bc}{膚}};
\node[Square] at (34.200000, -29.700000) {};
\node[Onyomi] at (34.250000, -29.600000) {\hbox{\tate フ}};
\node[Kunyomi] at (34.150000, -29.600000) {\hbox{\tate はだ}};
\node[Meaning] at (34.200000, -27.950000) {skin};
\node[Kanji] at (36.250000, -29.200000) {\textcolor[HTML]{91b7c3}{虚}};
\node[Square] at (36.250000, -29.700000) {};
\node[Onyomi] at (36.300000, -29.600000) {\hbox{\tate キョ・コ}};
\node[Kunyomi] at (36.200000, -29.600000) {\hbox{\tate むな.しい}};
\node[Meaning] at (36.250000, -27.950000) {void};
\node[Kanji] at (38.300000, -29.200000) {\textcolor[HTML]{68a4bc}{戯}};
\node[Square] at (38.300000, -29.700000) {};
\node[Onyomi] at (38.350000, -29.600000) {\hbox{\tate ギ・ゲ}};
\node[Kunyomi] at (38.250000, -29.600000) {\hbox{\tate ざ・じゃ}};
\node[Meaning] at (38.300000, -27.950000) {play};
\node[Kanji] at (40.350000, -29.200000) {\textcolor[HTML]{408dba}{虞}};
\node[Square] at (40.350000, -29.700000) {};
\node[Kunyomi] at (40.300000, -29.600000) {\hbox{\tate おそれ}};
\node[Meaning] at (40.350000, -27.950000) {uneasiness};
\node[Kanji] at (42.400000, -29.200000) {\textcolor[HTML]{b0b0b5}{慮}};
\node[Square] at (42.400000, -29.700000) {};
\node[Onyomi] at (42.450000, -29.600000) {\hbox{\tate リョ}};
\node[Kunyomi] at (42.350000, -29.600000) {\hbox{\tate おもんぱく}};
\node[Meaning] at (42.400000, -27.950000) {consider};
\node[Kanji] at (44.450000, -29.200000) {\textcolor[HTML]{d2a293}{劇}};
\node[Square] at (44.450000, -29.700000) {};
\node[Onyomi] at (44.500000, -29.600000) {\hbox{\tate ゲキ}};
\node[Meaning] at (44.450000, -27.950000) {drama};
\node[Kanji] at (46.500000, -29.200000) {\textcolor[HTML]{91b7c3}{虐}};
\node[Square] at (46.500000, -29.700000) {};
\node[Onyomi] at (46.550000, -29.600000) {\hbox{\tate ギャク}};
\node[Kunyomi] at (46.450000, -29.600000) {\hbox{\tate しいた}};
\node[Meaning] at (46.500000, -27.950000) {oppress};
\node[Kanji] at (48.550000, -29.200000) {\textcolor[HTML]{c8a59d}{鹿}};
\node[Square] at (48.550000, -29.700000) {};
\node[Onyomi] at (48.600000, -29.600000) {\hbox{\tate ロク}};
\node[Kunyomi] at (48.500000, -29.600000) {\hbox{\tate か・しか}};
\node[Meaning] at (48.550000, -27.950000) {deer};
\node[Kanji] at (50.600000, -29.200000) {\textcolor[HTML]{a3bac2}{薦}};
\node[Square] at (50.600000, -29.700000) {};
\node[Onyomi] at (50.650000, -29.600000) {\hbox{\tate セン}};
\node[Kunyomi] at (50.550000, -29.600000) {\hbox{\tate すす.*}};
\node[Meaning] at (50.600000, -27.950000) {recommend};
\node[Kanji] at (52.650000, -29.200000) {\textcolor[HTML]{c8a59d}{慶}};
\node[Square] at (52.650000, -29.700000) {};
\node[Onyomi] at (52.700000, -29.600000) {\hbox{\tate ケイ}};
\node[Kunyomi] at (52.600000, -29.600000) {\hbox{\tate よろこ}};
\node[Meaning] at (52.650000, -27.950000) {congratulate};
\node[Kanji] at (54.700000, -29.200000) {\textcolor[HTML]{a3bac2}{麗}};
\node[Square] at (54.700000, -29.700000) {};
\node[Onyomi] at (54.750000, -29.600000) {\hbox{\tate レイ}};
\node[Kunyomi] at (54.650000, -29.600000) {\hbox{\tate うるわ.しい}};
\node[Meaning] at (54.700000, -27.950000) {lovely};
\node[Kanji] at (56.750000, -29.200000) {\textcolor[HTML]{c8a59d}{熊}};
\node[Square] at (56.750000, -29.700000) {};
\node[Kunyomi] at (56.700000, -29.600000) {\hbox{\tate くま}};
\node[Meaning] at (56.750000, -27.950000) {bear};
\node[Meaning] at (-58.050000, -29.100000) {97.01\%};
\node[Kanji] at (-56.000000, -31.250000) {\textcolor[HTML]{c36143}{能}};
\node[Square] at (-56.000000, -31.750000) {};
\node[Onyomi] at (-55.950000, -31.650000) {\hbox{\tate ノウ}};
\node[Meaning] at (-56.000000, -30.000000) {ability};
\node[Kanji] at (-53.950000, -31.250000) {\textcolor[HTML]{d69f8d}{態}};
\node[Square] at (-53.950000, -31.750000) {};
\node[Onyomi] at (-53.900000, -31.650000) {\hbox{\tate タイ}};
\node[Kunyomi] at (-54.000000, -31.650000) {\hbox{\tate わざ}};
\node[Meaning] at (-53.950000, -30.000000) {appearance};
\node[Kanji] at (-51.900000, -31.250000) {\textcolor[HTML]{cd8268}{演}};
\node[Square] at (-51.900000, -31.750000) {};
\node[Onyomi] at (-51.850000, -31.650000) {\hbox{\tate エン}};
\node[Meaning] at (-51.900000, -30.000000) {perform};
\node[Kanji] at (-49.850000, -31.250000) {\textcolor[HTML]{408dba}{辱}};
\node[Square] at (-49.850000, -31.750000) {};
\node[Onyomi] at (-49.800000, -31.650000) {\hbox{\tate ジョク}};
\node[Kunyomi] at (-49.900000, -31.650000) {\hbox{\tate }};
\node[Meaning] at (-49.850000, -30.000000) {humiliate};
\node[Kanji] at (-47.800000, -31.250000) {\textcolor[HTML]{c8a59d}{震}};
\node[Square] at (-47.800000, -31.750000) {};
\node[Onyomi] at (-47.750000, -31.650000) {\hbox{\tate シン}};
\node[Kunyomi] at (-47.850000, -31.650000) {\hbox{\tate ふる.える}};
\node[Meaning] at (-47.800000, -30.000000) {earthquake};
\node[Kanji] at (-45.750000, -31.250000) {\textcolor[HTML]{d2a293}{振}};
\node[Square] at (-45.750000, -31.750000) {};
\node[Onyomi] at (-45.700000, -31.650000) {\hbox{\tate シン}};
\node[Kunyomi] at (-45.800000, -31.650000) {\hbox{\tate ふ.る}};
\node[Meaning] at (-45.750000, -30.000000) {shake};
\node[Kanji] at (-43.700000, -31.250000) {\textcolor[HTML]{68a4bc}{娠}};
\node[Square] at (-43.700000, -31.750000) {};
\node[Onyomi] at (-43.650000, -31.650000) {\hbox{\tate シン}};
\node[Meaning] at (-43.700000, -30.000000) {pregnant};
\node[Kanji] at (-41.650000, -31.250000) {\textcolor[HTML]{408dba}{唇}};
\node[Square] at (-41.650000, -31.750000) {};
\node[Onyomi] at (-41.600000, -31.650000) {\hbox{\tate シン}};
\node[Kunyomi] at (-41.700000, -31.650000) {\hbox{\tate くちびる}};
\node[Meaning] at (-41.650000, -30.000000) {lips};
\node[Kanji] at (-39.600000, -31.250000) {\textcolor[HTML]{d2a293}{農}};
\node[Square] at (-39.600000, -31.750000) {};
\node[Onyomi] at (-39.550000, -31.650000) {\hbox{\tate ノウ}};
\node[Meaning] at (-39.600000, -30.000000) {farming};
\node[Kanji] at (-37.550000, -31.250000) {\textcolor[HTML]{c8a59d}{濃}};
\node[Square] at (-37.550000, -31.750000) {};
\node[Onyomi] at (-37.500000, -31.650000) {\hbox{\tate ノウ}};
\node[Kunyomi] at (-37.600000, -31.650000) {\hbox{\tate こ.い}};
\node[Meaning] at (-37.550000, -30.000000) {thick};
\node[Kanji] at (-35.500000, -31.250000) {\textcolor[HTML]{c36143}{送}};
\node[Square] at (-35.500000, -31.750000) {};
\node[Onyomi] at (-35.450000, -31.650000) {\hbox{\tate ソウ}};
\node[Kunyomi] at (-35.550000, -31.650000) {\hbox{\tate おく.る}};
\node[Meaning] at (-35.500000, -30.000000) {send};
\node[Kanji] at (-33.450000, -31.250000) {\textcolor[HTML]{b74029}{関}};
\node[Square] at (-33.450000, -31.750000) {};
\node[Onyomi] at (-33.400000, -31.650000) {\hbox{\tate カン}};
\node[Kunyomi] at (-33.500000, -31.650000) {\hbox{\tate かか.わる}};
\node[Meaning] at (-33.450000, -30.000000) {related};
\node[Kanji] at (-31.400000, -31.250000) {\textcolor[HTML]{91b7c3}{咲}};
\node[Square] at (-31.400000, -31.750000) {};
\node[Onyomi] at (-31.350000, -31.650000) {\hbox{\tate ショウ}};
\node[Kunyomi] at (-31.450000, -31.650000) {\hbox{\tate さ}};
\node[Meaning] at (-31.400000, -30.000000) {blossom};
\node[Kanji] at (-29.350000, -31.250000) {\textcolor[HTML]{b0b0b5}{鬼}};
\node[Square] at (-29.350000, -31.750000) {};
\node[Onyomi] at (-29.300000, -31.650000) {\hbox{\tate キ}};
\node[Kunyomi] at (-29.400000, -31.650000) {\hbox{\tate おに}};
\node[Meaning] at (-29.350000, -30.000000) {demon};
\node[Kanji] at (-27.300000, -31.250000) {\textcolor[HTML]{1059be}{醜}};
\node[Square] at (-27.300000, -31.750000) {};
\node[Onyomi] at (-27.250000, -31.650000) {\hbox{\tate シュウ}};
\node[Kunyomi] at (-27.350000, -31.650000) {\hbox{\tate しこ・みにく}};
\node[Meaning] at (-27.300000, -30.000000) {ugly};
\node[Kanji] at (-25.250000, -31.250000) {\textcolor[HTML]{91b7c3}{魂}};
\node[Square] at (-25.250000, -31.750000) {};
\node[Onyomi] at (-25.200000, -31.650000) {\hbox{\tate コン}};
\node[Kunyomi] at (-25.300000, -31.650000) {\hbox{\tate たましい}};
\node[Meaning] at (-25.250000, -30.000000) {soul};
\node[Kanji] at (-23.200000, -31.250000) {\textcolor[HTML]{c8a59d}{魔}};
\node[Square] at (-23.200000, -31.750000) {};
\node[Onyomi] at (-23.150000, -31.650000) {\hbox{\tate マ}};
\node[Meaning] at (-23.200000, -30.000000) {devil};
\node[Kanji] at (-21.150000, -31.250000) {\textcolor[HTML]{91b7c3}{魅}};
\node[Square] at (-21.150000, -31.750000) {};
\node[Onyomi] at (-21.100000, -31.650000) {\hbox{\tate ミ}};
\node[Meaning] at (-21.150000, -30.000000) {alluring};
\node[Kanji] at (-19.100000, -31.250000) {\textcolor[HTML]{68a4bc}{塊}};
\node[Square] at (-19.100000, -31.750000) {};
\node[Onyomi] at (-19.050000, -31.650000) {\hbox{\tate カイ}};
\node[Kunyomi] at (-19.150000, -31.650000) {\hbox{\tate かたまり}};
\node[Meaning] at (-19.100000, -30.000000) {lump};
\node[Kanji] at (-17.050000, -31.250000) {\textcolor[HTML]{c8a59d}{襲}};
\node[Square] at (-17.050000, -31.750000) {};
\node[Onyomi] at (-17.000000, -31.650000) {\hbox{\tate シュウ}};
\node[Kunyomi] at (-17.100000, -31.650000) {\hbox{\tate おそ.う}};
\node[Meaning] at (-17.050000, -30.000000) {attack};
\node[Kanji] at (-15.000000, -31.250000) {\textcolor[HTML]{29409e}{嚇}};
\node[Square] at (-15.000000, -31.750000) {};
\node[Onyomi] at (-14.950000, -31.650000) {\hbox{\tate カク}};
\node[Meaning] at (-15.000000, -30.000000) {menacing};
\node[Kanji] at (-12.950000, -31.250000) {\textcolor[HTML]{29409e}{朕}};
\node[Square] at (-12.950000, -31.750000) {};
\node[Onyomi] at (-12.900000, -31.650000) {\hbox{\tate チン}};
\node[Meaning] at (-12.950000, -30.000000) {majestic plural};
\node[Kanji] at (-10.900000, -31.250000) {\textcolor[HTML]{68a4bc}{雰}};
\node[Square] at (-10.900000, -31.750000) {};
\node[Onyomi] at (-10.850000, -31.650000) {\hbox{\tate フン}};
\node[Meaning] at (-10.900000, -30.000000) {atmosphere};
\node[Kanji] at (-8.850000, -31.250000) {\textcolor[HTML]{a3bac2}{箇}};
\node[Square] at (-8.850000, -31.750000) {};
\node[Onyomi] at (-8.800000, -31.650000) {\hbox{\tate カ}};
\node[Meaning] at (-8.850000, -30.000000) {counters};
\node[Kanji] at (-6.800000, -31.250000) {\textcolor[HTML]{68a4bc}{錬}};
\node[Square] at (-6.800000, -31.750000) {};
\node[Onyomi] at (-6.750000, -31.650000) {\hbox{\tate レン}};
\node[Kunyomi] at (-6.850000, -31.650000) {\hbox{\tate ね}};
\node[Meaning] at (-6.800000, -30.000000) {tempering};
\node[Kanji] at (-4.750000, -31.250000) {\textcolor[HTML]{1e76bb}{遵}};
\node[Square] at (-4.750000, -31.750000) {};
\node[Onyomi] at (-4.700000, -31.650000) {\hbox{\tate ジュン}};
\node[Meaning] at (-4.750000, -30.000000) {abide by};
\node[Kanji] at (-2.700000, -31.250000) {\textcolor[HTML]{1e76bb}{罷}};
\node[Square] at (-2.700000, -31.750000) {};
\node[Onyomi] at (-2.650000, -31.650000) {\hbox{\tate ヒ}};
\node[Kunyomi] at (-2.750000, -31.650000) {\hbox{\tate や}};
\node[Meaning] at (-2.700000, -30.000000) {quit};
\node[Kanji] at (-0.650000, -31.250000) {\textcolor[HTML]{68a4bc}{屯}};
\node[Square] at (-0.650000, -31.750000) {};
\node[Onyomi] at (-0.600000, -31.650000) {\hbox{\tate トン}};
\node[Meaning] at (-0.650000, -30.000000) {barracks};
\node[Kanji] at (1.400000, -31.250000) {\textcolor[HTML]{1e76bb}{且}};
\node[Square] at (1.400000, -31.750000) {};
\node[Onyomi] at (1.450000, -31.650000) {\hbox{\tate ショ・ショウ}};
\node[Kunyomi] at (1.350000, -31.650000) {\hbox{\tate か}};
\node[Meaning] at (1.400000, -30.000000) {also};
\node[Kanji] at (3.450000, -31.250000) {\textcolor[HTML]{68a4bc}{藻}};
\node[Square] at (3.450000, -31.750000) {};
\node[Onyomi] at (3.500000, -31.650000) {\hbox{\tate ソウ}};
\node[Kunyomi] at (3.400000, -31.650000) {\hbox{\tate も}};
\node[Meaning] at (3.450000, -30.000000) {seaweed};
\node[Kanji] at (5.500000, -31.250000) {\textcolor[HTML]{91b7c3}{隷}};
\node[Square] at (5.500000, -31.750000) {};
\node[Onyomi] at (5.550000, -31.650000) {\hbox{\tate レイ}};
\node[Meaning] at (5.500000, -30.000000) {slave};
\node[Kanji] at (7.550000, -31.250000) {\textcolor[HTML]{68a4bc}{癒}};
\node[Square] at (7.550000, -31.750000) {};
\node[Onyomi] at (7.600000, -31.650000) {\hbox{\tate ユ}};
\node[Kunyomi] at (7.500000, -31.650000) {\hbox{\tate い・いや}};
\node[Meaning] at (7.550000, -30.000000) {healing};
\node[Kanji] at (9.600000, -31.250000) {\textcolor[HTML]{b0b0b5}{丹}};
\node[Square] at (9.600000, -31.750000) {};
\node[Onyomi] at (9.650000, -31.650000) {\hbox{\tate タン}};
\node[Kunyomi] at (9.550000, -31.650000) {\hbox{\tate に}};
\node[Meaning] at (9.600000, -30.000000) {rust colored};
\node[Kanji] at (11.650000, -31.250000) {\textcolor[HTML]{b0b0b5}{潟}};
\node[Square] at (11.650000, -31.750000) {};
\node[Onyomi] at (11.700000, -31.650000) {\hbox{\tate セキ}};
\node[Kunyomi] at (11.600000, -31.650000) {\hbox{\tate かた}};
\node[Meaning] at (11.650000, -30.000000) {lagoon};
\node[Kanji] at (13.700000, -31.250000) {\textcolor[HTML]{91b7c3}{柴}};
\node[Square] at (13.700000, -31.750000) {};
\node[Onyomi] at (13.750000, -31.650000) {\hbox{\tate サイ・シ}};
\node[Kunyomi] at (13.650000, -31.650000) {\hbox{\tate しば}};
\node[Meaning] at (13.700000, -30.000000) {brushwood};
\node[Kanji] at (15.750000, -31.250000) {\textcolor[HTML]{1e76bb}{璃}};
\node[Square] at (15.750000, -31.750000) {};
\node[Onyomi] at (15.800000, -31.650000) {\hbox{\tate リ}};
\node[Meaning] at (15.750000, -30.000000) {glassy};
\node[Kanji] at (17.800000, -31.250000) {\textcolor[HTML]{68a4bc}{俺}};
\node[Square] at (17.800000, -31.750000) {};
\node[Kunyomi] at (17.750000, -31.650000) {\hbox{\tate おれ}};
\node[Meaning] at (17.800000, -30.000000) {i};
\node[Kanji] at (19.850000, -31.250000) {\textcolor[HTML]{91b7c3}{駿}};
\node[Square] at (19.850000, -31.750000) {};
\node[Onyomi] at (19.900000, -31.650000) {\hbox{\tate シュン・スン}};
\node[Kunyomi] at (19.800000, -31.650000) {\hbox{\tate すぐ}};
\node[Meaning] at (19.850000, -30.000000) {speed};
\node[Kanji] at (21.900000, -31.250000) {\textcolor[HTML]{68a4bc}{臼}};
\node[Square] at (21.900000, -31.750000) {};
\node[Onyomi] at (21.950000, -31.650000) {\hbox{\tate キュウ}};
\node[Kunyomi] at (21.850000, -31.650000) {\hbox{\tate うす}};
\node[Meaning] at (21.900000, -30.000000) {mortar};
\node[Kanji] at (23.950000, -31.250000) {\textcolor[HTML]{1e76bb}{毀}};
\node[Square] at (23.950000, -31.750000) {};
\node[Onyomi] at (24.000000, -31.650000) {\hbox{\tate キ}};
\node[Meaning] at (23.950000, -30.000000) {destroy};
\node[Kanji] at (26.000000, -31.250000) {\textcolor[HTML]{68a4bc}{脊}};
\node[Square] at (26.000000, -31.750000) {};
\node[Onyomi] at (26.050000, -31.650000) {\hbox{\tate セキ}};
\node[Kunyomi] at (25.950000, -31.650000) {\hbox{\tate せせい}};
\node[Meaning] at (26.000000, -30.000000) {stature};
\node[Kanji] at (28.050000, -31.250000) {\textcolor[HTML]{1059be}{璽}};
\node[Square] at (28.050000, -31.750000) {};
\node[Onyomi] at (28.100000, -31.650000) {\hbox{\tate ジ}};
\node[Meaning] at (28.050000, -30.000000) {emperor's seal};
\node[Kanji] at (30.100000, -31.250000) {\textcolor[HTML]{91b7c3}{妖}};
\node[Square] at (30.100000, -31.750000) {};
\node[Onyomi] at (30.150000, -31.650000) {\hbox{\tate ヨウ}};
\node[Kunyomi] at (30.050000, -31.650000) {\hbox{\tate あや.しい}};
\node[Meaning] at (30.100000, -30.000000) {bewitching};
\node[Kanji] at (32.150000, -31.250000) {\textcolor[HTML]{1059be}{沃}};
\node[Square] at (32.150000, -31.750000) {};
\node[Onyomi] at (32.200000, -31.650000) {\hbox{\tate ヨク}};
\node[Meaning] at (32.150000, -30.000000) {fertility};
\node[Kanji] at (34.200000, -31.250000) {\textcolor[HTML]{68a4bc}{稽}};
\node[Square] at (34.200000, -31.750000) {};
\node[Onyomi] at (34.250000, -31.650000) {\hbox{\tate ケイ}};
\node[Meaning] at (34.200000, -30.000000) {consider};
\node[Kanji] at (36.250000, -31.250000) {\textcolor[HTML]{1e76bb}{采}};
\node[Square] at (36.250000, -31.750000) {};
\node[Onyomi] at (36.300000, -31.650000) {\hbox{\tate サイ}};
\node[Meaning] at (36.250000, -30.000000) {form};
\node[Kanji] at (38.300000, -31.250000) {\textcolor[HTML]{91b7c3}{斬}};
\node[Square] at (38.300000, -31.750000) {};
\node[Onyomi] at (38.350000, -31.650000) {\hbox{\tate ザン}};
\node[Kunyomi] at (38.250000, -31.650000) {\hbox{\tate き.る}};
\node[Meaning] at (38.300000, -30.000000) {slice};
\node[Kanji] at (40.350000, -31.250000) {\textcolor[HTML]{a3bac2}{也}};
\node[Square] at (40.350000, -31.750000) {};
\node[Kunyomi] at (40.300000, -31.650000) {\hbox{\tate なり}};
\node[Meaning] at (40.350000, -30.000000) {considerably};
\node[Kanji] at (42.400000, -31.250000) {\textcolor[HTML]{1e76bb}{巾}};
\node[Square] at (42.400000, -31.750000) {};
\node[Onyomi] at (42.450000, -31.650000) {\hbox{\tate キン}};
\node[Meaning] at (42.400000, -30.000000) {towel};
\node[Kanji] at (44.450000, -31.250000) {\textcolor[HTML]{68a4bc}{僅}};
\node[Square] at (44.450000, -31.750000) {};
\node[Onyomi] at (44.500000, -31.650000) {\hbox{\tate キン}};
\node[Kunyomi] at (44.400000, -31.650000) {\hbox{\tate わず.か}};
\node[Meaning] at (44.450000, -30.000000) {a wee bit};
\node[Kanji] at (46.500000, -31.250000) {\textcolor[HTML]{408dba}{侶}};
\node[Square] at (46.500000, -31.750000) {};
\node[Onyomi] at (46.550000, -31.650000) {\hbox{\tate リョ}};
\node[Meaning] at (46.500000, -30.000000) {companion};
\node[Kanji] at (48.550000, -31.250000) {\textcolor[HTML]{68a4bc}{伎}};
\node[Square] at (48.550000, -31.750000) {};
\node[Onyomi] at (48.600000, -31.650000) {\hbox{\tate キ}};
\node[Kunyomi] at (48.500000, -31.650000) {\hbox{\tate わざ}};
\node[Meaning] at (48.550000, -30.000000) {deed};
\node[Kanji] at (50.600000, -31.250000) {\textcolor[HTML]{1e76bb}{凄}};
\node[Square] at (50.600000, -31.750000) {};
\node[Onyomi] at (50.650000, -31.650000) {\hbox{\tate セイ}};
\node[Meaning] at (50.600000, -30.000000) {uncanny};
\node[Kanji] at (52.650000, -31.250000) {\textcolor[HTML]{408dba}{凌}};
\node[Square] at (52.650000, -31.750000) {};
\node[Onyomi] at (52.700000, -31.650000) {\hbox{\tate リョウ}};
\node[Kunyomi] at (52.600000, -31.650000) {\hbox{\tate しの}};
\node[Meaning] at (52.650000, -30.000000) {endure};
\node[Kanji] at (54.700000, -31.250000) {\textcolor[HTML]{408dba}{冶}};
\node[Square] at (54.700000, -31.750000) {};
\node[Onyomi] at (54.750000, -31.650000) {\hbox{\tate ヤ}};
\node[Meaning] at (54.700000, -30.000000) {melting};
\node[Kanji] at (56.750000, -31.250000) {\textcolor[HTML]{29409e}{凛}};
\node[Square] at (56.750000, -31.750000) {};
\node[Onyomi] at (56.800000, -31.650000) {\hbox{\tate リン}};
\node[Kunyomi] at (56.700000, -31.650000) {\hbox{\tate きびし}};
\node[Meaning] at (56.750000, -30.000000) {cold};
\node[Meaning] at (-58.050000, -31.150000) {98.05\%};
\node[Kanji] at (-56.000000, -33.300000) {\textcolor[HTML]{1059be}{刹}};
\node[Square] at (-56.000000, -33.800000) {};
\node[Onyomi] at (-55.950000, -33.700000) {\hbox{\tate サツ・セツ}};
\node[Meaning] at (-56.000000, -32.050000) {temple};
\node[Kanji] at (-53.950000, -33.300000) {\textcolor[HTML]{68a4bc}{剥}};
\node[Square] at (-53.950000, -33.800000) {};
\node[Onyomi] at (-53.900000, -33.700000) {\hbox{\tate ハク}};
\node[Kunyomi] at (-54.000000, -33.700000) {\hbox{\tate は.がす}};
\node[Meaning] at (-53.950000, -32.050000) {peel};
\node[Kanji] at (-51.900000, -33.300000) {\textcolor[HTML]{1e76bb}{匂}};
\node[Square] at (-51.900000, -33.800000) {};
\node[Kunyomi] at (-51.950000, -33.700000) {\hbox{\tate にお.う}};
\node[Meaning] at (-51.900000, -32.050000) {scent};
\node[Kanji] at (-49.850000, -33.300000) {\textcolor[HTML]{68a4bc}{勾}};
\node[Square] at (-49.850000, -33.800000) {};
\node[Onyomi] at (-49.800000, -33.700000) {\hbox{\tate コウ}};
\node[Meaning] at (-49.850000, -32.050000) {capture};
\node[Kanji] at (-47.800000, -33.300000) {\textcolor[HTML]{29409e}{嘲}};
\node[Square] at (-47.800000, -33.800000) {};
\node[Onyomi] at (-47.750000, -33.700000) {\hbox{\tate チョウ}};
\node[Kunyomi] at (-47.850000, -33.700000) {\hbox{\tate あざけ.る}};
\node[Meaning] at (-47.800000, -32.050000) {ridicule};
\node[Kanji] at (-45.750000, -33.300000) {\textcolor[HTML]{1059be}{咽}};
\node[Square] at (-45.750000, -33.800000) {};
\node[Onyomi] at (-45.700000, -33.700000) {\hbox{\tate イン}};
\node[Meaning] at (-45.750000, -32.050000) {throat};
\node[Kanji] at (-43.700000, -33.300000) {\textcolor[HTML]{408dba}{喉}};
\node[Square] at (-43.700000, -33.800000) {};
\node[Onyomi] at (-43.650000, -33.700000) {\hbox{\tate コウ}};
\node[Kunyomi] at (-43.750000, -33.700000) {\hbox{\tate のど}};
\node[Meaning] at (-43.700000, -32.050000) {throat};
\node[Kanji] at (-41.650000, -33.300000) {\textcolor[HTML]{29409e}{唾}};
\node[Square] at (-41.650000, -33.800000) {};
\node[Onyomi] at (-41.600000, -33.700000) {\hbox{\tate ダ}};
\node[Kunyomi] at (-41.700000, -33.700000) {\hbox{\tate つば}};
\node[Meaning] at (-41.650000, -32.050000) {saliva};
\node[Kanji] at (-39.600000, -33.300000) {\textcolor[HTML]{68a4bc}{呪}};
\node[Square] at (-39.600000, -33.800000) {};
\node[Kunyomi] at (-39.650000, -33.700000) {\hbox{\tate のろ}};
\node[Meaning] at (-39.600000, -32.050000) {curse};
\node[Kanji] at (-37.550000, -33.300000) {\textcolor[HTML]{1e76bb}{噌}};
\node[Square] at (-37.550000, -33.800000) {};
\node[Onyomi] at (-37.500000, -33.700000) {\hbox{\tate ソ}};
\node[Meaning] at (-37.550000, -32.050000) {boisterous};
\node[Kanji] at (-35.500000, -33.300000) {\textcolor[HTML]{68a4bc}{唄}};
\node[Square] at (-35.500000, -33.800000) {};
\node[Onyomi] at (-35.450000, -33.700000) {\hbox{\tate バイ}};
\node[Kunyomi] at (-35.550000, -33.700000) {\hbox{\tate うた}};
\node[Meaning] at (-35.500000, -32.050000) {shamisen song};
\node[Kanji] at (-33.450000, -33.300000) {\textcolor[HTML]{1e76bb}{叱}};
\node[Square] at (-33.450000, -33.800000) {};
\node[Kunyomi] at (-33.500000, -33.700000) {\hbox{\tate しか}};
\node[Meaning] at (-33.450000, -32.050000) {scold};
\node[Kanji] at (-31.400000, -33.300000) {\textcolor[HTML]{1059be}{呆}};
\node[Square] at (-31.400000, -33.800000) {};
\node[Onyomi] at (-31.350000, -33.700000) {\hbox{\tate ホウ}};
\node[Kunyomi] at (-31.450000, -33.700000) {\hbox{\tate あき・おろか}};
\node[Meaning] at (-31.400000, -32.050000) {shock};
\node[Kanji] at (-29.350000, -33.300000) {\textcolor[HTML]{68a4bc}{堆}};
\node[Square] at (-29.350000, -33.800000) {};
\node[Onyomi] at (-29.300000, -33.700000) {\hbox{\tate タイ}};
\node[Meaning] at (-29.350000, -32.050000) {piled high};
\node[Kanji] at (-27.300000, -33.300000) {\textcolor[HTML]{68a4bc}{填}};
\node[Square] at (-27.300000, -33.800000) {};
\node[Onyomi] at (-27.250000, -33.700000) {\hbox{\tate テン}};
\node[Meaning] at (-27.300000, -32.050000) {fill in};
\node[Kanji] at (-25.250000, -33.300000) {\textcolor[HTML]{408dba}{堰}};
\node[Square] at (-25.250000, -33.800000) {};
\node[Onyomi] at (-25.200000, -33.700000) {\hbox{\tate セキ}};
\node[Meaning] at (-25.250000, -32.050000) {dam};
\node[Kanji] at (-23.200000, -33.300000) {\textcolor[HTML]{1e76bb}{妬}};
\node[Square] at (-23.200000, -33.800000) {};
\node[Onyomi] at (-23.150000, -33.700000) {\hbox{\tate ト}};
\node[Kunyomi] at (-23.250000, -33.700000) {\hbox{\tate ねた.む}};
\node[Meaning] at (-23.200000, -32.050000) {jealousy};
\node[Kanji] at (-21.150000, -33.300000) {\textcolor[HTML]{1059be}{嫉}};
\node[Square] at (-21.150000, -33.800000) {};
\node[Onyomi] at (-21.100000, -33.700000) {\hbox{\tate シツ}};
\node[Meaning] at (-21.150000, -32.050000) {envy};
\node[Kanji] at (-19.100000, -33.300000) {\textcolor[HTML]{a3bac2}{塞}};
\node[Square] at (-19.100000, -33.800000) {};
\node[Onyomi] at (-19.050000, -33.700000) {\hbox{\tate サイ・ソク}};
\node[Kunyomi] at (-19.150000, -33.700000) {\hbox{\tate ふさ.ぐ}};
\node[Meaning] at (-19.100000, -32.050000) {obstruct};
\node[Kanji] at (-17.050000, -33.300000) {\textcolor[HTML]{91b7c3}{尻}};
\node[Square] at (-17.050000, -33.800000) {};
\node[Kunyomi] at (-17.100000, -33.700000) {\hbox{\tate しり}};
\node[Meaning] at (-17.050000, -32.050000) {butt};
\node[Kanji] at (-15.000000, -33.300000) {\textcolor[HTML]{68a4bc}{崖}};
\node[Square] at (-15.000000, -33.800000) {};
\node[Onyomi] at (-14.950000, -33.700000) {\hbox{\tate ガイ}};
\node[Kunyomi] at (-15.050000, -33.700000) {\hbox{\tate がけ}};
\node[Meaning] at (-15.000000, -32.050000) {cliff};
\node[Kanji] at (-12.950000, -33.300000) {\textcolor[HTML]{a3bac2}{庄}};
\node[Square] at (-12.950000, -33.800000) {};
\node[Onyomi] at (-12.900000, -33.700000) {\hbox{\tate ショウ}};
\node[Meaning] at (-12.950000, -32.050000) {manor};
\node[Kanji] at (-10.900000, -33.300000) {\textcolor[HTML]{a3bac2}{弥}};
\node[Square] at (-10.900000, -33.800000) {};
\node[Onyomi] at (-10.850000, -33.700000) {\hbox{\tate ビ・ミ}};
\node[Kunyomi] at (-10.950000, -33.700000) {\hbox{\tate や}};
\node[Meaning] at (-10.900000, -32.050000) {increasing};
\node[Kanji] at (-8.850000, -33.300000) {\textcolor[HTML]{408dba}{挨}};
\node[Square] at (-8.850000, -33.800000) {};
\node[Onyomi] at (-8.800000, -33.700000) {\hbox{\tate アイ}};
\node[Meaning] at (-8.850000, -32.050000) {push open};
\node[Kanji] at (-6.800000, -33.300000) {\textcolor[HTML]{1e76bb}{捻}};
\node[Square] at (-6.800000, -33.800000) {};
\node[Onyomi] at (-6.750000, -33.700000) {\hbox{\tate ネン}};
\node[Kunyomi] at (-6.850000, -33.700000) {\hbox{\tate ひね.る}};
\node[Meaning] at (-6.800000, -32.050000) {twist};
\node[Kanji] at (-4.750000, -33.300000) {\textcolor[HTML]{1e76bb}{拭}};
\node[Square] at (-4.750000, -33.800000) {};
\node[Onyomi] at (-4.700000, -33.700000) {\hbox{\tate ショク}};
\node[Kunyomi] at (-4.800000, -33.700000) {\hbox{\tate ふ.く}};
\node[Meaning] at (-4.750000, -32.050000) {wipe};
\node[Kanji] at (-2.700000, -33.300000) {\textcolor[HTML]{91b7c3}{捉}};
\node[Square] at (-2.700000, -33.800000) {};
\node[Onyomi] at (-2.650000, -33.700000) {\hbox{\tate ソク}};
\node[Kunyomi] at (-2.750000, -33.700000) {\hbox{\tate とら.える}};
\node[Meaning] at (-2.700000, -32.050000) {capture};
\node[Kanji] at (-0.650000, -33.300000) {\textcolor[HTML]{408dba}{拶}};
\node[Square] at (-0.650000, -33.800000) {};
\node[Onyomi] at (-0.600000, -33.700000) {\hbox{\tate サツ}};
\node[Meaning] at (-0.650000, -32.050000) {be imminent};
\node[Kanji] at (1.400000, -33.300000) {\textcolor[HTML]{1059be}{捗}};
\node[Square] at (1.400000, -33.800000) {};
\node[Onyomi] at (1.450000, -33.700000) {\hbox{\tate チョク}};
\node[Meaning] at (1.400000, -32.050000) {make progress};
\node[Kanji] at (3.450000, -33.300000) {\textcolor[HTML]{68a4bc}{憧}};
\node[Square] at (3.450000, -33.800000) {};
\node[Onyomi] at (3.500000, -33.700000) {\hbox{\tate ショウ・トウ}};
\node[Kunyomi] at (3.400000, -33.700000) {\hbox{\tate あこが}};
\node[Meaning] at (3.450000, -32.050000) {long for};
\node[Kanji] at (5.500000, -33.300000) {\textcolor[HTML]{68a4bc}{湧}};
\node[Square] at (5.500000, -33.800000) {};
\node[Onyomi] at (5.550000, -33.700000) {\hbox{\tate ユウ・ユ}};
\node[Kunyomi] at (5.450000, -33.700000) {\hbox{\tate わ}};
\node[Meaning] at (5.500000, -32.050000) {well};
\node[Kanji] at (7.550000, -33.300000) {\textcolor[HTML]{91b7c3}{沙}};
\node[Square] at (7.550000, -33.800000) {};
\node[Onyomi] at (7.600000, -33.700000) {\hbox{\tate サ・シャ}};
\node[Kunyomi] at (7.500000, -33.700000) {\hbox{\tate すな}};
\node[Meaning] at (7.550000, -32.050000) {sand};
\node[Kanji] at (9.600000, -33.300000) {\textcolor[HTML]{1059be}{淫}};
\node[Square] at (9.600000, -33.800000) {};
\node[Onyomi] at (9.650000, -33.700000) {\hbox{\tate イン}};
\node[Kunyomi] at (9.550000, -33.700000) {\hbox{\tate みだ.ら}};
\node[Meaning] at (9.600000, -32.050000) {lewdness};
\node[Kanji] at (11.650000, -33.300000) {\textcolor[HTML]{1e76bb}{氾}};
\node[Square] at (11.650000, -33.800000) {};
\node[Onyomi] at (11.700000, -33.700000) {\hbox{\tate ハン}};
\node[Meaning] at (11.650000, -32.050000) {spread out};
\node[Kanji] at (13.700000, -33.300000) {\textcolor[HTML]{1e76bb}{溺}};
\node[Square] at (13.700000, -33.800000) {};
\node[Onyomi] at (13.750000, -33.700000) {\hbox{\tate デキ}};
\node[Kunyomi] at (13.650000, -33.700000) {\hbox{\tate おぼ.れる}};
\node[Meaning] at (13.700000, -32.050000) {drown};
\node[Kanji] at (15.750000, -33.300000) {\textcolor[HTML]{408dba}{汰}};
\node[Square] at (15.750000, -33.800000) {};
\node[Onyomi] at (15.800000, -33.700000) {\hbox{\tate タ・タイ}};
\node[Kunyomi] at (15.700000, -33.700000) {\hbox{\tate おご・にご}};
\node[Meaning] at (15.750000, -32.050000) {select};
\node[Kanji] at (17.800000, -33.300000) {\textcolor[HTML]{68a4bc}{潰}};
\node[Square] at (17.800000, -33.800000) {};
\node[Onyomi] at (17.850000, -33.700000) {\hbox{\tate カイ}};
\node[Kunyomi] at (17.750000, -33.700000) {\hbox{\tate つぶ.す}};
\node[Meaning] at (17.800000, -32.050000) {crush};
\node[Kanji] at (19.850000, -33.300000) {\textcolor[HTML]{68a4bc}{汎}};
\node[Square] at (19.850000, -33.800000) {};
\node[Onyomi] at (19.900000, -33.700000) {\hbox{\tate ハン}};
\node[Meaning] at (19.850000, -32.050000) {pan-};
\node[Kanji] at (21.900000, -33.300000) {\textcolor[HTML]{68a4bc}{淀}};
\node[Square] at (21.900000, -33.800000) {};
\node[Kunyomi] at (21.850000, -33.700000) {\hbox{\tate よど}};
\node[Meaning] at (21.900000, -32.050000) {eddy};
\node[Kanji] at (23.950000, -33.300000) {\textcolor[HTML]{91b7c3}{釜}};
\node[Square] at (23.950000, -33.800000) {};
\node[Kunyomi] at (23.900000, -33.700000) {\hbox{\tate かま}};
\node[Meaning] at (23.950000, -32.050000) {kettle};
\node[Kanji] at (26.000000, -33.300000) {\textcolor[HTML]{1e76bb}{狐}};
\node[Square] at (26.000000, -33.800000) {};
\node[Onyomi] at (26.050000, -33.700000) {\hbox{\tate コ}};
\node[Kunyomi] at (25.950000, -33.700000) {\hbox{\tate きつね}};
\node[Meaning] at (26.000000, -32.050000) {fox};
\node[Kanji] at (28.050000, -33.300000) {\textcolor[HTML]{a3bac2}{狙}};
\node[Square] at (28.050000, -33.800000) {};
\node[Onyomi] at (28.100000, -33.700000) {\hbox{\tate ソ}};
\node[Kunyomi] at (28.000000, -33.700000) {\hbox{\tate ねら.い}};
\node[Meaning] at (28.050000, -32.050000) {aim};
\node[Kanji] at (30.100000, -33.300000) {\textcolor[HTML]{1e76bb}{莉}};
\node[Square] at (30.100000, -33.800000) {};
\node[Onyomi] at (30.150000, -33.700000) {\hbox{\tate リ・レイ}};
\node[Meaning] at (30.100000, -32.050000) {jasmine};
\node[Kanji] at (32.150000, -33.300000) {\textcolor[HTML]{1059be}{萎}};
\node[Square] at (32.150000, -33.800000) {};
\node[Onyomi] at (32.200000, -33.700000) {\hbox{\tate イ}};
\node[Kunyomi] at (32.100000, -33.700000) {\hbox{\tate な.える}};
\node[Meaning] at (32.150000, -32.050000) {wither};
\node[Kanji] at (34.200000, -33.300000) {\textcolor[HTML]{1e76bb}{蔽}};
\node[Square] at (34.200000, -33.800000) {};
\node[Onyomi] at (34.250000, -33.700000) {\hbox{\tate ヘイ}};
\node[Meaning] at (34.200000, -32.050000) {cover};
\node[Kanji] at (36.250000, -33.300000) {\textcolor[HTML]{91b7c3}{蓮}};
\node[Square] at (36.250000, -33.800000) {};
\node[Onyomi] at (36.300000, -33.700000) {\hbox{\tate レン}};
\node[Kunyomi] at (36.200000, -33.700000) {\hbox{\tate はす・はちす}};
\node[Meaning] at (36.250000, -32.050000) {lotus};
\node[Kanji] at (38.300000, -33.300000) {\textcolor[HTML]{1e76bb}{芯}};
\node[Square] at (38.300000, -33.800000) {};
\node[Onyomi] at (38.350000, -33.700000) {\hbox{\tate シン}};
\node[Meaning] at (38.300000, -32.050000) {wick};
\node[Kanji] at (40.350000, -33.300000) {\textcolor[HTML]{68a4bc}{藍}};
\node[Square] at (40.350000, -33.800000) {};
\node[Onyomi] at (40.400000, -33.700000) {\hbox{\tate ラン}};
\node[Kunyomi] at (40.300000, -33.700000) {\hbox{\tate あい}};
\node[Meaning] at (40.350000, -32.050000) {indigo};
\node[Kanji] at (42.400000, -33.300000) {\textcolor[HTML]{1e76bb}{苛}};
\node[Square] at (42.400000, -33.800000) {};
\node[Onyomi] at (42.450000, -33.700000) {\hbox{\tate カ}};
\node[Meaning] at (42.400000, -32.050000) {torment};
\node[Kanji] at (44.450000, -33.300000) {\textcolor[HTML]{68a4bc}{萌}};
\node[Square] at (44.450000, -33.800000) {};
\node[Onyomi] at (44.500000, -33.700000) {\hbox{\tate ホウ}};
\node[Kunyomi] at (44.400000, -33.700000) {\hbox{\tate きざ・めばえ}};
\node[Meaning] at (44.450000, -32.050000) {sprout};
\node[Kanji] at (46.500000, -33.300000) {\textcolor[HTML]{68a4bc}{蒙}};
\node[Square] at (46.500000, -33.800000) {};
\node[Onyomi] at (46.550000, -33.700000) {\hbox{\tate モウ・ボウ}};
\node[Kunyomi] at (46.450000, -33.700000) {\hbox{\tate おお・くら}};
\node[Meaning] at (46.500000, -32.050000) {darkness};
\node[Kanji] at (48.550000, -33.300000) {\textcolor[HTML]{91b7c3}{蓋}};
\node[Square] at (48.550000, -33.800000) {};
\node[Onyomi] at (48.600000, -33.700000) {\hbox{\tate ガイ}};
\node[Kunyomi] at (48.500000, -33.700000) {\hbox{\tate ふた}};
\node[Meaning] at (48.550000, -32.050000) {cover};
\node[Kanji] at (50.600000, -33.300000) {\textcolor[HTML]{1e76bb}{蔑}};
\node[Square] at (50.600000, -33.800000) {};
\node[Onyomi] at (50.650000, -33.700000) {\hbox{\tate ベツ}};
\node[Kunyomi] at (50.550000, -33.700000) {\hbox{\tate さげす}};
\node[Meaning] at (50.600000, -32.050000) {scorn};
\node[Kanji] at (52.650000, -33.300000) {\textcolor[HTML]{1e76bb}{葵}};
\node[Square] at (52.650000, -33.800000) {};
\node[Onyomi] at (52.700000, -33.700000) {\hbox{\tate キ}};
\node[Kunyomi] at (52.600000, -33.700000) {\hbox{\tate あおい}};
\node[Meaning] at (52.650000, -32.050000) {hollyhock};
\node[Kanji] at (54.700000, -33.300000) {\textcolor[HTML]{91b7c3}{葛}};
\node[Square] at (54.700000, -33.800000) {};
\node[Onyomi] at (54.750000, -33.700000) {\hbox{\tate カツ}};
\node[Kunyomi] at (54.650000, -33.700000) {\hbox{\tate くず}};
\node[Meaning] at (54.700000, -32.050000) {arrowroot};
\node[Kanji] at (56.750000, -33.300000) {\textcolor[HTML]{408dba}{蒼}};
\node[Square] at (56.750000, -33.800000) {};
\node[Onyomi] at (56.800000, -33.700000) {\hbox{\tate ソウ}};
\node[Kunyomi] at (56.700000, -33.700000) {\hbox{\tate あお}};
\node[Meaning] at (56.750000, -32.050000) {pale};
\node[Meaning] at (-58.050000, -33.200000) {98.19\%};
\node[Kanji] at (-56.000000, -35.350000) {\textcolor[HTML]{29409e}{茜}};
\node[Square] at (-56.000000, -35.850000) {};
\node[Onyomi] at (-55.950000, -35.750000) {\hbox{\tate セン}};
\node[Kunyomi] at (-56.050000, -35.750000) {\hbox{\tate あかね}};
\node[Meaning] at (-56.000000, -34.100000) {red dye};
\node[Kanji] at (-53.950000, -35.350000) {\textcolor[HTML]{91b7c3}{菅}};
\node[Square] at (-53.950000, -35.850000) {};
\node[Onyomi] at (-53.900000, -35.750000) {\hbox{\tate カン・ケン}};
\node[Kunyomi] at (-54.000000, -35.750000) {\hbox{\tate すげ}};
\node[Meaning] at (-53.950000, -34.100000) {sedge};
\node[Kanji] at (-51.900000, -35.350000) {\textcolor[HTML]{408dba}{遥}};
\node[Square] at (-51.900000, -35.850000) {};
\node[Onyomi] at (-51.850000, -35.750000) {\hbox{\tate ヨウ}};
\node[Kunyomi] at (-51.950000, -35.750000) {\hbox{\tate はる}};
\node[Meaning] at (-51.900000, -34.100000) {far off};
\node[Kanji] at (-49.850000, -35.350000) {\textcolor[HTML]{68a4bc}{遼}};
\node[Square] at (-49.850000, -35.850000) {};
\node[Onyomi] at (-49.800000, -35.750000) {\hbox{\tate リョウ}};
\node[Meaning] at (-49.850000, -34.100000) {distant};
\node[Kanji] at (-47.800000, -35.350000) {\textcolor[HTML]{1059be}{遜}};
\node[Square] at (-47.800000, -35.850000) {};
\node[Onyomi] at (-47.750000, -35.750000) {\hbox{\tate ソン}};
\node[Kunyomi] at (-47.850000, -35.750000) {\hbox{\tate したが.う}};
\node[Meaning] at (-47.800000, -34.100000) {humble};
\node[Kanji] at (-45.750000, -35.350000) {\textcolor[HTML]{68a4bc}{隙}};
\node[Square] at (-45.750000, -35.850000) {};
\node[Onyomi] at (-45.700000, -35.750000) {\hbox{\tate ゲキ}};
\node[Kunyomi] at (-45.800000, -35.750000) {\hbox{\tate すき}};
\node[Meaning] at (-45.750000, -34.100000) {crevice};
\node[Kanji] at (-43.700000, -35.350000) {\textcolor[HTML]{68a4bc}{曖}};
\node[Square] at (-43.700000, -35.850000) {};
\node[Onyomi] at (-43.650000, -35.750000) {\hbox{\tate アイ}};
\node[Meaning] at (-43.700000, -34.100000) {not clear};
\node[Kanji] at (-41.650000, -35.350000) {\textcolor[HTML]{91b7c3}{昧}};
\node[Square] at (-41.650000, -35.850000) {};
\node[Onyomi] at (-41.600000, -35.750000) {\hbox{\tate マイ}};
\node[Meaning] at (-41.650000, -34.100000) {foolish};
\node[Kanji] at (-39.600000, -35.350000) {\textcolor[HTML]{1059be}{曙}};
\node[Square] at (-39.600000, -35.850000) {};
\node[Onyomi] at (-39.550000, -35.750000) {\hbox{\tate ショ}};
\node[Kunyomi] at (-39.650000, -35.750000) {\hbox{\tate あけぼの}};
\node[Meaning] at (-39.600000, -34.100000) {dawn};
\node[Kanji] at (-37.550000, -35.350000) {\textcolor[HTML]{1e76bb}{旺}};
\node[Square] at (-37.550000, -35.850000) {};
\node[Onyomi] at (-37.500000, -35.750000) {\hbox{\tate オウ}};
\node[Meaning] at (-37.550000, -34.100000) {flourishing};
\node[Kanji] at (-35.500000, -35.350000) {\textcolor[HTML]{408dba}{腎}};
\node[Square] at (-35.500000, -35.850000) {};
\node[Onyomi] at (-35.450000, -35.750000) {\hbox{\tate ジン}};
\node[Meaning] at (-35.500000, -34.100000) {kidney};
\node[Kanji] at (-33.450000, -35.350000) {\textcolor[HTML]{68a4bc}{股}};
\node[Square] at (-33.450000, -35.850000) {};
\node[Onyomi] at (-33.400000, -35.750000) {\hbox{\tate コ}};
\node[Kunyomi] at (-33.500000, -35.750000) {\hbox{\tate また}};
\node[Meaning] at (-33.450000, -34.100000) {crotch};
\node[Kanji] at (-31.400000, -35.350000) {\textcolor[HTML]{29409e}{臆}};
\node[Square] at (-31.400000, -35.850000) {};
\node[Onyomi] at (-31.350000, -35.750000) {\hbox{\tate オク}};
\node[Meaning] at (-31.400000, -34.100000) {timidity};
\node[Kanji] at (-29.350000, -35.350000) {\textcolor[HTML]{408dba}{膝}};
\node[Square] at (-29.350000, -35.850000) {};
\node[Kunyomi] at (-29.400000, -35.750000) {\hbox{\tate ひざ}};
\node[Meaning] at (-29.350000, -34.100000) {knee};
\node[Kanji] at (-27.300000, -35.350000) {\textcolor[HTML]{1e76bb}{肘}};
\node[Square] at (-27.300000, -35.850000) {};
\node[Kunyomi] at (-27.350000, -35.750000) {\hbox{\tate ひじ}};
\node[Meaning] at (-27.300000, -34.100000) {elbow};
\node[Kanji] at (-25.250000, -35.350000) {\textcolor[HTML]{68a4bc}{腺}};
\node[Square] at (-25.250000, -35.850000) {};
\node[Onyomi] at (-25.200000, -35.750000) {\hbox{\tate セン}};
\node[Meaning] at (-25.250000, -34.100000) {gland};
\node[Kanji] at (-23.200000, -35.350000) {\textcolor[HTML]{91b7c3}{腫}};
\node[Square] at (-23.200000, -35.850000) {};
\node[Onyomi] at (-23.150000, -35.750000) {\hbox{\tate シュ}};
\node[Kunyomi] at (-23.250000, -35.750000) {\hbox{\tate は.れる}};
\node[Meaning] at (-23.200000, -34.100000) {tumor};
\node[Kanji] at (-21.150000, -35.350000) {\textcolor[HTML]{408dba}{膳}};
\node[Square] at (-21.150000, -35.850000) {};
\node[Onyomi] at (-21.100000, -35.750000) {\hbox{\tate ゼン}};
\node[Meaning] at (-21.150000, -34.100000) {tray};
\node[Kanji] at (-19.100000, -35.350000) {\textcolor[HTML]{68a4bc}{胡}};
\node[Square] at (-19.100000, -35.850000) {};
\node[Onyomi] at (-19.050000, -35.750000) {\hbox{\tate コ・ウ・ゴ}};
\node[Kunyomi] at (-19.150000, -35.750000) {\hbox{\tate なんぞ}};
\node[Meaning] at (-19.100000, -34.100000) {barbarian};
\node[Kanji] at (-17.050000, -35.350000) {\textcolor[HTML]{1059be}{楓}};
\node[Square] at (-17.050000, -35.850000) {};
\node[Onyomi] at (-17.000000, -35.750000) {\hbox{\tate フウ}};
\node[Kunyomi] at (-17.100000, -35.750000) {\hbox{\tate かえで}};
\node[Meaning] at (-17.050000, -34.100000) {maple};
\node[Kanji] at (-15.000000, -35.350000) {\textcolor[HTML]{408dba}{枕}};
\node[Square] at (-15.000000, -35.850000) {};
\node[Onyomi] at (-14.950000, -35.750000) {\hbox{\tate シン}};
\node[Kunyomi] at (-15.050000, -35.750000) {\hbox{\tate まくら}};
\node[Meaning] at (-15.000000, -34.100000) {pillow};
\node[Kanji] at (-12.950000, -35.350000) {\textcolor[HTML]{68a4bc}{椅}};
\node[Square] at (-12.950000, -35.850000) {};
\node[Onyomi] at (-12.900000, -35.750000) {\hbox{\tate イ}};
\node[Meaning] at (-12.950000, -34.100000) {chair};
\node[Kanji] at (-10.900000, -35.350000) {\textcolor[HTML]{408dba}{柿}};
\node[Square] at (-10.900000, -35.850000) {};
\node[Kunyomi] at (-10.950000, -35.750000) {\hbox{\tate かき}};
\node[Meaning] at (-10.900000, -34.100000) {persimmon};
\node[Kanji] at (-8.850000, -35.350000) {\textcolor[HTML]{91b7c3}{桁}};
\node[Square] at (-8.850000, -35.850000) {};
\node[Kunyomi] at (-8.900000, -35.750000) {\hbox{\tate けた}};
\node[Meaning] at (-8.850000, -34.100000) {beam};
\node[Kanji] at (-6.800000, -35.350000) {\textcolor[HTML]{1e76bb}{梗}};
\node[Square] at (-6.800000, -35.850000) {};
\node[Onyomi] at (-6.750000, -35.750000) {\hbox{\tate コウ}};
\node[Meaning] at (-6.800000, -34.100000) {close up};
\node[Kanji] at (-4.750000, -35.350000) {\textcolor[HTML]{91b7c3}{椎}};
\node[Square] at (-4.750000, -35.850000) {};
\node[Onyomi] at (-4.700000, -35.750000) {\hbox{\tate ツイ・スイ}};
\node[Kunyomi] at (-4.800000, -35.750000) {\hbox{\tate う・つち}};
\node[Meaning] at (-4.750000, -34.100000) {oak};
\node[Kanji] at (-2.700000, -35.350000) {\textcolor[HTML]{408dba}{柵}};
\node[Square] at (-2.700000, -35.850000) {};
\node[Onyomi] at (-2.650000, -35.750000) {\hbox{\tate サク}};
\node[Meaning] at (-2.700000, -34.100000) {fence};
\node[Kanji] at (-0.650000, -35.350000) {\textcolor[HTML]{242e6c}{栞}};
\node[Square] at (-0.650000, -35.850000) {};
\node[Onyomi] at (-0.600000, -35.750000) {\hbox{\tate カン}};
\node[Kunyomi] at (-0.700000, -35.750000) {\hbox{\tate しおり}};
\node[Meaning] at (-0.650000, -34.100000) {bookmark};
\node[Kanji] at (1.400000, -35.350000) {\textcolor[HTML]{1e76bb}{煎}};
\node[Square] at (1.400000, -35.850000) {};
\node[Onyomi] at (1.450000, -35.750000) {\hbox{\tate セン}};
\node[Kunyomi] at (1.350000, -35.750000) {\hbox{\tate い.る}};
\node[Meaning] at (1.400000, -34.100000) {broil};
\node[Kanji] at (3.450000, -35.350000) {\textcolor[HTML]{408dba}{瑠}};
\node[Square] at (3.450000, -35.850000) {};
\node[Onyomi] at (3.500000, -35.750000) {\hbox{\tate ル・リュウ}};
\node[Meaning] at (3.450000, -34.100000) {lapis lazuli};
\node[Kanji] at (5.500000, -35.350000) {\textcolor[HTML]{91b7c3}{斑}};
\node[Square] at (5.500000, -35.850000) {};
\node[Onyomi] at (5.550000, -35.750000) {\hbox{\tate ハン}};
\node[Meaning] at (5.500000, -34.100000) {blemish};
\node[Kanji] at (7.550000, -35.350000) {\textcolor[HTML]{1e76bb}{弄}};
\node[Square] at (7.550000, -35.850000) {};
\node[Onyomi] at (7.600000, -35.750000) {\hbox{\tate ロウ}};
\node[Kunyomi] at (7.500000, -35.750000) {\hbox{\tate もてあそ.ぶ}};
\node[Meaning] at (7.550000, -34.100000) {tamper with};
\node[Kanji] at (9.600000, -35.350000) {\textcolor[HTML]{91b7c3}{瑞}};
\node[Square] at (9.600000, -35.850000) {};
\node[Onyomi] at (9.650000, -35.750000) {\hbox{\tate スイ・ズイ}};
\node[Kunyomi] at (9.550000, -35.750000) {\hbox{\tate みず・しるし}};
\node[Meaning] at (9.600000, -34.100000) {congratulations};
\node[Kanji] at (11.650000, -35.350000) {\textcolor[HTML]{29409e}{瑛}};
\node[Square] at (11.650000, -35.850000) {};
\node[Onyomi] at (11.700000, -35.750000) {\hbox{\tate エイ}};
\node[Meaning] at (11.650000, -34.100000) {crystal};
\node[Kanji] at (13.700000, -35.350000) {\textcolor[HTML]{68a4bc}{玩}};
\node[Square] at (13.700000, -35.850000) {};
\node[Onyomi] at (13.750000, -35.750000) {\hbox{\tate ガン}};
\node[Meaning] at (13.700000, -34.100000) {trifle with};
\node[Kanji] at (15.750000, -35.350000) {\textcolor[HTML]{1059be}{畏}};
\node[Square] at (15.750000, -35.850000) {};
\node[Onyomi] at (15.800000, -35.750000) {\hbox{\tate イ}};
\node[Kunyomi] at (15.700000, -35.750000) {\hbox{\tate おそ.れる}};
\node[Meaning] at (15.750000, -34.100000) {fear};
\node[Kanji] at (17.800000, -35.350000) {\textcolor[HTML]{1059be}{痩}};
\node[Square] at (17.800000, -35.850000) {};
\node[Onyomi] at (17.850000, -35.750000) {\hbox{\tate ソウ}};
\node[Kunyomi] at (17.750000, -35.750000) {\hbox{\tate や.せる}};
\node[Meaning] at (17.800000, -34.100000) {get thin};
\node[Kanji] at (19.850000, -35.350000) {\textcolor[HTML]{68a4bc}{痕}};
\node[Square] at (19.850000, -35.850000) {};
\node[Onyomi] at (19.900000, -35.750000) {\hbox{\tate コン}};
\node[Kunyomi] at (19.800000, -35.750000) {\hbox{\tate あと}};
\node[Meaning] at (19.850000, -34.100000) {mark};
\node[Kanji] at (21.900000, -35.350000) {\textcolor[HTML]{408dba}{瞭}};
\node[Square] at (21.900000, -35.850000) {};
\node[Onyomi] at (21.950000, -35.750000) {\hbox{\tate リョウ}};
\node[Meaning] at (21.900000, -34.100000) {clear};
\node[Kanji] at (23.950000, -35.350000) {\textcolor[HTML]{1e76bb}{眉}};
\node[Square] at (23.950000, -35.850000) {};
\node[Onyomi] at (24.000000, -35.750000) {\hbox{\tate ビ・(ミ)}};
\node[Kunyomi] at (23.900000, -35.750000) {\hbox{\tate まゆ}};
\node[Meaning] at (23.950000, -34.100000) {eyebrow};
\node[Kanji] at (26.000000, -35.350000) {\textcolor[HTML]{408dba}{窟}};
\node[Square] at (26.000000, -35.850000) {};
\node[Onyomi] at (26.050000, -35.750000) {\hbox{\tate クツ}};
\node[Meaning] at (26.000000, -34.100000) {cavern};
\node[Kanji] at (28.050000, -35.350000) {\textcolor[HTML]{242e6c}{颯}};
\node[Square] at (28.050000, -35.850000) {};
\node[Onyomi] at (28.100000, -35.750000) {\hbox{\tate サツ・ソウ}};
\node[Kunyomi] at (28.000000, -35.750000) {\hbox{\tate さっ.と}};
\node[Meaning] at (28.050000, -34.100000) {quick};
\node[Kanji] at (30.100000, -35.350000) {\textcolor[HTML]{68a4bc}{靖}};
\node[Square] at (30.100000, -35.850000) {};
\node[Onyomi] at (30.150000, -35.750000) {\hbox{\tate ジョウ・セイ}};
\node[Kunyomi] at (30.050000, -35.750000) {\hbox{\tate やす}};
\node[Meaning] at (30.100000, -34.100000) {peaceful};
\node[Kanji] at (32.150000, -35.350000) {\textcolor[HTML]{1e76bb}{裾}};
\node[Square] at (32.150000, -35.850000) {};
\node[Kunyomi] at (32.100000, -35.750000) {\hbox{\tate すそ}};
\node[Meaning] at (32.150000, -34.100000) {cuff};
\node[Kanji] at (34.200000, -35.350000) {\textcolor[HTML]{1e76bb}{箸}};
\node[Square] at (34.200000, -35.850000) {};
\node[Onyomi] at (34.250000, -35.750000) {\hbox{\tate チャク}};
\node[Kunyomi] at (34.150000, -35.750000) {\hbox{\tate はし}};
\node[Meaning] at (34.200000, -34.100000) {chopsticks};
\node[Kanji] at (36.250000, -35.350000) {\textcolor[HTML]{29409e}{緋}};
\node[Square] at (36.250000, -35.850000) {};
\node[Onyomi] at (36.300000, -35.750000) {\hbox{\tate ヒ}};
\node[Kunyomi] at (36.200000, -35.750000) {\hbox{\tate あか・あけ}};
\node[Meaning] at (36.250000, -34.100000) {scarlet};
\node[Kanji] at (38.300000, -35.350000) {\textcolor[HTML]{1059be}{綺}};
\node[Square] at (38.300000, -35.850000) {};
\node[Onyomi] at (38.350000, -35.750000) {\hbox{\tate キ}};
\node[Meaning] at (38.300000, -34.100000) {beautiful};
\node[Kanji] at (40.350000, -35.350000) {\textcolor[HTML]{68a4bc}{綾}};
\node[Square] at (40.350000, -35.850000) {};
\node[Onyomi] at (40.400000, -35.750000) {\hbox{\tate リン}};
\node[Kunyomi] at (40.300000, -35.750000) {\hbox{\tate あや}};
\node[Meaning] at (40.350000, -34.100000) {design};
\node[Kanji] at (42.400000, -35.350000) {\textcolor[HTML]{408dba}{綻}};
\node[Square] at (42.400000, -35.850000) {};
\node[Onyomi] at (42.450000, -35.750000) {\hbox{\tate タン}};
\node[Kunyomi] at (42.350000, -35.750000) {\hbox{\tate ほころ.びる}};
\node[Meaning] at (42.400000, -34.100000) {rip};
\node[Kanji] at (44.450000, -35.350000) {\textcolor[HTML]{91b7c3}{舷}};
\node[Square] at (44.450000, -35.850000) {};
\node[Onyomi] at (44.500000, -35.750000) {\hbox{\tate ゲン}};
\node[Meaning] at (44.450000, -34.100000) {gunwale};
\node[Kanji] at (46.500000, -35.350000) {\textcolor[HTML]{408dba}{聡}};
\node[Square] at (46.500000, -35.850000) {};
\node[Onyomi] at (46.550000, -35.750000) {\hbox{\tate ソウ}};
\node[Kunyomi] at (46.450000, -35.750000) {\hbox{\tate さと.い}};
\node[Meaning] at (46.500000, -34.100000) {wise};
\node[Kanji] at (48.550000, -35.350000) {\textcolor[HTML]{1059be}{蟹}};
\node[Square] at (48.550000, -35.850000) {};
\node[Kunyomi] at (48.500000, -35.750000) {\hbox{\tate かに}};
\node[Meaning] at (48.550000, -34.100000) {crab};
\node[Kanji] at (50.600000, -35.350000) {\textcolor[HTML]{68a4bc}{蜂}};
\node[Square] at (50.600000, -35.850000) {};
\node[Onyomi] at (50.650000, -35.750000) {\hbox{\tate ホウ}};
\node[Kunyomi] at (50.550000, -35.750000) {\hbox{\tate はち}};
\node[Meaning] at (50.600000, -34.100000) {bee};
\node[Kanji] at (52.650000, -35.350000) {\textcolor[HTML]{1e76bb}{罵}};
\node[Square] at (52.650000, -35.850000) {};
\node[Onyomi] at (52.700000, -35.750000) {\hbox{\tate バ}};
\node[Kunyomi] at (52.600000, -35.750000) {\hbox{\tate ののし.る}};
\node[Meaning] at (52.650000, -34.100000) {insult};
\node[Kanji] at (54.700000, -35.350000) {\textcolor[HTML]{408dba}{戴}};
\node[Square] at (54.700000, -35.850000) {};
\node[Onyomi] at (54.750000, -35.750000) {\hbox{\tate タイ}};
\node[Kunyomi] at (54.650000, -35.750000) {\hbox{\tate いただ}};
\node[Meaning] at (54.700000, -34.100000) {receive};
\node[Kanji] at (56.750000, -35.350000) {\textcolor[HTML]{68a4bc}{哉}};
\node[Square] at (56.750000, -35.850000) {};
\node[Onyomi] at (56.800000, -35.750000) {\hbox{\tate サイ}};
\node[Kunyomi] at (56.700000, -35.750000) {\hbox{\tate や・かな}};
\node[Meaning] at (56.750000, -34.100000) {question mark};
\node[Meaning] at (-58.050000, -35.250000) {98.30\%};
\node[Kanji] at (-56.000000, -37.400000) {\textcolor[HTML]{91b7c3}{謎}};
\node[Square] at (-56.000000, -37.900000) {};
\node[Kunyomi] at (-56.050000, -37.800000) {\hbox{\tate なぞ}};
\node[Meaning] at (-56.000000, -36.150000) {riddle};
\node[Kanji] at (-53.950000, -37.400000) {\textcolor[HTML]{408dba}{諒}};
\node[Square] at (-53.950000, -37.900000) {};
\node[Onyomi] at (-53.900000, -37.800000) {\hbox{\tate リョウ}};
\node[Kunyomi] at (-54.000000, -37.800000) {\hbox{\tate あきら.か}};
\node[Meaning] at (-53.950000, -36.150000) {comprehend};
\node[Kanji] at (-51.900000, -37.400000) {\textcolor[HTML]{a3bac2}{誰}};
\node[Square] at (-51.900000, -37.900000) {};
\node[Kunyomi] at (-51.950000, -37.800000) {\hbox{\tate だれ}};
\node[Meaning] at (-51.900000, -36.150000) {who};
\node[Kanji] at (-49.850000, -37.400000) {\textcolor[HTML]{68a4bc}{詣}};
\node[Square] at (-49.850000, -37.900000) {};
\node[Onyomi] at (-49.800000, -37.800000) {\hbox{\tate ケイ}};
\node[Kunyomi] at (-49.900000, -37.800000) {\hbox{\tate もう.でる}};
\node[Meaning] at (-49.850000, -36.150000) {visit a temple};
\node[Kanji] at (-47.800000, -37.400000) {\textcolor[HTML]{408dba}{諦}};
\node[Square] at (-47.800000, -37.900000) {};
\node[Onyomi] at (-47.750000, -37.800000) {\hbox{\tate テイ}};
\node[Kunyomi] at (-47.850000, -37.800000) {\hbox{\tate あきら.める}};
\node[Meaning] at (-47.800000, -36.150000) {abandon};
\node[Kanji] at (-45.750000, -37.400000) {\textcolor[HTML]{1e76bb}{詮}};
\node[Square] at (-45.750000, -37.900000) {};
\node[Onyomi] at (-45.700000, -37.800000) {\hbox{\tate セン}};
\node[Meaning] at (-45.750000, -36.150000) {discussion};
\node[Kanji] at (-43.700000, -37.400000) {\textcolor[HTML]{91b7c3}{輔}};
\node[Square] at (-43.700000, -37.900000) {};
\node[Onyomi] at (-43.650000, -37.800000) {\hbox{\tate フ・ホ}};
\node[Kunyomi] at (-43.750000, -37.800000) {\hbox{\tate たす.ける}};
\node[Meaning] at (-43.700000, -36.150000) {help};
\node[Kanji] at (-41.650000, -37.400000) {\textcolor[HTML]{68a4bc}{貌}};
\node[Square] at (-41.650000, -37.900000) {};
\node[Onyomi] at (-41.600000, -37.800000) {\hbox{\tate ボウ}};
\node[Meaning] at (-41.650000, -36.150000) {appearance};
\node[Kanji] at (-39.600000, -37.400000) {\textcolor[HTML]{91b7c3}{貼}};
\node[Square] at (-39.600000, -37.900000) {};
\node[Onyomi] at (-39.550000, -37.800000) {\hbox{\tate チョウ}};
\node[Kunyomi] at (-39.650000, -37.800000) {\hbox{\tate は}};
\node[Meaning] at (-39.600000, -36.150000) {paste};
\node[Kanji] at (-37.550000, -37.400000) {\textcolor[HTML]{29409e}{賂}};
\node[Square] at (-37.550000, -37.900000) {};
\node[Onyomi] at (-37.500000, -37.800000) {\hbox{\tate ロ}};
\node[Meaning] at (-37.550000, -36.150000) {bribe};
\node[Kanji] at (-35.500000, -37.400000) {\textcolor[HTML]{68a4bc}{蹴}};
\node[Square] at (-35.500000, -37.900000) {};
\node[Onyomi] at (-35.450000, -37.800000) {\hbox{\tate シュウ}};
\node[Kunyomi] at (-35.550000, -37.800000) {\hbox{\tate け.る}};
\node[Meaning] at (-35.500000, -36.150000) {kick};
\node[Kanji] at (-33.450000, -37.400000) {\textcolor[HTML]{408dba}{醤}};
\node[Square] at (-33.450000, -37.900000) {};
\node[Onyomi] at (-33.400000, -37.800000) {\hbox{\tate ショウ}};
\node[Meaning] at (-33.450000, -36.150000) {soy sauce};
\node[Kanji] at (-31.400000, -37.400000) {\textcolor[HTML]{1059be}{酎}};
\node[Square] at (-31.400000, -37.900000) {};
\node[Onyomi] at (-31.350000, -37.800000) {\hbox{\tate チュウ・チュ}};
\node[Kunyomi] at (-31.450000, -37.800000) {\hbox{\tate かも.す}};
\node[Meaning] at (-31.400000, -36.150000) {sake};
\node[Kanji] at (-29.350000, -37.400000) {\textcolor[HTML]{1e76bb}{醒}};
\node[Square] at (-29.350000, -37.900000) {};
\node[Onyomi] at (-29.300000, -37.800000) {\hbox{\tate セイ}};
\node[Meaning] at (-29.350000, -36.150000) {disillusioned};
\node[Kanji] at (-27.300000, -37.400000) {\textcolor[HTML]{68a4bc}{麺}};
\node[Square] at (-27.300000, -37.900000) {};
\node[Onyomi] at (-27.250000, -37.800000) {\hbox{\tate メン}};
\node[Meaning] at (-27.300000, -36.150000) {noodles};
\node[Kanji] at (-25.250000, -37.400000) {\textcolor[HTML]{68a4bc}{鍋}};
\node[Square] at (-25.250000, -37.900000) {};
\node[Kunyomi] at (-25.300000, -37.800000) {\hbox{\tate なべ}};
\node[Meaning] at (-25.250000, -36.150000) {pot};
\node[Kanji] at (-23.200000, -37.400000) {\textcolor[HTML]{68a4bc}{鍵}};
\node[Square] at (-23.200000, -37.900000) {};
\node[Onyomi] at (-23.150000, -37.800000) {\hbox{\tate ケン}};
\node[Kunyomi] at (-23.250000, -37.800000) {\hbox{\tate かぎ}};
\node[Meaning] at (-23.200000, -36.150000) {key};
\node[Kanji] at (-21.150000, -37.400000) {\textcolor[HTML]{68a4bc}{闇}};
\node[Square] at (-21.150000, -37.900000) {};
\node[Onyomi] at (-21.100000, -37.800000) {\hbox{\tate アン・オン}};
\node[Kunyomi] at (-21.200000, -37.800000) {\hbox{\tate やみ}};
\node[Meaning] at (-21.150000, -36.150000) {darkness};
\node[Kanji] at (-19.100000, -37.400000) {\textcolor[HTML]{408dba}{頓}};
\node[Square] at (-19.100000, -37.900000) {};
\node[Onyomi] at (-19.050000, -37.800000) {\hbox{\tate トン}};
\node[Meaning] at (-19.100000, -36.150000) {suddenly};
\node[Kanji] at (-17.050000, -37.400000) {\textcolor[HTML]{d2a293}{頃}};
\node[Square] at (-17.050000, -37.900000) {};
\node[Kunyomi] at (-17.100000, -37.800000) {\hbox{\tate ころ・ごろ}};
\node[Meaning] at (-17.050000, -36.150000) {approximate};
\node[Kanji] at (-15.000000, -37.400000) {\textcolor[HTML]{1059be}{頬}};
\node[Square] at (-15.000000, -37.900000) {};
\node[Kunyomi] at (-15.050000, -37.800000) {\hbox{\tate ほお}};
\node[Meaning] at (-15.000000, -36.150000) {cheek};
\node[Kanji] at (-12.950000, -37.400000) {\textcolor[HTML]{68a4bc}{顎}};
\node[Square] at (-12.950000, -37.900000) {};
\node[Onyomi] at (-12.900000, -37.800000) {\hbox{\tate ガク}};
\node[Kunyomi] at (-13.000000, -37.800000) {\hbox{\tate あご}};
\node[Meaning] at (-12.950000, -36.150000) {jaw};
\node[Kanji] at (-10.900000, -37.400000) {\textcolor[HTML]{68a4bc}{餌}};
\node[Square] at (-10.900000, -37.900000) {};
\node[Onyomi] at (-10.850000, -37.800000) {\hbox{\tate ジ}};
\node[Kunyomi] at (-10.950000, -37.800000) {\hbox{\tate えさ・え}};
\node[Meaning] at (-10.900000, -36.150000) {bait};
\node[Kanji] at (-8.850000, -37.400000) {\textcolor[HTML]{408dba}{餅}};
\node[Square] at (-8.850000, -37.900000) {};
\node[Onyomi] at (-8.800000, -37.800000) {\hbox{\tate ヘイ}};
\node[Kunyomi] at (-8.900000, -37.800000) {\hbox{\tate もち}};
\node[Meaning] at (-8.850000, -36.150000) {mochi};
\node[Kanji] at (-6.800000, -37.400000) {\textcolor[HTML]{29409e}{鰐}};
\node[Square] at (-6.800000, -37.900000) {};
\node[Kunyomi] at (-6.850000, -37.800000) {\hbox{\tate わに}};
\node[Meaning] at (-6.800000, -36.150000) {alligator};
\node[Kanji] at (-4.750000, -37.400000) {\textcolor[HTML]{91b7c3}{麓}};
\node[Square] at (-4.750000, -37.900000) {};
\node[Onyomi] at (-4.700000, -37.800000) {\hbox{\tate ロク}};
\node[Kunyomi] at (-4.800000, -37.800000) {\hbox{\tate ふもと}};
\node[Meaning] at (-4.750000, -36.150000) {foothills};
\node[Kanji] at (-2.700000, -37.400000) {\textcolor[HTML]{408dba}{冥}};
\node[Square] at (-2.700000, -37.900000) {};
\node[Onyomi] at (-2.650000, -37.800000) {\hbox{\tate メイ・ミョウ}};
\node[Meaning] at (-2.700000, -36.150000) {dark};
\node[Kanji] at (-0.650000, -37.400000) {\textcolor[HTML]{408dba}{挫}};
\node[Square] at (-0.650000, -37.900000) {};
\node[Onyomi] at (-0.600000, -37.800000) {\hbox{\tate ザ}};
\node[Meaning] at (-0.650000, -36.150000) {sprain};
\node[Kanji] at (1.400000, -37.400000) {\textcolor[HTML]{68a4bc}{遡}};
\node[Square] at (1.400000, -37.900000) {};
\node[Onyomi] at (1.450000, -37.800000) {\hbox{\tate ソ}};
\node[Kunyomi] at (1.350000, -37.800000) {\hbox{\tate さかのぼ.る}};
\node[Meaning] at (1.400000, -36.150000) {go upstream};
\node[Kanji] at (3.450000, -37.400000) {\textcolor[HTML]{a3bac2}{嘉}};
\node[Square] at (3.450000, -37.900000) {};
\node[Onyomi] at (3.500000, -37.800000) {\hbox{\tate カ}};
\node[Kunyomi] at (3.400000, -37.800000) {\hbox{\tate よい}};
\node[Meaning] at (3.450000, -36.150000) {esteem};
\node[Kanji] at (5.500000, -37.400000) {\textcolor[HTML]{1059be}{爽}};
\node[Square] at (5.500000, -37.900000) {};
\node[Onyomi] at (5.550000, -37.800000) {\hbox{\tate ソウ}};
\node[Kunyomi] at (5.450000, -37.800000) {\hbox{\tate さわ}};
\node[Meaning] at (5.500000, -36.150000) {refreshing};
\node[Kanji] at (7.550000, -37.400000) {\textcolor[HTML]{68a4bc}{勃}};
\node[Square] at (7.550000, -37.900000) {};
\node[Onyomi] at (7.600000, -37.800000) {\hbox{\tate ボツ}};
\node[Meaning] at (7.550000, -36.150000) {rise};
\node[Kanji] at (9.600000, -37.400000) {\textcolor[HTML]{68a4bc}{骸}};
\node[Square] at (9.600000, -37.900000) {};
\node[Onyomi] at (9.650000, -37.800000) {\hbox{\tate ガイ}};
\node[Meaning] at (9.600000, -36.150000) {dead remains};
\node[Kanji] at (11.650000, -37.400000) {\textcolor[HTML]{408dba}{隼}};
\node[Square] at (11.650000, -37.900000) {};
\node[Onyomi] at (11.700000, -37.800000) {\hbox{\tate シュン}};
\node[Kunyomi] at (11.600000, -37.800000) {\hbox{\tate はやぶさ}};
\node[Meaning] at (11.650000, -36.150000) {falcon};
\node[Kanji] at (13.700000, -37.400000) {\textcolor[HTML]{68a4bc}{戚}};
\node[Square] at (13.700000, -37.900000) {};
\node[Onyomi] at (13.750000, -37.800000) {\hbox{\tate セキ}};
\node[Meaning] at (13.700000, -36.150000) {grieve};
\node[Kanji] at (15.750000, -37.400000) {\textcolor[HTML]{1059be}{丼}};
\node[Square] at (15.750000, -37.900000) {};
\node[Onyomi] at (15.800000, -37.800000) {\hbox{\tate ドン}};
\node[Kunyomi] at (15.700000, -37.800000) {\hbox{\tate どんぶり}};
\node[Meaning] at (15.750000, -36.150000) {rice bowl};
\node[Kanji] at (17.800000, -37.400000) {\textcolor[HTML]{91b7c3}{畿}};
\node[Square] at (17.800000, -37.900000) {};
\node[Onyomi] at (17.850000, -37.800000) {\hbox{\tate キ}};
\node[Meaning] at (17.800000, -36.150000) {capital};
\node[Kanji] at (19.850000, -37.400000) {\textcolor[HTML]{a3bac2}{斐}};
\node[Square] at (19.850000, -37.900000) {};
\node[Onyomi] at (19.900000, -37.800000) {\hbox{\tate イ}};
\node[Meaning] at (19.850000, -36.150000) {patterned};
\node[Kanji] at (21.900000, -37.400000) {\textcolor[HTML]{91b7c3}{拳}};
\node[Square] at (21.900000, -37.900000) {};
\node[Onyomi] at (21.950000, -37.800000) {\hbox{\tate ケン}};
\node[Kunyomi] at (21.850000, -37.800000) {\hbox{\tate こぶし}};
\node[Meaning] at (21.900000, -36.150000) {fist};
\node[Kanji] at (23.950000, -37.400000) {\textcolor[HTML]{91b7c3}{亮}};
\node[Square] at (23.950000, -37.900000) {};
\node[Onyomi] at (24.000000, -37.800000) {\hbox{\tate リョウ}};
\node[Kunyomi] at (23.900000, -37.800000) {\hbox{\tate あきらか}};
\node[Meaning] at (23.950000, -36.150000) {clear};
\node[Kanji] at (26.000000, -37.400000) {\textcolor[HTML]{a3bac2}{阜}};
\node[Square] at (26.000000, -37.900000) {};
\node[Onyomi] at (26.050000, -37.800000) {\hbox{\tate フ}};
\node[Meaning] at (26.000000, -36.150000) {mound};
\node[Kanji] at (28.050000, -37.400000) {\textcolor[HTML]{91b7c3}{翔}};
\node[Square] at (28.050000, -37.900000) {};
\node[Onyomi] at (28.100000, -37.800000) {\hbox{\tate ショウ}};
\node[Kunyomi] at (28.000000, -37.800000) {\hbox{\tate かけ}};
\node[Meaning] at (28.050000, -36.150000) {fly};
\node[Kanji] at (30.100000, -37.400000) {\textcolor[HTML]{b0b0b5}{那}};
\node[Square] at (30.100000, -37.900000) {};
\node[Onyomi] at (30.150000, -37.800000) {\hbox{\tate ナ・ダ}};
\node[Kunyomi] at (30.050000, -37.800000) {\hbox{\tate いかん・なに}};
\node[Meaning] at (30.100000, -36.150000) {what};
\node[Kanji] at (32.150000, -37.400000) {\textcolor[HTML]{b0b0b5}{龍}};
\node[Square] at (32.150000, -37.900000) {};
\node[Onyomi] at (32.200000, -37.800000) {\hbox{\tate リュウ}};
\node[Kunyomi] at (32.100000, -37.800000) {\hbox{\tate たつ}};
\node[Meaning] at (32.150000, -36.150000) {imperial};
\node[Kanji] at (34.200000, -37.400000) {\textcolor[HTML]{29409e}{箋}};
\node[Square] at (34.200000, -37.900000) {};
\node[Onyomi] at (34.250000, -37.800000) {\hbox{\tate セン}};
\node[Meaning] at (34.200000, -36.150000) {paper};
\node[Kanji] at (36.250000, -37.400000) {\textcolor[HTML]{408dba}{彙}};
\node[Square] at (36.250000, -37.900000) {};
\node[Onyomi] at (36.300000, -37.800000) {\hbox{\tate イ}};
\node[Meaning] at (36.250000, -36.150000) {same kind};
\node[Kanji] at (38.300000, -37.400000) {\textcolor[HTML]{91b7c3}{籠}};
\node[Square] at (38.300000, -37.900000) {};
\node[Onyomi] at (38.350000, -37.800000) {\hbox{\tate ロウ}};
\node[Kunyomi] at (38.250000, -37.800000) {\hbox{\tate かご}};
\node[Meaning] at (38.300000, -36.150000) {basket};
\node[Kanji] at (40.350000, -37.400000) {\textcolor[HTML]{68a4bc}{瘍}};
\node[Square] at (40.350000, -37.900000) {};
\node[Onyomi] at (40.400000, -37.800000) {\hbox{\tate ヨウ}};
\node[Meaning] at (40.350000, -36.150000) {boil (medical)};
\node[Kanji] at (42.400000, -37.400000) {\textcolor[HTML]{408dba}{哺}};
\node[Square] at (42.400000, -37.900000) {};
\node[Onyomi] at (42.450000, -37.800000) {\hbox{\tate ホ}};
\node[Kunyomi] at (42.350000, -37.800000) {\hbox{\tate ほぐく・ふく}};
\node[Meaning] at (42.400000, -36.150000) {nurse};
\node[Kanji] at (44.450000, -37.400000) {\textcolor[HTML]{408dba}{惧}};
\node[Square] at (44.450000, -37.900000) {};
\node[Onyomi] at (44.500000, -37.800000) {\hbox{\tate グ}};
\node[Meaning] at (44.450000, -36.150000) {dread};
\node[Kanji] at (46.500000, -37.400000) {\textcolor[HTML]{408dba}{拉}};
\node[Square] at (46.500000, -37.900000) {};
\node[Onyomi] at (46.550000, -37.800000) {\hbox{\tate ラ}};
\node[Meaning] at (46.500000, -36.150000) {crush};
\node[Kanji] at (48.550000, -37.400000) {\textcolor[HTML]{1e76bb}{璧}};
\node[Square] at (48.550000, -37.900000) {};
\node[Onyomi] at (48.600000, -37.800000) {\hbox{\tate ヘキ}};
\node[Meaning] at (48.550000, -36.150000) {sphere};
\node[Kanji] at (50.600000, -37.400000) {\textcolor[HTML]{1e76bb}{鬱}};
\node[Square] at (50.600000, -37.900000) {};
\node[Onyomi] at (50.650000, -37.800000) {\hbox{\tate ウツ}};
\node[Meaning] at (50.600000, -36.150000) {gloom};
\node[Kanji] at (52.650000, -37.400000) {\textcolor[HTML]{1e76bb}{踪}};
\node[Square] at (52.650000, -37.900000) {};
\node[Onyomi] at (52.700000, -37.800000) {\hbox{\tate ソウ}};
\node[Meaning] at (52.650000, -36.150000) {remains};
\node[Kanji] at (54.700000, -37.400000) {\textcolor[HTML]{1e76bb}{喩}};
\node[Square] at (54.700000, -37.900000) {};
\node[Onyomi] at (54.750000, -37.800000) {\hbox{\tate ユ}};
\node[Meaning] at (54.700000, -36.150000) {metaphor};
\node[Kanji] at (56.750000, -37.400000) {\textcolor[HTML]{1e76bb}{緻}};
\node[Square] at (56.750000, -37.900000) {};
\node[Onyomi] at (56.800000, -37.800000) {\hbox{\tate チ}};
\node[Meaning] at (56.750000, -36.150000) {fine};
\node[Meaning] at (-58.050000, -37.300000) {98.51\%};
\node[Kanji] at (-56.000000, -39.450000) {\textcolor[HTML]{1059be}{墟}};
\node[Square] at (-56.000000, -39.950000) {};
\node[Onyomi] at (-55.950000, -39.850000) {\hbox{\tate キョ}};
\node[Meaning] at (-56.000000, -38.200000) {ruins};
\node[Kanji] at (-53.950000, -39.450000) {\textcolor[HTML]{1059be}{嗅}};
\node[Square] at (-53.950000, -39.950000) {};
\node[Onyomi] at (-53.900000, -39.850000) {\hbox{\tate キュウ}};
\node[Kunyomi] at (-54.000000, -39.850000) {\hbox{\tate か.ぐ}};
\node[Meaning] at (-53.950000, -38.200000) {smell};
\node[Kanji] at (-51.900000, -39.450000) {\textcolor[HTML]{1059be}{訃}};
\node[Square] at (-51.900000, -39.950000) {};
\node[Onyomi] at (-51.850000, -39.850000) {\hbox{\tate フ}};
\node[Meaning] at (-51.900000, -38.200000) {obituary};
\node[Kanji] at (-49.850000, -39.450000) {\textcolor[HTML]{1059be}{贅}};
\node[Square] at (-49.850000, -39.950000) {};
\node[Onyomi] at (-49.800000, -39.850000) {\hbox{\tate ゼイ}};
\node[Kunyomi] at (-49.900000, -39.850000) {\hbox{\tate いぼ}};
\node[Meaning] at (-49.850000, -38.200000) {luxury};
\node[Kanji] at (-47.800000, -39.450000) {\textcolor[HTML]{1059be}{諧}};
\node[Square] at (-47.800000, -39.950000) {};
\node[Onyomi] at (-47.750000, -39.850000) {\hbox{\tate カイ}};
\node[Meaning] at (-47.800000, -38.200000) {harmony};
\node[Kanji] at (-45.750000, -39.450000) {\textcolor[HTML]{29409e}{傲}};
\node[Square] at (-45.750000, -39.950000) {};
\node[Onyomi] at (-45.700000, -39.850000) {\hbox{\tate ゴウ}};
\node[Kunyomi] at (-45.800000, -39.850000) {\hbox{\tate あなど・おご}};
\node[Meaning] at (-45.750000, -38.200000) {proud};
\node[Kanji] at (-43.700000, -39.450000) {\textcolor[HTML]{29409e}{楷}};
\node[Square] at (-43.700000, -39.950000) {};
\node[Onyomi] at (-43.650000, -39.850000) {\hbox{\tate カイ}};
\node[Meaning] at (-43.700000, -38.200000) {printed style};
\node[Kanji] at (-41.650000, -39.450000) {\textcolor[HTML]{29409e}{恣}};
\node[Square] at (-41.650000, -39.950000) {};
\node[Onyomi] at (-41.600000, -39.850000) {\hbox{\tate シ}};
\node[Meaning] at (-41.650000, -38.200000) {selfish};
\node[Kanji] at (-39.600000, -39.450000) {\textcolor[HTML]{29409e}{貪}};
\node[Square] at (-39.600000, -39.950000) {};
\node[Onyomi] at (-39.550000, -39.850000) {\hbox{\tate ドン}};
\node[Kunyomi] at (-39.650000, -39.850000) {\hbox{\tate むさぼ.る}};
\node[Meaning] at (-39.600000, -38.200000) {covet};
\node[Kanji] at (-37.550000, -39.450000) {\textcolor[HTML]{29409e}{辣}};
\node[Square] at (-37.550000, -39.950000) {};
\node[Onyomi] at (-37.500000, -39.850000) {\hbox{\tate ラツ}};
\node[Meaning] at (-37.550000, -38.200000) {bitter};
\node[Kanji] at (-35.500000, -39.450000) {\textcolor[HTML]{29409e}{漣}};
\node[Square] at (-35.500000, -39.950000) {};
\node[Onyomi] at (-35.450000, -39.850000) {\hbox{\tate レン・ラン}};
\node[Kunyomi] at (-35.550000, -39.850000) {\hbox{\tate さざなみ}};
\node[Meaning] at (-35.500000, -38.200000) {ripples};
\node[Kanji] at (-33.450000, -39.450000) {\textcolor[HTML]{242e6c}{摯}};
\node[Square] at (-33.450000, -39.950000) {};
\node[Onyomi] at (-33.400000, -39.850000) {\hbox{\tate シ}};
\node[Meaning] at (-33.450000, -38.200000) {seriousness};
\node[Kanji] at (-31.400000, -39.450000) {\textcolor[HTML]{242e6c}{錮}};
\node[Square] at (-31.400000, -39.950000) {};
\node[Onyomi] at (-31.350000, -39.850000) {\hbox{\tate コ}};
\node[Meaning] at (-31.400000, -38.200000) {tie up};
\node[Kanji] at (-29.350000, -39.450000) {\textcolor[HTML]{242e6c}{羞}};
\node[Square] at (-29.350000, -39.950000) {};
\node[Onyomi] at (-29.300000, -39.850000) {\hbox{\tate シュウ}};
\node[Meaning] at (-29.350000, -38.200000) {feel ashamed};
\node[Kanji] at (-27.300000, -39.450000) {\textcolor[HTML]{181c43}{慄}};
\node[Square] at (-27.300000, -39.950000) {};
\node[Onyomi] at (-27.250000, -39.850000) {\hbox{\tate リツ}};
\node[Meaning] at (-27.300000, -38.200000) {fear};
\node[Kanji] at (-25.250000, -39.450000) {\textcolor[HTML]{181c43}{憬}};
\node[Square] at (-25.250000, -39.950000) {};
\node[Onyomi] at (-25.200000, -39.850000) {\hbox{\tate ケイ}};
\node[Meaning] at (-25.250000, -38.200000) {long for};
\node[Meaning] at (-58.050000, -39.350000) {98.52\%};
\node[Meaning] at (-58.050000, 41.200000) {1 - 56};
\node[Meaning] at (-58.050000, 39.150000) {57 - 112};
\node[Meaning] at (-58.050000, 37.100000) {113 - 168};
\node[Meaning] at (-58.050000, 35.050000) {169 - 224};
\node[Meaning] at (-58.050000, 33.000000) {225 - 280};
\node[Meaning] at (-58.050000, 30.950000) {281 - 336};
\node[Meaning] at (-58.050000, 28.900000) {337 - 392};
\node[Meaning] at (-58.050000, 26.850000) {393 - 448};
\node[Meaning] at (-58.050000, 24.800000) {449 - 504};
\node[Meaning] at (-58.050000, 22.750000) {505 - 560};
\node[Meaning] at (-58.050000, 20.700000) {561 - 616};
\node[Meaning] at (-58.050000, 18.650000) {617 - 672};
\node[Meaning] at (-58.050000, 16.600000) {673 - 728};
\node[Meaning] at (-58.050000, 14.550000) {729 - 784};
\node[Meaning] at (-58.050000, 12.500000) {785 - 840};
\node[Meaning] at (-58.050000, 10.450000) {841 - 896};
\node[Meaning] at (-58.050000, 8.400000) {897 - 952};
\node[Meaning] at (-58.050000, 6.350000) {953 - 1008};
\node[Meaning] at (-58.050000, 4.300000) {1009 - 1064};
\node[Meaning] at (-58.050000, 2.250000) {1065 - 1120};
\node[Meaning] at (-58.050000, 0.200000) {1121 - 1176};
\node[Meaning] at (-58.050000, -1.850000) {1177 - 1232};
\node[Meaning] at (-58.050000, -3.900000) {1233 - 1288};
\node[Meaning] at (-58.050000, -5.950000) {1289 - 1344};
\node[Meaning] at (-58.050000, -8.000000) {1345 - 1400};
\node[Meaning] at (-58.050000, -10.050000) {1401 - 1456};
\node[Meaning] at (-58.050000, -12.100000) {1457 - 1512};
\node[Meaning] at (-58.050000, -14.150000) {1513 - 1568};
\node[Meaning] at (-58.050000, -16.200000) {1569 - 1624};
\node[Meaning] at (-58.050000, -18.250000) {1625 - 1680};
\node[Meaning] at (-58.050000, -20.300000) {1681 - 1736};
\node[Meaning] at (-58.050000, -22.350000) {1737 - 1792};
\node[Meaning] at (-58.050000, -24.400000) {1793 - 1848};
\node[Meaning] at (-58.050000, -26.450000) {1849 - 1904};
\node[Meaning] at (-58.050000, -28.500000) {1905 - 1960};
\node[Meaning] at (-58.050000, -30.550000) {1961 - 2016};
\node[Meaning] at (-58.050000, -32.600000) {2017 - 2072};
\node[Meaning] at (-58.050000, -34.650000) {2073 - 2128};
\node[Meaning] at (-58.050000, -36.700000) {2129 - 2184};
\node[Meaning] at (-58.050000, -38.750000) {2185 - 2240};

\node [above right,outer sep=10pt,minimum width=\paperwidth,align=center] at (bottomleft) {
  2200 kanji covering 98.52\% of common Japanese text. Data from \url{https://www.wanikani.com} and \url{https://en.wikipedia.org/wiki/List_of_joyo_kanji}. Kanji colors using colormap Balance, most to least frequent: \textcolor[HTML]{3c0912}{█}\textcolor[HTML]{830e29}{█}\textcolor[HTML]{a11d25}{█}\textcolor[HTML]{b74029}{█}\textcolor[HTML]{cd8268}{█}\textcolor[HTML]{d69f8d}{█}\textcolor[HTML]{d2a293}{█}\textcolor[HTML]{c8a59d}{█}\textcolor[HTML]{a3bac2}{█}\textcolor[HTML]{91b7c3}{█}\textcolor[HTML]{68a4bc}{█}\textcolor[HTML]{408dba}{█}\textcolor[HTML]{1059be}{█}\textcolor[HTML]{29409e}{█}\textcolor[HTML]{242e6c}{█}\textcolor[HTML]{181c43}{█}
  Kanji data from \url{https://www.wanikani.com} and \url{https://en.wikipedia.org/wiki/List_of_joyo_kanji}.
};

\end{document}
