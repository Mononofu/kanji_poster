\documentclass[12pt, a0paper, landscape]{tikzposter}
\usepackage[utf8]{inputenc}
\usepackage{CJKutf8}
\usepackage{anyfontsize}
\usepackage{hyperref}

\urlstyle{same}

\usetheme{Simple}

\newcommand{\Size}{2.05cm}
\tikzset{Square/.style={
    inner sep=0pt,
    text width=\Size,
    text height=\Size,
    minimum size=\Size,
    draw=lightgray,
    ultra thin,
    align=center,
    }
}

\tikzset{Kanji/.style={
    inner sep=0pt,
    text width=\Size,
    text height=\Size,
    minimum size=\Size,
    font=\fontsize{36}{43},
    align=center,
    }
}

\tikzset{Onyomi/.style={
    inner sep=0pt,
    text width=\Size,
    text height=\Size,
    minimum size=\Size,
    font=\fontsize{8}{9},
    align=left,
    }
}

\tikzset{Kunyomi/.style={
    inner sep=0pt,
    text width=\Size,
    text height=\Size,
    minimum size=\Size,
    font=\fontsize{8}{9},
    align=right,
    }
}

\tikzset{Meaning/.style={
    inner sep=0pt,
    text width=\Size,
    text height=\Size,
    minimum size=\Size,
    font=\scriptsize,
    align=center,
    text=gray,
    }
}

\begin{document}


\begin{CJK}{UTF8}{min}
\node[Square] at (-56.500000, 40.000000) {};
\node[Kanji] at (-56.500000, 40.500000) {人};
\node[Onyomi] at (-56.450000, 40.100000) {ニン};
\node[Kunyomi] at (-56.550000, 40.100000) {ひと};
\node[Meaning] at (-56.500000, 41.750000) {person};
\node[Square] at (-54.450000, 40.000000) {};
\node[Kanji] at (-54.450000, 40.500000) {言};
\node[Onyomi] at (-54.400000, 40.100000) {ゲン};
\node[Kunyomi] at (-54.500000, 40.100000) {い.う};
\node[Meaning] at (-54.450000, 41.750000) {say};
\node[Square] at (-52.400000, 40.000000) {};
\node[Kanji] at (-52.400000, 40.500000) {見};
\node[Onyomi] at (-52.350000, 40.100000) {ケン};
\node[Kunyomi] at (-52.450000, 40.100000) {み};
\node[Meaning] at (-52.400000, 41.750000) {see};
\node[Square] at (-50.350000, 40.000000) {};
\node[Kanji] at (-50.350000, 40.500000) {一};
\node[Onyomi] at (-50.300000, 40.100000) {イチ};
\node[Kunyomi] at (-50.400000, 40.100000) {ひと.*};
\node[Meaning] at (-50.350000, 41.750000) {one};
\node[Square] at (-48.300000, 40.000000) {};
\node[Kanji] at (-48.300000, 40.500000) {日};
\node[Onyomi] at (-48.250000, 40.100000) {ニチ};
\node[Kunyomi] at (-48.350000, 40.100000) {ひ};
\node[Meaning] at (-48.300000, 41.750000) {sun};
\node[Square] at (-46.250000, 40.000000) {};
\node[Kanji] at (-46.250000, 40.500000) {出};
\node[Onyomi] at (-46.200000, 40.100000) {シュツ};
\node[Kunyomi] at (-46.300000, 40.100000) {で.る};
\node[Meaning] at (-46.250000, 41.750000) {exit};
\node[Square] at (-44.200000, 40.000000) {};
\node[Kanji] at (-44.200000, 40.500000) {大};
\node[Onyomi] at (-44.150000, 40.100000) {タイ};
\node[Kunyomi] at (-44.250000, 40.100000) {おお};
\node[Meaning] at (-44.200000, 41.750000) {big};
\node[Square] at (-42.150000, 40.000000) {};
\node[Kanji] at (-42.150000, 40.500000) {上};
\node[Onyomi] at (-42.100000, 40.100000) {ジョウ};
\node[Kunyomi] at (-42.200000, 40.100000) {うえ};
\node[Meaning] at (-42.150000, 41.750000) {above};
\node[Square] at (-40.100000, 40.000000) {};
\node[Kanji] at (-40.100000, 40.500000) {生};
\node[Onyomi] at (-40.050000, 40.100000) {セイ};
\node[Kunyomi] at (-40.150000, 40.100000) {い.きる};
\node[Meaning] at (-40.100000, 41.750000) {life};
\node[Square] at (-38.050000, 40.000000) {};
\node[Kanji] at (-38.050000, 40.500000) {中};
\node[Onyomi] at (-38.000000, 40.100000) {チュウ};
\node[Kunyomi] at (-38.100000, 40.100000) {なか};
\node[Meaning] at (-38.050000, 41.750000) {middle};
\node[Square] at (-36.000000, 40.000000) {};
\node[Kanji] at (-36.000000, 40.500000) {気};
\node[Onyomi] at (-35.950000, 40.100000) {キ};
\node[Kunyomi] at (-36.050000, 40.100000) {いき};
\node[Meaning] at (-36.000000, 41.750000) {energy};
\node[Square] at (-33.950000, 40.000000) {};
\node[Kanji] at (-33.950000, 40.500000) {何};
\node[Onyomi] at (-33.900000, 40.100000) {カ};
\node[Kunyomi] at (-34.000000, 40.100000) {なに};
\node[Meaning] at (-33.950000, 41.750000) {what};
\node[Square] at (-31.900000, 40.000000) {};
\node[Kanji] at (-31.900000, 40.500000) {子};
\node[Onyomi] at (-31.850000, 40.100000) {シ};
\node[Kunyomi] at (-31.950000, 40.100000) {こ};
\node[Meaning] at (-31.900000, 41.750000) {child};
\node[Square] at (-29.850000, 40.000000) {};
\node[Kanji] at (-29.850000, 40.500000) {行};
\node[Onyomi] at (-29.800000, 40.100000) {コウ};
\node[Kunyomi] at (-29.900000, 40.100000) {い.く};
\node[Meaning] at (-29.850000, 41.750000) {go};
\node[Square] at (-27.800000, 40.000000) {};
\node[Kanji] at (-27.800000, 40.500000) {分};
\node[Onyomi] at (-27.750000, 40.100000) {ブン};
\node[Kunyomi] at (-27.850000, 40.100000) {わ.かる};
\node[Meaning] at (-27.800000, 41.750000) {part};
\node[Square] at (-25.750000, 40.000000) {};
\node[Kanji] at (-25.750000, 40.500000) {思};
\node[Onyomi] at (-25.700000, 40.100000) {シ};
\node[Kunyomi] at (-25.800000, 40.100000) {おも.う};
\node[Meaning] at (-25.750000, 41.750000) {think};
\node[Square] at (-23.700000, 40.000000) {};
\node[Kanji] at (-23.700000, 40.500000) {手};
\node[Onyomi] at (-23.650000, 40.100000) {シュ};
\node[Kunyomi] at (-23.750000, 40.100000) {て};
\node[Meaning] at (-23.700000, 41.750000) {hand};
\node[Square] at (-21.650000, 40.000000) {};
\node[Kanji] at (-21.650000, 40.500000) {間};
\node[Onyomi] at (-21.600000, 40.100000) {カン};
\node[Kunyomi] at (-21.700000, 40.100000) {あいだ};
\node[Meaning] at (-21.650000, 41.750000) {interval};
\node[Square] at (-19.600000, 40.000000) {};
\node[Kanji] at (-19.600000, 40.500000) {話};
\node[Onyomi] at (-19.550000, 40.100000) {ワ};
\node[Kunyomi] at (-19.650000, 40.100000) {はな.す};
\node[Meaning] at (-19.600000, 41.750000) {talk};
\node[Square] at (-17.550000, 40.000000) {};
\node[Kanji] at (-17.550000, 40.500000) {年};
\node[Onyomi] at (-17.500000, 40.100000) {ネン};
\node[Kunyomi] at (-17.600000, 40.100000) {とし};
\node[Meaning] at (-17.550000, 41.750000) {year};
\node[Square] at (-15.500000, 40.000000) {};
\node[Kanji] at (-15.500000, 40.500000) {本};
\node[Onyomi] at (-15.450000, 40.100000) {ホン};
\node[Kunyomi] at (-15.550000, 40.100000) {もと};
\node[Meaning] at (-15.500000, 41.750000) {book};
\node[Square] at (-13.450000, 40.000000) {};
\node[Kanji] at (-13.450000, 40.500000) {目};
\node[Onyomi] at (-13.400000, 40.100000) {モク};
\node[Kunyomi] at (-13.500000, 40.100000) {め};
\node[Meaning] at (-13.450000, 41.750000) {eye};
\node[Square] at (-11.400000, 40.000000) {};
\node[Kanji] at (-11.400000, 40.500000) {聞};
\node[Onyomi] at (-11.350000, 40.100000) {ブン};
\node[Kunyomi] at (-11.450000, 40.100000) {き.く};
\node[Meaning] at (-11.400000, 41.750000) {hear};
\node[Square] at (-9.350000, 40.000000) {};
\node[Kanji] at (-9.350000, 40.500000) {入};
\node[Onyomi] at (-9.300000, 40.100000) {ニュウ};
\node[Kunyomi] at (-9.400000, 40.100000) {はい.る};
\node[Meaning] at (-9.350000, 41.750000) {enter};
\node[Square] at (-7.300000, 40.000000) {};
\node[Kanji] at (-7.300000, 40.500000) {二};
\node[Onyomi] at (-7.250000, 40.100000) {ニ};
\node[Kunyomi] at (-7.350000, 40.100000) {ふた.*};
\node[Meaning] at (-7.300000, 41.750000) {two};
\node[Square] at (-5.250000, 40.000000) {};
\node[Kanji] at (-5.250000, 40.500000) {前};
\node[Onyomi] at (-5.200000, 40.100000) {ゼン};
\node[Kunyomi] at (-5.300000, 40.100000) {まえ};
\node[Meaning] at (-5.250000, 41.750000) {front};
\node[Square] at (-3.200000, 40.000000) {};
\node[Kanji] at (-3.200000, 40.500000) {声};
\node[Onyomi] at (-3.150000, 40.100000) {セイ};
\node[Kunyomi] at (-3.250000, 40.100000) {こえ};
\node[Meaning] at (-3.200000, 41.750000) {voice};
\node[Square] at (-1.150000, 40.000000) {};
\node[Kanji] at (-1.150000, 40.500000) {先};
\node[Onyomi] at (-1.100000, 40.100000) {セン};
\node[Kunyomi] at (-1.200000, 40.100000) {さき};
\node[Meaning] at (-1.150000, 41.750000) {previous};
\node[Square] at (0.900000, 40.000000) {};
\node[Kanji] at (0.900000, 40.500000) {僕};
\node[Onyomi] at (0.950000, 40.100000) {ボク};
\node[Meaning] at (0.900000, 41.750000) {i};
\node[Square] at (2.950000, 40.000000) {};
\node[Kanji] at (2.950000, 40.500000) {来};
\node[Onyomi] at (3.000000, 40.100000) {ライ};
\node[Kunyomi] at (2.900000, 40.100000) {く.る};
\node[Meaning] at (2.950000, 41.750000) {come};
\node[Square] at (5.000000, 40.000000) {};
\node[Kanji] at (5.000000, 40.500000) {自};
\node[Onyomi] at (5.050000, 40.100000) {ジ};
\node[Meaning] at (5.000000, 41.750000) {self};
\node[Square] at (7.050000, 40.000000) {};
\node[Kanji] at (7.050000, 40.500000) {時};
\node[Onyomi] at (7.100000, 40.100000) {ジ};
\node[Kunyomi] at (7.000000, 40.100000) {とき};
\node[Meaning] at (7.050000, 41.750000) {time};
\node[Square] at (9.100000, 40.000000) {};
\node[Kanji] at (9.100000, 40.500000) {事};
\node[Onyomi] at (9.150000, 40.100000) {ジ};
\node[Kunyomi] at (9.050000, 40.100000) {こと};
\node[Meaning] at (9.100000, 41.750000) {action};
\node[Square] at (11.150000, 40.000000) {};
\node[Kanji] at (11.150000, 40.500000) {下};
\node[Onyomi] at (11.200000, 40.100000) {カ};
\node[Kunyomi] at (11.100000, 40.100000) {した};
\node[Meaning] at (11.150000, 41.750000) {below};
\node[Square] at (13.200000, 40.000000) {};
\node[Kanji] at (13.200000, 40.500000) {物};
\node[Onyomi] at (13.250000, 40.100000) {ブツ};
\node[Kunyomi] at (13.150000, 40.100000) {もの};
\node[Meaning] at (13.200000, 41.750000) {thing};
\node[Square] at (15.250000, 40.000000) {};
\node[Kanji] at (15.250000, 40.500000) {会};
\node[Onyomi] at (15.300000, 40.100000) {カイ};
\node[Kunyomi] at (15.200000, 40.100000) {あ.う};
\node[Meaning] at (15.250000, 41.750000) {meet};
\node[Square] at (17.300000, 40.000000) {};
\node[Kanji] at (17.300000, 40.500000) {立};
\node[Onyomi] at (17.350000, 40.100000) {リツ};
\node[Kunyomi] at (17.250000, 40.100000) {た.つ};
\node[Meaning] at (17.300000, 41.750000) {stand};
\node[Square] at (19.350000, 40.000000) {};
\node[Kanji] at (19.350000, 40.500000) {魔};
\node[Onyomi] at (19.400000, 40.100000) {マ};
\node[Meaning] at (19.350000, 41.750000) {devil};
\node[Square] at (21.400000, 40.000000) {};
\node[Kanji] at (21.400000, 40.500000) {学};
\node[Onyomi] at (21.450000, 40.100000) {ガク};
\node[Kunyomi] at (21.350000, 40.100000) {まな.ぶ};
\node[Meaning] at (21.400000, 41.750000) {study};
\node[Square] at (23.450000, 40.000000) {};
\node[Kanji] at (23.450000, 40.500000) {法};
\node[Onyomi] at (23.500000, 40.100000) {ホウ};
\node[Meaning] at (23.450000, 41.750000) {method};
\node[Square] at (25.500000, 40.000000) {};
\node[Kanji] at (25.500000, 40.500000) {月};
\node[Onyomi] at (25.550000, 40.100000) {ゲツ};
\node[Kunyomi] at (25.450000, 40.100000) {つき};
\node[Meaning] at (25.500000, 41.750000) {moon};
\node[Square] at (27.550000, 40.000000) {};
\node[Kanji] at (27.550000, 40.500000) {顔};
\node[Onyomi] at (27.600000, 40.100000) {ガン};
\node[Kunyomi] at (27.500000, 40.100000) {かお};
\node[Meaning] at (27.550000, 41.750000) {face};
\node[Square] at (29.600000, 40.000000) {};
\node[Kanji] at (29.600000, 40.500000) {私};
\node[Onyomi] at (29.650000, 40.100000) {シ};
\node[Kunyomi] at (29.550000, 40.100000) {わたし};
\node[Meaning] at (29.600000, 41.750000) {i};
\node[Square] at (31.650000, 40.000000) {};
\node[Kanji] at (31.650000, 40.500000) {部};
\node[Onyomi] at (31.700000, 40.100000) {ブ};
\node[Kunyomi] at (31.600000, 40.100000) {へ};
\node[Meaning] at (31.650000, 41.750000) {part};
\node[Square] at (33.700000, 40.000000) {};
\node[Kanji] at (33.700000, 40.500000) {使};
\node[Onyomi] at (33.750000, 40.100000) {シ};
\node[Kunyomi] at (33.650000, 40.100000) {つか.う};
\node[Meaning] at (33.700000, 41.750000) {use};
\node[Square] at (35.750000, 40.000000) {};
\node[Kanji] at (35.750000, 40.500000) {合};
\node[Onyomi] at (35.800000, 40.100000) {ゴウ};
\node[Kunyomi] at (35.700000, 40.100000) {あ};
\node[Meaning] at (35.750000, 41.750000) {suit};
\node[Square] at (37.800000, 40.000000) {};
\node[Kanji] at (37.800000, 40.500000) {今};
\node[Onyomi] at (37.850000, 40.100000) {コン};
\node[Kunyomi] at (37.750000, 40.100000) {いま};
\node[Meaning] at (37.800000, 41.750000) {now};
\node[Square] at (39.850000, 40.000000) {};
\node[Kanji] at (39.850000, 40.500000) {知};
\node[Onyomi] at (39.900000, 40.100000) {チ};
\node[Kunyomi] at (39.800000, 40.100000) {し.る};
\node[Meaning] at (39.850000, 41.750000) {know};
\node[Square] at (41.900000, 40.000000) {};
\node[Kanji] at (41.900000, 40.500000) {後};
\node[Onyomi] at (41.950000, 40.100000) {ゴ};
\node[Kunyomi] at (41.850000, 40.100000) {うし.ろ};
\node[Meaning] at (41.900000, 41.750000) {behind};
\node[Square] at (43.950000, 40.000000) {};
\node[Kanji] at (43.950000, 40.500000) {国};
\node[Onyomi] at (44.000000, 40.100000) {コク};
\node[Kunyomi] at (43.900000, 40.100000) {くに};
\node[Meaning] at (43.950000, 41.750000) {country};
\node[Square] at (46.000000, 40.000000) {};
\node[Kanji] at (46.000000, 40.500000) {取};
\node[Onyomi] at (46.050000, 40.100000) {シュ};
\node[Kunyomi] at (45.950000, 40.100000) {と};
\node[Meaning] at (46.000000, 41.750000) {take};
\node[Square] at (48.050000, 40.000000) {};
\node[Kanji] at (48.050000, 40.500000) {小};
\node[Onyomi] at (48.100000, 40.100000) {ショウ};
\node[Kunyomi] at (48.000000, 40.100000) {ちい.さい};
\node[Meaning] at (48.050000, 41.750000) {small};
\node[Square] at (50.100000, 40.000000) {};
\node[Kanji] at (50.100000, 40.500000) {向};
\node[Onyomi] at (50.150000, 40.100000) {コウ};
\node[Kunyomi] at (50.050000, 40.100000) {む.き};
\node[Meaning] at (50.100000, 41.750000) {yonder};
\node[Square] at (52.150000, 40.000000) {};
\node[Kanji] at (52.150000, 40.500000) {明};
\node[Onyomi] at (52.200000, 40.100000) {メイ};
\node[Kunyomi] at (52.100000, 40.100000) {あ};
\node[Meaning] at (52.150000, 41.750000) {bright};
\node[Square] at (54.200000, 40.000000) {};
\node[Kanji] at (54.200000, 40.500000) {考};
\node[Onyomi] at (54.250000, 40.100000) {コウ};
\node[Kunyomi] at (54.150000, 40.100000) {かんが};
\node[Meaning] at (54.200000, 41.750000) {think};
\node[Square] at (56.250000, 40.000000) {};
\node[Kanji] at (56.250000, 40.500000) {場};
\node[Onyomi] at (56.300000, 40.100000) {ジョウ};
\node[Kunyomi] at (56.200000, 40.100000) {ば};
\node[Meaning] at (56.250000, 41.750000) {location};
\node[Meaning] at (-58.500000, 40.550000) {32.59\%};
\node[Square] at (-56.500000, 37.950000) {};
\node[Kanji] at (-56.500000, 38.450000) {込};
\node[Kunyomi] at (-56.550000, 38.050000) {こ};
\node[Meaning] at (-56.500000, 39.700000) {crowded};
\node[Square] at (-54.450000, 37.950000) {};
\node[Kanji] at (-54.450000, 38.450000) {方};
\node[Onyomi] at (-54.400000, 38.050000) {ホウ};
\node[Kunyomi] at (-54.500000, 38.050000) {かた};
\node[Meaning] at (-54.450000, 39.700000) {direction};
\node[Square] at (-52.400000, 37.950000) {};
\node[Kanji] at (-52.400000, 38.450000) {家};
\node[Onyomi] at (-52.350000, 38.050000) {カ};
\node[Kunyomi] at (-52.450000, 38.050000) {いえ};
\node[Meaning] at (-52.400000, 39.700000) {house};
\node[Square] at (-50.350000, 37.950000) {};
\node[Kanji] at (-50.350000, 38.450000) {君};
\node[Onyomi] at (-50.300000, 38.050000) {クン};
\node[Kunyomi] at (-50.400000, 38.050000) {きみ};
\node[Meaning] at (-50.350000, 39.700000) {buddy};
\node[Square] at (-48.300000, 37.950000) {};
\node[Kanji] at (-48.300000, 38.450000) {全};
\node[Onyomi] at (-48.250000, 38.050000) {ゼン};
\node[Kunyomi] at (-48.350000, 38.050000) {すべ.て};
\node[Meaning] at (-48.300000, 39.700000) {all};
\node[Square] at (-46.250000, 37.950000) {};
\node[Kanji] at (-46.250000, 38.450000) {動};
\node[Onyomi] at (-46.200000, 38.050000) {ドウ};
\node[Kunyomi] at (-46.300000, 38.050000) {うご.*};
\node[Meaning] at (-46.250000, 39.700000) {move};
\node[Square] at (-44.200000, 37.950000) {};
\node[Kanji] at (-44.200000, 38.450000) {持};
\node[Onyomi] at (-44.150000, 38.050000) {ジ};
\node[Kunyomi] at (-44.250000, 38.050000) {も};
\node[Meaning] at (-44.200000, 39.700000) {hold};
\node[Square] at (-42.150000, 37.950000) {};
\node[Kanji] at (-42.150000, 38.450000) {地};
\node[Onyomi] at (-42.100000, 38.050000) {チ};
\node[Meaning] at (-42.150000, 39.700000) {earth};
\node[Square] at (-40.100000, 37.950000) {};
\node[Kanji] at (-40.100000, 38.450000) {飛};
\node[Onyomi] at (-40.050000, 38.050000) {ヒ};
\node[Kunyomi] at (-40.150000, 38.050000) {と};
\node[Meaning] at (-40.100000, 39.700000) {fly};
\node[Square] at (-38.050000, 37.950000) {};
\node[Kanji] at (-38.050000, 38.450000) {屋};
\node[Onyomi] at (-38.000000, 38.050000) {オク};
\node[Kunyomi] at (-38.100000, 38.050000) {や};
\node[Meaning] at (-38.050000, 39.700000) {roof};
\node[Square] at (-36.000000, 37.950000) {};
\node[Kanji] at (-36.000000, 38.450000) {食};
\node[Onyomi] at (-35.950000, 38.050000) {ショク};
\node[Kunyomi] at (-36.050000, 38.050000) {た.べる};
\node[Meaning] at (-36.000000, 39.700000) {eat};
\node[Square] at (-33.950000, 37.950000) {};
\node[Kanji] at (-33.950000, 38.450000) {高};
\node[Onyomi] at (-33.900000, 38.050000) {コウ};
\node[Kunyomi] at (-34.000000, 38.050000) {たか.い};
\node[Meaning] at (-33.950000, 39.700000) {tall};
\node[Square] at (-31.900000, 37.950000) {};
\node[Kanji] at (-31.900000, 38.450000) {三};
\node[Onyomi] at (-31.850000, 38.050000) {サン};
\node[Kunyomi] at (-31.950000, 38.050000) {み.*};
\node[Meaning] at (-31.900000, 39.700000) {three};
\node[Square] at (-29.850000, 37.950000) {};
\node[Kanji] at (-29.850000, 38.450000) {所};
\node[Onyomi] at (-29.800000, 38.050000) {ショ};
\node[Kunyomi] at (-29.900000, 38.050000) {ところ};
\node[Meaning] at (-29.850000, 39.700000) {place};
\node[Square] at (-27.800000, 37.950000) {};
\node[Kanji] at (-27.800000, 38.450000) {回};
\node[Onyomi] at (-27.750000, 38.050000) {カイ};
\node[Kunyomi] at (-27.850000, 38.050000) {まわ.*};
\node[Meaning] at (-27.800000, 39.700000) {times};
\node[Square] at (-25.750000, 37.950000) {};
\node[Kanji] at (-25.750000, 38.450000) {誰};
\node[Kunyomi] at (-25.800000, 38.050000) {だれ};
\node[Meaning] at (-25.750000, 39.700000) {who};
\node[Square] at (-23.700000, 37.950000) {};
\node[Kanji] at (-23.700000, 38.450000) {度};
\node[Onyomi] at (-23.650000, 38.050000) {ド};
\node[Kunyomi] at (-23.750000, 38.050000) {たび};
\node[Meaning] at (-23.700000, 39.700000) {degrees};
\node[Square] at (-21.650000, 37.950000) {};
\node[Kanji] at (-21.650000, 38.450000) {女};
\node[Onyomi] at (-21.600000, 38.050000) {ジョ};
\node[Kunyomi] at (-21.700000, 38.050000) {おんな};
\node[Meaning] at (-21.650000, 39.700000) {woman};
\node[Square] at (-19.600000, 37.950000) {};
\node[Kanji] at (-19.600000, 38.450000) {心};
\node[Onyomi] at (-19.550000, 38.050000) {シン};
\node[Kunyomi] at (-19.650000, 38.050000) {こころ};
\node[Meaning] at (-19.600000, 39.700000) {heart};
\node[Square] at (-17.550000, 37.950000) {};
\node[Kanji] at (-17.550000, 38.450000) {口};
\node[Onyomi] at (-17.500000, 38.050000) {コウ};
\node[Kunyomi] at (-17.600000, 38.050000) {くち};
\node[Meaning] at (-17.550000, 39.700000) {mouth};
\node[Square] at (-15.500000, 37.950000) {};
\node[Kanji] at (-15.500000, 38.450000) {最};
\node[Onyomi] at (-15.450000, 38.050000) {サイ};
\node[Kunyomi] at (-15.550000, 38.050000) {もっと};
\node[Meaning] at (-15.500000, 39.700000) {most};
\node[Square] at (-13.450000, 37.950000) {};
\node[Kanji] at (-13.450000, 38.450000) {開};
\node[Onyomi] at (-13.400000, 38.050000) {カイ};
\node[Kunyomi] at (-13.500000, 38.050000) {あ.ける};
\node[Meaning] at (-13.450000, 39.700000) {open};
\node[Square] at (-11.400000, 37.950000) {};
\node[Kanji] at (-11.400000, 38.450000) {死};
\node[Onyomi] at (-11.350000, 38.050000) {シ};
\node[Kunyomi] at (-11.450000, 38.050000) {し.ぬ};
\node[Meaning] at (-11.400000, 39.700000) {death};
\node[Square] at (-9.350000, 37.950000) {};
\node[Kanji] at (-9.350000, 38.450000) {少};
\node[Onyomi] at (-9.300000, 38.050000) {ショウ};
\node[Kunyomi] at (-9.400000, 38.050000) {すこ.し};
\node[Meaning] at (-9.350000, 39.700000) {few};
\node[Square] at (-7.300000, 37.950000) {};
\node[Kanji] at (-7.300000, 38.450000) {者};
\node[Onyomi] at (-7.250000, 38.050000) {シャ};
\node[Kunyomi] at (-7.350000, 38.050000) {もの};
\node[Meaning] at (-7.300000, 39.700000) {someone};
\node[Square] at (-5.250000, 37.950000) {};
\node[Kanji] at (-5.250000, 38.450000) {長};
\node[Onyomi] at (-5.200000, 38.050000) {チョウ};
\node[Kunyomi] at (-5.300000, 38.050000) {なが.い};
\node[Meaning] at (-5.250000, 39.700000) {long};
\node[Square] at (-3.200000, 37.950000) {};
\node[Kanji] at (-3.200000, 38.450000) {校};
\node[Onyomi] at (-3.150000, 38.050000) {コウ};
\node[Meaning] at (-3.200000, 39.700000) {school};
\node[Square] at (-1.150000, 37.950000) {};
\node[Kanji] at (-1.150000, 38.450000) {通};
\node[Onyomi] at (-1.100000, 38.050000) {ツウ};
\node[Kunyomi] at (-1.200000, 38.050000) {とお.る};
\node[Meaning] at (-1.150000, 39.700000) {pass through};
\node[Square] at (0.900000, 37.950000) {};
\node[Kanji] at (0.900000, 38.450000) {近};
\node[Onyomi] at (0.950000, 38.050000) {キン};
\node[Kunyomi] at (0.850000, 38.050000) {ちか.い};
\node[Meaning] at (0.900000, 39.700000) {near};
\node[Square] at (2.950000, 37.950000) {};
\node[Kanji] at (2.950000, 38.450000) {教};
\node[Onyomi] at (3.000000, 38.050000) {キョウ};
\node[Kunyomi] at (2.900000, 38.050000) {おし.える};
\node[Meaning] at (2.950000, 39.700000) {teach};
\node[Square] at (5.000000, 37.950000) {};
\node[Kanji] at (5.000000, 38.450000) {体};
\node[Onyomi] at (5.050000, 38.050000) {タイ};
\node[Kunyomi] at (4.950000, 38.050000) {からだ};
\node[Meaning] at (5.000000, 39.700000) {body};
\node[Square] at (7.050000, 37.950000) {};
\node[Kanji] at (7.050000, 38.450000) {金};
\node[Onyomi] at (7.100000, 38.050000) {キン};
\node[Kunyomi] at (7.000000, 38.050000) {かね};
\node[Meaning] at (7.050000, 39.700000) {gold};
\node[Square] at (9.100000, 37.950000) {};
\node[Kanji] at (9.100000, 38.450000) {着};
\node[Onyomi] at (9.150000, 38.050000) {チャク};
\node[Kunyomi] at (9.050000, 38.050000) {き};
\node[Meaning] at (9.100000, 39.700000) {wear};
\node[Square] at (11.150000, 37.950000) {};
\node[Kanji] at (11.150000, 38.450000) {新};
\node[Onyomi] at (11.200000, 38.050000) {シン};
\node[Kunyomi] at (11.100000, 38.050000) {あたら.しい};
\node[Meaning] at (11.150000, 39.700000) {new};
\node[Square] at (13.200000, 37.950000) {};
\node[Kanji] at (13.200000, 38.450000) {笑};
\node[Onyomi] at (13.250000, 38.050000) {ショウ};
\node[Kunyomi] at (13.150000, 38.050000) {わら};
\node[Meaning] at (13.200000, 39.700000) {laugh};
\node[Square] at (15.250000, 37.950000) {};
\node[Kanji] at (15.250000, 38.450000) {続};
\node[Onyomi] at (15.300000, 38.050000) {ゾク};
\node[Kunyomi] at (15.200000, 38.050000) {つづ.く};
\node[Meaning] at (15.250000, 39.700000) {continue};
\node[Square] at (17.300000, 37.950000) {};
\node[Kanji] at (17.300000, 38.450000) {力};
\node[Onyomi] at (17.350000, 38.050000) {リョク};
\node[Kunyomi] at (17.250000, 38.050000) {ちから};
\node[Meaning] at (17.300000, 39.700000) {power};
\node[Square] at (19.350000, 37.950000) {};
\node[Kanji] at (19.350000, 38.450000) {作};
\node[Onyomi] at (19.400000, 38.050000) {サク};
\node[Kunyomi] at (19.300000, 38.050000) {つく.る};
\node[Meaning] at (19.350000, 39.700000) {make};
\node[Square] at (21.400000, 37.950000) {};
\node[Kanji] at (21.400000, 38.450000) {同};
\node[Onyomi] at (21.450000, 38.050000) {ドウ};
\node[Kunyomi] at (21.350000, 38.050000) {おな.じ};
\node[Meaning] at (21.400000, 39.700000) {same};
\node[Square] at (23.450000, 37.950000) {};
\node[Kanji] at (23.450000, 38.450000) {東};
\node[Onyomi] at (23.500000, 38.050000) {トウ};
\node[Kunyomi] at (23.400000, 38.050000) {ひがし};
\node[Meaning] at (23.450000, 39.700000) {east};
\node[Square] at (25.500000, 37.950000) {};
\node[Kanji] at (25.500000, 38.450000) {以};
\node[Onyomi] at (25.550000, 38.050000) {イ};
\node[Meaning] at (25.500000, 39.700000) {by means of};
\node[Square] at (27.550000, 37.950000) {};
\node[Kanji] at (27.550000, 38.450000) {々};
\node[Onyomi] at (27.600000, 38.050000) {ノマ};
\node[Kunyomi] at (27.500000, 38.050000) {のま};
\node[Meaning] at (27.550000, 39.700000) {repeater};
\node[Square] at (29.600000, 37.950000) {};
\node[Kanji] at (29.600000, 38.450000) {車};
\node[Onyomi] at (29.650000, 38.050000) {シャ};
\node[Kunyomi] at (29.550000, 38.050000) {くるま};
\node[Meaning] at (29.600000, 39.700000) {car};
\node[Square] at (31.650000, 37.950000) {};
\node[Kanji] at (31.650000, 38.450000) {十};
\node[Onyomi] at (31.700000, 38.050000) {ジュウ};
\node[Kunyomi] at (31.600000, 38.050000) {とお.*};
\node[Meaning] at (31.650000, 39.700000) {ten};
\node[Square] at (33.700000, 37.950000) {};
\node[Kanji] at (33.700000, 38.450000) {切};
\node[Onyomi] at (33.750000, 38.050000) {セツ};
\node[Kunyomi] at (33.650000, 38.050000) {き.る};
\node[Meaning] at (33.700000, 39.700000) {cut};
\node[Square] at (35.750000, 37.950000) {};
\node[Kanji] at (35.750000, 38.450000) {約};
\node[Onyomi] at (35.800000, 38.050000) {ヤク};
\node[Meaning] at (35.750000, 39.700000) {promise};
\node[Square] at (37.800000, 37.950000) {};
\node[Kanji] at (37.800000, 38.450000) {書};
\node[Onyomi] at (37.850000, 38.050000) {ショ};
\node[Kunyomi] at (37.750000, 38.050000) {か.く};
\node[Meaning] at (37.800000, 39.700000) {write};
\node[Square] at (39.850000, 37.950000) {};
\node[Kanji] at (39.850000, 38.450000) {空};
\node[Onyomi] at (39.900000, 38.050000) {クウ};
\node[Kunyomi] at (39.800000, 38.050000) {そら};
\node[Meaning] at (39.850000, 39.700000) {sky};
\node[Square] at (41.900000, 37.950000) {};
\node[Kanji] at (41.900000, 38.450000) {感};
\node[Onyomi] at (41.950000, 38.050000) {カン};
\node[Meaning] at (41.900000, 39.700000) {feeling};
\node[Square] at (43.950000, 37.950000) {};
\node[Kanji] at (43.950000, 38.450000) {真};
\node[Onyomi] at (44.000000, 38.050000) {シン};
\node[Kunyomi] at (43.900000, 38.050000) {ま};
\node[Meaning] at (43.950000, 39.700000) {reality};
\node[Square] at (46.000000, 37.950000) {};
\node[Kanji] at (46.000000, 38.450000) {発};
\node[Onyomi] at (46.050000, 38.050000) {ハツ};
\node[Meaning] at (46.000000, 39.700000) {departure};
\node[Square] at (48.050000, 37.950000) {};
\node[Kanji] at (48.050000, 38.450000) {引};
\node[Onyomi] at (48.100000, 38.050000) {イン};
\node[Kunyomi] at (48.000000, 38.050000) {ひ};
\node[Meaning] at (48.050000, 39.700000) {pull};
\node[Square] at (50.100000, 37.950000) {};
\node[Kanji] at (50.100000, 38.450000) {当};
\node[Onyomi] at (50.150000, 38.050000) {トウ};
\node[Kunyomi] at (50.050000, 38.050000) {あ.たる};
\node[Meaning] at (50.100000, 39.700000) {right};
\node[Square] at (52.150000, 37.950000) {};
\node[Kanji] at (52.150000, 38.450000) {戻};
\node[Onyomi] at (52.200000, 38.050000) {レイ};
\node[Kunyomi] at (52.100000, 38.050000) {もど};
\node[Meaning] at (52.150000, 39.700000) {return};
\node[Square] at (54.200000, 37.950000) {};
\node[Kanji] at (54.200000, 38.450000) {色};
\node[Onyomi] at (54.250000, 38.050000) {シキ};
\node[Kunyomi] at (54.150000, 38.050000) {いろ};
\node[Meaning] at (54.200000, 39.700000) {color};
\node[Square] at (56.250000, 37.950000) {};
\node[Kanji] at (56.250000, 38.450000) {調};
\node[Onyomi] at (56.300000, 38.050000) {チョウ};
\node[Kunyomi] at (56.200000, 38.050000) {しら.べる};
\node[Meaning] at (56.250000, 39.700000) {investigate};
\node[Meaning] at (-58.500000, 38.500000) {45.39\%};
\node[Square] at (-56.500000, 35.900000) {};
\node[Kanji] at (-56.500000, 36.400000) {理};
\node[Onyomi] at (-56.450000, 36.000000) {リ};
\node[Kunyomi] at (-56.550000, 36.000000) {ことわり};
\node[Meaning] at (-56.500000, 37.650000) {reason};
\node[Square] at (-54.450000, 35.900000) {};
\node[Kanji] at (-54.450000, 36.400000) {返};
\node[Onyomi] at (-54.400000, 36.000000) {ヘン};
\node[Kunyomi] at (-54.500000, 36.000000) {かえ.る};
\node[Meaning] at (-54.450000, 37.650000) {return};
\node[Square] at (-52.400000, 35.900000) {};
\node[Kanji] at (-52.400000, 36.400000) {音};
\node[Onyomi] at (-52.350000, 36.000000) {オン};
\node[Kunyomi] at (-52.450000, 36.000000) {おと};
\node[Meaning] at (-52.400000, 37.650000) {sound};
\node[Square] at (-50.350000, 35.900000) {};
\node[Kanji] at (-50.350000, 36.400000) {違};
\node[Onyomi] at (-50.300000, 36.000000) {イ};
\node[Kunyomi] at (-50.400000, 36.000000) {ちが};
\node[Meaning] at (-50.350000, 37.650000) {different};
\node[Square] at (-48.300000, 35.900000) {};
\node[Kanji] at (-48.300000, 36.400000) {外};
\node[Onyomi] at (-48.250000, 36.000000) {ガイ};
\node[Kunyomi] at (-48.350000, 36.000000) {そと};
\node[Meaning] at (-48.300000, 37.650000) {outside};
\node[Square] at (-46.250000, 35.900000) {};
\node[Kanji] at (-46.250000, 36.400000) {京};
\node[Onyomi] at (-46.200000, 36.000000) {キョウ};
\node[Kunyomi] at (-46.300000, 36.000000) {みやこ};
\node[Meaning] at (-46.250000, 37.650000) {capital};
\node[Square] at (-44.200000, 35.900000) {};
\node[Kanji] at (-44.200000, 36.400000) {頭};
\node[Onyomi] at (-44.150000, 36.000000) {ズ};
\node[Kunyomi] at (-44.250000, 36.000000) {あたま};
\node[Meaning] at (-44.200000, 37.650000) {head};
\node[Square] at (-42.150000, 35.900000) {};
\node[Kanji] at (-42.150000, 36.400000) {足};
\node[Onyomi] at (-42.100000, 36.000000) {ソク};
\node[Kunyomi] at (-42.200000, 36.000000) {あし};
\node[Meaning] at (-42.150000, 37.650000) {foot};
\node[Square] at (-40.100000, 35.900000) {};
\node[Kanji] at (-40.100000, 36.400000) {電};
\node[Onyomi] at (-40.050000, 36.000000) {デン};
\node[Meaning] at (-40.100000, 37.650000) {electricity};
\node[Square] at (-38.050000, 35.900000) {};
\node[Kanji] at (-38.050000, 36.400000) {文};
\node[Onyomi] at (-38.000000, 36.000000) {ブン};
\node[Meaning] at (-38.050000, 37.650000) {writing};
\node[Square] at (-36.000000, 35.900000) {};
\node[Kanji] at (-36.000000, 36.400000) {急};
\node[Onyomi] at (-35.950000, 36.000000) {キュウ};
\node[Kunyomi] at (-36.050000, 36.000000) {いそ.ぐ};
\node[Meaning] at (-36.000000, 37.650000) {hurry};
\node[Square] at (-33.950000, 35.900000) {};
\node[Kanji] at (-33.950000, 36.400000) {落};
\node[Onyomi] at (-33.900000, 36.000000) {ラク};
\node[Kunyomi] at (-34.000000, 36.000000) {お.ちる};
\node[Meaning] at (-33.950000, 37.650000) {fall};
\node[Square] at (-31.900000, 35.900000) {};
\node[Kanji] at (-31.900000, 36.400000) {性};
\node[Onyomi] at (-31.850000, 36.000000) {セイ};
\node[Meaning] at (-31.900000, 37.650000) {gender};
\node[Square] at (-29.850000, 35.900000) {};
\node[Kanji] at (-29.850000, 36.400000) {振};
\node[Onyomi] at (-29.800000, 36.000000) {シン};
\node[Kunyomi] at (-29.900000, 36.000000) {ふ.る};
\node[Meaning] at (-29.850000, 37.650000) {shake};
\node[Square] at (-27.800000, 35.900000) {};
\node[Kanji] at (-27.800000, 36.400000) {変};
\node[Onyomi] at (-27.750000, 36.000000) {ヘン};
\node[Kunyomi] at (-27.850000, 36.000000) {か.*};
\node[Meaning] at (-27.800000, 37.650000) {change};
\node[Square] at (-25.750000, 35.900000) {};
\node[Kanji] at (-25.750000, 36.400000) {名};
\node[Onyomi] at (-25.700000, 36.000000) {メイ};
\node[Kunyomi] at (-25.800000, 36.000000) {な};
\node[Meaning] at (-25.750000, 37.650000) {name};
\node[Square] at (-23.700000, 35.900000) {};
\node[Kanji] at (-23.700000, 36.400000) {第};
\node[Onyomi] at (-23.650000, 36.000000) {ダイ};
\node[Meaning] at (-23.700000, 37.650000) {ordinal number};
\node[Square] at (-21.650000, 35.900000) {};
\node[Kanji] at (-21.650000, 36.400000) {社};
\node[Onyomi] at (-21.600000, 36.000000) {シャ};
\node[Kunyomi] at (-21.700000, 36.000000) {やしろ};
\node[Meaning] at (-21.650000, 37.650000) {company};
\node[Square] at (-19.600000, 35.900000) {};
\node[Kanji] at (-19.600000, 36.400000) {決};
\node[Onyomi] at (-19.550000, 36.000000) {ケツ};
\node[Kunyomi] at (-19.650000, 36.000000) {き.める};
\node[Meaning] at (-19.600000, 37.650000) {decide};
\node[Square] at (-17.550000, 35.900000) {};
\node[Kanji] at (-17.550000, 36.400000) {歩};
\node[Onyomi] at (-17.500000, 36.000000) {ホ};
\node[Kunyomi] at (-17.600000, 36.000000) {ある.く};
\node[Meaning] at (-17.550000, 37.650000) {walk};
\node[Square] at (-15.500000, 35.900000) {};
\node[Kanji] at (-15.500000, 36.400000) {意};
\node[Onyomi] at (-15.450000, 36.000000) {イ};
\node[Meaning] at (-15.500000, 37.650000) {idea};
\node[Square] at (-13.450000, 35.900000) {};
\node[Kanji] at (-13.450000, 36.400000) {水};
\node[Onyomi] at (-13.400000, 36.000000) {スイ};
\node[Kunyomi] at (-13.500000, 36.000000) {みず};
\node[Meaning] at (-13.450000, 37.650000) {water};
\node[Square] at (-11.400000, 35.900000) {};
\node[Kanji] at (-11.400000, 36.400000) {様};
\node[Onyomi] at (-11.350000, 36.000000) {ヨウ};
\node[Kunyomi] at (-11.450000, 36.000000) {さま};
\node[Meaning] at (-11.400000, 37.650000) {Mr., Mrs.};
\node[Square] at (-9.350000, 35.900000) {};
\node[Kanji] at (-9.350000, 36.400000) {起};
\node[Onyomi] at (-9.300000, 36.000000) {キ};
\node[Kunyomi] at (-9.400000, 36.000000) {お};
\node[Meaning] at (-9.350000, 37.650000) {wake up};
\node[Square] at (-7.300000, 35.900000) {};
\node[Kanji] at (-7.300000, 36.400000) {多};
\node[Onyomi] at (-7.250000, 36.000000) {タ};
\node[Kunyomi] at (-7.350000, 36.000000) {おお.い};
\node[Meaning] at (-7.300000, 37.650000) {many};
\node[Square] at (-5.250000, 35.900000) {};
\node[Kanji] at (-5.250000, 36.400000) {県};
\node[Onyomi] at (-5.200000, 36.000000) {ケン};
\node[Meaning] at (-5.250000, 37.650000) {prefecture};
\node[Square] at (-3.200000, 35.900000) {};
\node[Kanji] at (-3.200000, 36.400000) {初};
\node[Onyomi] at (-3.150000, 36.000000) {ショ};
\node[Kunyomi] at (-3.250000, 36.000000) {はじ};
\node[Meaning] at (-3.200000, 37.650000) {first};
\node[Square] at (-1.150000, 35.900000) {};
\node[Kanji] at (-1.150000, 36.400000) {対};
\node[Onyomi] at (-1.100000, 36.000000) {タイ};
\node[Meaning] at (-1.150000, 37.650000) {versus};
\node[Square] at (0.900000, 35.900000) {};
\node[Kanji] at (0.900000, 36.400000) {男};
\node[Onyomi] at (0.950000, 36.000000) {ダン};
\node[Kunyomi] at (0.850000, 36.000000) {おとこ};
\node[Meaning] at (0.900000, 37.650000) {man};
\node[Square] at (2.950000, 35.900000) {};
\node[Kanji] at (2.950000, 36.400000) {身};
\node[Onyomi] at (3.000000, 36.000000) {シン};
\node[Kunyomi] at (2.900000, 36.000000) {み};
\node[Meaning] at (2.950000, 37.650000) {somebody};
\node[Square] at (5.000000, 35.900000) {};
\node[Kanji] at (5.000000, 36.400000) {待};
\node[Onyomi] at (5.050000, 36.000000) {タイ};
\node[Kunyomi] at (4.950000, 36.000000) {ま};
\node[Meaning] at (5.000000, 37.650000) {wait};
\node[Square] at (7.050000, 35.900000) {};
\node[Kanji] at (7.050000, 36.400000) {始};
\node[Onyomi] at (7.100000, 36.000000) {シ};
\node[Kunyomi] at (7.000000, 36.000000) {はじ};
\node[Meaning] at (7.050000, 37.650000) {begin};
\node[Square] at (9.100000, 35.900000) {};
\node[Kanji] at (9.100000, 36.400000) {選};
\node[Onyomi] at (9.150000, 36.000000) {セン};
\node[Kunyomi] at (9.050000, 36.000000) {えら.ぶ};
\node[Meaning] at (9.100000, 37.650000) {choose};
\node[Square] at (11.150000, 35.900000) {};
\node[Kanji] at (11.150000, 36.400000) {道};
\node[Onyomi] at (11.200000, 36.000000) {ドウ};
\node[Kunyomi] at (11.100000, 36.000000) {みち};
\node[Meaning] at (11.150000, 37.650000) {road};
\node[Square] at (13.200000, 35.900000) {};
\node[Kanji] at (13.200000, 36.400000) {突};
\node[Onyomi] at (13.250000, 36.000000) {トツ};
\node[Kunyomi] at (13.150000, 36.000000) {つ.く};
\node[Meaning] at (13.200000, 37.650000) {stab};
\node[Square] at (15.250000, 35.900000) {};
\node[Kanji] at (15.250000, 36.400000) {答};
\node[Onyomi] at (15.300000, 36.000000) {トウ};
\node[Kunyomi] at (15.200000, 36.000000) {こた};
\node[Meaning] at (15.250000, 37.650000) {answer};
\node[Square] at (17.300000, 35.900000) {};
\node[Kanji] at (17.300000, 36.400000) {親};
\node[Onyomi] at (17.350000, 36.000000) {シン};
\node[Kunyomi] at (17.250000, 36.000000) {おや};
\node[Meaning] at (17.300000, 37.650000) {parent};
\node[Square] at (19.350000, 35.900000) {};
\node[Kanji] at (19.350000, 36.400000) {室};
\node[Onyomi] at (19.400000, 36.000000) {シツ};
\node[Meaning] at (19.350000, 37.650000) {room};
\node[Square] at (21.400000, 35.900000) {};
\node[Kanji] at (21.400000, 36.400000) {員};
\node[Onyomi] at (21.450000, 36.000000) {イン};
\node[Meaning] at (21.400000, 37.650000) {member};
\node[Square] at (23.450000, 35.900000) {};
\node[Kanji] at (23.450000, 36.400000) {必};
\node[Onyomi] at (23.500000, 36.000000) {ヒツ};
\node[Kunyomi] at (23.400000, 36.000000) {かなら.ず};
\node[Meaning] at (23.450000, 37.650000) {certain};
\node[Square] at (25.500000, 35.900000) {};
\node[Kanji] at (25.500000, 36.400000) {山};
\node[Onyomi] at (25.550000, 36.000000) {サン};
\node[Kunyomi] at (25.450000, 36.000000) {やま};
\node[Meaning] at (25.500000, 37.650000) {mountain};
\node[Square] at (27.550000, 35.900000) {};
\node[Kanji] at (27.550000, 36.400000) {不};
\node[Onyomi] at (27.600000, 36.000000) {フ};
\node[Meaning] at (27.550000, 37.650000) {not};
\node[Square] at (29.600000, 35.900000) {};
\node[Kanji] at (29.600000, 36.400000) {番};
\node[Onyomi] at (29.650000, 36.000000) {バン};
\node[Meaning] at (29.600000, 37.650000) {number (series)};
\node[Square] at (31.650000, 35.900000) {};
\node[Kanji] at (31.650000, 36.400000) {表};
\node[Onyomi] at (31.700000, 36.000000) {ヒョウ};
\node[Kunyomi] at (31.600000, 36.000000) {あらわ.す};
\node[Meaning] at (31.650000, 37.650000) {express};
\node[Square] at (33.700000, 35.900000) {};
\node[Kanji] at (33.700000, 36.400000) {機};
\node[Onyomi] at (33.750000, 36.000000) {キ};
\node[Kunyomi] at (33.650000, 36.000000) {はた};
\node[Meaning] at (33.700000, 37.650000) {machine};
\node[Square] at (35.750000, 35.900000) {};
\node[Kanji] at (35.750000, 36.400000) {俺};
\node[Kunyomi] at (35.700000, 36.000000) {おれ};
\node[Meaning] at (35.750000, 37.650000) {i};
\node[Square] at (37.800000, 35.900000) {};
\node[Kanji] at (37.800000, 36.400000) {問};
\node[Onyomi] at (37.850000, 36.000000) {モン};
\node[Kunyomi] at (37.750000, 36.000000) {と};
\node[Meaning] at (37.800000, 37.650000) {problem};
\node[Square] at (39.850000, 35.900000) {};
\node[Kanji] at (39.850000, 36.400000) {段};
\node[Onyomi] at (39.900000, 36.000000) {ダン};
\node[Meaning] at (39.850000, 37.650000) {steps};
\node[Square] at (41.900000, 35.900000) {};
\node[Kanji] at (41.900000, 36.400000) {張};
\node[Onyomi] at (41.950000, 36.000000) {チョウ};
\node[Kunyomi] at (41.850000, 36.000000) {は.る};
\node[Meaning] at (41.900000, 37.650000) {stretch};
\node[Square] at (43.950000, 35.900000) {};
\node[Kanji] at (43.950000, 36.400000) {安};
\node[Onyomi] at (44.000000, 36.000000) {アン};
\node[Kunyomi] at (43.900000, 36.000000) {やす.い};
\node[Meaning] at (43.950000, 37.650000) {relax};
\node[Square] at (46.000000, 35.900000) {};
\node[Kanji] at (46.000000, 36.400000) {緒};
\node[Onyomi] at (46.050000, 36.000000) {ショ};
\node[Meaning] at (46.000000, 37.650000) {together};
\node[Square] at (48.050000, 35.900000) {};
\node[Kanji] at (48.050000, 36.400000) {連};
\node[Onyomi] at (48.100000, 36.000000) {レン};
\node[Kunyomi] at (48.000000, 36.000000) {つ};
\node[Meaning] at (48.050000, 37.650000) {take along};
\node[Square] at (50.100000, 35.900000) {};
\node[Kanji] at (50.100000, 36.400000) {走};
\node[Onyomi] at (50.150000, 36.000000) {ソウ};
\node[Kunyomi] at (50.050000, 36.000000) {はし.る};
\node[Meaning] at (50.100000, 37.650000) {run};
\node[Square] at (52.150000, 35.900000) {};
\node[Kanji] at (52.150000, 36.400000) {叫};
\node[Onyomi] at (52.200000, 36.000000) {キョウ};
\node[Kunyomi] at (52.100000, 36.000000) {さけ.ぶ};
\node[Meaning] at (52.150000, 37.650000) {shout};
\node[Square] at (54.200000, 35.900000) {};
\node[Kanji] at (54.200000, 36.400000) {面};
\node[Onyomi] at (54.250000, 36.000000) {メン};
\node[Kunyomi] at (54.150000, 36.000000) {おも};
\node[Meaning] at (54.200000, 37.650000) {surface};
\node[Square] at (56.250000, 35.900000) {};
\node[Kanji] at (56.250000, 36.400000) {木};
\node[Onyomi] at (56.300000, 36.000000) {モク};
\node[Kunyomi] at (56.200000, 36.000000) {き};
\node[Meaning] at (56.250000, 37.650000) {tree};
\node[Meaning] at (-58.500000, 36.450000) {54.12\%};
\node[Square] at (-56.500000, 33.850000) {};
\node[Kanji] at (-56.500000, 34.350000) {無};
\node[Onyomi] at (-56.450000, 33.950000) {ム};
\node[Kunyomi] at (-56.550000, 33.950000) {な.い};
\node[Meaning] at (-56.500000, 35.600000) {nothing};
\node[Square] at (-54.450000, 33.850000) {};
\node[Kanji] at (-54.450000, 34.350000) {予};
\node[Onyomi] at (-54.400000, 33.950000) {ヨ};
\node[Kunyomi] at (-54.500000, 33.950000) {あらかじ};
\node[Meaning] at (-54.450000, 35.600000) {beforehand};
\node[Square] at (-52.400000, 33.850000) {};
\node[Kanji] at (-52.400000, 34.350000) {座};
\node[Onyomi] at (-52.350000, 33.950000) {ザ};
\node[Kunyomi] at (-52.450000, 33.950000) {すわ.る};
\node[Meaning] at (-52.400000, 35.600000) {sit};
\node[Square] at (-50.350000, 33.850000) {};
\node[Kanji] at (-50.350000, 34.350000) {要};
\node[Onyomi] at (-50.300000, 33.950000) {ヨウ};
\node[Kunyomi] at (-50.400000, 33.950000) {い};
\node[Meaning] at (-50.350000, 35.600000) {need};
\node[Square] at (-48.300000, 33.850000) {};
\node[Kanji] at (-48.300000, 34.350000) {光};
\node[Onyomi] at (-48.250000, 33.950000) {コウ};
\node[Kunyomi] at (-48.350000, 33.950000) {ひかり};
\node[Meaning] at (-48.300000, 35.600000) {sunlight};
\node[Square] at (-46.250000, 33.850000) {};
\node[Kanji] at (-46.250000, 34.350000) {悪};
\node[Onyomi] at (-46.200000, 33.950000) {アク};
\node[Kunyomi] at (-46.300000, 33.950000) {わる.い};
\node[Meaning] at (-46.250000, 35.600000) {bad};
\node[Square] at (-44.200000, 33.850000) {};
\node[Kanji] at (-44.200000, 34.350000) {市};
\node[Onyomi] at (-44.150000, 33.950000) {シ};
\node[Kunyomi] at (-44.250000, 33.950000) {いち};
\node[Meaning] at (-44.200000, 35.600000) {city};
\node[Square] at (-42.150000, 33.850000) {};
\node[Kanji] at (-42.150000, 34.350000) {呪};
\node[Kunyomi] at (-42.200000, 33.950000) {のろ};
\node[Meaning] at (-42.150000, 35.600000) {curse};
\node[Square] at (-40.100000, 33.850000) {};
\node[Kanji] at (-40.100000, 34.350000) {去};
\node[Onyomi] at (-40.050000, 33.950000) {キョ};
\node[Kunyomi] at (-40.150000, 33.950000) {さ.る};
\node[Meaning] at (-40.100000, 35.600000) {past};
\node[Square] at (-38.050000, 33.850000) {};
\node[Kanji] at (-38.050000, 34.350000) {世};
\node[Onyomi] at (-38.000000, 33.950000) {セ};
\node[Kunyomi] at (-38.100000, 33.950000) {よ};
\node[Meaning] at (-38.050000, 35.600000) {generation};
\node[Square] at (-36.000000, 33.850000) {};
\node[Kanji] at (-36.000000, 34.350000) {画};
\node[Onyomi] at (-35.950000, 33.950000) {ガ};
\node[Meaning] at (-36.000000, 35.600000) {drawing};
\node[Square] at (-33.950000, 33.850000) {};
\node[Kanji] at (-33.950000, 34.350000) {然};
\node[Onyomi] at (-33.900000, 33.950000) {ゼン};
\node[Kunyomi] at (-34.000000, 33.950000) {しか};
\node[Meaning] at (-33.950000, 35.600000) {nature};
\node[Square] at (-31.900000, 33.850000) {};
\node[Kanji] at (-31.900000, 34.350000) {強};
\node[Onyomi] at (-31.850000, 33.950000) {キョウ};
\node[Kunyomi] at (-31.950000, 33.950000) {つよ.い};
\node[Meaning] at (-31.900000, 35.600000) {strong};
\node[Square] at (-29.850000, 33.850000) {};
\node[Kanji] at (-29.850000, 34.350000) {震};
\node[Onyomi] at (-29.800000, 33.950000) {シン};
\node[Kunyomi] at (-29.900000, 33.950000) {ふる.える};
\node[Meaning] at (-29.850000, 35.600000) {earthquake};
\node[Square] at (-27.800000, 33.850000) {};
\node[Kanji] at (-27.800000, 34.350000) {止};
\node[Onyomi] at (-27.750000, 33.950000) {シ};
\node[Kunyomi] at (-27.850000, 33.950000) {と.まる};
\node[Meaning] at (-27.800000, 35.600000) {stop};
\node[Square] at (-25.750000, 33.850000) {};
\node[Kanji] at (-25.750000, 34.350000) {姿};
\node[Onyomi] at (-25.700000, 33.950000) {シ};
\node[Kunyomi] at (-25.800000, 33.950000) {すがた};
\node[Meaning] at (-25.750000, 35.600000) {figure};
\node[Square] at (-23.700000, 33.850000) {};
\node[Kanji] at (-23.700000, 34.350000) {題};
\node[Onyomi] at (-23.650000, 33.950000) {ダイ};
\node[Meaning] at (-23.700000, 35.600000) {topic};
\node[Square] at (-21.650000, 33.850000) {};
\node[Kanji] at (-21.650000, 34.350000) {夜};
\node[Onyomi] at (-21.600000, 33.950000) {ヤ};
\node[Kunyomi] at (-21.700000, 33.950000) {よ};
\node[Meaning] at (-21.650000, 35.600000) {night};
\node[Square] at (-19.600000, 33.850000) {};
\node[Kanji] at (-19.600000, 34.350000) {数};
\node[Onyomi] at (-19.550000, 33.950000) {スウ};
\node[Kunyomi] at (-19.650000, 33.950000) {かぞ.える};
\node[Meaning] at (-19.600000, 35.600000) {count};
\node[Square] at (-17.550000, 33.850000) {};
\node[Kanji] at (-17.550000, 34.350000) {両};
\node[Onyomi] at (-17.500000, 33.950000) {リョウ};
\node[Meaning] at (-17.550000, 35.600000) {both};
\node[Square] at (-15.500000, 33.850000) {};
\node[Kanji] at (-15.500000, 34.350000) {終};
\node[Onyomi] at (-15.450000, 33.950000) {シュウ};
\node[Kunyomi] at (-15.550000, 33.950000) {おわ.り};
\node[Meaning] at (-15.500000, 35.600000) {end};
\node[Square] at (-13.450000, 33.850000) {};
\node[Kanji] at (-13.450000, 34.350000) {残};
\node[Onyomi] at (-13.400000, 33.950000) {ザン};
\node[Kunyomi] at (-13.500000, 33.950000) {のこ.*};
\node[Meaning] at (-13.450000, 35.600000) {remainder};
\node[Square] at (-11.400000, 33.850000) {};
\node[Kanji] at (-11.400000, 34.350000) {仕};
\node[Onyomi] at (-11.350000, 33.950000) {シ};
\node[Kunyomi] at (-11.450000, 33.950000) {つか.える};
\node[Meaning] at (-11.400000, 35.600000) {doing};
\node[Square] at (-9.350000, 33.850000) {};
\node[Kanji] at (-9.350000, 34.350000) {記};
\node[Onyomi] at (-9.300000, 33.950000) {キ};
\node[Kunyomi] at (-9.400000, 33.950000) {しる.す};
\node[Meaning] at (-9.350000, 35.600000) {write down};
\node[Square] at (-7.300000, 33.850000) {};
\node[Kanji] at (-7.300000, 34.350000) {受};
\node[Onyomi] at (-7.250000, 33.950000) {ジュ};
\node[Kunyomi] at (-7.350000, 33.950000) {う};
\node[Meaning] at (-7.300000, 35.600000) {accept};
\node[Square] at (-5.250000, 33.850000) {};
\node[Kanji] at (-5.250000, 34.350000) {省};
\node[Onyomi] at (-5.200000, 33.950000) {ショウ};
\node[Kunyomi] at (-5.300000, 33.950000) {はぶ.く};
\node[Meaning] at (-5.250000, 35.600000) {conserve};
\node[Square] at (-3.200000, 33.850000) {};
\node[Kanji] at (-3.200000, 34.350000) {界};
\node[Onyomi] at (-3.150000, 33.950000) {カイ};
\node[Meaning] at (-3.200000, 35.600000) {world};
\node[Square] at (-1.150000, 33.850000) {};
\node[Kanji] at (-1.150000, 34.350000) {乗};
\node[Onyomi] at (-1.100000, 33.950000) {ジョウ};
\node[Kunyomi] at (-1.200000, 33.950000) {の};
\node[Meaning] at (-1.150000, 35.600000) {ride};
\node[Square] at (0.900000, 33.850000) {};
\node[Kanji] at (0.900000, 34.350000) {指};
\node[Onyomi] at (0.950000, 33.950000) {シ};
\node[Kunyomi] at (0.850000, 33.950000) {ゆび};
\node[Meaning] at (0.900000, 35.600000) {finger};
\node[Square] at (2.950000, 33.850000) {};
\node[Kanji] at (2.950000, 34.350000) {関};
\node[Onyomi] at (3.000000, 33.950000) {カン};
\node[Kunyomi] at (2.900000, 33.950000) {かか.わる};
\node[Meaning] at (2.950000, 35.600000) {related};
\node[Square] at (5.000000, 33.850000) {};
\node[Kanji] at (5.000000, 34.350000) {海};
\node[Onyomi] at (5.050000, 33.950000) {カイ};
\node[Kunyomi] at (4.950000, 33.950000) {うみ};
\node[Meaning] at (5.000000, 35.600000) {sea};
\node[Square] at (7.050000, 33.850000) {};
\node[Kanji] at (7.050000, 34.350000) {現};
\node[Onyomi] at (7.100000, 33.950000) {ゲン};
\node[Kunyomi] at (7.000000, 33.950000) {あらわ.*};
\node[Meaning] at (7.050000, 35.600000) {present time};
\node[Square] at (9.100000, 33.850000) {};
\node[Kanji] at (9.100000, 34.350000) {用};
\node[Onyomi] at (9.150000, 33.950000) {ヨウ};
\node[Kunyomi] at (9.050000, 33.950000) {もち.いる};
\node[Meaning] at (9.100000, 35.600000) {task};
\node[Square] at (11.150000, 33.850000) {};
\node[Kanji] at (11.150000, 34.350000) {息};
\node[Onyomi] at (11.200000, 33.950000) {ソク};
\node[Kunyomi] at (11.100000, 33.950000) {いき};
\node[Meaning] at (11.150000, 35.600000) {breath};
\node[Square] at (13.200000, 33.850000) {};
\node[Kanji] at (13.200000, 34.350000) {石};
\node[Onyomi] at (13.250000, 33.950000) {セキ};
\node[Kunyomi] at (13.150000, 33.950000) {いし};
\node[Meaning] at (13.200000, 35.600000) {stone};
\node[Square] at (15.250000, 33.850000) {};
\node[Kanji] at (15.250000, 34.350000) {万};
\node[Onyomi] at (15.300000, 33.950000) {マン};
\node[Meaning] at (15.250000, 35.600000) {ten thousand};
\node[Square] at (17.300000, 33.850000) {};
\node[Kanji] at (17.300000, 34.350000) {葉};
\node[Kunyomi] at (17.250000, 33.950000) {は};
\node[Meaning] at (17.300000, 35.600000) {leaf};
\node[Square] at (19.350000, 33.850000) {};
\node[Kanji] at (19.350000, 34.350000) {横};
\node[Onyomi] at (19.400000, 33.950000) {オウ};
\node[Kunyomi] at (19.300000, 33.950000) {よこ};
\node[Meaning] at (19.350000, 35.600000) {side};
\node[Square] at (21.400000, 33.850000) {};
\node[Kanji] at (21.400000, 34.350000) {平};
\node[Onyomi] at (21.450000, 33.950000) {ヘイ};
\node[Kunyomi] at (21.350000, 33.950000) {たいら};
\node[Meaning] at (21.400000, 35.600000) {flat};
\node[Square] at (23.450000, 33.850000) {};
\node[Kanji] at (23.450000, 34.350000) {消};
\node[Onyomi] at (23.500000, 33.950000) {ショウ};
\node[Kunyomi] at (23.400000, 33.950000) {き.*};
\node[Meaning] at (23.450000, 35.600000) {extinguish};
\node[Square] at (25.500000, 33.850000) {};
\node[Kanji] at (25.500000, 34.350000) {置};
\node[Onyomi] at (25.550000, 33.950000) {チ};
\node[Kunyomi] at (25.450000, 33.950000) {お.く};
\node[Meaning] at (25.500000, 35.600000) {put};
\node[Square] at (27.550000, 33.850000) {};
\node[Kanji] at (27.550000, 34.350000) {紙};
\node[Onyomi] at (27.600000, 33.950000) {シ};
\node[Kunyomi] at (27.500000, 33.950000) {かみ};
\node[Meaning] at (27.550000, 35.600000) {paper};
\node[Square] at (29.600000, 33.850000) {};
\node[Kanji] at (29.600000, 34.350000) {集};
\node[Onyomi] at (29.650000, 33.950000) {シュウ};
\node[Kunyomi] at (29.550000, 33.950000) {あつ.まる};
\node[Meaning] at (29.600000, 35.600000) {collect};
\node[Square] at (31.650000, 33.850000) {};
\node[Kanji] at (31.650000, 34.350000) {火};
\node[Onyomi] at (31.700000, 33.950000) {カ};
\node[Kunyomi] at (31.600000, 33.950000) {ひ};
\node[Meaning] at (31.650000, 35.600000) {fire};
\node[Square] at (33.700000, 33.850000) {};
\node[Kanji] at (33.700000, 34.350000) {広};
\node[Onyomi] at (33.750000, 33.950000) {コウ};
\node[Kunyomi] at (33.650000, 33.950000) {ひろ};
\node[Meaning] at (33.700000, 35.600000) {wide};
\node[Square] at (35.750000, 33.850000) {};
\node[Kanji] at (35.750000, 34.350000) {階};
\node[Onyomi] at (35.800000, 33.950000) {カイ};
\node[Meaning] at (35.750000, 35.600000) {floor};
\node[Square] at (37.800000, 33.850000) {};
\node[Kanji] at (37.800000, 34.350000) {首};
\node[Onyomi] at (37.850000, 33.950000) {シュ};
\node[Kunyomi] at (37.750000, 33.950000) {くび};
\node[Meaning] at (37.800000, 35.600000) {neck};
\node[Square] at (39.850000, 33.850000) {};
\node[Kanji] at (39.850000, 34.350000) {味};
\node[Onyomi] at (39.900000, 33.950000) {ミ};
\node[Kunyomi] at (39.800000, 33.950000) {あじ};
\node[Meaning] at (39.850000, 35.600000) {flavor};
\node[Square] at (41.900000, 33.850000) {};
\node[Kanji] at (41.900000, 34.350000) {業};
\node[Onyomi] at (41.950000, 33.950000) {ギョウ};
\node[Meaning] at (41.900000, 35.600000) {business};
\node[Square] at (43.950000, 33.850000) {};
\node[Kanji] at (43.950000, 34.350000) {早};
\node[Onyomi] at (44.000000, 33.950000) {ソウ};
\node[Kunyomi] at (43.900000, 33.950000) {はや.い};
\node[Meaning] at (43.950000, 35.600000) {early};
\node[Square] at (46.000000, 33.850000) {};
\node[Kanji] at (46.000000, 34.350000) {術};
\node[Onyomi] at (46.050000, 33.950000) {ジュツ};
\node[Meaning] at (46.000000, 35.600000) {art};
\node[Square] at (48.050000, 33.850000) {};
\node[Kanji] at (48.050000, 34.350000) {楽};
\node[Onyomi] at (48.100000, 33.950000) {ガク};
\node[Kunyomi] at (48.000000, 33.950000) {たの.しい};
\node[Meaning] at (48.050000, 35.600000) {comfort};
\node[Square] at (50.100000, 33.850000) {};
\node[Kanji] at (50.100000, 34.350000) {実};
\node[Onyomi] at (50.150000, 33.950000) {ジツ};
\node[Kunyomi] at (50.050000, 33.950000) {み};
\node[Meaning] at (50.100000, 35.600000) {truth};
\node[Square] at (52.150000, 33.850000) {};
\node[Kanji] at (52.150000, 34.350000) {赤};
\node[Onyomi] at (52.200000, 33.950000) {セキ};
\node[Kunyomi] at (52.100000, 33.950000) {あか};
\node[Meaning] at (52.150000, 35.600000) {red};
\node[Square] at (54.200000, 33.850000) {};
\node[Kanji] at (54.200000, 34.350000) {恐};
\node[Onyomi] at (54.250000, 33.950000) {キョウ};
\node[Kunyomi] at (54.150000, 33.950000) {おそ.*};
\node[Meaning] at (54.200000, 35.600000) {fear};
\node[Square] at (56.250000, 33.850000) {};
\node[Kanji] at (56.250000, 34.350000) {配};
\node[Onyomi] at (56.300000, 33.950000) {ハイ};
\node[Kunyomi] at (56.200000, 33.950000) {くば.る};
\node[Meaning] at (56.250000, 35.600000) {distribute};
\node[Meaning] at (-58.500000, 34.400000) {60.86\%};
\node[Square] at (-56.500000, 31.800000) {};
\node[Kanji] at (-56.500000, 32.300000) {結};
\node[Onyomi] at (-56.450000, 31.900000) {ケツ};
\node[Kunyomi] at (-56.550000, 31.900000) {むす.ぶ};
\node[Meaning] at (-56.500000, 33.550000) {bind};
\node[Square] at (-54.450000, 31.800000) {};
\node[Kanji] at (-54.450000, 32.300000) {信};
\node[Onyomi] at (-54.400000, 31.900000) {シン};
\node[Kunyomi] at (-54.500000, 31.900000) {しん};
\node[Meaning] at (-54.450000, 33.550000) {believe};
\node[Square] at (-52.400000, 31.800000) {};
\node[Kanji] at (-52.400000, 32.300000) {情};
\node[Onyomi] at (-52.350000, 31.900000) {ジョウ};
\node[Kunyomi] at (-52.450000, 31.900000) {なさけ};
\node[Meaning] at (-52.400000, 33.550000) {feeling};
\node[Square] at (-50.350000, 31.800000) {};
\node[Kanji] at (-50.350000, 32.300000) {飲};
\node[Onyomi] at (-50.300000, 31.900000) {イン};
\node[Kunyomi] at (-50.400000, 31.900000) {の};
\node[Meaning] at (-50.350000, 33.550000) {drink};
\node[Square] at (-48.300000, 31.800000) {};
\node[Kanji] at (-48.300000, 32.300000) {試};
\node[Onyomi] at (-48.250000, 31.900000) {シ};
\node[Kunyomi] at (-48.350000, 31.900000) {こころ.みる};
\node[Meaning] at (-48.300000, 33.550000) {try};
\node[Square] at (-46.250000, 31.800000) {};
\node[Kanji] at (-46.250000, 32.300000) {次};
\node[Onyomi] at (-46.200000, 31.900000) {ジ};
\node[Kunyomi] at (-46.300000, 31.900000) {つぎ};
\node[Meaning] at (-46.250000, 33.550000) {next};
\node[Square] at (-44.200000, 31.800000) {};
\node[Kanji] at (-44.200000, 32.300000) {達};
\node[Onyomi] at (-44.150000, 31.900000) {タツ};
\node[Kunyomi] at (-44.250000, 31.900000) {たち};
\node[Meaning] at (-44.200000, 33.550000) {attain};
\node[Square] at (-42.150000, 31.800000) {};
\node[Kanji] at (-42.150000, 32.300000) {鳴};
\node[Onyomi] at (-42.100000, 31.900000) {メイ};
\node[Kunyomi] at (-42.200000, 31.900000) {な};
\node[Meaning] at (-42.150000, 33.550000) {chirp};
\node[Square] at (-40.100000, 31.800000) {};
\node[Kanji] at (-40.100000, 32.300000) {離};
\node[Onyomi] at (-40.050000, 31.900000) {リ};
\node[Kunyomi] at (-40.150000, 31.900000) {はな.*};
\node[Meaning] at (-40.100000, 33.550000) {detach};
\node[Square] at (-38.050000, 31.800000) {};
\node[Kanji] at (-38.050000, 32.300000) {増};
\node[Onyomi] at (-38.000000, 31.900000) {ゾウ};
\node[Kunyomi] at (-38.100000, 31.900000) {ふ.える};
\node[Meaning] at (-38.050000, 33.550000) {increase};
\node[Square] at (-36.000000, 31.800000) {};
\node[Kanji] at (-36.000000, 32.300000) {掛};
\node[Onyomi] at (-35.950000, 31.900000) {ガイ};
\node[Kunyomi] at (-36.050000, 31.900000) {か};
\node[Meaning] at (-36.000000, 33.550000) {hang};
\node[Square] at (-33.950000, 31.800000) {};
\node[Kanji] at (-33.950000, 32.300000) {殺};
\node[Onyomi] at (-33.900000, 31.900000) {サツ};
\node[Kunyomi] at (-34.000000, 31.900000) {ころ.す};
\node[Meaning] at (-33.950000, 33.550000) {kill};
\node[Square] at (-31.900000, 31.800000) {};
\node[Kanji] at (-31.900000, 32.300000) {計};
\node[Onyomi] at (-31.850000, 31.900000) {ケイ};
\node[Kunyomi] at (-31.950000, 31.900000) {はか.る};
\node[Meaning] at (-31.900000, 33.550000) {measure};
\node[Square] at (-29.850000, 31.800000) {};
\node[Kanji] at (-29.850000, 32.300000) {運};
\node[Onyomi] at (-29.800000, 31.900000) {ウン};
\node[Kunyomi] at (-29.900000, 31.900000) {はこ.ぶ};
\node[Meaning] at (-29.850000, 33.550000) {carry};
\node[Square] at (-27.800000, 31.800000) {};
\node[Kanji] at (-27.800000, 32.300000) {原};
\node[Onyomi] at (-27.750000, 31.900000) {ゲン};
\node[Kunyomi] at (-27.850000, 31.900000) {はら};
\node[Meaning] at (-27.800000, 33.550000) {original};
\node[Square] at (-25.750000, 31.800000) {};
\node[Kanji] at (-25.750000, 32.300000) {背};
\node[Onyomi] at (-25.700000, 31.900000) {ハイ};
\node[Kunyomi] at (-25.800000, 31.900000) {せ};
\node[Meaning] at (-25.750000, 33.550000) {back};
\node[Square] at (-23.700000, 31.800000) {};
\node[Kanji] at (-23.700000, 32.300000) {付};
\node[Onyomi] at (-23.650000, 31.900000) {フ};
\node[Kunyomi] at (-23.750000, 31.900000) {つ};
\node[Meaning] at (-23.700000, 33.550000) {attach};
\node[Square] at (-21.650000, 31.800000) {};
\node[Kanji] at (-21.650000, 32.300000) {夫};
\node[Onyomi] at (-21.600000, 31.900000) {フウ};
\node[Kunyomi] at (-21.700000, 31.900000) {おっと};
\node[Meaning] at (-21.650000, 33.550000) {husband};
\node[Square] at (-19.600000, 31.800000) {};
\node[Kanji] at (-19.600000, 32.300000) {周};
\node[Onyomi] at (-19.550000, 31.900000) {シュウ};
\node[Kunyomi] at (-19.650000, 31.900000) {まわ.り};
\node[Meaning] at (-19.600000, 33.550000) {circumference};
\node[Square] at (-17.550000, 31.800000) {};
\node[Kanji] at (-17.550000, 32.300000) {正};
\node[Onyomi] at (-17.500000, 31.900000) {セイ};
\node[Kunyomi] at (-17.600000, 31.900000) {ただ.しい};
\node[Meaning] at (-17.550000, 33.550000) {correct};
\node[Square] at (-15.500000, 31.800000) {};
\node[Kanji] at (-15.500000, 32.300000) {元};
\node[Onyomi] at (-15.450000, 31.900000) {ゲン};
\node[Kunyomi] at (-15.550000, 31.900000) {もと};
\node[Meaning] at (-15.500000, 33.550000) {origin};
\node[Square] at (-13.450000, 31.800000) {};
\node[Kanji] at (-13.450000, 32.300000) {父};
\node[Onyomi] at (-13.400000, 31.900000) {フ};
\node[Kunyomi] at (-13.500000, 31.900000) {ちち};
\node[Meaning] at (-13.450000, 33.550000) {father};
\node[Square] at (-11.400000, 31.800000) {};
\node[Kanji] at (-11.400000, 32.300000) {代};
\node[Onyomi] at (-11.350000, 31.900000) {ダイ};
\node[Kunyomi] at (-11.450000, 31.900000) {か};
\node[Meaning] at (-11.400000, 33.550000) {substitute};
\node[Square] at (-9.350000, 31.800000) {};
\node[Kanji] at (-9.350000, 32.300000) {呼};
\node[Onyomi] at (-9.300000, 31.900000) {コ};
\node[Kunyomi] at (-9.400000, 31.900000) {よ};
\node[Meaning] at (-9.350000, 33.550000) {call};
\node[Square] at (-7.300000, 31.800000) {};
\node[Kanji] at (-7.300000, 32.300000) {闇};
\node[Onyomi] at (-7.250000, 31.900000) {アン};
\node[Kunyomi] at (-7.350000, 31.900000) {やみ};
\node[Meaning] at (-7.300000, 33.550000) {darkness};
\node[Square] at (-5.250000, 31.800000) {};
\node[Kanji] at (-5.250000, 32.300000) {組};
\node[Onyomi] at (-5.200000, 31.900000) {ソ};
\node[Kunyomi] at (-5.300000, 31.900000) {くみ};
\node[Meaning] at (-5.250000, 33.550000) {group};
\node[Square] at (-3.200000, 31.800000) {};
\node[Kanji] at (-3.200000, 32.300000) {帰};
\node[Onyomi] at (-3.150000, 31.900000) {キ};
\node[Kunyomi] at (-3.250000, 31.900000) {かえ};
\node[Meaning] at (-3.200000, 33.550000) {return};
\node[Square] at (-1.150000, 31.800000) {};
\node[Kanji] at (-1.150000, 32.300000) {線};
\node[Onyomi] at (-1.100000, 31.900000) {セン};
\node[Meaning] at (-1.150000, 33.550000) {line};
\node[Square] at (0.900000, 31.800000) {};
\node[Kanji] at (0.900000, 32.300000) {徒};
\node[Onyomi] at (0.950000, 31.900000) {ト};
\node[Meaning] at (0.900000, 33.550000) {junior};
\node[Square] at (2.950000, 31.800000) {};
\node[Kanji] at (2.950000, 32.300000) {戦};
\node[Onyomi] at (3.000000, 31.900000) {セン};
\node[Kunyomi] at (2.900000, 31.900000) {たたか.*};
\node[Meaning] at (2.950000, 33.550000) {war};
\node[Square] at (5.000000, 31.800000) {};
\node[Kanji] at (5.000000, 32.300000) {暗};
\node[Onyomi] at (5.050000, 31.900000) {アン};
\node[Kunyomi] at (4.950000, 31.900000) {くら.い};
\node[Meaning] at (5.000000, 33.550000) {dark};
\node[Square] at (7.050000, 31.800000) {};
\node[Kanji] at (7.050000, 32.300000) {怒};
\node[Onyomi] at (7.100000, 31.900000) {ド};
\node[Kunyomi] at (7.000000, 31.900000) {おこ.る};
\node[Meaning] at (7.050000, 33.550000) {angry};
\node[Square] at (9.100000, 31.800000) {};
\node[Kanji] at (9.100000, 32.300000) {定};
\node[Onyomi] at (9.150000, 31.900000) {テイ};
\node[Kunyomi] at (9.050000, 31.900000) {さだ};
\node[Meaning] at (9.100000, 33.550000) {determine};
\node[Square] at (11.150000, 31.800000) {};
\node[Kanji] at (11.150000, 32.300000) {島};
\node[Onyomi] at (11.200000, 31.900000) {トウ};
\node[Kunyomi] at (11.100000, 31.900000) {しま};
\node[Meaning] at (11.150000, 33.550000) {island};
\node[Square] at (13.200000, 31.800000) {};
\node[Kanji] at (13.200000, 32.300000) {我};
\node[Onyomi] at (13.250000, 31.900000) {ガ};
\node[Kunyomi] at (13.150000, 31.900000) {われ};
\node[Meaning] at (13.200000, 33.550000) {i};
\node[Square] at (15.250000, 31.800000) {};
\node[Kanji] at (15.250000, 32.300000) {四};
\node[Onyomi] at (15.300000, 31.900000) {シ};
\node[Kunyomi] at (15.200000, 31.900000) {よん};
\node[Meaning] at (15.250000, 33.550000) {four};
\node[Square] at (17.300000, 31.800000) {};
\node[Kanji] at (17.300000, 32.300000) {的};
\node[Onyomi] at (17.350000, 31.900000) {テキ};
\node[Kunyomi] at (17.250000, 31.900000) {まと};
\node[Meaning] at (17.300000, 33.550000) {target};
\node[Square] at (19.350000, 31.800000) {};
\node[Kanji] at (19.350000, 32.300000) {病};
\node[Onyomi] at (19.400000, 31.900000) {ビョウ};
\node[Kunyomi] at (19.300000, 31.900000) {や};
\node[Meaning] at (19.350000, 33.550000) {sick};
\node[Square] at (21.400000, 31.800000) {};
\node[Kanji] at (21.400000, 32.300000) {抜};
\node[Onyomi] at (21.450000, 31.900000) {バツ};
\node[Kunyomi] at (21.350000, 31.900000) {ぬ};
\node[Meaning] at (21.400000, 33.550000) {extract};
\node[Square] at (23.450000, 31.800000) {};
\node[Kanji] at (23.450000, 32.300000) {章};
\node[Onyomi] at (23.500000, 31.900000) {ショウ};
\node[Meaning] at (23.450000, 33.550000) {chapter};
\node[Square] at (25.500000, 31.800000) {};
\node[Kanji] at (25.500000, 32.300000) {好};
\node[Onyomi] at (25.550000, 31.900000) {コウ};
\node[Kunyomi] at (25.450000, 31.900000) {す.き};
\node[Meaning] at (25.500000, 33.550000) {like};
\node[Square] at (27.550000, 31.800000) {};
\node[Kanji] at (27.550000, 32.300000) {転};
\node[Onyomi] at (27.600000, 31.900000) {テン};
\node[Kunyomi] at (27.500000, 31.900000) {ころ.ぶ};
\node[Meaning] at (27.550000, 33.550000) {revolve};
\node[Square] at (29.600000, 31.800000) {};
\node[Kanji] at (29.600000, 32.300000) {過};
\node[Onyomi] at (29.650000, 31.900000) {カ};
\node[Kunyomi] at (29.550000, 31.900000) {す.ぎ};
\node[Meaning] at (29.600000, 33.550000) {surpass};
\node[Square] at (31.650000, 31.800000) {};
\node[Kanji] at (31.650000, 32.300000) {押};
\node[Onyomi] at (31.700000, 31.900000) {オウ};
\node[Kunyomi] at (31.600000, 31.900000) {お};
\node[Meaning] at (31.650000, 33.550000) {push};
\node[Square] at (33.700000, 31.800000) {};
\node[Kanji] at (33.700000, 32.300000) {週};
\node[Onyomi] at (33.750000, 31.900000) {シュウ};
\node[Meaning] at (33.700000, 33.550000) {week};
\node[Square] at (35.750000, 31.800000) {};
\node[Kanji] at (35.750000, 32.300000) {伝};
\node[Onyomi] at (35.800000, 31.900000) {デン};
\node[Kunyomi] at (35.700000, 31.900000) {つた};
\node[Meaning] at (35.750000, 33.550000) {transmit};
\node[Square] at (37.800000, 31.800000) {};
\node[Kanji] at (37.800000, 32.300000) {半};
\node[Onyomi] at (37.850000, 31.900000) {ハン};
\node[Kunyomi] at (37.750000, 31.900000) {なか.ば};
\node[Meaning] at (37.800000, 33.550000) {half};
\node[Square] at (39.850000, 31.800000) {};
\node[Kanji] at (39.850000, 32.300000) {母};
\node[Onyomi] at (39.900000, 31.900000) {ボ};
\node[Kunyomi] at (39.800000, 31.900000) {はは};
\node[Meaning] at (39.850000, 33.550000) {mother};
\node[Square] at (41.900000, 31.800000) {};
\node[Kanji] at (41.900000, 32.300000) {相};
\node[Onyomi] at (41.950000, 31.900000) {ソウ};
\node[Kunyomi] at (41.850000, 31.900000) {あい};
\node[Meaning] at (41.900000, 33.550000) {mutual};
\node[Square] at (43.950000, 31.800000) {};
\node[Kanji] at (43.950000, 32.300000) {売};
\node[Onyomi] at (44.000000, 31.900000) {バイ};
\node[Kunyomi] at (43.900000, 31.900000) {う};
\node[Meaning] at (43.950000, 33.550000) {sell};
\node[Square] at (46.000000, 31.800000) {};
\node[Kanji] at (46.000000, 32.300000) {黒};
\node[Onyomi] at (46.050000, 31.900000) {コク};
\node[Kunyomi] at (45.950000, 31.900000) {くろ.い};
\node[Meaning] at (46.000000, 33.550000) {black};
\node[Square] at (48.050000, 31.800000) {};
\node[Kanji] at (48.050000, 32.300000) {朝};
\node[Onyomi] at (48.100000, 31.900000) {チョウ};
\node[Kunyomi] at (48.000000, 31.900000) {あさ};
\node[Meaning] at (48.050000, 33.550000) {morning};
\node[Square] at (50.100000, 31.800000) {};
\node[Kanji] at (50.100000, 32.300000) {別};
\node[Onyomi] at (50.150000, 31.900000) {ベツ};
\node[Kunyomi] at (50.050000, 31.900000) {わか.*};
\node[Meaning] at (50.100000, 33.550000) {separate};
\node[Square] at (52.150000, 31.800000) {};
\node[Kanji] at (52.150000, 32.300000) {授};
\node[Onyomi] at (52.200000, 31.900000) {ジュ};
\node[Kunyomi] at (52.100000, 31.900000) {さず.ける};
\node[Meaning] at (52.150000, 33.550000) {instruct};
\node[Square] at (54.200000, 31.800000) {};
\node[Kanji] at (54.200000, 32.300000) {直};
\node[Onyomi] at (54.250000, 31.900000) {チョク};
\node[Kunyomi] at (54.150000, 31.900000) {なお.す};
\node[Meaning] at (54.200000, 33.550000) {fix};
\node[Square] at (56.250000, 31.800000) {};
\node[Kanji] at (56.250000, 32.300000) {放};
\node[Onyomi] at (56.300000, 31.900000) {ホウ};
\node[Kunyomi] at (56.200000, 31.900000) {はな};
\node[Meaning] at (56.250000, 33.550000) {release};
\node[Meaning] at (-58.500000, 32.350000) {66.13\%};
\node[Square] at (-56.500000, 29.750000) {};
\node[Kanji] at (-56.500000, 30.250000) {読};
\node[Onyomi] at (-56.450000, 29.850000) {トウ};
\node[Kunyomi] at (-56.550000, 29.850000) {よ};
\node[Meaning] at (-56.500000, 31.500000) {read};
\node[Square] at (-54.450000, 29.750000) {};
\node[Kanji] at (-54.450000, 30.250000) {亡};
\node[Onyomi] at (-54.400000, 29.850000) {ボウ};
\node[Kunyomi] at (-54.500000, 29.850000) {な.く};
\node[Meaning] at (-54.450000, 31.500000) {deceased};
\node[Square] at (-52.400000, 29.750000) {};
\node[Kanji] at (-52.400000, 30.250000) {追};
\node[Onyomi] at (-52.350000, 29.850000) {ツイ};
\node[Kunyomi] at (-52.450000, 29.850000) {お};
\node[Meaning] at (-52.400000, 31.500000) {follow};
\node[Square] at (-50.350000, 29.750000) {};
\node[Kanji] at (-50.350000, 30.250000) {血};
\node[Onyomi] at (-50.300000, 29.850000) {ケツ};
\node[Kunyomi] at (-50.400000, 29.850000) {ち};
\node[Meaning] at (-50.350000, 31.500000) {blood};
\node[Square] at (-48.300000, 29.750000) {};
\node[Kanji] at (-48.300000, 30.250000) {薬};
\node[Onyomi] at (-48.250000, 29.850000) {ヤク};
\node[Kunyomi] at (-48.350000, 29.850000) {くすり};
\node[Meaning] at (-48.300000, 31.500000) {medicine};
\node[Square] at (-46.250000, 29.750000) {};
\node[Kanji] at (-46.250000, 30.250000) {歳};
\node[Onyomi] at (-46.200000, 29.850000) {サイ};
\node[Meaning] at (-46.250000, 31.500000) {years old};
\node[Square] at (-44.200000, 29.750000) {};
\node[Kanji] at (-44.200000, 30.250000) {内};
\node[Onyomi] at (-44.150000, 29.850000) {ナイ};
\node[Kunyomi] at (-44.250000, 29.850000) {うち};
\node[Meaning] at (-44.200000, 31.500000) {inside};
\node[Square] at (-42.150000, 29.750000) {};
\node[Kanji] at (-42.150000, 30.250000) {痛};
\node[Onyomi] at (-42.100000, 29.850000) {ツウ};
\node[Kunyomi] at (-42.200000, 29.850000) {いた.い};
\node[Meaning] at (-42.150000, 31.500000) {pain};
\node[Square] at (-40.100000, 29.750000) {};
\node[Kanji] at (-40.100000, 30.250000) {探};
\node[Onyomi] at (-40.050000, 29.850000) {タン};
\node[Kunyomi] at (-40.150000, 29.850000) {さが.す};
\node[Meaning] at (-40.100000, 31.500000) {look for};
\node[Square] at (-38.050000, 29.750000) {};
\node[Kanji] at (-38.050000, 30.250000) {瞬};
\node[Onyomi] at (-38.000000, 29.850000) {シュン};
\node[Kunyomi] at (-38.100000, 29.850000) {またた.く};
\node[Meaning] at (-38.050000, 31.500000) {blink};
\node[Square] at (-36.000000, 29.750000) {};
\node[Kanji] at (-36.000000, 30.250000) {流};
\node[Onyomi] at (-35.950000, 29.850000) {リュウ};
\node[Kunyomi] at (-36.050000, 29.850000) {なが.*};
\node[Meaning] at (-36.000000, 31.500000) {stream};
\node[Square] at (-33.950000, 29.750000) {};
\node[Kanji] at (-33.950000, 30.250000) {寄};
\node[Onyomi] at (-33.900000, 29.850000) {キ};
\node[Kunyomi] at (-34.000000, 29.850000) {よ.る};
\node[Meaning] at (-33.950000, 31.500000) {approach};
\node[Square] at (-31.900000, 29.750000) {};
\node[Kanji] at (-31.900000, 30.250000) {険};
\node[Onyomi] at (-31.850000, 29.850000) {ケン};
\node[Kunyomi] at (-31.950000, 29.850000) {けわ.しい};
\node[Meaning] at (-31.900000, 31.500000) {risky};
\node[Square] at (-29.850000, 29.750000) {};
\node[Kanji] at (-29.850000, 30.250000) {隠};
\node[Onyomi] at (-29.800000, 29.850000) {イン};
\node[Kunyomi] at (-29.900000, 29.850000) {かく.*};
\node[Meaning] at (-29.850000, 31.500000) {hide};
\node[Square] at (-27.800000, 29.750000) {};
\node[Kanji] at (-27.800000, 30.250000) {能};
\node[Onyomi] at (-27.750000, 29.850000) {ノウ};
\node[Meaning] at (-27.800000, 31.500000) {ability};
\node[Square] at (-25.750000, 29.750000) {};
\node[Kanji] at (-25.750000, 30.250000) {円};
\node[Onyomi] at (-25.700000, 29.850000) {エン};
\node[Kunyomi] at (-25.800000, 29.850000) {まる.い};
\node[Meaning] at (-25.750000, 31.500000) {yen};
\node[Square] at (-23.700000, 29.750000) {};
\node[Kanji] at (-23.700000, 30.250000) {語};
\node[Onyomi] at (-23.650000, 29.850000) {ゴ};
\node[Kunyomi] at (-23.750000, 29.850000) {かた.る};
\node[Meaning] at (-23.700000, 31.500000) {language};
\node[Square] at (-21.650000, 29.750000) {};
\node[Kanji] at (-21.650000, 30.250000) {点};
\node[Onyomi] at (-21.600000, 29.850000) {テン};
\node[Kunyomi] at (-21.700000, 29.850000) {つ.ける};
\node[Meaning] at (-21.650000, 31.500000) {point};
\node[Square] at (-19.600000, 29.750000) {};
\node[Kanji] at (-19.600000, 30.250000) {壁};
\node[Onyomi] at (-19.550000, 29.850000) {ヘキ};
\node[Kunyomi] at (-19.650000, 29.850000) {かべ};
\node[Meaning] at (-19.600000, 31.500000) {wall};
\node[Square] at (-17.550000, 29.750000) {};
\node[Kanji] at (-17.550000, 30.250000) {警};
\node[Onyomi] at (-17.500000, 29.850000) {ケイ};
\node[Meaning] at (-17.550000, 31.500000) {warn};
\node[Square] at (-15.500000, 29.750000) {};
\node[Kanji] at (-15.500000, 30.250000) {白};
\node[Onyomi] at (-15.450000, 29.850000) {ハク};
\node[Kunyomi] at (-15.550000, 29.850000) {しろ};
\node[Meaning] at (-15.500000, 31.500000) {white};
\node[Square] at (-13.450000, 29.750000) {};
\node[Kanji] at (-13.450000, 30.250000) {重};
\node[Onyomi] at (-13.400000, 29.850000) {ジュウ};
\node[Kunyomi] at (-13.500000, 29.850000) {おも.い};
\node[Meaning] at (-13.450000, 31.500000) {heavy};
\node[Square] at (-11.400000, 29.750000) {};
\node[Kanji] at (-11.400000, 30.250000) {危};
\node[Onyomi] at (-11.350000, 29.850000) {キ};
\node[Kunyomi] at (-11.450000, 29.850000) {あぶ.ない};
\node[Meaning] at (-11.400000, 31.500000) {dangerous};
\node[Square] at (-9.350000, 29.750000) {};
\node[Kanji] at (-9.350000, 30.250000) {天};
\node[Onyomi] at (-9.300000, 29.850000) {テン};
\node[Kunyomi] at (-9.400000, 29.850000) {あま};
\node[Meaning] at (-9.350000, 31.500000) {heaven};
\node[Square] at (-7.300000, 29.750000) {};
\node[Kanji] at (-7.300000, 30.250000) {傷};
\node[Onyomi] at (-7.250000, 29.850000) {ショウ};
\node[Kunyomi] at (-7.350000, 29.850000) {きず};
\node[Meaning] at (-7.300000, 31.500000) {wound};
\node[Square] at (-5.250000, 29.750000) {};
\node[Kanji] at (-5.250000, 30.250000) {覚};
\node[Onyomi] at (-5.200000, 29.850000) {カク};
\node[Kunyomi] at (-5.300000, 29.850000) {おぼ};
\node[Meaning] at (-5.250000, 31.500000) {memorize};
\node[Square] at (-3.200000, 29.750000) {};
\node[Kanji] at (-3.200000, 30.250000) {解};
\node[Onyomi] at (-3.150000, 29.850000) {カイ};
\node[Kunyomi] at (-3.250000, 29.850000) {と.く};
\node[Meaning] at (-3.200000, 31.500000) {untie};
\node[Square] at (-1.150000, 29.750000) {};
\node[Kanji] at (-1.150000, 30.250000) {確};
\node[Onyomi] at (-1.100000, 29.850000) {カク};
\node[Kunyomi] at (-1.200000, 29.850000) {たし.か};
\node[Meaning] at (-1.150000, 31.500000) {certain};
\node[Square] at (0.900000, 29.750000) {};
\node[Kanji] at (0.900000, 30.250000) {打};
\node[Onyomi] at (0.950000, 29.850000) {ダ};
\node[Kunyomi] at (0.850000, 29.850000) {う};
\node[Meaning] at (0.900000, 31.500000) {hit};
\node[Square] at (2.950000, 29.750000) {};
\node[Kanji] at (2.950000, 30.250000) {反};
\node[Onyomi] at (3.000000, 29.850000) {ハン};
\node[Meaning] at (2.950000, 31.500000) {anti};
\node[Square] at (5.000000, 29.750000) {};
\node[Kanji] at (5.000000, 30.250000) {閉};
\node[Onyomi] at (5.050000, 29.850000) {ヘイ};
\node[Kunyomi] at (4.950000, 29.850000) {し};
\node[Meaning] at (5.000000, 31.500000) {closed};
\node[Square] at (7.050000, 29.750000) {};
\node[Kanji] at (7.050000, 30.250000) {勝};
\node[Onyomi] at (7.100000, 29.850000) {ショウ};
\node[Kunyomi] at (7.000000, 29.850000) {か.つ};
\node[Meaning] at (7.050000, 31.500000) {win};
\node[Square] at (9.100000, 29.750000) {};
\node[Kanji] at (9.100000, 30.250000) {可};
\node[Onyomi] at (9.150000, 29.850000) {カ};
\node[Meaning] at (9.100000, 31.500000) {possible};
\node[Square] at (11.150000, 29.750000) {};
\node[Kanji] at (11.150000, 30.250000) {降};
\node[Onyomi] at (11.200000, 29.850000) {コウ};
\node[Kunyomi] at (11.100000, 29.850000) {お.りる};
\node[Meaning] at (11.150000, 31.500000) {descend};
\node[Square] at (13.200000, 29.750000) {};
\node[Kanji] at (13.200000, 30.250000) {髪};
\node[Onyomi] at (13.250000, 29.850000) {ハツ};
\node[Kunyomi] at (13.150000, 29.850000) {かみ};
\node[Meaning] at (13.200000, 31.500000) {hair};
\node[Square] at (15.250000, 29.750000) {};
\node[Kanji] at (15.250000, 30.250000) {買};
\node[Onyomi] at (15.300000, 29.850000) {バイ};
\node[Kunyomi] at (15.200000, 29.850000) {か};
\node[Meaning] at (15.250000, 31.500000) {buy};
\node[Square] at (17.300000, 29.750000) {};
\node[Kanji] at (17.300000, 30.250000) {議};
\node[Onyomi] at (17.350000, 29.850000) {ギ};
\node[Meaning] at (17.300000, 31.500000) {deliberation};
\node[Square] at (19.350000, 29.750000) {};
\node[Kanji] at (19.350000, 30.250000) {進};
\node[Onyomi] at (19.400000, 29.850000) {シン};
\node[Kunyomi] at (19.300000, 29.850000) {すす.む};
\node[Meaning] at (19.350000, 31.500000) {advance};
\node[Square] at (21.400000, 29.750000) {};
\node[Kanji] at (21.400000, 30.250000) {経};
\node[Onyomi] at (21.450000, 29.850000) {ケイ};
\node[Kunyomi] at (21.350000, 29.850000) {た.つ};
\node[Meaning] at (21.400000, 31.500000) {passage of time};
\node[Square] at (23.450000, 29.750000) {};
\node[Kanji] at (23.450000, 30.250000) {店};
\node[Onyomi] at (23.500000, 29.850000) {テン};
\node[Kunyomi] at (23.400000, 29.850000) {みせ};
\node[Meaning] at (23.450000, 31.500000) {shop};
\node[Square] at (25.500000, 29.750000) {};
\node[Kanji] at (25.500000, 30.250000) {助};
\node[Onyomi] at (25.550000, 29.850000) {ジョ};
\node[Kunyomi] at (25.450000, 29.850000) {たす};
\node[Meaning] at (25.500000, 31.500000) {help};
\node[Square] at (27.550000, 29.750000) {};
\node[Kanji] at (27.550000, 30.250000) {静};
\node[Onyomi] at (27.600000, 29.850000) {セイ};
\node[Kunyomi] at (27.500000, 29.850000) {しず.か};
\node[Meaning] at (27.550000, 31.500000) {quiet};
\node[Square] at (29.600000, 29.750000) {};
\node[Kanji] at (29.600000, 30.250000) {住};
\node[Onyomi] at (29.650000, 29.850000) {ジュウ};
\node[Kunyomi] at (29.550000, 29.850000) {す.む};
\node[Meaning] at (29.600000, 31.500000) {dwelling};
\node[Square] at (31.650000, 29.750000) {};
\node[Kanji] at (31.650000, 30.250000) {鬼};
\node[Onyomi] at (31.700000, 29.850000) {キ};
\node[Kunyomi] at (31.600000, 29.850000) {おに};
\node[Meaning] at (31.650000, 31.500000) {demon};
\node[Square] at (33.700000, 29.750000) {};
\node[Kanji] at (33.700000, 30.250000) {腕};
\node[Onyomi] at (33.750000, 29.850000) {ワン};
\node[Kunyomi] at (33.650000, 29.850000) {うで};
\node[Meaning] at (33.700000, 31.500000) {arm};
\node[Square] at (35.750000, 29.750000) {};
\node[Kanji] at (35.750000, 30.250000) {説};
\node[Onyomi] at (35.800000, 29.850000) {セツ};
\node[Kunyomi] at (35.700000, 29.850000) {と.く};
\node[Meaning] at (35.750000, 31.500000) {theory};
\node[Square] at (37.800000, 29.750000) {};
\node[Kanji] at (37.800000, 30.250000) {逃};
\node[Onyomi] at (37.850000, 29.850000) {トウ};
\node[Kunyomi] at (37.750000, 29.850000) {に.げる};
\node[Meaning] at (37.800000, 31.500000) {escape};
\node[Square] at (39.850000, 29.750000) {};
\node[Kanji] at (39.850000, 30.250000) {台};
\node[Onyomi] at (39.900000, 29.850000) {ダイ};
\node[Meaning] at (39.850000, 31.500000) {machine};
\node[Square] at (41.900000, 29.750000) {};
\node[Kanji] at (41.900000, 30.250000) {冷};
\node[Onyomi] at (41.950000, 29.850000) {レイ};
\node[Kunyomi] at (41.850000, 29.850000) {つめ.たい};
\node[Meaning] at (41.900000, 31.500000) {cool};
\node[Square] at (43.950000, 29.750000) {};
\node[Kanji] at (43.950000, 30.250000) {料};
\node[Onyomi] at (44.000000, 29.850000) {リョウ};
\node[Meaning] at (43.950000, 31.500000) {fee};
\node[Square] at (46.000000, 29.750000) {};
\node[Kanji] at (46.000000, 30.250000) {成};
\node[Onyomi] at (46.050000, 29.850000) {セイ};
\node[Kunyomi] at (45.950000, 29.850000) {な.る};
\node[Meaning] at (46.000000, 31.500000) {become};
\node[Square] at (48.050000, 29.750000) {};
\node[Kanji] at (48.050000, 30.250000) {窓};
\node[Onyomi] at (48.100000, 29.850000) {ソウ};
\node[Kunyomi] at (48.000000, 29.850000) {まど};
\node[Meaning] at (48.050000, 31.500000) {window};
\node[Square] at (50.100000, 29.750000) {};
\node[Kanji] at (50.100000, 30.250000) {耳};
\node[Onyomi] at (50.150000, 29.850000) {ジ};
\node[Kunyomi] at (50.050000, 29.850000) {みみ};
\node[Meaning] at (50.100000, 31.500000) {ear};
\node[Square] at (52.150000, 29.750000) {};
\node[Kanji] at (52.150000, 30.250000) {席};
\node[Onyomi] at (52.200000, 29.850000) {セキ};
\node[Meaning] at (52.150000, 31.500000) {seat};
\node[Square] at (54.200000, 29.750000) {};
\node[Kanji] at (54.200000, 30.250000) {彼};
\node[Onyomi] at (54.250000, 29.850000) {ヒ};
\node[Kunyomi] at (54.150000, 29.850000) {かれ};
\node[Meaning] at (54.200000, 31.500000) {he};
\node[Square] at (56.250000, 29.750000) {};
\node[Kanji] at (56.250000, 30.250000) {研};
\node[Onyomi] at (56.300000, 29.850000) {ケン};
\node[Kunyomi] at (56.200000, 29.850000) {と};
\node[Meaning] at (56.250000, 31.500000) {sharpen};
\node[Meaning] at (-58.500000, 30.300000) {70.54\%};
\node[Square] at (-56.500000, 27.700000) {};
\node[Kanji] at (-56.500000, 28.200000) {果};
\node[Onyomi] at (-56.450000, 27.800000) {カ};
\node[Kunyomi] at (-56.550000, 27.800000) {くだ};
\node[Meaning] at (-56.500000, 29.450000) {fruit};
\node[Square] at (-54.450000, 27.700000) {};
\node[Kanji] at (-54.450000, 28.200000) {風};
\node[Onyomi] at (-54.400000, 27.800000) {フウ};
\node[Kunyomi] at (-54.500000, 27.800000) {かぜ};
\node[Meaning] at (-54.450000, 29.450000) {wind};
\node[Square] at (-52.400000, 27.700000) {};
\node[Kanji] at (-52.400000, 28.200000) {吸};
\node[Onyomi] at (-52.350000, 27.800000) {キュウ};
\node[Kunyomi] at (-52.450000, 27.800000) {す.う};
\node[Meaning] at (-52.400000, 29.450000) {suck};
\node[Square] at (-50.350000, 27.700000) {};
\node[Kanji] at (-50.350000, 28.200000) {五};
\node[Onyomi] at (-50.300000, 27.800000) {ゴ};
\node[Kunyomi] at (-50.400000, 27.800000) {いつ.つ};
\node[Meaning] at (-50.350000, 29.450000) {five};
\node[Square] at (-48.300000, 27.700000) {};
\node[Kanji] at (-48.300000, 28.200000) {捕};
\node[Onyomi] at (-48.250000, 27.800000) {ホ};
\node[Kunyomi] at (-48.350000, 27.800000) {つか.まる};
\node[Meaning] at (-48.300000, 29.450000) {catch};
\node[Square] at (-46.250000, 27.700000) {};
\node[Kanji] at (-46.250000, 28.200000) {族};
\node[Onyomi] at (-46.200000, 27.800000) {ゾク};
\node[Meaning] at (-46.250000, 29.450000) {tribe};
\node[Square] at (-44.200000, 27.700000) {};
\node[Kanji] at (-44.200000, 28.200000) {熱};
\node[Onyomi] at (-44.150000, 27.800000) {ネツ};
\node[Kunyomi] at (-44.250000, 27.800000) {あつ.い};
\node[Meaning] at (-44.200000, 29.450000) {heat};
\node[Square] at (-42.150000, 27.700000) {};
\node[Kanji] at (-42.150000, 28.200000) {報};
\node[Onyomi] at (-42.100000, 27.800000) {ホウ};
\node[Kunyomi] at (-42.200000, 27.800000) {むく.いる};
\node[Meaning] at (-42.150000, 29.450000) {news};
\node[Square] at (-40.100000, 27.700000) {};
\node[Kanji] at (-40.100000, 28.200000) {政};
\node[Onyomi] at (-40.050000, 27.800000) {セイ};
\node[Meaning] at (-40.100000, 29.450000) {politics};
\node[Square] at (-38.050000, 27.700000) {};
\node[Kanji] at (-38.050000, 28.200000) {側};
\node[Onyomi] at (-38.000000, 27.800000) {ソク};
\node[Kunyomi] at (-38.100000, 27.800000) {がわ};
\node[Meaning] at (-38.050000, 29.450000) {side};
\node[Square] at (-36.000000, 27.700000) {};
\node[Kanji] at (-36.000000, 28.200000) {質};
\node[Onyomi] at (-35.950000, 27.800000) {シツ};
\node[Meaning] at (-36.000000, 29.450000) {quality};
\node[Square] at (-33.950000, 27.700000) {};
\node[Kanji] at (-33.950000, 28.200000) {扉};
\node[Onyomi] at (-33.900000, 27.800000) {ヒ};
\node[Kunyomi] at (-34.000000, 27.800000) {とびら};
\node[Meaning] at (-33.950000, 29.450000) {front door};
\node[Square] at (-31.900000, 27.700000) {};
\node[Kanji] at (-31.900000, 28.200000) {失};
\node[Onyomi] at (-31.850000, 27.800000) {シツ};
\node[Kunyomi] at (-31.950000, 27.800000) {うしな.う};
\node[Meaning] at (-31.900000, 29.450000) {fault};
\node[Square] at (-29.850000, 27.700000) {};
\node[Kanji] at (-29.850000, 28.200000) {品};
\node[Onyomi] at (-29.800000, 27.800000) {ヒン};
\node[Kunyomi] at (-29.900000, 27.800000) {しな};
\node[Meaning] at (-29.850000, 29.450000) {product};
\node[Square] at (-27.800000, 27.700000) {};
\node[Kanji] at (-27.800000, 28.200000) {府};
\node[Onyomi] at (-27.750000, 27.800000) {フ};
\node[Meaning] at (-27.800000, 29.450000) {government};
\node[Square] at (-25.750000, 27.700000) {};
\node[Kanji] at (-25.750000, 28.200000) {難};
\node[Onyomi] at (-25.700000, 27.800000) {ナン};
\node[Kunyomi] at (-25.800000, 27.800000) {むずか.しい};
\node[Meaning] at (-25.750000, 29.450000) {difficult};
\node[Square] at (-23.700000, 27.700000) {};
\node[Kanji] at (-23.700000, 28.200000) {深};
\node[Onyomi] at (-23.650000, 27.800000) {シン};
\node[Kunyomi] at (-23.750000, 27.800000) {ふか.*};
\node[Meaning] at (-23.700000, 29.450000) {deep};
\node[Square] at (-21.650000, 27.700000) {};
\node[Kanji] at (-21.650000, 28.200000) {害};
\node[Onyomi] at (-21.600000, 27.800000) {ガイ};
\node[Meaning] at (-21.650000, 29.450000) {damage};
\node[Square] at (-19.600000, 27.700000) {};
\node[Kanji] at (-19.600000, 28.200000) {究};
\node[Onyomi] at (-19.550000, 27.800000) {キュウ};
\node[Kunyomi] at (-19.650000, 27.800000) {きわ.める};
\node[Meaning] at (-19.600000, 29.450000) {research};
\node[Square] at (-17.550000, 27.700000) {};
\node[Kanji] at (-17.550000, 28.200000) {都};
\node[Onyomi] at (-17.500000, 27.800000) {ト};
\node[Kunyomi] at (-17.600000, 27.800000) {みやこ};
\node[Meaning] at (-17.550000, 29.450000) {metropolis};
\node[Square] at (-15.500000, 27.700000) {};
\node[Kanji] at (-15.500000, 28.200000) {活};
\node[Onyomi] at (-15.450000, 27.800000) {カツ};
\node[Meaning] at (-15.500000, 29.450000) {lively};
\node[Square] at (-13.450000, 27.700000) {};
\node[Kanji] at (-13.450000, 28.200000) {遠};
\node[Onyomi] at (-13.400000, 27.800000) {エン};
\node[Kunyomi] at (-13.500000, 27.800000) {とお};
\node[Meaning] at (-13.450000, 29.450000) {far};
\node[Square] at (-11.400000, 27.700000) {};
\node[Kanji] at (-11.400000, 28.200000) {渡};
\node[Onyomi] at (-11.350000, 27.800000) {ト};
\node[Kunyomi] at (-11.450000, 27.800000) {わた};
\node[Meaning] at (-11.400000, 29.450000) {transit};
\node[Square] at (-9.350000, 27.700000) {};
\node[Kanji] at (-9.350000, 28.200000) {太};
\node[Onyomi] at (-9.300000, 27.800000) {タイ};
\node[Kunyomi] at (-9.400000, 27.800000) {ふと.い};
\node[Meaning] at (-9.350000, 29.450000) {fat};
\node[Square] at (-7.300000, 27.700000) {};
\node[Kanji] at (-7.300000, 28.200000) {田};
\node[Onyomi] at (-7.250000, 27.800000) {デン};
\node[Kunyomi] at (-7.350000, 27.800000) {た};
\node[Meaning] at (-7.300000, 29.450000) {rice paddy};
\node[Square] at (-5.250000, 27.700000) {};
\node[Kanji] at (-5.250000, 28.200000) {友};
\node[Onyomi] at (-5.200000, 27.800000) {ユウ};
\node[Kunyomi] at (-5.300000, 27.800000) {とも};
\node[Meaning] at (-5.250000, 29.450000) {friend};
\node[Square] at (-3.200000, 27.700000) {};
\node[Kanji] at (-3.200000, 28.200000) {野};
\node[Onyomi] at (-3.150000, 27.800000) {ヤ};
\node[Kunyomi] at (-3.250000, 27.800000) {の};
\node[Meaning] at (-3.200000, 29.450000) {field};
\node[Square] at (-1.150000, 27.700000) {};
\node[Kanji] at (-1.150000, 28.200000) {像};
\node[Onyomi] at (-1.100000, 27.800000) {ゾウ};
\node[Meaning] at (-1.150000, 29.450000) {statue};
\node[Square] at (0.900000, 27.700000) {};
\node[Kanji] at (0.900000, 28.200000) {城};
\node[Onyomi] at (0.950000, 27.800000) {ジョウ};
\node[Kunyomi] at (0.850000, 27.800000) {しろ};
\node[Meaning] at (0.900000, 29.450000) {castle};
\node[Square] at (2.950000, 27.700000) {};
\node[Kanji] at (2.950000, 28.200000) {北};
\node[Onyomi] at (3.000000, 27.800000) {ホク};
\node[Kunyomi] at (2.900000, 27.800000) {きた};
\node[Meaning] at (2.950000, 29.450000) {north};
\node[Square] at (5.000000, 27.700000) {};
\node[Kanji] at (5.000000, 28.200000) {送};
\node[Onyomi] at (5.050000, 27.800000) {ソウ};
\node[Kunyomi] at (4.950000, 27.800000) {おく.る};
\node[Meaning] at (5.000000, 29.450000) {send};
\node[Square] at (7.050000, 27.700000) {};
\node[Kanji] at (7.050000, 28.200000) {士};
\node[Onyomi] at (7.100000, 27.800000) {シ};
\node[Kunyomi] at (7.000000, 27.800000) {さむらい};
\node[Meaning] at (7.050000, 29.450000) {samurai};
\node[Square] at (9.100000, 27.700000) {};
\node[Kanji] at (9.100000, 28.200000) {客};
\node[Onyomi] at (9.150000, 27.800000) {キャク};
\node[Meaning] at (9.100000, 29.450000) {guest};
\node[Square] at (11.150000, 27.700000) {};
\node[Kanji] at (11.150000, 28.200000) {願};
\node[Onyomi] at (11.200000, 27.800000) {ガン};
\node[Kunyomi] at (11.100000, 27.800000) {ねが};
\node[Meaning] at (11.150000, 29.450000) {request};
\node[Square] at (13.200000, 27.700000) {};
\node[Kanji] at (13.200000, 28.200000) {速};
\node[Onyomi] at (13.250000, 27.800000) {ソク};
\node[Kunyomi] at (13.150000, 27.800000) {はや.い};
\node[Meaning] at (13.200000, 29.450000) {fast};
\node[Square] at (15.250000, 27.700000) {};
\node[Kanji] at (15.250000, 28.200000) {差};
\node[Onyomi] at (15.300000, 27.800000) {サ};
\node[Kunyomi] at (15.200000, 27.800000) {さ};
\node[Meaning] at (15.250000, 29.450000) {distinction};
\node[Square] at (17.300000, 27.700000) {};
\node[Kanji] at (17.300000, 28.200000) {寝};
\node[Onyomi] at (17.350000, 27.800000) {シン};
\node[Kunyomi] at (17.250000, 27.800000) {ね};
\node[Meaning] at (17.300000, 29.450000) {lie down};
\node[Square] at (19.350000, 27.700000) {};
\node[Kanji] at (19.350000, 28.200000) {守};
\node[Onyomi] at (19.400000, 27.800000) {ス};
\node[Kunyomi] at (19.300000, 27.800000) {まも.る};
\node[Meaning] at (19.350000, 29.450000) {protect};
\node[Square] at (21.400000, 27.700000) {};
\node[Kanji] at (21.400000, 28.200000) {験};
\node[Onyomi] at (21.450000, 27.800000) {ケン};
\node[Kunyomi] at (21.350000, 27.800000) {ため};
\node[Meaning] at (21.400000, 29.450000) {test};
\node[Square] at (23.450000, 27.700000) {};
\node[Kanji] at (23.450000, 28.200000) {羽};
\node[Kunyomi] at (23.400000, 27.800000) {はね};
\node[Meaning] at (23.450000, 29.450000) {feather};
\node[Square] at (25.500000, 27.700000) {};
\node[Kanji] at (25.500000, 28.200000) {康};
\node[Onyomi] at (25.550000, 27.800000) {コウ};
\node[Meaning] at (25.500000, 29.450000) {health};
\node[Square] at (27.550000, 27.700000) {};
\node[Kanji] at (27.550000, 28.200000) {丈};
\node[Onyomi] at (27.600000, 27.800000) {ジョウ};
\node[Kunyomi] at (27.500000, 27.800000) {たけ};
\node[Meaning] at (27.550000, 29.450000) {height};
\node[Square] at (29.600000, 27.700000) {};
\node[Kanji] at (29.600000, 28.200000) {命};
\node[Onyomi] at (29.650000, 27.800000) {メイ};
\node[Kunyomi] at (29.550000, 27.800000) {いのち};
\node[Meaning] at (29.600000, 29.450000) {fate};
\node[Square] at (31.650000, 27.700000) {};
\node[Kanji] at (31.650000, 28.200000) {察};
\node[Onyomi] at (31.700000, 27.800000) {サツ};
\node[Kunyomi] at (31.600000, 27.800000) {さっ.する};
\node[Meaning] at (31.650000, 29.450000) {guess};
\node[Square] at (33.700000, 27.700000) {};
\node[Kanji] at (33.700000, 28.200000) {絶};
\node[Onyomi] at (33.750000, 27.800000) {ゼツ};
\node[Kunyomi] at (33.650000, 27.800000) {た.*};
\node[Meaning] at (33.700000, 29.450000) {extinction};
\node[Square] at (35.750000, 27.700000) {};
\node[Kanji] at (35.750000, 28.200000) {図};
\node[Onyomi] at (35.800000, 27.800000) {ズ};
\node[Kunyomi] at (35.700000, 27.800000) {え};
\node[Meaning] at (35.750000, 29.450000) {diagram};
\node[Square] at (37.800000, 27.700000) {};
\node[Kanji] at (37.800000, 28.200000) {驚};
\node[Onyomi] at (37.850000, 27.800000) {キョウ};
\node[Kunyomi] at (37.750000, 27.800000) {おどろ.*};
\node[Meaning] at (37.800000, 29.450000) {surprised};
\node[Square] at (39.850000, 27.700000) {};
\node[Kanji] at (39.850000, 28.200000) {利};
\node[Onyomi] at (39.900000, 27.800000) {リ};
\node[Kunyomi] at (39.800000, 27.800000) {き.く};
\node[Meaning] at (39.850000, 29.450000) {profit};
\node[Square] at (41.900000, 27.700000) {};
\node[Kanji] at (41.900000, 28.200000) {特};
\node[Onyomi] at (41.950000, 27.800000) {トク};
\node[Meaning] at (41.900000, 29.450000) {special};
\node[Square] at (43.950000, 27.700000) {};
\node[Kanji] at (43.950000, 28.200000) {怖};
\node[Onyomi] at (44.000000, 27.800000) {フ};
\node[Kunyomi] at (43.900000, 27.800000) {こわ.*};
\node[Meaning] at (43.950000, 29.450000) {scary};
\node[Square] at (46.000000, 27.700000) {};
\node[Kanji] at (46.000000, 28.200000) {浮};
\node[Onyomi] at (46.050000, 27.800000) {フ};
\node[Kunyomi] at (45.950000, 27.800000) {う};
\node[Meaning] at (46.000000, 29.450000) {float};
\node[Square] at (48.050000, 27.700000) {};
\node[Kanji] at (48.050000, 28.200000) {査};
\node[Onyomi] at (48.100000, 27.800000) {サ};
\node[Meaning] at (48.050000, 29.450000) {inspect};
\node[Square] at (50.100000, 27.700000) {};
\node[Kanji] at (50.100000, 28.200000) {椅};
\node[Onyomi] at (50.150000, 27.800000) {イ};
\node[Meaning] at (50.100000, 29.450000) {chair};
\node[Square] at (52.150000, 27.700000) {};
\node[Kanji] at (52.150000, 28.200000) {化};
\node[Onyomi] at (52.200000, 27.800000) {カ};
\node[Kunyomi] at (52.100000, 27.800000) {ば.ける};
\node[Meaning] at (52.150000, 29.450000) {change};
\node[Square] at (54.200000, 27.700000) {};
\node[Kanji] at (54.200000, 28.200000) {主};
\node[Onyomi] at (54.250000, 27.800000) {シュ};
\node[Kunyomi] at (54.150000, 27.800000) {おも};
\node[Meaning] at (54.200000, 29.450000) {master};
\node[Square] at (56.250000, 27.700000) {};
\node[Kanji] at (56.250000, 28.200000) {団};
\node[Onyomi] at (56.300000, 27.800000) {ダン};
\node[Meaning] at (56.250000, 29.450000) {group};
\node[Meaning] at (-58.500000, 28.250000) {74.20\%};
\node[Square] at (-56.500000, 25.650000) {};
\node[Kanji] at (-56.500000, 26.150000) {細};
\node[Onyomi] at (-56.450000, 25.750000) {サイ};
\node[Kunyomi] at (-56.550000, 25.750000) {ほそ};
\node[Meaning] at (-56.500000, 27.400000) {thin};
\node[Square] at (-54.450000, 25.650000) {};
\node[Kanji] at (-54.450000, 26.150000) {神};
\node[Onyomi] at (-54.400000, 25.750000) {シン};
\node[Kunyomi] at (-54.500000, 25.750000) {かみ};
\node[Meaning] at (-54.450000, 27.400000) {god};
\node[Square] at (-52.400000, 25.650000) {};
\node[Kanji] at (-52.400000, 26.150000) {加};
\node[Onyomi] at (-52.350000, 25.750000) {カ};
\node[Kunyomi] at (-52.450000, 25.750000) {くわ.える};
\node[Meaning] at (-52.400000, 27.400000) {add};
\node[Square] at (-50.350000, 25.650000) {};
\node[Kanji] at (-50.350000, 26.150000) {形};
\node[Onyomi] at (-50.300000, 25.750000) {ケイ};
\node[Kunyomi] at (-50.400000, 25.750000) {かた};
\node[Meaning] at (-50.350000, 27.400000) {shape};
\node[Square] at (-48.300000, 25.650000) {};
\node[Kanji] at (-48.300000, 26.150000) {伸};
\node[Onyomi] at (-48.250000, 25.750000) {シン};
\node[Kunyomi] at (-48.350000, 25.750000) {の};
\node[Meaning] at (-48.300000, 27.400000) {stretch};
\node[Square] at (-46.250000, 25.650000) {};
\node[Kanji] at (-46.250000, 26.150000) {根};
\node[Onyomi] at (-46.200000, 25.750000) {コン};
\node[Kunyomi] at (-46.300000, 25.750000) {ね};
\node[Meaning] at (-46.250000, 27.400000) {root};
\node[Square] at (-44.200000, 25.650000) {};
\node[Kanji] at (-44.200000, 26.150000) {鼻};
\node[Onyomi] at (-44.150000, 25.750000) {ビ};
\node[Kunyomi] at (-44.250000, 25.750000) {はな};
\node[Meaning] at (-44.200000, 27.400000) {nose};
\node[Square] at (-42.150000, 25.650000) {};
\node[Kanji] at (-42.150000, 26.150000) {写};
\node[Onyomi] at (-42.100000, 25.750000) {シャ};
\node[Kunyomi] at (-42.200000, 25.750000) {うつ.す};
\node[Meaning] at (-42.150000, 27.400000) {copy};
\node[Square] at (-40.100000, 25.650000) {};
\node[Kanji] at (-40.100000, 26.150000) {故};
\node[Onyomi] at (-40.050000, 25.750000) {コ};
\node[Kunyomi] at (-40.150000, 25.750000) {ゆえ};
\node[Meaning] at (-40.100000, 27.400000) {circumstance};
\node[Square] at (-38.050000, 25.650000) {};
\node[Kanji] at (-38.050000, 26.150000) {夢};
\node[Onyomi] at (-38.000000, 25.750000) {ム};
\node[Kunyomi] at (-38.100000, 25.750000) {ゆめ};
\node[Meaning] at (-38.050000, 27.400000) {dream};
\node[Square] at (-36.000000, 25.650000) {};
\node[Kanji] at (-36.000000, 26.150000) {期};
\node[Onyomi] at (-35.950000, 25.750000) {キ};
\node[Meaning] at (-36.000000, 27.400000) {period of time};
\node[Square] at (-33.950000, 25.650000) {};
\node[Kanji] at (-33.950000, 26.150000) {曜};
\node[Onyomi] at (-33.900000, 25.750000) {ヨウ};
\node[Meaning] at (-33.950000, 27.400000) {weekday};
\node[Square] at (-31.900000, 25.650000) {};
\node[Kanji] at (-31.900000, 26.150000) {宙};
\node[Onyomi] at (-31.850000, 25.750000) {チュウ};
\node[Meaning] at (-31.900000, 27.400000) {mid air};
\node[Square] at (-29.850000, 25.650000) {};
\node[Kanji] at (-29.850000, 26.150000) {床};
\node[Onyomi] at (-29.800000, 25.750000) {ショウ};
\node[Kunyomi] at (-29.900000, 25.750000) {ゆか};
\node[Meaning] at (-29.850000, 27.400000) {floor};
\node[Square] at (-27.800000, 25.650000) {};
\node[Kanji] at (-27.800000, 26.150000) {優};
\node[Onyomi] at (-27.750000, 25.750000) {ユウ};
\node[Kunyomi] at (-27.850000, 25.750000) {やさ.しい};
\node[Meaning] at (-27.800000, 27.400000) {superior};
\node[Square] at (-25.750000, 25.650000) {};
\node[Kanji] at (-25.750000, 26.150000) {星};
\node[Onyomi] at (-25.700000, 25.750000) {セイ};
\node[Kunyomi] at (-25.800000, 25.750000) {ほし};
\node[Meaning] at (-25.750000, 27.400000) {star};
\node[Square] at (-23.700000, 25.650000) {};
\node[Kanji] at (-23.700000, 26.150000) {撃};
\node[Onyomi] at (-23.650000, 25.750000) {ゲキ};
\node[Kunyomi] at (-23.750000, 25.750000) {う.つ};
\node[Meaning] at (-23.700000, 27.400000) {attack};
\node[Square] at (-21.650000, 25.650000) {};
\node[Kanji] at (-21.650000, 26.150000) {興};
\node[Onyomi] at (-21.600000, 25.750000) {キョウ};
\node[Meaning] at (-21.650000, 27.400000) {interest};
\node[Square] at (-19.600000, 25.650000) {};
\node[Kanji] at (-19.600000, 26.150000) {忘};
\node[Onyomi] at (-19.550000, 25.750000) {ボウ};
\node[Kunyomi] at (-19.650000, 25.750000) {わす.れる};
\node[Meaning] at (-19.600000, 27.400000) {forget};
\node[Square] at (-17.550000, 25.650000) {};
\node[Kanji] at (-17.550000, 26.150000) {銀};
\node[Onyomi] at (-17.500000, 25.750000) {ギン};
\node[Meaning] at (-17.550000, 27.400000) {silver};
\node[Square] at (-15.500000, 25.650000) {};
\node[Kanji] at (-15.500000, 26.150000) {休};
\node[Onyomi] at (-15.450000, 25.750000) {キュウ};
\node[Kunyomi] at (-15.550000, 25.750000) {やす.み};
\node[Meaning] at (-15.500000, 27.400000) {rest};
\node[Square] at (-13.450000, 25.650000) {};
\node[Kanji] at (-13.450000, 26.150000) {氏};
\node[Onyomi] at (-13.400000, 25.750000) {シ};
\node[Kunyomi] at (-13.500000, 25.750000) {うじ};
\node[Meaning] at (-13.450000, 27.400000) {family name};
\node[Square] at (-11.400000, 25.650000) {};
\node[Kanji] at (-11.400000, 26.150000) {再};
\node[Onyomi] at (-11.350000, 25.750000) {サ};
\node[Kunyomi] at (-11.450000, 25.750000) {ふたた.び};
\node[Meaning] at (-11.400000, 27.400000) {again};
\node[Square] at (-9.350000, 25.650000) {};
\node[Kanji] at (-9.350000, 26.150000) {勢};
\node[Onyomi] at (-9.300000, 25.750000) {セイ};
\node[Kunyomi] at (-9.400000, 25.750000) {いきお.い};
\node[Meaning] at (-9.350000, 27.400000) {force};
\node[Square] at (-7.300000, 25.650000) {};
\node[Kanji] at (-7.300000, 26.150000) {由};
\node[Onyomi] at (-7.250000, 25.750000) {ユウ};
\node[Kunyomi] at (-7.350000, 25.750000) {よし};
\node[Meaning] at (-7.300000, 27.400000) {reason};
\node[Square] at (-5.250000, 25.650000) {};
\node[Kanji] at (-5.250000, 26.150000) {育};
\node[Onyomi] at (-5.200000, 25.750000) {イク};
\node[Kunyomi] at (-5.300000, 25.750000) {そだ};
\node[Meaning] at (-5.250000, 27.400000) {nurture};
\node[Square] at (-3.200000, 25.650000) {};
\node[Kanji] at (-3.200000, 26.150000) {箱};
\node[Kunyomi] at (-3.250000, 25.750000) {はこ};
\node[Meaning] at (-3.200000, 27.400000) {box};
\node[Square] at (-1.150000, 25.650000) {};
\node[Kanji] at (-1.150000, 26.150000) {町};
\node[Onyomi] at (-1.100000, 25.750000) {チョウ};
\node[Kunyomi] at (-1.200000, 25.750000) {まち};
\node[Meaning] at (-1.150000, 27.400000) {town};
\node[Square] at (0.900000, 25.650000) {};
\node[Kanji] at (0.900000, 26.150000) {映};
\node[Onyomi] at (0.950000, 25.750000) {エイ};
\node[Kunyomi] at (0.850000, 25.750000) {うつ};
\node[Meaning] at (0.900000, 27.400000) {reflect};
\node[Square] at (2.950000, 25.650000) {};
\node[Kanji] at (2.950000, 26.150000) {役};
\node[Onyomi] at (3.000000, 25.750000) {ヤク};
\node[Meaning] at (2.950000, 27.400000) {service};
\node[Square] at (5.000000, 25.650000) {};
\node[Kanji] at (5.000000, 26.150000) {午};
\node[Onyomi] at (5.050000, 25.750000) {ゴ};
\node[Meaning] at (5.000000, 27.400000) {noon};
\node[Square] at (7.050000, 25.650000) {};
\node[Kanji] at (7.050000, 26.150000) {毎};
\node[Onyomi] at (7.100000, 25.750000) {マイ};
\node[Kunyomi] at (7.000000, 25.750000) {ごと};
\node[Meaning] at (7.050000, 27.400000) {every};
\node[Square] at (9.100000, 25.650000) {};
\node[Kanji] at (9.100000, 26.150000) {巨};
\node[Onyomi] at (9.150000, 25.750000) {キョ};
\node[Meaning] at (9.100000, 27.400000) {giant};
\node[Square] at (11.150000, 25.650000) {};
\node[Kanji] at (11.150000, 26.150000) {護};
\node[Onyomi] at (11.200000, 25.750000) {ゴ};
\node[Meaning] at (11.150000, 27.400000) {defend};
\node[Square] at (13.200000, 25.650000) {};
\node[Kanji] at (13.200000, 26.150000) {眠};
\node[Onyomi] at (13.250000, 25.750000) {ミン};
\node[Kunyomi] at (13.150000, 25.750000) {ねむ.*};
\node[Meaning] at (13.200000, 27.400000) {sleep};
\node[Square] at (15.250000, 25.650000) {};
\node[Kanji] at (15.250000, 26.150000) {際};
\node[Onyomi] at (15.300000, 25.750000) {サイ};
\node[Kunyomi] at (15.200000, 25.750000) {きわ};
\node[Meaning] at (15.250000, 27.400000) {occasion};
\node[Square] at (17.300000, 25.650000) {};
\node[Kanji] at (17.300000, 26.150000) {黙};
\node[Onyomi] at (17.350000, 25.750000) {モク};
\node[Kunyomi] at (17.250000, 25.750000) {だま.る};
\node[Meaning] at (17.300000, 27.400000) {shut up};
\node[Square] at (19.350000, 25.650000) {};
\node[Kanji] at (19.350000, 26.150000) {片};
\node[Onyomi] at (19.400000, 25.750000) {ヘン};
\node[Kunyomi] at (19.300000, 25.750000) {かた};
\node[Meaning] at (19.350000, 27.400000) {one sided};
\node[Square] at (21.400000, 25.650000) {};
\node[Kanji] at (21.400000, 26.150000) {談};
\node[Onyomi] at (21.450000, 25.750000) {ダン};
\node[Meaning] at (21.400000, 27.400000) {discuss};
\node[Square] at (23.450000, 25.650000) {};
\node[Kanji] at (23.450000, 26.150000) {注};
\node[Onyomi] at (23.500000, 25.750000) {チュウ};
\node[Kunyomi] at (23.400000, 25.750000) {そそ.ぐ};
\node[Meaning] at (23.450000, 27.400000) {pour};
\node[Square] at (25.500000, 25.650000) {};
\node[Kanji] at (25.500000, 26.150000) {建};
\node[Onyomi] at (25.550000, 25.750000) {ケン};
\node[Kunyomi] at (25.450000, 25.750000) {た.*};
\node[Meaning] at (25.500000, 27.400000) {build};
\node[Square] at (27.550000, 25.650000) {};
\node[Kanji] at (27.550000, 26.150000) {倒};
\node[Onyomi] at (27.600000, 25.750000) {トウ};
\node[Kunyomi] at (27.500000, 25.750000) {たお.す};
\node[Meaning] at (27.550000, 27.400000) {overthrow};
\node[Square] at (29.600000, 25.650000) {};
\node[Kanji] at (29.600000, 26.150000) {登};
\node[Onyomi] at (29.650000, 25.750000) {トウ};
\node[Kunyomi] at (29.550000, 25.750000) {のぼ.る};
\node[Meaning] at (29.600000, 27.400000) {climb};
\node[Square] at (31.650000, 25.650000) {};
\node[Kanji] at (31.650000, 26.150000) {技};
\node[Onyomi] at (31.700000, 25.750000) {ギ};
\node[Kunyomi] at (31.600000, 25.750000) {わざ};
\node[Meaning] at (31.650000, 27.400000) {skill};
\node[Square] at (33.700000, 25.650000) {};
\node[Kanji] at (33.700000, 26.150000) {古};
\node[Onyomi] at (33.750000, 25.750000) {コ};
\node[Kunyomi] at (33.650000, 25.750000) {ふる.い};
\node[Meaning] at (33.700000, 27.400000) {old};
\node[Square] at (35.750000, 25.650000) {};
\node[Kanji] at (35.750000, 26.150000) {科};
\node[Onyomi] at (35.800000, 25.750000) {カ};
\node[Meaning] at (35.750000, 27.400000) {science};
\node[Square] at (37.800000, 25.650000) {};
\node[Kanji] at (37.800000, 26.150000) {産};
\node[Onyomi] at (37.850000, 25.750000) {サン};
\node[Kunyomi] at (37.750000, 25.750000) {う.む};
\node[Meaning] at (37.800000, 27.400000) {give birth};
\node[Square] at (39.850000, 25.650000) {};
\node[Kanji] at (39.850000, 26.150000) {魂};
\node[Onyomi] at (39.900000, 25.750000) {コン};
\node[Kunyomi] at (39.800000, 25.750000) {たましい};
\node[Meaning] at (39.850000, 27.400000) {soul};
\node[Square] at (41.900000, 25.650000) {};
\node[Kanji] at (41.900000, 26.150000) {土};
\node[Onyomi] at (41.950000, 25.750000) {ド};
\node[Kunyomi] at (41.850000, 25.750000) {つち};
\node[Meaning] at (41.900000, 27.400000) {soil};
\node[Square] at (43.950000, 25.650000) {};
\node[Kanji] at (43.950000, 26.150000) {川};
\node[Onyomi] at (44.000000, 25.750000) {セン};
\node[Kunyomi] at (43.900000, 25.750000) {かわ};
\node[Meaning] at (43.950000, 27.400000) {river};
\node[Square] at (46.000000, 25.650000) {};
\node[Kanji] at (46.000000, 26.150000) {庭};
\node[Onyomi] at (46.050000, 25.750000) {テイ};
\node[Kunyomi] at (45.950000, 25.750000) {にわ};
\node[Meaning] at (46.000000, 27.400000) {garden};
\node[Square] at (48.050000, 25.650000) {};
\node[Kanji] at (48.050000, 26.150000) {穴};
\node[Onyomi] at (48.100000, 25.750000) {ケツ};
\node[Kunyomi] at (48.000000, 25.750000) {あな};
\node[Meaning] at (48.050000, 27.400000) {hole};
\node[Square] at (50.100000, 25.650000) {};
\node[Kanji] at (50.100000, 26.150000) {美};
\node[Onyomi] at (50.150000, 25.750000) {ビ};
\node[Kunyomi] at (50.050000, 25.750000) {うつく.しい};
\node[Meaning] at (50.100000, 27.400000) {beauty};
\node[Square] at (52.150000, 25.650000) {};
\node[Kanji] at (52.150000, 26.150000) {破};
\node[Onyomi] at (52.200000, 25.750000) {ハ};
\node[Kunyomi] at (52.100000, 25.750000) {やぶ.*};
\node[Meaning] at (52.150000, 27.400000) {tear};
\node[Square] at (54.200000, 25.650000) {};
\node[Kanji] at (54.200000, 26.150000) {交};
\node[Onyomi] at (54.250000, 25.750000) {コウ};
\node[Kunyomi] at (54.150000, 25.750000) {まじ};
\node[Meaning] at (54.200000, 27.400000) {mix};
\node[Square] at (56.250000, 25.650000) {};
\node[Kanji] at (56.250000, 26.150000) {防};
\node[Onyomi] at (56.300000, 25.750000) {ボウ};
\node[Kunyomi] at (56.200000, 25.750000) {ふせ.ぐ};
\node[Meaning] at (56.250000, 27.400000) {prevent};
\node[Meaning] at (-58.500000, 26.200000) {77.33\%};
\node[Square] at (-56.500000, 23.600000) {};
\node[Kanji] at (-56.500000, 24.100000) {件};
\node[Onyomi] at (-56.450000, 23.700000) {ケン};
\node[Meaning] at (-56.500000, 25.350000) {matter};
\node[Square] at (-54.450000, 23.600000) {};
\node[Kanji] at (-54.450000, 24.100000) {激};
\node[Onyomi] at (-54.400000, 23.700000) {ゲキ};
\node[Kunyomi] at (-54.500000, 23.700000) {はげ.しい};
\node[Meaning] at (-54.450000, 25.350000) {fierce};
\node[Square] at (-52.400000, 23.600000) {};
\node[Kanji] at (-52.400000, 24.100000) {館};
\node[Onyomi] at (-52.350000, 23.700000) {カン};
\node[Meaning] at (-52.400000, 25.350000) {public building};
\node[Square] at (-50.350000, 23.600000) {};
\node[Kanji] at (-50.350000, 24.100000) {端};
\node[Onyomi] at (-50.300000, 23.700000) {タン};
\node[Kunyomi] at (-50.400000, 23.700000) {はし};
\node[Meaning] at (-50.350000, 25.350000) {edge};
\node[Square] at (-48.300000, 23.600000) {};
\node[Kanji] at (-48.300000, 24.100000) {払};
\node[Kunyomi] at (-48.350000, 23.700000) {はら};
\node[Meaning] at (-48.300000, 25.350000) {pay};
\node[Square] at (-46.250000, 23.600000) {};
\node[Kanji] at (-46.250000, 24.100000) {状};
\node[Onyomi] at (-46.200000, 23.700000) {ジョウ};
\node[Meaning] at (-46.250000, 25.350000) {condition};
\node[Square] at (-44.200000, 23.600000) {};
\node[Kanji] at (-44.200000, 24.100000) {低};
\node[Onyomi] at (-44.150000, 23.700000) {テイ};
\node[Kunyomi] at (-44.250000, 23.700000) {ひく.い};
\node[Meaning] at (-44.200000, 25.350000) {low};
\node[Square] at (-42.150000, 23.600000) {};
\node[Kanji] at (-42.150000, 24.100000) {望};
\node[Onyomi] at (-42.100000, 23.700000) {ボウ};
\node[Kunyomi] at (-42.200000, 23.700000) {のぞ.む};
\node[Meaning] at (-42.150000, 25.350000) {hope};
\node[Square] at (-40.100000, 23.600000) {};
\node[Kanji] at (-40.100000, 24.100000) {握};
\node[Onyomi] at (-40.050000, 23.700000) {アク};
\node[Kunyomi] at (-40.150000, 23.700000) {にぎ.る};
\node[Meaning] at (-40.100000, 25.350000) {grip};
\node[Square] at (-38.050000, 23.600000) {};
\node[Kanji] at (-38.050000, 24.100000) {皮};
\node[Onyomi] at (-38.000000, 23.700000) {ヒ};
\node[Kunyomi] at (-38.100000, 23.700000) {かわ};
\node[Meaning] at (-38.050000, 25.350000) {skin};
\node[Square] at (-36.000000, 23.600000) {};
\node[Kanji] at (-36.000000, 24.100000) {投};
\node[Onyomi] at (-35.950000, 23.700000) {トウ};
\node[Kunyomi] at (-36.050000, 23.700000) {な};
\node[Meaning] at (-36.000000, 25.350000) {throw};
\node[Square] at (-33.950000, 23.600000) {};
\node[Kanji] at (-33.950000, 24.100000) {悲};
\node[Onyomi] at (-33.900000, 23.700000) {ヒ};
\node[Kunyomi] at (-34.000000, 23.700000) {かな};
\node[Meaning] at (-33.950000, 25.350000) {sad};
\node[Square] at (-31.900000, 23.600000) {};
\node[Kanji] at (-31.900000, 24.100000) {駅};
\node[Onyomi] at (-31.850000, 23.700000) {エキ};
\node[Meaning] at (-31.900000, 25.350000) {station};
\node[Square] at (-29.850000, 23.600000) {};
\node[Kanji] at (-29.850000, 24.100000) {練};
\node[Onyomi] at (-29.800000, 23.700000) {レン};
\node[Kunyomi] at (-29.900000, 23.700000) {ね};
\node[Meaning] at (-29.850000, 25.350000) {practice};
\node[Square] at (-27.800000, 23.600000) {};
\node[Kanji] at (-27.800000, 24.100000) {習};
\node[Onyomi] at (-27.750000, 23.700000) {シュウ};
\node[Kunyomi] at (-27.850000, 23.700000) {なら.う};
\node[Meaning] at (-27.800000, 25.350000) {learn};
\node[Square] at (-25.750000, 23.600000) {};
\node[Kanji] at (-25.750000, 24.100000) {夏};
\node[Onyomi] at (-25.700000, 23.700000) {ゲ};
\node[Kunyomi] at (-25.800000, 23.700000) {なつ};
\node[Meaning] at (-25.750000, 25.350000) {summer};
\node[Square] at (-23.700000, 23.600000) {};
\node[Kanji] at (-23.700000, 24.100000) {泣};
\node[Onyomi] at (-23.650000, 23.700000) {キュウ};
\node[Kunyomi] at (-23.750000, 23.700000) {な};
\node[Meaning] at (-23.700000, 25.350000) {cry};
\node[Square] at (-21.650000, 23.600000) {};
\node[Kanji] at (-21.650000, 24.100000) {値};
\node[Onyomi] at (-21.600000, 23.700000) {チ};
\node[Kunyomi] at (-21.700000, 23.700000) {ね};
\node[Meaning] at (-21.650000, 25.350000) {value};
\node[Square] at (-19.600000, 23.600000) {};
\node[Kanji] at (-19.600000, 24.100000) {苦};
\node[Onyomi] at (-19.550000, 23.700000) {ク};
\node[Kunyomi] at (-19.650000, 23.700000) {くる};
\node[Meaning] at (-19.600000, 25.350000) {suffering};
\node[Square] at (-17.550000, 23.600000) {};
\node[Kanji] at (-17.550000, 24.100000) {院};
\node[Onyomi] at (-17.500000, 23.700000) {イン};
\node[Meaning] at (-17.550000, 25.350000) {institution};
\node[Square] at (-15.500000, 23.600000) {};
\node[Kanji] at (-15.500000, 24.100000) {働};
\node[Onyomi] at (-15.450000, 23.700000) {ドウ};
\node[Kunyomi] at (-15.550000, 23.700000) {はたら.*};
\node[Meaning] at (-15.500000, 25.350000) {work};
\node[Square] at (-13.450000, 23.600000) {};
\node[Kanji] at (-13.450000, 24.100000) {嫌};
\node[Onyomi] at (-13.400000, 23.700000) {ケン};
\node[Kunyomi] at (-13.500000, 23.700000) {いや};
\node[Meaning] at (-13.450000, 25.350000) {dislike};
\node[Square] at (-11.400000, 23.600000) {};
\node[Kanji] at (-11.400000, 24.100000) {樹};
\node[Onyomi] at (-11.350000, 23.700000) {ジュ};
\node[Kunyomi] at (-11.450000, 23.700000) {き};
\node[Meaning] at (-11.400000, 25.350000) {wood};
\node[Square] at (-9.350000, 23.600000) {};
\node[Kanji] at (-9.350000, 24.100000) {旅};
\node[Onyomi] at (-9.300000, 23.700000) {リョ};
\node[Kunyomi] at (-9.400000, 23.700000) {たび};
\node[Meaning] at (-9.350000, 25.350000) {trip};
\node[Square] at (-7.300000, 23.600000) {};
\node[Kanji] at (-7.300000, 24.100000) {秘};
\node[Onyomi] at (-7.250000, 23.700000) {ヒ};
\node[Kunyomi] at (-7.350000, 23.700000) {ひ.める};
\node[Meaning] at (-7.300000, 25.350000) {secret};
\node[Square] at (-5.250000, 23.600000) {};
\node[Kanji] at (-5.250000, 24.100000) {寮};
\node[Onyomi] at (-5.200000, 23.700000) {リョウ};
\node[Meaning] at (-5.250000, 25.350000) {dormitory};
\node[Square] at (-3.200000, 23.600000) {};
\node[Kanji] at (-3.200000, 24.100000) {戸};
\node[Onyomi] at (-3.150000, 23.700000) {コ};
\node[Kunyomi] at (-3.250000, 23.700000) {と};
\node[Meaning] at (-3.200000, 25.350000) {door};
\node[Square] at (-1.150000, 23.600000) {};
\node[Kanji] at (-1.150000, 24.100000) {路};
\node[Onyomi] at (-1.100000, 23.700000) {ロ};
\node[Kunyomi] at (-1.200000, 23.700000) {じ};
\node[Meaning] at (-1.150000, 25.350000) {road};
\node[Square] at (0.900000, 23.600000) {};
\node[Kanji] at (0.900000, 24.100000) {保};
\node[Onyomi] at (0.950000, 23.700000) {ホ};
\node[Kunyomi] at (0.850000, 23.700000) {たも.つ};
\node[Meaning] at (0.900000, 25.350000) {preserve};
\node[Square] at (2.950000, 23.600000) {};
\node[Kanji] at (2.950000, 24.100000) {精};
\node[Onyomi] at (3.000000, 23.700000) {セイ};
\node[Meaning] at (2.950000, 25.350000) {spirit};
\node[Square] at (5.000000, 23.600000) {};
\node[Kanji] at (5.000000, 24.100000) {肩};
\node[Onyomi] at (5.050000, 23.700000) {ケン};
\node[Kunyomi] at (4.950000, 23.700000) {かた};
\node[Meaning] at (5.000000, 25.350000) {shoulder};
\node[Square] at (7.050000, 23.600000) {};
\node[Kanji] at (7.050000, 24.100000) {字};
\node[Onyomi] at (7.100000, 23.700000) {ジ};
\node[Meaning] at (7.050000, 25.350000) {letter};
\node[Square] at (9.100000, 23.600000) {};
\node[Kanji] at (9.100000, 24.100000) {鳥};
\node[Onyomi] at (9.150000, 23.700000) {チョウ};
\node[Kunyomi] at (9.050000, 23.700000) {とり};
\node[Meaning] at (9.100000, 25.350000) {bird};
\node[Square] at (11.150000, 23.600000) {};
\node[Kanji] at (11.150000, 24.100000) {園};
\node[Onyomi] at (11.200000, 23.700000) {エン};
\node[Meaning] at (11.150000, 25.350000) {garden};
\node[Square] at (13.200000, 23.600000) {};
\node[Kanji] at (13.200000, 24.100000) {観};
\node[Onyomi] at (13.250000, 23.700000) {カン};
\node[Kunyomi] at (13.150000, 23.700000) {み.る};
\node[Meaning] at (13.200000, 25.350000) {view};
\node[Square] at (15.250000, 23.600000) {};
\node[Kanji] at (15.250000, 24.100000) {有};
\node[Onyomi] at (15.300000, 23.700000) {ユウ};
\node[Kunyomi] at (15.200000, 23.700000) {あ.る};
\node[Meaning] at (15.250000, 25.350000) {have};
\node[Square] at (17.300000, 23.600000) {};
\node[Kanji] at (17.300000, 24.100000) {森};
\node[Onyomi] at (17.350000, 23.700000) {シン};
\node[Kunyomi] at (17.250000, 23.700000) {もり};
\node[Meaning] at (17.300000, 25.350000) {forest};
\node[Square] at (19.350000, 23.600000) {};
\node[Kanji] at (19.350000, 24.100000) {夕};
\node[Onyomi] at (19.400000, 23.700000) {セキ};
\node[Kunyomi] at (19.300000, 23.700000) {ゆう};
\node[Meaning] at (19.350000, 25.350000) {evening};
\node[Square] at (21.400000, 23.600000) {};
\node[Kanji] at (21.400000, 24.100000) {被};
\node[Onyomi] at (21.450000, 23.700000) {ヒ};
\node[Kunyomi] at (21.350000, 23.700000) {かぶ.る};
\node[Meaning] at (21.400000, 25.350000) {incur};
\node[Square] at (23.450000, 23.600000) {};
\node[Kanji] at (23.450000, 24.100000) {廊};
\node[Onyomi] at (23.500000, 23.700000) {ロウ};
\node[Meaning] at (23.450000, 25.350000) {corridor};
\node[Square] at (25.500000, 23.600000) {};
\node[Kanji] at (25.500000, 24.100000) {災};
\node[Onyomi] at (25.550000, 23.700000) {サイ};
\node[Kunyomi] at (25.450000, 23.700000) {わざわ.い};
\node[Meaning] at (25.500000, 25.350000) {disaster};
\node[Square] at (27.550000, 23.600000) {};
\node[Kanji] at (27.550000, 24.100000) {妖};
\node[Onyomi] at (27.600000, 23.700000) {ヨウ};
\node[Kunyomi] at (27.500000, 23.700000) {あや-しい};
\node[Meaning] at (27.550000, 25.350000) {bewitching};
\node[Square] at (29.600000, 23.600000) {};
\node[Kanji] at (29.600000, 24.100000) {丸};
\node[Onyomi] at (29.650000, 23.700000) {ガン};
\node[Kunyomi] at (29.550000, 23.700000) {まる};
\node[Meaning] at (29.600000, 25.350000) {circle};
\node[Square] at (31.650000, 23.600000) {};
\node[Kanji] at (31.650000, 24.100000) {花};
\node[Onyomi] at (31.700000, 23.700000) {カ};
\node[Kunyomi] at (31.600000, 23.700000) {はな};
\node[Meaning] at (31.650000, 25.350000) {flower};
\node[Square] at (33.700000, 23.600000) {};
\node[Kanji] at (33.700000, 24.100000) {港};
\node[Onyomi] at (33.750000, 23.700000) {コウ};
\node[Kunyomi] at (33.650000, 23.700000) {みなと};
\node[Meaning] at (33.700000, 25.350000) {harbor};
\node[Square] at (35.750000, 23.600000) {};
\node[Kanji] at (35.750000, 24.100000) {想};
\node[Onyomi] at (35.800000, 23.700000) {ソウ};
\node[Meaning] at (35.750000, 25.350000) {concept};
\node[Square] at (37.800000, 23.600000) {};
\node[Kanji] at (37.800000, 24.100000) {宇};
\node[Onyomi] at (37.850000, 23.700000) {ウ};
\node[Meaning] at (37.800000, 25.350000) {outer space};
\node[Square] at (39.850000, 23.600000) {};
\node[Kanji] at (39.850000, 24.100000) {暖};
\node[Onyomi] at (39.900000, 23.700000) {ダン};
\node[Kunyomi] at (39.800000, 23.700000) {あたた.かい};
\node[Meaning] at (39.850000, 25.350000) {warm};
\node[Square] at (41.900000, 23.600000) {};
\node[Kanji] at (41.900000, 24.100000) {壊};
\node[Onyomi] at (41.950000, 23.700000) {カイ};
\node[Kunyomi] at (41.850000, 23.700000) {こわ.*};
\node[Meaning] at (41.900000, 25.350000) {break};
\node[Square] at (43.950000, 23.600000) {};
\node[Kanji] at (43.950000, 24.100000) {触};
\node[Onyomi] at (44.000000, 23.700000) {ショク};
\node[Kunyomi] at (43.900000, 23.700000) {さわ.る};
\node[Meaning] at (43.950000, 25.350000) {touch};
\node[Square] at (46.000000, 23.600000) {};
\node[Kanji] at (46.000000, 24.100000) {工};
\node[Onyomi] at (46.050000, 23.700000) {コウ};
\node[Meaning] at (46.000000, 25.350000) {industry};
\node[Square] at (48.050000, 23.600000) {};
\node[Kanji] at (48.050000, 24.100000) {温};
\node[Onyomi] at (48.100000, 23.700000) {オン};
\node[Kunyomi] at (48.000000, 23.700000) {あたた.*};
\node[Meaning] at (48.050000, 25.350000) {warm};
\node[Square] at (50.100000, 23.600000) {};
\node[Kanji] at (50.100000, 24.100000) {視};
\node[Onyomi] at (50.150000, 23.700000) {シ};
\node[Meaning] at (50.100000, 25.350000) {look at};
\node[Square] at (52.150000, 23.600000) {};
\node[Kanji] at (52.150000, 24.100000) {満};
\node[Onyomi] at (52.200000, 23.700000) {マン};
\node[Kunyomi] at (52.100000, 23.700000) {み};
\node[Meaning] at (52.150000, 25.350000) {full};
\node[Square] at (54.200000, 23.600000) {};
\node[Kanji] at (54.200000, 24.100000) {舞};
\node[Onyomi] at (54.250000, 23.700000) {ブ};
\node[Kunyomi] at (54.150000, 23.700000) {まい};
\node[Meaning] at (54.200000, 25.350000) {dance};
\node[Square] at (56.250000, 23.600000) {};
\node[Kanji] at (56.250000, 24.100000) {得};
\node[Onyomi] at (56.300000, 23.700000) {トク};
\node[Kunyomi] at (56.200000, 23.700000) {え.る};
\node[Meaning] at (56.250000, 25.350000) {acquire};
\node[Meaning] at (-58.500000, 24.150000) {80.05\%};
\node[Square] at (-56.500000, 21.550000) {};
\node[Kanji] at (-56.500000, 22.050000) {巻};
\node[Onyomi] at (-56.450000, 21.650000) {カン};
\node[Kunyomi] at (-56.550000, 21.650000) {ま.く};
\node[Meaning] at (-56.500000, 23.300000) {scroll};
\node[Square] at (-54.450000, 21.550000) {};
\node[Kanji] at (-54.450000, 22.050000) {遺};
\node[Onyomi] at (-54.400000, 21.650000) {イ};
\node[Kunyomi] at (-54.500000, 21.650000) {のこ.す};
\node[Meaning] at (-54.450000, 23.300000) {leave behind};
\node[Square] at (-52.400000, 21.550000) {};
\node[Kanji] at (-52.400000, 22.050000) {杯};
\node[Onyomi] at (-52.350000, 21.650000) {ハイ};
\node[Meaning] at (-52.400000, 23.300000) {cup of liquid};
\node[Square] at (-50.350000, 21.550000) {};
\node[Kanji] at (-50.350000, 22.050000) {詰};
\node[Onyomi] at (-50.300000, 21.650000) {キツ};
\node[Kunyomi] at (-50.400000, 21.650000) {つ};
\node[Meaning] at (-50.350000, 23.300000) {stuffed};
\node[Square] at (-48.300000, 21.550000) {};
\node[Kanji] at (-48.300000, 22.050000) {告};
\node[Onyomi] at (-48.250000, 21.650000) {コク};
\node[Kunyomi] at (-48.350000, 21.650000) {つ.げる};
\node[Meaning] at (-48.300000, 23.300000) {announce};
\node[Square] at (-46.250000, 21.550000) {};
\node[Kanji] at (-46.250000, 22.050000) {駆};
\node[Onyomi] at (-46.200000, 21.650000) {ク};
\node[Kunyomi] at (-46.300000, 21.650000) {か};
\node[Meaning] at (-46.250000, 23.300000) {gallop};
\node[Square] at (-44.200000, 21.550000) {};
\node[Kanji] at (-44.200000, 22.050000) {越};
\node[Onyomi] at (-44.150000, 21.650000) {エツ};
\node[Kunyomi] at (-44.250000, 21.650000) {こ.*};
\node[Meaning] at (-44.200000, 23.300000) {go beyond};
\node[Square] at (-42.150000, 21.550000) {};
\node[Kanji] at (-42.150000, 22.050000) {脇};
\node[Onyomi] at (-42.100000, 21.650000) {キョウ};
\node[Kunyomi] at (-42.200000, 21.650000) {わき};
\node[Meaning] at (-42.150000, 23.300000) {armpit};
\node[Square] at (-40.100000, 21.550000) {};
\node[Kanji] at (-40.100000, 22.050000) {滑};
\node[Onyomi] at (-40.050000, 21.650000) {カツ};
\node[Kunyomi] at (-40.150000, 21.650000) {すべ.る};
\node[Meaning] at (-40.100000, 23.300000) {slippery};
\node[Square] at (-38.050000, 21.550000) {};
\node[Kanji] at (-38.050000, 22.050000) {認};
\node[Onyomi] at (-38.000000, 21.650000) {ニン};
\node[Kunyomi] at (-38.100000, 21.650000) {みと.める};
\node[Meaning] at (-38.050000, 23.300000) {recognize};
\node[Square] at (-36.000000, 21.550000) {};
\node[Kanji] at (-36.000000, 22.050000) {福};
\node[Onyomi] at (-35.950000, 21.650000) {フク};
\node[Meaning] at (-36.000000, 23.300000) {luck};
\node[Square] at (-33.950000, 21.550000) {};
\node[Kanji] at (-33.950000, 22.050000) {良};
\node[Onyomi] at (-33.900000, 21.650000) {リョウ};
\node[Kunyomi] at (-34.000000, 21.650000) {よ};
\node[Meaning] at (-33.950000, 23.300000) {good};
\node[Square] at (-31.900000, 21.550000) {};
\node[Kanji] at (-31.900000, 22.050000) {響};
\node[Onyomi] at (-31.850000, 21.650000) {キョウ};
\node[Kunyomi] at (-31.950000, 21.650000) {ひび.く};
\node[Meaning] at (-31.900000, 23.300000) {echo};
\node[Square] at (-29.850000, 21.550000) {};
\node[Kanji] at (-29.850000, 22.050000) {毛};
\node[Onyomi] at (-29.800000, 21.650000) {モウ};
\node[Kunyomi] at (-29.900000, 21.650000) {け};
\node[Meaning] at (-29.850000, 23.300000) {fur};
\node[Square] at (-27.800000, 21.550000) {};
\node[Kanji] at (-27.800000, 22.050000) {並};
\node[Onyomi] at (-27.750000, 21.650000) {ヘイ};
\node[Kunyomi] at (-27.850000, 21.650000) {なら.*};
\node[Meaning] at (-27.800000, 23.300000) {line up};
\node[Square] at (-25.750000, 21.550000) {};
\node[Kanji] at (-25.750000, 22.050000) {歌};
\node[Onyomi] at (-25.700000, 21.650000) {カ};
\node[Kunyomi] at (-25.800000, 21.650000) {うた};
\node[Meaning] at (-25.750000, 23.300000) {song};
\node[Square] at (-23.700000, 21.550000) {};
\node[Kanji] at (-23.700000, 22.050000) {抱};
\node[Onyomi] at (-23.650000, 21.650000) {ホウ};
\node[Kunyomi] at (-23.750000, 21.650000) {だ};
\node[Meaning] at (-23.700000, 23.300000) {hug};
\node[Square] at (-21.650000, 21.550000) {};
\node[Kanji] at (-21.650000, 22.050000) {雪};
\node[Onyomi] at (-21.600000, 21.650000) {セツ};
\node[Kunyomi] at (-21.700000, 21.650000) {ゆき};
\node[Meaning] at (-21.650000, 23.300000) {snow};
\node[Square] at (-19.600000, 21.550000) {};
\node[Kanji] at (-19.600000, 22.050000) {公};
\node[Onyomi] at (-19.550000, 21.650000) {コウ};
\node[Meaning] at (-19.600000, 23.300000) {public};
\node[Square] at (-17.550000, 21.550000) {};
\node[Kanji] at (-17.550000, 22.050000) {退};
\node[Onyomi] at (-17.500000, 21.650000) {タイ};
\node[Kunyomi] at (-17.600000, 21.650000) {しりぞ.く};
\node[Meaning] at (-17.550000, 23.300000) {retreat};
\node[Square] at (-15.500000, 21.550000) {};
\node[Kanji] at (-15.500000, 22.050000) {霊};
\node[Onyomi] at (-15.450000, 21.650000) {レイ};
\node[Meaning] at (-15.500000, 23.300000) {ghost};
\node[Square] at (-13.450000, 21.550000) {};
\node[Kanji] at (-13.450000, 22.050000) {胸};
\node[Onyomi] at (-13.400000, 21.650000) {キョウ};
\node[Kunyomi] at (-13.500000, 21.650000) {むね};
\node[Meaning] at (-13.450000, 23.300000) {chest};
\node[Square] at (-11.400000, 21.550000) {};
\node[Kanji] at (-11.400000, 22.050000) {王};
\node[Onyomi] at (-11.350000, 21.650000) {オウ};
\node[Meaning] at (-11.400000, 23.300000) {king};
\node[Square] at (-9.350000, 21.550000) {};
\node[Kanji] at (-9.350000, 22.050000) {号};
\node[Onyomi] at (-9.300000, 21.650000) {ゴウ};
\node[Meaning] at (-9.350000, 23.300000) {number};
\node[Square] at (-7.300000, 21.550000) {};
\node[Kanji] at (-7.300000, 22.050000) {青};
\node[Onyomi] at (-7.250000, 21.650000) {ショウ};
\node[Kunyomi] at (-7.350000, 21.650000) {あお};
\node[Meaning] at (-7.300000, 23.300000) {blue};
\node[Square] at (-5.250000, 21.550000) {};
\node[Kanji] at (-5.250000, 22.050000) {衛};
\node[Onyomi] at (-5.200000, 21.650000) {エイ};
\node[Meaning] at (-5.250000, 23.300000) {defense};
\node[Square] at (-3.200000, 21.550000) {};
\node[Kanji] at (-3.200000, 22.050000) {微};
\node[Onyomi] at (-3.150000, 21.650000) {ビ};
\node[Kunyomi] at (-3.250000, 21.650000) {かす.か};
\node[Meaning] at (-3.200000, 23.300000) {delicate};
\node[Square] at (-1.150000, 21.550000) {};
\node[Kanji] at (-1.150000, 22.050000) {影};
\node[Onyomi] at (-1.100000, 21.650000) {エイ};
\node[Kunyomi] at (-1.200000, 21.650000) {かげ};
\node[Meaning] at (-1.150000, 23.300000) {shadow};
\node[Square] at (0.900000, 21.550000) {};
\node[Kanji] at (0.900000, 22.050000) {波};
\node[Onyomi] at (0.950000, 21.650000) {ハ};
\node[Kunyomi] at (0.850000, 21.650000) {なみ};
\node[Meaning] at (0.900000, 23.300000) {wave};
\node[Square] at (2.950000, 21.550000) {};
\node[Kanji] at (2.950000, 22.050000) {爆};
\node[Onyomi] at (3.000000, 21.650000) {バク};
\node[Kunyomi] at (2.900000, 21.650000) {は.ぜる};
\node[Meaning] at (2.950000, 23.300000) {explode};
\node[Square] at (5.000000, 21.550000) {};
\node[Kanji] at (5.000000, 22.050000) {念};
\node[Onyomi] at (5.050000, 21.650000) {ネン};
\node[Meaning] at (5.000000, 23.300000) {thought};
\node[Square] at (7.050000, 21.550000) {};
\node[Kanji] at (7.050000, 22.050000) {球};
\node[Onyomi] at (7.100000, 21.650000) {キュウ};
\node[Kunyomi] at (7.000000, 21.650000) {たま};
\node[Meaning] at (7.050000, 23.300000) {sphere};
\node[Square] at (9.100000, 21.550000) {};
\node[Kanji] at (9.100000, 22.050000) {備};
\node[Onyomi] at (9.150000, 21.650000) {ビ};
\node[Kunyomi] at (9.050000, 21.650000) {そな.える};
\node[Meaning] at (9.100000, 23.300000) {provide};
\node[Square] at (11.150000, 21.550000) {};
\node[Kanji] at (11.150000, 22.050000) {船};
\node[Onyomi] at (11.200000, 21.650000) {セン};
\node[Kunyomi] at (11.100000, 21.650000) {ふね};
\node[Meaning] at (11.150000, 23.300000) {boat};
\node[Square] at (13.200000, 21.550000) {};
\node[Kanji] at (13.200000, 22.050000) {村};
\node[Onyomi] at (13.250000, 21.650000) {ソン};
\node[Kunyomi] at (13.150000, 21.650000) {むら};
\node[Meaning] at (13.200000, 23.300000) {village};
\node[Square] at (15.250000, 21.550000) {};
\node[Kanji] at (15.250000, 22.050000) {透};
\node[Onyomi] at (15.300000, 21.650000) {トウ};
\node[Kunyomi] at (15.200000, 21.650000) {す.ける};
\node[Meaning] at (15.250000, 23.300000) {transparent};
\node[Square] at (17.300000, 21.550000) {};
\node[Kanji] at (17.300000, 22.050000) {六};
\node[Onyomi] at (17.350000, 21.650000) {ロク};
\node[Kunyomi] at (17.250000, 21.650000) {む.つ};
\node[Meaning] at (17.300000, 23.300000) {six};
\node[Square] at (19.350000, 21.550000) {};
\node[Kanji] at (19.350000, 22.050000) {服};
\node[Onyomi] at (19.400000, 21.650000) {フク};
\node[Meaning] at (19.350000, 23.300000) {clothes};
\node[Square] at (21.400000, 21.550000) {};
\node[Kanji] at (21.400000, 22.050000) {襲};
\node[Onyomi] at (21.450000, 21.650000) {シュウ};
\node[Kunyomi] at (21.350000, 21.650000) {おそ.う};
\node[Meaning] at (21.400000, 23.300000) {attack};
\node[Square] at (23.450000, 21.550000) {};
\node[Kanji] at (23.450000, 22.050000) {式};
\node[Onyomi] at (23.500000, 21.650000) {シキ};
\node[Meaning] at (23.450000, 23.300000) {ritual};
\node[Square] at (25.500000, 21.550000) {};
\node[Kanji] at (25.500000, 22.050000) {介};
\node[Onyomi] at (25.550000, 21.650000) {カイ};
\node[Meaning] at (25.500000, 23.300000) {jammed in};
\node[Square] at (27.550000, 21.550000) {};
\node[Kanji] at (27.550000, 22.050000) {務};
\node[Onyomi] at (27.600000, 21.650000) {ム};
\node[Kunyomi] at (27.500000, 21.650000) {つと.める};
\node[Meaning] at (27.550000, 23.300000) {task};
\node[Square] at (29.600000, 21.550000) {};
\node[Kanji] at (29.600000, 22.050000) {遅};
\node[Onyomi] at (29.650000, 21.650000) {チ};
\node[Kunyomi] at (29.550000, 21.650000) {おそ.い};
\node[Meaning] at (29.600000, 23.300000) {slow};
\node[Square] at (31.650000, 21.550000) {};
\node[Kanji] at (31.650000, 22.050000) {馬};
\node[Onyomi] at (31.700000, 21.650000) {バ};
\node[Kunyomi] at (31.600000, 21.650000) {うま};
\node[Meaning] at (31.650000, 23.300000) {horse};
\node[Square] at (33.700000, 21.550000) {};
\node[Kanji] at (33.700000, 22.050000) {割};
\node[Onyomi] at (33.750000, 21.650000) {カツ};
\node[Kunyomi] at (33.650000, 21.650000) {わり};
\node[Meaning] at (33.700000, 23.300000) {divide};
\node[Square] at (35.750000, 21.550000) {};
\node[Kanji] at (35.750000, 22.050000) {腹};
\node[Onyomi] at (35.800000, 21.650000) {フク};
\node[Kunyomi] at (35.700000, 21.650000) {はら};
\node[Meaning] at (35.750000, 23.300000) {belly};
\node[Square] at (37.800000, 21.550000) {};
\node[Kanji] at (37.800000, 22.050000) {移};
\node[Onyomi] at (37.850000, 21.650000) {イ};
\node[Kunyomi] at (37.750000, 21.650000) {うつ.*};
\node[Meaning] at (37.800000, 23.300000) {shift};
\node[Square] at (39.850000, 21.550000) {};
\node[Kanji] at (39.850000, 22.050000) {器};
\node[Onyomi] at (39.900000, 21.650000) {キ};
\node[Kunyomi] at (39.800000, 21.650000) {うつわ};
\node[Meaning] at (39.850000, 23.300000) {container};
\node[Square] at (41.900000, 21.550000) {};
\node[Kanji] at (41.900000, 22.050000) {他};
\node[Onyomi] at (41.950000, 21.650000) {タ};
\node[Kunyomi] at (41.850000, 21.650000) {ほか};
\node[Meaning] at (41.900000, 23.300000) {other};
\node[Square] at (43.950000, 21.550000) {};
\node[Kanji] at (43.950000, 22.050000) {曲};
\node[Onyomi] at (44.000000, 21.650000) {キョク};
\node[Kunyomi] at (43.900000, 21.650000) {ま.げる};
\node[Meaning] at (43.950000, 23.300000) {music};
\node[Square] at (46.000000, 21.550000) {};
\node[Kanji] at (46.000000, 22.050000) {隣};
\node[Onyomi] at (46.050000, 21.650000) {リン};
\node[Kunyomi] at (45.950000, 21.650000) {となり};
\node[Meaning] at (46.000000, 23.300000) {neighbor};
\node[Square] at (48.050000, 21.550000) {};
\node[Kanji] at (48.050000, 22.050000) {喜};
\node[Onyomi] at (48.100000, 21.650000) {キ};
\node[Kunyomi] at (48.000000, 21.650000) {よろこ};
\node[Meaning] at (48.050000, 23.300000) {rejoice};
\node[Square] at (50.100000, 21.550000) {};
\node[Kanji] at (50.100000, 22.050000) {井};
\node[Onyomi] at (50.150000, 21.650000) {ショウ};
\node[Kunyomi] at (50.050000, 21.650000) {い};
\node[Meaning] at (50.100000, 23.300000) {well};
\node[Square] at (52.150000, 21.550000) {};
\node[Kanji] at (52.150000, 22.050000) {七};
\node[Onyomi] at (52.200000, 21.650000) {シチ};
\node[Kunyomi] at (52.100000, 21.650000) {なな.*};
\node[Meaning] at (52.150000, 23.300000) {seven};
\node[Square] at (54.200000, 21.550000) {};
\node[Kanji] at (54.200000, 22.050000) {許};
\node[Onyomi] at (54.250000, 21.650000) {キョ};
\node[Kunyomi] at (54.150000, 21.650000) {ゆる.す};
\node[Meaning] at (54.200000, 23.300000) {permit};
\node[Square] at (56.250000, 21.550000) {};
\node[Kanji] at (56.250000, 22.050000) {桜};
\node[Kunyomi] at (56.200000, 21.650000) {さくら};
\node[Meaning] at (56.250000, 23.300000) {sakura};
\node[Meaning] at (-58.500000, 22.100000) {82.40\%};
\node[Square] at (-56.500000, 19.500000) {};
\node[Kanji] at (-56.500000, 20.000000) {鏡};
\node[Onyomi] at (-56.450000, 19.600000) {キョウ};
\node[Kunyomi] at (-56.550000, 19.600000) {かがみ};
\node[Meaning] at (-56.500000, 21.250000) {mirror};
\node[Square] at (-54.450000, 19.500000) {};
\node[Kanji] at (-54.450000, 20.000000) {宿};
\node[Onyomi] at (-54.400000, 19.600000) {シュク};
\node[Kunyomi] at (-54.500000, 19.600000) {やど};
\node[Meaning] at (-54.450000, 21.250000) {lodge};
\node[Square] at (-52.400000, 19.500000) {};
\node[Kanji] at (-52.400000, 20.000000) {疑};
\node[Onyomi] at (-52.350000, 19.600000) {ギ};
\node[Kunyomi] at (-52.450000, 19.600000) {うたが.う};
\node[Meaning] at (-52.400000, 21.250000) {doubt};
\node[Square] at (-50.350000, 19.500000) {};
\node[Kanji] at (-50.350000, 20.000000) {若};
\node[Onyomi] at (-50.300000, 19.600000) {ジャク};
\node[Kunyomi] at (-50.400000, 19.600000) {わか};
\node[Meaning] at (-50.350000, 21.250000) {young};
\node[Square] at (-48.300000, 19.500000) {};
\node[Kanji] at (-48.300000, 20.000000) {態};
\node[Onyomi] at (-48.250000, 19.600000) {タイ};
\node[Kunyomi] at (-48.350000, 19.600000) {わざ};
\node[Meaning] at (-48.300000, 21.250000) {appearance};
\node[Square] at (-46.250000, 19.500000) {};
\node[Kanji] at (-46.250000, 20.000000) {職};
\node[Onyomi] at (-46.200000, 19.600000) {ショク};
\node[Meaning] at (-46.250000, 21.250000) {employment};
\node[Square] at (-44.200000, 19.500000) {};
\node[Kanji] at (-44.200000, 20.000000) {示};
\node[Onyomi] at (-44.150000, 19.600000) {ジ};
\node[Kunyomi] at (-44.250000, 19.600000) {しめ.す};
\node[Meaning] at (-44.200000, 21.250000) {indicate};
\node[Square] at (-42.150000, 19.500000) {};
\node[Kanji] at (-42.150000, 20.000000) {参};
\node[Onyomi] at (-42.100000, 19.600000) {サン};
\node[Kunyomi] at (-42.200000, 19.600000) {まい.る};
\node[Meaning] at (-42.150000, 21.250000) {participate};
\node[Square] at (-40.100000, 19.500000) {};
\node[Kanji] at (-40.100000, 20.000000) {机};
\node[Kunyomi] at (-40.150000, 19.600000) {つくえ};
\node[Meaning] at (-40.100000, 21.250000) {desk};
\node[Square] at (-38.050000, 19.500000) {};
\node[Kanji] at (-38.050000, 20.000000) {蛇};
\node[Onyomi] at (-38.000000, 19.600000) {ジャ};
\node[Kunyomi] at (-38.100000, 19.600000) {へび};
\node[Meaning] at (-38.050000, 21.250000) {snake};
\node[Square] at (-36.000000, 19.500000) {};
\node[Kanji] at (-36.000000, 20.000000) {炉};
\node[Onyomi] at (-35.950000, 19.600000) {ロ};
\node[Kunyomi] at (-36.050000, 19.600000) {いろり};
\node[Meaning] at (-36.000000, 21.250000) {furnace};
\node[Square] at (-33.950000, 19.500000) {};
\node[Kanji] at (-33.950000, 20.000000) {密};
\node[Onyomi] at (-33.900000, 19.600000) {ミツ};
\node[Kunyomi] at (-34.000000, 19.600000) {ひそ.か};
\node[Meaning] at (-33.950000, 21.250000) {secrecy};
\node[Square] at (-31.900000, 19.500000) {};
\node[Kanji] at (-31.900000, 20.000000) {陽};
\node[Onyomi] at (-31.850000, 19.600000) {ヨウ};
\node[Kunyomi] at (-31.950000, 19.600000) {ひ};
\node[Meaning] at (-31.900000, 21.250000) {sunshine};
\node[Square] at (-29.850000, 19.500000) {};
\node[Kanji] at (-29.850000, 20.000000) {裏};
\node[Onyomi] at (-29.800000, 19.600000) {リ};
\node[Kunyomi] at (-29.900000, 19.600000) {うら};
\node[Meaning] at (-29.850000, 21.250000) {backside};
\node[Square] at (-27.800000, 19.500000) {};
\node[Kanji] at (-27.800000, 20.000000) {帽};
\node[Onyomi] at (-27.750000, 19.600000) {ボウ};
\node[Meaning] at (-27.800000, 21.250000) {hat};
\node[Square] at (-25.750000, 19.500000) {};
\node[Kanji] at (-25.750000, 20.000000) {常};
\node[Onyomi] at (-25.700000, 19.600000) {ジョウ};
\node[Kunyomi] at (-25.800000, 19.600000) {つね};
\node[Meaning] at (-25.750000, 21.250000) {normal};
\node[Square] at (-23.700000, 19.500000) {};
\node[Kanji] at (-23.700000, 20.000000) {庁};
\node[Onyomi] at (-23.650000, 19.600000) {チョウ};
\node[Meaning] at (-23.700000, 21.250000) {agency};
\node[Square] at (-21.650000, 19.500000) {};
\node[Kanji] at (-21.650000, 20.000000) {角};
\node[Onyomi] at (-21.600000, 19.600000) {カク};
\node[Kunyomi] at (-21.700000, 19.600000) {かど};
\node[Meaning] at (-21.650000, 21.250000) {angle};
\node[Square] at (-19.600000, 19.500000) {};
\node[Kanji] at (-19.600000, 20.000000) {治};
\node[Onyomi] at (-19.550000, 19.600000) {ジ};
\node[Kunyomi] at (-19.650000, 19.600000) {なお.す};
\node[Meaning] at (-19.600000, 21.250000) {cure};
\node[Square] at (-17.550000, 19.500000) {};
\node[Kanji] at (-17.550000, 20.000000) {倍};
\node[Onyomi] at (-17.500000, 19.600000) {バイ};
\node[Meaning] at (-17.550000, 21.250000) {double};
\node[Square] at (-15.500000, 19.500000) {};
\node[Kanji] at (-15.500000, 20.000000) {途};
\node[Onyomi] at (-15.450000, 19.600000) {ト};
\node[Meaning] at (-15.500000, 21.250000) {route};
\node[Square] at (-13.450000, 19.500000) {};
\node[Kanji] at (-13.450000, 20.000000) {鈴};
\node[Onyomi] at (-13.400000, 19.600000) {リン};
\node[Kunyomi] at (-13.500000, 19.600000) {すず};
\node[Meaning] at (-13.450000, 21.250000) {buzzer};
\node[Square] at (-11.400000, 19.500000) {};
\node[Kanji] at (-11.400000, 20.000000) {猫};
\node[Kunyomi] at (-11.450000, 19.600000) {ねこ};
\node[Meaning] at (-11.400000, 21.250000) {cat};
\node[Square] at (-9.350000, 19.500000) {};
\node[Kanji] at (-9.350000, 20.000000) {右};
\node[Onyomi] at (-9.300000, 19.600000) {ウ};
\node[Kunyomi] at (-9.400000, 19.600000) {みぎ};
\node[Meaning] at (-9.350000, 21.250000) {right};
\node[Square] at (-7.300000, 19.500000) {};
\node[Kanji] at (-7.300000, 20.000000) {和};
\node[Onyomi] at (-7.250000, 19.600000) {ワ};
\node[Kunyomi] at (-7.350000, 19.600000) {なご};
\node[Meaning] at (-7.300000, 21.250000) {peace};
\node[Square] at (-5.250000, 19.500000) {};
\node[Kanji] at (-5.250000, 20.000000) {幸};
\node[Onyomi] at (-5.200000, 19.600000) {コウ};
\node[Kunyomi] at (-5.300000, 19.600000) {しあわ.せ};
\node[Meaning] at (-5.250000, 21.250000) {happiness};
\node[Square] at (-3.200000, 19.500000) {};
\node[Kanji] at (-3.200000, 20.000000) {揺};
\node[Onyomi] at (-3.150000, 19.600000) {ヨウ};
\node[Kunyomi] at (-3.250000, 19.600000) {ゆ.*};
\node[Meaning] at (-3.200000, 21.250000) {shake};
\node[Square] at (-1.150000, 19.500000) {};
\node[Kanji] at (-1.150000, 20.000000) {臣};
\node[Onyomi] at (-1.100000, 19.600000) {シン};
\node[Meaning] at (-1.150000, 21.250000) {servant};
\node[Square] at (0.900000, 19.500000) {};
\node[Kanji] at (0.900000, 20.000000) {奥};
\node[Onyomi] at (0.950000, 19.600000) {オウ};
\node[Kunyomi] at (0.850000, 19.600000) {おく};
\node[Meaning] at (0.900000, 21.250000) {interior};
\node[Square] at (2.950000, 19.500000) {};
\node[Kanji] at (2.950000, 20.000000) {門};
\node[Onyomi] at (3.000000, 19.600000) {モン};
\node[Meaning] at (2.950000, 21.250000) {gates};
\node[Square] at (5.000000, 19.500000) {};
\node[Kanji] at (5.000000, 20.000000) {茶};
\node[Onyomi] at (5.050000, 19.600000) {チャ};
\node[Meaning] at (5.000000, 21.250000) {tea};
\node[Square] at (7.050000, 19.500000) {};
\node[Kanji] at (7.050000, 20.000000) {焼};
\node[Onyomi] at (7.100000, 19.600000) {ショウ};
\node[Kunyomi] at (7.000000, 19.600000) {や};
\node[Meaning] at (7.050000, 21.250000) {bake};
\node[Square] at (9.100000, 19.500000) {};
\node[Kanji] at (9.100000, 20.000000) {絵};
\node[Onyomi] at (9.150000, 19.600000) {エ};
\node[Meaning] at (9.100000, 21.250000) {drawing};
\node[Square] at (11.150000, 19.500000) {};
\node[Kanji] at (11.150000, 20.000000) {例};
\node[Onyomi] at (11.200000, 19.600000) {レイ};
\node[Kunyomi] at (11.100000, 19.600000) {たと};
\node[Meaning] at (11.150000, 21.250000) {example};
\node[Square] at (13.200000, 19.500000) {};
\node[Kanji] at (13.200000, 20.000000) {監};
\node[Onyomi] at (13.250000, 19.600000) {カン};
\node[Meaning] at (13.200000, 21.250000) {oversee};
\node[Square] at (15.250000, 19.500000) {};
\node[Kanji] at (15.250000, 20.000000) {薄};
\node[Onyomi] at (15.300000, 19.600000) {ハク};
\node[Kunyomi] at (15.200000, 19.600000) {うす.*};
\node[Meaning] at (15.250000, 21.250000) {dilute};
\node[Square] at (17.300000, 19.500000) {};
\node[Kanji] at (17.300000, 20.000000) {囲};
\node[Onyomi] at (17.350000, 19.600000) {イ};
\node[Kunyomi] at (17.250000, 19.600000) {かこ.む};
\node[Meaning] at (17.300000, 21.250000) {surround};
\node[Square] at (19.350000, 19.500000) {};
\node[Kanji] at (19.350000, 20.000000) {応};
\node[Onyomi] at (19.400000, 19.600000) {オウ};
\node[Meaning] at (19.350000, 21.250000) {respond};
\node[Square] at (21.400000, 19.500000) {};
\node[Kanji] at (21.400000, 20.000000) {吹};
\node[Onyomi] at (21.450000, 19.600000) {スイ};
\node[Kunyomi] at (21.350000, 19.600000) {ふ};
\node[Meaning] at (21.400000, 21.250000) {blow};
\node[Square] at (23.450000, 19.500000) {};
\node[Kanji] at (23.450000, 20.000000) {単};
\node[Onyomi] at (23.500000, 19.600000) {タン};
\node[Meaning] at (23.450000, 21.250000) {simple};
\node[Square] at (25.500000, 19.500000) {};
\node[Kanji] at (25.500000, 20.000000) {玄};
\node[Onyomi] at (25.550000, 19.600000) {ゲン};
\node[Kunyomi] at (25.450000, 19.600000) {くろ};
\node[Meaning] at (25.500000, 21.250000) {mysterious};
\node[Square] at (27.550000, 19.500000) {};
\node[Kanji] at (27.550000, 20.000000) {係};
\node[Onyomi] at (27.600000, 19.600000) {ケイ};
\node[Kunyomi] at (27.500000, 19.600000) {かか};
\node[Meaning] at (27.550000, 21.250000) {connection};
\node[Square] at (29.600000, 19.500000) {};
\node[Kanji] at (29.600000, 20.000000) {射};
\node[Onyomi] at (29.650000, 19.600000) {シャ};
\node[Kunyomi] at (29.550000, 19.600000) {い.る};
\node[Meaning] at (29.600000, 21.250000) {shoot};
\node[Square] at (31.650000, 19.500000) {};
\node[Kanji] at (31.650000, 20.000000) {厳};
\node[Onyomi] at (31.700000, 19.600000) {ゲン};
\node[Kunyomi] at (31.600000, 19.600000) {きび.しい};
\node[Meaning] at (31.650000, 21.250000) {strict};
\node[Square] at (33.700000, 19.500000) {};
\node[Kanji] at (33.700000, 20.000000) {額};
\node[Onyomi] at (33.750000, 19.600000) {ガク};
\node[Kunyomi] at (33.650000, 19.600000) {ひたい};
\node[Meaning] at (33.700000, 21.250000) {amount};
\node[Square] at (35.750000, 19.500000) {};
\node[Kanji] at (35.750000, 20.000000) {軽};
\node[Onyomi] at (35.800000, 19.600000) {ケイ};
\node[Kunyomi] at (35.700000, 19.600000) {かる};
\node[Meaning] at (35.750000, 21.250000) {lightweight};
\node[Square] at (37.800000, 19.500000) {};
\node[Kanji] at (37.800000, 20.000000) {減};
\node[Onyomi] at (37.850000, 19.600000) {ゲン};
\node[Kunyomi] at (37.750000, 19.600000) {へ.る};
\node[Meaning] at (37.800000, 21.250000) {decrease};
\node[Square] at (39.850000, 19.500000) {};
\node[Kanji] at (39.850000, 20.000000) {汚};
\node[Onyomi] at (39.900000, 19.600000) {オ};
\node[Kunyomi] at (39.800000, 19.600000) {よご};
\node[Meaning] at (39.850000, 21.250000) {dirty};
\node[Square] at (41.900000, 19.500000) {};
\node[Kanji] at (41.900000, 20.000000) {届};
\node[Kunyomi] at (41.850000, 19.600000) {とど};
\node[Meaning] at (41.900000, 21.250000) {deliver};
\node[Square] at (43.950000, 19.500000) {};
\node[Kanji] at (43.950000, 20.000000) {染};
\node[Onyomi] at (44.000000, 19.600000) {セン};
\node[Kunyomi] at (43.900000, 19.600000) {しみ};
\node[Meaning] at (43.950000, 21.250000) {dye};
\node[Square] at (46.000000, 19.500000) {};
\node[Kanji] at (46.000000, 20.000000) {準};
\node[Onyomi] at (46.050000, 19.600000) {ジュン};
\node[Meaning] at (46.000000, 21.250000) {standard};
\node[Square] at (48.050000, 19.500000) {};
\node[Kanji] at (48.050000, 20.000000) {怪};
\node[Onyomi] at (48.100000, 19.600000) {カイ};
\node[Kunyomi] at (48.000000, 19.600000) {あや.しい};
\node[Meaning] at (48.050000, 21.250000) {suspicious};
\node[Square] at (50.100000, 19.500000) {};
\node[Kanji] at (50.100000, 20.000000) {昔};
\node[Kunyomi] at (50.050000, 19.600000) {むかし};
\node[Meaning] at (50.100000, 21.250000) {long ago};
\node[Square] at (52.150000, 19.500000) {};
\node[Kanji] at (52.150000, 20.000000) {包};
\node[Onyomi] at (52.200000, 19.600000) {ホウ};
\node[Kunyomi] at (52.100000, 19.600000) {つつ.み};
\node[Meaning] at (52.150000, 21.250000) {wrap};
\node[Square] at (54.200000, 19.500000) {};
\node[Kanji] at (54.200000, 20.000000) {留};
\node[Onyomi] at (54.250000, 19.600000) {ル};
\node[Kunyomi] at (54.150000, 19.600000) {と};
\node[Meaning] at (54.200000, 21.250000) {detain};
\node[Square] at (56.250000, 19.500000) {};
\node[Kanji] at (56.250000, 20.000000) {種};
\node[Onyomi] at (56.300000, 19.600000) {シュ};
\node[Kunyomi] at (56.200000, 19.600000) {たね};
\node[Meaning] at (56.250000, 21.250000) {kind};
\node[Meaning] at (-58.500000, 20.050000) {84.45\%};
\node[Square] at (-56.500000, 17.450000) {};
\node[Kanji] at (-56.500000, 17.950000) {輝};
\node[Onyomi] at (-56.450000, 17.550000) {キ};
\node[Kunyomi] at (-56.550000, 17.550000) {かがやき};
\node[Meaning] at (-56.500000, 19.200000) {radiance};
\node[Square] at (-54.450000, 17.450000) {};
\node[Kanji] at (-54.450000, 17.950000) {競};
\node[Onyomi] at (-54.400000, 17.550000) {キョウ};
\node[Kunyomi] at (-54.500000, 17.550000) {きそ.*};
\node[Meaning] at (-54.450000, 19.200000) {compete};
\node[Square] at (-52.400000, 17.450000) {};
\node[Kanji] at (-52.400000, 17.950000) {争};
\node[Onyomi] at (-52.350000, 17.550000) {ソウ};
\node[Kunyomi] at (-52.450000, 17.550000) {あらそ.う};
\node[Meaning] at (-52.400000, 19.200000) {conflict};
\node[Square] at (-50.350000, 17.450000) {};
\node[Kanji] at (-50.350000, 17.950000) {憶};
\node[Onyomi] at (-50.300000, 17.550000) {オク};
\node[Meaning] at (-50.350000, 19.200000) {recollection};
\node[Square] at (-48.300000, 17.450000) {};
\node[Kanji] at (-48.300000, 17.950000) {兄};
\node[Onyomi] at (-48.250000, 17.550000) {キョウ};
\node[Kunyomi] at (-48.350000, 17.550000) {あに};
\node[Meaning] at (-48.300000, 19.200000) {older brother};
\node[Square] at (-46.250000, 17.450000) {};
\node[Kanji] at (-46.250000, 17.950000) {雨};
\node[Onyomi] at (-46.200000, 17.550000) {ウ};
\node[Kunyomi] at (-46.300000, 17.550000) {あめ};
\node[Meaning] at (-46.250000, 19.200000) {rain};
\node[Square] at (-44.200000, 17.450000) {};
\node[Kanji] at (-44.200000, 17.950000) {歯};
\node[Kunyomi] at (-44.250000, 17.550000) {は};
\node[Meaning] at (-44.200000, 19.200000) {tooth};
\node[Square] at (-42.150000, 17.450000) {};
\node[Kanji] at (-42.150000, 17.950000) {個};
\node[Onyomi] at (-42.100000, 17.550000) {コ};
\node[Meaning] at (-42.150000, 19.200000) {individual};
\node[Square] at (-40.100000, 17.450000) {};
\node[Kanji] at (-40.100000, 17.950000) {素};
\node[Onyomi] at (-40.050000, 17.550000) {ス};
\node[Meaning] at (-40.100000, 19.200000) {element};
\node[Square] at (-38.050000, 17.450000) {};
\node[Kanji] at (-38.050000, 17.950000) {炎};
\node[Onyomi] at (-38.000000, 17.550000) {エン};
\node[Kunyomi] at (-38.100000, 17.550000) {ほのお};
\node[Meaning] at (-38.050000, 19.200000) {flame};
\node[Square] at (-36.000000, 17.450000) {};
\node[Kanji] at (-36.000000, 17.950000) {避};
\node[Onyomi] at (-35.950000, 17.550000) {ヒ};
\node[Kunyomi] at (-36.050000, 17.550000) {さ.ける};
\node[Meaning] at (-36.000000, 19.200000) {dodge};
\node[Square] at (-33.950000, 17.450000) {};
\node[Kanji] at (-33.950000, 17.950000) {混};
\node[Onyomi] at (-33.900000, 17.550000) {コン};
\node[Kunyomi] at (-34.000000, 17.550000) {ま.*};
\node[Meaning] at (-33.950000, 19.200000) {mix};
\node[Square] at (-31.900000, 17.450000) {};
\node[Kanji] at (-31.900000, 17.950000) {左};
\node[Onyomi] at (-31.850000, 17.550000) {サ};
\node[Kunyomi] at (-31.950000, 17.550000) {ひだり};
\node[Meaning] at (-31.900000, 19.200000) {left};
\node[Square] at (-29.850000, 17.450000) {};
\node[Kanji] at (-29.850000, 17.950000) {印};
\node[Onyomi] at (-29.800000, 17.550000) {イン};
\node[Kunyomi] at (-29.900000, 17.550000) {しるし};
\node[Meaning] at (-29.850000, 19.200000) {seal};
\node[Square] at (-27.800000, 17.450000) {};
\node[Kanji] at (-27.800000, 17.950000) {犬};
\node[Onyomi] at (-27.750000, 17.550000) {ケン};
\node[Kunyomi] at (-27.850000, 17.550000) {いぬ};
\node[Meaning] at (-27.800000, 19.200000) {dog};
\node[Square] at (-25.750000, 17.450000) {};
\node[Kanji] at (-25.750000, 17.950000) {負};
\node[Onyomi] at (-25.700000, 17.550000) {フ};
\node[Kunyomi] at (-25.800000, 17.550000) {ま.ける};
\node[Meaning] at (-25.750000, 19.200000) {lose};
\node[Square] at (-23.700000, 17.450000) {};
\node[Kanji] at (-23.700000, 17.950000) {愛};
\node[Onyomi] at (-23.650000, 17.550000) {アイ};
\node[Kunyomi] at (-23.750000, 17.550000) {まな};
\node[Meaning] at (-23.700000, 19.200000) {love};
\node[Square] at (-21.650000, 17.450000) {};
\node[Kanji] at (-21.650000, 17.950000) {民};
\node[Onyomi] at (-21.600000, 17.550000) {ミン};
\node[Kunyomi] at (-21.700000, 17.550000) {たみ};
\node[Meaning] at (-21.650000, 19.200000) {peoples};
\node[Square] at (-19.600000, 17.450000) {};
\node[Kanji] at (-19.600000, 17.950000) {判};
\node[Onyomi] at (-19.550000, 17.550000) {ハン};
\node[Meaning] at (-19.600000, 19.200000) {judge};
\node[Square] at (-17.550000, 17.450000) {};
\node[Kanji] at (-17.550000, 17.950000) {断};
\node[Onyomi] at (-17.500000, 17.550000) {ダン};
\node[Kunyomi] at (-17.600000, 17.550000) {ことわ.る};
\node[Meaning] at (-17.550000, 19.200000) {cut off};
\node[Square] at (-15.500000, 17.450000) {};
\node[Kanji] at (-15.500000, 17.950000) {袋};
\node[Onyomi] at (-15.450000, 17.550000) {タイ};
\node[Kunyomi] at (-15.550000, 17.550000) {ふくろ};
\node[Meaning] at (-15.500000, 19.200000) {sack};
\node[Square] at (-13.450000, 17.450000) {};
\node[Kanji] at (-13.450000, 17.950000) {列};
\node[Onyomi] at (-13.400000, 17.550000) {レツ};
\node[Meaning] at (-13.450000, 19.200000) {row};
\node[Square] at (-11.400000, 17.450000) {};
\node[Kanji] at (-11.400000, 17.950000) {湖};
\node[Onyomi] at (-11.350000, 17.550000) {コ};
\node[Kunyomi] at (-11.450000, 17.550000) {みずうみ};
\node[Meaning] at (-11.400000, 19.200000) {lake};
\node[Square] at (-9.350000, 17.450000) {};
\node[Kanji] at (-9.350000, 17.950000) {八};
\node[Onyomi] at (-9.300000, 17.550000) {ハチ};
\node[Kunyomi] at (-9.400000, 17.550000) {や.*};
\node[Meaning] at (-9.350000, 19.200000) {eight};
\node[Square] at (-7.300000, 17.450000) {};
\node[Kanji] at (-7.300000, 17.950000) {玉};
\node[Onyomi] at (-7.250000, 17.550000) {ギョク};
\node[Kunyomi] at (-7.350000, 17.550000) {たま};
\node[Meaning] at (-7.300000, 19.200000) {ball};
\node[Square] at (-5.250000, 17.450000) {};
\node[Kanji] at (-5.250000, 17.950000) {英};
\node[Onyomi] at (-5.200000, 17.550000) {エイ};
\node[Meaning] at (-5.250000, 19.200000) {england};
\node[Square] at (-3.200000, 17.450000) {};
\node[Kanji] at (-3.200000, 17.950000) {涙};
\node[Onyomi] at (-3.150000, 17.550000) {ルイ};
\node[Kunyomi] at (-3.250000, 17.550000) {なみだ};
\node[Meaning] at (-3.200000, 19.200000) {teardrop};
\node[Square] at (-1.150000, 17.450000) {};
\node[Kanji] at (-1.150000, 17.950000) {灯};
\node[Onyomi] at (-1.100000, 17.550000) {トウ};
\node[Kunyomi] at (-1.200000, 17.550000) {あかり};
\node[Meaning] at (-1.150000, 19.200000) {lamp};
\node[Square] at (0.900000, 17.450000) {};
\node[Kanji] at (0.900000, 17.950000) {格};
\node[Onyomi] at (0.950000, 17.550000) {カク};
\node[Meaning] at (0.900000, 19.200000) {status};
\node[Square] at (2.950000, 17.450000) {};
\node[Kanji] at (2.950000, 17.950000) {頑};
\node[Onyomi] at (3.000000, 17.550000) {ガン};
\node[Meaning] at (2.950000, 19.200000) {stubborn};
\node[Square] at (5.000000, 17.450000) {};
\node[Kanji] at (5.000000, 17.950000) {散};
\node[Onyomi] at (5.050000, 17.550000) {サン};
\node[Kunyomi] at (4.950000, 17.550000) {ち.*};
\node[Meaning] at (5.000000, 19.200000) {scatter};
\node[Square] at (7.050000, 17.450000) {};
\node[Kanji] at (7.050000, 17.950000) {局};
\node[Onyomi] at (7.100000, 17.550000) {キョク};
\node[Meaning] at (7.050000, 19.200000) {bureau};
\node[Square] at (9.100000, 17.450000) {};
\node[Kanji] at (9.100000, 17.950000) {腰};
\node[Onyomi] at (9.150000, 17.550000) {ヨウ};
\node[Kunyomi] at (9.050000, 17.550000) {こし};
\node[Meaning] at (9.100000, 19.200000) {waist};
\node[Square] at (11.150000, 17.450000) {};
\node[Kanji] at (11.150000, 17.950000) {盗};
\node[Onyomi] at (11.200000, 17.550000) {トウ};
\node[Kunyomi] at (11.100000, 17.550000) {ぬす.む};
\node[Meaning] at (11.150000, 19.200000) {steal};
\node[Square] at (13.200000, 17.450000) {};
\node[Kanji] at (13.200000, 17.950000) {津};
\node[Onyomi] at (13.250000, 17.550000) {シン};
\node[Kunyomi] at (13.150000, 17.550000) {つ};
\node[Meaning] at (13.200000, 19.200000) {haven};
\node[Square] at (15.250000, 17.450000) {};
\node[Kanji] at (15.250000, 17.950000) {敷};
\node[Onyomi] at (15.300000, 17.550000) {フ};
\node[Kunyomi] at (15.200000, 17.550000) {しき};
\node[Meaning] at (15.250000, 19.200000) {spread};
\node[Square] at (17.300000, 17.450000) {};
\node[Kanji] at (17.300000, 17.950000) {毒};
\node[Onyomi] at (17.350000, 17.550000) {ドク};
\node[Meaning] at (17.300000, 19.200000) {poison};
\node[Square] at (19.350000, 17.450000) {};
\node[Kanji] at (19.350000, 17.950000) {遊};
\node[Onyomi] at (19.400000, 17.550000) {ユウ};
\node[Kunyomi] at (19.300000, 17.550000) {あそ};
\node[Meaning] at (19.350000, 19.200000) {play};
\node[Square] at (21.400000, 17.450000) {};
\node[Kanji] at (21.400000, 17.950000) {申};
\node[Onyomi] at (21.450000, 17.550000) {シン};
\node[Kunyomi] at (21.350000, 17.550000) {もう};
\node[Meaning] at (21.400000, 19.200000) {say humbly};
\node[Square] at (23.450000, 17.450000) {};
\node[Kanji] at (23.450000, 17.950000) {在};
\node[Onyomi] at (23.500000, 17.550000) {ザイ};
\node[Meaning] at (23.450000, 19.200000) {exist};
\node[Square] at (25.500000, 17.450000) {};
\node[Kanji] at (25.500000, 17.950000) {沈};
\node[Onyomi] at (25.550000, 17.550000) {チン};
\node[Kunyomi] at (25.450000, 17.550000) {しず.*};
\node[Meaning] at (25.500000, 19.200000) {sink};
\node[Square] at (27.550000, 17.450000) {};
\node[Kanji] at (27.550000, 17.950000) {覆};
\node[Onyomi] at (27.600000, 17.550000) {フク};
\node[Kunyomi] at (27.500000, 17.550000) {おお.う};
\node[Meaning] at (27.550000, 19.200000) {capsize};
\node[Square] at (29.600000, 17.450000) {};
\node[Kanji] at (29.600000, 17.950000) {昨};
\node[Onyomi] at (29.650000, 17.550000) {サク};
\node[Meaning] at (29.600000, 19.200000) {previous};
\node[Square] at (31.650000, 17.450000) {};
\node[Kanji] at (31.650000, 17.950000) {容};
\node[Onyomi] at (31.700000, 17.550000) {ヨウ};
\node[Meaning] at (31.650000, 19.200000) {form};
\node[Square] at (33.700000, 17.450000) {};
\node[Kanji] at (33.700000, 17.950000) {剣};
\node[Onyomi] at (33.750000, 17.550000) {ケン};
\node[Kunyomi] at (33.650000, 17.550000) {つるぎ};
\node[Meaning] at (33.700000, 19.200000) {sword};
\node[Square] at (35.750000, 17.450000) {};
\node[Kanji] at (35.750000, 17.950000) {頼};
\node[Onyomi] at (35.800000, 17.550000) {ライ};
\node[Kunyomi] at (35.700000, 17.550000) {たの};
\node[Meaning] at (35.750000, 19.200000) {trust};
\node[Square] at (37.800000, 17.450000) {};
\node[Kanji] at (37.800000, 17.950000) {支};
\node[Onyomi] at (37.850000, 17.550000) {シ};
\node[Kunyomi] at (37.750000, 17.550000) {ささ.える};
\node[Meaning] at (37.800000, 19.200000) {support};
\node[Square] at (39.850000, 17.450000) {};
\node[Kanji] at (39.850000, 17.950000) {岩};
\node[Onyomi] at (39.900000, 17.550000) {ガン};
\node[Kunyomi] at (39.800000, 17.550000) {いわ};
\node[Meaning] at (39.850000, 19.200000) {boulder};
\node[Square] at (41.900000, 17.450000) {};
\node[Kanji] at (41.900000, 17.950000) {製};
\node[Onyomi] at (41.950000, 17.550000) {セイ};
\node[Meaning] at (41.900000, 19.200000) {manufacture};
\node[Square] at (43.950000, 17.450000) {};
\node[Kanji] at (43.950000, 17.950000) {居};
\node[Onyomi] at (44.000000, 17.550000) {キョ};
\node[Kunyomi] at (43.900000, 17.550000) {い};
\node[Meaning] at (43.950000, 19.200000) {alive};
\node[Square] at (46.000000, 17.450000) {};
\node[Kanji] at (46.000000, 17.950000) {西};
\node[Onyomi] at (46.050000, 17.550000) {セイ};
\node[Kunyomi] at (45.950000, 17.550000) {にし};
\node[Meaning] at (46.000000, 19.200000) {west};
\node[Square] at (48.050000, 17.450000) {};
\node[Kanji] at (48.050000, 17.950000) {枚};
\node[Onyomi] at (48.100000, 17.550000) {マイ};
\node[Meaning] at (48.050000, 19.200000) {counter: sheets};
\node[Square] at (50.100000, 17.450000) {};
\node[Kanji] at (50.100000, 17.950000) {制};
\node[Onyomi] at (50.150000, 17.550000) {セイ};
\node[Meaning] at (50.100000, 19.200000) {control};
\node[Square] at (52.150000, 17.450000) {};
\node[Kanji] at (52.150000, 17.950000) {九};
\node[Onyomi] at (52.200000, 17.550000) {ク};
\node[Kunyomi] at (52.100000, 17.550000) {ここの.*};
\node[Meaning] at (52.150000, 19.200000) {nine};
\node[Square] at (54.200000, 17.450000) {};
\node[Kanji] at (54.200000, 17.950000) {構};
\node[Onyomi] at (54.250000, 17.550000) {コウ};
\node[Kunyomi] at (54.150000, 17.550000) {かま.*};
\node[Meaning] at (54.200000, 19.200000) {set up};
\node[Square] at (56.250000, 17.450000) {};
\node[Kanji] at (56.250000, 17.950000) {末};
\node[Onyomi] at (56.300000, 17.550000) {マツ};
\node[Kunyomi] at (56.200000, 17.550000) {すえ};
\node[Meaning] at (56.250000, 19.200000) {end};
\node[Meaning] at (-58.500000, 18.000000) {86.28\%};
\node[Square] at (-56.500000, 15.400000) {};
\node[Kanji] at (-56.500000, 15.900000) {完};
\node[Onyomi] at (-56.450000, 15.500000) {カン};
\node[Meaning] at (-56.500000, 17.150000) {perfect};
\node[Square] at (-54.450000, 15.400000) {};
\node[Kanji] at (-54.450000, 15.900000) {区};
\node[Onyomi] at (-54.400000, 15.500000) {ク};
\node[Meaning] at (-54.450000, 17.150000) {district};
\node[Square] at (-52.400000, 15.400000) {};
\node[Kanji] at (-52.400000, 15.900000) {沖};
\node[Onyomi] at (-52.350000, 15.500000) {チュウ};
\node[Kunyomi] at (-52.450000, 15.500000) {おき};
\node[Meaning] at (-52.400000, 17.150000) {open sea};
\node[Square] at (-50.350000, 15.400000) {};
\node[Kanji] at (-50.350000, 15.900000) {肉};
\node[Onyomi] at (-50.300000, 15.500000) {ニク};
\node[Meaning] at (-50.350000, 17.150000) {meat};
\node[Square] at (-48.300000, 15.400000) {};
\node[Kanji] at (-48.300000, 15.900000) {供};
\node[Onyomi] at (-48.250000, 15.500000) {キョウ};
\node[Kunyomi] at (-48.350000, 15.500000) {とも};
\node[Meaning] at (-48.300000, 17.150000) {servant};
\node[Square] at (-46.250000, 15.400000) {};
\node[Kanji] at (-46.250000, 15.900000) {普};
\node[Onyomi] at (-46.200000, 15.500000) {フ};
\node[Meaning] at (-46.250000, 17.150000) {normal};
\node[Square] at (-44.200000, 15.400000) {};
\node[Kanji] at (-44.200000, 15.900000) {象};
\node[Onyomi] at (-44.150000, 15.500000) {ショウ};
\node[Meaning] at (-44.200000, 17.150000) {elephant};
\node[Square] at (-42.150000, 15.400000) {};
\node[Kanji] at (-42.150000, 15.900000) {羊};
\node[Onyomi] at (-42.100000, 15.500000) {ヨウ};
\node[Kunyomi] at (-42.200000, 15.500000) {ひつじ};
\node[Meaning] at (-42.150000, 17.150000) {sheep};
\node[Square] at (-40.100000, 15.400000) {};
\node[Kanji] at (-40.100000, 15.900000) {証};
\node[Onyomi] at (-40.050000, 15.500000) {ショウ};
\node[Kunyomi] at (-40.150000, 15.500000) {あかし};
\node[Meaning] at (-40.100000, 17.150000) {evidence};
\node[Square] at (-38.050000, 15.400000) {};
\node[Kanji] at (-38.050000, 15.900000) {禁};
\node[Onyomi] at (-38.000000, 15.500000) {キン};
\node[Meaning] at (-38.050000, 17.150000) {prohibition};
\node[Square] at (-36.000000, 15.400000) {};
\node[Kanji] at (-36.000000, 15.900000) {症};
\node[Onyomi] at (-35.950000, 15.500000) {ショウ};
\node[Meaning] at (-36.000000, 17.150000) {symptom};
\node[Square] at (-33.950000, 15.400000) {};
\node[Kanji] at (-33.950000, 15.900000) {弱};
\node[Onyomi] at (-33.900000, 15.500000) {ジャク};
\node[Kunyomi] at (-34.000000, 15.500000) {よわ.い};
\node[Meaning] at (-33.950000, 17.150000) {weak};
\node[Square] at (-31.900000, 15.400000) {};
\node[Kanji] at (-31.900000, 15.900000) {妻};
\node[Onyomi] at (-31.850000, 15.500000) {サイ};
\node[Kunyomi] at (-31.950000, 15.500000) {つま};
\node[Meaning] at (-31.900000, 17.150000) {wife};
\node[Square] at (-29.850000, 15.400000) {};
\node[Kanji] at (-29.850000, 15.900000) {縄};
\node[Onyomi] at (-29.800000, 15.500000) {ジョウ};
\node[Kunyomi] at (-29.900000, 15.500000) {なわ};
\node[Meaning] at (-29.850000, 17.150000) {rope};
\node[Square] at (-27.800000, 15.400000) {};
\node[Kanji] at (-27.800000, 15.900000) {任};
\node[Onyomi] at (-27.750000, 15.500000) {ニン};
\node[Kunyomi] at (-27.850000, 15.500000) {まか.せる};
\node[Meaning] at (-27.800000, 17.150000) {duty};
\node[Square] at (-25.750000, 15.400000) {};
\node[Kanji] at (-25.750000, 15.900000) {奇};
\node[Onyomi] at (-25.700000, 15.500000) {キ};
\node[Meaning] at (-25.750000, 17.150000) {odd};
\node[Square] at (-23.700000, 15.400000) {};
\node[Kanji] at (-23.700000, 15.900000) {賞};
\node[Onyomi] at (-23.650000, 15.500000) {ショウ};
\node[Meaning] at (-23.700000, 17.150000) {prize};
\node[Square] at (-21.650000, 15.400000) {};
\node[Kanji] at (-21.650000, 15.900000) {吐};
\node[Onyomi] at (-21.600000, 15.500000) {ト};
\node[Kunyomi] at (-21.700000, 15.500000) {は};
\node[Meaning] at (-21.650000, 17.150000) {throw up};
\node[Square] at (-19.600000, 15.400000) {};
\node[Kanji] at (-19.600000, 15.900000) {妹};
\node[Onyomi] at (-19.550000, 15.500000) {マイ};
\node[Kunyomi] at (-19.650000, 15.500000) {いもうと};
\node[Meaning] at (-19.600000, 17.150000) {younger sister};
\node[Square] at (-17.550000, 15.400000) {};
\node[Kanji] at (-17.550000, 15.900000) {勉};
\node[Onyomi] at (-17.500000, 15.500000) {ベン};
\node[Meaning] at (-17.550000, 17.150000) {exertion};
\node[Square] at (-15.500000, 15.400000) {};
\node[Kanji] at (-15.500000, 15.900000) {抗};
\node[Onyomi] at (-15.450000, 15.500000) {コウ};
\node[Kunyomi] at (-15.550000, 15.500000) {あらが.う};
\node[Meaning] at (-15.500000, 17.150000) {confront};
\node[Square] at (-13.450000, 15.400000) {};
\node[Kanji] at (-13.450000, 15.900000) {則};
\node[Onyomi] at (-13.400000, 15.500000) {ソク};
\node[Kunyomi] at (-13.500000, 15.500000) {のっと.る};
\node[Meaning] at (-13.450000, 17.150000) {rule};
\node[Square] at (-11.400000, 15.400000) {};
\node[Kanji] at (-11.400000, 15.900000) {惑};
\node[Onyomi] at (-11.350000, 15.500000) {ワク};
\node[Kunyomi] at (-11.450000, 15.500000) {まど.う};
\node[Meaning] at (-11.400000, 17.150000) {misguided};
\node[Square] at (-9.350000, 15.400000) {};
\node[Kanji] at (-9.350000, 15.900000) {飾};
\node[Onyomi] at (-9.300000, 15.500000) {ショク};
\node[Kunyomi] at (-9.400000, 15.500000) {かざ.る};
\node[Meaning] at (-9.350000, 17.150000) {decorate};
\node[Square] at (-7.300000, 15.400000) {};
\node[Kanji] at (-7.300000, 15.900000) {縦};
\node[Onyomi] at (-7.250000, 15.500000) {ジュウ};
\node[Kunyomi] at (-7.350000, 15.500000) {たて};
\node[Meaning] at (-7.300000, 17.150000) {vertical};
\node[Square] at (-5.250000, 15.400000) {};
\node[Kanji] at (-5.250000, 15.900000) {折};
\node[Onyomi] at (-5.200000, 15.500000) {セツ};
\node[Kunyomi] at (-5.300000, 15.500000) {お.る};
\node[Meaning] at (-5.250000, 17.150000) {fold};
\node[Square] at (-3.200000, 15.400000) {};
\node[Kanji] at (-3.200000, 15.900000) {屈};
\node[Onyomi] at (-3.150000, 15.500000) {クツ};
\node[Kunyomi] at (-3.250000, 15.500000) {かが};
\node[Meaning] at (-3.200000, 17.150000) {yield};
\node[Square] at (-1.150000, 15.400000) {};
\node[Kanji] at (-1.150000, 15.900000) {奮};
\node[Onyomi] at (-1.100000, 15.500000) {フン};
\node[Kunyomi] at (-1.200000, 15.500000) {ふる.*};
\node[Meaning] at (-1.150000, 17.150000) {stirred up};
\node[Square] at (0.900000, 15.400000) {};
\node[Kanji] at (0.900000, 15.900000) {類};
\node[Onyomi] at (0.950000, 15.500000) {ルイ};
\node[Kunyomi] at (0.850000, 15.500000) {たぐ.い};
\node[Meaning] at (0.900000, 17.150000) {type};
\node[Square] at (2.950000, 15.400000) {};
\node[Kanji] at (2.950000, 15.900000) {帯};
\node[Onyomi] at (3.000000, 15.500000) {タイ};
\node[Kunyomi] at (2.900000, 15.500000) {おび};
\node[Meaning] at (2.950000, 17.150000) {belt};
\node[Square] at (5.000000, 15.400000) {};
\node[Kanji] at (5.000000, 15.900000) {刻};
\node[Onyomi] at (5.050000, 15.500000) {コク};
\node[Kunyomi] at (4.950000, 15.500000) {きざ.む};
\node[Meaning] at (5.000000, 17.150000) {carve};
\node[Square] at (7.050000, 15.400000) {};
\node[Kanji] at (7.050000, 15.900000) {緑};
\node[Onyomi] at (7.100000, 15.500000) {リョク};
\node[Kunyomi] at (7.000000, 15.500000) {みどり};
\node[Meaning] at (7.050000, 17.150000) {green};
\node[Square] at (9.100000, 15.400000) {};
\node[Kanji] at (9.100000, 15.900000) {骨};
\node[Onyomi] at (9.150000, 15.500000) {コツ};
\node[Kunyomi] at (9.050000, 15.500000) {ほね};
\node[Meaning] at (9.100000, 17.150000) {bone};
\node[Square] at (11.150000, 15.400000) {};
\node[Kanji] at (11.150000, 15.900000) {非};
\node[Onyomi] at (11.200000, 15.500000) {ヒ};
\node[Meaning] at (11.150000, 17.150000) {injustice};
\node[Square] at (13.200000, 15.400000) {};
\node[Kanji] at (13.200000, 15.900000) {撮};
\node[Onyomi] at (13.250000, 15.500000) {サツ};
\node[Kunyomi] at (13.150000, 15.500000) {と.る};
\node[Meaning] at (13.200000, 17.150000) {photograph};
\node[Square] at (15.250000, 15.400000) {};
\node[Kanji] at (15.250000, 15.900000) {狂};
\node[Onyomi] at (15.300000, 15.500000) {キョウ};
\node[Kunyomi] at (15.200000, 15.500000) {くる.*};
\node[Meaning] at (15.250000, 17.150000) {lunatic};
\node[Square] at (17.300000, 15.400000) {};
\node[Kanji] at (17.300000, 15.900000) {存};
\node[Onyomi] at (17.350000, 15.500000) {ソン};
\node[Meaning] at (17.300000, 17.150000) {suppose};
\node[Square] at (19.350000, 15.400000) {};
\node[Kanji] at (19.350000, 15.900000) {痕};
\node[Onyomi] at (19.400000, 15.500000) {コン};
\node[Kunyomi] at (19.300000, 15.500000) {あと};
\node[Meaning] at (19.350000, 17.150000) {mark};
\node[Square] at (21.400000, 15.400000) {};
\node[Kanji] at (21.400000, 15.900000) {卵};
\node[Onyomi] at (21.450000, 15.500000) {ラン};
\node[Kunyomi] at (21.350000, 15.500000) {たまご};
\node[Meaning] at (21.400000, 17.150000) {egg};
\node[Square] at (23.450000, 15.400000) {};
\node[Kanji] at (23.450000, 15.900000) {眼};
\node[Onyomi] at (23.500000, 15.500000) {ガン};
\node[Kunyomi] at (23.400000, 15.500000) {め};
\node[Meaning] at (23.450000, 17.150000) {eyeball};
\node[Square] at (25.500000, 15.400000) {};
\node[Kanji] at (25.500000, 15.900000) {眺};
\node[Onyomi] at (25.550000, 15.500000) {チョウ};
\node[Kunyomi] at (25.450000, 15.500000) {なが.める};
\node[Meaning] at (25.500000, 17.150000) {stare};
\node[Square] at (27.550000, 15.400000) {};
\node[Kanji] at (27.550000, 15.900000) {噴};
\node[Onyomi] at (27.600000, 15.500000) {フン};
\node[Kunyomi] at (27.500000, 15.500000) {ふ};
\node[Meaning] at (27.550000, 17.150000) {erupt};
\node[Square] at (29.600000, 15.400000) {};
\node[Kanji] at (29.600000, 15.900000) {仲};
\node[Onyomi] at (29.650000, 15.500000) {チュウ};
\node[Kunyomi] at (29.550000, 15.500000) {なか};
\node[Meaning] at (29.600000, 17.150000) {relationship};
\node[Square] at (31.650000, 15.400000) {};
\node[Kanji] at (31.650000, 15.900000) {便};
\node[Onyomi] at (31.700000, 15.500000) {ベン};
\node[Kunyomi] at (31.600000, 15.500000) {たよ.*};
\node[Meaning] at (31.650000, 17.150000) {convenience};
\node[Square] at (33.700000, 15.400000) {};
\node[Kanji] at (33.700000, 15.900000) {従};
\node[Onyomi] at (33.750000, 15.500000) {ジュウ};
\node[Kunyomi] at (33.650000, 15.500000) {したが.う};
\node[Meaning] at (33.700000, 17.150000) {obey};
\node[Square] at (35.750000, 15.400000) {};
\node[Kanji] at (35.750000, 15.900000) {億};
\node[Onyomi] at (35.800000, 15.500000) {オク};
\node[Meaning] at (35.750000, 17.150000) {100 million};
\node[Square] at (37.800000, 15.400000) {};
\node[Kanji] at (37.800000, 15.900000) {到};
\node[Onyomi] at (37.850000, 15.500000) {トウ};
\node[Meaning] at (37.800000, 17.150000) {arrival};
\node[Square] at (39.850000, 15.400000) {};
\node[Kanji] at (39.850000, 15.900000) {棚};
\node[Onyomi] at (39.900000, 15.500000) {ホウ};
\node[Kunyomi] at (39.800000, 15.500000) {たな};
\node[Meaning] at (39.850000, 17.150000) {shelf};
\node[Square] at (41.900000, 15.400000) {};
\node[Kanji] at (41.900000, 15.900000) {南};
\node[Onyomi] at (41.950000, 15.500000) {ナン};
\node[Kunyomi] at (41.850000, 15.500000) {みなみ};
\node[Meaning] at (41.900000, 17.150000) {south};
\node[Square] at (43.950000, 15.400000) {};
\node[Kanji] at (43.950000, 15.900000) {協};
\node[Onyomi] at (44.000000, 15.500000) {キョウ};
\node[Meaning] at (43.950000, 17.150000) {cooperation};
\node[Square] at (46.000000, 15.400000) {};
\node[Kanji] at (46.000000, 15.900000) {乱};
\node[Onyomi] at (46.050000, 15.500000) {ラン};
\node[Kunyomi] at (45.950000, 15.500000) {みだ.す};
\node[Meaning] at (46.000000, 17.150000) {riot};
\node[Square] at (48.050000, 15.400000) {};
\node[Kanji] at (48.050000, 15.900000) {挙};
\node[Onyomi] at (48.100000, 15.500000) {キョ};
\node[Kunyomi] at (48.000000, 15.500000) {あ.がる};
\node[Meaning] at (48.050000, 17.150000) {raise};
\node[Square] at (50.100000, 15.400000) {};
\node[Kanji] at (50.100000, 15.900000) {群};
\node[Onyomi] at (50.150000, 15.500000) {グン};
\node[Kunyomi] at (50.050000, 15.500000) {む};
\node[Meaning] at (50.100000, 17.150000) {flock};
\node[Square] at (52.150000, 15.400000) {};
\node[Kanji] at (52.150000, 15.900000) {燃};
\node[Onyomi] at (52.200000, 15.500000) {ネン};
\node[Kunyomi] at (52.100000, 15.500000) {も.*};
\node[Meaning] at (52.150000, 17.150000) {burn};
\node[Square] at (54.200000, 15.400000) {};
\node[Kanji] at (54.200000, 15.900000) {与};
\node[Onyomi] at (54.250000, 15.500000) {ヨ};
\node[Kunyomi] at (54.150000, 15.500000) {あた.える};
\node[Meaning] at (54.200000, 17.150000) {give};
\node[Square] at (56.250000, 15.400000) {};
\node[Kanji] at (56.250000, 15.900000) {未};
\node[Onyomi] at (56.300000, 15.500000) {ミ};
\node[Kunyomi] at (56.200000, 15.500000) {ま.だ};
\node[Meaning] at (56.250000, 17.150000) {not yet};
\node[Meaning] at (-58.500000, 15.950000) {87.91\%};
\node[Square] at (-56.500000, 13.350000) {};
\node[Kanji] at (-56.500000, 13.850000) {煙};
\node[Onyomi] at (-56.450000, 13.450000) {エン};
\node[Kunyomi] at (-56.550000, 13.450000) {けむ.り};
\node[Meaning] at (-56.500000, 15.100000) {smoke};
\node[Square] at (-54.450000, 13.350000) {};
\node[Kanji] at (-54.450000, 13.850000) {久};
\node[Onyomi] at (-54.400000, 13.450000) {キュウ};
\node[Kunyomi] at (-54.500000, 13.450000) {ひさ};
\node[Meaning] at (-54.450000, 15.100000) {long time};
\node[Square] at (-52.400000, 13.350000) {};
\node[Kanji] at (-52.400000, 13.850000) {論};
\node[Onyomi] at (-52.350000, 13.450000) {ロン};
\node[Meaning] at (-52.400000, 15.100000) {theory};
\node[Square] at (-50.350000, 13.350000) {};
\node[Kanji] at (-50.350000, 13.850000) {輩};
\node[Onyomi] at (-50.300000, 13.450000) {ハイ};
\node[Meaning] at (-50.350000, 15.100000) {comrade};
\node[Square] at (-48.300000, 13.350000) {};
\node[Kanji] at (-48.300000, 13.850000) {紹};
\node[Onyomi] at (-48.250000, 13.450000) {ショウ};
\node[Meaning] at (-48.300000, 15.100000) {introduce};
\node[Square] at (-46.250000, 13.350000) {};
\node[Kanji] at (-46.250000, 13.850000) {脚};
\node[Onyomi] at (-46.200000, 13.450000) {キャク};
\node[Kunyomi] at (-46.300000, 13.450000) {あし};
\node[Meaning] at (-46.250000, 15.100000) {leg};
\node[Square] at (-44.200000, 13.350000) {};
\node[Kanji] at (-44.200000, 13.850000) {陰};
\node[Onyomi] at (-44.150000, 13.450000) {イン};
\node[Kunyomi] at (-44.250000, 13.450000) {かげ};
\node[Meaning] at (-44.200000, 15.100000) {shade};
\node[Square] at (-42.150000, 13.350000) {};
\node[Kanji] at (-42.150000, 13.850000) {束};
\node[Onyomi] at (-42.100000, 13.450000) {ソク};
\node[Kunyomi] at (-42.200000, 13.450000) {たば};
\node[Meaning] at (-42.150000, 15.100000) {bundle};
\node[Square] at (-40.100000, 13.350000) {};
\node[Kanji] at (-40.100000, 13.850000) {厚};
\node[Onyomi] at (-40.050000, 13.450000) {コウ};
\node[Kunyomi] at (-40.150000, 13.450000) {あつ};
\node[Meaning] at (-40.100000, 15.100000) {thick};
\node[Square] at (-38.050000, 13.350000) {};
\node[Kanji] at (-38.050000, 13.850000) {慌};
\node[Onyomi] at (-38.000000, 13.450000) {コウ};
\node[Kunyomi] at (-38.100000, 13.450000) {あわ-てる};
\node[Meaning] at (-38.050000, 15.100000) {disconcerted};
\node[Square] at (-36.000000, 13.350000) {};
\node[Kanji] at (-36.000000, 13.850000) {因};
\node[Onyomi] at (-35.950000, 13.450000) {イン};
\node[Kunyomi] at (-36.050000, 13.450000) {よ};
\node[Meaning] at (-36.000000, 15.100000) {cause};
\node[Square] at (-33.950000, 13.350000) {};
\node[Kanji] at (-33.950000, 13.850000) {妙};
\node[Onyomi] at (-33.900000, 13.450000) {ミョウ};
\node[Kunyomi] at (-34.000000, 13.450000) {たえ.なる};
\node[Meaning] at (-33.950000, 15.100000) {strange};
\node[Square] at (-31.900000, 13.350000) {};
\node[Kanji] at (-31.900000, 13.850000) {除};
\node[Onyomi] at (-31.850000, 13.450000) {ジョ};
\node[Kunyomi] at (-31.950000, 13.450000) {のぞ.く};
\node[Meaning] at (-31.900000, 15.100000) {exclude};
\node[Square] at (-29.850000, 13.350000) {};
\node[Kanji] at (-29.850000, 13.850000) {弾};
\node[Onyomi] at (-29.800000, 13.450000) {ダン};
\node[Kunyomi] at (-29.900000, 13.450000) {ひ.く};
\node[Meaning] at (-29.850000, 15.100000) {bullet};
\node[Square] at (-27.800000, 13.350000) {};
\node[Kanji] at (-27.800000, 13.850000) {労};
\node[Onyomi] at (-27.750000, 13.450000) {ロウ};
\node[Kunyomi] at (-27.850000, 13.450000) {いたわ.る};
\node[Meaning] at (-27.800000, 15.100000) {labor};
\node[Square] at (-25.750000, 13.350000) {};
\node[Kanji] at (-25.750000, 13.850000) {鍋};
\node[Kunyomi] at (-25.800000, 13.450000) {なべ};
\node[Meaning] at (-25.750000, 15.100000) {pot};
\node[Square] at (-23.700000, 13.350000) {};
\node[Kanji] at (-23.700000, 13.850000) {晩};
\node[Onyomi] at (-23.650000, 13.450000) {バン};
\node[Meaning] at (-23.700000, 15.100000) {night};
\node[Square] at (-21.650000, 13.350000) {};
\node[Kanji] at (-21.650000, 13.850000) {復};
\node[Onyomi] at (-21.600000, 13.450000) {フク};
\node[Meaning] at (-21.650000, 15.100000) {restore};
\node[Square] at (-19.600000, 13.350000) {};
\node[Kanji] at (-19.600000, 13.850000) {医};
\node[Onyomi] at (-19.550000, 13.450000) {イ};
\node[Meaning] at (-19.600000, 15.100000) {medicine};
\node[Square] at (-17.550000, 13.350000) {};
\node[Kanji] at (-17.550000, 13.850000) {軍};
\node[Onyomi] at (-17.500000, 13.450000) {グン};
\node[Meaning] at (-17.550000, 15.100000) {army};
\node[Square] at (-15.500000, 13.350000) {};
\node[Kanji] at (-15.500000, 13.850000) {照};
\node[Onyomi] at (-15.450000, 13.450000) {ショウ};
\node[Kunyomi] at (-15.550000, 13.450000) {て.*};
\node[Meaning] at (-15.500000, 15.100000) {illuminate};
\node[Square] at (-13.450000, 13.350000) {};
\node[Kanji] at (-13.450000, 13.850000) {膝};
\node[Kunyomi] at (-13.500000, 13.450000) {ひざ};
\node[Meaning] at (-13.450000, 15.100000) {knee};
\node[Square] at (-11.400000, 13.350000) {};
\node[Kanji] at (-11.400000, 13.850000) {脱};
\node[Onyomi] at (-11.350000, 13.450000) {ダツ};
\node[Kunyomi] at (-11.450000, 13.450000) {ぬ.ぐ};
\node[Meaning] at (-11.400000, 15.100000) {undress};
\node[Square] at (-9.350000, 13.350000) {};
\node[Kanji] at (-9.350000, 13.850000) {草};
\node[Onyomi] at (-9.300000, 13.450000) {ソウ};
\node[Kunyomi] at (-9.400000, 13.450000) {くさ};
\node[Meaning] at (-9.350000, 15.100000) {grass};
\node[Square] at (-7.300000, 13.350000) {};
\node[Kanji] at (-7.300000, 13.850000) {械};
\node[Onyomi] at (-7.250000, 13.450000) {カイ};
\node[Kunyomi] at (-7.350000, 13.450000) {かせ};
\node[Meaning] at (-7.300000, 15.100000) {contraption};
\node[Square] at (-5.250000, 13.350000) {};
\node[Kanji] at (-5.250000, 13.850000) {救};
\node[Onyomi] at (-5.200000, 13.450000) {キュウ};
\node[Kunyomi] at (-5.300000, 13.450000) {すく.う};
\node[Meaning] at (-5.250000, 15.100000) {rescue};
\node[Square] at (-3.200000, 13.350000) {};
\node[Kanji] at (-3.200000, 13.850000) {瓶};
\node[Onyomi] at (-3.150000, 13.450000) {ビン};
\node[Kunyomi] at (-3.250000, 13.450000) {かめ};
\node[Meaning] at (-3.200000, 15.100000) {bottle};
\node[Square] at (-1.150000, 13.350000) {};
\node[Kanji] at (-1.150000, 13.850000) {甲};
\node[Onyomi] at (-1.100000, 13.450000) {コウ};
\node[Kunyomi] at (-1.200000, 13.450000) {か};
\node[Meaning] at (-1.150000, 15.100000) {turtle shell};
\node[Square] at (0.900000, 13.350000) {};
\node[Kanji] at (0.900000, 13.850000) {案};
\node[Onyomi] at (0.950000, 13.450000) {アン};
\node[Meaning] at (0.900000, 15.100000) {plan};
\node[Square] at (2.950000, 13.350000) {};
\node[Kanji] at (2.950000, 13.850000) {富};
\node[Onyomi] at (3.000000, 13.450000) {フ};
\node[Kunyomi] at (2.900000, 13.450000) {と};
\node[Meaning] at (2.950000, 15.100000) {rich};
\node[Square] at (5.000000, 13.350000) {};
\node[Kanji] at (5.000000, 13.850000) {州};
\node[Onyomi] at (5.050000, 13.450000) {シュウ};
\node[Meaning] at (5.000000, 15.100000) {state};
\node[Square] at (7.050000, 13.350000) {};
\node[Kanji] at (7.050000, 13.850000) {固};
\node[Onyomi] at (7.100000, 13.450000) {コ};
\node[Kunyomi] at (7.000000, 13.450000) {かた.い};
\node[Meaning] at (7.050000, 15.100000) {hard};
\node[Square] at (9.100000, 13.350000) {};
\node[Kanji] at (9.100000, 13.850000) {領};
\node[Onyomi] at (9.150000, 13.450000) {リョウ};
\node[Meaning] at (9.100000, 15.100000) {territory};
\node[Square] at (11.150000, 13.350000) {};
\node[Kanji] at (11.150000, 13.850000) {肖};
\node[Onyomi] at (11.200000, 13.450000) {ショウ};
\node[Kunyomi] at (11.100000, 13.450000) {あやか};
\node[Meaning] at (11.150000, 15.100000) {resemblance};
\node[Square] at (13.200000, 13.350000) {};
\node[Kanji] at (13.200000, 13.850000) {老};
\node[Onyomi] at (13.250000, 13.450000) {ロウ};
\node[Meaning] at (13.200000, 15.100000) {elderly};
\node[Square] at (15.250000, 13.350000) {};
\node[Kanji] at (15.250000, 13.850000) {専};
\node[Onyomi] at (15.300000, 13.450000) {セン};
\node[Kunyomi] at (15.200000, 13.450000) {もっぱ.ら};
\node[Meaning] at (15.250000, 15.100000) {specialty};
\node[Square] at (17.300000, 13.350000) {};
\node[Kanji] at (17.300000, 13.850000) {宮};
\node[Onyomi] at (17.350000, 13.450000) {キュウ};
\node[Kunyomi] at (17.250000, 13.450000) {みや};
\node[Meaning] at (17.300000, 15.100000) {shinto shrine};
\node[Square] at (19.350000, 13.350000) {};
\node[Kanji] at (19.350000, 13.850000) {鉄};
\node[Onyomi] at (19.400000, 13.450000) {テツ};
\node[Meaning] at (19.350000, 15.100000) {iron};
\node[Square] at (21.400000, 13.350000) {};
\node[Kanji] at (21.400000, 13.850000) {比};
\node[Onyomi] at (21.450000, 13.450000) {ヒ};
\node[Kunyomi] at (21.350000, 13.450000) {くら.べる};
\node[Meaning] at (21.400000, 15.100000) {compare};
\node[Square] at (23.450000, 13.350000) {};
\node[Kanji] at (23.450000, 13.850000) {洋};
\node[Onyomi] at (23.500000, 13.450000) {ヨウ};
\node[Meaning] at (23.450000, 15.100000) {western style};
\node[Square] at (25.500000, 13.350000) {};
\node[Kanji] at (25.500000, 13.850000) {統};
\node[Onyomi] at (25.550000, 13.450000) {トウ};
\node[Kunyomi] at (25.450000, 13.450000) {す.べる};
\node[Meaning] at (25.500000, 15.100000) {unite};
\node[Square] at (27.550000, 13.350000) {};
\node[Kanji] at (27.550000, 13.850000) {繰};
\node[Onyomi] at (27.600000, 13.450000) {ソウ};
\node[Kunyomi] at (27.500000, 13.450000) {く};
\node[Meaning] at (27.550000, 15.100000) {spin};
\node[Square] at (29.600000, 13.350000) {};
\node[Kanji] at (29.600000, 13.850000) {婚};
\node[Onyomi] at (29.650000, 13.450000) {コン};
\node[Meaning] at (29.600000, 15.100000) {marriage};
\node[Square] at (31.650000, 13.350000) {};
\node[Kanji] at (31.650000, 13.850000) {跡};
\node[Onyomi] at (31.700000, 13.450000) {セキ};
\node[Kunyomi] at (31.600000, 13.450000) {あと};
\node[Meaning] at (31.650000, 15.100000) {traces};
\node[Square] at (33.700000, 13.350000) {};
\node[Kanji] at (33.700000, 13.850000) {鍵};
\node[Onyomi] at (33.750000, 13.450000) {ケン};
\node[Kunyomi] at (33.650000, 13.450000) {かぎ};
\node[Meaning] at (33.700000, 15.100000) {key};
\node[Square] at (35.750000, 13.350000) {};
\node[Kanji] at (35.750000, 13.850000) {描};
\node[Onyomi] at (35.800000, 13.450000) {ビョウ};
\node[Kunyomi] at (35.700000, 13.450000) {か.く};
\node[Meaning] at (35.750000, 15.100000) {draw};
\node[Square] at (37.800000, 13.350000) {};
\node[Kanji] at (37.800000, 13.850000) {短};
\node[Onyomi] at (37.850000, 13.450000) {タン};
\node[Kunyomi] at (37.750000, 13.450000) {みじか.い};
\node[Meaning] at (37.800000, 15.100000) {short};
\node[Square] at (39.850000, 13.350000) {};
\node[Kanji] at (39.850000, 13.850000) {百};
\node[Onyomi] at (39.900000, 13.450000) {ヒャク};
\node[Meaning] at (39.850000, 15.100000) {hundred};
\node[Square] at (41.900000, 13.350000) {};
\node[Kanji] at (41.900000, 13.850000) {識};
\node[Onyomi] at (41.950000, 13.450000) {シキ};
\node[Meaning] at (41.900000, 15.100000) {discerning};
\node[Square] at (43.950000, 13.350000) {};
\node[Kanji] at (43.950000, 13.850000) {尾};
\node[Onyomi] at (44.000000, 13.450000) {ビ};
\node[Kunyomi] at (43.900000, 13.450000) {お};
\node[Meaning] at (43.950000, 15.100000) {tail};
\node[Square] at (46.000000, 13.350000) {};
\node[Kanji] at (46.000000, 13.850000) {刺};
\node[Onyomi] at (46.050000, 13.450000) {シ};
\node[Kunyomi] at (45.950000, 13.450000) {さ.*};
\node[Meaning] at (46.000000, 15.100000) {stab};
\node[Square] at (48.050000, 13.350000) {};
\node[Kanji] at (48.050000, 13.850000) {谷};
\node[Onyomi] at (48.100000, 13.450000) {コク};
\node[Kunyomi] at (48.000000, 13.450000) {たに};
\node[Meaning] at (48.050000, 15.100000) {valley};
\node[Square] at (50.100000, 13.350000) {};
\node[Kanji] at (50.100000, 13.850000) {逆};
\node[Onyomi] at (50.150000, 13.450000) {ギャク};
\node[Kunyomi] at (50.050000, 13.450000) {さか.らう};
\node[Meaning] at (50.100000, 15.100000) {reverse};
\node[Square] at (52.150000, 13.350000) {};
\node[Kanji] at (52.150000, 13.850000) {載};
\node[Onyomi] at (52.200000, 13.450000) {サイ};
\node[Kunyomi] at (52.100000, 13.450000) {の.せる};
\node[Meaning] at (52.150000, 15.100000) {publish};
\node[Square] at (54.200000, 13.350000) {};
\node[Kanji] at (54.200000, 13.850000) {録};
\node[Onyomi] at (54.250000, 13.450000) {ロク};
\node[Meaning] at (54.200000, 15.100000) {record};
\node[Square] at (56.250000, 13.350000) {};
\node[Kanji] at (56.250000, 13.850000) {簡};
\node[Onyomi] at (56.300000, 13.450000) {カン};
\node[Meaning] at (56.250000, 15.100000) {simplicity};
\node[Meaning] at (-58.500000, 13.900000) {89.34\%};
\node[Square] at (-56.500000, 11.300000) {};
\node[Kanji] at (-56.500000, 11.800000) {輪};
\node[Onyomi] at (-56.450000, 11.400000) {リン};
\node[Kunyomi] at (-56.550000, 11.400000) {わ};
\node[Meaning] at (-56.500000, 13.050000) {wheel};
\node[Square] at (-54.450000, 11.300000) {};
\node[Kanji] at (-54.450000, 11.800000) {臓};
\node[Onyomi] at (-54.400000, 11.400000) {ゾウ};
\node[Meaning] at (-54.450000, 13.050000) {internal organs};
\node[Square] at (-52.400000, 11.300000) {};
\node[Kanji] at (-52.400000, 11.800000) {底};
\node[Onyomi] at (-52.350000, 11.400000) {テイ};
\node[Kunyomi] at (-52.450000, 11.400000) {そこ};
\node[Meaning] at (-52.400000, 13.050000) {bottom};
\node[Square] at (-50.350000, 11.300000) {};
\node[Kanji] at (-50.350000, 11.800000) {迷};
\node[Onyomi] at (-50.300000, 11.400000) {メイ};
\node[Kunyomi] at (-50.400000, 11.400000) {まよ.う};
\node[Meaning] at (-50.350000, 13.050000) {astray};
\node[Square] at (-48.300000, 11.300000) {};
\node[Kanji] at (-48.300000, 11.800000) {互};
\node[Onyomi] at (-48.250000, 11.400000) {ゴ};
\node[Kunyomi] at (-48.350000, 11.400000) {たが.い};
\node[Meaning] at (-48.300000, 13.050000) {mutual};
\node[Square] at (-46.250000, 11.300000) {};
\node[Kanji] at (-46.250000, 11.800000) {宝};
\node[Onyomi] at (-46.200000, 11.400000) {ホウ};
\node[Kunyomi] at (-46.300000, 11.400000) {たから};
\node[Meaning] at (-46.250000, 13.050000) {treasure};
\node[Square] at (-44.200000, 11.300000) {};
\node[Kanji] at (-44.200000, 11.800000) {騒};
\node[Onyomi] at (-44.150000, 11.400000) {ソウ};
\node[Kunyomi] at (-44.250000, 11.400000) {さわ.ぐ};
\node[Meaning] at (-44.200000, 13.050000) {boisterous};
\node[Square] at (-42.150000, 11.300000) {};
\node[Kanji] at (-42.150000, 11.800000) {督};
\node[Onyomi] at (-42.100000, 11.400000) {トク};
\node[Meaning] at (-42.150000, 13.050000) {coach};
\node[Square] at (-40.100000, 11.300000) {};
\node[Kanji] at (-40.100000, 11.800000) {匹};
\node[Kunyomi] at (-40.150000, 11.400000) {ひき};
\node[Meaning] at (-40.100000, 13.050000) {small animal};
\node[Square] at (-38.050000, 11.300000) {};
\node[Kanji] at (-38.050000, 11.800000) {絡};
\node[Onyomi] at (-38.000000, 11.400000) {ラク};
\node[Kunyomi] at (-38.100000, 11.400000) {から.む};
\node[Meaning] at (-38.050000, 13.050000) {entangle};
\node[Square] at (-36.000000, 11.300000) {};
\node[Kanji] at (-36.000000, 11.800000) {疲};
\node[Onyomi] at (-35.950000, 11.400000) {ヒ};
\node[Kunyomi] at (-36.050000, 11.400000) {つか.れる};
\node[Meaning] at (-36.000000, 13.050000) {exhausted};
\node[Square] at (-33.950000, 11.300000) {};
\node[Kanji] at (-33.950000, 11.800000) {礼};
\node[Onyomi] at (-33.900000, 11.400000) {レイ};
\node[Meaning] at (-33.950000, 13.050000) {thanks};
\node[Square] at (-31.900000, 11.300000) {};
\node[Kanji] at (-31.900000, 11.800000) {洗};
\node[Onyomi] at (-31.850000, 11.400000) {セン};
\node[Kunyomi] at (-31.950000, 11.400000) {あら.う};
\node[Meaning] at (-31.900000, 13.050000) {wash};
\node[Square] at (-29.850000, 11.300000) {};
\node[Kanji] at (-29.850000, 11.800000) {罰};
\node[Onyomi] at (-29.800000, 11.400000) {バツ};
\node[Kunyomi] at (-29.900000, 11.400000) {ばつ};
\node[Meaning] at (-29.850000, 13.050000) {penalty};
\node[Square] at (-27.800000, 11.300000) {};
\node[Kanji] at (-27.800000, 11.800000) {昼};
\node[Kunyomi] at (-27.850000, 11.400000) {ひる};
\node[Meaning] at (-27.800000, 13.050000) {noon};
\node[Square] at (-25.750000, 11.300000) {};
\node[Kanji] at (-25.750000, 11.800000) {迎};
\node[Onyomi] at (-25.700000, 11.400000) {ゲイ};
\node[Kunyomi] at (-25.800000, 11.400000) {むか.える};
\node[Meaning] at (-25.750000, 13.050000) {welcome};
\node[Square] at (-23.700000, 11.300000) {};
\node[Kanji] at (-23.700000, 11.800000) {尋};
\node[Onyomi] at (-23.650000, 11.400000) {ジン};
\node[Kunyomi] at (-23.750000, 11.400000) {たず.ねる};
\node[Meaning] at (-23.700000, 13.050000) {inquire};
\node[Square] at (-21.650000, 11.300000) {};
\node[Kanji] at (-21.650000, 11.800000) {求};
\node[Onyomi] at (-21.600000, 11.400000) {キュウ};
\node[Kunyomi] at (-21.700000, 11.400000) {もと.める};
\node[Meaning] at (-21.650000, 13.050000) {request};
\node[Square] at (-19.600000, 11.300000) {};
\node[Kanji] at (-19.600000, 11.800000) {隅};
\node[Onyomi] at (-19.550000, 11.400000) {グウ};
\node[Kunyomi] at (-19.650000, 11.400000) {すみ};
\node[Meaning] at (-19.600000, 13.050000) {corner};
\node[Square] at (-17.550000, 11.300000) {};
\node[Kanji] at (-17.550000, 11.800000) {効};
\node[Onyomi] at (-17.500000, 11.400000) {コウ};
\node[Kunyomi] at (-17.600000, 11.400000) {き.く};
\node[Meaning] at (-17.550000, 13.050000) {effective};
\node[Square] at (-15.500000, 11.300000) {};
\node[Kanji] at (-15.500000, 11.800000) {健};
\node[Onyomi] at (-15.450000, 11.400000) {ケン};
\node[Meaning] at (-15.500000, 13.050000) {healthy};
\node[Square] at (-13.450000, 11.300000) {};
\node[Kanji] at (-13.450000, 11.800000) {踏};
\node[Onyomi] at (-13.400000, 11.400000) {トウ};
\node[Kunyomi] at (-13.500000, 11.400000) {ふ.む};
\node[Meaning] at (-13.450000, 13.050000) {step};
\node[Square] at (-11.400000, 11.300000) {};
\node[Kanji] at (-11.400000, 11.800000) {攻};
\node[Onyomi] at (-11.350000, 11.400000) {コウ};
\node[Kunyomi] at (-11.450000, 11.400000) {せ.める};
\node[Meaning] at (-11.400000, 13.050000) {aggression};
\node[Square] at (-9.350000, 11.300000) {};
\node[Kanji] at (-9.350000, 11.800000) {設};
\node[Onyomi] at (-9.300000, 11.400000) {セツ};
\node[Kunyomi] at (-9.400000, 11.400000) {もう.ける};
\node[Meaning] at (-9.350000, 13.050000) {establish};
\node[Square] at (-7.300000, 11.300000) {};
\node[Kanji] at (-7.300000, 11.800000) {板};
\node[Onyomi] at (-7.250000, 11.400000) {ハン};
\node[Kunyomi] at (-7.350000, 11.400000) {いた};
\node[Meaning] at (-7.300000, 13.050000) {board};
\node[Square] at (-5.250000, 11.300000) {};
\node[Kanji] at (-5.250000, 11.800000) {荷};
\node[Onyomi] at (-5.200000, 11.400000) {カ};
\node[Kunyomi] at (-5.300000, 11.400000) {に};
\node[Meaning] at (-5.250000, 13.050000) {luggage};
\node[Square] at (-3.200000, 11.300000) {};
\node[Kanji] at (-3.200000, 11.800000) {官};
\node[Onyomi] at (-3.150000, 11.400000) {カン};
\node[Meaning] at (-3.200000, 13.050000) {government};
\node[Square] at (-1.150000, 11.300000) {};
\node[Kanji] at (-1.150000, 11.800000) {頂};
\node[Onyomi] at (-1.100000, 11.400000) {チョウ};
\node[Kunyomi] at (-1.200000, 11.400000) {いただき};
\node[Meaning] at (-1.150000, 13.050000) {summit};
\node[Square] at (0.900000, 11.300000) {};
\node[Kanji] at (0.900000, 11.800000) {令};
\node[Onyomi] at (0.950000, 11.400000) {レイ};
\node[Meaning] at (0.900000, 13.050000) {orders};
\node[Square] at (2.950000, 11.300000) {};
\node[Kanji] at (2.950000, 11.800000) {収};
\node[Onyomi] at (3.000000, 11.400000) {シュウ};
\node[Kunyomi] at (2.900000, 11.400000) {おさ.める};
\node[Meaning] at (2.950000, 13.050000) {obtain};
\node[Square] at (5.000000, 11.300000) {};
\node[Kanji] at (5.000000, 11.800000) {具};
\node[Onyomi] at (5.050000, 11.400000) {グ};
\node[Meaning] at (5.000000, 13.050000) {tool};
\node[Square] at (7.050000, 11.300000) {};
\node[Kanji] at (7.050000, 11.800000) {困};
\node[Onyomi] at (7.100000, 11.400000) {コン};
\node[Kunyomi] at (7.000000, 11.400000) {こま};
\node[Meaning] at (7.050000, 13.050000) {distressed};
\node[Square] at (9.100000, 11.300000) {};
\node[Kanji] at (9.100000, 11.800000) {衝};
\node[Onyomi] at (9.150000, 11.400000) {ショウ};
\node[Kunyomi] at (9.050000, 11.400000) {つ.く};
\node[Meaning] at (9.100000, 13.050000) {collide};
\node[Square] at (11.150000, 11.300000) {};
\node[Kanji] at (11.150000, 11.800000) {権};
\node[Onyomi] at (11.200000, 11.400000) {ケン};
\node[Meaning] at (11.150000, 13.050000) {rights};
\node[Square] at (13.200000, 11.300000) {};
\node[Kanji] at (13.200000, 11.800000) {型};
\node[Onyomi] at (13.250000, 11.400000) {ケイ};
\node[Kunyomi] at (13.150000, 11.400000) {かた};
\node[Meaning] at (13.200000, 13.050000) {model};
\node[Square] at (15.250000, 11.300000) {};
\node[Kanji] at (15.250000, 11.800000) {敵};
\node[Onyomi] at (15.300000, 11.400000) {テキ};
\node[Kunyomi] at (15.200000, 11.400000) {かな.う};
\node[Meaning] at (15.250000, 13.050000) {enemy};
\node[Square] at (17.300000, 11.300000) {};
\node[Kanji] at (17.300000, 11.800000) {騎};
\node[Onyomi] at (17.350000, 11.400000) {キ};
\node[Meaning] at (17.300000, 13.050000) {horse};
\node[Square] at (19.350000, 11.300000) {};
\node[Kanji] at (19.350000, 11.800000) {暴};
\node[Onyomi] at (19.400000, 11.400000) {ボウ};
\node[Kunyomi] at (19.300000, 11.400000) {あば.れる};
\node[Meaning] at (19.350000, 13.050000) {violence};
\node[Square] at (21.400000, 11.300000) {};
\node[Kanji] at (21.400000, 11.800000) {臭};
\node[Onyomi] at (21.450000, 11.400000) {シュウ};
\node[Kunyomi] at (21.350000, 11.400000) {くさ};
\node[Meaning] at (21.400000, 13.050000) {stinking};
\node[Square] at (23.450000, 11.300000) {};
\node[Kanji] at (23.450000, 11.800000) {委};
\node[Onyomi] at (23.500000, 11.400000) {イ};
\node[Meaning] at (23.450000, 13.050000) {committee};
\node[Square] at (25.500000, 11.300000) {};
\node[Kanji] at (25.500000, 11.800000) {締};
\node[Onyomi] at (25.550000, 11.400000) {テイ};
\node[Kunyomi] at (25.450000, 11.400000) {し};
\node[Meaning] at (25.500000, 13.050000) {tighten};
\node[Square] at (27.550000, 11.300000) {};
\node[Kanji] at (27.550000, 11.800000) {藤};
\node[Onyomi] at (27.600000, 11.400000) {トウ};
\node[Kunyomi] at (27.500000, 11.400000) {ふじ};
\node[Meaning] at (27.550000, 13.050000) {wisteria};
\node[Square] at (29.600000, 11.300000) {};
\node[Kanji] at (29.600000, 11.800000) {凍};
\node[Onyomi] at (29.650000, 11.400000) {トウ};
\node[Kunyomi] at (29.550000, 11.400000) {こお.る};
\node[Meaning] at (29.600000, 13.050000) {frozen};
\node[Square] at (31.650000, 11.300000) {};
\node[Kanji] at (31.650000, 11.800000) {捨};
\node[Onyomi] at (31.700000, 11.400000) {シャ};
\node[Kunyomi] at (31.600000, 11.400000) {す};
\node[Meaning] at (31.650000, 13.050000) {throw away};
\node[Square] at (33.700000, 11.300000) {};
\node[Kanji] at (33.700000, 11.800000) {総};
\node[Onyomi] at (33.750000, 11.400000) {ソウ};
\node[Meaning] at (33.700000, 13.050000) {whole};
\node[Square] at (35.750000, 11.300000) {};
\node[Kanji] at (35.750000, 11.800000) {規};
\node[Onyomi] at (35.800000, 11.400000) {キ};
\node[Meaning] at (35.750000, 13.050000) {standard};
\node[Square] at (37.800000, 11.300000) {};
\node[Kanji] at (37.800000, 11.800000) {庫};
\node[Onyomi] at (37.850000, 11.400000) {コ};
\node[Kunyomi] at (37.750000, 11.400000) {くら};
\node[Meaning] at (37.800000, 13.050000) {storage};
\node[Square] at (39.850000, 11.300000) {};
\node[Kanji] at (39.850000, 11.800000) {検};
\node[Onyomi] at (39.900000, 11.400000) {ケン};
\node[Meaning] at (39.850000, 13.050000) {examine};
\node[Square] at (41.900000, 11.300000) {};
\node[Kanji] at (41.900000, 11.800000) {黄};
\node[Onyomi] at (41.950000, 11.400000) {オウ};
\node[Kunyomi] at (41.850000, 11.400000) {き};
\node[Meaning] at (41.900000, 13.050000) {yellow};
\node[Square] at (43.950000, 11.300000) {};
\node[Kanji] at (43.950000, 11.800000) {済};
\node[Onyomi] at (44.000000, 11.400000) {サイ};
\node[Kunyomi] at (43.900000, 11.400000) {す.ます};
\node[Meaning] at (43.950000, 13.050000) {come to an end};
\node[Square] at (46.000000, 11.300000) {};
\node[Kanji] at (46.000000, 11.800000) {幹};
\node[Onyomi] at (46.050000, 11.400000) {カン};
\node[Kunyomi] at (45.950000, 11.400000) {みき};
\node[Meaning] at (46.000000, 13.050000) {tree trunk};
\node[Square] at (48.050000, 11.300000) {};
\node[Kanji] at (48.050000, 11.800000) {荒};
\node[Onyomi] at (48.100000, 11.400000) {コウ};
\node[Kunyomi] at (48.000000, 11.400000) {あ};
\node[Meaning] at (48.050000, 13.050000) {wild};
\node[Square] at (50.100000, 11.300000) {};
\node[Kanji] at (50.100000, 11.800000) {塔};
\node[Onyomi] at (50.150000, 11.400000) {トウ};
\node[Meaning] at (50.100000, 13.050000) {tower};
\node[Square] at (52.150000, 11.300000) {};
\node[Kanji] at (52.150000, 11.800000) {千};
\node[Onyomi] at (52.200000, 11.400000) {セン};
\node[Kunyomi] at (52.100000, 11.400000) {ち};
\node[Meaning] at (52.150000, 13.050000) {thousand};
\node[Square] at (54.200000, 11.300000) {};
\node[Kanji] at (54.200000, 11.800000) {鹿};
\node[Onyomi] at (54.250000, 11.400000) {ロク};
\node[Kunyomi] at (54.150000, 11.400000) {か};
\node[Meaning] at (54.200000, 13.050000) {deer};
\node[Square] at (56.250000, 11.300000) {};
\node[Kanji] at (56.250000, 11.800000) {輸};
\node[Onyomi] at (56.300000, 11.400000) {ユ};
\node[Meaning] at (56.250000, 13.050000) {transport};
\node[Meaning] at (-58.500000, 11.850000) {90.58\%};
\node[Square] at (-56.500000, 9.250000) {};
\node[Kanji] at (-56.500000, 9.750000) {唱};
\node[Onyomi] at (-56.450000, 9.350000) {ショウ};
\node[Kunyomi] at (-56.550000, 9.350000) {とな.える};
\node[Meaning] at (-56.500000, 11.000000) {chant};
\node[Square] at (-54.450000, 9.250000) {};
\node[Kanji] at (-54.450000, 9.750000) {罪};
\node[Onyomi] at (-54.400000, 9.350000) {ザイ};
\node[Kunyomi] at (-54.500000, 9.350000) {つみ};
\node[Meaning] at (-54.450000, 11.000000) {guilt};
\node[Square] at (-52.400000, 9.250000) {};
\node[Kanji] at (-52.400000, 9.750000) {岡};
\node[Kunyomi] at (-52.450000, 9.350000) {おか};
\node[Meaning] at (-52.400000, 11.000000) {hill};
\node[Square] at (-50.350000, 9.250000) {};
\node[Kanji] at (-50.350000, 9.750000) {援};
\node[Onyomi] at (-50.300000, 9.350000) {エン};
\node[Meaning] at (-50.350000, 11.000000) {aid};
\node[Square] at (-48.300000, 9.250000) {};
\node[Kanji] at (-48.300000, 9.750000) {帝};
\node[Onyomi] at (-48.250000, 9.350000) {テイ};
\node[Kunyomi] at (-48.350000, 9.350000) {みかど};
\node[Meaning] at (-48.300000, 11.000000) {sovereign};
\node[Square] at (-46.250000, 9.250000) {};
\node[Kanji] at (-46.250000, 9.750000) {酒};
\node[Onyomi] at (-46.200000, 9.350000) {シュ};
\node[Kunyomi] at (-46.300000, 9.350000) {さけ};
\node[Meaning] at (-46.250000, 11.000000) {alcohol};
\node[Square] at (-44.200000, 9.250000) {};
\node[Kanji] at (-44.200000, 9.750000) {布};
\node[Onyomi] at (-44.150000, 9.350000) {フ};
\node[Kunyomi] at (-44.250000, 9.350000) {ぬの};
\node[Meaning] at (-44.200000, 11.000000) {cloth};
\node[Square] at (-42.150000, 9.250000) {};
\node[Kanji] at (-42.150000, 9.750000) {管};
\node[Onyomi] at (-42.100000, 9.350000) {カン};
\node[Kunyomi] at (-42.200000, 9.350000) {くだ};
\node[Meaning] at (-42.150000, 11.000000) {pipe};
\node[Square] at (-40.100000, 9.250000) {};
\node[Kanji] at (-40.100000, 9.750000) {景};
\node[Onyomi] at (-40.050000, 9.350000) {ケイ};
\node[Meaning] at (-40.100000, 11.000000) {scene};
\node[Square] at (-38.050000, 9.250000) {};
\node[Kanji] at (-38.050000, 9.750000) {忍};
\node[Onyomi] at (-38.000000, 9.350000) {ニン};
\node[Kunyomi] at (-38.100000, 9.350000) {しの.ぶ};
\node[Meaning] at (-38.050000, 11.000000) {endure};
\node[Square] at (-36.000000, 9.250000) {};
\node[Kanji] at (-36.000000, 9.750000) {恋};
\node[Onyomi] at (-35.950000, 9.350000) {レン};
\node[Kunyomi] at (-36.050000, 9.350000) {こい};
\node[Meaning] at (-36.000000, 11.000000) {romance};
\node[Square] at (-33.950000, 9.250000) {};
\node[Kanji] at (-33.950000, 9.750000) {祈};
\node[Onyomi] at (-33.900000, 9.350000) {キ};
\node[Kunyomi] at (-34.000000, 9.350000) {いの.る};
\node[Meaning] at (-33.950000, 11.000000) {pray};
\node[Square] at (-31.900000, 9.250000) {};
\node[Kanji] at (-31.900000, 9.750000) {武};
\node[Onyomi] at (-31.850000, 9.350000) {ブ};
\node[Kunyomi] at (-31.950000, 9.350000) {たけ};
\node[Meaning] at (-31.900000, 11.000000) {military};
\node[Square] at (-29.850000, 9.250000) {};
\node[Kanji] at (-29.850000, 9.750000) {独};
\node[Onyomi] at (-29.800000, 9.350000) {ドク};
\node[Kunyomi] at (-29.900000, 9.350000) {ひと.り};
\node[Meaning] at (-29.850000, 11.000000) {alone};
\node[Square] at (-27.800000, 9.250000) {};
\node[Kanji] at (-27.800000, 9.750000) {筋};
\node[Onyomi] at (-27.750000, 9.350000) {キン};
\node[Kunyomi] at (-27.850000, 9.350000) {すじ};
\node[Meaning] at (-27.800000, 11.000000) {muscle};
\node[Square] at (-25.750000, 9.250000) {};
\node[Kanji] at (-25.750000, 9.750000) {陸};
\node[Onyomi] at (-25.700000, 9.350000) {リク};
\node[Meaning] at (-25.750000, 11.000000) {land};
\node[Square] at (-23.700000, 9.250000) {};
\node[Kanji] at (-23.700000, 9.750000) {携};
\node[Onyomi] at (-23.650000, 9.350000) {ケイ};
\node[Kunyomi] at (-23.750000, 9.350000) {たずさ.わる};
\node[Meaning] at (-23.700000, 11.000000) {portable};
\node[Square] at (-21.650000, 9.250000) {};
\node[Kanji] at (-21.650000, 9.750000) {位};
\node[Onyomi] at (-21.600000, 9.350000) {イ};
\node[Kunyomi] at (-21.700000, 9.350000) {くらい};
\node[Meaning] at (-21.650000, 11.000000) {rank};
\node[Square] at (-19.600000, 9.250000) {};
\node[Kanji] at (-19.600000, 9.750000) {犯};
\node[Onyomi] at (-19.550000, 9.350000) {ハン};
\node[Kunyomi] at (-19.650000, 9.350000) {おか.す};
\node[Meaning] at (-19.600000, 11.000000) {crime};
\node[Square] at (-17.550000, 9.250000) {};
\node[Kanji] at (-17.550000, 9.750000) {境};
\node[Onyomi] at (-17.500000, 9.350000) {キョウ};
\node[Kunyomi] at (-17.600000, 9.350000) {さかい};
\node[Meaning] at (-17.550000, 11.000000) {boundary};
\node[Square] at (-15.500000, 9.250000) {};
\node[Kanji] at (-15.500000, 9.750000) {春};
\node[Onyomi] at (-15.450000, 9.350000) {シュン};
\node[Kunyomi] at (-15.550000, 9.350000) {はる};
\node[Meaning] at (-15.500000, 11.000000) {spring};
\node[Square] at (-13.450000, 9.250000) {};
\node[Kanji] at (-13.450000, 9.750000) {史};
\node[Onyomi] at (-13.400000, 9.350000) {シ};
\node[Meaning] at (-13.450000, 11.000000) {history};
\node[Square] at (-11.400000, 9.250000) {};
\node[Kanji] at (-11.400000, 9.750000) {積};
\node[Onyomi] at (-11.350000, 9.350000) {セキ};
\node[Kunyomi] at (-11.450000, 9.350000) {つ.む};
\node[Meaning] at (-11.400000, 11.000000) {accumulate};
\node[Square] at (-9.350000, 9.250000) {};
\node[Kanji] at (-9.350000, 9.750000) {雑};
\node[Onyomi] at (-9.300000, 9.350000) {ザツ};
\node[Meaning] at (-9.350000, 11.000000) {random};
\node[Square] at (-7.300000, 9.250000) {};
\node[Kanji] at (-7.300000, 9.750000) {皿};
\node[Kunyomi] at (-7.350000, 9.350000) {さら};
\node[Meaning] at (-7.300000, 11.000000) {plate};
\node[Square] at (-5.250000, 9.250000) {};
\node[Kanji] at (-5.250000, 9.750000) {弟};
\node[Onyomi] at (-5.200000, 9.350000) {ダイ};
\node[Kunyomi] at (-5.300000, 9.350000) {おとうと};
\node[Meaning] at (-5.250000, 11.000000) {little brother};
\node[Square] at (-3.200000, 9.250000) {};
\node[Kanji] at (-3.200000, 9.750000) {魚};
\node[Onyomi] at (-3.150000, 9.350000) {ギョ};
\node[Kunyomi] at (-3.250000, 9.350000) {さかな};
\node[Meaning] at (-3.200000, 11.000000) {fish};
\node[Square] at (-1.150000, 9.250000) {};
\node[Kanji] at (-1.150000, 9.750000) {婦};
\node[Onyomi] at (-1.100000, 9.350000) {フ};
\node[Meaning] at (-1.150000, 11.000000) {wife};
\node[Square] at (0.900000, 9.250000) {};
\node[Kanji] at (0.900000, 9.750000) {韓};
\node[Onyomi] at (0.950000, 9.350000) {カン};
\node[Meaning] at (0.900000, 11.000000) {korea};
\node[Square] at (2.950000, 9.250000) {};
\node[Kanji] at (2.950000, 9.750000) {墓};
\node[Onyomi] at (3.000000, 9.350000) {ボ};
\node[Kunyomi] at (2.900000, 9.350000) {はか};
\node[Meaning] at (2.950000, 11.000000) {grave};
\node[Square] at (5.000000, 9.250000) {};
\node[Kanji] at (5.000000, 9.750000) {岸};
\node[Onyomi] at (5.050000, 9.350000) {ガン};
\node[Kunyomi] at (4.950000, 9.350000) {きし};
\node[Meaning] at (5.000000, 11.000000) {coast};
\node[Square] at (7.050000, 9.250000) {};
\node[Kanji] at (7.050000, 9.750000) {松};
\node[Onyomi] at (7.100000, 9.350000) {ショウ};
\node[Kunyomi] at (7.000000, 9.350000) {まつ};
\node[Meaning] at (7.050000, 11.000000) {pine};
\node[Square] at (9.100000, 9.250000) {};
\node[Kanji] at (9.100000, 9.750000) {辞};
\node[Onyomi] at (9.150000, 9.350000) {ジ};
\node[Kunyomi] at (9.050000, 9.350000) {や.める};
\node[Meaning] at (9.100000, 11.000000) {quit};
\node[Square] at (11.150000, 9.250000) {};
\node[Kanji] at (11.150000, 9.750000) {節};
\node[Onyomi] at (11.200000, 9.350000) {セツ};
\node[Kunyomi] at (11.100000, 9.350000) {ふし};
\node[Meaning] at (11.150000, 11.000000) {season};
\node[Square] at (13.200000, 9.250000) {};
\node[Kanji] at (13.200000, 9.750000) {貨};
\node[Onyomi] at (13.250000, 9.350000) {カ};
\node[Meaning] at (13.200000, 11.000000) {freight};
\node[Square] at (15.250000, 9.250000) {};
\node[Kanji] at (15.250000, 9.750000) {踊};
\node[Onyomi] at (15.300000, 9.350000) {ヨウ};
\node[Kunyomi] at (15.200000, 9.350000) {おど};
\node[Meaning] at (15.250000, 11.000000) {dance};
\node[Square] at (17.300000, 9.250000) {};
\node[Kanji] at (17.300000, 9.750000) {装};
\node[Onyomi] at (17.350000, 9.350000) {ソウ};
\node[Kunyomi] at (17.250000, 9.350000) {よそお.う};
\node[Meaning] at (17.300000, 11.000000) {attire};
\node[Square] at (19.350000, 9.250000) {};
\node[Kanji] at (19.350000, 9.750000) {泊};
\node[Onyomi] at (19.400000, 9.350000) {ハク};
\node[Kunyomi] at (19.300000, 9.350000) {と.まる};
\node[Meaning] at (19.350000, 11.000000) {overnight};
\node[Square] at (21.400000, 9.250000) {};
\node[Kanji] at (21.400000, 9.750000) {師};
\node[Onyomi] at (21.450000, 9.350000) {シ};
\node[Meaning] at (21.400000, 11.000000) {teacher};
\node[Square] at (23.450000, 9.250000) {};
\node[Kanji] at (23.450000, 9.750000) {拍};
\node[Onyomi] at (23.500000, 9.350000) {ハク};
\node[Meaning] at (23.450000, 11.000000) {beat};
\node[Square] at (25.500000, 9.250000) {};
\node[Kanji] at (25.500000, 9.750000) {課};
\node[Onyomi] at (25.550000, 9.350000) {カ};
\node[Meaning] at (25.500000, 11.000000) {section};
\node[Square] at (27.550000, 9.250000) {};
\node[Kanji] at (27.550000, 9.750000) {隊};
\node[Onyomi] at (27.600000, 9.350000) {タイ};
\node[Meaning] at (27.550000, 11.000000) {squad};
\node[Square] at (29.600000, 9.250000) {};
\node[Kanji] at (29.600000, 9.750000) {油};
\node[Onyomi] at (29.650000, 9.350000) {ユ};
\node[Kunyomi] at (29.550000, 9.350000) {あぶら};
\node[Meaning] at (29.600000, 11.000000) {oil};
\node[Square] at (31.650000, 9.250000) {};
\node[Kanji] at (31.650000, 9.750000) {処};
\node[Onyomi] at (31.700000, 9.350000) {ショ};
\node[Meaning] at (31.650000, 11.000000) {deal with};
\node[Square] at (33.700000, 9.250000) {};
\node[Kanji] at (33.700000, 9.750000) {緊};
\node[Onyomi] at (33.750000, 9.350000) {キン};
\node[Meaning] at (33.700000, 11.000000) {tense};
\node[Square] at (35.750000, 9.250000) {};
\node[Kanji] at (35.750000, 9.750000) {訳};
\node[Onyomi] at (35.800000, 9.350000) {ヤク};
\node[Kunyomi] at (35.700000, 9.350000) {わけ};
\node[Meaning] at (35.750000, 11.000000) {translation};
\node[Square] at (37.800000, 9.250000) {};
\node[Kanji] at (37.800000, 9.750000) {坊};
\node[Onyomi] at (37.850000, 9.350000) {ボウ};
\node[Meaning] at (37.800000, 11.000000) {monk};
\node[Square] at (39.850000, 9.250000) {};
\node[Kanji] at (39.850000, 9.750000) {量};
\node[Onyomi] at (39.900000, 9.350000) {リョウ};
\node[Kunyomi] at (39.800000, 9.350000) {はか.る};
\node[Meaning] at (39.850000, 11.000000) {quantity};
\node[Square] at (41.900000, 9.250000) {};
\node[Kanji] at (41.900000, 9.750000) {漏};
\node[Onyomi] at (41.950000, 9.350000) {ロウ};
\node[Kunyomi] at (41.850000, 9.350000) {も.らす};
\node[Meaning] at (41.900000, 11.000000) {leak};
\node[Square] at (43.950000, 9.250000) {};
\node[Kanji] at (43.950000, 9.750000) {房};
\node[Onyomi] at (44.000000, 9.350000) {ボウ};
\node[Kunyomi] at (43.900000, 9.350000) {ふさ};
\node[Meaning] at (43.950000, 11.000000) {cluster};
\node[Square] at (46.000000, 9.250000) {};
\node[Kanji] at (46.000000, 9.750000) {裂};
\node[Onyomi] at (46.050000, 9.350000) {レツ};
\node[Kunyomi] at (45.950000, 9.350000) {さ.*};
\node[Meaning] at (46.000000, 11.000000) {split};
\node[Square] at (48.050000, 9.250000) {};
\node[Kanji] at (48.050000, 9.750000) {秒};
\node[Onyomi] at (48.100000, 9.350000) {ビョウ};
\node[Meaning] at (48.050000, 11.000000) {second};
\node[Square] at (50.100000, 9.250000) {};
\node[Kanji] at (50.100000, 9.750000) {希};
\node[Onyomi] at (50.150000, 9.350000) {キ};
\node[Kunyomi] at (50.050000, 9.350000) {まれ};
\node[Meaning] at (50.100000, 11.000000) {wish};
\node[Square] at (52.150000, 9.250000) {};
\node[Kanji] at (52.150000, 9.750000) {誘};
\node[Onyomi] at (52.200000, 9.350000) {ユウ};
\node[Kunyomi] at (52.100000, 9.350000) {さそ.う};
\node[Meaning] at (52.150000, 11.000000) {invite};
\node[Square] at (54.200000, 9.250000) {};
\node[Kanji] at (54.200000, 9.750000) {喉};
\node[Onyomi] at (54.250000, 9.350000) {コウ};
\node[Kunyomi] at (54.150000, 9.350000) {のど};
\node[Meaning] at (54.200000, 11.000000) {throat};
\node[Square] at (56.250000, 9.250000) {};
\node[Kanji] at (56.250000, 9.750000) {儀};
\node[Onyomi] at (56.300000, 9.350000) {ギ};
\node[Meaning] at (56.250000, 11.000000) {ceremony};
\node[Meaning] at (-58.500000, 9.800000) {91.65\%};
\node[Square] at (-56.500000, 7.200000) {};
\node[Kanji] at (-56.500000, 7.700000) {祭};
\node[Onyomi] at (-56.450000, 7.300000) {サイ};
\node[Kunyomi] at (-56.550000, 7.300000) {まつり};
\node[Meaning] at (-56.500000, 8.950000) {festival};
\node[Square] at (-54.450000, 7.200000) {};
\node[Kanji] at (-54.450000, 7.700000) {橋};
\node[Onyomi] at (-54.400000, 7.300000) {キョウ};
\node[Kunyomi] at (-54.500000, 7.300000) {はし};
\node[Meaning] at (-54.450000, 8.950000) {bridge};
\node[Square] at (-52.400000, 7.200000) {};
\node[Kanji] at (-52.400000, 7.700000) {策};
\node[Onyomi] at (-52.350000, 7.300000) {サク};
\node[Kunyomi] at (-52.450000, 7.300000) {さく};
\node[Meaning] at (-52.400000, 8.950000) {plan};
\node[Square] at (-50.350000, 7.200000) {};
\node[Kanji] at (-50.350000, 7.700000) {造};
\node[Onyomi] at (-50.300000, 7.300000) {ゾウ};
\node[Kunyomi] at (-50.400000, 7.300000) {つく.る};
\node[Meaning] at (-50.350000, 8.950000) {create};
\node[Square] at (-48.300000, 7.200000) {};
\node[Kanji] at (-48.300000, 7.700000) {挨};
\node[Onyomi] at (-48.250000, 7.300000) {アイ};
\node[Meaning] at (-48.300000, 8.950000) {push open};
\node[Square] at (-46.250000, 7.200000) {};
\node[Kanji] at (-46.250000, 7.700000) {拶};
\node[Onyomi] at (-46.200000, 7.300000) {サツ};
\node[Meaning] at (-46.250000, 8.950000) {be imminent};
\node[Square] at (-44.200000, 7.200000) {};
\node[Kanji] at (-44.200000, 7.700000) {基};
\node[Onyomi] at (-44.150000, 7.300000) {キ};
\node[Kunyomi] at (-44.250000, 7.300000) {もと};
\node[Meaning] at (-44.200000, 8.950000) {foundation};
\node[Square] at (-42.150000, 7.200000) {};
\node[Kanji] at (-42.150000, 7.700000) {埋};
\node[Onyomi] at (-42.100000, 7.300000) {マイ};
\node[Kunyomi] at (-42.200000, 7.300000) {う};
\node[Meaning] at (-42.150000, 8.950000) {bury};
\node[Square] at (-40.100000, 7.200000) {};
\node[Kanji] at (-40.100000, 7.700000) {辛};
\node[Onyomi] at (-40.050000, 7.300000) {シン};
\node[Kunyomi] at (-40.150000, 7.300000) {から.い};
\node[Meaning] at (-40.100000, 8.950000) {spicy};
\node[Square] at (-38.050000, 7.200000) {};
\node[Kanji] at (-38.050000, 7.700000) {邪};
\node[Onyomi] at (-38.000000, 7.300000) {ジャ};
\node[Kunyomi] at (-38.100000, 7.300000) {よこし.ま};
\node[Meaning] at (-38.050000, 8.950000) {wicked};
\node[Square] at (-36.000000, 7.200000) {};
\node[Kanji] at (-36.000000, 7.700000) {替};
\node[Onyomi] at (-35.950000, 7.300000) {タイ};
\node[Kunyomi] at (-36.050000, 7.300000) {か};
\node[Meaning] at (-36.000000, 8.950000) {replace};
\node[Square] at (-33.950000, 7.200000) {};
\node[Kanji] at (-33.950000, 7.700000) {灰};
\node[Onyomi] at (-33.900000, 7.300000) {カイ};
\node[Kunyomi] at (-34.000000, 7.300000) {はい};
\node[Meaning] at (-33.950000, 8.950000) {ashes};
\node[Square] at (-31.900000, 7.200000) {};
\node[Kanji] at (-31.900000, 7.700000) {紅};
\node[Onyomi] at (-31.850000, 7.300000) {コウ};
\node[Kunyomi] at (-31.950000, 7.300000) {べに};
\node[Meaning] at (-31.900000, 8.950000) {deep red};
\node[Square] at (-29.850000, 7.200000) {};
\node[Kanji] at (-29.850000, 7.700000) {歓};
\node[Onyomi] at (-29.800000, 7.300000) {カン};
\node[Meaning] at (-29.850000, 8.950000) {delight};
\node[Square] at (-27.800000, 7.200000) {};
\node[Kanji] at (-27.800000, 7.700000) {崩};
\node[Onyomi] at (-27.750000, 7.300000) {ホウ};
\node[Kunyomi] at (-27.850000, 7.300000) {くず.*};
\node[Meaning] at (-27.800000, 8.950000) {crumble};
\node[Square] at (-25.750000, 7.200000) {};
\node[Kanji] at (-25.750000, 7.700000) {演};
\node[Onyomi] at (-25.700000, 7.300000) {エン};
\node[Meaning] at (-25.750000, 8.950000) {perform};
\node[Square] at (-23.700000, 7.200000) {};
\node[Kanji] at (-23.700000, 7.700000) {環};
\node[Onyomi] at (-23.650000, 7.300000) {カン};
\node[Meaning] at (-23.700000, 8.950000) {loop};
\node[Square] at (-21.650000, 7.200000) {};
\node[Kanji] at (-21.650000, 7.700000) {責};
\node[Onyomi] at (-21.600000, 7.300000) {セキ};
\node[Kunyomi] at (-21.700000, 7.300000) {せ.める};
\node[Meaning] at (-21.650000, 8.950000) {blame};
\node[Square] at (-19.600000, 7.200000) {};
\node[Kanji] at (-19.600000, 7.700000) {砂};
\node[Onyomi] at (-19.550000, 7.300000) {サ};
\node[Kunyomi] at (-19.650000, 7.300000) {すな};
\node[Meaning] at (-19.600000, 8.950000) {sand};
\node[Square] at (-17.550000, 7.200000) {};
\node[Kanji] at (-17.550000, 7.700000) {欠};
\node[Onyomi] at (-17.500000, 7.300000) {ケツ};
\node[Kunyomi] at (-17.600000, 7.300000) {か.ける};
\node[Meaning] at (-17.550000, 8.950000) {lack};
\node[Square] at (-15.500000, 7.200000) {};
\node[Kanji] at (-15.500000, 7.700000) {接};
\node[Onyomi] at (-15.450000, 7.300000) {セツ};
\node[Kunyomi] at (-15.550000, 7.300000) {つ.ぐ};
\node[Meaning] at (-15.500000, 8.950000) {adjoin};
\node[Square] at (-13.450000, 7.200000) {};
\node[Kanji] at (-13.450000, 7.700000) {占};
\node[Onyomi] at (-13.400000, 7.300000) {セン};
\node[Kunyomi] at (-13.500000, 7.300000) {うらな.い};
\node[Meaning] at (-13.450000, 8.950000) {fortune};
\node[Square] at (-11.400000, 7.200000) {};
\node[Kanji] at (-11.400000, 7.700000) {鋭};
\node[Onyomi] at (-11.350000, 7.300000) {エイ};
\node[Kunyomi] at (-11.450000, 7.300000) {するど.い};
\node[Meaning] at (-11.400000, 8.950000) {sharp};
\node[Square] at (-9.350000, 7.200000) {};
\node[Kanji] at (-9.350000, 7.700000) {芝};
\node[Onyomi] at (-9.300000, 7.300000) {シ};
\node[Kunyomi] at (-9.400000, 7.300000) {しば};
\node[Meaning] at (-9.350000, 8.950000) {lawn};
\node[Square] at (-7.300000, 7.200000) {};
\node[Kanji] at (-7.300000, 7.700000) {丁};
\node[Onyomi] at (-7.250000, 7.300000) {チョウ};
\node[Meaning] at (-7.300000, 8.950000) {street};
\node[Square] at (-5.250000, 7.200000) {};
\node[Kanji] at (-5.250000, 7.700000) {暑};
\node[Onyomi] at (-5.200000, 7.300000) {ショ};
\node[Kunyomi] at (-5.300000, 7.300000) {あつ};
\node[Meaning] at (-5.250000, 8.950000) {hot};
\node[Square] at (-3.200000, 7.200000) {};
\node[Kanji] at (-3.200000, 7.700000) {晴};
\node[Onyomi] at (-3.150000, 7.300000) {セイ};
\node[Kunyomi] at (-3.250000, 7.300000) {は};
\node[Meaning] at (-3.200000, 8.950000) {clear up};
\node[Square] at (-1.150000, 7.200000) {};
\node[Kanji] at (-1.150000, 7.700000) {慣};
\node[Onyomi] at (-1.100000, 7.300000) {カン};
\node[Kunyomi] at (-1.200000, 7.300000) {な.れる};
\node[Meaning] at (-1.150000, 8.950000) {accustomed};
\node[Square] at (0.900000, 7.200000) {};
\node[Kanji] at (0.900000, 7.700000) {狙};
\node[Onyomi] at (0.950000, 7.300000) {ソ};
\node[Kunyomi] at (0.850000, 7.300000) {ねら.い};
\node[Meaning] at (0.900000, 8.950000) {aim};
\node[Square] at (2.950000, 7.200000) {};
\node[Kanji] at (2.950000, 7.700000) {虫};
\node[Onyomi] at (3.000000, 7.300000) {チュウ};
\node[Kunyomi] at (2.900000, 7.300000) {むし};
\node[Meaning] at (2.950000, 8.950000) {insect};
\node[Square] at (5.000000, 7.200000) {};
\node[Kanji] at (5.000000, 7.700000) {脳};
\node[Onyomi] at (5.050000, 7.300000) {ノウ};
\node[Meaning] at (5.000000, 8.950000) {brain};
\node[Square] at (7.050000, 7.200000) {};
\node[Kanji] at (7.050000, 7.700000) {飼};
\node[Onyomi] at (7.100000, 7.300000) {シ};
\node[Kunyomi] at (7.000000, 7.300000) {か};
\node[Meaning] at (7.050000, 8.950000) {domesticate};
\node[Square] at (9.100000, 7.200000) {};
\node[Kanji] at (9.100000, 7.700000) {眉};
\node[Onyomi] at (9.150000, 7.300000) {ビ};
\node[Kunyomi] at (9.050000, 7.300000) {まゆ};
\node[Meaning] at (9.100000, 8.950000) {eyebrow};
\node[Square] at (11.150000, 7.200000) {};
\node[Kanji] at (11.150000, 7.700000) {冬};
\node[Onyomi] at (11.200000, 7.300000) {トウ};
\node[Kunyomi] at (11.100000, 7.300000) {ふゆ};
\node[Meaning] at (11.150000, 8.950000) {winter};
\node[Square] at (13.200000, 7.200000) {};
\node[Kanji] at (13.200000, 7.700000) {寒};
\node[Onyomi] at (13.250000, 7.300000) {カン};
\node[Kunyomi] at (13.150000, 7.300000) {さむ};
\node[Meaning] at (13.200000, 8.950000) {cold};
\node[Square] at (15.250000, 7.200000) {};
\node[Kanji] at (15.250000, 7.700000) {飯};
\node[Onyomi] at (15.300000, 7.300000) {ハン};
\node[Kunyomi] at (15.200000, 7.300000) {めし};
\node[Meaning] at (15.250000, 8.950000) {meal};
\node[Square] at (17.300000, 7.200000) {};
\node[Kanji] at (17.300000, 7.700000) {余};
\node[Onyomi] at (17.350000, 7.300000) {ヨ};
\node[Kunyomi] at (17.250000, 7.300000) {あま.る};
\node[Meaning] at (17.300000, 8.950000) {surplus};
\node[Square] at (19.350000, 7.200000) {};
\node[Kanji] at (19.350000, 7.700000) {植};
\node[Onyomi] at (19.400000, 7.300000) {ショク};
\node[Kunyomi] at (19.300000, 7.300000) {う.*};
\node[Meaning] at (19.350000, 8.950000) {plant};
\node[Square] at (21.400000, 7.200000) {};
\node[Kanji] at (21.400000, 7.700000) {裁};
\node[Onyomi] at (21.450000, 7.300000) {サイ};
\node[Kunyomi] at (21.350000, 7.300000) {さば.く};
\node[Meaning] at (21.400000, 8.950000) {judge};
\node[Square] at (23.450000, 7.200000) {};
\node[Kanji] at (23.450000, 7.700000) {祝};
\node[Onyomi] at (23.500000, 7.300000) {シュク};
\node[Kunyomi] at (23.400000, 7.300000) {いわ.う};
\node[Meaning] at (23.450000, 8.950000) {celebrate};
\node[Square] at (25.500000, 7.200000) {};
\node[Kanji] at (25.500000, 7.700000) {鮮};
\node[Onyomi] at (25.550000, 7.300000) {セン};
\node[Kunyomi] at (25.450000, 7.300000) {あざ.やか};
\node[Meaning] at (25.500000, 8.950000) {fresh};
\node[Square] at (27.550000, 7.200000) {};
\node[Kanji] at (27.550000, 7.700000) {阪};
\node[Onyomi] at (27.600000, 7.300000) {ハン};
\node[Kunyomi] at (27.500000, 7.300000) {さか};
\node[Meaning] at (27.550000, 8.950000) {heights};
\node[Square] at (29.600000, 7.200000) {};
\node[Kanji] at (29.600000, 7.700000) {快};
\node[Onyomi] at (29.650000, 7.300000) {カイ};
\node[Kunyomi] at (29.550000, 7.300000) {こころよ.い};
\node[Meaning] at (29.600000, 8.950000) {pleasant};
\node[Square] at (31.650000, 7.200000) {};
\node[Kanji] at (31.650000, 7.700000) {刊};
\node[Onyomi] at (31.700000, 7.300000) {カン};
\node[Meaning] at (31.650000, 8.950000) {edition};
\node[Square] at (33.700000, 7.200000) {};
\node[Kanji] at (33.700000, 7.700000) {致};
\node[Onyomi] at (33.750000, 7.300000) {チ};
\node[Kunyomi] at (33.650000, 7.300000) {いた.す};
\node[Meaning] at (33.700000, 8.950000) {do};
\node[Square] at (35.750000, 7.200000) {};
\node[Kanji] at (35.750000, 7.700000) {借};
\node[Onyomi] at (35.800000, 7.300000) {シャク};
\node[Kunyomi] at (35.700000, 7.300000) {か.りる};
\node[Meaning] at (35.750000, 8.950000) {borrow};
\node[Square] at (37.800000, 7.200000) {};
\node[Kanji] at (37.800000, 7.700000) {貴};
\node[Onyomi] at (37.850000, 7.300000) {キ};
\node[Kunyomi] at (37.750000, 7.300000) {とうと.い};
\node[Meaning] at (37.800000, 8.950000) {valuable};
\node[Square] at (39.850000, 7.200000) {};
\node[Kanji] at (39.850000, 7.700000) {柄};
\node[Onyomi] at (39.900000, 7.300000) {ヘイ};
\node[Kunyomi] at (39.800000, 7.300000) {がら};
\node[Meaning] at (39.850000, 8.950000) {pattern};
\node[Square] at (41.900000, 7.200000) {};
\node[Kanji] at (41.900000, 7.700000) {皆};
\node[Onyomi] at (41.950000, 7.300000) {カイ};
\node[Kunyomi] at (41.850000, 7.300000) {みな};
\node[Meaning] at (41.900000, 8.950000) {all};
\node[Square] at (43.950000, 7.200000) {};
\node[Kanji] at (43.950000, 7.700000) {訓};
\node[Onyomi] at (44.000000, 7.300000) {クン};
\node[Kunyomi] at (43.900000, 7.300000) {よ.む};
\node[Meaning] at (43.950000, 8.950000) {instruction};
\node[Square] at (46.000000, 7.200000) {};
\node[Kanji] at (46.000000, 7.700000) {障};
\node[Onyomi] at (46.050000, 7.300000) {ショウ};
\node[Kunyomi] at (45.950000, 7.300000) {さわ.る};
\node[Meaning] at (46.000000, 8.950000) {hinder};
\node[Square] at (48.050000, 7.200000) {};
\node[Kanji] at (48.050000, 7.700000) {商};
\node[Onyomi] at (48.100000, 7.300000) {ショウ};
\node[Kunyomi] at (48.000000, 7.300000) {あきな.い};
\node[Meaning] at (48.050000, 8.950000) {merchandise};
\node[Square] at (50.100000, 7.200000) {};
\node[Kanji] at (50.100000, 7.700000) {価};
\node[Onyomi] at (50.150000, 7.300000) {カ};
\node[Kunyomi] at (50.050000, 7.300000) {あたい};
\node[Meaning] at (50.100000, 8.950000) {value};
\node[Square] at (52.150000, 7.200000) {};
\node[Kanji] at (52.150000, 7.700000) {航};
\node[Onyomi] at (52.200000, 7.300000) {コウ};
\node[Meaning] at (52.150000, 8.950000) {navigation};
\node[Square] at (54.200000, 7.200000) {};
\node[Kanji] at (54.200000, 7.700000) {慢};
\node[Onyomi] at (54.250000, 7.300000) {マン};
\node[Meaning] at (54.200000, 8.950000) {ridicule};
\node[Square] at (56.250000, 7.200000) {};
\node[Kanji] at (56.250000, 7.700000) {辺};
\node[Onyomi] at (56.300000, 7.300000) {ヘン};
\node[Kunyomi] at (56.200000, 7.300000) {あた.り};
\node[Meaning] at (56.250000, 8.950000) {area};
\node[Meaning] at (-58.500000, 7.750000) {92.57\%};
\node[Square] at (-56.500000, 5.150000) {};
\node[Kanji] at (-56.500000, 5.650000) {算};
\node[Onyomi] at (-56.450000, 5.250000) {サン};
\node[Kunyomi] at (-56.550000, 5.250000) {そろ};
\node[Meaning] at (-56.500000, 6.900000) {calculate};
\node[Square] at (-54.450000, 5.150000) {};
\node[Kanji] at (-54.450000, 5.650000) {靴};
\node[Onyomi] at (-54.400000, 5.250000) {カ};
\node[Kunyomi] at (-54.500000, 5.250000) {くつ};
\node[Meaning] at (-54.450000, 6.900000) {shoes};
\node[Square] at (-52.400000, 5.150000) {};
\node[Kanji] at (-52.400000, 5.650000) {頬};
\node[Kunyomi] at (-52.450000, 5.250000) {ほお};
\node[Meaning] at (-52.400000, 6.900000) {cheek};
\node[Square] at (-50.350000, 5.150000) {};
\node[Kanji] at (-50.350000, 5.650000) {謝};
\node[Onyomi] at (-50.300000, 5.250000) {シャ};
\node[Kunyomi] at (-50.400000, 5.250000) {あやま.る};
\node[Meaning] at (-50.350000, 6.900000) {apologize};
\node[Square] at (-48.300000, 5.150000) {};
\node[Kanji] at (-48.300000, 5.650000) {冒};
\node[Onyomi] at (-48.250000, 5.250000) {ボウ};
\node[Kunyomi] at (-48.350000, 5.250000) {おか.す};
\node[Meaning] at (-48.300000, 6.900000) {dare};
\node[Square] at (-46.250000, 5.150000) {};
\node[Kanji] at (-46.250000, 5.650000) {尻};
\node[Kunyomi] at (-46.300000, 5.250000) {しり};
\node[Meaning] at (-46.250000, 6.900000) {butt};
\node[Square] at (-44.200000, 5.150000) {};
\node[Kanji] at (-44.200000, 5.650000) {律};
\node[Onyomi] at (-44.150000, 5.250000) {リツ};
\node[Meaning] at (-44.200000, 6.900000) {law};
\node[Square] at (-42.150000, 5.150000) {};
\node[Kanji] at (-42.150000, 5.650000) {補};
\node[Onyomi] at (-42.100000, 5.250000) {ホ};
\node[Kunyomi] at (-42.200000, 5.250000) {おぎな.う};
\node[Meaning] at (-42.150000, 6.900000) {supplement};
\node[Square] at (-40.100000, 5.150000) {};
\node[Kanji] at (-40.100000, 5.650000) {詳};
\node[Onyomi] at (-40.050000, 5.250000) {ショウ};
\node[Kunyomi] at (-40.150000, 5.250000) {くわ.しい};
\node[Meaning] at (-40.100000, 6.900000) {detailed};
\node[Square] at (-38.050000, 5.150000) {};
\node[Kanji] at (-38.050000, 5.650000) {誕};
\node[Onyomi] at (-38.000000, 5.250000) {タン};
\node[Meaning] at (-38.050000, 6.900000) {birth};
\node[Square] at (-36.000000, 5.150000) {};
\node[Kanji] at (-36.000000, 5.650000) {焦};
\node[Onyomi] at (-35.950000, 5.250000) {ショウ};
\node[Kunyomi] at (-36.050000, 5.250000) {こ.*};
\node[Meaning] at (-36.000000, 6.900000) {char};
\node[Square] at (-33.950000, 5.150000) {};
\node[Kanji] at (-33.950000, 5.650000) {紫};
\node[Onyomi] at (-33.900000, 5.250000) {シ};
\node[Kunyomi] at (-34.000000, 5.250000) {むらさき};
\node[Meaning] at (-33.950000, 6.900000) {purple};
\node[Square] at (-31.900000, 5.150000) {};
\node[Kanji] at (-31.900000, 5.650000) {費};
\node[Onyomi] at (-31.850000, 5.250000) {ヒ};
\node[Kunyomi] at (-31.950000, 5.250000) {つい.やす};
\node[Meaning] at (-31.900000, 6.900000) {expense};
\node[Square] at (-29.850000, 5.150000) {};
\node[Kanji] at (-29.850000, 5.650000) {唇};
\node[Onyomi] at (-29.800000, 5.250000) {シン};
\node[Kunyomi] at (-29.900000, 5.250000) {くちびる};
\node[Meaning] at (-29.850000, 6.900000) {lips};
\node[Square] at (-27.800000, 5.150000) {};
\node[Kanji] at (-27.800000, 5.650000) {拭};
\node[Onyomi] at (-27.750000, 5.250000) {ショク};
\node[Kunyomi] at (-27.850000, 5.250000) {ふ-く};
\node[Meaning] at (-27.800000, 6.900000) {wipe};
\node[Square] at (-25.750000, 5.150000) {};
\node[Kanji] at (-25.750000, 5.650000) {氷};
\node[Onyomi] at (-25.700000, 5.250000) {ヒョウ};
\node[Kunyomi] at (-25.800000, 5.250000) {こおり};
\node[Meaning] at (-25.750000, 6.900000) {ice};
\node[Square] at (-23.700000, 5.150000) {};
\node[Kanji] at (-23.700000, 5.650000) {訪};
\node[Onyomi] at (-23.650000, 5.250000) {ホウ};
\node[Kunyomi] at (-23.750000, 5.250000) {たず.ねる};
\node[Meaning] at (-23.700000, 6.900000) {visit};
\node[Square] at (-21.650000, 5.150000) {};
\node[Kanji] at (-21.650000, 5.650000) {純};
\node[Onyomi] at (-21.600000, 5.250000) {ジュン};
\node[Meaning] at (-21.650000, 6.900000) {pure};
\node[Square] at (-19.600000, 5.150000) {};
\node[Kanji] at (-19.600000, 5.650000) {掲};
\node[Onyomi] at (-19.550000, 5.250000) {ケイ};
\node[Kunyomi] at (-19.650000, 5.250000) {かか.げる};
\node[Meaning] at (-19.600000, 6.900000) {display};
\node[Square] at (-17.550000, 5.150000) {};
\node[Kanji] at (-17.550000, 5.650000) {双};
\node[Onyomi] at (-17.500000, 5.250000) {ソウ};
\node[Kunyomi] at (-17.600000, 5.250000) {ふた};
\node[Meaning] at (-17.550000, 6.900000) {pair};
\node[Square] at (-15.500000, 5.150000) {};
\node[Kanji] at (-15.500000, 5.650000) {停};
\node[Onyomi] at (-15.450000, 5.250000) {テイ};
\node[Meaning] at (-15.500000, 6.900000) {halt};
\node[Square] at (-13.450000, 5.150000) {};
\node[Kanji] at (-13.450000, 5.650000) {似};
\node[Onyomi] at (-13.400000, 5.250000) {ネ};
\node[Kunyomi] at (-13.500000, 5.250000) {に.る};
\node[Meaning] at (-13.450000, 6.900000) {resemble};
\node[Square] at (-11.400000, 5.150000) {};
\node[Kanji] at (-11.400000, 5.650000) {脅};
\node[Onyomi] at (-11.350000, 5.250000) {キョウ};
\node[Kunyomi] at (-11.450000, 5.250000) {おど};
\node[Meaning] at (-11.400000, 6.900000) {threaten};
\node[Square] at (-9.350000, 5.150000) {};
\node[Kanji] at (-9.350000, 5.650000) {拾};
\node[Kunyomi] at (-9.400000, 5.250000) {ひろ};
\node[Meaning] at (-9.350000, 6.900000) {pick up};
\node[Square] at (-7.300000, 5.150000) {};
\node[Kanji] at (-7.300000, 5.650000) {忙};
\node[Onyomi] at (-7.250000, 5.250000) {ボウ};
\node[Kunyomi] at (-7.350000, 5.250000) {いそが};
\node[Meaning] at (-7.300000, 6.900000) {busy};
\node[Square] at (-5.250000, 5.150000) {};
\node[Kanji] at (-5.250000, 5.650000) {斉};
\node[Onyomi] at (-5.200000, 5.250000) {セイ};
\node[Meaning] at (-5.250000, 6.900000) {simultaneous};
\node[Square] at (-3.200000, 5.150000) {};
\node[Kanji] at (-3.200000, 5.650000) {肘};
\node[Kunyomi] at (-3.250000, 5.250000) {ひじ};
\node[Meaning] at (-3.200000, 6.900000) {elbow};
\node[Square] at (-1.150000, 5.150000) {};
\node[Kanji] at (-1.150000, 5.650000) {派};
\node[Onyomi] at (-1.100000, 5.250000) {ハ};
\node[Meaning] at (-1.150000, 6.900000) {sect};
\node[Square] at (0.900000, 5.150000) {};
\node[Kanji] at (0.900000, 5.650000) {闘};
\node[Onyomi] at (0.950000, 5.250000) {トウ};
\node[Kunyomi] at (0.850000, 5.250000) {たたか.う};
\node[Meaning] at (0.900000, 6.900000) {struggle};
\node[Square] at (2.950000, 5.150000) {};
\node[Kanji] at (2.950000, 5.650000) {勇};
\node[Onyomi] at (3.000000, 5.250000) {ユウ};
\node[Kunyomi] at (2.900000, 5.250000) {いさ};
\node[Meaning] at (2.950000, 6.900000) {courage};
\node[Square] at (5.000000, 5.150000) {};
\node[Kanji] at (5.000000, 5.650000) {審};
\node[Onyomi] at (5.050000, 5.250000) {シン};
\node[Meaning] at (5.000000, 6.900000) {judge};
\node[Square] at (7.050000, 5.150000) {};
\node[Kanji] at (7.050000, 5.650000) {域};
\node[Onyomi] at (7.100000, 5.250000) {イキ};
\node[Meaning] at (7.050000, 6.900000) {region};
\node[Square] at (9.100000, 5.150000) {};
\node[Kanji] at (9.100000, 5.650000) {宅};
\node[Onyomi] at (9.150000, 5.250000) {タク};
\node[Meaning] at (9.100000, 6.900000) {house};
\node[Square] at (11.150000, 5.150000) {};
\node[Kanji] at (11.150000, 5.650000) {粉};
\node[Onyomi] at (11.200000, 5.250000) {フン};
\node[Kunyomi] at (11.100000, 5.250000) {こな};
\node[Meaning] at (11.150000, 6.900000) {powder};
\node[Square] at (13.200000, 5.150000) {};
\node[Kanji] at (13.200000, 5.650000) {諸};
\node[Onyomi] at (13.250000, 5.250000) {ショ};
\node[Kunyomi] at (13.150000, 5.250000) {もろ};
\node[Meaning] at (13.200000, 6.900000) {various};
\node[Square] at (15.250000, 5.150000) {};
\node[Kanji] at (15.250000, 5.650000) {挟};
\node[Onyomi] at (15.300000, 5.250000) {キョウ};
\node[Kunyomi] at (15.200000, 5.250000) {はさ};
\node[Meaning] at (15.250000, 6.900000) {between};
\node[Square] at (17.300000, 5.150000) {};
\node[Kanji] at (17.300000, 5.650000) {濃};
\node[Onyomi] at (17.350000, 5.250000) {ノウ};
\node[Kunyomi] at (17.250000, 5.250000) {こ.い};
\node[Meaning] at (17.300000, 6.900000) {thick};
\node[Square] at (19.350000, 5.150000) {};
\node[Kanji] at (19.350000, 5.650000) {棒};
\node[Onyomi] at (19.400000, 5.250000) {ボウ};
\node[Kunyomi] at (19.300000, 5.250000) {ぼう};
\node[Meaning] at (19.350000, 6.900000) {pole};
\node[Square] at (21.400000, 5.150000) {};
\node[Kanji] at (21.400000, 5.650000) {暮};
\node[Onyomi] at (21.450000, 5.250000) {ボ};
\node[Kunyomi] at (21.350000, 5.250000) {く.*};
\node[Meaning] at (21.400000, 6.900000) {livelihood};
\node[Square] at (23.450000, 5.150000) {};
\node[Kanji] at (23.450000, 5.650000) {漁};
\node[Onyomi] at (23.500000, 5.250000) {ギョ};
\node[Kunyomi] at (23.400000, 5.250000) {あさ.る};
\node[Meaning] at (23.450000, 6.900000) {fishing};
\node[Square] at (25.500000, 5.150000) {};
\node[Kanji] at (25.500000, 5.650000) {賢};
\node[Onyomi] at (25.550000, 5.250000) {ケン};
\node[Kunyomi] at (25.450000, 5.250000) {かしこ.い};
\node[Meaning] at (25.500000, 6.900000) {clever};
\node[Square] at (27.550000, 5.150000) {};
\node[Kanji] at (27.550000, 5.650000) {顎};
\node[Onyomi] at (27.600000, 5.250000) {ガク};
\node[Kunyomi] at (27.500000, 5.250000) {あご};
\node[Meaning] at (27.550000, 6.900000) {jaw};
\node[Square] at (29.600000, 5.150000) {};
\node[Kanji] at (29.600000, 5.650000) {材};
\node[Onyomi] at (29.650000, 5.250000) {ザイ};
\node[Meaning] at (29.600000, 6.900000) {lumber};
\node[Square] at (31.650000, 5.150000) {};
\node[Kanji] at (31.650000, 5.650000) {修};
\node[Onyomi] at (31.700000, 5.250000) {シュウ};
\node[Kunyomi] at (31.600000, 5.250000) {おさ.まる};
\node[Meaning] at (31.650000, 6.900000) {discipline};
\node[Square] at (33.700000, 5.150000) {};
\node[Kanji] at (33.700000, 5.650000) {郎};
\node[Onyomi] at (33.750000, 5.250000) {ロウ};
\node[Meaning] at (33.700000, 6.900000) {guy};
\node[Square] at (35.750000, 5.150000) {};
\node[Kanji] at (35.750000, 5.650000) {娘};
\node[Kunyomi] at (35.700000, 5.250000) {むすめ};
\node[Meaning] at (35.750000, 6.900000) {daughter};
\node[Square] at (37.800000, 5.150000) {};
\node[Kanji] at (37.800000, 5.650000) {党};
\node[Onyomi] at (37.850000, 5.250000) {トウ};
\node[Meaning] at (37.800000, 6.900000) {group};
\node[Square] at (39.850000, 5.150000) {};
\node[Kanji] at (39.850000, 5.650000) {欲};
\node[Onyomi] at (39.900000, 5.250000) {ヨク};
\node[Kunyomi] at (39.800000, 5.250000) {ほ.しい};
\node[Meaning] at (39.850000, 6.900000) {want};
\node[Square] at (41.900000, 5.150000) {};
\node[Kanji] at (41.900000, 5.650000) {籠};
\node[Onyomi] at (41.950000, 5.250000) {ロウ};
\node[Kunyomi] at (41.850000, 5.250000) {かご};
\node[Meaning] at (41.900000, 6.900000) {basket};
\node[Square] at (43.950000, 5.150000) {};
\node[Kanji] at (43.950000, 5.650000) {泥};
\node[Onyomi] at (44.000000, 5.250000) {デイ};
\node[Kunyomi] at (43.900000, 5.250000) {どろ};
\node[Meaning] at (43.950000, 6.900000) {mud};
\node[Square] at (46.000000, 5.150000) {};
\node[Kanji] at (46.000000, 5.650000) {聖};
\node[Onyomi] at (46.050000, 5.250000) {セイ};
\node[Meaning] at (46.000000, 6.900000) {holy};
\node[Square] at (48.050000, 5.150000) {};
\node[Kanji] at (48.050000, 5.650000) {展};
\node[Onyomi] at (48.100000, 5.250000) {テン};
\node[Kunyomi] at (48.000000, 5.250000) {のぶ};
\node[Meaning] at (48.050000, 6.900000) {expand};
\node[Square] at (50.100000, 5.150000) {};
\node[Kanji] at (50.100000, 5.650000) {枝};
\node[Onyomi] at (50.150000, 5.250000) {シ};
\node[Kunyomi] at (50.050000, 5.250000) {えだ};
\node[Meaning] at (50.100000, 6.900000) {branch};
\node[Square] at (52.150000, 5.150000) {};
\node[Kanji] at (52.150000, 5.650000) {跳};
\node[Onyomi] at (52.200000, 5.250000) {チョウ};
\node[Kunyomi] at (52.100000, 5.250000) {と.ぶ};
\node[Meaning] at (52.150000, 6.900000) {hop};
\node[Square] at (54.200000, 5.150000) {};
\node[Kanji] at (54.200000, 5.650000) {牛};
\node[Onyomi] at (54.250000, 5.250000) {ギュウ};
\node[Kunyomi] at (54.150000, 5.250000) {うし};
\node[Meaning] at (54.200000, 6.900000) {cow};
\node[Square] at (56.250000, 5.150000) {};
\node[Kanji] at (56.250000, 5.650000) {兵};
\node[Onyomi] at (56.300000, 5.250000) {ヘイ};
\node[Meaning] at (56.250000, 6.900000) {soldier};
\node[Meaning] at (-58.500000, 5.700000) {93.40\%};
\node[Square] at (-56.500000, 3.100000) {};
\node[Kanji] at (-56.500000, 3.600000) {資};
\node[Onyomi] at (-56.450000, 3.200000) {シ};
\node[Meaning] at (-56.500000, 4.850000) {resources};
\node[Square] at (-54.450000, 3.100000) {};
\node[Kanji] at (-54.450000, 3.600000) {爪};
\node[Onyomi] at (-54.400000, 3.200000) {ソウ};
\node[Kunyomi] at (-54.500000, 3.200000) {つま};
\node[Meaning] at (-54.450000, 4.850000) {claw};
\node[Square] at (-52.400000, 3.100000) {};
\node[Kanji] at (-52.400000, 3.600000) {抑};
\node[Onyomi] at (-52.350000, 3.200000) {ヨク};
\node[Kunyomi] at (-52.450000, 3.200000) {おさ.える};
\node[Meaning] at (-52.400000, 4.850000) {suppress};
\node[Square] at (-50.350000, 3.100000) {};
\node[Kanji] at (-50.350000, 3.600000) {縛};
\node[Onyomi] at (-50.300000, 3.200000) {バク};
\node[Kunyomi] at (-50.400000, 3.200000) {しば};
\node[Meaning] at (-50.350000, 4.850000) {bind};
\node[Square] at (-48.300000, 3.100000) {};
\node[Kanji] at (-48.300000, 3.600000) {往};
\node[Onyomi] at (-48.250000, 3.200000) {オウ};
\node[Meaning] at (-48.300000, 4.850000) {depart};
\node[Square] at (-46.250000, 3.100000) {};
\node[Kanji] at (-46.250000, 3.600000) {逮};
\node[Onyomi] at (-46.200000, 3.200000) {タイ};
\node[Meaning] at (-46.250000, 4.850000) {apprehend};
\node[Square] at (-44.200000, 3.100000) {};
\node[Kanji] at (-44.200000, 3.600000) {香};
\node[Onyomi] at (-44.150000, 3.200000) {コウ};
\node[Kunyomi] at (-44.250000, 3.200000) {かお};
\node[Meaning] at (-44.200000, 4.850000) {fragrance};
\node[Square] at (-42.150000, 3.100000) {};
\node[Kanji] at (-42.150000, 3.600000) {崎};
\node[Onyomi] at (-42.100000, 3.200000) {キ};
\node[Kunyomi] at (-42.200000, 3.200000) {さき};
\node[Meaning] at (-42.150000, 4.850000) {cape};
\node[Square] at (-40.100000, 3.100000) {};
\node[Kanji] at (-40.100000, 3.600000) {限};
\node[Onyomi] at (-40.050000, 3.200000) {ゲン};
\node[Kunyomi] at (-40.150000, 3.200000) {かぎ.る};
\node[Meaning] at (-40.100000, 4.850000) {limit};
\node[Square] at (-38.050000, 3.100000) {};
\node[Kanji] at (-38.050000, 3.600000) {迫};
\node[Onyomi] at (-38.000000, 3.200000) {ハク};
\node[Kunyomi] at (-38.100000, 3.200000) {せま.る};
\node[Meaning] at (-38.050000, 4.850000) {urge};
\node[Square] at (-36.000000, 3.100000) {};
\node[Kanji] at (-36.000000, 3.600000) {将};
\node[Onyomi] at (-35.950000, 3.200000) {ショウ};
\node[Meaning] at (-36.000000, 4.850000) {commander};
\node[Square] at (-33.950000, 3.100000) {};
\node[Kanji] at (-33.950000, 3.600000) {湯};
\node[Onyomi] at (-33.900000, 3.200000) {トウ};
\node[Kunyomi] at (-34.000000, 3.200000) {ゆ};
\node[Meaning] at (-33.950000, 4.850000) {hot water};
\node[Square] at (-31.900000, 3.100000) {};
\node[Kanji] at (-31.900000, 3.600000) {税};
\node[Onyomi] at (-31.850000, 3.200000) {ゼイ};
\node[Meaning] at (-31.900000, 4.850000) {tax};
\node[Square] at (-29.850000, 3.100000) {};
\node[Kanji] at (-29.850000, 3.600000) {極};
\node[Onyomi] at (-29.800000, 3.200000) {キョク};
\node[Kunyomi] at (-29.900000, 3.200000) {きわ.める};
\node[Meaning] at (-29.850000, 4.850000) {extreme};
\node[Square] at (-27.800000, 3.100000) {};
\node[Kanji] at (-27.800000, 3.600000) {秋};
\node[Kunyomi] at (-27.850000, 3.200000) {あき};
\node[Meaning] at (-27.800000, 4.850000) {autumn};
\node[Square] at (-25.750000, 3.100000) {};
\node[Kanji] at (-25.750000, 3.600000) {糸};
\node[Onyomi] at (-25.700000, 3.200000) {シ};
\node[Kunyomi] at (-25.800000, 3.200000) {いと};
\node[Meaning] at (-25.750000, 4.850000) {thread};
\node[Square] at (-23.700000, 3.100000) {};
\node[Kanji] at (-23.700000, 3.600000) {織};
\node[Onyomi] at (-23.650000, 3.200000) {シキ};
\node[Kunyomi] at (-23.750000, 3.200000) {お.る};
\node[Meaning] at (-23.700000, 4.850000) {weave};
\node[Square] at (-21.650000, 3.100000) {};
\node[Kanji] at (-21.650000, 3.600000) {傾};
\node[Onyomi] at (-21.600000, 3.200000) {ケイ};
\node[Kunyomi] at (-21.700000, 3.200000) {かたむ.*};
\node[Meaning] at (-21.650000, 4.850000) {lean};
\node[Square] at (-19.600000, 3.100000) {};
\node[Kanji] at (-19.600000, 3.600000) {封};
\node[Onyomi] at (-19.550000, 3.200000) {フウ};
\node[Meaning] at (-19.600000, 4.850000) {seal};
\node[Square] at (-17.550000, 3.100000) {};
\node[Kanji] at (-17.550000, 3.600000) {歴};
\node[Onyomi] at (-17.500000, 3.200000) {レキ};
\node[Kunyomi] at (-17.600000, 3.200000) {へ.る};
\node[Meaning] at (-17.550000, 4.850000) {continuation};
\node[Square] at (-15.500000, 3.100000) {};
\node[Kanji] at (-15.500000, 3.600000) {模};
\node[Onyomi] at (-15.450000, 3.200000) {モ};
\node[Meaning] at (-15.500000, 4.850000) {imitation};
\node[Square] at (-13.450000, 3.100000) {};
\node[Kanji] at (-13.450000, 3.600000) {菜};
\node[Onyomi] at (-13.400000, 3.200000) {サイ};
\node[Meaning] at (-13.450000, 4.850000) {vegetable};
\node[Square] at (-11.400000, 3.100000) {};
\node[Kanji] at (-11.400000, 3.600000) {忠};
\node[Onyomi] at (-11.350000, 3.200000) {チュウ};
\node[Meaning] at (-11.400000, 4.850000) {loyalty};
\node[Square] at (-9.350000, 3.100000) {};
\node[Kanji] at (-9.350000, 3.600000) {盛};
\node[Onyomi] at (-9.300000, 3.200000) {セイ};
\node[Kunyomi] at (-9.400000, 3.200000) {も.る};
\node[Meaning] at (-9.350000, 4.850000) {pile};
\node[Square] at (-7.300000, 3.100000) {};
\node[Kanji] at (-7.300000, 3.600000) {誇};
\node[Onyomi] at (-7.250000, 3.200000) {コ};
\node[Kunyomi] at (-7.350000, 3.200000) {ほこ.る};
\node[Meaning] at (-7.300000, 4.850000) {pride};
\node[Square] at (-5.250000, 3.100000) {};
\node[Kanji] at (-5.250000, 3.600000) {蹴};
\node[Onyomi] at (-5.200000, 3.200000) {シュウ};
\node[Kunyomi] at (-5.300000, 3.200000) {け-る};
\node[Meaning] at (-5.250000, 4.850000) {kick};
\node[Square] at (-3.200000, 3.100000) {};
\node[Kanji] at (-3.200000, 3.600000) {仮};
\node[Onyomi] at (-3.150000, 3.200000) {カ};
\node[Kunyomi] at (-3.250000, 3.200000) {かり};
\node[Meaning] at (-3.200000, 4.850000) {temporary};
\node[Square] at (-1.150000, 3.100000) {};
\node[Kanji] at (-1.150000, 3.600000) {掃};
\node[Onyomi] at (-1.100000, 3.200000) {ソウ};
\node[Kunyomi] at (-1.200000, 3.200000) {は.く};
\node[Meaning] at (-1.150000, 4.850000) {sweep};
\node[Square] at (0.900000, 3.100000) {};
\node[Kanji] at (0.900000, 3.600000) {縁};
\node[Onyomi] at (0.950000, 3.200000) {エン};
\node[Kunyomi] at (0.850000, 3.200000) {ふち};
\node[Meaning] at (0.900000, 4.850000) {edge};
\node[Square] at (2.950000, 3.100000) {};
\node[Kanji] at (2.950000, 3.600000) {逸};
\node[Onyomi] at (3.000000, 3.200000) {イツ};
\node[Kunyomi] at (2.900000, 3.200000) {そ};
\node[Meaning] at (2.950000, 4.850000) {deviate};
\node[Square] at (5.000000, 3.100000) {};
\node[Kanji] at (5.000000, 3.600000) {招};
\node[Onyomi] at (5.050000, 3.200000) {ショウ};
\node[Kunyomi] at (4.950000, 3.200000) {まね.く};
\node[Meaning] at (5.000000, 4.850000) {beckon};
\node[Square] at (7.050000, 3.100000) {};
\node[Kanji] at (7.050000, 3.600000) {旗};
\node[Onyomi] at (7.100000, 3.200000) {キ};
\node[Kunyomi] at (7.000000, 3.200000) {はた};
\node[Meaning] at (7.050000, 4.850000) {flag};
\node[Square] at (9.100000, 3.100000) {};
\node[Kanji] at (9.100000, 3.600000) {級};
\node[Onyomi] at (9.150000, 3.200000) {キュウ};
\node[Meaning] at (9.100000, 4.850000) {rank};
\node[Square] at (11.150000, 3.100000) {};
\node[Kanji] at (11.150000, 3.600000) {努};
\node[Onyomi] at (11.200000, 3.200000) {ド};
\node[Kunyomi] at (11.100000, 3.200000) {つと.める};
\node[Meaning] at (11.150000, 4.850000) {toil};
\node[Square] at (13.200000, 3.100000) {};
\node[Kanji] at (13.200000, 3.600000) {功};
\node[Onyomi] at (13.250000, 3.200000) {コウ};
\node[Meaning] at (13.200000, 4.850000) {achievement};
\node[Square] at (15.250000, 3.100000) {};
\node[Kanji] at (15.250000, 3.600000) {浴};
\node[Onyomi] at (15.300000, 3.200000) {ヨク};
\node[Kunyomi] at (15.200000, 3.200000) {あ};
\node[Meaning] at (15.250000, 4.850000) {bathe};
\node[Square] at (17.300000, 3.100000) {};
\node[Kanji] at (17.300000, 3.600000) {候};
\node[Onyomi] at (17.350000, 3.200000) {コウ};
\node[Meaning] at (17.300000, 4.850000) {climate};
\node[Square] at (19.350000, 3.100000) {};
\node[Kanji] at (19.350000, 3.600000) {汗};
\node[Onyomi] at (19.400000, 3.200000) {カン};
\node[Kunyomi] at (19.300000, 3.200000) {あせ};
\node[Meaning] at (19.350000, 4.850000) {sweat};
\node[Square] at (21.400000, 3.100000) {};
\node[Kanji] at (21.400000, 3.600000) {液};
\node[Onyomi] at (21.450000, 3.200000) {エキ};
\node[Meaning] at (21.400000, 4.850000) {fluid};
\node[Square] at (23.450000, 3.100000) {};
\node[Kanji] at (23.450000, 3.600000) {伏};
\node[Onyomi] at (23.500000, 3.200000) {フク};
\node[Kunyomi] at (23.400000, 3.200000) {ふ};
\node[Meaning] at (23.450000, 4.850000) {bow};
\node[Square] at (25.500000, 3.100000) {};
\node[Kanji] at (25.500000, 3.600000) {街};
\node[Onyomi] at (25.550000, 3.200000) {ガイ};
\node[Kunyomi] at (25.450000, 3.200000) {まち};
\node[Meaning] at (25.500000, 4.850000) {street};
\node[Square] at (27.550000, 3.100000) {};
\node[Kanji] at (27.550000, 3.600000) {衆};
\node[Onyomi] at (27.600000, 3.200000) {シュウ};
\node[Meaning] at (27.550000, 4.850000) {populace};
\node[Square] at (29.600000, 3.100000) {};
\node[Kanji] at (29.600000, 3.600000) {均};
\node[Onyomi] at (29.650000, 3.200000) {キン};
\node[Kunyomi] at (29.550000, 3.200000) {ひと.しい};
\node[Meaning] at (29.600000, 4.850000) {equal};
\node[Square] at (31.650000, 3.100000) {};
\node[Kanji] at (31.650000, 3.600000) {納};
\node[Onyomi] at (31.700000, 3.200000) {ノウ};
\node[Kunyomi] at (31.600000, 3.200000) {おさ};
\node[Meaning] at (31.650000, 4.850000) {supply};
\node[Square] at (33.700000, 3.100000) {};
\node[Kanji] at (33.700000, 3.600000) {恥};
\node[Onyomi] at (33.750000, 3.200000) {チ};
\node[Kunyomi] at (33.650000, 3.200000) {は};
\node[Meaning] at (33.700000, 4.850000) {shame};
\node[Square] at (35.750000, 3.100000) {};
\node[Kanji] at (35.750000, 3.600000) {御};
\node[Onyomi] at (35.800000, 3.200000) {ゴ};
\node[Kunyomi] at (35.700000, 3.200000) {お};
\node[Meaning] at (35.750000, 4.850000) {honorable};
\node[Square] at (37.800000, 3.100000) {};
\node[Kanji] at (37.800000, 3.600000) {盆};
\node[Onyomi] at (37.850000, 3.200000) {ボン};
\node[Meaning] at (37.800000, 4.850000) {lantern festival};
\node[Square] at (39.850000, 3.100000) {};
\node[Kanji] at (39.850000, 3.600000) {憎};
\node[Onyomi] at (39.900000, 3.200000) {ゾウ};
\node[Kunyomi] at (39.800000, 3.200000) {にく.*};
\node[Meaning] at (39.850000, 4.850000) {hate};
\node[Square] at (41.900000, 3.100000) {};
\node[Kanji] at (41.900000, 3.600000) {雲};
\node[Onyomi] at (41.950000, 3.200000) {ウン};
\node[Kunyomi] at (41.850000, 3.200000) {くも};
\node[Meaning] at (41.900000, 4.850000) {cloud};
\node[Square] at (43.950000, 3.100000) {};
\node[Kanji] at (43.950000, 3.600000) {担};
\node[Onyomi] at (44.000000, 3.200000) {タン};
\node[Kunyomi] at (43.900000, 3.200000) {にな.う};
\node[Meaning] at (43.950000, 4.850000) {carry};
\node[Square] at (46.000000, 3.100000) {};
\node[Kanji] at (46.000000, 3.600000) {暇};
\node[Onyomi] at (46.050000, 3.200000) {カ};
\node[Kunyomi] at (45.950000, 3.200000) {ひま};
\node[Meaning] at (46.000000, 4.850000) {spare time};
\node[Square] at (48.050000, 3.100000) {};
\node[Kanji] at (48.050000, 3.600000) {呆};
\node[Onyomi] at (48.100000, 3.200000) {ホウ};
\node[Kunyomi] at (48.000000, 3.200000) {あき};
\node[Meaning] at (48.050000, 4.850000) {shock};
\node[Square] at (50.100000, 3.100000) {};
\node[Kanji] at (50.100000, 3.600000) {属};
\node[Onyomi] at (50.150000, 3.200000) {ゾク};
\node[Meaning] at (50.100000, 4.850000) {belong};
\node[Square] at (52.150000, 3.100000) {};
\node[Kanji] at (52.150000, 3.600000) {豆};
\node[Onyomi] at (52.200000, 3.200000) {トウ};
\node[Kunyomi] at (52.100000, 3.200000) {まめ};
\node[Meaning] at (52.150000, 4.850000) {beans};
\node[Square] at (54.200000, 3.100000) {};
\node[Kanji] at (54.200000, 3.600000) {奪};
\node[Onyomi] at (54.250000, 3.200000) {ダツ};
\node[Kunyomi] at (54.150000, 3.200000) {うば};
\node[Meaning] at (54.200000, 4.850000) {rob};
\node[Square] at (56.250000, 3.100000) {};
\node[Kanji] at (56.250000, 3.600000) {骸};
\node[Onyomi] at (56.300000, 3.200000) {ガイ};
\node[Meaning] at (56.250000, 4.850000) {dead remains};
\node[Meaning] at (-58.500000, 3.650000) {94.13\%};
\node[Square] at (-56.500000, 1.050000) {};
\node[Kanji] at (-56.500000, 1.550000) {齢};
\node[Onyomi] at (-56.450000, 1.150000) {レイ};
\node[Kunyomi] at (-56.550000, 1.150000) {よわい};
\node[Meaning] at (-56.500000, 2.800000) {age};
\node[Square] at (-54.450000, 1.050000) {};
\node[Kanji] at (-54.450000, 1.550000) {況};
\node[Onyomi] at (-54.400000, 1.150000) {キョウ};
\node[Meaning] at (-54.450000, 2.800000) {condition};
\node[Square] at (-52.400000, 1.050000) {};
\node[Kanji] at (-52.400000, 1.550000) {胃};
\node[Onyomi] at (-52.350000, 1.150000) {イ};
\node[Meaning] at (-52.400000, 2.800000) {stomach};
\node[Square] at (-50.350000, 1.050000) {};
\node[Kanji] at (-50.350000, 1.550000) {悔};
\node[Onyomi] at (-50.300000, 1.150000) {カイ};
\node[Kunyomi] at (-50.400000, 1.150000) {くや.しい};
\node[Meaning] at (-50.350000, 2.800000) {regret};
\node[Square] at (-48.300000, 1.050000) {};
\node[Kanji] at (-48.300000, 1.550000) {耐};
\node[Onyomi] at (-48.250000, 1.150000) {タイ};
\node[Kunyomi] at (-48.350000, 1.150000) {た.える};
\node[Meaning] at (-48.300000, 2.800000) {resistant};
\node[Square] at (-46.250000, 1.050000) {};
\node[Kanji] at (-46.250000, 1.550000) {鶏};
\node[Onyomi] at (-46.200000, 1.150000) {ケイ};
\node[Kunyomi] at (-46.300000, 1.150000) {とり};
\node[Meaning] at (-46.250000, 2.800000) {chicken};
\node[Square] at (-44.200000, 1.050000) {};
\node[Kanji] at (-44.200000, 1.550000) {冗};
\node[Onyomi] at (-44.150000, 1.150000) {ジョウ};
\node[Meaning] at (-44.200000, 2.800000) {superfluous};
\node[Square] at (-42.150000, 1.050000) {};
\node[Kanji] at (-42.150000, 1.550000) {乳};
\node[Onyomi] at (-42.100000, 1.150000) {ニュウ};
\node[Meaning] at (-42.150000, 2.800000) {milk};
\node[Square] at (-40.100000, 1.050000) {};
\node[Kanji] at (-40.100000, 1.550000) {施};
\node[Onyomi] at (-40.050000, 1.150000) {シ};
\node[Kunyomi] at (-40.150000, 1.150000) {ほどこ.す};
\node[Meaning] at (-40.100000, 2.800000) {carry out};
\node[Square] at (-38.050000, 1.050000) {};
\node[Kanji] at (-38.050000, 1.550000) {測};
\node[Onyomi] at (-38.000000, 1.150000) {ソク};
\node[Kunyomi] at (-38.100000, 1.150000) {はか.る};
\node[Meaning] at (-38.050000, 2.800000) {measure};
\node[Square] at (-36.000000, 1.050000) {};
\node[Kanji] at (-36.000000, 1.550000) {躍};
\node[Onyomi] at (-35.950000, 1.150000) {ヤク};
\node[Kunyomi] at (-36.050000, 1.150000) {おど.る};
\node[Meaning] at (-36.000000, 2.800000) {leap};
\node[Square] at (-33.950000, 1.050000) {};
\node[Kanji] at (-33.950000, 1.550000) {汽};
\node[Onyomi] at (-33.900000, 1.150000) {キ};
\node[Meaning] at (-33.950000, 2.800000) {steam};
\node[Square] at (-31.900000, 1.050000) {};
\node[Kanji] at (-31.900000, 1.550000) {獣};
\node[Onyomi] at (-31.850000, 1.150000) {ジュウ};
\node[Kunyomi] at (-31.950000, 1.150000) {けもの};
\node[Meaning] at (-31.900000, 2.800000) {beast};
\node[Square] at (-29.850000, 1.050000) {};
\node[Kanji] at (-29.850000, 1.550000) {系};
\node[Onyomi] at (-29.800000, 1.150000) {ケイ};
\node[Meaning] at (-29.850000, 2.800000) {lineage};
\node[Square] at (-27.800000, 1.050000) {};
\node[Kanji] at (-27.800000, 1.550000) {熊};
\node[Kunyomi] at (-27.850000, 1.150000) {くま};
\node[Meaning] at (-27.800000, 2.800000) {bear};
\node[Square] at (-25.750000, 1.050000) {};
\node[Kanji] at (-25.750000, 1.550000) {仰};
\node[Onyomi] at (-25.700000, 1.150000) {ギョウ};
\node[Kunyomi] at (-25.800000, 1.150000) {あお.ぐ};
\node[Meaning] at (-25.750000, 2.800000) {look up to};
\node[Square] at (-23.700000, 1.050000) {};
\node[Kanji] at (-23.700000, 1.550000) {喚};
\node[Onyomi] at (-23.650000, 1.150000) {カン};
\node[Kunyomi] at (-23.750000, 1.150000) {わめ};
\node[Meaning] at (-23.700000, 2.800000) {scream};
\node[Square] at (-21.650000, 1.050000) {};
\node[Kanji] at (-21.650000, 1.550000) {営};
\node[Onyomi] at (-21.600000, 1.150000) {エイ};
\node[Kunyomi] at (-21.700000, 1.150000) {いとな.む};
\node[Meaning] at (-21.650000, 2.800000) {manage};
\node[Square] at (-19.600000, 1.050000) {};
\node[Kanji] at (-19.600000, 1.550000) {承};
\node[Onyomi] at (-19.550000, 1.150000) {ショウ};
\node[Kunyomi] at (-19.650000, 1.150000) {うけたまわ.る};
\node[Meaning] at (-19.600000, 2.800000) {consent};
\node[Square] at (-17.550000, 1.050000) {};
\node[Kanji] at (-17.550000, 1.550000) {膨};
\node[Onyomi] at (-17.500000, 1.150000) {ボウ};
\node[Kunyomi] at (-17.600000, 1.150000) {ふく};
\node[Meaning] at (-17.550000, 2.800000) {swell};
\node[Square] at (-15.500000, 1.050000) {};
\node[Kanji] at (-15.500000, 1.550000) {憂};
\node[Onyomi] at (-15.450000, 1.150000) {ユウ};
\node[Kunyomi] at (-15.550000, 1.150000) {う};
\node[Meaning] at (-15.500000, 2.800000) {grief};
\node[Square] at (-13.450000, 1.050000) {};
\node[Kanji] at (-13.450000, 1.550000) {貼};
\node[Onyomi] at (-13.400000, 1.150000) {チョウ};
\node[Kunyomi] at (-13.500000, 1.150000) {は};
\node[Meaning] at (-13.450000, 2.800000) {paste};
\node[Square] at (-11.400000, 1.050000) {};
\node[Kanji] at (-11.400000, 1.550000) {郵};
\node[Onyomi] at (-11.350000, 1.150000) {ユウ};
\node[Meaning] at (-11.400000, 2.800000) {mail};
\node[Square] at (-9.350000, 1.050000) {};
\node[Kanji] at (-9.350000, 1.550000) {昇};
\node[Onyomi] at (-9.300000, 1.150000) {ショウ};
\node[Kunyomi] at (-9.400000, 1.150000) {のぼ.る};
\node[Meaning] at (-9.350000, 2.800000) {ascend};
\node[Square] at (-7.300000, 1.050000) {};
\node[Kanji] at (-7.300000, 1.550000) {甘};
\node[Onyomi] at (-7.250000, 1.150000) {カン};
\node[Kunyomi] at (-7.350000, 1.150000) {あま};
\node[Meaning] at (-7.300000, 2.800000) {sweet};
\node[Square] at (-5.250000, 1.050000) {};
\node[Kanji] at (-5.250000, 1.550000) {憤};
\node[Onyomi] at (-5.200000, 1.150000) {フン};
\node[Kunyomi] at (-5.300000, 1.150000) {いきどお};
\node[Meaning] at (-5.250000, 2.800000) {resent};
\node[Square] at (-3.200000, 1.050000) {};
\node[Kanji] at (-3.200000, 1.550000) {標};
\node[Onyomi] at (-3.150000, 1.150000) {ヒョウ};
\node[Kunyomi] at (-3.250000, 1.150000) {しるし};
\node[Meaning] at (-3.200000, 2.800000) {signpost};
\node[Square] at (-1.150000, 1.050000) {};
\node[Kanji] at (-1.150000, 1.550000) {垣};
\node[Kunyomi] at (-1.200000, 1.150000) {かき};
\node[Meaning] at (-1.150000, 2.800000) {hedge};
\node[Square] at (0.900000, 1.050000) {};
\node[Kanji] at (0.900000, 1.550000) {央};
\node[Onyomi] at (0.950000, 1.150000) {オウ};
\node[Meaning] at (0.900000, 2.800000) {central};
\node[Square] at (2.950000, 1.050000) {};
\node[Kanji] at (2.950000, 1.550000) {農};
\node[Onyomi] at (3.000000, 1.150000) {ノウ};
\node[Meaning] at (2.950000, 2.800000) {farming};
\node[Square] at (5.000000, 1.050000) {};
\node[Kanji] at (5.000000, 1.550000) {敗};
\node[Onyomi] at (5.050000, 1.150000) {ハイ};
\node[Kunyomi] at (4.950000, 1.150000) {やぶ.れる};
\node[Meaning] at (5.000000, 2.800000) {failure};
\node[Square] at (7.050000, 1.050000) {};
\node[Kanji] at (7.050000, 1.550000) {永};
\node[Onyomi] at (7.100000, 1.150000) {エイ};
\node[Meaning] at (7.050000, 2.800000) {eternity};
\node[Square] at (9.100000, 1.050000) {};
\node[Kanji] at (9.100000, 1.550000) {異};
\node[Onyomi] at (9.150000, 1.150000) {イ};
\node[Kunyomi] at (9.050000, 1.150000) {こと.*};
\node[Meaning] at (9.100000, 2.800000) {differ};
\node[Square] at (11.150000, 1.050000) {};
\node[Kanji] at (11.150000, 1.550000) {慎};
\node[Onyomi] at (11.200000, 1.150000) {シン};
\node[Kunyomi] at (11.100000, 1.150000) {つつし.む};
\node[Meaning] at (11.150000, 2.800000) {humility};
\node[Square] at (13.200000, 1.050000) {};
\node[Kanji] at (13.200000, 1.550000) {頃};
\node[Kunyomi] at (13.150000, 1.150000) {ころ};
\node[Meaning] at (13.200000, 2.800000) {approximate};
\node[Square] at (15.250000, 1.050000) {};
\node[Kanji] at (15.250000, 1.550000) {隙};
\node[Onyomi] at (15.300000, 1.150000) {ゲキ};
\node[Kunyomi] at (15.200000, 1.150000) {すき};
\node[Meaning] at (15.250000, 2.800000) {crevice};
\node[Square] at (17.300000, 1.050000) {};
\node[Kanji] at (17.300000, 1.550000) {鎖};
\node[Onyomi] at (17.350000, 1.150000) {サ};
\node[Kunyomi] at (17.250000, 1.150000) {くさり};
\node[Meaning] at (17.300000, 2.800000) {chain};
\node[Square] at (19.350000, 1.050000) {};
\node[Kanji] at (19.350000, 1.550000) {蒼};
\node[Onyomi] at (19.400000, 1.150000) {ソウ};
\node[Kunyomi] at (19.300000, 1.150000) {あお};
\node[Meaning] at (19.350000, 2.800000) {pale};
\node[Square] at (21.400000, 1.050000) {};
\node[Kanji] at (21.400000, 1.550000) {票};
\node[Onyomi] at (21.450000, 1.150000) {ヒョウ};
\node[Meaning] at (21.400000, 2.800000) {ballot};
\node[Square] at (23.450000, 1.050000) {};
\node[Kanji] at (23.450000, 1.550000) {江};
\node[Onyomi] at (23.500000, 1.150000) {コウ};
\node[Kunyomi] at (23.400000, 1.150000) {え};
\node[Meaning] at (23.450000, 2.800000) {inlet};
\node[Square] at (25.500000, 1.050000) {};
\node[Kanji] at (25.500000, 1.550000) {児};
\node[Onyomi] at (25.550000, 1.150000) {ジ};
\node[Kunyomi] at (25.450000, 1.150000) {こ};
\node[Meaning] at (25.500000, 2.800000) {child};
\node[Square] at (27.550000, 1.050000) {};
\node[Kanji] at (27.550000, 1.550000) {博};
\node[Onyomi] at (27.600000, 1.150000) {ハク};
\node[Meaning] at (27.550000, 2.800000) {exhibition};
\node[Square] at (29.600000, 1.050000) {};
\node[Kanji] at (29.600000, 1.550000) {否};
\node[Onyomi] at (29.650000, 1.150000) {ヒ};
\node[Kunyomi] at (29.550000, 1.150000) {いな};
\node[Meaning] at (29.600000, 2.800000) {no};
\node[Square] at (31.650000, 1.050000) {};
\node[Kanji] at (31.650000, 1.550000) {贈};
\node[Onyomi] at (31.700000, 1.150000) {ゾウ};
\node[Kunyomi] at (31.600000, 1.150000) {おく.*};
\node[Meaning] at (31.650000, 2.800000) {presents};
\node[Square] at (33.700000, 1.050000) {};
\node[Kanji] at (33.700000, 1.550000) {札};
\node[Onyomi] at (33.750000, 1.150000) {サツ};
\node[Kunyomi] at (33.650000, 1.150000) {ふだ};
\node[Meaning] at (33.700000, 2.800000) {bill};
\node[Square] at (35.750000, 1.050000) {};
\node[Kanji] at (35.750000, 1.550000) {提};
\node[Onyomi] at (35.800000, 1.150000) {テイ};
\node[Meaning] at (35.750000, 2.800000) {present};
\node[Square] at (37.800000, 1.050000) {};
\node[Kanji] at (37.800000, 1.550000) {狭};
\node[Onyomi] at (37.850000, 1.150000) {キョウ};
\node[Kunyomi] at (37.750000, 1.150000) {せま};
\node[Meaning] at (37.800000, 2.800000) {narrow};
\node[Square] at (39.850000, 1.050000) {};
\node[Kanji] at (39.850000, 1.550000) {給};
\node[Onyomi] at (39.900000, 1.150000) {キュウ};
\node[Kunyomi] at (39.800000, 1.150000) {たま.う};
\node[Meaning] at (39.850000, 2.800000) {salary};
\node[Square] at (41.900000, 1.050000) {};
\node[Kanji] at (41.900000, 1.550000) {浜};
\node[Onyomi] at (41.950000, 1.150000) {ヒン};
\node[Kunyomi] at (41.850000, 1.150000) {はま};
\node[Meaning] at (41.900000, 2.800000) {beach};
\node[Square] at (43.950000, 1.050000) {};
\node[Kanji] at (43.950000, 1.550000) {敬};
\node[Onyomi] at (44.000000, 1.150000) {ケイ};
\node[Kunyomi] at (43.900000, 1.150000) {うやま.う};
\node[Meaning] at (43.950000, 2.800000) {respect};
\node[Square] at (46.000000, 1.050000) {};
\node[Kanji] at (46.000000, 1.550000) {剤};
\node[Onyomi] at (46.050000, 1.150000) {ザイ};
\node[Meaning] at (46.000000, 2.800000) {dose};
\node[Square] at (48.050000, 1.050000) {};
\node[Kanji] at (48.050000, 1.550000) {咲};
\node[Onyomi] at (48.100000, 1.150000) {ショウ};
\node[Kunyomi] at (48.000000, 1.150000) {さ};
\node[Meaning] at (48.050000, 2.800000) {blossom};
\node[Square] at (50.100000, 1.050000) {};
\node[Kanji] at (50.100000, 1.550000) {惨};
\node[Onyomi] at (50.150000, 1.150000) {サン};
\node[Kunyomi] at (50.050000, 1.150000) {みじ};
\node[Meaning] at (50.100000, 2.800000) {disaster};
\node[Square] at (52.150000, 1.050000) {};
\node[Kanji] at (52.150000, 1.550000) {換};
\node[Onyomi] at (52.200000, 1.150000) {カン};
\node[Kunyomi] at (52.100000, 1.150000) {か.える};
\node[Meaning] at (52.150000, 2.800000) {exchange};
\node[Square] at (54.200000, 1.050000) {};
\node[Kanji] at (54.200000, 1.550000) {療};
\node[Onyomi] at (54.250000, 1.150000) {リョウ};
\node[Meaning] at (54.200000, 2.800000) {heal};
\node[Square] at (56.250000, 1.050000) {};
\node[Kanji] at (56.250000, 1.550000) {賛};
\node[Onyomi] at (56.300000, 1.150000) {サン};
\node[Meaning] at (56.250000, 2.800000) {agree};
\node[Meaning] at (-58.500000, 1.600000) {94.77\%};
\node[Square] at (-56.500000, -1.000000) {};
\node[Kanji] at (-56.500000, -0.500000) {祖};
\node[Onyomi] at (-56.450000, -0.900000) {ソ};
\node[Meaning] at (-56.500000, 0.750000) {ancestor};
\node[Square] at (-54.450000, -1.000000) {};
\node[Kanji] at (-54.450000, -0.500000) {滅};
\node[Onyomi] at (-54.400000, -0.900000) {メツ};
\node[Kunyomi] at (-54.500000, -0.900000) {ほろ.*};
\node[Meaning] at (-54.450000, 0.750000) {destroy};
\node[Square] at (-52.400000, -1.000000) {};
\node[Kanji] at (-52.400000, -0.500000) {旋};
\node[Onyomi] at (-52.350000, -0.900000) {セン};
\node[Meaning] at (-52.400000, 0.750000) {rotation};
\node[Square] at (-50.350000, -1.000000) {};
\node[Kanji] at (-50.350000, -0.500000) {奈};
\node[Onyomi] at (-50.300000, -0.900000) {ナ};
\node[Kunyomi] at (-50.400000, -0.900000) {な};
\node[Meaning] at (-50.350000, 0.750000) {nara};
\node[Square] at (-48.300000, -1.000000) {};
\node[Kanji] at (-48.300000, -0.500000) {湾};
\node[Onyomi] at (-48.250000, -0.900000) {ワン};
\node[Meaning] at (-48.300000, 0.750000) {gulf};
\node[Square] at (-46.250000, -1.000000) {};
\node[Kanji] at (-46.250000, -0.500000) {吉};
\node[Onyomi] at (-46.200000, -0.900000) {キツ};
\node[Kunyomi] at (-46.300000, -0.900000) {よし};
\node[Meaning] at (-46.250000, 0.750000) {good luck};
\node[Square] at (-44.200000, -1.000000) {};
\node[Kanji] at (-44.200000, -0.500000) {哀};
\node[Onyomi] at (-44.150000, -0.900000) {アイ};
\node[Kunyomi] at (-44.250000, -0.900000) {あわ.れ*};
\node[Meaning] at (-44.200000, 0.750000) {pathetic};
\node[Square] at (-42.150000, -1.000000) {};
\node[Kanji] at (-42.150000, -0.500000) {縮};
\node[Onyomi] at (-42.100000, -0.900000) {シュク};
\node[Kunyomi] at (-42.200000, -0.900000) {ちぢ};
\node[Meaning] at (-42.150000, 0.750000) {shrink};
\node[Square] at (-40.100000, -1.000000) {};
\node[Kanji] at (-40.100000, -0.500000) {継};
\node[Onyomi] at (-40.050000, -0.900000) {ケイ};
\node[Kunyomi] at (-40.150000, -0.900000) {つ.ぐ};
\node[Meaning] at (-40.100000, 0.750000) {inherit};
\node[Square] at (-38.050000, -1.000000) {};
\node[Kanji] at (-38.050000, -0.500000) {拳};
\node[Onyomi] at (-38.000000, -0.900000) {ケン};
\node[Kunyomi] at (-38.100000, -0.900000) {こぶし};
\node[Meaning] at (-38.050000, 0.750000) {fist};
\node[Square] at (-36.000000, -1.000000) {};
\node[Kanji] at (-36.000000, -0.500000) {漂};
\node[Onyomi] at (-35.950000, -0.900000) {ヒョウ};
\node[Kunyomi] at (-36.050000, -0.900000) {ただよ.う};
\node[Meaning] at (-36.000000, 0.750000) {drift};
\node[Square] at (-33.950000, -1.000000) {};
\node[Kanji] at (-33.950000, -0.500000) {泳};
\node[Onyomi] at (-33.900000, -0.900000) {エイ};
\node[Kunyomi] at (-34.000000, -0.900000) {およ};
\node[Meaning] at (-33.950000, 0.750000) {swim};
\node[Square] at (-31.900000, -1.000000) {};
\node[Kanji] at (-31.900000, -0.500000) {猛};
\node[Onyomi] at (-31.850000, -0.900000) {モウ};
\node[Meaning] at (-31.900000, 0.750000) {fierce};
\node[Square] at (-29.850000, -1.000000) {};
\node[Kanji] at (-29.850000, -0.500000) {柱};
\node[Onyomi] at (-29.800000, -0.900000) {チュウ};
\node[Kunyomi] at (-29.900000, -0.900000) {はしら};
\node[Meaning] at (-29.850000, 0.750000) {pillar};
\node[Square] at (-27.800000, -1.000000) {};
\node[Kanji] at (-27.800000, -0.500000) {潜};
\node[Onyomi] at (-27.750000, -0.900000) {セン};
\node[Kunyomi] at (-27.850000, -0.900000) {くぐ.る};
\node[Meaning] at (-27.800000, 0.750000) {conceal};
\node[Square] at (-25.750000, -1.000000) {};
\node[Kanji] at (-25.750000, -0.500000) {刑};
\node[Onyomi] at (-25.700000, -0.900000) {ケイ};
\node[Meaning] at (-25.750000, 0.750000) {punish};
\node[Square] at (-23.700000, -1.000000) {};
\node[Kanji] at (-23.700000, -0.500000) {池};
\node[Onyomi] at (-23.650000, -0.900000) {チ};
\node[Kunyomi] at (-23.750000, -0.900000) {いけ};
\node[Meaning] at (-23.700000, 0.750000) {pond};
\node[Square] at (-21.650000, -1.000000) {};
\node[Kanji] at (-21.650000, -0.500000) {企};
\node[Onyomi] at (-21.600000, -0.900000) {キ};
\node[Kunyomi] at (-21.700000, -0.900000) {くわだ.てる};
\node[Meaning] at (-21.650000, 0.750000) {plan};
\node[Square] at (-19.600000, -1.000000) {};
\node[Kanji] at (-19.600000, -0.500000) {雄};
\node[Onyomi] at (-19.550000, -0.900000) {ユウ};
\node[Kunyomi] at (-19.650000, -0.900000) {おす};
\node[Meaning] at (-19.600000, 0.750000) {male};
\node[Square] at (-17.550000, -1.000000) {};
\node[Kanji] at (-17.550000, -0.500000) {導};
\node[Onyomi] at (-17.500000, -0.900000) {ドウ};
\node[Kunyomi] at (-17.600000, -0.900000) {みちび.く};
\node[Meaning] at (-17.550000, 0.750000) {lead};
\node[Square] at (-15.500000, -1.000000) {};
\node[Kanji] at (-15.500000, -0.500000) {垂};
\node[Onyomi] at (-15.450000, -0.900000) {スイ};
\node[Kunyomi] at (-15.550000, -0.900000) {た.*};
\node[Meaning] at (-15.500000, 0.750000) {dangle};
\node[Square] at (-13.450000, -1.000000) {};
\node[Kanji] at (-13.450000, -0.500000) {筒};
\node[Onyomi] at (-13.400000, -0.900000) {トウ};
\node[Kunyomi] at (-13.500000, -0.900000) {つつ};
\node[Meaning] at (-13.450000, 0.750000) {cylinder};
\node[Square] at (-11.400000, -1.000000) {};
\node[Kanji] at (-11.400000, -0.500000) {姉};
\node[Onyomi] at (-11.350000, -0.900000) {シ};
\node[Kunyomi] at (-11.450000, -0.900000) {お.ねえ};
\node[Meaning] at (-11.400000, 0.750000) {older sister};
\node[Square] at (-9.350000, -1.000000) {};
\node[Kanji] at (-9.350000, -0.500000) {貸};
\node[Onyomi] at (-9.300000, -0.900000) {タイ};
\node[Kunyomi] at (-9.400000, -0.900000) {か};
\node[Meaning] at (-9.350000, 0.750000) {lend};
\node[Square] at (-7.300000, -1.000000) {};
\node[Kanji] at (-7.300000, -0.500000) {霧};
\node[Onyomi] at (-7.250000, -0.900000) {ム};
\node[Kunyomi] at (-7.350000, -0.900000) {きり};
\node[Meaning] at (-7.300000, 0.750000) {fog};
\node[Square] at (-5.250000, -1.000000) {};
\node[Kanji] at (-5.250000, -0.500000) {尽};
\node[Onyomi] at (-5.200000, -0.900000) {ジン};
\node[Kunyomi] at (-5.300000, -0.900000) {つ.くす};
\node[Meaning] at (-5.250000, 0.750000) {exhaust};
\node[Square] at (-3.200000, -1.000000) {};
\node[Kanji] at (-3.200000, -0.500000) {堂};
\node[Onyomi] at (-3.150000, -0.900000) {ドウ};
\node[Meaning] at (-3.200000, 0.750000) {hall};
\node[Square] at (-1.150000, -1.000000) {};
\node[Kanji] at (-1.150000, -0.500000) {拠};
\node[Onyomi] at (-1.100000, -0.900000) {キョ};
\node[Kunyomi] at (-1.200000, -0.900000) {よ.る};
\node[Meaning] at (-1.150000, 0.750000) {based on};
\node[Square] at (0.900000, -1.000000) {};
\node[Kanji] at (0.900000, -0.500000) {揚};
\node[Onyomi] at (0.950000, -0.900000) {ヨウ};
\node[Kunyomi] at (0.850000, -0.900000) {あげ};
\node[Meaning] at (0.900000, 0.750000) {hoist};
\node[Square] at (2.950000, -1.000000) {};
\node[Kanji] at (2.950000, -0.500000) {等};
\node[Onyomi] at (3.000000, -0.900000) {トウ};
\node[Kunyomi] at (2.900000, -0.900000) {ひと.しい};
\node[Meaning] at (2.950000, 0.750000) {equal};
\node[Square] at (5.000000, -1.000000) {};
\node[Kanji] at (5.000000, -0.500000) {梨};
\node[Kunyomi] at (4.950000, -0.900000) {なし};
\node[Meaning] at (5.000000, 0.750000) {pear};
\node[Square] at (7.050000, -1.000000) {};
\node[Kanji] at (7.050000, -0.500000) {畑};
\node[Kunyomi] at (7.000000, -0.900000) {はたけ};
\node[Meaning] at (7.050000, 0.750000) {field};
\node[Square] at (9.100000, -1.000000) {};
\node[Kanji] at (9.100000, -0.500000) {寂};
\node[Onyomi] at (9.150000, -0.900000) {ジャク};
\node[Kunyomi] at (9.050000, -0.900000) {さび};
\node[Meaning] at (9.100000, 0.750000) {lonely};
\node[Square] at (11.150000, -1.000000) {};
\node[Kanji] at (11.150000, -0.500000) {超};
\node[Onyomi] at (11.200000, -0.900000) {チョウ};
\node[Kunyomi] at (11.100000, -0.900000) {こ.*};
\node[Meaning] at (11.150000, 0.750000) {ultra};
\node[Square] at (13.200000, -1.000000) {};
\node[Kanji] at (13.200000, -0.500000) {兆};
\node[Onyomi] at (13.250000, -0.900000) {チョウ};
\node[Meaning] at (13.200000, 0.750000) {omen};
\node[Square] at (15.250000, -1.000000) {};
\node[Kanji] at (15.250000, -0.500000) {翼};
\node[Onyomi] at (15.300000, -0.900000) {ヨク};
\node[Kunyomi] at (15.200000, -0.900000) {つばさ};
\node[Meaning] at (15.250000, 0.750000) {wing};
\node[Square] at (17.300000, -1.000000) {};
\node[Kanji] at (17.300000, -0.500000) {胞};
\node[Onyomi] at (17.350000, -0.900000) {ホウ};
\node[Meaning] at (17.300000, 0.750000) {cell};
\node[Square] at (19.350000, -1.000000) {};
\node[Kanji] at (19.350000, -0.500000) {愚};
\node[Onyomi] at (19.400000, -0.900000) {グ};
\node[Kunyomi] at (19.300000, -0.900000) {おろ};
\node[Meaning] at (19.350000, 0.750000) {foolish};
\node[Square] at (21.400000, -1.000000) {};
\node[Kanji] at (21.400000, -0.500000) {腫};
\node[Onyomi] at (21.450000, -0.900000) {シュ};
\node[Kunyomi] at (21.350000, -0.900000) {は-れる};
\node[Meaning] at (21.400000, 0.750000) {tumor};
\node[Square] at (23.450000, -1.000000) {};
\node[Kanji] at (23.450000, -0.500000) {義};
\node[Onyomi] at (23.500000, -0.900000) {ギ};
\node[Meaning] at (23.450000, 0.750000) {righteousness};
\node[Square] at (25.500000, -1.000000) {};
\node[Kanji] at (25.500000, -0.500000) {損};
\node[Onyomi] at (25.550000, -0.900000) {ソン};
\node[Kunyomi] at (25.450000, -0.900000) {そこ.なう};
\node[Meaning] at (25.500000, 0.750000) {loss};
\node[Square] at (27.550000, -1.000000) {};
\node[Kanji] at (27.550000, -0.500000) {聴};
\node[Onyomi] at (27.600000, -0.900000) {チョウ};
\node[Kunyomi] at (27.500000, -0.900000) {き.く};
\node[Meaning] at (27.550000, 0.750000) {listen};
\node[Square] at (29.600000, -1.000000) {};
\node[Kanji] at (29.600000, -0.500000) {珍};
\node[Onyomi] at (29.650000, -0.900000) {チン};
\node[Kunyomi] at (29.550000, -0.900000) {めずら.しい};
\node[Meaning] at (29.600000, 0.750000) {rare};
\node[Square] at (31.650000, -1.000000) {};
\node[Kanji] at (31.650000, -0.500000) {卒};
\node[Onyomi] at (31.700000, -0.900000) {ソツ};
\node[Meaning] at (31.650000, 0.750000) {graduate};
\node[Square] at (33.700000, -1.000000) {};
\node[Kanji] at (33.700000, -0.500000) {弁};
\node[Onyomi] at (33.750000, -0.900000) {ベン};
\node[Meaning] at (33.700000, 0.750000) {dialect};
\node[Square] at (35.750000, -1.000000) {};
\node[Kanji] at (35.750000, -0.500000) {虚};
\node[Onyomi] at (35.800000, -0.900000) {キョ};
\node[Kunyomi] at (35.700000, -0.900000) {むな.しい	};
\node[Meaning] at (35.750000, 0.750000) {void};
\node[Square] at (37.800000, -1.000000) {};
\node[Kanji] at (37.800000, -0.500000) {句};
\node[Onyomi] at (37.850000, -0.900000) {ク};
\node[Meaning] at (37.800000, 0.750000) {paragraph};
\node[Square] at (39.850000, -1.000000) {};
\node[Kanji] at (39.850000, -0.500000) {烈};
\node[Onyomi] at (39.900000, -0.900000) {レツ};
\node[Kunyomi] at (39.800000, -0.900000) {はげ.しい};
\node[Meaning] at (39.850000, 0.750000) {violent};
\node[Square] at (41.900000, -1.000000) {};
\node[Kanji] at (41.900000, -0.500000) {嘆};
\node[Onyomi] at (41.950000, -0.900000) {タン};
\node[Kunyomi] at (41.850000, -0.900000) {なげ.く};
\node[Meaning] at (41.900000, 0.750000) {sigh};
\node[Square] at (43.950000, -1.000000) {};
\node[Kanji] at (43.950000, -0.500000) {遭};
\node[Onyomi] at (44.000000, -0.900000) {ソウ};
\node[Kunyomi] at (43.900000, -0.900000) {あ};
\node[Meaning] at (43.950000, 0.750000) {encounter};
\node[Square] at (46.000000, -1.000000) {};
\node[Kanji] at (46.000000, -0.500000) {嘲};
\node[Onyomi] at (46.050000, -0.900000) {チョウ};
\node[Kunyomi] at (45.950000, -0.900000) {あざけ-る};
\node[Meaning] at (46.000000, 0.750000) {ridicule};
\node[Square] at (48.050000, -1.000000) {};
\node[Kanji] at (48.050000, -0.500000) {矢};
\node[Onyomi] at (48.100000, -0.900000) {シ};
\node[Kunyomi] at (48.000000, -0.900000) {や};
\node[Meaning] at (48.050000, 0.750000) {arrow};
\node[Square] at (50.100000, -1.000000) {};
\node[Kanji] at (50.100000, -0.500000) {溶};
\node[Onyomi] at (50.150000, -0.900000) {ヨウ};
\node[Kunyomi] at (50.050000, -0.900000) {と.ける};
\node[Meaning] at (50.100000, 0.750000) {melt};
\node[Square] at (52.150000, -1.000000) {};
\node[Kanji] at (52.150000, -0.500000) {閣};
\node[Onyomi] at (52.200000, -0.900000) {カク};
\node[Meaning] at (52.150000, 0.750000) {the cabinet};
\node[Square] at (54.200000, -1.000000) {};
\node[Kanji] at (54.200000, -0.500000) {延};
\node[Onyomi] at (54.250000, -0.900000) {エン};
\node[Kunyomi] at (54.150000, -0.900000) {のば.す};
\node[Meaning] at (54.200000, 0.750000) {prolong};
\node[Square] at (56.250000, -1.000000) {};
\node[Kanji] at (56.250000, -0.500000) {整};
\node[Onyomi] at (56.300000, -0.900000) {セイ};
\node[Kunyomi] at (56.200000, -0.900000) {ととの.*};
\node[Meaning] at (56.250000, 0.750000) {arrange};
\node[Meaning] at (-58.500000, -0.450000) {95.32\%};
\node[Square] at (-56.500000, -3.050000) {};
\node[Kanji] at (-56.500000, -2.550000) {懸};
\node[Onyomi] at (-56.450000, -2.950000) {ケン};
\node[Kunyomi] at (-56.550000, -2.950000) {か.*};
\node[Meaning] at (-56.500000, -1.300000) {suspend};
\node[Square] at (-54.450000, -3.050000) {};
\node[Kanji] at (-54.450000, -2.550000) {硬};
\node[Onyomi] at (-54.400000, -2.950000) {コウ};
\node[Kunyomi] at (-54.500000, -2.950000) {かた.い};
\node[Meaning] at (-54.450000, -1.300000) {stiff};
\node[Square] at (-52.400000, -3.050000) {};
\node[Kanji] at (-52.400000, -2.550000) {佐};
\node[Onyomi] at (-52.350000, -2.950000) {サ};
\node[Meaning] at (-52.400000, -1.300000) {help};
\node[Square] at (-50.350000, -3.050000) {};
\node[Kanji] at (-50.350000, -2.550000) {舌};
\node[Onyomi] at (-50.300000, -2.950000) {ゼツ};
\node[Kunyomi] at (-50.400000, -2.950000) {した};
\node[Meaning] at (-50.350000, -1.300000) {tongue};
\node[Square] at (-48.300000, -3.050000) {};
\node[Kanji] at (-48.300000, -2.550000) {創};
\node[Onyomi] at (-48.250000, -2.950000) {ソウ};
\node[Meaning] at (-48.300000, -1.300000) {create};
\node[Square] at (-46.250000, -3.050000) {};
\node[Kanji] at (-46.250000, -2.550000) {程};
\node[Onyomi] at (-46.200000, -2.950000) {テイ};
\node[Kunyomi] at (-46.300000, -2.950000) {ほど};
\node[Meaning] at (-46.250000, -1.300000) {extent};
\node[Square] at (-44.200000, -3.050000) {};
\node[Kanji] at (-44.200000, -2.550000) {塊};
\node[Onyomi] at (-44.150000, -2.950000) {カイ};
\node[Kunyomi] at (-44.250000, -2.950000) {かたまり};
\node[Meaning] at (-44.200000, -1.300000) {lump};
\node[Square] at (-42.150000, -3.050000) {};
\node[Kanji] at (-42.150000, -2.550000) {膚};
\node[Onyomi] at (-42.100000, -2.950000) {フ};
\node[Kunyomi] at (-42.200000, -2.950000) {はだ};
\node[Meaning] at (-42.150000, -1.300000) {skin};
\node[Square] at (-40.100000, -3.050000) {};
\node[Kanji] at (-40.100000, -2.550000) {就};
\node[Onyomi] at (-40.050000, -2.950000) {シュウ};
\node[Kunyomi] at (-40.150000, -2.950000) {つ.く};
\node[Meaning] at (-40.100000, -1.300000) {settle in};
\node[Square] at (-38.050000, -3.050000) {};
\node[Kanji] at (-38.050000, -2.550000) {糖};
\node[Onyomi] at (-38.000000, -2.950000) {トウ};
\node[Meaning] at (-38.050000, -1.300000) {sugar};
\node[Square] at (-36.000000, -3.050000) {};
\node[Kanji] at (-36.000000, -2.550000) {滴};
\node[Onyomi] at (-35.950000, -2.950000) {テキ};
\node[Kunyomi] at (-36.050000, -2.950000) {したた.る};
\node[Meaning] at (-36.000000, -1.300000) {drip};
\node[Square] at (-33.950000, -3.050000) {};
\node[Kanji] at (-33.950000, -2.550000) {紀};
\node[Onyomi] at (-33.900000, -2.950000) {キ};
\node[Meaning] at (-33.950000, -1.300000) {account};
\node[Square] at (-31.900000, -3.050000) {};
\node[Kanji] at (-31.900000, -2.550000) {仏};
\node[Onyomi] at (-31.850000, -2.950000) {ブツ};
\node[Kunyomi] at (-31.950000, -2.950000) {ほとけ};
\node[Meaning] at (-31.900000, -1.300000) {buddha};
\node[Square] at (-29.850000, -3.050000) {};
\node[Kanji] at (-29.850000, -2.550000) {豚};
\node[Onyomi] at (-29.800000, -2.950000) {トン};
\node[Kunyomi] at (-29.900000, -2.950000) {ぶた};
\node[Meaning] at (-29.850000, -1.300000) {pork};
\node[Square] at (-27.800000, -3.050000) {};
\node[Kanji] at (-27.800000, -2.550000) {慮};
\node[Onyomi] at (-27.750000, -2.950000) {リョ};
\node[Kunyomi] at (-27.850000, -2.950000) {おもんぱく};
\node[Meaning] at (-27.800000, -1.300000) {consider};
\node[Square] at (-25.750000, -3.050000) {};
\node[Kanji] at (-25.750000, -2.550000) {磨};
\node[Onyomi] at (-25.700000, -2.950000) {マ};
\node[Kunyomi] at (-25.800000, -2.950000) {みが};
\node[Meaning] at (-25.750000, -1.300000) {polish};
\node[Square] at (-23.700000, -3.050000) {};
\node[Kanji] at (-23.700000, -2.550000) {壇};
\node[Onyomi] at (-23.650000, -2.950000) {ダン};
\node[Meaning] at (-23.700000, -1.300000) {podium};
\node[Square] at (-21.650000, -3.050000) {};
\node[Kanji] at (-21.650000, -2.550000) {偉};
\node[Onyomi] at (-21.600000, -2.950000) {イ};
\node[Kunyomi] at (-21.700000, -2.950000) {えら};
\node[Meaning] at (-21.650000, -1.300000) {greatness};
\node[Square] at (-19.600000, -3.050000) {};
\node[Kanji] at (-19.600000, -2.550000) {湧};
\node[Onyomi] at (-19.550000, -2.950000) {ユウ};
\node[Kunyomi] at (-19.650000, -2.950000) {わ};
\node[Meaning] at (-19.600000, -1.300000) {well};
\node[Square] at (-17.550000, -3.050000) {};
\node[Kanji] at (-17.550000, -2.550000) {看};
\node[Onyomi] at (-17.500000, -2.950000) {カン};
\node[Meaning] at (-17.550000, -1.300000) {watch over};
\node[Square] at (-15.500000, -3.050000) {};
\node[Kanji] at (-15.500000, -2.550000) {絞};
\node[Onyomi] at (-15.450000, -2.950000) {コウ};
\node[Kunyomi] at (-15.550000, -2.950000) {し};
\node[Meaning] at (-15.500000, -1.300000) {strangle};
\node[Square] at (-13.450000, -3.050000) {};
\node[Kanji] at (-13.450000, -2.550000) {酸};
\node[Onyomi] at (-13.400000, -2.950000) {サン};
\node[Kunyomi] at (-13.500000, -2.950000) {す};
\node[Meaning] at (-13.450000, -1.300000) {acid};
\node[Square] at (-11.400000, -3.050000) {};
\node[Kanji] at (-11.400000, -2.550000) {肌};
\node[Kunyomi] at (-11.450000, -2.950000) {はだ};
\node[Meaning] at (-11.400000, -1.300000) {skin};
\node[Square] at (-9.350000, -3.050000) {};
\node[Kanji] at (-9.350000, -2.550000) {股};
\node[Onyomi] at (-9.300000, -2.950000) {コ};
\node[Kunyomi] at (-9.400000, -2.950000) {また};
\node[Meaning] at (-9.350000, -1.300000) {crotch};
\node[Square] at (-7.300000, -3.050000) {};
\node[Kanji] at (-7.300000, -2.550000) {干};
\node[Onyomi] at (-7.250000, -2.950000) {カン};
\node[Kunyomi] at (-7.350000, -2.950000) {ほ.す};
\node[Meaning] at (-7.300000, -1.300000) {dry};
\node[Square] at (-5.250000, -3.050000) {};
\node[Kanji] at (-5.250000, -2.550000) {率};
\node[Onyomi] at (-5.200000, -2.950000) {リツ};
\node[Kunyomi] at (-5.300000, -2.950000) {ひき.いる};
\node[Meaning] at (-5.250000, -1.300000) {percent};
\node[Square] at (-3.200000, -3.050000) {};
\node[Kanji] at (-3.200000, -2.550000) {推};
\node[Onyomi] at (-3.150000, -2.950000) {スイ};
\node[Kunyomi] at (-3.250000, -2.950000) {お.す};
\node[Meaning] at (-3.200000, -1.300000) {infer};
\node[Square] at (-1.150000, -3.050000) {};
\node[Kanji] at (-1.150000, -2.550000) {遣};
\node[Onyomi] at (-1.100000, -2.950000) {ケン};
\node[Kunyomi] at (-1.200000, -2.950000) {つか.う};
\node[Meaning] at (-1.150000, -1.300000) {dispatch};
\node[Square] at (0.900000, -3.050000) {};
\node[Kanji] at (0.900000, -2.550000) {肝};
\node[Onyomi] at (0.950000, -2.950000) {カン};
\node[Kunyomi] at (0.850000, -2.950000) {きも};
\node[Meaning] at (0.900000, -1.300000) {liver};
\node[Square] at (2.950000, -3.050000) {};
\node[Kanji] at (2.950000, -2.550000) {共};
\node[Onyomi] at (3.000000, -2.950000) {キョウ};
\node[Kunyomi] at (2.900000, -2.950000) {とも};
\node[Meaning] at (2.950000, -1.300000) {together};
\node[Square] at (5.000000, -3.050000) {};
\node[Kanji] at (5.000000, -2.550000) {獄};
\node[Onyomi] at (5.050000, -2.950000) {ゴク};
\node[Meaning] at (5.000000, -1.300000) {prison};
\node[Square] at (7.050000, -3.050000) {};
\node[Kanji] at (7.050000, -2.550000) {威};
\node[Onyomi] at (7.100000, -2.950000) {イ};
\node[Meaning] at (7.050000, -1.300000) {majesty};
\node[Square] at (9.100000, -3.050000) {};
\node[Kanji] at (9.100000, -2.550000) {凝};
\node[Onyomi] at (9.150000, -2.950000) {ギョウ};
\node[Kunyomi] at (9.050000, -2.950000) {こ};
\node[Meaning] at (9.100000, -1.300000) {congeal};
\node[Square] at (11.150000, -3.050000) {};
\node[Kanji] at (11.150000, -2.550000) {袖};
\node[Onyomi] at (11.200000, -2.950000) {シュウ};
\node[Kunyomi] at (11.100000, -2.950000) {そで};
\node[Meaning] at (11.150000, -1.300000) {sleeve};
\node[Square] at (13.200000, -3.050000) {};
\node[Kanji] at (13.200000, -2.550000) {林};
\node[Onyomi] at (13.250000, -2.950000) {リン};
\node[Kunyomi] at (13.150000, -2.950000) {はやし};
\node[Meaning] at (13.200000, -1.300000) {forest};
\node[Square] at (15.250000, -3.050000) {};
\node[Kanji] at (15.250000, -2.550000) {賀};
\node[Onyomi] at (15.300000, -2.950000) {ガ};
\node[Meaning] at (15.250000, -1.300000) {congratulate};
\node[Square] at (17.300000, -3.050000) {};
\node[Kanji] at (17.300000, -2.550000) {更};
\node[Onyomi] at (17.350000, -2.950000) {コウ};
\node[Kunyomi] at (17.250000, -2.950000) {さら};
\node[Meaning] at (17.300000, -1.300000) {again};
\node[Square] at (19.350000, -3.050000) {};
\node[Kanji] at (19.350000, -2.550000) {柔};
\node[Onyomi] at (19.400000, -2.950000) {ジュウ};
\node[Kunyomi] at (19.300000, -2.950000) {やわ.*};
\node[Meaning] at (19.350000, -1.300000) {gentle};
\node[Square] at (21.400000, -3.050000) {};
\node[Kanji] at (21.400000, -2.550000) {才};
\node[Onyomi] at (21.450000, -2.950000) {サイ};
\node[Meaning] at (21.400000, -1.300000) {genius};
\node[Square] at (23.450000, -3.050000) {};
\node[Kanji] at (23.450000, -2.550000) {養};
\node[Onyomi] at (23.500000, -2.950000) {ヨウ};
\node[Kunyomi] at (23.400000, -2.950000) {やしな.う};
\node[Meaning] at (23.450000, -1.300000) {foster};
\node[Square] at (25.500000, -3.050000) {};
\node[Kanji] at (25.500000, -2.550000) {僧};
\node[Onyomi] at (25.550000, -2.950000) {ソウ};
\node[Meaning] at (25.500000, -1.300000) {priest};
\node[Square] at (27.550000, -3.050000) {};
\node[Kanji] at (27.550000, -2.550000) {順};
\node[Onyomi] at (27.600000, -2.950000) {ジュン};
\node[Meaning] at (27.550000, -1.300000) {order};
\node[Square] at (29.600000, -3.050000) {};
\node[Kanji] at (29.600000, -2.550000) {埼};
\node[Onyomi] at (29.650000, -2.950000) {キ};
\node[Kunyomi] at (29.550000, -2.950000) {さい};
\node[Meaning] at (29.600000, -1.300000) {cape};
\node[Square] at (31.650000, -3.050000) {};
\node[Kanji] at (31.650000, -2.550000) {那};
\node[Onyomi] at (31.700000, -2.950000) {ナ};
\node[Kunyomi] at (31.600000, -2.950000) {いかん};
\node[Meaning] at (31.650000, -1.300000) {what};
\node[Square] at (33.700000, -3.050000) {};
\node[Kanji] at (33.700000, -2.550000) {剥};
\node[Onyomi] at (33.750000, -2.950000) {ハク};
\node[Kunyomi] at (33.650000, -2.950000) {は-がす};
\node[Meaning] at (33.700000, -1.300000) {peel};
\node[Square] at (35.750000, -3.050000) {};
\node[Kanji] at (35.750000, -2.550000) {茂};
\node[Onyomi] at (35.800000, -2.950000) {モ};
\node[Kunyomi] at (35.700000, -2.950000) {しげ.る};
\node[Meaning] at (35.750000, -1.300000) {luxuriant};
\node[Square] at (37.800000, -3.050000) {};
\node[Kanji] at (37.800000, -2.550000) {即};
\node[Onyomi] at (37.850000, -2.950000) {ソク};
\node[Kunyomi] at (37.750000, -2.950000) {すなわ.ち};
\node[Meaning] at (37.800000, -1.300000) {instant};
\node[Square] at (39.850000, -3.050000) {};
\node[Kanji] at (39.850000, -2.550000) {殴};
\node[Onyomi] at (39.900000, -2.950000) {オウ};
\node[Kunyomi] at (39.800000, -2.950000) {なぐ.る};
\node[Meaning] at (39.850000, -1.300000) {assault};
\node[Square] at (41.900000, -3.050000) {};
\node[Kanji] at (41.900000, -2.550000) {斜};
\node[Onyomi] at (41.950000, -2.950000) {シャ};
\node[Kunyomi] at (41.850000, -2.950000) {なな.め};
\node[Meaning] at (41.900000, -1.300000) {diagonal};
\node[Square] at (43.950000, -3.050000) {};
\node[Kanji] at (43.950000, -2.550000) {慨};
\node[Onyomi] at (44.000000, -2.950000) {ガイ};
\node[Meaning] at (43.950000, -1.300000) {sigh};
\node[Square] at (46.000000, -3.050000) {};
\node[Kanji] at (46.000000, -2.550000) {層};
\node[Onyomi] at (46.050000, -2.950000) {ソウ};
\node[Meaning] at (46.000000, -1.300000) {layer};
\node[Square] at (48.050000, -3.050000) {};
\node[Kanji] at (48.050000, -2.550000) {促};
\node[Onyomi] at (48.100000, -2.950000) {ソク};
\node[Kunyomi] at (48.000000, -2.950000) {うなが.す};
\node[Meaning] at (48.050000, -1.300000) {urge};
\node[Square] at (50.100000, -3.050000) {};
\node[Kanji] at (50.100000, -2.550000) {翌};
\node[Onyomi] at (50.150000, -2.950000) {ヨク};
\node[Meaning] at (50.100000, -1.300000) {the following};
\node[Square] at (52.150000, -3.050000) {};
\node[Kanji] at (52.150000, -2.550000) {渋};
\node[Onyomi] at (52.200000, -2.950000) {ジュウ};
\node[Kunyomi] at (52.100000, -2.950000) {しぶ.い};
\node[Meaning] at (52.150000, -1.300000) {bitter};
\node[Square] at (54.200000, -3.050000) {};
\node[Kanji] at (54.200000, -2.550000) {渦};
\node[Onyomi] at (54.250000, -2.950000) {カ};
\node[Kunyomi] at (54.150000, -2.950000) {うず};
\node[Meaning] at (54.200000, -1.300000) {whirlpool};
\node[Square] at (56.250000, -3.050000) {};
\node[Kanji] at (56.250000, -2.550000) {戯};
\node[Onyomi] at (56.300000, -2.950000) {ギ};
\node[Kunyomi] at (56.200000, -2.950000) {ざ};
\node[Meaning] at (56.250000, -1.300000) {play};
\node[Meaning] at (-58.500000, -2.500000) {95.80\%};
\node[Square] at (-56.500000, -5.100000) {};
\node[Kanji] at (-56.500000, -4.600000) {革};
\node[Onyomi] at (-56.450000, -5.000000) {カク};
\node[Kunyomi] at (-56.550000, -5.000000) {かわ};
\node[Meaning] at (-56.500000, -3.350000) {leather};
\node[Square] at (-54.450000, -5.100000) {};
\node[Kanji] at (-54.450000, -4.600000) {巣};
\node[Onyomi] at (-54.400000, -5.000000) {ソウ};
\node[Kunyomi] at (-54.500000, -5.000000) {す};
\node[Meaning] at (-54.450000, -3.350000) {nest};
\node[Square] at (-52.400000, -5.100000) {};
\node[Kanji] at (-52.400000, -4.600000) {績};
\node[Onyomi] at (-52.350000, -5.000000) {セキ};
\node[Meaning] at (-52.400000, -3.350000) {exploits};
\node[Square] at (-50.350000, -5.100000) {};
\node[Kanji] at (-50.350000, -4.600000) {昭};
\node[Onyomi] at (-50.300000, -5.000000) {ショウ};
\node[Meaning] at (-50.350000, -3.350000) {shining};
\node[Square] at (-48.300000, -5.100000) {};
\node[Kanji] at (-48.300000, -4.600000) {銃};
\node[Onyomi] at (-48.250000, -5.000000) {ジュウ};
\node[Meaning] at (-48.300000, -3.350000) {gun};
\node[Square] at (-46.250000, -5.100000) {};
\node[Kanji] at (-46.250000, -4.600000) {遮};
\node[Onyomi] at (-46.200000, -5.000000) {シャ};
\node[Kunyomi] at (-46.300000, -5.000000) {さえぎ};
\node[Meaning] at (-46.250000, -3.350000) {intercept};
\node[Square] at (-44.200000, -5.100000) {};
\node[Kanji] at (-44.200000, -4.600000) {泉};
\node[Onyomi] at (-44.150000, -5.000000) {セン};
\node[Kunyomi] at (-44.250000, -5.000000) {いずみ};
\node[Meaning] at (-44.200000, -3.350000) {spring};
\node[Square] at (-42.150000, -5.100000) {};
\node[Kanji] at (-42.150000, -4.600000) {築};
\node[Onyomi] at (-42.100000, -5.000000) {チク};
\node[Kunyomi] at (-42.200000, -5.000000) {きず.く};
\node[Meaning] at (-42.150000, -3.350000) {construct};
\node[Square] at (-40.100000, -5.100000) {};
\node[Kanji] at (-40.100000, -4.600000) {栄};
\node[Onyomi] at (-40.050000, -5.000000) {エイ};
\node[Kunyomi] at (-40.150000, -5.000000) {さか.える};
\node[Meaning] at (-40.100000, -3.350000) {prosperity};
\node[Square] at (-38.050000, -5.100000) {};
\node[Kanji] at (-38.050000, -4.600000) {評};
\node[Onyomi] at (-38.000000, -5.000000) {ヒョウ};
\node[Meaning] at (-38.050000, -3.350000) {evaluate};
\node[Square] at (-36.000000, -5.100000) {};
\node[Kanji] at (-36.000000, -4.600000) {挑};
\node[Onyomi] at (-35.950000, -5.000000) {チョウ};
\node[Kunyomi] at (-36.050000, -5.000000) {いど.む};
\node[Meaning] at (-36.000000, -3.350000) {challenge};
\node[Square] at (-33.950000, -5.100000) {};
\node[Kanji] at (-33.950000, -4.600000) {洞};
\node[Onyomi] at (-33.900000, -5.000000) {ドウ};
\node[Kunyomi] at (-34.000000, -5.000000) {ほら};
\node[Meaning] at (-33.950000, -3.350000) {cave};
\node[Square] at (-31.900000, -5.100000) {};
\node[Kanji] at (-31.900000, -4.600000) {蓋};
\node[Onyomi] at (-31.850000, -5.000000) {ガイ};
\node[Kunyomi] at (-31.950000, -5.000000) {ふた};
\node[Meaning] at (-31.900000, -3.350000) {cover};
\node[Square] at (-29.850000, -5.100000) {};
\node[Kanji] at (-29.850000, -4.600000) {寺};
\node[Onyomi] at (-29.800000, -5.000000) {ジ};
\node[Kunyomi] at (-29.900000, -5.000000) {てら};
\node[Meaning] at (-29.850000, -3.350000) {temple};
\node[Square] at (-27.800000, -5.100000) {};
\node[Kanji] at (-27.800000, -4.600000) {易};
\node[Onyomi] at (-27.750000, -5.000000) {イ};
\node[Kunyomi] at (-27.850000, -5.000000) {やさ.しい};
\node[Meaning] at (-27.800000, -3.350000) {easy};
\node[Square] at (-25.750000, -5.100000) {};
\node[Kanji] at (-25.750000, -4.600000) {針};
\node[Onyomi] at (-25.700000, -5.000000) {シン};
\node[Kunyomi] at (-25.800000, -5.000000) {はり};
\node[Meaning] at (-25.750000, -3.350000) {needle};
\node[Square] at (-23.700000, -5.100000) {};
\node[Kanji] at (-23.700000, -4.600000) {菓};
\node[Onyomi] at (-23.650000, -5.000000) {カ};
\node[Meaning] at (-23.700000, -3.350000) {cake};
\node[Square] at (-21.650000, -5.100000) {};
\node[Kanji] at (-21.650000, -4.600000) {腐};
\node[Onyomi] at (-21.600000, -5.000000) {フ};
\node[Kunyomi] at (-21.700000, -5.000000) {くさ.る};
\node[Meaning] at (-21.650000, -3.350000) {rot};
\node[Square] at (-19.600000, -5.100000) {};
\node[Kanji] at (-19.600000, -4.600000) {粒};
\node[Onyomi] at (-19.550000, -5.000000) {リュウ};
\node[Kunyomi] at (-19.650000, -5.000000) {つぶ};
\node[Meaning] at (-19.600000, -3.350000) {grains};
\node[Square] at (-17.550000, -5.100000) {};
\node[Kanji] at (-17.550000, -4.600000) {匂};
\node[Kunyomi] at (-17.600000, -5.000000) {にお-う};
\node[Meaning] at (-17.550000, -3.350000) {scent};
\node[Square] at (-15.500000, -5.100000) {};
\node[Kanji] at (-15.500000, -4.600000) {捜};
\node[Onyomi] at (-15.450000, -5.000000) {ソウ};
\node[Kunyomi] at (-15.550000, -5.000000) {さが.す};
\node[Meaning] at (-15.500000, -3.350000) {search};
\node[Square] at (-13.450000, -5.100000) {};
\node[Kanji] at (-13.450000, -4.600000) {添};
\node[Onyomi] at (-13.400000, -5.000000) {テン};
\node[Kunyomi] at (-13.500000, -5.000000) {そ.える};
\node[Meaning] at (-13.450000, -3.350000) {append};
\node[Square] at (-11.400000, -5.100000) {};
\node[Kanji] at (-11.400000, -4.600000) {牙};
\node[Onyomi] at (-11.350000, -5.000000) {ゲ};
\node[Kunyomi] at (-11.450000, -5.000000) {きば};
\node[Meaning] at (-11.400000, -3.350000) {fang};
\node[Square] at (-9.350000, -5.100000) {};
\node[Kanji] at (-9.350000, -4.600000) {扱};
\node[Onyomi] at (-9.300000, -5.000000) {キュウ};
\node[Kunyomi] at (-9.400000, -5.000000) {あつか};
\node[Meaning] at (-9.350000, -3.350000) {handle};
\node[Square] at (-7.300000, -5.100000) {};
\node[Kanji] at (-7.300000, -4.600000) {戒};
\node[Onyomi] at (-7.250000, -5.000000) {カイ};
\node[Kunyomi] at (-7.350000, -5.000000) {いまし.める};
\node[Meaning] at (-7.300000, -3.350000) {commandment};
\node[Square] at (-5.250000, -5.100000) {};
\node[Kanji] at (-5.250000, -4.600000) {釣};
\node[Onyomi] at (-5.200000, -5.000000) {チョウ};
\node[Kunyomi] at (-5.300000, -5.000000) {つ};
\node[Meaning] at (-5.250000, -3.350000) {fishing};
\node[Square] at (-3.200000, -5.100000) {};
\node[Kanji] at (-3.200000, -4.600000) {浸};
\node[Onyomi] at (-3.150000, -5.000000) {シン};
\node[Kunyomi] at (-3.250000, -5.000000) {ひた.*};
\node[Meaning] at (-3.200000, -3.350000) {immersed};
\node[Square] at (-1.150000, -5.100000) {};
\node[Kanji] at (-1.150000, -4.600000) {柵};
\node[Onyomi] at (-1.100000, -5.000000) {サク};
\node[Meaning] at (-1.150000, -3.350000) {fence};
\node[Square] at (0.900000, -5.100000) {};
\node[Kanji] at (0.900000, -4.600000) {抵};
\node[Onyomi] at (0.950000, -5.000000) {テイ};
\node[Meaning] at (0.900000, -3.350000) {resist};
\node[Square] at (2.950000, -5.100000) {};
\node[Kanji] at (2.950000, -4.600000) {勧};
\node[Onyomi] at (3.000000, -5.000000) {カン};
\node[Kunyomi] at (2.900000, -5.000000) {すす.める};
\node[Meaning] at (2.950000, -3.350000) {recommend};
\node[Square] at (5.000000, -5.100000) {};
\node[Kanji] at (5.000000, -4.600000) {嗅};
\node[Onyomi] at (5.050000, -5.000000) {キュウ};
\node[Kunyomi] at (4.950000, -5.000000) {か-ぐ};
\node[Meaning] at (5.000000, -3.350000) {smell};
\node[Square] at (7.050000, -5.100000) {};
\node[Kanji] at (7.050000, -4.600000) {秀};
\node[Onyomi] at (7.100000, -5.000000) {シュウ};
\node[Kunyomi] at (7.000000, -5.000000) {ひい.でる};
\node[Meaning] at (7.050000, -3.350000) {excel};
\node[Square] at (9.100000, -5.100000) {};
\node[Kanji] at (9.100000, -4.600000) {励};
\node[Onyomi] at (9.150000, -5.000000) {レイ};
\node[Kunyomi] at (9.050000, -5.000000) {はげ.*};
\node[Meaning] at (9.100000, -3.350000) {encourage};
\node[Square] at (11.150000, -5.100000) {};
\node[Kanji] at (11.150000, -4.600000) {請};
\node[Onyomi] at (11.200000, -5.000000) {セイ};
\node[Kunyomi] at (11.100000, -5.000000) {う.ける};
\node[Meaning] at (11.150000, -3.350000) {request};
\node[Square] at (13.200000, -5.100000) {};
\node[Kanji] at (13.200000, -4.600000) {倉};
\node[Onyomi] at (13.250000, -5.000000) {ソウ};
\node[Kunyomi] at (13.150000, -5.000000) {くら};
\node[Meaning] at (13.200000, -3.350000) {warehouse};
\node[Square] at (15.250000, -5.100000) {};
\node[Kanji] at (15.250000, -4.600000) {酔};
\node[Onyomi] at (15.300000, -5.000000) {スイ};
\node[Kunyomi] at (15.200000, -5.000000) {よ.う};
\node[Meaning] at (15.250000, -3.350000) {drunk};
\node[Square] at (17.300000, -5.100000) {};
\node[Kanji] at (17.300000, -4.600000) {誓};
\node[Onyomi] at (17.350000, -5.000000) {セイ};
\node[Kunyomi] at (17.250000, -5.000000) {ちか.う};
\node[Meaning] at (17.300000, -3.350000) {vow};
\node[Square] at (19.350000, -5.100000) {};
\node[Kanji] at (19.350000, -4.600000) {筆};
\node[Onyomi] at (19.400000, -5.000000) {ヒツ};
\node[Kunyomi] at (19.300000, -5.000000) {ふで};
\node[Meaning] at (19.350000, -3.350000) {writing brush};
\node[Square] at (21.400000, -5.100000) {};
\node[Kanji] at (21.400000, -4.600000) {複};
\node[Onyomi] at (21.450000, -5.000000) {フク};
\node[Meaning] at (21.400000, -3.350000) {duplicate};
\node[Square] at (23.450000, -5.100000) {};
\node[Kanji] at (23.450000, -4.600000) {舎};
\node[Onyomi] at (23.500000, -5.000000) {シャ};
\node[Meaning] at (23.450000, -3.350000) {cottage};
\node[Square] at (25.500000, -5.100000) {};
\node[Kanji] at (25.500000, -4.600000) {滞};
\node[Onyomi] at (25.550000, -5.000000) {タイ};
\node[Kunyomi] at (25.450000, -5.000000) {とどこお.る};
\node[Meaning] at (25.500000, -3.350000) {stagnate};
\node[Square] at (27.550000, -5.100000) {};
\node[Kanji] at (27.550000, -4.600000) {餌};
\node[Onyomi] at (27.600000, -5.000000) {ジ};
\node[Kunyomi] at (27.500000, -5.000000) {えさ};
\node[Meaning] at (27.550000, -3.350000) {bait};
\node[Square] at (29.600000, -5.100000) {};
\node[Kanji] at (29.600000, -4.600000) {財};
\node[Onyomi] at (29.650000, -5.000000) {サイ};
\node[Meaning] at (29.600000, -3.350000) {wealth};
\node[Square] at (31.650000, -5.100000) {};
\node[Kanji] at (31.650000, -4.600000) {預};
\node[Onyomi] at (31.700000, -5.000000) {ヨ};
\node[Kunyomi] at (31.600000, -5.000000) {あず.ける};
\node[Meaning] at (31.650000, -3.350000) {deposit};
\node[Square] at (33.700000, -5.100000) {};
\node[Kanji] at (33.700000, -4.600000) {源};
\node[Onyomi] at (33.750000, -5.000000) {ゲン};
\node[Kunyomi] at (33.650000, -5.000000) {みなもと};
\node[Meaning] at (33.700000, -3.350000) {origin};
\node[Square] at (35.750000, -5.100000) {};
\node[Kanji] at (35.750000, -4.600000) {葬};
\node[Onyomi] at (35.800000, -5.000000) {ソウ};
\node[Kunyomi] at (35.700000, -5.000000) {ほうむ.る};
\node[Meaning] at (35.750000, -3.350000) {burial};
\node[Square] at (37.800000, -5.100000) {};
\node[Kanji] at (37.800000, -4.600000) {炭};
\node[Onyomi] at (37.850000, -5.000000) {タン};
\node[Kunyomi] at (37.750000, -5.000000) {すみ};
\node[Meaning] at (37.800000, -3.350000) {charcoal};
\node[Square] at (39.850000, -5.100000) {};
\node[Kanji] at (39.850000, -4.600000) {謎};
\node[Kunyomi] at (39.800000, -5.000000) {なぞ};
\node[Meaning] at (39.850000, -3.350000) {riddle};
\node[Square] at (41.900000, -5.100000) {};
\node[Kanji] at (41.900000, -4.600000) {竜};
\node[Onyomi] at (41.950000, -5.000000) {リュウ};
\node[Kunyomi] at (41.850000, -5.000000) {たつ};
\node[Meaning] at (41.900000, -3.350000) {dragon};
\node[Square] at (43.950000, -5.100000) {};
\node[Kanji] at (43.950000, -4.600000) {煮};
\node[Onyomi] at (44.000000, -5.000000) {シャ};
\node[Kunyomi] at (43.900000, -5.000000) {に};
\node[Meaning] at (43.950000, -3.350000) {boil};
\node[Square] at (46.000000, -5.100000) {};
\node[Kanji] at (46.000000, -4.600000) {呂};
\node[Onyomi] at (46.050000, -5.000000) {ロ};
\node[Kunyomi] at (45.950000, -5.000000) {せぼね};
\node[Meaning] at (46.000000, -3.350000) {bath};
\node[Square] at (48.050000, -5.100000) {};
\node[Kanji] at (48.050000, -4.600000) {改};
\node[Onyomi] at (48.100000, -5.000000) {カイ};
\node[Kunyomi] at (48.000000, -5.000000) {あらた.*};
\node[Meaning] at (48.050000, -3.350000) {renew};
\node[Square] at (50.100000, -5.100000) {};
\node[Kanji] at (50.100000, -4.600000) {冊};
\node[Onyomi] at (50.150000, -5.000000) {サツ};
\node[Meaning] at (50.100000, -3.350000) {books counter};
\node[Square] at (52.150000, -5.100000) {};
\node[Kanji] at (52.150000, -4.600000) {条};
\node[Onyomi] at (52.200000, -5.000000) {ジョウ};
\node[Meaning] at (52.150000, -3.350000) {clause};
\node[Square] at (54.200000, -5.100000) {};
\node[Kanji] at (54.200000, -4.600000) {覧};
\node[Onyomi] at (54.250000, -5.000000) {ラン};
\node[Meaning] at (54.200000, -3.350000) {look at};
\node[Square] at (56.250000, -5.100000) {};
\node[Kanji] at (56.250000, -4.600000) {肺};
\node[Onyomi] at (56.300000, -5.000000) {ハイ};
\node[Kunyomi] at (56.200000, -5.000000) {はい};
\node[Meaning] at (56.250000, -3.350000) {lung};
\node[Meaning] at (-58.500000, -4.550000) {96.23\%};
\node[Square] at (-56.500000, -7.150000) {};
\node[Kanji] at (-56.500000, -6.650000) {勤};
\node[Onyomi] at (-56.450000, -7.050000) {キン};
\node[Kunyomi] at (-56.550000, -7.050000) {つと.*};
\node[Meaning] at (-56.500000, -5.400000) {work};
\node[Square] at (-54.450000, -7.150000) {};
\node[Kanji] at (-54.450000, -6.650000) {誉};
\node[Onyomi] at (-54.400000, -7.050000) {ヨ};
\node[Kunyomi] at (-54.500000, -7.050000) {ほ.める};
\node[Meaning] at (-54.450000, -5.400000) {honor};
\node[Square] at (-52.400000, -7.150000) {};
\node[Kanji] at (-52.400000, -6.650000) {網};
\node[Onyomi] at (-52.350000, -7.050000) {モウ};
\node[Kunyomi] at (-52.450000, -7.050000) {あみ};
\node[Meaning] at (-52.400000, -5.400000) {netting};
\node[Square] at (-50.350000, -7.150000) {};
\node[Kanji] at (-50.350000, -6.650000) {慰};
\node[Onyomi] at (-50.300000, -7.050000) {イ};
\node[Kunyomi] at (-50.400000, -7.050000) {なぐさ.*};
\node[Meaning] at (-50.350000, -5.400000) {consolation};
\node[Square] at (-48.300000, -7.150000) {};
\node[Kanji] at (-48.300000, -6.650000) {唯};
\node[Onyomi] at (-48.250000, -7.050000) {ユイ};
\node[Kunyomi] at (-48.350000, -7.050000) {ただ};
\node[Meaning] at (-48.300000, -5.400000) {solely};
\node[Square] at (-46.250000, -7.150000) {};
\node[Kanji] at (-46.250000, -6.650000) {敢};
\node[Onyomi] at (-46.200000, -7.050000) {カン};
\node[Kunyomi] at (-46.300000, -7.050000) {あ};
\node[Meaning] at (-46.250000, -5.400000) {daring};
\node[Square] at (-44.200000, -7.150000) {};
\node[Kanji] at (-44.200000, -6.650000) {米};
\node[Onyomi] at (-44.150000, -7.050000) {ベイ};
\node[Kunyomi] at (-44.250000, -7.050000) { こめ};
\node[Meaning] at (-44.200000, -5.400000) {rice};
\node[Square] at (-42.150000, -7.150000) {};
\node[Kanji] at (-42.150000, -6.650000) {署};
\node[Onyomi] at (-42.100000, -7.050000) {ショ};
\node[Meaning] at (-42.150000, -5.400000) {govt. office};
\node[Square] at (-40.100000, -7.150000) {};
\node[Kanji] at (-40.100000, -6.650000) {善};
\node[Onyomi] at (-40.050000, -7.050000) {ゼン};
\node[Kunyomi] at (-40.150000, -7.050000) {ぜん};
\node[Meaning] at (-40.100000, -5.400000) {morally good};
\node[Square] at (-38.050000, -7.150000) {};
\node[Kanji] at (-38.050000, -6.650000) {編};
\node[Onyomi] at (-38.000000, -7.050000) {ヘン};
\node[Kunyomi] at (-38.100000, -7.050000) {あ.む};
\node[Meaning] at (-38.050000, -5.400000) {knit};
\node[Square] at (-36.000000, -7.150000) {};
\node[Kanji] at (-36.000000, -6.650000) {宣};
\node[Onyomi] at (-35.950000, -7.050000) {セン};
\node[Kunyomi] at (-36.050000, -7.050000) {のたま.う};
\node[Meaning] at (-36.000000, -5.400000) {proclaim};
\node[Square] at (-33.950000, -7.150000) {};
\node[Kanji] at (-33.950000, -6.650000) {劇};
\node[Onyomi] at (-33.900000, -7.050000) {ゲキ};
\node[Meaning] at (-33.950000, -5.400000) {drama};
\node[Square] at (-31.900000, -7.150000) {};
\node[Kanji] at (-31.900000, -6.650000) {執};
\node[Onyomi] at (-31.850000, -7.050000) {シュウ};
\node[Kunyomi] at (-31.950000, -7.050000) {と.る};
\node[Meaning] at (-31.900000, -5.400000) {tenacious};
\node[Square] at (-29.850000, -7.150000) {};
\node[Kanji] at (-29.850000, -6.650000) {稲};
\node[Kunyomi] at (-29.900000, -7.050000) {いね};
\node[Meaning] at (-29.850000, -5.400000) {rice plant};
\node[Square] at (-27.800000, -7.150000) {};
\node[Kanji] at (-27.800000, -6.650000) {至};
\node[Onyomi] at (-27.750000, -7.050000) {シ};
\node[Kunyomi] at (-27.850000, -7.050000) {いた.る};
\node[Meaning] at (-27.800000, -5.400000) {attain};
\node[Square] at (-25.750000, -7.150000) {};
\node[Kanji] at (-25.750000, -6.650000) {偽};
\node[Onyomi] at (-25.700000, -7.050000) {ギ};
\node[Kunyomi] at (-25.800000, -7.050000) {にせ};
\node[Meaning] at (-25.750000, -5.400000) {fake};
\node[Square] at (-23.700000, -7.150000) {};
\node[Kanji] at (-23.700000, -6.650000) {曇};
\node[Kunyomi] at (-23.750000, -7.050000) {くも};
\node[Meaning] at (-23.700000, -5.400000) {cloudy};
\node[Square] at (-21.650000, -7.150000) {};
\node[Kanji] at (-21.650000, -6.650000) {疾};
\node[Onyomi] at (-21.600000, -7.050000) {シツ};
\node[Kunyomi] at (-21.700000, -7.050000) {はや};
\node[Meaning] at (-21.650000, -5.400000) {rapid};
\node[Square] at (-19.600000, -7.150000) {};
\node[Kanji] at (-19.600000, -6.650000) {清};
\node[Onyomi] at (-19.550000, -7.050000) {セイ};
\node[Kunyomi] at (-19.650000, -7.050000) {きよ.い};
\node[Meaning] at (-19.600000, -5.400000) {pure};
\node[Square] at (-17.550000, -7.150000) {};
\node[Kanji] at (-17.550000, -6.650000) {適};
\node[Onyomi] at (-17.500000, -7.050000) {テキ};
\node[Meaning] at (-17.550000, -5.400000) {suitable};
\node[Square] at (-15.500000, -7.150000) {};
\node[Kanji] at (-15.500000, -6.650000) {醜};
\node[Onyomi] at (-15.450000, -7.050000) {シュウ};
\node[Kunyomi] at (-15.550000, -7.050000) {しこ};
\node[Meaning] at (-15.500000, -5.400000) {ugly};
\node[Square] at (-13.450000, -7.150000) {};
\node[Kanji] at (-13.450000, -6.650000) {圧};
\node[Onyomi] at (-13.400000, -7.050000) {アツ};
\node[Meaning] at (-13.450000, -5.400000) {pressure};
\node[Square] at (-11.400000, -7.150000) {};
\node[Kanji] at (-11.400000, -6.650000) {懐};
\node[Onyomi] at (-11.350000, -7.050000) {カイ};
\node[Kunyomi] at (-11.450000, -7.050000) {なつ};
\node[Meaning] at (-11.400000, -5.400000) {nostalgia};
\node[Square] at (-9.350000, -7.150000) {};
\node[Kanji] at (-9.350000, -6.650000) {巡};
\node[Onyomi] at (-9.300000, -7.050000) {ジュン};
\node[Kunyomi] at (-9.400000, -7.050000) {めぐ.る};
\node[Meaning] at (-9.350000, -5.400000) {patrol};
\node[Square] at (-7.300000, -7.150000) {};
\node[Kanji] at (-7.300000, -6.650000) {操};
\node[Onyomi] at (-7.250000, -7.050000) {ソウ};
\node[Kunyomi] at (-7.350000, -7.050000) {あやつ.る};
\node[Meaning] at (-7.300000, -5.400000) {manipulate};
\node[Square] at (-5.250000, -7.150000) {};
\node[Kanji] at (-5.250000, -6.650000) {盤};
\node[Onyomi] at (-5.200000, -7.050000) {バン};
\node[Kunyomi] at (-5.300000, -7.050000) {ばん};
\node[Meaning] at (-5.250000, -5.400000) {tray};
\node[Square] at (-3.200000, -7.150000) {};
\node[Kanji] at (-3.200000, -6.650000) {侵};
\node[Onyomi] at (-3.150000, -7.050000) {シン};
\node[Kunyomi] at (-3.250000, -7.050000) {おか.す};
\node[Meaning] at (-3.200000, -5.400000) {invade};
\node[Square] at (-1.150000, -7.150000) {};
\node[Kanji] at (-1.150000, -6.650000) {辱};
\node[Onyomi] at (-1.100000, -7.050000) {ジョク};
\node[Kunyomi] at (-1.200000, -7.050000) {はずかし.める};
\node[Meaning] at (-1.150000, -5.400000) {humiliate};
\node[Square] at (0.900000, -7.150000) {};
\node[Kanji] at (0.900000, -6.650000) {癒};
\node[Onyomi] at (0.950000, -7.050000) {ユ};
\node[Kunyomi] at (0.850000, -7.050000) {い};
\node[Meaning] at (0.900000, -5.400000) {healing};
\node[Square] at (2.950000, -7.150000) {};
\node[Kanji] at (2.950000, -6.650000) {販};
\node[Onyomi] at (3.000000, -7.050000) {ハン};
\node[Meaning] at (2.950000, -5.400000) {sell};
\node[Square] at (5.000000, -7.150000) {};
\node[Kanji] at (5.000000, -6.650000) {枕};
\node[Onyomi] at (5.050000, -7.050000) {シン};
\node[Kunyomi] at (4.950000, -7.050000) {まくら};
\node[Meaning] at (5.000000, -5.400000) {pillow};
\node[Square] at (7.050000, -7.150000) {};
\node[Kanji] at (7.050000, -6.650000) {誌};
\node[Onyomi] at (7.100000, -7.050000) {シ};
\node[Meaning] at (7.050000, -5.400000) {magazine};
\node[Square] at (9.100000, -7.150000) {};
\node[Kanji] at (9.100000, -6.650000) {湿};
\node[Onyomi] at (9.150000, -7.050000) {シツ};
\node[Kunyomi] at (9.050000, -7.050000) {しめ.らせる};
\node[Meaning] at (9.100000, -5.400000) {damp};
\node[Square] at (11.150000, -7.150000) {};
\node[Kanji] at (11.150000, -6.650000) {貯};
\node[Onyomi] at (11.200000, -7.050000) {チョ};
\node[Kunyomi] at (11.100000, -7.050000) {たくわ.える};
\node[Meaning] at (11.150000, -5.400000) {savings};
\node[Square] at (13.200000, -7.150000) {};
\node[Kanji] at (13.200000, -6.650000) {拝};
\node[Onyomi] at (13.250000, -7.050000) {ハイ};
\node[Kunyomi] at (13.150000, -7.050000) {おが.む};
\node[Meaning] at (13.200000, -5.400000) {worship};
\node[Square] at (15.250000, -7.150000) {};
\node[Kanji] at (15.250000, -6.650000) {澄};
\node[Onyomi] at (15.300000, -7.050000) {チョウ};
\node[Kunyomi] at (15.200000, -7.050000) {す.*};
\node[Meaning] at (15.250000, -5.400000) {lucidity};
\node[Square] at (17.300000, -7.150000) {};
\node[Kanji] at (17.300000, -6.650000) {砕};
\node[Onyomi] at (17.350000, -7.050000) {サイ};
\node[Kunyomi] at (17.250000, -7.050000) {くだ.*};
\node[Meaning] at (17.300000, -5.400000) {smash};
\node[Square] at (19.350000, -7.150000) {};
\node[Kanji] at (19.350000, -6.650000) {塞};
\node[Onyomi] at (19.400000, -7.050000) {サイ};
\node[Kunyomi] at (19.300000, -7.050000) {ふさ-ぐ};
\node[Meaning] at (19.350000, -5.400000) {obstruct};
\node[Square] at (21.400000, -7.150000) {};
\node[Kanji] at (21.400000, -6.650000) {司};
\node[Onyomi] at (21.450000, -7.050000) {シ};
\node[Kunyomi] at (21.350000, -7.050000) {つかさど.る};
\node[Meaning] at (21.400000, -5.400000) {director};
\node[Square] at (23.450000, -7.150000) {};
\node[Kanji] at (23.450000, -6.650000) {幼};
\node[Onyomi] at (23.500000, -7.050000) {ヨウ};
\node[Kunyomi] at (23.400000, -7.050000) {おさな.い};
\node[Meaning] at (23.450000, -5.400000) {infancy};
\node[Square] at (25.500000, -7.150000) {};
\node[Kanji] at (25.500000, -6.650000) {尊};
\node[Onyomi] at (25.550000, -7.050000) {ソン};
\node[Kunyomi] at (25.450000, -7.050000) {とうと.い};
\node[Meaning] at (25.500000, -5.400000) {revered};
\node[Square] at (27.550000, -7.150000) {};
\node[Kanji] at (27.550000, -6.650000) {銅};
\node[Onyomi] at (27.600000, -7.050000) {ドウ};
\node[Kunyomi] at (27.500000, -7.050000) {あかがね};
\node[Meaning] at (27.550000, -5.400000) {copper};
\node[Square] at (29.600000, -7.150000) {};
\node[Kanji] at (29.600000, -6.650000) {殿};
\node[Onyomi] at (29.650000, -7.050000) {デン};
\node[Kunyomi] at (29.550000, -7.050000) {との};
\node[Meaning] at (29.600000, -5.400000) {milord};
\node[Square] at (31.650000, -7.150000) {};
\node[Kanji] at (31.650000, -6.650000) {掘};
\node[Onyomi] at (31.700000, -7.050000) {クツ};
\node[Kunyomi] at (31.600000, -7.050000) {ほ.る};
\node[Meaning] at (31.650000, -5.400000) {dig};
\node[Square] at (33.700000, -7.150000) {};
\node[Kanji] at (33.700000, -6.650000) {塗};
\node[Onyomi] at (33.750000, -7.050000) {ト};
\node[Kunyomi] at (33.650000, -7.050000) {ぬる};
\node[Meaning] at (33.700000, -5.400000) {paint};
\node[Square] at (35.750000, -7.150000) {};
\node[Kanji] at (35.750000, -6.650000) {浅};
\node[Onyomi] at (35.800000, -7.050000) {セン};
\node[Kunyomi] at (35.700000, -7.050000) {あさ};
\node[Meaning] at (35.750000, -5.400000) {shallow};
\node[Square] at (37.800000, -7.150000) {};
\node[Kanji] at (37.800000, -6.650000) {乾};
\node[Onyomi] at (37.850000, -7.050000) {カン};
\node[Kunyomi] at (37.750000, -7.050000) {かわ};
\node[Meaning] at (37.800000, -5.400000) {dry};
\node[Square] at (39.850000, -7.150000) {};
\node[Kanji] at (39.850000, -6.650000) {採};
\node[Onyomi] at (39.900000, -7.050000) {サイ};
\node[Kunyomi] at (39.800000, -7.050000) {と.る};
\node[Meaning] at (39.850000, -5.400000) {gather};
\node[Square] at (41.900000, -7.150000) {};
\node[Kanji] at (41.900000, -6.650000) {為};
\node[Onyomi] at (41.950000, -7.050000) {イ};
\node[Kunyomi] at (41.850000, -7.050000) {ため};
\node[Meaning] at (41.900000, -5.400000) {sake};
\node[Square] at (43.950000, -7.150000) {};
\node[Kanji] at (43.950000, -6.650000) {距};
\node[Onyomi] at (44.000000, -7.050000) {キョ};
\node[Meaning] at (43.950000, -5.400000) {distance};
\node[Square] at (46.000000, -7.150000) {};
\node[Kanji] at (46.000000, -6.650000) {泡};
\node[Onyomi] at (46.050000, -7.050000) {ホウ};
\node[Kunyomi] at (45.950000, -7.050000) {あわ};
\node[Meaning] at (46.000000, -5.400000) {bubbles};
\node[Square] at (48.050000, -7.150000) {};
\node[Kanji] at (48.050000, -6.650000) {遂};
\node[Onyomi] at (48.100000, -7.050000) {スイ};
\node[Kunyomi] at (48.000000, -7.050000) {と.げる};
\node[Meaning] at (48.050000, -5.400000) {accomplish};
\node[Square] at (50.100000, -7.150000) {};
\node[Kanji] at (50.100000, -6.650000) {悟};
\node[Onyomi] at (50.150000, -7.050000) {ゴ};
\node[Kunyomi] at (50.050000, -7.050000) {さと.る};
\node[Meaning] at (50.100000, -5.400000) {comprehension};
\node[Square] at (52.150000, -7.150000) {};
\node[Kanji] at (52.150000, -6.650000) {没};
\node[Onyomi] at (52.200000, -7.050000) {ボツ};
\node[Kunyomi] at (52.100000, -7.050000) {おぼ};
\node[Meaning] at (52.150000, -5.400000) {die};
\node[Square] at (54.200000, -7.150000) {};
\node[Kanji] at (54.200000, -6.650000) {据};
\node[Onyomi] at (54.250000, -7.050000) {キョ};
\node[Kunyomi] at (54.150000, -7.050000) {す};
\node[Meaning] at (54.200000, -5.400000) {install};
\node[Square] at (56.250000, -7.150000) {};
\node[Kanji] at (56.250000, -6.650000) {痩};
\node[Onyomi] at (56.300000, -7.050000) {ソウ};
\node[Kunyomi] at (56.200000, -7.050000) {や-せる};
\node[Meaning] at (56.250000, -5.400000) {get thin};
\node[Meaning] at (-58.500000, -6.600000) {96.60\%};
\node[Square] at (-56.500000, -9.200000) {};
\node[Kanji] at (-56.500000, -8.700000) {雰};
\node[Onyomi] at (-56.450000, -9.100000) {フン};
\node[Meaning] at (-56.500000, -7.450000) {atmosphere};
\node[Square] at (-54.450000, -9.200000) {};
\node[Kanji] at (-54.450000, -8.700000) {各};
\node[Onyomi] at (-54.400000, -9.100000) {カク};
\node[Kunyomi] at (-54.500000, -9.100000) {おの};
\node[Meaning] at (-54.450000, -7.450000) {each};
\node[Square] at (-52.400000, -9.200000) {};
\node[Kanji] at (-52.400000, -8.700000) {華};
\node[Onyomi] at (-52.350000, -9.100000) {カ};
\node[Kunyomi] at (-52.450000, -9.100000) {はな};
\node[Meaning] at (-52.400000, -7.450000) {showy};
\node[Square] at (-50.350000, -9.200000) {};
\node[Kanji] at (-50.350000, -8.700000) {蔵};
\node[Onyomi] at (-50.300000, -9.100000) {ゾウ};
\node[Kunyomi] at (-50.400000, -9.100000) {くら};
\node[Meaning] at (-50.350000, -7.450000) {storehouse};
\node[Square] at (-48.300000, -9.200000) {};
\node[Kanji] at (-48.300000, -8.700000) {雷};
\node[Onyomi] at (-48.250000, -9.100000) {ライ};
\node[Kunyomi] at (-48.350000, -9.100000) {かみなり};
\node[Meaning] at (-48.300000, -7.450000) {thunder};
\node[Square] at (-46.250000, -9.200000) {};
\node[Kanji] at (-46.250000, -8.700000) {沢};
\node[Onyomi] at (-46.200000, -9.100000) {タク};
\node[Kunyomi] at (-46.300000, -9.100000) {さわ};
\node[Meaning] at (-46.250000, -7.450000) {swamp};
\node[Square] at (-44.200000, -9.200000) {};
\node[Kanji] at (-44.200000, -8.700000) {訴};
\node[Onyomi] at (-44.150000, -9.100000) {ソ};
\node[Kunyomi] at (-44.250000, -9.100000) {うった.える};
\node[Meaning] at (-44.200000, -7.450000) {sue};
\node[Square] at (-42.150000, -9.200000) {};
\node[Kanji] at (-42.150000, -8.700000) {及};
\node[Onyomi] at (-42.100000, -9.100000) {キュウ};
\node[Kunyomi] at (-42.200000, -9.100000) {およ.*};
\node[Meaning] at (-42.150000, -7.450000) {reach};
\node[Square] at (-40.100000, -9.200000) {};
\node[Kanji] at (-40.100000, -8.700000) {棟};
\node[Onyomi] at (-40.050000, -9.100000) {トウ};
\node[Meaning] at (-40.100000, -7.450000) {pillar};
\node[Square] at (-38.050000, -9.200000) {};
\node[Kanji] at (-38.050000, -8.700000) {偶};
\node[Onyomi] at (-38.000000, -9.100000) {グウ};
\node[Kunyomi] at (-38.100000, -9.100000) {たま};
\node[Meaning] at (-38.050000, -7.450000) {accidentally};
\node[Square] at (-36.000000, -9.200000) {};
\node[Kanji] at (-36.000000, -8.700000) {召};
\node[Onyomi] at (-35.950000, -9.100000) {ショウ};
\node[Kunyomi] at (-36.050000, -9.100000) {め.す};
\node[Meaning] at (-36.000000, -7.450000) {call};
\node[Square] at (-33.950000, -9.200000) {};
\node[Kanji] at (-33.950000, -8.700000) {唾};
\node[Onyomi] at (-33.900000, -9.100000) {ダ};
\node[Kunyomi] at (-34.000000, -9.100000) {つば};
\node[Meaning] at (-33.950000, -7.450000) {saliva};
\node[Square] at (-31.900000, -9.200000) {};
\node[Kanji] at (-31.900000, -8.700000) {璧};
\node[Onyomi] at (-31.850000, -9.100000) {ヘキ};
\node[Meaning] at (-31.900000, -7.450000) {sphere};
\node[Square] at (-29.850000, -9.200000) {};
\node[Kanji] at (-29.850000, -8.700000) {符};
\node[Onyomi] at (-29.800000, -9.100000) {フ};
\node[Meaning] at (-29.850000, -7.450000) {token};
\node[Square] at (-27.800000, -9.200000) {};
\node[Kanji] at (-27.800000, -8.700000) {衣};
\node[Onyomi] at (-27.750000, -9.100000) {イ};
\node[Kunyomi] at (-27.850000, -9.100000) {ころも};
\node[Meaning] at (-27.800000, -7.450000) {clothes};
\node[Square] at (-25.750000, -9.200000) {};
\node[Kanji] at (-25.750000, -8.700000) {潮};
\node[Onyomi] at (-25.700000, -9.100000) {チョウ};
\node[Kunyomi] at (-25.800000, -9.100000) {しお};
\node[Meaning] at (-25.750000, -7.450000) {tide};
\node[Square] at (-23.700000, -9.200000) {};
\node[Kanji] at (-23.700000, -8.700000) {寧};
\node[Onyomi] at (-23.650000, -9.100000) {ネイ};
\node[Kunyomi] at (-23.750000, -9.100000) {むし.ろ};
\node[Meaning] at (-23.700000, -7.450000) {rather};
\node[Square] at (-21.650000, -9.200000) {};
\node[Kanji] at (-21.650000, -8.700000) {濯};
\node[Onyomi] at (-21.600000, -9.100000) {タク};
\node[Kunyomi] at (-21.700000, -9.100000) {すす.ぐ};
\node[Meaning] at (-21.650000, -7.450000) {wash};
\node[Square] at (-19.600000, -9.200000) {};
\node[Kanji] at (-19.600000, -8.700000) {槽};
\node[Onyomi] at (-19.550000, -9.100000) {ソウ};
\node[Kunyomi] at (-19.650000, -9.100000) {ふね};
\node[Meaning] at (-19.600000, -7.450000) {tank};
\node[Square] at (-17.550000, -9.200000) {};
\node[Kanji] at (-17.550000, -8.700000) {季};
\node[Onyomi] at (-17.500000, -9.100000) {キ};
\node[Meaning] at (-17.550000, -7.450000) {seasons};
\node[Square] at (-15.500000, -9.200000) {};
\node[Kanji] at (-15.500000, -8.700000) {豊};
\node[Onyomi] at (-15.450000, -9.100000) {ホウ};
\node[Kunyomi] at (-15.550000, -9.100000) {ゆた.か};
\node[Meaning] at (-15.500000, -7.450000) {plentiful};
\node[Square] at (-13.450000, -9.200000) {};
\node[Kanji] at (-13.450000, -8.700000) {含};
\node[Onyomi] at (-13.400000, -9.100000) {ガン};
\node[Kunyomi] at (-13.500000, -9.100000) {ふく.む};
\node[Meaning] at (-13.450000, -7.450000) {include};
\node[Square] at (-11.400000, -9.200000) {};
\node[Kanji] at (-11.400000, -8.700000) {索};
\node[Onyomi] at (-11.350000, -9.100000) {サク};
\node[Meaning] at (-11.400000, -7.450000) {search};
\node[Square] at (-9.350000, -9.200000) {};
\node[Kanji] at (-9.350000, -8.700000) {獲};
\node[Onyomi] at (-9.300000, -9.100000) {カク};
\node[Kunyomi] at (-9.400000, -9.100000) {え.る};
\node[Meaning] at (-9.350000, -7.450000) {seize};
\node[Square] at (-7.300000, -9.200000) {};
\node[Kanji] at (-7.300000, -8.700000) {擦};
\node[Onyomi] at (-7.250000, -9.100000) {サツ};
\node[Kunyomi] at (-7.350000, -9.100000) {こす};
\node[Meaning] at (-7.300000, -7.450000) {grate};
\node[Square] at (-5.250000, -9.200000) {};
\node[Kanji] at (-5.250000, -8.700000) {賭};
\node[Kunyomi] at (-5.300000, -9.100000) {か};
\node[Meaning] at (-5.250000, -7.450000) {gamble};
\node[Square] at (-3.200000, -9.200000) {};
\node[Kanji] at (-3.200000, -8.700000) {巾};
\node[Onyomi] at (-3.150000, -9.100000) {キン};
\node[Meaning] at (-3.200000, -7.450000) {towel};
\node[Square] at (-1.150000, -9.200000) {};
\node[Kanji] at (-1.150000, -8.700000) {麻};
\node[Onyomi] at (-1.100000, -9.100000) {マ};
\node[Kunyomi] at (-1.200000, -9.100000) {あさ};
\node[Meaning] at (-1.150000, -7.450000) {hemp};
\node[Square] at (0.900000, -9.200000) {};
\node[Kanji] at (0.900000, -8.700000) {淡};
\node[Onyomi] at (0.950000, -9.100000) {タン};
\node[Kunyomi] at (0.850000, -9.100000) {あわ.い};
\node[Meaning] at (0.900000, -7.450000) {faint};
\node[Square] at (2.950000, -9.200000) {};
\node[Kanji] at (2.950000, -8.700000) {茨};
\node[Onyomi] at (3.000000, -9.100000) {シ};
\node[Kunyomi] at (2.900000, -9.100000) {いばら};
\node[Meaning] at (2.950000, -7.450000) {briar};
\node[Square] at (5.000000, -9.200000) {};
\node[Kanji] at (5.000000, -8.700000) {塩};
\node[Onyomi] at (5.050000, -9.100000) {エン};
\node[Kunyomi] at (4.950000, -9.100000) {しお};
\node[Meaning] at (5.000000, -7.450000) {salt};
\node[Square] at (7.050000, -9.200000) {};
\node[Kanji] at (7.050000, -8.700000) {摘};
\node[Onyomi] at (7.100000, -9.100000) {テキ};
\node[Kunyomi] at (7.000000, -9.100000) {つ.む};
\node[Meaning] at (7.050000, -7.450000) {pluck};
\node[Square] at (9.100000, -9.200000) {};
\node[Kanji] at (9.100000, -8.700000) {旬};
\node[Onyomi] at (9.150000, -9.100000) {シュン};
\node[Meaning] at (9.100000, -7.450000) {in season};
\node[Square] at (11.150000, -9.200000) {};
\node[Kanji] at (11.150000, -8.700000) {恨};
\node[Onyomi] at (11.200000, -9.100000) {コン};
\node[Kunyomi] at (11.100000, -9.100000) {うら.む};
\node[Meaning] at (11.150000, -7.450000) {grudge};
\node[Square] at (13.200000, -9.200000) {};
\node[Kanji] at (13.200000, -8.700000) {柳};
\node[Onyomi] at (13.250000, -9.100000) {リュウ};
\node[Kunyomi] at (13.150000, -9.100000) {やなぎ};
\node[Meaning] at (13.200000, -7.450000) {willow};
\node[Square] at (15.250000, -9.200000) {};
\node[Kanji] at (15.250000, -8.700000) {勘};
\node[Onyomi] at (15.300000, -9.100000) {カン};
\node[Meaning] at (15.250000, -7.450000) {intuition};
\node[Square] at (17.300000, -9.200000) {};
\node[Kanji] at (17.300000, -8.700000) {縫};
\node[Onyomi] at (17.350000, -9.100000) {ホウ};
\node[Kunyomi] at (17.250000, -9.100000) {ぬ};
\node[Meaning] at (17.300000, -7.450000) {sew};
\node[Square] at (19.350000, -9.200000) {};
\node[Kanji] at (19.350000, -8.700000) {鼓};
\node[Onyomi] at (19.400000, -9.100000) {コ};
\node[Kunyomi] at (19.300000, -9.100000) {つづみ};
\node[Meaning] at (19.350000, -7.450000) {beat};
\node[Square] at (21.400000, -9.200000) {};
\node[Kanji] at (21.400000, -8.700000) {萎};
\node[Onyomi] at (21.450000, -9.100000) {イ};
\node[Kunyomi] at (21.350000, -9.100000) {な-える};
\node[Meaning] at (21.400000, -7.450000) {wither};
\node[Square] at (23.450000, -9.200000) {};
\node[Kanji] at (23.450000, -8.700000) {諦};
\node[Onyomi] at (23.500000, -9.100000) {テイ};
\node[Kunyomi] at (23.400000, -9.100000) {あきら-める};
\node[Meaning] at (23.450000, -7.450000) {abandon};
\node[Square] at (25.500000, -9.200000) {};
\node[Kanji] at (25.500000, -8.700000) {坂};
\node[Onyomi] at (25.550000, -9.100000) {ハン};
\node[Kunyomi] at (25.450000, -9.100000) {さか};
\node[Meaning] at (25.500000, -7.450000) {slope};
\node[Square] at (27.550000, -9.200000) {};
\node[Kanji] at (27.550000, -8.700000) {脈};
\node[Onyomi] at (27.600000, -9.100000) {ミャク};
\node[Meaning] at (27.550000, -7.450000) {vein};
\node[Square] at (29.600000, -9.200000) {};
\node[Kanji] at (29.600000, -8.700000) {著};
\node[Onyomi] at (29.650000, -9.100000) {チョ};
\node[Kunyomi] at (29.550000, -9.100000) {いちじる.しい};
\node[Meaning] at (29.600000, -7.450000) {author};
\node[Square] at (31.650000, -9.200000) {};
\node[Kanji] at (31.650000, -8.700000) {沿};
\node[Onyomi] at (31.700000, -9.100000) {エン};
\node[Kunyomi] at (31.600000, -9.100000) {そ.う};
\node[Meaning] at (31.650000, -7.450000) {run alongside};
\node[Square] at (33.700000, -9.200000) {};
\node[Kanji] at (33.700000, -8.700000) {犠};
\node[Onyomi] at (33.750000, -9.100000) {ギ};
\node[Meaning] at (33.700000, -7.450000) {sacrifice};
\node[Square] at (35.750000, -9.200000) {};
\node[Kanji] at (35.750000, -8.700000) {牲};
\node[Onyomi] at (35.800000, -9.100000) {セイ};
\node[Meaning] at (35.750000, -7.450000) {offering};
\node[Square] at (37.800000, -9.200000) {};
\node[Kanji] at (37.800000, -8.700000) {揮};
\node[Onyomi] at (37.850000, -9.100000) {キ};
\node[Meaning] at (37.800000, -7.450000) {brandish};
\node[Square] at (39.850000, -9.200000) {};
\node[Kanji] at (39.850000, -8.700000) {講};
\node[Onyomi] at (39.900000, -9.100000) {コウ};
\node[Meaning] at (39.850000, -7.450000) {lecture};
\node[Square] at (41.900000, -9.200000) {};
\node[Kanji] at (41.900000, -8.700000) {唐};
\node[Onyomi] at (41.950000, -9.100000) {トウ};
\node[Meaning] at (41.900000, -7.450000) {china};
\node[Square] at (43.950000, -9.200000) {};
\node[Kanji] at (43.950000, -8.700000) {孤};
\node[Onyomi] at (44.000000, -9.100000) {コ};
\node[Meaning] at (43.950000, -7.450000) {orphan};
\node[Square] at (46.000000, -9.200000) {};
\node[Kanji] at (46.000000, -8.700000) {緩};
\node[Onyomi] at (46.050000, -9.100000) {カン};
\node[Kunyomi] at (45.950000, -9.100000) {ゆる};
\node[Meaning] at (46.000000, -7.450000) {loose};
\node[Square] at (48.050000, -9.200000) {};
\node[Kanji] at (48.050000, -8.700000) {免};
\node[Onyomi] at (48.100000, -9.100000) {メン};
\node[Kunyomi] at (48.000000, -9.100000) {まぬか.れる};
\node[Meaning] at (48.050000, -7.450000) {excuse};
\node[Square] at (50.100000, -9.200000) {};
\node[Kanji] at (50.100000, -8.700000) {伊};
\node[Onyomi] at (50.150000, -9.100000) {イ};
\node[Kunyomi] at (50.050000, -9.100000) {だ};
\node[Meaning] at (50.100000, -7.450000) {italy};
\node[Square] at (52.150000, -9.200000) {};
\node[Kanji] at (52.150000, -8.700000) {苛};
\node[Onyomi] at (52.200000, -9.100000) {カ};
\node[Meaning] at (52.150000, -7.450000) {torment};
\node[Square] at (54.200000, -9.200000) {};
\node[Kanji] at (54.200000, -8.700000) {豪};
\node[Onyomi] at (54.250000, -9.100000) {ゴウ};
\node[Meaning] at (54.200000, -7.450000) {luxurious};
\node[Square] at (56.250000, -9.200000) {};
\node[Kanji] at (56.250000, -8.700000) {鈍};
\node[Onyomi] at (56.300000, -9.100000) {ドン};
\node[Kunyomi] at (56.200000, -9.100000) {にぶ.い};
\node[Meaning] at (56.250000, -7.450000) {dull};
\node[Meaning] at (-58.500000, -8.650000) {96.93\%};
\node[Square] at (-56.500000, -11.250000) {};
\node[Kanji] at (-56.500000, -10.750000) {殻};
\node[Onyomi] at (-56.450000, -11.150000) {カク};
\node[Kunyomi] at (-56.550000, -11.150000) {から};
\node[Meaning] at (-56.500000, -9.500000) {husk};
\node[Square] at (-54.450000, -11.250000) {};
\node[Kanji] at (-54.450000, -10.750000) {株};
\node[Onyomi] at (-54.400000, -11.150000) {シュ};
\node[Kunyomi] at (-54.500000, -11.150000) {かぶ};
\node[Meaning] at (-54.450000, -9.500000) {stocks};
\node[Square] at (-52.400000, -11.250000) {};
\node[Kanji] at (-52.400000, -10.750000) {核};
\node[Onyomi] at (-52.350000, -11.150000) {カク};
\node[Kunyomi] at (-52.450000, -11.150000) {かく};
\node[Meaning] at (-52.400000, -9.500000) {nucleus};
\node[Square] at (-50.350000, -11.250000) {};
\node[Kanji] at (-50.350000, -10.750000) {瞳};
\node[Onyomi] at (-50.300000, -11.150000) {トウ};
\node[Kunyomi] at (-50.400000, -11.150000) {ひとみ};
\node[Meaning] at (-50.350000, -9.500000) {pupil};
\node[Square] at (-48.300000, -11.250000) {};
\node[Kanji] at (-48.300000, -10.750000) {漫};
\node[Onyomi] at (-48.250000, -11.150000) {マン};
\node[Meaning] at (-48.300000, -9.500000) {manga};
\node[Square] at (-46.250000, -11.250000) {};
\node[Kanji] at (-46.250000, -10.750000) {裕};
\node[Onyomi] at (-46.200000, -11.150000) {ユウ};
\node[Meaning] at (-46.250000, -9.500000) {abundant};
\node[Square] at (-44.200000, -11.250000) {};
\node[Kanji] at (-44.200000, -10.750000) {削};
\node[Onyomi] at (-44.150000, -11.150000) {サク};
\node[Kunyomi] at (-44.250000, -11.150000) {けず.る};
\node[Meaning] at (-44.200000, -9.500000) {whittle down};
\node[Square] at (-42.150000, -11.250000) {};
\node[Kanji] at (-42.150000, -10.750000) {拒};
\node[Onyomi] at (-42.100000, -11.150000) {キョ};
\node[Kunyomi] at (-42.200000, -11.150000) {こば.む};
\node[Meaning] at (-42.150000, -9.500000) {refusal};
\node[Square] at (-40.100000, -11.250000) {};
\node[Kanji] at (-40.100000, -10.750000) {鐘};
\node[Onyomi] at (-40.050000, -11.150000) {ショウ};
\node[Kunyomi] at (-40.150000, -11.150000) {かね};
\node[Meaning] at (-40.100000, -9.500000) {bell};
\node[Square] at (-38.050000, -11.250000) {};
\node[Kanji] at (-38.050000, -10.750000) {蔑};
\node[Onyomi] at (-38.000000, -11.150000) {ベツ};
\node[Kunyomi] at (-38.100000, -11.150000) {さげす};
\node[Meaning] at (-38.050000, -9.500000) {scorn};
\node[Square] at (-36.000000, -11.250000) {};
\node[Kanji] at (-36.000000, -10.750000) {副};
\node[Onyomi] at (-35.950000, -11.150000) {フク};
\node[Meaning] at (-36.000000, -9.500000) {vice};
\node[Square] at (-33.950000, -11.250000) {};
\node[Kanji] at (-33.950000, -10.750000) {診};
\node[Onyomi] at (-33.900000, -11.150000) {シン};
\node[Kunyomi] at (-34.000000, -11.150000) {み.る};
\node[Meaning] at (-33.950000, -9.500000) {diagnose};
\node[Square] at (-31.900000, -11.250000) {};
\node[Kanji] at (-31.900000, -10.750000) {幕};
\node[Onyomi] at (-31.850000, -11.150000) {マク};
\node[Kunyomi] at (-31.950000, -11.150000) {とばり};
\node[Meaning] at (-31.900000, -9.500000) {curtain};
\node[Square] at (-29.850000, -11.250000) {};
\node[Kanji] at (-29.850000, -10.750000) {仙};
\node[Onyomi] at (-29.800000, -11.150000) {セン};
\node[Meaning] at (-29.850000, -9.500000) {hermit};
\node[Square] at (-27.800000, -11.250000) {};
\node[Kanji] at (-27.800000, -10.750000) {棄};
\node[Onyomi] at (-27.750000, -11.150000) {キ};
\node[Meaning] at (-27.800000, -9.500000) {abandon};
\node[Square] at (-25.750000, -11.250000) {};
\node[Kanji] at (-25.750000, -10.750000) {阻};
\node[Onyomi] at (-25.700000, -11.150000) {ソ};
\node[Kunyomi] at (-25.800000, -11.150000) {はば.む};
\node[Meaning] at (-25.750000, -9.500000) {thwart};
\node[Square] at (-23.700000, -11.250000) {};
\node[Kanji] at (-23.700000, -10.750000) {帳};
\node[Onyomi] at (-23.650000, -11.150000) {チョウ};
\node[Kunyomi] at (-23.750000, -11.150000) {とばり};
\node[Meaning] at (-23.700000, -9.500000) {notebook};
\node[Square] at (-21.650000, -11.250000) {};
\node[Kanji] at (-21.650000, -10.750000) {胴};
\node[Onyomi] at (-21.600000, -11.150000) {ドウ};
\node[Meaning] at (-21.650000, -9.500000) {torso};
\node[Square] at (-19.600000, -11.250000) {};
\node[Kanji] at (-19.600000, -10.750000) {畳};
\node[Onyomi] at (-19.550000, -11.150000) {ジョウ};
\node[Kunyomi] at (-19.650000, -11.150000) {たたみ};
\node[Meaning] at (-19.600000, -9.500000) {tatami mat};
\node[Square] at (-17.550000, -11.250000) {};
\node[Kanji] at (-17.550000, -10.750000) {了};
\node[Onyomi] at (-17.500000, -11.150000) {リョウ};
\node[Meaning] at (-17.550000, -9.500000) {finish};
\node[Square] at (-15.500000, -11.250000) {};
\node[Kanji] at (-15.500000, -10.750000) {童};
\node[Onyomi] at (-15.450000, -11.150000) {ドウ};
\node[Meaning] at (-15.500000, -9.500000) {juvenile};
\node[Square] at (-13.450000, -11.250000) {};
\node[Kanji] at (-13.450000, -10.750000) {笛};
\node[Onyomi] at (-13.400000, -11.150000) {テキ};
\node[Kunyomi] at (-13.500000, -11.150000) {ふえ};
\node[Meaning] at (-13.450000, -9.500000) {flute};
\node[Square] at (-11.400000, -11.250000) {};
\node[Kanji] at (-11.400000, -10.750000) {孫};
\node[Onyomi] at (-11.350000, -11.150000) {ソン};
\node[Kunyomi] at (-11.450000, -11.150000) {まご};
\node[Meaning] at (-11.400000, -9.500000) {grandchild};
\node[Square] at (-9.350000, -11.250000) {};
\node[Kanji] at (-9.350000, -10.750000) {彫};
\node[Onyomi] at (-9.300000, -11.150000) {チョウ};
\node[Kunyomi] at (-9.400000, -11.150000) {ほ.る};
\node[Meaning] at (-9.350000, -9.500000) {carve};
\node[Square] at (-7.300000, -11.250000) {};
\node[Kanji] at (-7.300000, -10.750000) {伎};
\node[Onyomi] at (-7.250000, -11.150000) {キ};
\node[Kunyomi] at (-7.350000, -11.150000) {わざ};
\node[Meaning] at (-7.300000, -9.500000) {deed};
\node[Square] at (-5.250000, -11.250000) {};
\node[Kanji] at (-5.250000, -10.750000) {妊};
\node[Onyomi] at (-5.200000, -11.150000) {ニン};
\node[Meaning] at (-5.250000, -9.500000) {pregnant};
\node[Square] at (-3.200000, -11.250000) {};
\node[Kanji] at (-3.200000, -10.750000) {綱};
\node[Onyomi] at (-3.150000, -11.150000) {コウ};
\node[Kunyomi] at (-3.250000, -11.150000) {つな};
\node[Meaning] at (-3.200000, -9.500000) {cable};
\node[Square] at (-1.150000, -11.250000) {};
\node[Kanji] at (-1.150000, -10.750000) {傍};
\node[Onyomi] at (-1.100000, -11.150000) {ボウ};
\node[Kunyomi] at (-1.200000, -11.150000) {かたわ};
\node[Meaning] at (-1.150000, -9.500000) {nearby};
\node[Square] at (0.900000, -11.250000) {};
\node[Kanji] at (0.900000, -10.750000) {侮};
\node[Onyomi] at (0.950000, -11.150000) {ブ};
\node[Kunyomi] at (0.850000, -11.150000) {あなず};
\node[Meaning] at (0.900000, -9.500000) {despise};
\node[Square] at (2.950000, -11.250000) {};
\node[Kanji] at (2.950000, -10.750000) {拷};
\node[Onyomi] at (3.000000, -11.150000) {ゴウ};
\node[Meaning] at (2.950000, -9.500000) {torture};
\node[Square] at (5.000000, -11.250000) {};
\node[Kanji] at (5.000000, -10.750000) {竹};
\node[Kunyomi] at (4.950000, -11.150000) {たけ};
\node[Meaning] at (5.000000, -9.500000) {bamboo};
\node[Square] at (7.050000, -11.250000) {};
\node[Kanji] at (7.050000, -10.750000) {述};
\node[Onyomi] at (7.100000, -11.150000) {ジュツ};
\node[Kunyomi] at (7.000000, -11.150000) {の.べる};
\node[Meaning] at (7.050000, -9.500000) {mention};
\node[Square] at (9.100000, -11.250000) {};
\node[Kanji] at (9.100000, -10.750000) {宴};
\node[Onyomi] at (9.150000, -11.150000) {エン};
\node[Kunyomi] at (9.050000, -11.150000) {うたげ};
\node[Meaning] at (9.100000, -9.500000) {banquet};
\node[Square] at (11.150000, -11.250000) {};
\node[Kanji] at (11.150000, -10.750000) {愉};
\node[Onyomi] at (11.200000, -11.150000) {ユ};
\node[Kunyomi] at (11.100000, -11.150000) {たの};
\node[Meaning] at (11.150000, -9.500000) {pleasant};
\node[Square] at (13.200000, -11.250000) {};
\node[Kanji] at (13.200000, -10.750000) {履};
\node[Onyomi] at (13.250000, -11.150000) {リ};
\node[Kunyomi] at (13.150000, -11.150000) {は.く};
\node[Meaning] at (13.200000, -9.500000) {boots};
\node[Square] at (15.250000, -11.250000) {};
\node[Kanji] at (15.250000, -10.750000) {河};
\node[Onyomi] at (15.300000, -11.150000) {カ};
\node[Kunyomi] at (15.200000, -11.150000) {かわ};
\node[Meaning] at (15.250000, -9.500000) {river};
\node[Square] at (17.300000, -11.250000) {};
\node[Kanji] at (17.300000, -10.750000) {宗};
\node[Onyomi] at (17.350000, -11.150000) {シュウ};
\node[Meaning] at (17.300000, -9.500000) {religion};
\node[Square] at (19.350000, -11.250000) {};
\node[Kanji] at (19.350000, -10.750000) {魅};
\node[Onyomi] at (19.400000, -11.150000) {ミ};
\node[Meaning] at (19.350000, -9.500000) {alluring};
\node[Square] at (21.400000, -11.250000) {};
\node[Kanji] at (21.400000, -10.750000) {皇};
\node[Onyomi] at (21.450000, -11.150000) {コウ};
\node[Meaning] at (21.400000, -9.500000) {emperor};
\node[Square] at (23.450000, -11.250000) {};
\node[Kanji] at (23.450000, -10.750000) {蒸};
\node[Onyomi] at (23.500000, -11.150000) {ジョウ};
\node[Kunyomi] at (23.400000, -11.150000) {む.れる};
\node[Meaning] at (23.450000, -9.500000) {steam};
\node[Square] at (25.500000, -11.250000) {};
\node[Kanji] at (25.500000, -10.750000) {娠};
\node[Onyomi] at (25.550000, -11.150000) {シン};
\node[Meaning] at (25.500000, -9.500000) {pregnant};
\node[Square] at (27.550000, -11.250000) {};
\node[Kanji] at (27.550000, -10.750000) {駐};
\node[Onyomi] at (27.600000, -11.150000) {チュウ};
\node[Meaning] at (27.550000, -9.500000) {resident};
\node[Square] at (29.600000, -11.250000) {};
\node[Kanji] at (29.600000, -10.750000) {稼};
\node[Onyomi] at (29.650000, -11.150000) {カ};
\node[Kunyomi] at (29.550000, -11.150000) {かせ.ぐ};
\node[Meaning] at (29.600000, -9.500000) {earnings};
\node[Square] at (31.650000, -11.250000) {};
\node[Kanji] at (31.650000, -10.750000) {涼};
\node[Onyomi] at (31.700000, -11.150000) {リョウ};
\node[Kunyomi] at (31.600000, -11.150000) {すず.しい};
\node[Meaning] at (31.650000, -9.500000) {cool};
\node[Square] at (33.700000, -11.250000) {};
\node[Kanji] at (33.700000, -10.750000) {盾};
\node[Onyomi] at (33.750000, -11.150000) {ジュン};
\node[Kunyomi] at (33.650000, -11.150000) {たて};
\node[Meaning] at (33.700000, -9.500000) {shield};
\node[Square] at (35.750000, -11.250000) {};
\node[Kanji] at (35.750000, -10.750000) {釈};
\node[Onyomi] at (35.800000, -11.150000) {シャク};
\node[Kunyomi] at (35.700000, -11.150000) {す};
\node[Meaning] at (35.750000, -9.500000) {explanation};
\node[Square] at (37.800000, -11.250000) {};
\node[Kanji] at (37.800000, -10.750000) {襟};
\node[Onyomi] at (37.850000, -11.150000) {キン};
\node[Kunyomi] at (37.750000, -11.150000) {えり};
\node[Meaning] at (37.800000, -9.500000) {collar};
\node[Square] at (39.850000, -11.250000) {};
\node[Kanji] at (39.850000, -10.750000) {旦};
\node[Onyomi] at (39.900000, -11.150000) {タン};
\node[Kunyomi] at (39.800000, -11.150000) {あきら};
\node[Meaning] at (39.850000, -9.500000) {dawn};
\node[Square] at (41.900000, -11.250000) {};
\node[Kanji] at (41.900000, -10.750000) {幻};
\node[Onyomi] at (41.950000, -11.150000) {ゲン};
\node[Kunyomi] at (41.850000, -11.150000) {まぼろし};
\node[Meaning] at (41.900000, -9.500000) {illusion};
\node[Square] at (43.950000, -11.250000) {};
\node[Kanji] at (43.950000, -10.750000) {鉛};
\node[Onyomi] at (44.000000, -11.150000) {エン};
\node[Kunyomi] at (43.900000, -11.150000) {なまり};
\node[Meaning] at (43.950000, -9.500000) {lead};
\node[Square] at (46.000000, -11.250000) {};
\node[Kanji] at (46.000000, -10.750000) {催};
\node[Onyomi] at (46.050000, -11.150000) {サイ};
\node[Kunyomi] at (45.950000, -11.150000) {もよお.す};
\node[Meaning] at (46.000000, -9.500000) {sponsor};
\node[Square] at (48.050000, -11.250000) {};
\node[Kanji] at (48.050000, -10.750000) {契};
\node[Onyomi] at (48.100000, -11.150000) {ケイ};
\node[Meaning] at (48.050000, -9.500000) {pledge};
\node[Square] at (50.100000, -11.250000) {};
\node[Kanji] at (50.100000, -10.750000) {却};
\node[Onyomi] at (50.150000, -11.150000) {キャク};
\node[Kunyomi] at (50.050000, -11.150000) {かえって};
\node[Meaning] at (50.100000, -9.500000) {contrary};
\node[Square] at (52.150000, -11.250000) {};
\node[Kanji] at (52.150000, -10.750000) {嬢};
\node[Onyomi] at (52.200000, -11.150000) {ジョウ};
\node[Kunyomi] at (52.100000, -11.150000) {むすめ};
\node[Meaning] at (52.150000, -9.500000) {miss};
\node[Square] at (54.200000, -11.250000) {};
\node[Kanji] at (54.200000, -10.750000) {廷};
\node[Onyomi] at (54.250000, -11.150000) {テイ};
\node[Meaning] at (54.200000, -9.500000) {courts};
\node[Square] at (56.250000, -11.250000) {};
\node[Kanji] at (56.250000, -10.750000) {悩};
\node[Onyomi] at (56.300000, -11.150000) {ノウ};
\node[Kunyomi] at (56.200000, -11.150000) {なや};
\node[Meaning] at (56.250000, -9.500000) {worry};
\node[Meaning] at (-58.500000, -10.700000) {97.21\%};
\node[Square] at (-56.500000, -13.300000) {};
\node[Kanji] at (-56.500000, -12.800000) {陣};
\node[Onyomi] at (-56.450000, -13.200000) {ジン};
\node[Meaning] at (-56.500000, -11.550000) {army base};
\node[Square] at (-54.450000, -13.300000) {};
\node[Kanji] at (-54.450000, -12.800000) {奏};
\node[Onyomi] at (-54.400000, -13.200000) {ソウ};
\node[Kunyomi] at (-54.500000, -13.200000) {かな.でる};
\node[Meaning] at (-54.450000, -11.550000) {play music};
\node[Square] at (-52.400000, -13.300000) {};
\node[Kanji] at (-52.400000, -12.800000) {褒};
\node[Onyomi] at (-52.350000, -13.200000) {ホウ};
\node[Kunyomi] at (-52.450000, -13.200000) {ほ.める};
\node[Meaning] at (-52.400000, -11.550000) {praise};
\node[Square] at (-50.350000, -13.300000) {};
\node[Kanji] at (-50.350000, -12.800000) {誠};
\node[Onyomi] at (-50.300000, -13.200000) {セイ};
\node[Kunyomi] at (-50.400000, -13.200000) {まこと};
\node[Meaning] at (-50.350000, -11.550000) {sincerity};
\node[Square] at (-48.300000, -13.300000) {};
\node[Kanji] at (-48.300000, -12.800000) {蜜};
\node[Onyomi] at (-48.250000, -13.200000) {ミツ};
\node[Meaning] at (-48.300000, -11.550000) {honey};
\node[Square] at (-46.250000, -13.300000) {};
\node[Kanji] at (-46.250000, -12.800000) {瀬};
\node[Onyomi] at (-46.200000, -13.200000) {ライ};
\node[Kunyomi] at (-46.300000, -13.200000) {せ};
\node[Meaning] at (-46.250000, -11.550000) {rapids};
\node[Square] at (-44.200000, -13.300000) {};
\node[Kanji] at (-44.200000, -12.800000) {嵐};
\node[Kunyomi] at (-44.250000, -13.200000) {あらし};
\node[Meaning] at (-44.200000, -11.550000) {storm};
\node[Square] at (-42.150000, -13.300000) {};
\node[Kanji] at (-42.150000, -12.800000) {駒};
\node[Onyomi] at (-42.100000, -13.200000) {ク};
\node[Kunyomi] at (-42.200000, -13.200000) {こま};
\node[Meaning] at (-42.150000, -11.550000) {chess piece};
\node[Square] at (-40.100000, -13.300000) {};
\node[Kanji] at (-40.100000, -12.800000) {酷};
\node[Onyomi] at (-40.050000, -13.200000) {コク};
\node[Kunyomi] at (-40.150000, -13.200000) {ひど};
\node[Meaning] at (-40.100000, -11.550000) {cruel};
\node[Square] at (-38.050000, -13.300000) {};
\node[Kanji] at (-38.050000, -12.800000) {刀};
\node[Onyomi] at (-38.000000, -13.200000) {トウ};
\node[Kunyomi] at (-38.100000, -13.200000) {かたな};
\node[Meaning] at (-38.050000, -11.550000) {sword};
\node[Square] at (-36.000000, -13.300000) {};
\node[Kanji] at (-36.000000, -12.800000) {稚};
\node[Onyomi] at (-35.950000, -13.200000) {チ};
\node[Meaning] at (-36.000000, -11.550000) {immature};
\node[Square] at (-33.950000, -13.300000) {};
\node[Kanji] at (-33.950000, -12.800000) {熟};
\node[Onyomi] at (-33.900000, -13.200000) {ジュク};
\node[Kunyomi] at (-34.000000, -13.200000) {う.れる};
\node[Meaning] at (-33.950000, -11.550000) {ripen};
\node[Square] at (-31.900000, -13.300000) {};
\node[Kanji] at (-31.900000, -12.800000) {寿};
\node[Onyomi] at (-31.850000, -13.200000) {ジュ};
\node[Kunyomi] at (-31.950000, -13.200000) {ことぶき};
\node[Meaning] at (-31.900000, -11.550000) {lifespan};
\node[Square] at (-29.850000, -13.300000) {};
\node[Kanji] at (-29.850000, -12.800000) {幅};
\node[Onyomi] at (-29.800000, -13.200000) {フク};
\node[Kunyomi] at (-29.900000, -13.200000) {はば};
\node[Meaning] at (-29.850000, -11.550000) {width};
\node[Square] at (-27.800000, -13.300000) {};
\node[Kanji] at (-27.800000, -12.800000) {募};
\node[Onyomi] at (-27.750000, -13.200000) {ボ};
\node[Kunyomi] at (-27.850000, -13.200000) {つの.る};
\node[Meaning] at (-27.800000, -11.550000) {recruit};
\node[Square] at (-25.750000, -13.300000) {};
\node[Kanji] at (-25.750000, -12.800000) {紛};
\node[Onyomi] at (-25.700000, -13.200000) {フン};
\node[Kunyomi] at (-25.800000, -13.200000) {まぎ.*};
\node[Meaning] at (-25.750000, -11.550000) {distract};
\node[Square] at (-23.700000, -13.300000) {};
\node[Kanji] at (-23.700000, -12.800000) {舟};
\node[Onyomi] at (-23.650000, -13.200000) {シュウ};
\node[Kunyomi] at (-23.750000, -13.200000) {ふね};
\node[Meaning] at (-23.700000, -11.550000) {boat};
\node[Square] at (-21.650000, -13.300000) {};
\node[Kanji] at (-21.650000, -12.800000) {噌};
\node[Onyomi] at (-21.600000, -13.200000) {ソ};
\node[Meaning] at (-21.650000, -11.550000) {boisterous};
\node[Square] at (-19.600000, -13.300000) {};
\node[Kanji] at (-19.600000, -12.800000) {恵};
\node[Onyomi] at (-19.550000, -13.200000) {エ};
\node[Kunyomi] at (-19.650000, -13.200000) {めぐ.*};
\node[Meaning] at (-19.600000, -11.550000) {favor};
\node[Square] at (-17.550000, -13.300000) {};
\node[Kanji] at (-17.550000, -12.800000) {狩};
\node[Onyomi] at (-17.500000, -13.200000) {シュ};
\node[Kunyomi] at (-17.600000, -13.200000) {か};
\node[Meaning] at (-17.550000, -11.550000) {hunt};
\node[Square] at (-15.500000, -13.300000) {};
\node[Kanji] at (-15.500000, -12.800000) {翻};
\node[Onyomi] at (-15.450000, -13.200000) {ホン};
\node[Kunyomi] at (-15.550000, -13.200000) {ひるがえ.*};
\node[Meaning] at (-15.500000, -11.550000) {flip};
\node[Square] at (-13.450000, -13.300000) {};
\node[Kanji] at (-13.450000, -12.800000) {宛};
\node[Kunyomi] at (-13.500000, -13.200000) {あ-てる};
\node[Meaning] at (-13.450000, -11.550000) {allocate};
\node[Square] at (-11.400000, -13.300000) {};
\node[Kanji] at (-11.400000, -12.800000) {弓};
\node[Onyomi] at (-11.350000, -13.200000) {キュウ};
\node[Kunyomi] at (-11.450000, -13.200000) {ゆみ};
\node[Meaning] at (-11.400000, -11.550000) {bow};
\node[Square] at (-9.350000, -13.300000) {};
\node[Kanji] at (-9.350000, -12.800000) {志};
\node[Onyomi] at (-9.300000, -13.200000) {シ};
\node[Kunyomi] at (-9.400000, -13.200000) {こころざし};
\node[Meaning] at (-9.350000, -11.550000) {intention};
\node[Square] at (-7.300000, -13.300000) {};
\node[Kanji] at (-7.300000, -12.800000) {誤};
\node[Onyomi] at (-7.250000, -13.200000) {ゴ};
\node[Kunyomi] at (-7.350000, -13.200000) {あやま.*};
\node[Meaning] at (-7.300000, -11.550000) {mistake};
\node[Square] at (-5.250000, -13.300000) {};
\node[Kanji] at (-5.250000, -12.800000) {己};
\node[Onyomi] at (-5.200000, -13.200000) {コ};
\node[Kunyomi] at (-5.300000, -13.200000) {おのれ};
\node[Meaning] at (-5.250000, -11.550000) {oneself};
\node[Square] at (-3.200000, -13.300000) {};
\node[Kanji] at (-3.200000, -12.800000) {厄};
\node[Onyomi] at (-3.150000, -13.200000) {ヤク};
\node[Meaning] at (-3.200000, -11.550000) {unlucky};
\node[Square] at (-1.150000, -13.300000) {};
\node[Kanji] at (-1.150000, -12.800000) {旧};
\node[Onyomi] at (-1.100000, -13.200000) {キュウ};
\node[Meaning] at (-1.150000, -11.550000) {former};
\node[Square] at (0.900000, -13.300000) {};
\node[Kanji] at (0.900000, -12.800000) {患};
\node[Onyomi] at (0.950000, -13.200000) {カン};
\node[Kunyomi] at (0.850000, -13.200000) {わずら.う};
\node[Meaning] at (0.900000, -11.550000) {afflicted};
\node[Square] at (2.950000, -13.300000) {};
\node[Kanji] at (2.950000, -12.800000) {崖};
\node[Onyomi] at (3.000000, -13.200000) {ガイ};
\node[Kunyomi] at (2.900000, -13.200000) {がけ};
\node[Meaning] at (2.950000, -11.550000) {cliff};
\node[Square] at (5.000000, -13.300000) {};
\node[Kanji] at (5.000000, -12.800000) {軒};
\node[Onyomi] at (5.050000, -13.200000) {ケン};
\node[Kunyomi] at (4.950000, -13.200000) {のき        };
\node[Meaning] at (5.000000, -11.550000) {house counter};
\node[Square] at (7.050000, -13.300000) {};
\node[Kanji] at (7.050000, -12.800000) {充};
\node[Onyomi] at (7.100000, -13.200000) {ジュウ};
\node[Kunyomi] at (7.000000, -13.200000) {あ.てる};
\node[Meaning] at (7.050000, -11.550000) {allocate};
\node[Square] at (9.100000, -13.300000) {};
\node[Kanji] at (9.100000, -12.800000) {脂};
\node[Onyomi] at (9.150000, -13.200000) {シ};
\node[Kunyomi] at (9.050000, -13.200000) {あぶら        };
\node[Meaning] at (9.100000, -11.550000) {fat};
\node[Square] at (11.150000, -13.300000) {};
\node[Kanji] at (11.150000, -12.800000) {搭};
\node[Onyomi] at (11.200000, -13.200000) {トウ};
\node[Meaning] at (11.150000, -11.550000) {board};
\node[Square] at (13.200000, -13.300000) {};
\node[Kanji] at (13.200000, -12.800000) {窟};
\node[Onyomi] at (13.250000, -13.200000) {クツ};
\node[Meaning] at (13.200000, -11.550000) {cavern};
\node[Square] at (15.250000, -13.300000) {};
\node[Kanji] at (15.250000, -12.800000) {傘};
\node[Onyomi] at (15.300000, -13.200000) {サン};
\node[Kunyomi] at (15.200000, -13.200000) {かさ};
\node[Meaning] at (15.250000, -11.550000) {umbrella};
\node[Square] at (17.300000, -13.300000) {};
\node[Kanji] at (17.300000, -12.800000) {盟};
\node[Onyomi] at (17.350000, -13.200000) {メイ};
\node[Meaning] at (17.300000, -11.550000) {alliance};
\node[Square] at (19.350000, -13.300000) {};
\node[Kanji] at (19.350000, -12.800000) {般};
\node[Onyomi] at (19.400000, -13.200000) {ハン};
\node[Meaning] at (19.350000, -11.550000) {generally};
\node[Square] at (21.400000, -13.300000) {};
\node[Kanji] at (21.400000, -12.800000) {桃};
\node[Kunyomi] at (21.350000, -13.200000) {もも};
\node[Meaning] at (21.400000, -11.550000) {peach};
\node[Square] at (23.450000, -13.300000) {};
\node[Kanji] at (23.450000, -12.800000) {囚};
\node[Onyomi] at (23.500000, -13.200000) {シュウ};
\node[Kunyomi] at (23.400000, -13.200000) {とら};
\node[Meaning] at (23.450000, -11.550000) {criminal};
\node[Square] at (25.500000, -13.300000) {};
\node[Kanji] at (25.500000, -12.800000) {亀};
\node[Kunyomi] at (25.450000, -13.200000) {かめ};
\node[Meaning] at (25.500000, -11.550000) {turtle};
\node[Square] at (27.550000, -13.300000) {};
\node[Kanji] at (27.550000, -12.800000) {廃};
\node[Onyomi] at (27.600000, -13.200000) {ハイ};
\node[Kunyomi] at (27.500000, -13.200000) {すた};
\node[Meaning] at (27.550000, -11.550000) {obsolete};
\node[Square] at (29.600000, -13.300000) {};
\node[Kanji] at (29.600000, -12.800000) {繁};
\node[Onyomi] at (29.650000, -13.200000) {ハン};
\node[Kunyomi] at (29.550000, -13.200000) {しげ.*};
\node[Meaning] at (29.600000, -11.550000) {overgrown};
\node[Square] at (31.650000, -13.300000) {};
\node[Kanji] at (31.650000, -12.800000) {笠};
\node[Kunyomi] at (31.600000, -13.200000) {かさ};
\node[Meaning] at (31.650000, -11.550000) {conical hat};
\node[Square] at (33.700000, -13.300000) {};
\node[Kanji] at (33.700000, -12.800000) {菌};
\node[Onyomi] at (33.750000, -13.200000) {キン};
\node[Meaning] at (33.700000, -11.550000) {bacteria};
\node[Square] at (35.750000, -13.300000) {};
\node[Kanji] at (35.750000, -12.800000) {塚};
\node[Onyomi] at (35.800000, -13.200000) {チョウ};
\node[Kunyomi] at (35.700000, -13.200000) {つか};
\node[Meaning] at (35.750000, -11.550000) {mound};
\node[Square] at (37.800000, -13.300000) {};
\node[Kanji] at (37.800000, -12.800000) {之};
\node[Onyomi] at (37.850000, -13.200000) {シ};
\node[Kunyomi] at (37.750000, -13.200000) {これ};
\node[Meaning] at (37.800000, -11.550000) {this};
\node[Square] at (39.850000, -13.300000) {};
\node[Kanji] at (39.850000, -12.800000) {咽};
\node[Onyomi] at (39.900000, -13.200000) {イン};
\node[Meaning] at (39.850000, -11.550000) {throat};
\node[Square] at (41.900000, -13.300000) {};
\node[Kanji] at (41.900000, -12.800000) {益};
\node[Onyomi] at (41.950000, -13.200000) {エキ};
\node[Meaning] at (41.900000, -11.550000) {benefit};
\node[Square] at (43.950000, -13.300000) {};
\node[Kanji] at (43.950000, -12.800000) {貿};
\node[Onyomi] at (44.000000, -13.200000) {ボウ};
\node[Meaning] at (43.950000, -11.550000) {trade};
\node[Square] at (46.000000, -13.300000) {};
\node[Kanji] at (46.000000, -12.800000) {枠};
\node[Kunyomi] at (45.950000, -13.200000) {わく};
\node[Meaning] at (46.000000, -11.550000) {frame};
\node[Square] at (48.050000, -13.300000) {};
\node[Kanji] at (48.050000, -12.800000) {伺};
\node[Onyomi] at (48.100000, -13.200000) {シ};
\node[Kunyomi] at (48.000000, -13.200000) {うかが};
\node[Meaning] at (48.050000, -11.550000) {pay respects};
\node[Square] at (50.100000, -13.300000) {};
\node[Kanji] at (50.100000, -12.800000) {潟};
\node[Onyomi] at (50.150000, -13.200000) {セキ};
\node[Kunyomi] at (50.050000, -13.200000) {かた};
\node[Meaning] at (50.100000, -11.550000) {lagoon};
\node[Square] at (52.150000, -13.300000) {};
\node[Kanji] at (52.150000, -12.800000) {丘};
\node[Onyomi] at (52.200000, -13.200000) {キュウ};
\node[Kunyomi] at (52.100000, -13.200000) {おか};
\node[Meaning] at (52.150000, -11.550000) {hill};
\node[Square] at (54.200000, -13.300000) {};
\node[Kanji] at (54.200000, -12.800000) {滝};
\node[Kunyomi] at (54.150000, -13.200000) {たき};
\node[Meaning] at (54.200000, -11.550000) {waterfall};
\node[Square] at (56.250000, -13.300000) {};
\node[Kanji] at (56.250000, -12.800000) {婆};
\node[Onyomi] at (56.300000, -13.200000) {バ};
\node[Kunyomi] at (56.200000, -13.200000) {ばあ};
\node[Meaning] at (56.250000, -11.550000) {old woman};
\node[Meaning] at (-58.500000, -12.750000) {97.45\%};
\node[Square] at (-56.500000, -15.350000) {};
\node[Kanji] at (-56.500000, -14.850000) {鉢};
\node[Onyomi] at (-56.450000, -15.250000) {ハチ};
\node[Meaning] at (-56.500000, -13.600000) {bowl};
\node[Square] at (-54.450000, -15.350000) {};
\node[Kanji] at (-54.450000, -14.850000) {蚊};
\node[Kunyomi] at (-54.500000, -15.250000) {か};
\node[Meaning] at (-54.450000, -13.600000) {mosquito};
\node[Square] at (-52.400000, -15.350000) {};
\node[Kanji] at (-52.400000, -14.850000) {鬱};
\node[Onyomi] at (-52.350000, -15.250000) {ウツ};
\node[Meaning] at (-52.400000, -13.600000) {gloom};
\node[Square] at (-50.350000, -15.350000) {};
\node[Kanji] at (-50.350000, -14.850000) {麦};
\node[Onyomi] at (-50.300000, -15.250000) {バク};
\node[Kunyomi] at (-50.400000, -15.250000) {むぎ};
\node[Meaning] at (-50.350000, -13.600000) {wheat};
\node[Square] at (-48.300000, -15.350000) {};
\node[Kanji] at (-48.300000, -14.850000) {芸};
\node[Onyomi] at (-48.250000, -15.250000) {ゲイ};
\node[Meaning] at (-48.300000, -13.600000) {acting};
\node[Square] at (-46.250000, -15.350000) {};
\node[Kanji] at (-46.250000, -14.850000) {汁};
\node[Onyomi] at (-46.200000, -15.250000) {ジュウ};
\node[Kunyomi] at (-46.300000, -15.250000) {しる};
\node[Meaning] at (-46.250000, -13.600000) {soup};
\node[Square] at (-44.200000, -15.350000) {};
\node[Kanji] at (-44.200000, -14.850000) {雅};
\node[Onyomi] at (-44.150000, -15.250000) {ガ};
\node[Kunyomi] at (-44.250000, -15.250000) {みや.び};
\node[Meaning] at (-44.200000, -13.600000) {elegant};
\node[Square] at (-42.150000, -15.350000) {};
\node[Kanji] at (-42.150000, -14.850000) {穏};
\node[Onyomi] at (-42.100000, -15.250000) {オン};
\node[Kunyomi] at (-42.200000, -15.250000) {おだ.やか};
\node[Meaning] at (-42.150000, -13.600000) {calm};
\node[Square] at (-40.100000, -15.350000) {};
\node[Kanji] at (-40.100000, -14.850000) {浦};
\node[Onyomi] at (-40.050000, -15.250000) {ホ};
\node[Kunyomi] at (-40.150000, -15.250000) {うら};
\node[Meaning] at (-40.100000, -13.600000) {bay};
\node[Square] at (-38.050000, -15.350000) {};
\node[Kanji] at (-38.050000, -14.850000) {貫};
\node[Onyomi] at (-38.000000, -15.250000) {カン};
\node[Kunyomi] at (-38.100000, -15.250000) {つらぬ};
\node[Meaning] at (-38.050000, -13.600000) {pierce};
\node[Square] at (-36.000000, -15.350000) {};
\node[Kanji] at (-36.000000, -14.850000) {里};
\node[Onyomi] at (-35.950000, -15.250000) {リ};
\node[Kunyomi] at (-36.050000, -15.250000) {さと};
\node[Meaning] at (-36.000000, -13.600000) {home village};
\node[Square] at (-33.950000, -15.350000) {};
\node[Kanji] at (-33.950000, -14.850000) {漢};
\node[Onyomi] at (-33.900000, -15.250000) {カン};
\node[Meaning] at (-33.950000, -13.600000) {chinese};
\node[Square] at (-31.900000, -15.350000) {};
\node[Kanji] at (-31.900000, -14.850000) {梅};
\node[Onyomi] at (-31.850000, -15.250000) {バイ};
\node[Kunyomi] at (-31.950000, -15.250000) {うめ};
\node[Meaning] at (-31.900000, -13.600000) {ume};
\node[Square] at (-29.850000, -15.350000) {};
\node[Kanji] at (-29.850000, -14.850000) {俳};
\node[Onyomi] at (-29.800000, -15.250000) {ハイ};
\node[Meaning] at (-29.850000, -13.600000) {haiku};
\node[Square] at (-27.800000, -15.350000) {};
\node[Kanji] at (-27.800000, -14.850000) {還};
\node[Onyomi] at (-27.750000, -15.250000) {カン};
\node[Kunyomi] at (-27.850000, -15.250000) {かえ.る};
\node[Meaning] at (-27.800000, -13.600000) {send back};
\node[Square] at (-25.750000, -15.350000) {};
\node[Kanji] at (-25.750000, -14.850000) {範};
\node[Onyomi] at (-25.700000, -15.250000) {ハン};
\node[Meaning] at (-25.750000, -13.600000) {example};
\node[Square] at (-23.700000, -15.350000) {};
\node[Kanji] at (-23.700000, -14.850000) {朗};
\node[Onyomi] at (-23.650000, -15.250000) {ロウ};
\node[Kunyomi] at (-23.750000, -15.250000) {ほが.らか};
\node[Meaning] at (-23.700000, -13.600000) {bright};
\node[Square] at (-21.650000, -15.350000) {};
\node[Kanji] at (-21.650000, -14.850000) {缶};
\node[Onyomi] at (-21.600000, -15.250000) {カン};
\node[Meaning] at (-21.650000, -13.600000) {tin can};
\node[Square] at (-19.600000, -15.350000) {};
\node[Kanji] at (-19.600000, -14.850000) {叱};
\node[Kunyomi] at (-19.650000, -15.250000) {しか};
\node[Meaning] at (-19.600000, -13.600000) {scold};
\node[Square] at (-17.550000, -15.350000) {};
\node[Kanji] at (-17.550000, -14.850000) {裸};
\node[Onyomi] at (-17.500000, -15.250000) {ラ};
\node[Kunyomi] at (-17.600000, -15.250000) {はだか};
\node[Meaning] at (-17.550000, -13.600000) {naked};
\node[Square] at (-15.500000, -15.350000) {};
\node[Kanji] at (-15.500000, -14.850000) {沸};
\node[Onyomi] at (-15.450000, -15.250000) {フツ};
\node[Kunyomi] at (-15.550000, -15.250000) {わ};
\node[Meaning] at (-15.500000, -13.600000) {boil};
\node[Square] at (-13.450000, -15.350000) {};
\node[Kanji] at (-13.450000, -14.850000) {虐};
\node[Onyomi] at (-13.400000, -15.250000) {ギャク};
\node[Kunyomi] at (-13.500000, -15.250000) {しいた};
\node[Meaning] at (-13.450000, -13.600000) {oppress};
\node[Square] at (-11.400000, -15.350000) {};
\node[Kanji] at (-11.400000, -14.850000) {堪};
\node[Onyomi] at (-11.350000, -15.250000) {カン};
\node[Kunyomi] at (-11.450000, -15.250000) {た};
\node[Meaning] at (-11.400000, -13.600000) {endure};
\node[Square] at (-9.350000, -15.350000) {};
\node[Kanji] at (-9.350000, -14.850000) {捻};
\node[Onyomi] at (-9.300000, -15.250000) {ネン};
\node[Kunyomi] at (-9.400000, -15.250000) {ひね-る};
\node[Meaning] at (-9.350000, -13.600000) {twist};
\node[Square] at (-7.300000, -15.350000) {};
\node[Kanji] at (-7.300000, -14.850000) {徴};
\node[Onyomi] at (-7.250000, -15.250000) {チョウ};
\node[Meaning] at (-7.300000, -13.600000) {indication};
\node[Square] at (-5.250000, -15.350000) {};
\node[Kanji] at (-5.250000, -14.850000) {径};
\node[Onyomi] at (-5.200000, -15.250000) {ケイ};
\node[Meaning] at (-5.250000, -13.600000) {diameter};
\node[Square] at (-3.200000, -15.350000) {};
\node[Kanji] at (-3.200000, -14.850000) {雇};
\node[Onyomi] at (-3.150000, -15.250000) {コ};
\node[Kunyomi] at (-3.250000, -15.250000) {やと.う};
\node[Meaning] at (-3.200000, -13.600000) {employ};
\node[Square] at (-1.150000, -15.350000) {};
\node[Kanji] at (-1.150000, -14.850000) {幽};
\node[Onyomi] at (-1.100000, -15.250000) {ユウ};
\node[Meaning] at (-1.150000, -13.600000) {secluded};
\node[Square] at (0.900000, -15.350000) {};
\node[Kanji] at (0.900000, -14.850000) {壮};
\node[Onyomi] at (0.950000, -15.250000) {ソウ};
\node[Meaning] at (0.900000, -13.600000) {robust};
\node[Square] at (2.950000, -15.350000) {};
\node[Kanji] at (2.950000, -14.850000) {紋};
\node[Onyomi] at (3.000000, -15.250000) {モン};
\node[Meaning] at (2.950000, -13.600000) {family crest};
\node[Square] at (5.000000, -15.350000) {};
\node[Kanji] at (5.000000, -14.850000) {勲};
\node[Onyomi] at (5.050000, -15.250000) {クン};
\node[Kunyomi] at (4.950000, -15.250000) {いさお};
\node[Meaning] at (5.000000, -13.600000) {merit};
\node[Square] at (7.050000, -15.350000) {};
\node[Kanji] at (7.050000, -14.850000) {郭};
\node[Onyomi] at (7.100000, -15.250000) {カク};
\node[Kunyomi] at (7.000000, -15.250000) {くるわ        };
\node[Meaning] at (7.050000, -13.600000) {enclosure};
\node[Square] at (9.100000, -15.350000) {};
\node[Kanji] at (9.100000, -14.850000) {批};
\node[Onyomi] at (9.150000, -15.250000) {ヒ};
\node[Meaning] at (9.100000, -13.600000) {criticism};
\node[Square] at (11.150000, -15.350000) {};
\node[Kanji] at (11.150000, -14.850000) {錦};
\node[Onyomi] at (11.200000, -15.250000) {キン};
\node[Kunyomi] at (11.100000, -15.250000) {にしき};
\node[Meaning] at (11.150000, -13.600000) {brocade};
\node[Square] at (13.200000, -15.350000) {};
\node[Kanji] at (13.200000, -14.850000) {悼};
\node[Onyomi] at (13.250000, -15.250000) {トウ};
\node[Kunyomi] at (13.150000, -15.250000) {いた};
\node[Meaning] at (13.200000, -13.600000) {grieve};
\node[Square] at (15.250000, -15.350000) {};
\node[Kanji] at (15.250000, -14.850000) {胆};
\node[Onyomi] at (15.300000, -15.250000) {タン};
\node[Kunyomi] at (15.200000, -15.250000) {きも};
\node[Meaning] at (15.250000, -13.600000) {guts};
\node[Square] at (17.300000, -15.350000) {};
\node[Kanji] at (17.300000, -14.850000) {羨};
\node[Onyomi] at (17.350000, -15.250000) {セン};
\node[Kunyomi] at (17.250000, -15.250000) {うらや-む};
\node[Meaning] at (17.300000, -13.600000) {envy};
\node[Square] at (19.350000, -15.350000) {};
\node[Kanji] at (19.350000, -14.850000) {妨};
\node[Onyomi] at (19.400000, -15.250000) {ボウ};
\node[Kunyomi] at (19.300000, -15.250000) {さまた.げる};
\node[Meaning] at (19.350000, -13.600000) {obstruct};
\node[Square] at (21.400000, -15.350000) {};
\node[Kanji] at (21.400000, -14.850000) {択};
\node[Onyomi] at (21.450000, -15.250000) {タク};
\node[Kunyomi] at (21.350000, -15.250000) {えら.ぶ};
\node[Meaning] at (21.400000, -13.600000) {select};
\node[Square] at (23.450000, -15.350000) {};
\node[Kanji] at (23.450000, -14.850000) {蜂};
\node[Onyomi] at (23.500000, -15.250000) {ホウ};
\node[Kunyomi] at (23.400000, -15.250000) {はち};
\node[Meaning] at (23.450000, -13.600000) {bee};
\node[Square] at (25.500000, -15.350000) {};
\node[Kanji] at (25.500000, -14.850000) {趣};
\node[Onyomi] at (25.550000, -15.250000) {シュ};
\node[Kunyomi] at (25.450000, -15.250000) {おもむき};
\node[Meaning] at (25.500000, -13.600000) {gist};
\node[Square] at (27.550000, -15.350000) {};
\node[Kanji] at (27.550000, -14.850000) {絹};
\node[Onyomi] at (27.600000, -15.250000) {ケン};
\node[Kunyomi] at (27.500000, -15.250000) {きぬ};
\node[Meaning] at (27.550000, -13.600000) {silk};
\node[Square] at (29.600000, -15.350000) {};
\node[Kanji] at (29.600000, -14.850000) {戚};
\node[Onyomi] at (29.650000, -15.250000) {セキ};
\node[Meaning] at (29.600000, -13.600000) {grieve};
\node[Square] at (31.650000, -15.350000) {};
\node[Kanji] at (31.650000, -14.850000) {煎};
\node[Onyomi] at (31.700000, -15.250000) {セン};
\node[Kunyomi] at (31.600000, -15.250000) {い-る};
\node[Meaning] at (31.650000, -13.600000) {broil};
\node[Square] at (33.700000, -15.350000) {};
\node[Kanji] at (33.700000, -14.850000) {貝};
\node[Kunyomi] at (33.650000, -15.250000) {かい};
\node[Meaning] at (33.700000, -13.600000) {shellfish};
\node[Square] at (35.750000, -15.350000) {};
\node[Kanji] at (35.750000, -14.850000) {殖};
\node[Onyomi] at (35.800000, -15.250000) {ショク};
\node[Kunyomi] at (35.700000, -15.250000) {ふ.える};
\node[Meaning] at (35.750000, -13.600000) {multiply};
\node[Square] at (37.800000, -15.350000) {};
\node[Kanji] at (37.800000, -14.850000) {猿};
\node[Onyomi] at (37.850000, -15.250000) {エン};
\node[Kunyomi] at (37.750000, -15.250000) {さる};
\node[Meaning] at (37.800000, -13.600000) {monkey};
\node[Square] at (39.850000, -15.350000) {};
\node[Kanji] at (39.850000, -14.850000) {劣};
\node[Onyomi] at (39.900000, -15.250000) {レツ};
\node[Kunyomi] at (39.800000, -15.250000) {おと.る};
\node[Meaning] at (39.850000, -13.600000) {inferiority};
\node[Square] at (41.900000, -15.350000) {};
\node[Kanji] at (41.900000, -14.850000) {晶};
\node[Onyomi] at (41.950000, -15.250000) {ショウ};
\node[Meaning] at (41.900000, -13.600000) {crystal};
\node[Square] at (43.950000, -15.350000) {};
\node[Kanji] at (43.950000, -14.850000) {陥};
\node[Onyomi] at (44.000000, -15.250000) {カン};
\node[Kunyomi] at (43.900000, -15.250000) {おちい};
\node[Meaning] at (43.950000, -13.600000) {cave in};
\node[Square] at (46.000000, -15.350000) {};
\node[Kanji] at (46.000000, -14.850000) {徐};
\node[Onyomi] at (46.050000, -15.250000) {ジョ};
\node[Kunyomi] at (45.950000, -15.250000) {おもむ};
\node[Meaning] at (46.000000, -13.600000) {gently};
\node[Square] at (48.050000, -15.350000) {};
\node[Kanji] at (48.050000, -14.850000) {虜};
\node[Onyomi] at (48.100000, -15.250000) {リョ};
\node[Kunyomi] at (48.000000, -15.250000) {とりく};
\node[Meaning] at (48.050000, -13.600000) {captive};
\node[Square] at (50.100000, -15.350000) {};
\node[Kanji] at (50.100000, -14.850000) {裾};
\node[Kunyomi] at (50.050000, -15.250000) {すそ};
\node[Meaning] at (50.100000, -13.600000) {cuff};
\node[Square] at (52.150000, -15.350000) {};
\node[Kanji] at (52.150000, -14.850000) {潔};
\node[Onyomi] at (52.200000, -15.250000) {ケツ};
\node[Kunyomi] at (52.100000, -15.250000) {いさぎよ.い};
\node[Meaning] at (52.150000, -13.600000) {pure};
\node[Square] at (54.200000, -15.350000) {};
\node[Kanji] at (54.200000, -14.850000) {券};
\node[Onyomi] at (54.250000, -15.250000) {ケン};
\node[Meaning] at (54.200000, -13.600000) {ticket};
\node[Square] at (56.250000, -15.350000) {};
\node[Kanji] at (56.250000, -14.850000) {措};
\node[Onyomi] at (56.300000, -15.250000) {ソ};
\node[Meaning] at (56.250000, -13.600000) {set aside};
\node[Meaning] at (-58.500000, -14.800000) {97.65\%};
\node[Square] at (-56.500000, -17.400000) {};
\node[Kanji] at (-56.500000, -16.900000) {砲};
\node[Onyomi] at (-56.450000, -17.300000) {ホウ};
\node[Meaning] at (-56.500000, -15.650000) {cannon};
\node[Square] at (-54.450000, -17.400000) {};
\node[Kanji] at (-54.450000, -16.900000) {卓};
\node[Onyomi] at (-54.400000, -17.300000) {タク};
\node[Meaning] at (-54.450000, -15.650000) {table};
\node[Square] at (-52.400000, -17.400000) {};
\node[Kanji] at (-52.400000, -16.900000) {癖};
\node[Onyomi] at (-52.350000, -17.300000) {ヘキ};
\node[Kunyomi] at (-52.450000, -17.300000) {くせ};
\node[Meaning] at (-52.400000, -15.650000) {habit};
\node[Square] at (-50.350000, -17.400000) {};
\node[Kanji] at (-50.350000, -16.900000) {架};
\node[Onyomi] at (-50.300000, -17.300000) {カ};
\node[Kunyomi] at (-50.400000, -17.300000) {か.*};
\node[Meaning] at (-50.350000, -15.650000) {shelf};
\node[Square] at (-48.300000, -17.400000) {};
\node[Kanji] at (-48.300000, -16.900000) {粗};
\node[Onyomi] at (-48.250000, -17.300000) {ソ};
\node[Kunyomi] at (-48.350000, -17.300000) {あら};
\node[Meaning] at (-48.300000, -15.650000) {coarse};
\node[Square] at (-46.250000, -17.400000) {};
\node[Kanji] at (-46.250000, -16.900000) {碑};
\node[Onyomi] at (-46.200000, -17.300000) {ヒ};
\node[Kunyomi] at (-46.300000, -17.300000) {いしぶみ};
\node[Meaning] at (-46.250000, -15.650000) {tombstone};
\node[Square] at (-44.200000, -17.400000) {};
\node[Kanji] at (-44.200000, -16.900000) {赦};
\node[Onyomi] at (-44.150000, -17.300000) {シャ};
\node[Meaning] at (-44.200000, -15.650000) {pardon};
\node[Square] at (-42.150000, -17.400000) {};
\node[Kanji] at (-42.150000, -16.900000) {奴};
\node[Onyomi] at (-42.100000, -17.300000) {ド};
\node[Kunyomi] at (-42.200000, -17.300000) {やつ};
\node[Meaning] at (-42.150000, -15.650000) {dude};
\node[Square] at (-40.100000, -17.400000) {};
\node[Kanji] at (-40.100000, -16.900000) {依};
\node[Onyomi] at (-40.050000, -17.300000) {イ};
\node[Kunyomi] at (-40.150000, -17.300000) {よ.る};
\node[Meaning] at (-40.100000, -15.650000) {reliant};
\node[Square] at (-38.050000, -17.400000) {};
\node[Kanji] at (-38.050000, -16.900000) {控};
\node[Onyomi] at (-38.000000, -17.300000) {コウ};
\node[Kunyomi] at (-38.100000, -17.300000) {ひか};
\node[Meaning] at (-38.050000, -15.650000) {abstain};
\node[Square] at (-36.000000, -17.400000) {};
\node[Kanji] at (-36.000000, -16.900000) {露};
\node[Onyomi] at (-35.950000, -17.300000) {ロ};
\node[Kunyomi] at (-36.050000, -17.300000) {つゆ};
\node[Meaning] at (-36.000000, -15.650000) {expose};
\node[Square] at (-33.950000, -17.400000) {};
\node[Kanji] at (-33.950000, -16.900000) {扇};
\node[Onyomi] at (-33.900000, -17.300000) {セン};
\node[Kunyomi] at (-34.000000, -17.300000) {おうぎ};
\node[Meaning] at (-33.950000, -15.650000) {folding fan};
\node[Square] at (-31.900000, -17.400000) {};
\node[Kanji] at (-31.900000, -16.900000) {栓};
\node[Onyomi] at (-31.850000, -17.300000) {セン};
\node[Meaning] at (-31.900000, -15.650000) {cork};
\node[Square] at (-29.850000, -17.400000) {};
\node[Kanji] at (-29.850000, -16.900000) {凄};
\node[Onyomi] at (-29.800000, -17.300000) {セイ};
\node[Meaning] at (-29.850000, -15.650000) {uncanny};
\node[Square] at (-27.800000, -17.400000) {};
\node[Kanji] at (-27.800000, -16.900000) {溺};
\node[Onyomi] at (-27.750000, -17.300000) {デキ};
\node[Kunyomi] at (-27.850000, -17.300000) {おぼ-れる};
\node[Meaning] at (-27.800000, -15.650000) {drown};
\node[Square] at (-25.750000, -17.400000) {};
\node[Kanji] at (-25.750000, -16.900000) {徳};
\node[Onyomi] at (-25.700000, -17.300000) {トク};
\node[Meaning] at (-25.750000, -15.650000) {virtue};
\node[Square] at (-23.700000, -17.400000) {};
\node[Kanji] at (-23.700000, -16.900000) {鎌};
\node[Onyomi] at (-23.650000, -17.300000) {ケン};
\node[Kunyomi] at (-23.750000, -17.300000) {かま};
\node[Meaning] at (-23.700000, -15.650000) {sickle};
\node[Square] at (-21.650000, -17.400000) {};
\node[Kanji] at (-21.650000, -16.900000) {腸};
\node[Onyomi] at (-21.600000, -17.300000) {チョウ};
\node[Kunyomi] at (-21.700000, -17.300000) {はらわた};
\node[Meaning] at (-21.650000, -15.650000) {intestines};
\node[Square] at (-19.600000, -17.400000) {};
\node[Kanji] at (-19.600000, -16.900000) {卑};
\node[Onyomi] at (-19.550000, -17.300000) {ヒ};
\node[Kunyomi] at (-19.650000, -17.300000) {いや};
\node[Meaning] at (-19.600000, -15.650000) {lowly};
\node[Square] at (-17.550000, -17.400000) {};
\node[Kanji] at (-17.550000, -16.900000) {臆};
\node[Onyomi] at (-17.500000, -17.300000) {オク};
\node[Meaning] at (-17.550000, -15.650000) {timidity};
\node[Square] at (-15.500000, -17.400000) {};
\node[Kanji] at (-15.500000, -16.900000) {喫};
\node[Onyomi] at (-15.450000, -17.300000) {キツ};
\node[Kunyomi] at (-15.550000, -17.300000) {の.む};
\node[Meaning] at (-15.500000, -15.650000) {consume};
\node[Square] at (-13.450000, -17.400000) {};
\node[Kanji] at (-13.450000, -16.900000) {憲};
\node[Onyomi] at (-13.400000, -17.300000) {ケン};
\node[Meaning] at (-13.450000, -15.650000) {constitution};
\node[Square] at (-11.400000, -17.400000) {};
\node[Kanji] at (-11.400000, -16.900000) {括};
\node[Onyomi] at (-11.350000, -17.300000) {カツ};
\node[Kunyomi] at (-11.450000, -17.300000) {くく.る};
\node[Meaning] at (-11.400000, -15.650000) {fasten};
\node[Square] at (-9.350000, -17.400000) {};
\node[Kanji] at (-9.350000, -16.900000) {牧};
\node[Onyomi] at (-9.300000, -17.300000) {ボク};
\node[Kunyomi] at (-9.400000, -17.300000) {まき};
\node[Meaning] at (-9.350000, -15.650000) {pasture};
\node[Square] at (-7.300000, -17.400000) {};
\node[Kanji] at (-7.300000, -16.900000) {粧};
\node[Onyomi] at (-7.250000, -17.300000) {ショウ};
\node[Meaning] at (-7.300000, -15.650000) {cosmetics};
\node[Square] at (-5.250000, -17.400000) {};
\node[Kanji] at (-5.250000, -16.900000) {隻};
\node[Onyomi] at (-5.200000, -17.300000) {セキ};
\node[Meaning] at (-5.250000, -15.650000) {ship counter};
\node[Square] at (-3.200000, -17.400000) {};
\node[Kanji] at (-3.200000, -16.900000) {胡};
\node[Onyomi] at (-3.150000, -17.300000) {コ};
\node[Kunyomi] at (-3.250000, -17.300000) {なんぞ};
\node[Meaning] at (-3.200000, -15.650000) {barbarian};
\node[Square] at (-1.150000, -17.400000) {};
\node[Kanji] at (-1.150000, -16.900000) {詩};
\node[Onyomi] at (-1.100000, -17.300000) {シ};
\node[Kunyomi] at (-1.200000, -17.300000) {し};
\node[Meaning] at (-1.150000, -15.650000) {poem};
\node[Square] at (0.900000, -17.400000) {};
\node[Kanji] at (0.900000, -16.900000) {版};
\node[Onyomi] at (0.950000, -17.300000) {ハン};
\node[Meaning] at (0.900000, -15.650000) {edition};
\node[Square] at (2.950000, -17.400000) {};
\node[Kanji] at (2.950000, -16.900000) {寸};
\node[Onyomi] at (3.000000, -17.300000) {スン};
\node[Meaning] at (2.950000, -15.650000) {measurement};
\node[Square] at (5.000000, -17.400000) {};
\node[Kanji] at (5.000000, -16.900000) {錯};
\node[Onyomi] at (5.050000, -17.300000) {サク};
\node[Meaning] at (5.000000, -15.650000) {confused};
\node[Square] at (7.050000, -17.400000) {};
\node[Kanji] at (7.050000, -16.900000) {濁};
\node[Onyomi] at (7.100000, -17.300000) {ダク};
\node[Kunyomi] at (7.000000, -17.300000) {にご};
\node[Meaning] at (7.050000, -15.650000) {muddy};
\node[Square] at (9.100000, -17.400000) {};
\node[Kanji] at (9.100000, -16.900000) {爵};
\node[Onyomi] at (9.150000, -17.300000) {シャク};
\node[Meaning] at (9.100000, -15.650000) {baron};
\node[Square] at (11.150000, -17.400000) {};
\node[Kanji] at (11.150000, -16.900000) {略};
\node[Onyomi] at (11.200000, -17.300000) {リャク};
\node[Kunyomi] at (11.100000, -17.300000) {りゃく.す};
\node[Meaning] at (11.150000, -15.650000) {abbreviation};
\node[Square] at (13.200000, -17.400000) {};
\node[Kanji] at (13.200000, -16.900000) {拡};
\node[Onyomi] at (13.250000, -17.300000) {カク};
\node[Kunyomi] at (13.150000, -17.300000) {ひろ.がる};
\node[Meaning] at (13.200000, -15.650000) {extend};
\node[Square] at (15.250000, -17.400000) {};
\node[Kanji] at (15.250000, -16.900000) {岐};
\node[Onyomi] at (15.300000, -17.300000) {キ};
\node[Meaning] at (15.250000, -15.650000) {branch off};
\node[Square] at (17.300000, -17.400000) {};
\node[Kanji] at (17.300000, -16.900000) {懲};
\node[Onyomi] at (17.350000, -17.300000) {チョウ};
\node[Kunyomi] at (17.250000, -17.300000) {こ.りる};
\node[Meaning] at (17.300000, -15.650000) {chastise};
\node[Square] at (19.350000, -17.400000) {};
\node[Kanji] at (19.350000, -16.900000) {也};
\node[Kunyomi] at (19.300000, -17.300000) {なり};
\node[Meaning] at (19.350000, -15.650000) {considerably};
\node[Square] at (21.400000, -17.400000) {};
\node[Kanji] at (21.400000, -16.900000) {憩};
\node[Onyomi] at (21.450000, -17.300000) {ケイ};
\node[Kunyomi] at (21.350000, -17.300000) {いこ.い};
\node[Meaning] at (21.400000, -15.650000) {rest};
\node[Square] at (23.450000, -17.400000) {};
\node[Kanji] at (23.450000, -16.900000) {刈};
\node[Kunyomi] at (23.400000, -17.300000) {か};
\node[Meaning] at (23.450000, -15.650000) {prune};
\node[Square] at (25.500000, -17.400000) {};
\node[Kanji] at (25.500000, -16.900000) {隔};
\node[Onyomi] at (25.550000, -17.300000) {カク};
\node[Kunyomi] at (25.450000, -17.300000) {へだ.*};
\node[Meaning] at (25.500000, -15.650000) {isolate};
\node[Square] at (27.550000, -17.400000) {};
\node[Kanji] at (27.550000, -16.900000) {訂};
\node[Onyomi] at (27.600000, -17.300000) {テイ};
\node[Meaning] at (27.550000, -15.650000) {revise};
\node[Square] at (29.600000, -17.400000) {};
\node[Kanji] at (29.600000, -16.900000) {覇};
\node[Onyomi] at (29.650000, -17.300000) {ハ};
\node[Kunyomi] at (29.550000, -17.300000) {はたがしら};
\node[Meaning] at (29.600000, -15.650000) {leadership};
\node[Square] at (31.650000, -17.400000) {};
\node[Kanji] at (31.650000, -16.900000) {遇};
\node[Onyomi] at (31.700000, -17.300000) {グウ};
\node[Kunyomi] at (31.600000, -17.300000) {あ};
\node[Meaning] at (31.650000, -15.650000) {treatment};
\node[Square] at (33.700000, -17.400000) {};
\node[Kanji] at (33.700000, -16.900000) {羅};
\node[Onyomi] at (33.750000, -17.300000) {ラ};
\node[Kunyomi] at (33.650000, -17.300000) {うすもの        };
\node[Meaning] at (33.700000, -15.650000) {spread out};
\node[Square] at (35.750000, -17.400000) {};
\node[Kanji] at (35.750000, -16.900000) {膜};
\node[Onyomi] at (35.800000, -17.300000) {マク};
\node[Meaning] at (35.750000, -15.650000) {membrane};
\node[Square] at (37.800000, -17.400000) {};
\node[Kanji] at (37.800000, -16.900000) {繕};
\node[Onyomi] at (37.850000, -17.300000) {ゼン};
\node[Kunyomi] at (37.750000, -17.300000) {つくろ-う};
\node[Meaning] at (37.800000, -15.650000) {darning};
\node[Square] at (39.850000, -17.400000) {};
\node[Kanji] at (39.850000, -16.900000) {昆};
\node[Onyomi] at (39.900000, -17.300000) {コン};
\node[Meaning] at (39.850000, -15.650000) {descendants};
\node[Square] at (41.900000, -17.400000) {};
\node[Kanji] at (41.900000, -16.900000) {臨};
\node[Onyomi] at (41.950000, -17.300000) {リン};
\node[Kunyomi] at (41.850000, -17.300000) {のぞ.む};
\node[Meaning] at (41.900000, -15.650000) {look to};
\node[Square] at (43.950000, -17.400000) {};
\node[Kanji] at (43.950000, -16.900000) {刷};
\node[Onyomi] at (44.000000, -17.300000) {サツ};
\node[Kunyomi] at (43.900000, -17.300000) {す.る};
\node[Meaning] at (43.950000, -15.650000) {printing};
\node[Square] at (46.000000, -17.400000) {};
\node[Kanji] at (46.000000, -16.900000) {孔};
\node[Onyomi] at (46.050000, -17.300000) {コウ};
\node[Kunyomi] at (45.950000, -17.300000) {あな};
\node[Meaning] at (46.000000, -15.650000) {cavity};
\node[Square] at (48.050000, -17.400000) {};
\node[Kanji] at (48.050000, -16.900000) {潤};
\node[Onyomi] at (48.100000, -17.300000) {ジュン};
\node[Kunyomi] at (48.000000, -17.300000) {うるお.*};
\node[Meaning] at (48.050000, -15.650000) {watered};
\node[Square] at (50.100000, -17.400000) {};
\node[Kanji] at (50.100000, -16.900000) {軟};
\node[Onyomi] at (50.150000, -17.300000) {ナン};
\node[Kunyomi] at (50.050000, -17.300000) {やわ};
\node[Meaning] at (50.100000, -15.650000) {soft};
\node[Square] at (52.150000, -17.400000) {};
\node[Kanji] at (52.150000, -16.900000) {傲};
\node[Onyomi] at (52.200000, -17.300000) {ゴウ};
\node[Kunyomi] at (52.100000, -17.300000) {あなど};
\node[Meaning] at (52.150000, -15.650000) {proud};
\node[Square] at (54.200000, -17.400000) {};
\node[Kanji] at (54.200000, -16.900000) {尿};
\node[Onyomi] at (54.250000, -17.300000) {ニョウ};
\node[Meaning] at (54.200000, -15.650000) {urine};
\node[Square] at (56.250000, -17.400000) {};
\node[Kanji] at (56.250000, -16.900000) {凹};
\node[Onyomi] at (56.300000, -17.300000) {オウ};
\node[Kunyomi] at (56.200000, -17.300000) {くぼ.む};
\node[Meaning] at (56.250000, -15.650000) {concave};
\node[Meaning] at (-58.500000, -16.850000) {97.80\%};
\node[Square] at (-56.500000, -19.450000) {};
\node[Kanji] at (-56.500000, -18.950000) {崇};
\node[Onyomi] at (-56.450000, -19.350000) {スウ};
\node[Kunyomi] at (-56.550000, -19.350000) {あが};
\node[Meaning] at (-56.500000, -17.700000) {worship};
\node[Square] at (-54.450000, -19.450000) {};
\node[Kanji] at (-54.450000, -18.950000) {曖};
\node[Onyomi] at (-54.400000, -19.350000) {アイ};
\node[Meaning] at (-54.450000, -17.700000) {not clear};
\node[Square] at (-52.400000, -19.450000) {};
\node[Kanji] at (-52.400000, -18.950000) {妬};
\node[Onyomi] at (-52.350000, -19.350000) {ト};
\node[Kunyomi] at (-52.450000, -19.350000) {ねた-む};
\node[Meaning] at (-52.400000, -17.700000) {jealousy};
\node[Square] at (-50.350000, -19.450000) {};
\node[Kanji] at (-50.350000, -18.950000) {昧};
\node[Onyomi] at (-50.300000, -19.350000) {マイ};
\node[Meaning] at (-50.350000, -17.700000) {foolish};
\node[Square] at (-48.300000, -19.450000) {};
\node[Kanji] at (-48.300000, -18.950000) {冥};
\node[Onyomi] at (-48.250000, -19.350000) {メイ};
\node[Meaning] at (-48.300000, -17.700000) {dark};
\node[Square] at (-46.250000, -19.450000) {};
\node[Kanji] at (-46.250000, -18.950000) {討};
\node[Onyomi] at (-46.200000, -19.350000) {トウ};
\node[Meaning] at (-46.250000, -17.700000) {chastise};
\node[Square] at (-44.200000, -19.450000) {};
\node[Kanji] at (-44.200000, -18.950000) {丼};
\node[Onyomi] at (-44.150000, -19.350000) {ドン};
\node[Kunyomi] at (-44.250000, -19.350000) {どんぶり};
\node[Meaning] at (-44.200000, -17.700000) {rice bowl};
\node[Square] at (-42.150000, -19.450000) {};
\node[Kanji] at (-42.150000, -18.950000) {圏};
\node[Onyomi] at (-42.100000, -19.350000) {ケン};
\node[Meaning] at (-42.150000, -17.700000) {range};
\node[Square] at (-40.100000, -19.450000) {};
\node[Kanji] at (-40.100000, -18.950000) {頻};
\node[Onyomi] at (-40.050000, -19.350000) {ヒン};
\node[Kunyomi] at (-40.150000, -19.350000) {しき.りに};
\node[Meaning] at (-40.100000, -17.700000) {frequent};
\node[Square] at (-38.050000, -19.450000) {};
\node[Kanji] at (-38.050000, -18.950000) {芯};
\node[Onyomi] at (-38.000000, -19.350000) {シン};
\node[Meaning] at (-38.050000, -17.700000) {wick};
\node[Square] at (-36.000000, -19.450000) {};
\node[Kanji] at (-36.000000, -18.950000) {墜};
\node[Onyomi] at (-35.950000, -19.350000) {ツイ};
\node[Meaning] at (-36.000000, -17.700000) {crash};
\node[Square] at (-33.950000, -19.450000) {};
\node[Kanji] at (-33.950000, -18.950000) {簿};
\node[Onyomi] at (-33.900000, -19.350000) {ボ};
\node[Meaning] at (-33.950000, -17.700000) {record book};
\node[Square] at (-31.900000, -19.450000) {};
\node[Kanji] at (-31.900000, -18.950000) {顕};
\node[Onyomi] at (-31.850000, -19.350000) {ケン};
\node[Kunyomi] at (-31.950000, -19.350000) {あきらか};
\node[Meaning] at (-31.900000, -17.700000) {appear};
\node[Square] at (-29.850000, -19.450000) {};
\node[Kanji] at (-29.850000, -18.950000) {栃};
\node[Kunyomi] at (-29.900000, -19.350000) {とち};
\node[Meaning] at (-29.850000, -17.700000) {horse chestnut};
\node[Square] at (-27.800000, -19.450000) {};
\node[Kanji] at (-27.800000, -18.950000) {賓};
\node[Onyomi] at (-27.750000, -19.350000) {ヒン};
\node[Meaning] at (-27.800000, -17.700000) {vip};
\node[Square] at (-25.750000, -19.450000) {};
\node[Kanji] at (-25.750000, -18.950000) {喝};
\node[Onyomi] at (-25.700000, -19.350000) {カツ};
\node[Meaning] at (-25.750000, -17.700000) {scold};
\node[Square] at (-23.700000, -19.450000) {};
\node[Kanji] at (-23.700000, -18.950000) {雌};
\node[Onyomi] at (-23.650000, -19.350000) {シ};
\node[Kunyomi] at (-23.750000, -19.350000) {めす};
\node[Meaning] at (-23.700000, -17.700000) {female};
\node[Square] at (-21.650000, -19.450000) {};
\node[Kanji] at (-21.650000, -18.950000) {葛};
\node[Onyomi] at (-21.600000, -19.350000) {カツ};
\node[Kunyomi] at (-21.700000, -19.350000) {くず};
\node[Meaning] at (-21.650000, -17.700000) {arrowroot};
\node[Square] at (-19.600000, -19.450000) {};
\node[Kanji] at (-19.600000, -18.950000) {挫};
\node[Onyomi] at (-19.550000, -19.350000) {ザ};
\node[Meaning] at (-19.600000, -17.700000) {sprain};
\node[Square] at (-17.550000, -19.450000) {};
\node[Kanji] at (-17.550000, -18.950000) {拉};
\node[Onyomi] at (-17.500000, -19.350000) {ラ};
\node[Meaning] at (-17.550000, -17.700000) {crush};
\node[Square] at (-15.500000, -19.450000) {};
\node[Kanji] at (-15.500000, -18.950000) {賃};
\node[Onyomi] at (-15.450000, -19.350000) {チン};
\node[Meaning] at (-15.500000, -17.700000) {rent};
\node[Square] at (-13.450000, -19.450000) {};
\node[Kanji] at (-13.450000, -18.950000) {称};
\node[Onyomi] at (-13.400000, -19.350000) {ショウ};
\node[Kunyomi] at (-13.500000, -19.350000) {とな.える};
\node[Meaning] at (-13.450000, -17.700000) {title};
\node[Square] at (-11.400000, -19.450000) {};
\node[Kanji] at (-11.400000, -18.950000) {項};
\node[Onyomi] at (-11.350000, -19.350000) {コウ};
\node[Meaning] at (-11.400000, -17.700000) {paragraph};
\node[Square] at (-9.350000, -19.450000) {};
\node[Kanji] at (-9.350000, -18.950000) {祉};
\node[Onyomi] at (-9.300000, -19.350000) {シ};
\node[Meaning] at (-9.350000, -17.700000) {welfare};
\node[Square] at (-7.300000, -19.450000) {};
\node[Kanji] at (-7.300000, -18.950000) {徹};
\node[Onyomi] at (-7.250000, -19.350000) {テツ};
\node[Kunyomi] at (-7.350000, -19.350000) {てっ.する};
\node[Meaning] at (-7.300000, -17.700000) {penetrate};
\node[Square] at (-5.250000, -19.450000) {};
\node[Kanji] at (-5.250000, -18.950000) {綿};
\node[Onyomi] at (-5.200000, -19.350000) {メン};
\node[Kunyomi] at (-5.300000, -19.350000) {わた};
\node[Meaning] at (-5.250000, -17.700000) {cotton};
\node[Square] at (-3.200000, -19.450000) {};
\node[Kanji] at (-3.200000, -18.950000) {菊};
\node[Onyomi] at (-3.150000, -19.350000) {キク};
\node[Meaning] at (-3.200000, -17.700000) {chrysanthemum};
\node[Square] at (-1.150000, -19.450000) {};
\node[Kanji] at (-1.150000, -18.950000) {箸};
\node[Onyomi] at (-1.100000, -19.350000) {チャク};
\node[Kunyomi] at (-1.200000, -19.350000) {はし};
\node[Meaning] at (-1.150000, -17.700000) {chopsticks};
\node[Square] at (0.900000, -19.450000) {};
\node[Kanji] at (0.900000, -18.950000) {征};
\node[Onyomi] at (0.950000, -19.350000) {セイ};
\node[Meaning] at (0.900000, -17.700000) {subjugate};
\node[Square] at (2.950000, -19.450000) {};
\node[Kanji] at (2.950000, -18.950000) {堤};
\node[Onyomi] at (3.000000, -19.350000) {テイ};
\node[Kunyomi] at (2.900000, -19.350000) {つつみ};
\node[Meaning] at (2.950000, -17.700000) {embankment};
\node[Square] at (5.000000, -19.450000) {};
\node[Kanji] at (5.000000, -18.950000) {窮};
\node[Onyomi] at (5.050000, -19.350000) {キュウ};
\node[Kunyomi] at (4.950000, -19.350000) {きわ};
\node[Meaning] at (5.000000, -17.700000) {destitute};
\node[Square] at (7.050000, -19.450000) {};
\node[Kanji] at (7.050000, -18.950000) {浪};
\node[Onyomi] at (7.100000, -19.350000) {ロウ};
\node[Meaning] at (7.050000, -17.700000) {wander};
\node[Square] at (9.100000, -19.450000) {};
\node[Kanji] at (9.100000, -18.950000) {稽};
\node[Onyomi] at (9.150000, -19.350000) {ケイ};
\node[Meaning] at (9.100000, -17.700000) {consider};
\node[Square] at (11.150000, -19.450000) {};
\node[Kanji] at (11.150000, -18.950000) {餅};
\node[Onyomi] at (11.200000, -19.350000) {ヘイ};
\node[Kunyomi] at (11.100000, -19.350000) {もち};
\node[Meaning] at (11.150000, -17.700000) {mochi};
\node[Square] at (13.200000, -19.450000) {};
\node[Kanji] at (13.200000, -18.950000) {肥};
\node[Onyomi] at (13.250000, -19.350000) {ヒ};
\node[Kunyomi] at (13.150000, -19.350000) {こ.える};
\node[Meaning] at (13.200000, -17.700000) {obese};
\node[Square] at (15.250000, -19.450000) {};
\node[Kanji] at (15.250000, -18.950000) {貧};
\node[Onyomi] at (15.300000, -19.350000) {ビン};
\node[Kunyomi] at (15.200000, -19.350000) {まず.しい};
\node[Meaning] at (15.250000, -17.700000) {poor};
\node[Square] at (17.300000, -19.450000) {};
\node[Kanji] at (17.300000, -18.950000) {阜};
\node[Onyomi] at (17.350000, -19.350000) {フ};
\node[Meaning] at (17.300000, -17.700000) {mound};
\node[Square] at (19.350000, -19.450000) {};
\node[Kanji] at (19.350000, -18.950000) {敏};
\node[Onyomi] at (19.400000, -19.350000) {ビン};
\node[Meaning] at (19.350000, -17.700000) {alert};
\node[Square] at (21.400000, -19.450000) {};
\node[Kanji] at (21.400000, -18.950000) {墨};
\node[Kunyomi] at (21.350000, -19.350000) {すみ};
\node[Meaning] at (21.400000, -17.700000) {black ink};
\node[Square] at (23.450000, -19.450000) {};
\node[Kanji] at (23.450000, -18.950000) {駄};
\node[Onyomi] at (23.500000, -19.350000) {ダ};
\node[Meaning] at (23.450000, -17.700000) {burdensome};
\node[Square] at (25.500000, -19.450000) {};
\node[Kanji] at (25.500000, -18.950000) {郊};
\node[Onyomi] at (25.550000, -19.350000) {コウ};
\node[Meaning] at (25.500000, -17.700000) {suburbs};
\node[Square] at (27.550000, -19.450000) {};
\node[Kanji] at (27.550000, -18.950000) {錠};
\node[Onyomi] at (27.600000, -19.350000) {ジョウ};
\node[Meaning] at (27.550000, -17.700000) {lock};
\node[Square] at (29.600000, -19.450000) {};
\node[Kanji] at (29.600000, -18.950000) {畏};
\node[Onyomi] at (29.650000, -19.350000) {イ};
\node[Kunyomi] at (29.550000, -19.350000) {おそ-れる};
\node[Meaning] at (29.600000, -17.700000) {fear};
\node[Square] at (31.650000, -19.450000) {};
\node[Kanji] at (31.650000, -18.950000) {翁};
\node[Onyomi] at (31.700000, -19.350000) {オウ};
\node[Meaning] at (31.650000, -17.700000) {old man};
\node[Square] at (33.700000, -19.450000) {};
\node[Kanji] at (33.700000, -18.950000) {漠};
\node[Onyomi] at (33.750000, -19.350000) {バク};
\node[Meaning] at (33.700000, -17.700000) {desert};
\node[Square] at (35.750000, -19.450000) {};
\node[Kanji] at (35.750000, -18.950000) {献};
\node[Onyomi] at (35.800000, -19.350000) {ケン};
\node[Kunyomi] at (35.700000, -19.350000) {たてまつ.る};
\node[Meaning] at (35.750000, -17.700000) {offer};
\node[Square] at (37.800000, -19.450000) {};
\node[Kanji] at (37.800000, -18.950000) {隆};
\node[Onyomi] at (37.850000, -19.350000) {リュウ};
\node[Meaning] at (37.800000, -17.700000) {prosperity};
\node[Square] at (39.850000, -19.450000) {};
\node[Kanji] at (39.850000, -18.950000) {克};
\node[Onyomi] at (39.900000, -19.350000) {コク};
\node[Meaning] at (39.850000, -17.700000) {overcome};
\node[Square] at (41.900000, -19.450000) {};
\node[Kanji] at (41.900000, -18.950000) {飢};
\node[Onyomi] at (41.950000, -19.350000) {キ};
\node[Kunyomi] at (41.850000, -19.350000) {う.える};
\node[Meaning] at (41.900000, -17.700000) {starve};
\node[Square] at (43.950000, -19.450000) {};
\node[Kanji] at (43.950000, -18.950000) {丹};
\node[Onyomi] at (44.000000, -19.350000) {タン};
\node[Kunyomi] at (43.900000, -19.350000) {に};
\node[Meaning] at (43.950000, -17.700000) {rust colored};
\node[Square] at (46.000000, -19.450000) {};
\node[Kanji] at (46.000000, -18.950000) {茎};
\node[Onyomi] at (46.050000, -19.350000) {キョウ};
\node[Kunyomi] at (45.950000, -19.350000) {くき};
\node[Meaning] at (46.000000, -17.700000) {stem};
\node[Square] at (48.050000, -19.450000) {};
\node[Kanji] at (48.050000, -18.950000) {睡};
\node[Onyomi] at (48.100000, -19.350000) {スイ};
\node[Meaning] at (48.050000, -17.700000) {drowsy};
\node[Square] at (50.100000, -19.450000) {};
\node[Kanji] at (50.100000, -18.950000) {僚};
\node[Onyomi] at (50.150000, -19.350000) {リョウ};
\node[Meaning] at (50.100000, -17.700000) {colleague};
\node[Square] at (52.150000, -19.450000) {};
\node[Kanji] at (52.150000, -18.950000) {隷};
\node[Onyomi] at (52.200000, -19.350000) {レイ};
\node[Meaning] at (52.150000, -17.700000) {slave};
\node[Square] at (54.200000, -19.450000) {};
\node[Kanji] at (54.200000, -18.950000) {杉};
\node[Kunyomi] at (54.150000, -19.350000) {すぎ};
\node[Meaning] at (54.200000, -17.700000) {cedar};
\node[Square] at (56.250000, -19.450000) {};
\node[Kanji] at (56.250000, -18.950000) {償};
\node[Onyomi] at (56.300000, -19.350000) {ショウ};
\node[Kunyomi] at (56.200000, -19.350000) {つぐな.う};
\node[Meaning] at (56.250000, -17.700000) {reparation};
\node[Meaning] at (-58.500000, -18.900000) {97.94\%};
\node[Square] at (-56.500000, -21.500000) {};
\node[Kanji] at (-56.500000, -21.000000) {邦};
\node[Onyomi] at (-56.450000, -21.400000) {ホウ};
\node[Kunyomi] at (-56.550000, -21.400000) {くに};
\node[Meaning] at (-56.500000, -19.750000) {home country};
\node[Square] at (-54.450000, -21.500000) {};
\node[Kanji] at (-54.450000, -21.000000) {棋};
\node[Onyomi] at (-54.400000, -21.400000) {キ};
\node[Kunyomi] at (-54.500000, -21.400000) {ご};
\node[Meaning] at (-54.450000, -19.750000) {chess piece};
\node[Square] at (-52.400000, -21.500000) {};
\node[Kanji] at (-52.400000, -21.000000) {排};
\node[Onyomi] at (-52.350000, -21.400000) {ハイ};
\node[Meaning] at (-52.400000, -19.750000) {reject};
\node[Square] at (-50.350000, -21.500000) {};
\node[Kanji] at (-50.350000, -21.000000) {郷};
\node[Onyomi] at (-50.300000, -21.400000) {キョウ};
\node[Meaning] at (-50.350000, -19.750000) {hometown};
\node[Square] at (-48.300000, -21.500000) {};
\node[Kanji] at (-48.300000, -21.000000) {鶴};
\node[Onyomi] at (-48.250000, -21.400000) {カク};
\node[Kunyomi] at (-48.350000, -21.400000) {つる};
\node[Meaning] at (-48.300000, -19.750000) {crane};
\node[Square] at (-46.250000, -21.500000) {};
\node[Kanji] at (-46.250000, -21.000000) {珠};
\node[Onyomi] at (-46.200000, -21.400000) {シュ};
\node[Kunyomi] at (-46.300000, -21.400000) {たましい};
\node[Meaning] at (-46.250000, -19.750000) {pearl};
\node[Square] at (-44.200000, -21.500000) {};
\node[Kanji] at (-44.200000, -21.000000) {奨};
\node[Onyomi] at (-44.150000, -21.400000) {ショウ};
\node[Meaning] at (-44.200000, -19.750000) {encourage};
\node[Square] at (-42.150000, -21.500000) {};
\node[Kanji] at (-42.150000, -21.000000) {陶};
\node[Onyomi] at (-42.100000, -21.400000) {トウ};
\node[Meaning] at (-42.150000, -19.750000) {pottery};
\node[Square] at (-40.100000, -21.500000) {};
\node[Kanji] at (-40.100000, -21.000000) {涯};
\node[Onyomi] at (-40.050000, -21.400000) {ガイ};
\node[Kunyomi] at (-40.150000, -21.400000) {はて};
\node[Meaning] at (-40.100000, -19.750000) {horizon};
\node[Square] at (-38.050000, -21.500000) {};
\node[Kanji] at (-38.050000, -21.000000) {繊};
\node[Onyomi] at (-38.000000, -21.400000) {セン};
\node[Meaning] at (-38.050000, -19.750000) {fiber};
\node[Square] at (-36.000000, -21.500000) {};
\node[Kanji] at (-36.000000, -21.000000) {蝶};
\node[Onyomi] at (-35.950000, -21.400000) {チョウ};
\node[Meaning] at (-36.000000, -19.750000) {butterfly};
\node[Square] at (-33.950000, -21.500000) {};
\node[Kanji] at (-33.950000, -21.000000) {詮};
\node[Onyomi] at (-33.900000, -21.400000) {セン};
\node[Meaning] at (-33.950000, -19.750000) {discussion};
\node[Square] at (-31.900000, -21.500000) {};
\node[Kanji] at (-31.900000, -21.000000) {罵};
\node[Onyomi] at (-31.850000, -21.400000) {バ};
\node[Kunyomi] at (-31.950000, -21.400000) {ののし-る};
\node[Meaning] at (-31.900000, -19.750000) {insult};
\node[Square] at (-29.850000, -21.500000) {};
\node[Kanji] at (-29.850000, -21.000000) {粋};
\node[Onyomi] at (-29.800000, -21.400000) {スイ};
\node[Kunyomi] at (-29.900000, -21.400000) {いき};
\node[Meaning] at (-29.850000, -19.750000) {stylish};
\node[Square] at (-27.800000, -21.500000) {};
\node[Kanji] at (-27.800000, -21.000000) {譲};
\node[Onyomi] at (-27.750000, -21.400000) {ジョウ};
\node[Kunyomi] at (-27.850000, -21.400000) {ゆず.る};
\node[Meaning] at (-27.800000, -19.750000) {defer};
\node[Square] at (-25.750000, -21.500000) {};
\node[Kanji] at (-25.750000, -21.000000) {姫};
\node[Kunyomi] at (-25.800000, -21.400000) {ひめ};
\node[Meaning] at (-25.750000, -19.750000) {princess};
\node[Square] at (-23.700000, -21.500000) {};
\node[Kanji] at (-23.700000, -21.000000) {鳩};
\node[Onyomi] at (-23.650000, -21.400000) {ク};
\node[Kunyomi] at (-23.750000, -21.400000) {はと};
\node[Meaning] at (-23.700000, -19.750000) {dove};
\node[Square] at (-21.650000, -21.500000) {};
\node[Kanji] at (-21.650000, -21.000000) {恒};
\node[Onyomi] at (-21.600000, -21.400000) {コウ};
\node[Kunyomi] at (-21.700000, -21.400000) {つね};
\node[Meaning] at (-21.650000, -19.750000) {constant};
\node[Square] at (-19.600000, -21.500000) {};
\node[Kanji] at (-19.600000, -21.000000) {漬};
\node[Onyomi] at (-19.550000, -21.400000) {シ};
\node[Kunyomi] at (-19.650000, -21.400000) {つ};
\node[Meaning] at (-19.600000, -19.750000) {pickle};
\node[Square] at (-17.550000, -21.500000) {};
\node[Kanji] at (-17.550000, -21.000000) {吟};
\node[Onyomi] at (-17.500000, -21.400000) {ギン};
\node[Meaning] at (-17.550000, -19.750000) {recital};
\node[Square] at (-15.500000, -21.500000) {};
\node[Kanji] at (-15.500000, -21.000000) {沙};
\node[Onyomi] at (-15.450000, -21.400000) {サ};
\node[Kunyomi] at (-15.550000, -21.400000) {すな};
\node[Meaning] at (-15.500000, -19.750000) {sand};
\node[Square] at (-13.450000, -21.500000) {};
\node[Kanji] at (-13.450000, -21.000000) {艶};
\node[Onyomi] at (-13.400000, -21.400000) {エン};
\node[Kunyomi] at (-13.500000, -21.400000) {つや};
\node[Meaning] at (-13.450000, -19.750000) {glossy};
\node[Square] at (-11.400000, -21.500000) {};
\node[Kanji] at (-11.400000, -21.000000) {瓦};
\node[Onyomi] at (-11.350000, -21.400000) {ガ};
\node[Kunyomi] at (-11.450000, -21.400000) {かわら};
\node[Meaning] at (-11.400000, -19.750000) {tile};
\node[Square] at (-9.350000, -21.500000) {};
\node[Kanji] at (-9.350000, -21.000000) {詞};
\node[Onyomi] at (-9.300000, -21.400000) {シ};
\node[Meaning] at (-9.350000, -19.750000) {part of speech};
\node[Square] at (-7.300000, -21.500000) {};
\node[Kanji] at (-7.300000, -21.000000) {麗};
\node[Onyomi] at (-7.250000, -21.400000) {レイ};
\node[Kunyomi] at (-7.350000, -21.400000) {うるわ.しい};
\node[Meaning] at (-7.300000, -19.750000) {lovely};
\node[Square] at (-5.250000, -21.500000) {};
\node[Kanji] at (-5.250000, -21.000000) {幾};
\node[Onyomi] at (-5.200000, -21.400000) {キ};
\node[Kunyomi] at (-5.300000, -21.400000) {いく};
\node[Meaning] at (-5.250000, -19.750000) {how many};
\node[Square] at (-3.200000, -21.500000) {};
\node[Kanji] at (-3.200000, -21.000000) {堅};
\node[Onyomi] at (-3.150000, -21.400000) {ケン};
\node[Kunyomi] at (-3.250000, -21.400000) {かた};
\node[Meaning] at (-3.200000, -19.750000) {solid};
\node[Square] at (-1.150000, -21.500000) {};
\node[Kanji] at (-1.150000, -21.000000) {撲};
\node[Onyomi] at (-1.100000, -21.400000) {ボク};
\node[Meaning] at (-1.150000, -19.750000) {slap};
\node[Square] at (0.900000, -21.500000) {};
\node[Kanji] at (0.900000, -21.000000) {凶};
\node[Onyomi] at (0.950000, -21.400000) {キョウ};
\node[Meaning] at (0.900000, -19.750000) {villain};
\node[Square] at (2.950000, -21.500000) {};
\node[Kanji] at (2.950000, -21.000000) {溝};
\node[Onyomi] at (3.000000, -21.400000) {コウ};
\node[Kunyomi] at (2.900000, -21.400000) {みぞ};
\node[Meaning] at (2.950000, -19.750000) {gutter};
\node[Square] at (5.000000, -21.500000) {};
\node[Kanji] at (5.000000, -21.000000) {須};
\node[Onyomi] at (5.050000, -21.400000) {ス};
\node[Kunyomi] at (4.950000, -21.400000) {すべから};
\node[Meaning] at (5.000000, -19.750000) {necessary};
\node[Square] at (7.050000, -21.500000) {};
\node[Kanji] at (7.050000, -21.000000) {渇};
\node[Onyomi] at (7.100000, -21.400000) {カツ};
\node[Kunyomi] at (7.000000, -21.400000) {かわ};
\node[Meaning] at (7.050000, -19.750000) {thirst};
\node[Square] at (9.100000, -21.500000) {};
\node[Kanji] at (9.100000, -21.000000) {序};
\node[Onyomi] at (9.150000, -21.400000) {ジョ};
\node[Kunyomi] at (9.050000, -21.400000) {つい};
\node[Meaning] at (9.100000, -19.750000) {preface};
\node[Square] at (11.150000, -21.500000) {};
\node[Kanji] at (11.150000, -21.000000) {恩};
\node[Onyomi] at (11.200000, -21.400000) {オン};
\node[Kunyomi] at (11.100000, -21.400000) {おん};
\node[Meaning] at (11.150000, -19.750000) {kindness};
\node[Square] at (13.200000, -21.500000) {};
\node[Kanji] at (13.200000, -21.000000) {艦};
\node[Onyomi] at (13.250000, -21.400000) {カン};
\node[Meaning] at (13.200000, -19.750000) {warship};
\node[Square] at (15.250000, -21.500000) {};
\node[Kanji] at (15.250000, -21.000000) {欺};
\node[Onyomi] at (15.300000, -21.400000) {ギ};
\node[Kunyomi] at (15.200000, -21.400000) {あざむ.く};
\node[Meaning] at (15.250000, -19.750000) {deceit};
\node[Square] at (17.300000, -21.500000) {};
\node[Kanji] at (17.300000, -21.000000) {掌};
\node[Onyomi] at (17.350000, -21.400000) {ショウ};
\node[Kunyomi] at (17.250000, -21.400000) {てのひら};
\node[Meaning] at (17.300000, -19.750000) {manipulate};
\node[Square] at (19.350000, -21.500000) {};
\node[Kanji] at (19.350000, -21.000000) {衰};
\node[Onyomi] at (19.400000, -21.400000) {スイ};
\node[Kunyomi] at (19.300000, -21.400000) {おとろ.える};
\node[Meaning] at (19.350000, -19.750000) {decline};
\node[Square] at (21.400000, -21.500000) {};
\node[Kanji] at (21.400000, -21.000000) {疫};
\node[Onyomi] at (21.450000, -21.400000) {エキ};
\node[Meaning] at (21.400000, -19.750000) {epidemic};
\node[Square] at (23.450000, -21.500000) {};
\node[Kanji] at (23.450000, -21.000000) {偏};
\node[Onyomi] at (23.500000, -21.400000) {ヘン};
\node[Kunyomi] at (23.400000, -21.400000) {かたよ};
\node[Meaning] at (23.450000, -19.750000) {biased};
\node[Square] at (25.500000, -21.500000) {};
\node[Kanji] at (25.500000, -21.000000) {冠};
\node[Onyomi] at (25.550000, -21.400000) {カン};
\node[Kunyomi] at (25.450000, -21.400000) {かんむり};
\node[Meaning] at (25.500000, -19.750000) {crown};
\node[Square] at (27.550000, -21.500000) {};
\node[Kanji] at (27.550000, -21.000000) {飽};
\node[Onyomi] at (27.600000, -21.400000) {ホウ};
\node[Kunyomi] at (27.500000, -21.400000) {あ};
\node[Meaning] at (27.550000, -19.750000) {bored};
\node[Square] at (29.600000, -21.500000) {};
\node[Kanji] at (29.600000, -21.000000) {紺};
\node[Onyomi] at (29.650000, -21.400000) {コン};
\node[Meaning] at (29.600000, -19.750000) {navy};
\node[Square] at (31.650000, -21.500000) {};
\node[Kanji] at (31.650000, -21.000000) {煩};
\node[Onyomi] at (31.700000, -21.400000) {ハン};
\node[Kunyomi] at (31.600000, -21.400000) {うるさ};
\node[Meaning] at (31.650000, -19.750000) {annoy};
\node[Square] at (33.700000, -21.500000) {};
\node[Kanji] at (33.700000, -21.000000) {采};
\node[Onyomi] at (33.750000, -21.400000) {サイ};
\node[Meaning] at (33.700000, -19.750000) {form};
\node[Square] at (35.750000, -21.500000) {};
\node[Kanji] at (35.750000, -21.000000) {踪};
\node[Onyomi] at (35.800000, -21.400000) {ソウ};
\node[Meaning] at (35.750000, -19.750000) {remains};
\node[Square] at (37.800000, -21.500000) {};
\node[Kanji] at (37.800000, -21.000000) {貪};
\node[Onyomi] at (37.850000, -21.400000) {ドン};
\node[Kunyomi] at (37.750000, -21.400000) {むさぼ-る};
\node[Meaning] at (37.800000, -19.750000) {covet};
\node[Square] at (39.850000, -21.500000) {};
\node[Kanji] at (39.850000, -21.000000) {是};
\node[Onyomi] at (39.900000, -21.400000) {ゼ};
\node[Meaning] at (39.850000, -19.750000) {absolutely};
\node[Square] at (41.900000, -21.500000) {};
\node[Kanji] at (41.900000, -21.000000) {芽};
\node[Onyomi] at (41.950000, -21.400000) {ガ};
\node[Kunyomi] at (41.850000, -21.400000) {め};
\node[Meaning] at (41.900000, -19.750000) {sprout};
\node[Square] at (43.950000, -21.500000) {};
\node[Kanji] at (43.950000, -21.000000) {如};
\node[Onyomi] at (44.000000, -21.400000) {ジョ};
\node[Kunyomi] at (43.900000, -21.400000) {ごと.し};
\node[Meaning] at (43.950000, -19.750000) {likeness};
\node[Square] at (46.000000, -21.500000) {};
\node[Kanji] at (46.000000, -21.000000) {肪};
\node[Onyomi] at (46.050000, -21.400000) {ボウ};
\node[Meaning] at (46.000000, -19.750000) {obese};
\node[Square] at (48.050000, -21.500000) {};
\node[Kanji] at (48.050000, -21.000000) {鯨};
\node[Onyomi] at (48.100000, -21.400000) {ゲイ};
\node[Kunyomi] at (48.000000, -21.400000) {くじら};
\node[Meaning] at (48.050000, -19.750000) {whale};
\node[Square] at (50.100000, -21.500000) {};
\node[Kanji] at (50.100000, -21.000000) {龍};
\node[Onyomi] at (50.150000, -21.400000) {リュウ};
\node[Kunyomi] at (50.050000, -21.400000) {たつ};
\node[Meaning] at (50.100000, -19.750000) {imperial};
\node[Square] at (52.150000, -21.500000) {};
\node[Kanji] at (52.150000, -21.000000) {狐};
\node[Onyomi] at (52.200000, -21.400000) {コ};
\node[Kunyomi] at (52.100000, -21.400000) {きつね};
\node[Meaning] at (52.150000, -19.750000) {fox};
\node[Square] at (54.200000, -21.500000) {};
\node[Kanji] at (54.200000, -21.000000) {斑};
\node[Onyomi] at (54.250000, -21.400000) {ハン};
\node[Meaning] at (54.200000, -19.750000) {blemish};
\node[Square] at (56.250000, -21.500000) {};
\node[Kanji] at (56.250000, -21.000000) {渉};
\node[Onyomi] at (56.300000, -21.400000) {ショウ};
\node[Kunyomi] at (56.200000, -21.400000) {わた.る};
\node[Meaning] at (56.250000, -19.750000) {ford};
\node[Meaning] at (-58.500000, -20.950000) {98.04\%};
\node[Square] at (-56.500000, -23.550000) {};
\node[Kanji] at (-56.500000, -23.050000) {薦};
\node[Onyomi] at (-56.450000, -23.450000) {セン};
\node[Kunyomi] at (-56.550000, -23.450000) {すす.*};
\node[Meaning] at (-56.500000, -21.800000) {recommend};
\node[Square] at (-54.450000, -23.550000) {};
\node[Kanji] at (-54.450000, -23.050000) {沼};
\node[Onyomi] at (-54.400000, -23.450000) {ショウ};
\node[Kunyomi] at (-54.500000, -23.450000) {ぬま};
\node[Meaning] at (-54.450000, -21.800000) {bog};
\node[Square] at (-52.400000, -23.550000) {};
\node[Kanji] at (-52.400000, -23.050000) {鍛};
\node[Onyomi] at (-52.350000, -23.450000) {タン};
\node[Kunyomi] at (-52.450000, -23.450000) {きた.える};
\node[Meaning] at (-52.400000, -21.800000) {forge};
\node[Square] at (-50.350000, -23.550000) {};
\node[Kanji] at (-50.350000, -23.050000) {阿};
\node[Onyomi] at (-50.300000, -23.450000) {ア};
\node[Kunyomi] at (-50.400000, -23.450000) {おもね};
\node[Meaning] at (-50.350000, -21.800000) {flatter};
\node[Square] at (-48.300000, -23.550000) {};
\node[Kanji] at (-48.300000, -23.050000) {粛};
\node[Onyomi] at (-48.250000, -23.450000) {シュク};
\node[Kunyomi] at (-48.350000, -23.450000) {つつし};
\node[Meaning] at (-48.300000, -21.800000) {solemn};
\node[Square] at (-46.250000, -23.550000) {};
\node[Kanji] at (-46.250000, -23.050000) {凡};
\node[Onyomi] at (-46.200000, -23.450000) {ボン};
\node[Kunyomi] at (-46.300000, -23.450000) {おうよ.そ};
\node[Meaning] at (-46.250000, -21.800000) {mediocre};
\node[Square] at (-44.200000, -23.550000) {};
\node[Kanji] at (-44.200000, -23.050000) {凸};
\node[Onyomi] at (-44.150000, -23.450000) {トツ};
\node[Kunyomi] at (-44.250000, -23.450000) {でこ};
\node[Meaning] at (-44.200000, -21.800000) {convex};
\node[Square] at (-42.150000, -23.550000) {};
\node[Kanji] at (-42.150000, -23.050000) {貌};
\node[Onyomi] at (-42.100000, -23.450000) {ボウ};
\node[Meaning] at (-42.150000, -21.800000) {appearance};
\node[Square] at (-40.100000, -23.550000) {};
\node[Kanji] at (-40.100000, -23.050000) {妥};
\node[Onyomi] at (-40.050000, -23.450000) {ダ};
\node[Meaning] at (-40.100000, -21.800000) {gentle};
\node[Square] at (-38.050000, -23.550000) {};
\node[Kanji] at (-38.050000, -23.050000) {欧};
\node[Onyomi] at (-38.000000, -23.450000) {オウ};
\node[Meaning] at (-38.050000, -21.800000) {europe};
\node[Square] at (-36.000000, -23.550000) {};
\node[Kanji] at (-36.000000, -23.050000) {託};
\node[Onyomi] at (-35.950000, -23.450000) {タク};
\node[Kunyomi] at (-36.050000, -23.450000) {かこ.*};
\node[Meaning] at (-36.000000, -21.800000) {consign};
\node[Square] at (-33.950000, -23.550000) {};
\node[Kanji] at (-33.950000, -23.050000) {俊};
\node[Onyomi] at (-33.900000, -23.450000) {シュン};
\node[Meaning] at (-33.950000, -21.800000) {genius};
\node[Square] at (-31.900000, -23.550000) {};
\node[Kanji] at (-31.900000, -23.050000) {籍};
\node[Onyomi] at (-31.850000, -23.450000) {セキ};
\node[Meaning] at (-31.900000, -21.800000) {enroll};
\node[Square] at (-29.850000, -23.550000) {};
\node[Kanji] at (-29.850000, -23.050000) {琴};
\node[Kunyomi] at (-29.900000, -23.450000) {こと};
\node[Meaning] at (-29.850000, -21.800000) {harp};
\node[Square] at (-27.800000, -23.550000) {};
\node[Kanji] at (-27.800000, -23.050000) {零};
\node[Onyomi] at (-27.750000, -23.450000) {レイ};
\node[Kunyomi] at (-27.850000, -23.450000) {こぼ.す};
\node[Meaning] at (-27.800000, -21.800000) {zero};
\node[Square] at (-25.750000, -23.550000) {};
\node[Kanji] at (-25.750000, -23.050000) {詐};
\node[Onyomi] at (-25.700000, -23.450000) {サ};
\node[Kunyomi] at (-25.800000, -23.450000) {いつわ.る};
\node[Meaning] at (-25.750000, -21.800000) {lie};
\node[Square] at (-23.700000, -23.550000) {};
\node[Kanji] at (-23.700000, -23.050000) {蓄};
\node[Onyomi] at (-23.650000, -23.450000) {チク};
\node[Kunyomi] at (-23.750000, -23.450000) {たくわ.える};
\node[Meaning] at (-23.700000, -21.800000) {amass};
\node[Square] at (-21.650000, -23.550000) {};
\node[Kanji] at (-21.650000, -23.050000) {枯};
\node[Onyomi] at (-21.600000, -23.450000) {コ};
\node[Kunyomi] at (-21.700000, -23.450000) {か};
\node[Meaning] at (-21.650000, -21.800000) {wither};
\node[Square] at (-19.600000, -23.550000) {};
\node[Kanji] at (-19.600000, -23.050000) {秩};
\node[Onyomi] at (-19.550000, -23.450000) {チツ};
\node[Meaning] at (-19.600000, -21.800000) {order};
\node[Square] at (-17.550000, -23.550000) {};
\node[Kanji] at (-17.550000, -23.050000) {啓};
\node[Onyomi] at (-17.500000, -23.450000) {ケイ};
\node[Kunyomi] at (-17.600000, -23.450000) {さと};
\node[Meaning] at (-17.550000, -21.800000) {enlighten};
\node[Square] at (-15.500000, -23.550000) {};
\node[Kanji] at (-15.500000, -23.050000) {抹};
\node[Onyomi] at (-15.450000, -23.450000) {マツ};
\node[Meaning] at (-15.500000, -21.800000) {erase};
\node[Square] at (-13.450000, -23.550000) {};
\node[Kanji] at (-13.450000, -23.050000) {恭};
\node[Onyomi] at (-13.400000, -23.450000) {キョウ};
\node[Kunyomi] at (-13.500000, -23.450000) {うやうや};
\node[Meaning] at (-13.450000, -21.800000) {respect};
\node[Square] at (-11.400000, -23.550000) {};
\node[Kanji] at (-11.400000, -23.050000) {妄};
\node[Onyomi] at (-11.350000, -23.450000) {モウ};
\node[Kunyomi] at (-11.450000, -23.450000) {みだ};
\node[Meaning] at (-11.400000, -21.800000) {reckless};
\node[Square] at (-9.350000, -23.550000) {};
\node[Kanji] at (-9.350000, -23.050000) {箇};
\node[Onyomi] at (-9.300000, -23.450000) {カ};
\node[Meaning] at (-9.350000, -21.800000) {counters};
\node[Square] at (-7.300000, -23.550000) {};
\node[Kanji] at (-7.300000, -23.050000) {潰};
\node[Onyomi] at (-7.250000, -23.450000) {カイ};
\node[Kunyomi] at (-7.350000, -23.450000) {つぶ-す};
\node[Meaning] at (-7.300000, -21.800000) {crush};
\node[Square] at (-5.250000, -23.550000) {};
\node[Kanji] at (-5.250000, -23.050000) {嚇};
\node[Onyomi] at (-5.200000, -23.450000) {カク};
\node[Meaning] at (-5.250000, -21.800000) {menacing};
\node[Square] at (-3.200000, -23.550000) {};
\node[Kanji] at (-3.200000, -23.050000) {腺};
\node[Onyomi] at (-3.150000, -23.450000) {セン};
\node[Meaning] at (-3.200000, -21.800000) {gland};
\node[Square] at (-1.150000, -23.550000) {};
\node[Kanji] at (-1.150000, -23.050000) {伴};
\node[Onyomi] at (-1.100000, -23.450000) {ハン};
\node[Kunyomi] at (-1.200000, -23.450000) {ともな.う};
\node[Meaning] at (-1.150000, -21.800000) {accompany};
\node[Square] at (0.900000, -23.550000) {};
\node[Kanji] at (0.900000, -23.050000) {喪};
\node[Onyomi] at (0.950000, -23.450000) {ソウ};
\node[Kunyomi] at (0.850000, -23.450000) {も};
\node[Meaning] at (0.900000, -21.800000) {mourning};
\node[Square] at (2.950000, -23.550000) {};
\node[Kanji] at (2.950000, -23.050000) {爽};
\node[Onyomi] at (3.000000, -23.450000) {ソウ};
\node[Kunyomi] at (2.900000, -23.450000) {さわ};
\node[Meaning] at (2.950000, -21.800000) {refreshing};
\node[Square] at (5.000000, -23.550000) {};
\node[Kanji] at (5.000000, -23.050000) {彩};
\node[Onyomi] at (5.050000, -23.450000) {サイ};
\node[Kunyomi] at (4.950000, -23.450000) {いろど.る};
\node[Meaning] at (5.000000, -21.800000) {coloring};
\node[Square] at (7.050000, -23.550000) {};
\node[Kanji] at (7.050000, -23.050000) {俗};
\node[Onyomi] at (7.100000, -23.450000) {ゾク};
\node[Meaning] at (7.050000, -21.800000) {vulgar};
\node[Square] at (9.100000, -23.550000) {};
\node[Kanji] at (9.100000, -23.050000) {巧};
\node[Onyomi] at (9.150000, -23.450000) {コウ};
\node[Kunyomi] at (9.050000, -23.450000) {うま.い};
\node[Meaning] at (9.100000, -21.800000) {adept};
\node[Square] at (11.150000, -23.550000) {};
\node[Kanji] at (11.150000, -23.050000) {燥};
\node[Onyomi] at (11.200000, -23.450000) {ソウ};
\node[Kunyomi] at (11.100000, -23.450000) {はしゃ.ぐ};
\node[Meaning] at (11.150000, -21.800000) {dry up};
\node[Square] at (13.200000, -23.550000) {};
\node[Kanji] at (13.200000, -23.050000) {悠};
\node[Onyomi] at (13.250000, -23.450000) {ユウ};
\node[Meaning] at (13.200000, -21.800000) {leisure};
\node[Square] at (15.250000, -23.550000) {};
\node[Kanji] at (15.250000, -23.050000) {鎮};
\node[Onyomi] at (15.300000, -23.450000) {チン};
\node[Kunyomi] at (15.200000, -23.450000) {おさえ};
\node[Meaning] at (15.250000, -21.800000) {tranquilize};
\node[Square] at (17.300000, -23.550000) {};
\node[Kanji] at (17.300000, -23.050000) {悦};
\node[Onyomi] at (17.350000, -23.450000) {エツ};
\node[Kunyomi] at (17.250000, -23.450000) {よろこ};
\node[Meaning] at (17.300000, -21.800000) {delight};
\node[Square] at (19.350000, -23.550000) {};
\node[Kanji] at (19.350000, -23.050000) {窒};
\node[Onyomi] at (19.400000, -23.450000) {チツ};
\node[Meaning] at (19.350000, -21.800000) {suffocate};
\node[Square] at (21.400000, -23.550000) {};
\node[Kanji] at (21.400000, -23.050000) {智};
\node[Onyomi] at (21.450000, -23.450000) {チ};
\node[Meaning] at (21.400000, -21.800000) {wisdom};
\node[Square] at (23.450000, -23.550000) {};
\node[Kanji] at (23.450000, -23.050000) {凌};
\node[Onyomi] at (23.500000, -23.450000) {リョウ};
\node[Kunyomi] at (23.400000, -23.450000) {しの};
\node[Meaning] at (23.450000, -21.800000) {endure};
\node[Square] at (25.500000, -23.550000) {};
\node[Kanji] at (25.500000, -23.050000) {憧};
\node[Onyomi] at (25.550000, -23.450000) {ショウ};
\node[Kunyomi] at (25.450000, -23.450000) {あこが};
\node[Meaning] at (25.500000, -21.800000) {long for};
\node[Square] at (27.550000, -23.550000) {};
\node[Kanji] at (27.550000, -23.050000) {汰};
\node[Onyomi] at (27.600000, -23.450000) {タ};
\node[Kunyomi] at (27.500000, -23.450000) {おご};
\node[Meaning] at (27.550000, -21.800000) {select};
\node[Square] at (29.600000, -23.550000) {};
\node[Kanji] at (29.600000, -23.050000) {廉};
\node[Onyomi] at (29.650000, -23.450000) {レン};
\node[Meaning] at (29.600000, -21.800000) {bargain};
\node[Square] at (31.650000, -23.550000) {};
\node[Kanji] at (31.650000, -23.050000) {釜};
\node[Kunyomi] at (31.600000, -23.450000) {かま};
\node[Meaning] at (31.650000, -21.800000) {kettle};
\node[Square] at (33.700000, -23.550000) {};
\node[Kanji] at (33.700000, -23.050000) {畿};
\node[Onyomi] at (33.750000, -23.450000) {キ};
\node[Meaning] at (33.700000, -21.800000) {capital};
\node[Square] at (35.750000, -23.550000) {};
\node[Kanji] at (35.750000, -23.050000) {肢};
\node[Onyomi] at (35.800000, -23.450000) {シ};
\node[Meaning] at (35.750000, -21.800000) {limb};
\node[Square] at (37.800000, -23.550000) {};
\node[Kanji] at (37.800000, -23.050000) {嫉};
\node[Onyomi] at (37.850000, -23.450000) {シツ};
\node[Meaning] at (37.800000, -21.800000) {envy};
\node[Square] at (39.850000, -23.550000) {};
\node[Kanji] at (39.850000, -23.050000) {薪};
\node[Onyomi] at (39.900000, -23.450000) {シン};
\node[Kunyomi] at (39.800000, -23.450000) {たきぎ};
\node[Meaning] at (39.850000, -21.800000) {fuel};
\node[Square] at (41.900000, -23.550000) {};
\node[Kanji] at (41.900000, -23.050000) {痘};
\node[Onyomi] at (41.950000, -23.450000) {トウ};
\node[Meaning] at (41.900000, -21.800000) {pox};
\node[Square] at (43.950000, -23.550000) {};
\node[Kanji] at (43.950000, -23.050000) {辣};
\node[Onyomi] at (44.000000, -23.450000) {ラツ};
\node[Meaning] at (43.950000, -21.800000) {bitter};
\node[Square] at (46.000000, -23.550000) {};
\node[Kanji] at (46.000000, -23.050000) {購};
\node[Onyomi] at (46.050000, -23.450000) {コウ};
\node[Meaning] at (46.000000, -21.800000) {subscription};
\node[Square] at (48.050000, -23.550000) {};
\node[Kanji] at (48.050000, -23.050000) {偵};
\node[Onyomi] at (48.100000, -23.450000) {テイ};
\node[Meaning] at (48.050000, -21.800000) {spy};
\node[Square] at (50.100000, -23.550000) {};
\node[Kanji] at (50.100000, -23.050000) {姓};
\node[Onyomi] at (50.150000, -23.450000) {セイ};
\node[Meaning] at (50.100000, -21.800000) {surname};
\node[Square] at (52.150000, -23.550000) {};
\node[Kanji] at (52.150000, -23.050000) {宜};
\node[Onyomi] at (52.200000, -23.450000) {ギ};
\node[Kunyomi] at (52.100000, -23.450000) {よろ};
\node[Meaning] at (52.150000, -21.800000) {best regards};
\node[Square] at (54.200000, -23.550000) {};
\node[Kanji] at (54.200000, -23.050000) {岳};
\node[Onyomi] at (54.250000, -23.450000) {ガク};
\node[Kunyomi] at (54.150000, -23.450000) {たけ};
\node[Meaning] at (54.200000, -21.800000) {peak};
\node[Square] at (56.250000, -23.550000) {};
\node[Kanji] at (56.250000, -23.050000) {翔};
\node[Onyomi] at (56.300000, -23.450000) {ショウ};
\node[Kunyomi] at (56.200000, -23.450000) {かけ};
\node[Meaning] at (56.250000, -21.800000) {fly};
\node[Meaning] at (-58.500000, -23.000000) {98.12\%};
\node[Square] at (-56.500000, -25.600000) {};
\node[Kanji] at (-56.500000, -25.100000) {嫁};
\node[Onyomi] at (-56.450000, -25.500000) {カ};
\node[Kunyomi] at (-56.550000, -25.500000) {よめ};
\node[Meaning] at (-56.500000, -23.850000) {bride};
\node[Square] at (-54.450000, -25.600000) {};
\node[Kanji] at (-54.450000, -25.100000) {礎};
\node[Onyomi] at (-54.400000, -25.500000) {ソ};
\node[Kunyomi] at (-54.500000, -25.500000) {いしずえ};
\node[Meaning] at (-54.450000, -23.850000) {foundation};
\node[Square] at (-52.400000, -25.600000) {};
\node[Kanji] at (-52.400000, -25.100000) {謀};
\node[Onyomi] at (-52.350000, -25.500000) {ボウ};
\node[Kunyomi] at (-52.450000, -25.500000) {はか.る};
\node[Meaning] at (-52.400000, -23.850000) {conspire};
\node[Square] at (-50.350000, -25.600000) {};
\node[Kanji] at (-50.350000, -25.100000) {墳};
\node[Onyomi] at (-50.300000, -25.500000) {フン};
\node[Meaning] at (-50.350000, -23.850000) {tomb};
\node[Square] at (-48.300000, -25.600000) {};
\node[Kanji] at (-48.300000, -25.100000) {慈};
\node[Onyomi] at (-48.250000, -25.500000) {ジ};
\node[Kunyomi] at (-48.350000, -25.500000) {いつく};
\node[Meaning] at (-48.300000, -23.850000) {mercy};
\node[Square] at (-46.250000, -25.600000) {};
\node[Kanji] at (-46.250000, -25.100000) {寛};
\node[Onyomi] at (-46.200000, -25.500000) {カン};
\node[Kunyomi] at (-46.300000, -25.500000) {くつろ.ぐ};
\node[Meaning] at (-46.250000, -23.850000) {tolerance};
\node[Square] at (-44.200000, -25.600000) {};
\node[Kanji] at (-44.200000, -25.100000) {痴};
\node[Onyomi] at (-44.150000, -25.500000) {チ};
\node[Kunyomi] at (-44.250000, -25.500000) {おろか};
\node[Meaning] at (-44.200000, -23.850000) {stupid};
\node[Square] at (-42.150000, -25.600000) {};
\node[Kanji] at (-42.150000, -25.100000) {遥};
\node[Onyomi] at (-42.100000, -25.500000) {ヨウ};
\node[Kunyomi] at (-42.200000, -25.500000) {はる};
\node[Meaning] at (-42.150000, -23.850000) {far off};
\node[Square] at (-40.100000, -25.600000) {};
\node[Kanji] at (-40.100000, -25.100000) {紳};
\node[Onyomi] at (-40.050000, -25.500000) {シン};
\node[Meaning] at (-40.100000, -23.850000) {gentleman};
\node[Square] at (-38.050000, -25.600000) {};
\node[Kanji] at (-38.050000, -25.100000) {逝};
\node[Onyomi] at (-38.000000, -25.500000) {セイ};
\node[Kunyomi] at (-38.100000, -25.500000) {い.く};
\node[Meaning] at (-38.050000, -23.850000) {die};
\node[Square] at (-36.000000, -25.600000) {};
\node[Kanji] at (-36.000000, -25.100000) {磁};
\node[Onyomi] at (-35.950000, -25.500000) {ジ};
\node[Meaning] at (-36.000000, -23.850000) {magnet};
\node[Square] at (-33.950000, -25.600000) {};
\node[Kanji] at (-33.950000, -25.100000) {暦};
\node[Onyomi] at (-33.900000, -25.500000) {レキ};
\node[Kunyomi] at (-34.000000, -25.500000) {こよみ};
\node[Meaning] at (-33.950000, -23.850000) {calendar};
\node[Square] at (-31.900000, -25.600000) {};
\node[Kanji] at (-31.900000, -25.100000) {碁};
\node[Onyomi] at (-31.850000, -25.500000) {ゴ};
\node[Meaning] at (-31.900000, -23.850000) {go};
\node[Square] at (-29.850000, -25.600000) {};
\node[Kanji] at (-29.850000, -25.100000) {粘};
\node[Onyomi] at (-29.800000, -25.500000) {ネン};
\node[Kunyomi] at (-29.900000, -25.500000) {ねば.る};
\node[Meaning] at (-29.850000, -23.850000) {sticky};
\node[Square] at (-27.800000, -25.600000) {};
\node[Kanji] at (-27.800000, -25.100000) {銘};
\node[Onyomi] at (-27.750000, -25.500000) {メイ};
\node[Meaning] at (-27.800000, -23.850000) {inscription};
\node[Square] at (-25.750000, -25.600000) {};
\node[Kanji] at (-25.750000, -25.100000) {彰};
\node[Onyomi] at (-25.700000, -25.500000) {ショウ};
\node[Meaning] at (-25.750000, -23.850000) {patent};
\node[Square] at (-23.700000, -25.600000) {};
\node[Kanji] at (-23.700000, -25.100000) {軌};
\node[Onyomi] at (-23.650000, -25.500000) {キ};
\node[Meaning] at (-23.700000, -23.850000) {rut};
\node[Square] at (-21.650000, -25.600000) {};
\node[Kanji] at (-21.650000, -25.100000) {殊};
\node[Onyomi] at (-21.600000, -25.500000) {シュ};
\node[Kunyomi] at (-21.700000, -25.500000) {こと};
\node[Meaning] at (-21.650000, -23.850000) {especially};
\node[Square] at (-19.600000, -25.600000) {};
\node[Kanji] at (-19.600000, -25.100000) {惜};
\node[Onyomi] at (-19.550000, -25.500000) {セキ};
\node[Kunyomi] at (-19.650000, -25.500000) {お};
\node[Meaning] at (-19.600000, -23.850000) {frugal};
\node[Square] at (-17.550000, -25.600000) {};
\node[Kanji] at (-17.550000, -25.100000) {峡};
\node[Onyomi] at (-17.500000, -25.500000) {キョウ};
\node[Kunyomi] at (-17.600000, -25.500000) {はざま};
\node[Meaning] at (-17.550000, -23.850000) {ravine};
\node[Square] at (-15.500000, -25.600000) {};
\node[Kanji] at (-15.500000, -25.100000) {匿};
\node[Onyomi] at (-15.450000, -25.500000) {トク};
\node[Kunyomi] at (-15.550000, -25.500000) {かくま.う};
\node[Meaning] at (-15.500000, -23.850000) {hide};
\node[Square] at (-13.450000, -25.600000) {};
\node[Kanji] at (-13.450000, -25.100000) {褐};
\node[Onyomi] at (-13.400000, -25.500000) {カツ};
\node[Meaning] at (-13.450000, -23.850000) {brown};
\node[Square] at (-11.400000, -25.600000) {};
\node[Kanji] at (-11.400000, -25.100000) {典};
\node[Onyomi] at (-11.350000, -25.500000) {テン};
\node[Meaning] at (-11.400000, -23.850000) {rule};
\node[Square] at (-9.350000, -25.600000) {};
\node[Kanji] at (-9.350000, -25.100000) {銭};
\node[Onyomi] at (-9.300000, -25.500000) {セン};
\node[Kunyomi] at (-9.400000, -25.500000) {ぜに};
\node[Meaning] at (-9.350000, -23.850000) {coin};
\node[Square] at (-7.300000, -25.600000) {};
\node[Kanji] at (-7.300000, -25.100000) {鉱};
\node[Onyomi] at (-7.250000, -25.500000) {コウ};
\node[Kunyomi] at (-7.350000, -25.500000) {あらがね};
\node[Meaning] at (-7.300000, -23.850000) {mineral};
\node[Square] at (-5.250000, -25.600000) {};
\node[Kanji] at (-5.250000, -25.100000) {仁};
\node[Onyomi] at (-5.200000, -25.500000) {ジン};
\node[Meaning] at (-5.250000, -23.850000) {humanity};
\node[Square] at (-3.200000, -25.600000) {};
\node[Kanji] at (-3.200000, -25.100000) {虎};
\node[Onyomi] at (-3.150000, -25.500000) {コ};
\node[Kunyomi] at (-3.250000, -25.500000) {とら};
\node[Meaning] at (-3.200000, -23.850000) {tiger};
\node[Square] at (-1.150000, -25.600000) {};
\node[Kanji] at (-1.150000, -25.100000) {哲};
\node[Onyomi] at (-1.100000, -25.500000) {テツ};
\node[Meaning] at (-1.150000, -23.850000) {philosophy};
\node[Square] at (0.900000, -25.600000) {};
\node[Kanji] at (0.900000, -25.100000) {暫};
\node[Onyomi] at (0.950000, -25.500000) {ザン};
\node[Kunyomi] at (0.850000, -25.500000) {しばら.く};
\node[Meaning] at (0.900000, -23.850000) {temporarily};
\node[Square] at (2.950000, -25.600000) {};
\node[Kanji] at (2.950000, -25.100000) {貢};
\node[Onyomi] at (3.000000, -25.500000) {コウ};
\node[Kunyomi] at (2.900000, -25.500000) {みつ.ぐ};
\node[Meaning] at (2.950000, -23.850000) {tribute};
\node[Square] at (5.000000, -25.600000) {};
\node[Kanji] at (5.000000, -25.100000) {滋};
\node[Onyomi] at (5.050000, -25.500000) {ジ};
\node[Meaning] at (5.000000, -23.850000) {nourishing};
\node[Square] at (7.050000, -25.600000) {};
\node[Kanji] at (7.050000, -25.100000) {塀};
\node[Onyomi] at (7.100000, -25.500000) {ヘイ};
\node[Meaning] at (7.050000, -23.850000) {fence};
\node[Square] at (9.100000, -25.600000) {};
\node[Kanji] at (9.100000, -25.100000) {霜};
\node[Kunyomi] at (9.050000, -25.500000) {しも};
\node[Meaning] at (9.100000, -23.850000) {frost};
\node[Square] at (11.150000, -25.600000) {};
\node[Kanji] at (11.150000, -25.100000) {拘};
\node[Onyomi] at (11.200000, -25.500000) {コウ};
\node[Kunyomi] at (11.100000, -25.500000) {かか.わる};
\node[Meaning] at (11.150000, -23.850000) {arrest};
\node[Square] at (13.200000, -25.600000) {};
\node[Kanji] at (13.200000, -25.100000) {亭};
\node[Onyomi] at (13.250000, -25.500000) {テイ};
\node[Meaning] at (13.200000, -23.850000) {restaurant};
\node[Square] at (15.250000, -25.600000) {};
\node[Kanji] at (15.250000, -25.100000) {聡};
\node[Onyomi] at (15.300000, -25.500000) {ソウ};
\node[Kunyomi] at (15.200000, -25.500000) {さと.い};
\node[Meaning] at (15.250000, -23.850000) {wise};
\node[Square] at (17.300000, -25.600000) {};
\node[Kanji] at (17.300000, -25.100000) {鯉};
\node[Onyomi] at (17.350000, -25.500000) {リ};
\node[Kunyomi] at (17.250000, -25.500000) {こい};
\node[Meaning] at (17.300000, -23.850000) {carp};
\node[Square] at (19.350000, -25.600000) {};
\node[Kanji] at (19.350000, -25.100000) {茜};
\node[Onyomi] at (19.400000, -25.500000) {セン};
\node[Kunyomi] at (19.300000, -25.500000) {あかね};
\node[Meaning] at (19.350000, -23.850000) {red dye};
\node[Square] at (21.400000, -25.600000) {};
\node[Kanji] at (21.400000, -25.100000) {疎};
\node[Onyomi] at (21.450000, -25.500000) {ソ};
\node[Kunyomi] at (21.350000, -25.500000) {うと};
\node[Meaning] at (21.400000, -23.850000) {neglect};
\node[Square] at (23.450000, -25.600000) {};
\node[Kanji] at (23.450000, -25.100000) {盲};
\node[Onyomi] at (23.500000, -25.500000) {モウ};
\node[Kunyomi] at (23.400000, -25.500000) {めくら};
\node[Meaning] at (23.450000, -23.850000) {blind};
\node[Square] at (25.500000, -25.600000) {};
\node[Kanji] at (25.500000, -25.100000) {痢};
\node[Onyomi] at (25.550000, -25.500000) {リ};
\node[Meaning] at (25.500000, -23.850000) {diarrhea};
\node[Square] at (27.550000, -25.600000) {};
\node[Kanji] at (27.550000, -25.100000) {蛮};
\node[Onyomi] at (27.600000, -25.500000) {バン};
\node[Kunyomi] at (27.500000, -25.500000) {えびす};
\node[Meaning] at (27.550000, -23.850000) {barbarian};
\node[Square] at (29.600000, -25.600000) {};
\node[Kanji] at (29.600000, -25.100000) {頓};
\node[Onyomi] at (29.650000, -25.500000) {トン};
\node[Meaning] at (29.600000, -23.850000) {suddenly};
\node[Square] at (31.650000, -25.600000) {};
\node[Kanji] at (31.650000, -25.100000) {鑑};
\node[Onyomi] at (31.700000, -25.500000) {カン};
\node[Meaning] at (31.650000, -23.850000) {model};
\node[Square] at (33.700000, -25.600000) {};
\node[Kanji] at (33.700000, -25.100000) {芋};
\node[Kunyomi] at (33.650000, -25.500000) {いも};
\node[Meaning] at (33.700000, -23.850000) {potato};
\node[Square] at (35.750000, -25.600000) {};
\node[Kanji] at (35.750000, -25.100000) {謙};
\node[Onyomi] at (35.800000, -25.500000) {ケン};
\node[Meaning] at (35.750000, -23.850000) {modesty};
\node[Square] at (37.800000, -25.600000) {};
\node[Kanji] at (37.800000, -25.100000) {弧};
\node[Onyomi] at (37.850000, -25.500000) {コ};
\node[Meaning] at (37.800000, -23.850000) {arc};
\node[Square] at (39.850000, -25.600000) {};
\node[Kanji] at (39.850000, -25.100000) {懇};
\node[Onyomi] at (39.900000, -25.500000) {コン};
\node[Meaning] at (39.850000, -23.850000) {courteous};
\node[Square] at (41.900000, -25.600000) {};
\node[Kanji] at (41.900000, -25.100000) {斗};
\node[Onyomi] at (41.950000, -25.500000) {ト};
\node[Meaning] at (41.900000, -23.850000) {ladle};
\node[Square] at (43.950000, -25.600000) {};
\node[Kanji] at (43.950000, -25.100000) {刃};
\node[Onyomi] at (44.000000, -25.500000) {ジン};
\node[Kunyomi] at (43.900000, -25.500000) {は};
\node[Meaning] at (43.950000, -23.850000) {blade};
\node[Square] at (46.000000, -25.600000) {};
\node[Kanji] at (46.000000, -25.100000) {剛};
\node[Onyomi] at (46.050000, -25.500000) {ゴウ};
\node[Meaning] at (46.000000, -23.850000) {sturdy};
\node[Square] at (48.050000, -25.600000) {};
\node[Kanji] at (48.050000, -25.100000) {唆};
\node[Onyomi] at (48.100000, -25.500000) {サ};
\node[Kunyomi] at (48.000000, -25.500000) {そそのか};
\node[Meaning] at (48.050000, -23.850000) {instigate};
\node[Square] at (50.100000, -25.600000) {};
\node[Kanji] at (50.100000, -25.100000) {舗};
\node[Onyomi] at (50.150000, -25.500000) {ホ};
\node[Meaning] at (50.100000, -23.850000) {shop};
\node[Square] at (52.150000, -25.600000) {};
\node[Kanji] at (52.150000, -25.100000) {拐};
\node[Onyomi] at (52.200000, -25.500000) {カイ};
\node[Meaning] at (52.150000, -23.850000) {kidnap};
\node[Square] at (54.200000, -25.600000) {};
\node[Kanji] at (54.200000, -25.100000) {弥};
\node[Onyomi] at (54.250000, -25.500000) {ビ};
\node[Kunyomi] at (54.150000, -25.500000) {や};
\node[Meaning] at (54.200000, -23.850000) {increasing};
\node[Square] at (56.250000, -25.600000) {};
\node[Kanji] at (56.250000, -25.100000) {酵};
\node[Onyomi] at (56.300000, -25.500000) {コウ};
\node[Meaning] at (56.250000, -23.850000) {ferment};
\node[Meaning] at (-58.500000, -25.050000) {98.19\%};
\node[Square] at (-56.500000, -27.650000) {};
\node[Kanji] at (-56.500000, -27.150000) {串};
\node[Kunyomi] at (-56.550000, -27.550000) {くし};
\node[Meaning] at (-56.500000, -25.900000) {skewer};
\node[Square] at (-54.450000, -27.650000) {};
\node[Kanji] at (-54.450000, -27.150000) {曽};
\node[Onyomi] at (-54.400000, -27.550000) {ソウ};
\node[Meaning] at (-54.450000, -25.900000) {formerly};
\node[Square] at (-52.400000, -27.650000) {};
\node[Kanji] at (-52.400000, -27.150000) {慄};
\node[Onyomi] at (-52.350000, -27.550000) {リツ};
\node[Meaning] at (-52.400000, -25.900000) {fear};
\node[Square] at (-50.350000, -27.650000) {};
\node[Kanji] at (-50.350000, -27.150000) {綺};
\node[Onyomi] at (-50.300000, -27.550000) {キ};
\node[Meaning] at (-50.350000, -25.900000) {beautiful};
\node[Square] at (-48.300000, -27.650000) {};
\node[Kanji] at (-48.300000, -27.150000) {融};
\node[Onyomi] at (-48.250000, -27.550000) {ユウ};
\node[Meaning] at (-48.300000, -25.900000) {dissolve};
\node[Square] at (-46.250000, -27.650000) {};
\node[Kanji] at (-46.250000, -27.150000) {鋼};
\node[Onyomi] at (-46.200000, -27.550000) {コウ};
\node[Kunyomi] at (-46.300000, -27.550000) {はがね};
\node[Meaning] at (-46.250000, -25.900000) {steel};
\node[Square] at (-44.200000, -27.650000) {};
\node[Kanji] at (-44.200000, -27.150000) {堀};
\node[Onyomi] at (-44.150000, -27.550000) {クツ};
\node[Kunyomi] at (-44.250000, -27.550000) {ほり};
\node[Meaning] at (-44.200000, -25.900000) {ditch};
\node[Square] at (-42.150000, -27.650000) {};
\node[Kanji] at (-42.150000, -27.150000) {泰};
\node[Onyomi] at (-42.100000, -27.550000) {タイ};
\node[Meaning] at (-42.150000, -25.900000) {peace};
\node[Square] at (-40.100000, -27.650000) {};
\node[Kanji] at (-40.100000, -27.150000) {侍};
\node[Kunyomi] at (-40.150000, -27.550000) {さむらい};
\node[Meaning] at (-40.100000, -25.900000) {samurai};
\node[Square] at (-38.050000, -27.650000) {};
\node[Kanji] at (-38.050000, -27.150000) {錬};
\node[Onyomi] at (-38.000000, -27.550000) {レン};
\node[Kunyomi] at (-38.100000, -27.550000) {ね};
\node[Meaning] at (-38.050000, -25.900000) {tempering};
\node[Square] at (-36.000000, -27.650000) {};
\node[Kanji] at (-36.000000, -27.150000) {虹};
\node[Onyomi] at (-35.950000, -27.550000) {コウ};
\node[Kunyomi] at (-36.050000, -27.550000) {にじ};
\node[Meaning] at (-36.000000, -25.900000) {rainbow};
\node[Square] at (-33.950000, -27.650000) {};
\node[Kanji] at (-33.950000, -27.150000) {穫};
\node[Onyomi] at (-33.900000, -27.550000) {カク};
\node[Meaning] at (-33.950000, -25.900000) {harvest};
\node[Square] at (-31.900000, -27.650000) {};
\node[Kanji] at (-31.900000, -27.150000) {畜};
\node[Onyomi] at (-31.850000, -27.550000) {チク};
\node[Meaning] at (-31.900000, -25.900000) {livestock};
\node[Square] at (-29.850000, -27.650000) {};
\node[Kanji] at (-29.850000, -27.150000) {概};
\node[Onyomi] at (-29.800000, -27.550000) {ガイ};
\node[Kunyomi] at (-29.900000, -27.550000) {おおむ.ね};
\node[Meaning] at (-29.850000, -25.900000) {approximation};
\node[Square] at (-27.800000, -27.650000) {};
\node[Kanji] at (-27.800000, -27.150000) {媛};
\node[Onyomi] at (-27.750000, -27.550000) {エン};
\node[Kunyomi] at (-27.850000, -27.550000) {ひめ};
\node[Meaning] at (-27.800000, -25.900000) {princess};
\node[Square] at (-25.750000, -27.650000) {};
\node[Kanji] at (-25.750000, -27.150000) {苗};
\node[Onyomi] at (-25.700000, -27.550000) {ミョウ};
\node[Kunyomi] at (-25.800000, -27.550000) {なえ};
\node[Meaning] at (-25.750000, -25.900000) {seedling};
\node[Square] at (-23.700000, -27.650000) {};
\node[Kanji] at (-23.700000, -27.150000) {怠};
\node[Onyomi] at (-23.650000, -27.550000) {タイ};
\node[Kunyomi] at (-23.750000, -27.550000) {おこた};
\node[Meaning] at (-23.700000, -25.900000) {lazy};
\node[Square] at (-21.650000, -27.650000) {};
\node[Kanji] at (-21.650000, -27.150000) {擁};
\node[Onyomi] at (-21.600000, -27.550000) {ヨウ};
\node[Meaning] at (-21.650000, -25.900000) {embrace};
\node[Square] at (-19.600000, -27.650000) {};
\node[Kanji] at (-19.600000, -27.150000) {緯};
\node[Onyomi] at (-19.550000, -27.550000) {イ};
\node[Kunyomi] at (-19.650000, -27.550000) {ぬき};
\node[Meaning] at (-19.600000, -25.900000) {latitude};
\node[Square] at (-17.550000, -27.650000) {};
\node[Kanji] at (-17.550000, -27.150000) {瓜};
\node[Onyomi] at (-17.500000, -27.550000) {カ};
\node[Kunyomi] at (-17.600000, -27.550000) {うり};
\node[Meaning] at (-17.550000, -25.900000) {melon};
\node[Square] at (-15.500000, -27.650000) {};
\node[Kanji] at (-15.500000, -27.150000) {欄};
\node[Onyomi] at (-15.450000, -27.550000) {ラン};
\node[Kunyomi] at (-15.550000, -27.550000) {てすり        };
\node[Meaning] at (-15.500000, -25.900000) {column};
\node[Square] at (-13.450000, -27.650000) {};
\node[Kanji] at (-13.450000, -27.150000) {奉};
\node[Onyomi] at (-13.400000, -27.550000) {ホウ};
\node[Kunyomi] at (-13.500000, -27.550000) {たてまつ};
\node[Meaning] at (-13.450000, -25.900000) {dedicate};
\node[Square] at (-11.400000, -27.650000) {};
\node[Kanji] at (-11.400000, -27.150000) {靖};
\node[Onyomi] at (-11.350000, -27.550000) {ジョウ};
\node[Kunyomi] at (-11.450000, -27.550000) {やす};
\node[Meaning] at (-11.400000, -25.900000) {peaceful};
\node[Square] at (-9.350000, -27.650000) {};
\node[Kanji] at (-9.350000, -27.150000) {搾};
\node[Onyomi] at (-9.300000, -27.550000) {サク};
\node[Kunyomi] at (-9.400000, -27.550000) {しぼ};
\node[Meaning] at (-9.350000, -25.900000) {squeeze};
\node[Square] at (-7.300000, -27.650000) {};
\node[Kanji] at (-7.300000, -27.150000) {弦};
\node[Onyomi] at (-7.250000, -27.550000) {ゲン};
\node[Kunyomi] at (-7.350000, -27.550000) {つる};
\node[Meaning] at (-7.300000, -25.900000) {chord};
\node[Square] at (-5.250000, -27.650000) {};
\node[Kanji] at (-5.250000, -27.150000) {閲};
\node[Onyomi] at (-5.200000, -27.550000) {エツ};
\node[Kunyomi] at (-5.300000, -27.550000) {けみ};
\node[Meaning] at (-5.250000, -25.900000) {inspection};
\node[Square] at (-3.200000, -27.650000) {};
\node[Kanji] at (-3.200000, -27.150000) {菅};
\node[Onyomi] at (-3.150000, -27.550000) {カン};
\node[Kunyomi] at (-3.250000, -27.550000) {すげ};
\node[Meaning] at (-3.200000, -25.900000) {sedge};
\node[Square] at (-1.150000, -27.650000) {};
\node[Kanji] at (-1.150000, -27.150000) {淑};
\node[Onyomi] at (-1.100000, -27.550000) {シュク};
\node[Kunyomi] at (-1.200000, -27.550000) {しと};
\node[Meaning] at (-1.150000, -25.900000) {graceful};
\node[Square] at (0.900000, -27.650000) {};
\node[Kanji] at (0.900000, -27.150000) {慶};
\node[Onyomi] at (0.950000, -27.550000) {ケイ};
\node[Kunyomi] at (0.850000, -27.550000) {よろこ};
\node[Meaning] at (0.900000, -25.900000) {congratulate};
\node[Square] at (2.950000, -27.650000) {};
\node[Kanji] at (2.950000, -27.150000) {桟};
\node[Onyomi] at (3.000000, -27.550000) {サン};
\node[Kunyomi] at (2.900000, -27.550000) {かけはし};
\node[Meaning] at (2.950000, -25.900000) {jetty};
\node[Square] at (5.000000, -27.650000) {};
\node[Kanji] at (5.000000, -27.150000) {婿};
\node[Onyomi] at (5.050000, -27.550000) {セイ};
\node[Kunyomi] at (4.950000, -27.550000) {むこ};
\node[Meaning] at (5.000000, -25.900000) {groom};
\node[Square] at (7.050000, -27.650000) {};
\node[Kanji] at (7.050000, -27.150000) {惧};
\node[Onyomi] at (7.100000, -27.550000) {グ};
\node[Meaning] at (7.050000, -25.900000) {dread};
\node[Square] at (9.100000, -27.650000) {};
\node[Kanji] at (9.100000, -27.150000) {醒};
\node[Onyomi] at (9.150000, -27.550000) {セイ};
\node[Meaning] at (9.100000, -25.900000) {disillusioned};
\node[Square] at (11.150000, -27.650000) {};
\node[Kanji] at (11.150000, -27.150000) {倣};
\node[Onyomi] at (11.200000, -27.550000) {ホウ};
\node[Kunyomi] at (11.100000, -27.550000) {なら-う};
\node[Meaning] at (11.150000, -25.900000) {emulate};
\node[Square] at (13.200000, -27.650000) {};
\node[Kanji] at (13.200000, -27.150000) {債};
\node[Onyomi] at (13.250000, -27.550000) {サイ};
\node[Meaning] at (13.200000, -25.900000) {debt};
\node[Square] at (15.250000, -27.650000) {};
\node[Kanji] at (15.250000, -27.150000) {維};
\node[Onyomi] at (15.300000, -27.550000) {イ};
\node[Meaning] at (15.250000, -25.900000) {maintain};
\node[Square] at (17.300000, -27.650000) {};
\node[Kanji] at (17.300000, -27.150000) {撤};
\node[Onyomi] at (17.350000, -27.550000) {テツ};
\node[Meaning] at (17.300000, -25.900000) {withdrawal};
\node[Square] at (19.350000, -27.650000) {};
\node[Kanji] at (19.350000, -27.150000) {摩};
\node[Onyomi] at (19.400000, -27.550000) {マ};
\node[Kunyomi] at (19.300000, -27.550000) {さす.る};
\node[Meaning] at (19.350000, -25.900000) {chafe};
\node[Square] at (21.400000, -27.650000) {};
\node[Kanji] at (21.400000, -27.150000) {抽};
\node[Onyomi] at (21.450000, -27.550000) {チュウ};
\node[Meaning] at (21.400000, -25.900000) {pluck};
\node[Square] at (23.450000, -27.650000) {};
\node[Kanji] at (23.450000, -27.150000) {堰};
\node[Onyomi] at (23.500000, -27.550000) {セキ};
\node[Meaning] at (23.450000, -25.900000) {dam};
\node[Square] at (25.500000, -27.650000) {};
\node[Kanji] at (25.500000, -27.150000) {蟹};
\node[Kunyomi] at (25.450000, -27.550000) {かに};
\node[Meaning] at (25.500000, -25.900000) {crab};
\node[Square] at (27.550000, -27.650000) {};
\node[Kanji] at (27.550000, -27.150000) {郡};
\node[Onyomi] at (27.600000, -27.550000) {グン};
\node[Kunyomi] at (27.500000, -27.550000) {こおり        };
\node[Meaning] at (27.550000, -25.900000) {county};
\node[Square] at (29.600000, -27.650000) {};
\node[Kanji] at (29.600000, -27.150000) {芳};
\node[Onyomi] at (29.650000, -27.550000) {ホウ};
\node[Kunyomi] at (29.550000, -27.550000) {かんば};
\node[Meaning] at (29.600000, -25.900000) {perfume};
\node[Square] at (31.650000, -27.650000) {};
\node[Kanji] at (31.650000, -27.150000) {剰};
\node[Onyomi] at (31.700000, -27.550000) {ジョウ};
\node[Kunyomi] at (31.600000, -27.550000) {あまつさえ};
\node[Meaning] at (31.650000, -25.900000) {surplus};
\node[Square] at (33.700000, -27.650000) {};
\node[Kanji] at (33.700000, -27.150000) {准};
\node[Onyomi] at (33.750000, -27.550000) {ジュン};
\node[Meaning] at (33.700000, -25.900000) {semi};
\node[Square] at (35.750000, -27.650000) {};
\node[Kanji] at (35.750000, -27.150000) {駿};
\node[Onyomi] at (35.800000, -27.550000) {シュン};
\node[Kunyomi] at (35.700000, -27.550000) {すぐ};
\node[Meaning] at (35.750000, -25.900000) {speed};
\node[Square] at (37.800000, -27.650000) {};
\node[Kanji] at (37.800000, -27.150000) {憾};
\node[Onyomi] at (37.850000, -27.550000) {カン};
\node[Kunyomi] at (37.750000, -27.550000) {うら};
\node[Meaning] at (37.800000, -25.900000) {remorse};
\node[Square] at (39.850000, -27.650000) {};
\node[Kanji] at (39.850000, -27.150000) {髄};
\node[Onyomi] at (39.900000, -27.550000) {ズイ};
\node[Meaning] at (39.850000, -25.900000) {marrow};
\node[Square] at (41.900000, -27.650000) {};
\node[Kanji] at (41.900000, -27.150000) {昌};
\node[Onyomi] at (41.950000, -27.550000) {ショウ};
\node[Kunyomi] at (41.850000, -27.550000) {さかん};
\node[Meaning] at (41.900000, -25.900000) {prosperous};
\node[Square] at (43.950000, -27.650000) {};
\node[Kanji] at (43.950000, -27.150000) {陵};
\node[Onyomi] at (44.000000, -27.550000) {リョウ};
\node[Kunyomi] at (43.900000, -27.550000) {みささぎ};
\node[Meaning] at (43.950000, -25.900000) {mausoleum};
\node[Square] at (46.000000, -27.650000) {};
\node[Kanji] at (46.000000, -27.150000) {忌};
\node[Onyomi] at (46.050000, -27.550000) {キ};
\node[Kunyomi] at (45.950000, -27.550000) {い};
\node[Meaning] at (46.000000, -25.900000) {mourning};
\node[Square] at (48.050000, -27.650000) {};
\node[Kanji] at (48.050000, -27.150000) {朽};
\node[Onyomi] at (48.100000, -27.550000) {キュウ};
\node[Kunyomi] at (48.000000, -27.550000) {く};
\node[Meaning] at (48.050000, -25.900000) {rot};
\node[Square] at (50.100000, -27.650000) {};
\node[Kanji] at (50.100000, -27.150000) {椎};
\node[Onyomi] at (50.150000, -27.550000) {ツイ};
\node[Kunyomi] at (50.050000, -27.550000) {う};
\node[Meaning] at (50.100000, -25.900000) {oak};
\node[Square] at (52.150000, -27.650000) {};
\node[Kanji] at (52.150000, -27.150000) {傑};
\node[Onyomi] at (52.200000, -27.550000) {ケツ};
\node[Kunyomi] at (52.100000, -27.550000) {すぐ};
\node[Meaning] at (52.150000, -25.900000) {greatness};
\node[Square] at (54.200000, -27.650000) {};
\node[Kanji] at (54.200000, -27.150000) {堆};
\node[Onyomi] at (54.250000, -27.550000) {タイ};
\node[Meaning] at (54.200000, -25.900000) {piled high};
\node[Square] at (56.250000, -27.650000) {};
\node[Kanji] at (56.250000, -27.150000) {贅};
\node[Onyomi] at (56.300000, -27.550000) {ゼイ};
\node[Kunyomi] at (56.200000, -27.550000) {いぼ};
\node[Meaning] at (56.250000, -25.900000) {luxury};
\node[Meaning] at (-58.500000, -27.100000) {98.23\%};
\node[Square] at (-56.500000, -29.700000) {};
\node[Kanji] at (-56.500000, -29.200000) {酢};
\node[Kunyomi] at (-56.550000, -29.600000) {す};
\node[Meaning] at (-56.500000, -27.950000) {vinegar};
\node[Square] at (-54.450000, -29.700000) {};
\node[Kanji] at (-54.450000, -29.200000) {塁};
\node[Onyomi] at (-54.400000, -29.600000) {ルイ};
\node[Meaning] at (-54.450000, -27.950000) {base};
\node[Square] at (-52.400000, -29.700000) {};
\node[Kanji] at (-52.400000, -29.200000) {顧};
\node[Onyomi] at (-52.350000, -29.600000) {コ};
\node[Kunyomi] at (-52.450000, -29.600000) {かえり.みる};
\node[Meaning] at (-52.400000, -27.950000) {review};
\node[Square] at (-50.350000, -29.700000) {};
\node[Kanji] at (-50.350000, -29.200000) {庄};
\node[Onyomi] at (-50.300000, -29.600000) {ショウ};
\node[Meaning] at (-50.350000, -27.950000) {manor};
\node[Square] at (-48.300000, -29.700000) {};
\node[Kanji] at (-48.300000, -29.200000) {炊};
\node[Onyomi] at (-48.250000, -29.600000) {スイ};
\node[Kunyomi] at (-48.350000, -29.600000) {た.く};
\node[Meaning] at (-48.300000, -27.950000) {cook};
\node[Square] at (-46.250000, -29.700000) {};
\node[Kanji] at (-46.250000, -29.200000) {陛};
\node[Onyomi] at (-46.200000, -29.600000) {ヘイ};
\node[Meaning] at (-46.250000, -27.950000) {highness};
\node[Square] at (-44.200000, -29.700000) {};
\node[Kanji] at (-44.200000, -29.200000) {把};
\node[Onyomi] at (-44.150000, -29.600000) {ワ};
\node[Meaning] at (-44.200000, -27.950000) {bundle};
\node[Square] at (-42.150000, -29.700000) {};
\node[Kanji] at (-42.150000, -29.200000) {邸};
\node[Onyomi] at (-42.100000, -29.600000) {テイ};
\node[Kunyomi] at (-42.200000, -29.600000) {やしき};
\node[Meaning] at (-42.150000, -27.950000) {residence};
\node[Square] at (-40.100000, -29.700000) {};
\node[Kanji] at (-40.100000, -29.200000) {稿};
\node[Onyomi] at (-40.050000, -29.600000) {コウ};
\node[Kunyomi] at (-40.150000, -29.600000) {したがき};
\node[Meaning] at (-40.100000, -27.950000) {draft};
\node[Square] at (-38.050000, -29.700000) {};
\node[Kanji] at (-38.050000, -29.200000) {騰};
\node[Onyomi] at (-38.000000, -29.600000) {トウ};
\node[Kunyomi] at (-38.100000, -29.600000) {あが};
\node[Meaning] at (-38.050000, -27.950000) {inflation};
\node[Square] at (-36.000000, -29.700000) {};
\node[Kanji] at (-36.000000, -29.200000) {酬};
\node[Onyomi] at (-35.950000, -29.600000) {シュウ};
\node[Kunyomi] at (-36.050000, -29.600000) {むく};
\node[Meaning] at (-36.000000, -27.950000) {repay};
\node[Square] at (-33.950000, -29.700000) {};
\node[Kanji] at (-33.950000, -29.200000) {洪};
\node[Onyomi] at (-33.900000, -29.600000) {コウ};
\node[Meaning] at (-33.950000, -27.950000) {flood};
\node[Square] at (-31.900000, -29.700000) {};
\node[Kanji] at (-31.900000, -29.200000) {剖};
\node[Onyomi] at (-31.850000, -29.600000) {ボウ};
\node[Meaning] at (-31.900000, -27.950000) {divide};
\node[Square] at (-29.850000, -29.700000) {};
\node[Kanji] at (-29.850000, -29.200000) {帆};
\node[Onyomi] at (-29.800000, -29.600000) {ハン};
\node[Kunyomi] at (-29.900000, -29.600000) {ほ};
\node[Meaning] at (-29.850000, -27.950000) {sail};
\node[Square] at (-27.800000, -29.700000) {};
\node[Kanji] at (-27.800000, -29.200000) {暁};
\node[Onyomi] at (-27.750000, -29.600000) {キョウ};
\node[Kunyomi] at (-27.850000, -29.600000) {あかつき};
\node[Meaning] at (-27.800000, -27.950000) {dawn};
\node[Square] at (-25.750000, -29.700000) {};
\node[Kanji] at (-25.750000, -29.200000) {礁};
\node[Onyomi] at (-25.700000, -29.600000) {ショウ};
\node[Meaning] at (-25.750000, -27.950000) {reef};
\node[Square] at (-23.700000, -29.700000) {};
\node[Kanji] at (-23.700000, -29.200000) {閑};
\node[Onyomi] at (-23.650000, -29.600000) {カン};
\node[Meaning] at (-23.700000, -27.950000) {leisure};
\node[Square] at (-21.650000, -29.700000) {};
\node[Kanji] at (-21.650000, -29.200000) {渓};
\node[Onyomi] at (-21.600000, -29.600000) {ケイ};
\node[Kunyomi] at (-21.700000, -29.600000) {たに};
\node[Meaning] at (-21.650000, -27.950000) {valley};
\node[Square] at (-19.600000, -29.700000) {};
\node[Kanji] at (-19.600000, -29.200000) {桁};
\node[Kunyomi] at (-19.650000, -29.600000) {けた};
\node[Meaning] at (-19.600000, -27.950000) {beam};
\node[Square] at (-17.550000, -29.700000) {};
\node[Kanji] at (-17.550000, -29.200000) {乞};
\node[Kunyomi] at (-17.600000, -29.600000) {こ-う};
\node[Meaning] at (-17.550000, -27.950000) {beg};
\node[Square] at (-15.500000, -29.700000) {};
\node[Kanji] at (-15.500000, -29.200000) {蔽};
\node[Onyomi] at (-15.450000, -29.600000) {ヘイ};
\node[Meaning] at (-15.500000, -27.950000) {cover};
\node[Square] at (-13.450000, -29.700000) {};
\node[Kanji] at (-13.450000, -29.200000) {較};
\node[Onyomi] at (-13.400000, -29.600000) {カク};
\node[Meaning] at (-13.450000, -27.950000) {contrast};
\node[Square] at (-11.400000, -29.700000) {};
\node[Kanji] at (-11.400000, -29.200000) {醤};
\node[Onyomi] at (-11.350000, -29.600000) {ショウ};
\node[Meaning] at (-11.400000, -27.950000) {soy sauce};
\node[Square] at (-9.350000, -29.700000) {};
\node[Kanji] at (-9.350000, -29.200000) {孝};
\node[Onyomi] at (-9.300000, -29.600000) {コウ};
\node[Meaning] at (-9.350000, -27.950000) {filial piety};
\node[Square] at (-7.300000, -29.700000) {};
\node[Kanji] at (-7.300000, -29.200000) {兼};
\node[Onyomi] at (-7.250000, -29.600000) {ケン};
\node[Kunyomi] at (-7.350000, -29.600000) {か.ねる};
\node[Meaning] at (-7.300000, -27.950000) {concurrently};
\node[Square] at (-5.250000, -29.700000) {};
\node[Kanji] at (-5.250000, -29.200000) {旨};
\node[Onyomi] at (-5.200000, -29.600000) {シ};
\node[Kunyomi] at (-5.300000, -29.600000) {うま.い};
\node[Meaning] at (-5.250000, -27.950000) {point};
\node[Square] at (-3.200000, -29.700000) {};
\node[Kanji] at (-3.200000, -29.200000) {矛};
\node[Onyomi] at (-3.150000, -29.600000) {ム};
\node[Kunyomi] at (-3.250000, -29.600000) {ほこ};
\node[Meaning] at (-3.200000, -27.950000) {halberd};
\node[Square] at (-1.150000, -29.700000) {};
\node[Kanji] at (-1.150000, -29.200000) {尺};
\node[Onyomi] at (-1.100000, -29.600000) {シャク};
\node[Meaning] at (-1.150000, -27.950000) {shaku};
\node[Square] at (0.900000, -29.700000) {};
\node[Kanji] at (0.900000, -29.200000) {挿};
\node[Onyomi] at (0.950000, -29.600000) {ソウ};
\node[Kunyomi] at (0.850000, -29.600000) {さ.す};
\node[Meaning] at (0.900000, -27.950000) {insert};
\node[Square] at (2.950000, -29.700000) {};
\node[Kanji] at (2.950000, -29.200000) {幣};
\node[Onyomi] at (3.000000, -29.600000) {ヘイ};
\node[Meaning] at (2.950000, -27.950000) {cash};
\node[Square] at (5.000000, -29.700000) {};
\node[Kanji] at (5.000000, -29.200000) {祥};
\node[Onyomi] at (5.050000, -29.600000) {ショウ};
\node[Kunyomi] at (4.950000, -29.600000) {きざ};
\node[Meaning] at (5.000000, -27.950000) {auspicious};
\node[Square] at (7.050000, -29.700000) {};
\node[Kanji] at (7.050000, -29.200000) {陳};
\node[Onyomi] at (7.100000, -29.600000) {チン};
\node[Kunyomi] at (7.000000, -29.600000) {ひ.ねる};
\node[Meaning] at (7.050000, -27.950000) {exhibit};
\node[Square] at (9.100000, -29.700000) {};
\node[Kanji] at (9.100000, -29.200000) {践};
\node[Onyomi] at (9.150000, -29.600000) {セン};
\node[Kunyomi] at (9.050000, -29.600000) {ふ};
\node[Meaning] at (9.100000, -27.950000) {practice};
\node[Square] at (11.150000, -29.700000) {};
\node[Kanji] at (11.150000, -29.200000) {佳};
\node[Onyomi] at (11.200000, -29.600000) {カ};
\node[Meaning] at (11.150000, -27.950000) {excellent};
\node[Square] at (13.200000, -29.700000) {};
\node[Kanji] at (13.200000, -29.200000) {循};
\node[Onyomi] at (13.250000, -29.600000) {ジュン};
\node[Meaning] at (13.200000, -27.950000) {circulation};
\node[Square] at (15.250000, -29.700000) {};
\node[Kanji] at (15.250000, -29.200000) {柴};
\node[Onyomi] at (15.300000, -29.600000) {サイ};
\node[Kunyomi] at (15.200000, -29.600000) {しば};
\node[Meaning] at (15.250000, -27.950000) {brushwood};
\node[Square] at (17.300000, -29.700000) {};
\node[Kanji] at (17.300000, -29.200000) {匠};
\node[Onyomi] at (17.350000, -29.600000) {ショウ};
\node[Kunyomi] at (17.250000, -29.600000) {たくみ};
\node[Meaning] at (17.300000, -27.950000) {artisan};
\node[Square] at (19.350000, -29.700000) {};
\node[Kanji] at (19.350000, -29.200000) {醸};
\node[Onyomi] at (19.400000, -29.600000) {ジョウ};
\node[Kunyomi] at (19.300000, -29.600000) {かも};
\node[Meaning] at (19.350000, -27.950000) {brew};
\node[Square] at (21.400000, -29.700000) {};
\node[Kanji] at (21.400000, -29.200000) {漣};
\node[Onyomi] at (21.450000, -29.600000) {レン};
\node[Kunyomi] at (21.350000, -29.600000) {さざなみ};
\node[Meaning] at (21.400000, -27.950000) {ripples};
\node[Square] at (23.450000, -29.700000) {};
\node[Kanji] at (23.450000, -29.200000) {姻};
\node[Onyomi] at (23.500000, -29.600000) {イン};
\node[Meaning] at (23.450000, -27.950000) {marry};
\node[Square] at (25.500000, -29.700000) {};
\node[Kanji] at (25.500000, -29.200000) {漆};
\node[Onyomi] at (25.550000, -29.600000) {シツ};
\node[Kunyomi] at (25.450000, -29.600000) {うるし};
\node[Meaning] at (25.500000, -27.950000) {lacquer};
\node[Square] at (27.550000, -29.700000) {};
\node[Kanji] at (27.550000, -29.200000) {僅};
\node[Onyomi] at (27.600000, -29.600000) {キン};
\node[Kunyomi] at (27.500000, -29.600000) {わず-か};
\node[Meaning] at (27.550000, -27.950000) {a wee bit};
\node[Square] at (29.600000, -29.700000) {};
\node[Kanji] at (29.600000, -29.200000) {宵};
\node[Onyomi] at (29.650000, -29.600000) {ショウ};
\node[Kunyomi] at (29.550000, -29.600000) {よい};
\node[Meaning] at (29.600000, -27.950000) {wee hours};
\node[Square] at (31.650000, -29.700000) {};
\node[Kanji] at (31.650000, -29.200000) {吏};
\node[Onyomi] at (31.700000, -29.600000) {リ};
\node[Meaning] at (31.650000, -27.950000) {officer};
\node[Square] at (33.700000, -29.700000) {};
\node[Kanji] at (33.700000, -29.200000) {乏};
\node[Onyomi] at (33.750000, -29.600000) {ボウ};
\node[Kunyomi] at (33.650000, -29.600000) {とぼ.しい};
\node[Meaning] at (33.700000, -27.950000) {scarce};
\node[Square] at (35.750000, -29.700000) {};
\node[Kanji] at (35.750000, -29.200000) {賄};
\node[Onyomi] at (35.800000, -29.600000) {ワイ};
\node[Kunyomi] at (35.700000, -29.600000) {まかな.う};
\node[Meaning] at (35.750000, -27.950000) {bribe};
\node[Square] at (37.800000, -29.700000) {};
\node[Kanji] at (37.800000, -29.200000) {賂};
\node[Onyomi] at (37.850000, -29.600000) {ロ};
\node[Meaning] at (37.800000, -27.950000) {bribe};
\node[Square] at (39.850000, -29.700000) {};
\node[Kanji] at (39.850000, -29.200000) {析};
\node[Onyomi] at (39.900000, -29.600000) {セキ};
\node[Meaning] at (39.850000, -27.950000) {analysis};
\node[Square] at (41.900000, -29.700000) {};
\node[Kanji] at (41.900000, -29.200000) {穂};
\node[Onyomi] at (41.950000, -29.600000) {スイ};
\node[Kunyomi] at (41.850000, -29.600000) {ほ};
\node[Meaning] at (41.900000, -27.950000) {head of plant};
\node[Square] at (43.950000, -29.700000) {};
\node[Kanji] at (43.950000, -29.200000) {瞭};
\node[Onyomi] at (44.000000, -29.600000) {リョウ};
\node[Meaning] at (43.950000, -27.950000) {clear};
\node[Square] at (46.000000, -29.700000) {};
\node[Kanji] at (46.000000, -29.200000) {餓};
\node[Onyomi] at (46.050000, -29.600000) {ガ};
\node[Kunyomi] at (45.950000, -29.600000) {う.える};
\node[Meaning] at (46.000000, -27.950000) {starve};
\node[Square] at (48.050000, -29.700000) {};
\node[Kanji] at (48.050000, -29.200000) {耕};
\node[Onyomi] at (48.100000, -29.600000) {コウ};
\node[Kunyomi] at (48.000000, -29.600000) {たがや.す};
\node[Meaning] at (48.050000, -27.950000) {plow};
\node[Square] at (50.100000, -29.700000) {};
\node[Kanji] at (50.100000, -29.200000) {賠};
\node[Onyomi] at (50.150000, -29.600000) {バイ};
\node[Meaning] at (50.100000, -27.950000) {compensation};
\node[Square] at (52.150000, -29.700000) {};
\node[Kanji] at (52.150000, -29.200000) {綾};
\node[Onyomi] at (52.200000, -29.600000) {リン};
\node[Kunyomi] at (52.100000, -29.600000) {あや};
\node[Meaning] at (52.150000, -27.950000) {design};
\node[Square] at (54.200000, -29.700000) {};
\node[Kanji] at (54.200000, -29.200000) {倫};
\node[Onyomi] at (54.250000, -29.600000) {リン};
\node[Meaning] at (54.200000, -27.950000) {ethics};
\node[Square] at (56.250000, -29.700000) {};
\node[Kanji] at (56.250000, -29.200000) {猟};
\node[Onyomi] at (56.300000, -29.600000) {リョウ};
\node[Kunyomi] at (56.200000, -29.600000) {かり};
\node[Meaning] at (56.250000, -27.950000) {hunting};
\node[Meaning] at (-58.500000, -29.150000) {98.27\%};
\node[Square] at (-56.500000, -31.750000) {};
\node[Kanji] at (-56.500000, -31.250000) {浄};
\node[Onyomi] at (-56.450000, -31.650000) {ジョウ};
\node[Kunyomi] at (-56.550000, -31.650000) {きよ.い};
\node[Meaning] at (-56.500000, -30.000000) {cleanse};
\node[Square] at (-54.450000, -31.750000) {};
\node[Kanji] at (-54.450000, -31.250000) {呈};
\node[Onyomi] at (-54.400000, -31.650000) {テイ};
\node[Meaning] at (-54.450000, -30.000000) {present};
\node[Square] at (-52.400000, -31.750000) {};
\node[Kanji] at (-52.400000, -31.250000) {哺};
\node[Onyomi] at (-52.350000, -31.650000) {ホ};
\node[Kunyomi] at (-52.450000, -31.650000) {ほぐく};
\node[Meaning] at (-52.400000, -30.000000) {nurse};
\node[Square] at (-50.350000, -31.750000) {};
\node[Kanji] at (-50.350000, -31.250000) {赴};
\node[Onyomi] at (-50.300000, -31.650000) {フ};
\node[Kunyomi] at (-50.400000, -31.650000) {おもむ};
\node[Meaning] at (-50.350000, -30.000000) {proceed};
\node[Square] at (-48.300000, -31.750000) {};
\node[Kanji] at (-48.300000, -31.250000) {脊};
\node[Onyomi] at (-48.250000, -31.650000) {セキ};
\node[Kunyomi] at (-48.350000, -31.650000) {せせい};
\node[Meaning] at (-48.300000, -30.000000) {stature};
\node[Square] at (-46.250000, -31.750000) {};
\node[Kanji] at (-46.250000, -31.250000) {轄};
\node[Onyomi] at (-46.200000, -31.650000) {カツ};
\node[Kunyomi] at (-46.300000, -31.650000) {くさび};
\node[Meaning] at (-46.250000, -30.000000) {control};
\node[Square] at (-44.200000, -31.750000) {};
\node[Kanji] at (-44.200000, -31.250000) {賊};
\node[Onyomi] at (-44.150000, -31.650000) {ゾク};
\node[Meaning] at (-44.200000, -30.000000) {robber};
\node[Square] at (-42.150000, -31.750000) {};
\node[Kanji] at (-42.150000, -31.250000) {陪};
\node[Onyomi] at (-42.100000, -31.650000) {バイ};
\node[Meaning] at (-42.150000, -30.000000) {accompany};
\node[Square] at (-40.100000, -31.750000) {};
\node[Kanji] at (-40.100000, -31.250000) {岬};
\node[Onyomi] at (-40.050000, -31.650000) {コウ};
\node[Kunyomi] at (-40.150000, -31.650000) {みさき};
\node[Meaning] at (-40.100000, -30.000000) {cape};
\node[Square] at (-38.050000, -31.750000) {};
\node[Kanji] at (-38.050000, -31.250000) {亜};
\node[Onyomi] at (-38.000000, -31.650000) {ア};
\node[Kunyomi] at (-38.100000, -31.650000) {つ};
\node[Meaning] at (-38.050000, -30.000000) {asia};
\node[Square] at (-36.000000, -31.750000) {};
\node[Kanji] at (-36.000000, -31.250000) {升};
\node[Onyomi] at (-35.950000, -31.650000) {ショウ};
\node[Kunyomi] at (-36.050000, -31.650000) {ます};
\node[Meaning] at (-36.000000, -30.000000) {grid};
\node[Square] at (-33.950000, -31.750000) {};
\node[Kanji] at (-33.950000, -31.250000) {坑};
\node[Onyomi] at (-33.900000, -31.650000) {コウ};
\node[Meaning] at (-33.950000, -30.000000) {pit};
\node[Square] at (-31.900000, -31.750000) {};
\node[Kanji] at (-31.900000, -31.250000) {謹};
\node[Onyomi] at (-31.850000, -31.650000) {キン};
\node[Kunyomi] at (-31.950000, -31.650000) {つつし};
\node[Meaning] at (-31.900000, -30.000000) {humble};
\node[Square] at (-29.850000, -31.750000) {};
\node[Kanji] at (-29.850000, -31.250000) {詣};
\node[Onyomi] at (-29.800000, -31.650000) {ケイ};
\node[Kunyomi] at (-29.900000, -31.650000) {もう-でる};
\node[Meaning] at (-29.850000, -30.000000) {visit a temple};
\node[Square] at (-27.800000, -31.750000) {};
\node[Kanji] at (-27.800000, -31.250000) {遵};
\node[Onyomi] at (-27.750000, -31.650000) {ジュン};
\node[Meaning] at (-27.800000, -30.000000) {abide by};
\node[Square] at (-25.750000, -31.750000) {};
\node[Kanji] at (-25.750000, -31.250000) {附};
\node[Onyomi] at (-25.700000, -31.650000) {フ};
\node[Meaning] at (-25.750000, -30.000000) {affixed};
\node[Square] at (-23.700000, -31.750000) {};
\node[Kanji] at (-23.700000, -31.250000) {麺};
\node[Onyomi] at (-23.650000, -31.650000) {メン};
\node[Meaning] at (-23.700000, -30.000000) {noodles};
\node[Square] at (-21.650000, -31.750000) {};
\node[Kanji] at (-21.650000, -31.250000) {侶};
\node[Onyomi] at (-21.600000, -31.650000) {リョ};
\node[Meaning] at (-21.650000, -30.000000) {companion};
\node[Square] at (-19.600000, -31.750000) {};
\node[Kanji] at (-19.600000, -31.250000) {弄};
\node[Onyomi] at (-19.550000, -31.650000) {ロウ};
\node[Kunyomi] at (-19.650000, -31.650000) {もてあそ-ぶ};
\node[Meaning] at (-19.600000, -30.000000) {tamper with};
\node[Square] at (-17.550000, -31.750000) {};
\node[Kanji] at (-17.550000, -31.250000) {又};
\node[Kunyomi] at (-17.600000, -31.650000) {また};
\node[Meaning] at (-17.550000, -30.000000) {again};
\node[Square] at (-15.500000, -31.750000) {};
\node[Kanji] at (-15.500000, -31.250000) {禅};
\node[Onyomi] at (-15.450000, -31.650000) {ゼン};
\node[Meaning] at (-15.500000, -30.000000) {zen};
\node[Square] at (-13.450000, -31.750000) {};
\node[Kanji] at (-13.450000, -31.250000) {既};
\node[Onyomi] at (-13.400000, -31.650000) {キ};
\node[Kunyomi] at (-13.500000, -31.650000) {すで};
\node[Meaning] at (-13.450000, -30.000000) {previously};
\node[Square] at (-11.400000, -31.750000) {};
\node[Kanji] at (-11.400000, -31.250000) {蛍};
\node[Onyomi] at (-11.350000, -31.650000) {ケイ};
\node[Kunyomi] at (-11.450000, -31.650000) {ほたる};
\node[Meaning] at (-11.400000, -30.000000) {firefly};
\node[Square] at (-9.350000, -31.750000) {};
\node[Kanji] at (-9.350000, -31.250000) {墟};
\node[Onyomi] at (-9.300000, -31.650000) {キョ};
\node[Meaning] at (-9.350000, -30.000000) {ruins};
\node[Square] at (-7.300000, -31.750000) {};
\node[Kanji] at (-7.300000, -31.250000) {遜};
\node[Onyomi] at (-7.250000, -31.650000) {ソン};
\node[Kunyomi] at (-7.350000, -31.650000) {したが.う};
\node[Meaning] at (-7.300000, -30.000000) {humble};
\node[Square] at (-5.250000, -31.750000) {};
\node[Kanji] at (-5.250000, -31.250000) {斎};
\node[Onyomi] at (-5.200000, -31.650000) {サイ};
\node[Kunyomi] at (-5.300000, -31.650000) {いつ.く};
\node[Meaning] at (-5.250000, -30.000000) {purification};
\node[Square] at (-3.200000, -31.750000) {};
\node[Kanji] at (-3.200000, -31.250000) {迅};
\node[Onyomi] at (-3.150000, -31.650000) {ジン};
\node[Meaning] at (-3.200000, -30.000000) {swift};
\node[Square] at (-1.150000, -31.750000) {};
\node[Kanji] at (-1.150000, -31.250000) {亮};
\node[Onyomi] at (-1.100000, -31.650000) {リョウ};
\node[Kunyomi] at (-1.200000, -31.650000) {あきらか        };
\node[Meaning] at (-1.150000, -30.000000) {clear};
\node[Square] at (0.900000, -31.750000) {};
\node[Kanji] at (0.900000, -31.250000) {乃};
\node[Onyomi] at (0.950000, -31.650000) {ナイ};
\node[Kunyomi] at (0.850000, -31.650000) {すなわ};
\node[Meaning] at (0.900000, -30.000000) {from};
\node[Square] at (2.950000, -31.750000) {};
\node[Kanji] at (2.950000, -31.250000) {諭};
\node[Onyomi] at (3.000000, -31.650000) {ユ};
\node[Kunyomi] at (2.900000, -31.650000) {さと};
\node[Meaning] at (2.950000, -30.000000) {admonish};
\node[Square] at (5.000000, -31.750000) {};
\node[Kanji] at (5.000000, -31.250000) {随};
\node[Onyomi] at (5.050000, -31.650000) {ズイ};
\node[Kunyomi] at (4.950000, -31.650000) {したが.う};
\node[Meaning] at (5.000000, -30.000000) {all};
\node[Square] at (7.050000, -31.750000) {};
\node[Kanji] at (7.050000, -31.250000) {栽};
\node[Onyomi] at (7.100000, -31.650000) {サイ};
\node[Meaning] at (7.050000, -30.000000) {planting};
\node[Square] at (9.100000, -31.750000) {};
\node[Kanji] at (9.100000, -31.250000) {猶};
\node[Onyomi] at (9.150000, -31.650000) {ユウ};
\node[Kunyomi] at (9.050000, -31.650000) {なお};
\node[Meaning] at (9.100000, -30.000000) {still};
\node[Square] at (11.150000, -31.750000) {};
\node[Kanji] at (11.150000, -31.250000) {庸};
\node[Onyomi] at (11.200000, -31.650000) {ヨウ};
\node[Meaning] at (11.150000, -30.000000) {common};
\node[Square] at (13.200000, -31.750000) {};
\node[Kanji] at (13.200000, -31.250000) {譜};
\node[Onyomi] at (13.250000, -31.650000) {フ};
\node[Meaning] at (13.200000, -30.000000) {genealogy};
\node[Square] at (15.250000, -31.750000) {};
\node[Kanji] at (15.250000, -31.250000) {窃};
\node[Onyomi] at (15.300000, -31.650000) {セツ};
\node[Kunyomi] at (15.200000, -31.650000) {ぬす};
\node[Meaning] at (15.250000, -30.000000) {steal};
\node[Square] at (17.300000, -31.750000) {};
\node[Kanji] at (17.300000, -31.250000) {奔};
\node[Onyomi] at (17.350000, -31.650000) {ホン};
\node[Kunyomi] at (17.250000, -31.650000) {はし.る};
\node[Meaning] at (17.300000, -30.000000) {run};
\node[Square] at (19.350000, -31.750000) {};
\node[Kanji] at (19.350000, -31.250000) {甚};
\node[Onyomi] at (19.400000, -31.650000) {ジン};
\node[Kunyomi] at (19.300000, -31.650000) {はなは};
\node[Meaning] at (19.350000, -30.000000) {very};
\node[Square] at (21.400000, -31.750000) {};
\node[Kanji] at (21.400000, -31.250000) {紡};
\node[Onyomi] at (21.450000, -31.650000) {ボウ};
\node[Kunyomi] at (21.350000, -31.650000) {つむ};
\node[Meaning] at (21.400000, -30.000000) {spinning};
\node[Square] at (23.450000, -31.750000) {};
\node[Kanji] at (23.450000, -31.250000) {唄};
\node[Onyomi] at (23.500000, -31.650000) {バイ};
\node[Kunyomi] at (23.400000, -31.650000) {うた};
\node[Meaning] at (23.450000, -30.000000) {shamisen song};
\node[Square] at (25.500000, -31.750000) {};
\node[Kanji] at (25.500000, -31.250000) {矯};
\node[Onyomi] at (25.550000, -31.650000) {キョウ};
\node[Kunyomi] at (25.450000, -31.650000) {た};
\node[Meaning] at (25.500000, -30.000000) {correct};
\node[Square] at (27.550000, -31.750000) {};
\node[Kanji] at (27.550000, -31.250000) {藻};
\node[Onyomi] at (27.600000, -31.650000) {ソウ};
\node[Kunyomi] at (27.500000, -31.650000) {も};
\node[Meaning] at (27.550000, -30.000000) {seaweed};
\node[Square] at (29.600000, -31.750000) {};
\node[Kanji] at (29.600000, -31.250000) {臼};
\node[Onyomi] at (29.650000, -31.650000) {キュウ};
\node[Kunyomi] at (29.550000, -31.650000) {うす};
\node[Meaning] at (29.600000, -30.000000) {mortar};
\node[Square] at (31.650000, -31.750000) {};
\node[Kanji] at (31.650000, -31.250000) {薫};
\node[Onyomi] at (31.700000, -31.650000) {クン};
\node[Kunyomi] at (31.600000, -31.650000) {かお-る};
\node[Meaning] at (31.650000, -30.000000) {fragrant};
\node[Square] at (33.700000, -31.750000) {};
\node[Kanji] at (33.700000, -31.250000) {繭};
\node[Onyomi] at (33.750000, -31.650000) {ケン};
\node[Kunyomi] at (33.650000, -31.650000) {まゆ};
\node[Meaning] at (33.700000, -30.000000) {cocoon};
\node[Square] at (35.750000, -31.750000) {};
\node[Kanji] at (35.750000, -31.250000) {嗣};
\node[Onyomi] at (35.800000, -31.650000) {シ};
\node[Meaning] at (35.750000, -30.000000) {heir};
\node[Square] at (37.800000, -31.750000) {};
\node[Kanji] at (37.800000, -31.250000) {逐};
\node[Onyomi] at (37.850000, -31.650000) {チク};
\node[Meaning] at (37.800000, -30.000000) {pursue};
\node[Square] at (39.850000, -31.750000) {};
\node[Kanji] at (39.850000, -31.250000) {濫};
\node[Onyomi] at (39.900000, -31.650000) {ラン};
\node[Meaning] at (39.850000, -30.000000) {excessive};
\node[Square] at (41.900000, -31.750000) {};
\node[Kanji] at (41.900000, -31.250000) {峰};
\node[Onyomi] at (41.950000, -31.650000) {ホウ};
\node[Kunyomi] at (41.850000, -31.650000) {みね};
\node[Meaning] at (41.900000, -30.000000) {summit};
\node[Square] at (43.950000, -31.750000) {};
\node[Kanji] at (43.950000, -31.250000) {輔};
\node[Onyomi] at (44.000000, -31.650000) {フ};
\node[Kunyomi] at (43.900000, -31.650000) {たす.ける};
\node[Meaning] at (43.950000, -30.000000) {help};
\node[Square] at (46.000000, -31.750000) {};
\node[Kanji] at (46.000000, -31.250000) {緋};
\node[Onyomi] at (46.050000, -31.650000) {ヒ};
\node[Kunyomi] at (45.950000, -31.650000) {あか};
\node[Meaning] at (46.000000, -30.000000) {scarlet};
\node[Square] at (48.050000, -31.750000) {};
\node[Kanji] at (48.050000, -31.250000) {蒙};
\node[Onyomi] at (48.100000, -31.650000) {モウ};
\node[Kunyomi] at (48.000000, -31.650000) {おお};
\node[Meaning] at (48.050000, -30.000000) {darkness};
\node[Square] at (50.100000, -31.750000) {};
\node[Kanji] at (50.100000, -31.250000) {戴};
\node[Onyomi] at (50.150000, -31.650000) {タイ};
\node[Kunyomi] at (50.050000, -31.650000) {いただ};
\node[Meaning] at (50.100000, -30.000000) {receive};
\node[Square] at (52.150000, -31.750000) {};
\node[Kanji] at (52.150000, -31.250000) {荘};
\node[Onyomi] at (52.200000, -31.650000) {ソウ};
\node[Kunyomi] at (52.100000, -31.650000) {あごそ};
\node[Meaning] at (52.150000, -30.000000) {villa};
\node[Square] at (54.200000, -31.750000) {};
\node[Kanji] at (54.200000, -31.250000) {卸};
\node[Onyomi] at (54.250000, -31.650000) {シャ};
\node[Kunyomi] at (54.150000, -31.650000) {おろし};
\node[Meaning] at (54.200000, -30.000000) {wholesale};
\node[Square] at (56.250000, -31.750000) {};
\node[Kanji] at (56.250000, -31.250000) {尚};
\node[Onyomi] at (56.300000, -31.650000) {ショウ};
\node[Kunyomi] at (56.200000, -31.650000) {なお};
\node[Meaning] at (56.250000, -30.000000) {furthermore};
\node[Meaning] at (-58.500000, -31.200000) {98.29\%};
\node[Square] at (-56.500000, -33.800000) {};
\node[Kanji] at (-56.500000, -33.300000) {累};
\node[Onyomi] at (-56.450000, -33.700000) {ルイ};
\node[Meaning] at (-56.500000, -32.050000) {accumulate};
\node[Square] at (-54.450000, -33.800000) {};
\node[Kanji] at (-54.450000, -33.300000) {該};
\node[Onyomi] at (-54.400000, -33.700000) {ガイ};
\node[Meaning] at (-54.450000, -32.050000) {the above};
\node[Square] at (-52.400000, -33.800000) {};
\node[Kanji] at (-52.400000, -33.300000) {遼};
\node[Onyomi] at (-52.350000, -33.700000) {リョウ};
\node[Meaning] at (-52.400000, -32.050000) {distant};
\node[Square] at (-50.350000, -33.800000) {};
\node[Kanji] at (-50.350000, -33.300000) {穀};
\node[Onyomi] at (-50.300000, -33.700000) {コク};
\node[Meaning] at (-50.350000, -32.050000) {grain};
\node[Square] at (-48.300000, -33.800000) {};
\node[Kanji] at (-48.300000, -33.300000) {凛};
\node[Onyomi] at (-48.250000, -33.700000) {リン};
\node[Kunyomi] at (-48.350000, -33.700000) {きびし};
\node[Meaning] at (-48.300000, -32.050000) {cold};
\node[Square] at (-46.250000, -33.800000) {};
\node[Kanji] at (-46.250000, -33.300000) {弔};
\node[Onyomi] at (-46.200000, -33.700000) {チョウ};
\node[Kunyomi] at (-46.300000, -33.700000) {とぶら.う};
\node[Meaning] at (-46.250000, -32.050000) {condolence};
\node[Square] at (-44.200000, -33.800000) {};
\node[Kanji] at (-44.200000, -33.300000) {楓};
\node[Onyomi] at (-44.150000, -33.700000) {フウ};
\node[Kunyomi] at (-44.250000, -33.700000) {かえで};
\node[Meaning] at (-44.200000, -32.050000) {maple};
\node[Square] at (-42.150000, -33.800000) {};
\node[Kanji] at (-42.150000, -33.300000) {藩};
\node[Onyomi] at (-42.100000, -33.700000) {ハン};
\node[Meaning] at (-42.150000, -32.050000) {fiefdom};
\node[Square] at (-40.100000, -33.800000) {};
\node[Kanji] at (-40.100000, -33.300000) {屯};
\node[Onyomi] at (-40.050000, -33.700000) {トン};
\node[Meaning] at (-40.100000, -32.050000) {barracks};
\node[Square] at (-38.050000, -33.800000) {};
\node[Kanji] at (-38.050000, -33.300000) {鋳};
\node[Onyomi] at (-38.000000, -33.700000) {チュウ};
\node[Kunyomi] at (-38.100000, -33.700000) {い};
\node[Meaning] at (-38.050000, -32.050000) {cast};
\node[Square] at (-36.000000, -33.800000) {};
\node[Kanji] at (-36.000000, -33.300000) {慕};
\node[Onyomi] at (-35.950000, -33.700000) {ボ};
\node[Kunyomi] at (-36.050000, -33.700000) {した};
\node[Meaning] at (-36.000000, -32.050000) {yearn for};
\node[Square] at (-33.950000, -33.800000) {};
\node[Kanji] at (-33.950000, -33.300000) {韻};
\node[Onyomi] at (-33.900000, -33.700000) {イン};
\node[Meaning] at (-33.950000, -32.050000) {rhyme};
\node[Square] at (-31.900000, -33.800000) {};
\node[Kanji] at (-31.900000, -33.300000) {怨};
\node[Onyomi] at (-31.850000, -33.700000) {エン};
\node[Meaning] at (-31.900000, -32.050000) {grudge};
\node[Square] at (-29.850000, -33.800000) {};
\node[Kanji] at (-29.850000, -33.300000) {旺};
\node[Onyomi] at (-29.800000, -33.700000) {オウ};
\node[Meaning] at (-29.850000, -32.050000) {flourishing};
\node[Square] at (-27.800000, -33.800000) {};
\node[Kanji] at (-27.800000, -33.300000) {勾};
\node[Onyomi] at (-27.750000, -33.700000) {コウ};
\node[Meaning] at (-27.800000, -32.050000) {capture};
\node[Square] at (-25.750000, -33.800000) {};
\node[Kanji] at (-25.750000, -33.300000) {愁};
\node[Onyomi] at (-25.700000, -33.700000) {シュウ};
\node[Kunyomi] at (-25.800000, -33.700000) {うれ-える};
\node[Meaning] at (-25.750000, -32.050000) {distress};
\node[Square] at (-23.700000, -33.800000) {};
\node[Kanji] at (-23.700000, -33.300000) {腎};
\node[Onyomi] at (-23.650000, -33.700000) {ジン};
\node[Meaning] at (-23.700000, -32.050000) {kidney};
\node[Square] at (-21.650000, -33.800000) {};
\node[Kanji] at (-21.650000, -33.300000) {嫡};
\node[Onyomi] at (-21.600000, -33.700000) {チャク};
\node[Meaning] at (-21.650000, -32.050000) {legitimate wife};
\node[Square] at (-19.600000, -33.800000) {};
\node[Kanji] at (-19.600000, -33.300000) {訃};
\node[Onyomi] at (-19.550000, -33.700000) {フ};
\node[Meaning] at (-19.600000, -32.050000) {obituary};
\node[Square] at (-17.550000, -33.800000) {};
\node[Kanji] at (-17.550000, -33.300000) {耗};
\node[Onyomi] at (-17.500000, -33.700000) {モウ};
\node[Meaning] at (-17.550000, -32.050000) {decrease};
\node[Square] at (-15.500000, -33.800000) {};
\node[Kanji] at (-15.500000, -33.300000) {麓};
\node[Onyomi] at (-15.450000, -33.700000) {ロク};
\node[Kunyomi] at (-15.550000, -33.700000) {ふもと};
\node[Meaning] at (-15.500000, -32.050000) {foothills};
\node[Square] at (-13.450000, -33.800000) {};
\node[Kanji] at (-13.450000, -33.300000) {需};
\node[Onyomi] at (-13.400000, -33.700000) {ジュ};
\node[Meaning] at (-13.450000, -32.050000) {demand};
\node[Square] at (-11.400000, -33.800000) {};
\node[Kanji] at (-11.400000, -33.300000) {斐};
\node[Onyomi] at (-11.350000, -33.700000) {イ};
\node[Meaning] at (-11.400000, -32.050000) {patterned};
\node[Square] at (-9.350000, -33.800000) {};
\node[Kanji] at (-9.350000, -33.300000) {軸};
\node[Onyomi] at (-9.300000, -33.700000) {ジク};
\node[Meaning] at (-9.350000, -32.050000) {axis};
\node[Square] at (-7.300000, -33.800000) {};
\node[Kanji] at (-7.300000, -33.300000) {吾};
\node[Onyomi] at (-7.250000, -33.700000) {ゴ};
\node[Kunyomi] at (-7.350000, -33.700000) {わが};
\node[Meaning] at (-7.300000, -32.050000) {i};
\node[Square] at (-5.250000, -33.800000) {};
\node[Kanji] at (-5.250000, -33.300000) {斬};
\node[Onyomi] at (-5.200000, -33.700000) {ザン};
\node[Kunyomi] at (-5.300000, -33.700000) {き.る};
\node[Meaning] at (-5.250000, -32.050000) {slice};
\node[Square] at (-3.200000, -33.800000) {};
\node[Kanji] at (-3.200000, -33.300000) {妃};
\node[Onyomi] at (-3.150000, -33.700000) {ヒ};
\node[Meaning] at (-3.200000, -32.050000) {princess};
\node[Square] at (-1.150000, -33.800000) {};
\node[Kanji] at (-1.150000, -33.300000) {后};
\node[Onyomi] at (-1.100000, -33.700000) {コウ};
\node[Kunyomi] at (-1.200000, -33.700000) {きさき};
\node[Meaning] at (-1.150000, -32.050000) {empress};
\node[Square] at (0.900000, -33.800000) {};
\node[Kanji] at (0.900000, -33.300000) {貞};
\node[Onyomi] at (0.950000, -33.700000) {テイ};
\node[Kunyomi] at (0.850000, -33.700000) {さだ};
\node[Meaning] at (0.900000, -32.050000) {chastity};
\node[Square] at (2.950000, -33.800000) {};
\node[Kanji] at (2.950000, -33.300000) {肯};
\node[Onyomi] at (3.000000, -33.700000) {コウ};
\node[Kunyomi] at (2.900000, -33.700000) {がえんじ.る};
\node[Meaning] at (2.950000, -32.050000) {agreement};
\node[Square] at (5.000000, -33.800000) {};
\node[Kanji] at (5.000000, -33.300000) {糧};
\node[Onyomi] at (5.050000, -33.700000) {リョウ};
\node[Kunyomi] at (4.950000, -33.700000) {かて};
\node[Meaning] at (5.000000, -32.050000) {provisions};
\node[Square] at (7.050000, -33.800000) {};
\node[Kanji] at (7.050000, -33.300000) {披};
\node[Onyomi] at (7.100000, -33.700000) {ヒ};
\node[Meaning] at (7.050000, -32.050000) {expose};
\node[Square] at (9.100000, -33.800000) {};
\node[Kanji] at (9.100000, -33.300000) {謡};
\node[Onyomi] at (9.150000, -33.700000) {ヨウ};
\node[Kunyomi] at (9.050000, -33.700000) {うた};
\node[Meaning] at (9.100000, -32.050000) {noh chanting};
\node[Square] at (11.150000, -33.800000) {};
\node[Kanji] at (11.150000, -33.300000) {杏};
\node[Onyomi] at (11.200000, -33.700000) {アン};
\node[Kunyomi] at (11.100000, -33.700000) {あんず};
\node[Meaning] at (11.150000, -32.050000) {apricot};
\node[Square] at (13.200000, -33.800000) {};
\node[Kanji] at (13.200000, -33.300000) {宰};
\node[Onyomi] at (13.250000, -33.700000) {サイ};
\node[Meaning] at (13.200000, -32.050000) {manager};
\node[Square] at (15.250000, -33.800000) {};
\node[Kanji] at (15.250000, -33.300000) {呉};
\node[Onyomi] at (15.300000, -33.700000) {ゴ};
\node[Kunyomi] at (15.200000, -33.700000) {くれ};
\node[Meaning] at (15.250000, -32.050000) {give};
\node[Square] at (17.300000, -33.800000) {};
\node[Kanji] at (17.300000, -33.300000) {摂};
\node[Onyomi] at (17.350000, -33.700000) {セツ};
\node[Kunyomi] at (17.250000, -33.700000) {おさ};
\node[Meaning] at (17.300000, -32.050000) {in addition};
\node[Square] at (19.350000, -33.800000) {};
\node[Kanji] at (19.350000, -33.300000) {遍};
\node[Onyomi] at (19.400000, -33.700000) {ヘン};
\node[Kunyomi] at (19.300000, -33.700000) {あまね};
\node[Meaning] at (19.350000, -32.050000) {universal};
\node[Square] at (21.400000, -33.800000) {};
\node[Kanji] at (21.400000, -33.300000) {烏};
\node[Onyomi] at (21.450000, -33.700000) {ウオ};
\node[Kunyomi] at (21.350000, -33.700000) {からす};
\node[Meaning] at (21.400000, -32.050000) {crow};
\node[Square] at (23.450000, -33.800000) {};
\node[Kanji] at (23.450000, -33.300000) {乙};
\node[Onyomi] at (23.500000, -33.700000) {オツ};
\node[Kunyomi] at (23.400000, -33.700000) {おと};
\node[Meaning] at (23.450000, -32.050000) {latter};
\node[Square] at (25.500000, -33.800000) {};
\node[Kanji] at (25.500000, -33.300000) {哉};
\node[Onyomi] at (25.550000, -33.700000) {サイ};
\node[Kunyomi] at (25.450000, -33.700000) {や};
\node[Meaning] at (25.500000, -32.050000) {question mark};
\node[Square] at (27.550000, -33.800000) {};
\node[Kanji] at (27.550000, -33.300000) {酌};
\node[Onyomi] at (27.600000, -33.700000) {シャク};
\node[Kunyomi] at (27.500000, -33.700000) {く};
\node[Meaning] at (27.550000, -32.050000) {serve};
\node[Square] at (29.600000, -33.800000) {};
\node[Kanji] at (29.600000, -33.300000) {媒};
\node[Onyomi] at (29.650000, -33.700000) {バイ};
\node[Kunyomi] at (29.550000, -33.700000) {なこうど};
\node[Meaning] at (29.600000, -32.050000) {mediator};
\node[Square] at (31.650000, -33.800000) {};
\node[Kanji] at (31.650000, -33.300000) {篤};
\node[Onyomi] at (31.700000, -33.700000) {トク};
\node[Kunyomi] at (31.600000, -33.700000) {あつ};
\node[Meaning] at (31.650000, -32.050000) {deliberate};
\node[Square] at (33.700000, -33.800000) {};
\node[Kanji] at (33.700000, -33.300000) {峠};
\node[Kunyomi] at (33.650000, -33.700000) {とうげ};
\node[Meaning] at (33.700000, -32.050000) {ridge};
\node[Square] at (35.750000, -33.800000) {};
\node[Kanji] at (35.750000, -33.300000) {擬};
\node[Onyomi] at (35.800000, -33.700000) {ギ};
\node[Kunyomi] at (35.700000, -33.700000) {まが};
\node[Meaning] at (35.750000, -32.050000) {imitate};
\node[Square] at (37.800000, -33.800000) {};
\node[Kanji] at (37.800000, -33.300000) {藍};
\node[Onyomi] at (37.850000, -33.700000) {ラン};
\node[Kunyomi] at (37.750000, -33.700000) {あい};
\node[Meaning] at (37.800000, -32.050000) {indigo};
\node[Square] at (39.850000, -33.800000) {};
\node[Kanji] at (39.850000, -33.300000) {彙};
\node[Onyomi] at (39.900000, -33.700000) {イ};
\node[Meaning] at (39.850000, -32.050000) {same kind};
\node[Square] at (41.900000, -33.800000) {};
\node[Kanji] at (41.900000, -33.300000) {壱};
\node[Onyomi] at (41.950000, -33.700000) {イチ};
\node[Meaning] at (41.900000, -32.050000) {1 (legal)};
\node[Square] at (43.950000, -33.800000) {};
\node[Kanji] at (43.950000, -33.300000) {刹};
\node[Onyomi] at (44.000000, -33.700000) {サツ};
\node[Meaning] at (43.950000, -32.050000) {temple};
\node[Square] at (46.000000, -33.800000) {};
\node[Kanji] at (46.000000, -33.300000) {賜};
\node[Onyomi] at (46.050000, -33.700000) {シ};
\node[Kunyomi] at (45.950000, -33.700000) {たまわ-る};
\node[Meaning] at (46.000000, -32.050000) {grant};
\node[Square] at (48.050000, -33.800000) {};
\node[Kanji] at (48.050000, -33.300000) {箋};
\node[Onyomi] at (48.100000, -33.700000) {セン};
\node[Meaning] at (48.050000, -32.050000) {paper};
\node[Square] at (50.100000, -33.800000) {};
\node[Kanji] at (50.100000, -33.300000) {膳};
\node[Onyomi] at (50.150000, -33.700000) {ゼン};
\node[Meaning] at (50.100000, -32.050000) {tray};
\node[Square] at (52.150000, -33.800000) {};
\node[Kanji] at (52.150000, -33.300000) {遡};
\node[Onyomi] at (52.200000, -33.700000) {ソ};
\node[Kunyomi] at (52.100000, -33.700000) {さかのぼ-る};
\node[Meaning] at (52.150000, -32.050000) {go upstream};
\node[Square] at (54.200000, -33.800000) {};
\node[Kanji] at (54.200000, -33.300000) {緻};
\node[Onyomi] at (54.250000, -33.700000) {チ};
\node[Meaning] at (54.200000, -32.050000) {fine};
\node[Square] at (56.250000, -33.800000) {};
\node[Kanji] at (56.250000, -33.300000) {逓};
\node[Onyomi] at (56.300000, -33.700000) {テイ};
\node[Meaning] at (56.250000, -32.050000) {relay};
\node[Meaning] at (-58.500000, -33.250000) {98.30\%};
\node[Square] at (-56.500000, -35.850000) {};
\node[Kanji] at (-56.500000, -35.350000) {諾};
\node[Onyomi] at (-56.450000, -35.750000) {ダク};
\node[Meaning] at (-56.500000, -34.100000) {agreement};
\node[Square] at (-54.450000, -35.850000) {};
\node[Kanji] at (-54.450000, -35.350000) {併};
\node[Onyomi] at (-54.400000, -35.750000) {ヘイ};
\node[Kunyomi] at (-54.500000, -35.750000) {あわ.せる};
\node[Meaning] at (-54.450000, -34.100000) {join};
\node[Square] at (-52.400000, -35.850000) {};
\node[Kanji] at (-52.400000, -35.350000) {枢};
\node[Onyomi] at (-52.350000, -35.750000) {スウ};
\node[Kunyomi] at (-52.450000, -35.750000) {からくり};
\node[Meaning] at (-52.400000, -34.100000) {hinge};
\node[Square] at (-50.350000, -35.850000) {};
\node[Kanji] at (-50.350000, -35.350000) {娯};
\node[Onyomi] at (-50.300000, -35.750000) {ゴ};
\node[Meaning] at (-50.350000, -34.100000) {recreation};
\node[Square] at (-48.300000, -35.850000) {};
\node[Kanji] at (-48.300000, -35.350000) {塾};
\node[Onyomi] at (-48.250000, -35.750000) {ジュク};
\node[Meaning] at (-48.300000, -34.100000) {cram school};
\node[Square] at (-46.250000, -35.850000) {};
\node[Kanji] at (-46.250000, -35.350000) {俵};
\node[Onyomi] at (-46.200000, -35.750000) {ヒョウ};
\node[Kunyomi] at (-46.300000, -35.750000) {たわら};
\node[Meaning] at (-46.250000, -34.100000) {sack};
\node[Square] at (-44.200000, -35.850000) {};
\node[Kanji] at (-44.200000, -35.350000) {培};
\node[Onyomi] at (-44.150000, -35.750000) {バイ};
\node[Kunyomi] at (-44.250000, -35.750000) {つちか.う};
\node[Meaning] at (-44.200000, -34.100000) {cultivate};
\node[Square] at (-42.150000, -35.850000) {};
\node[Kanji] at (-42.150000, -35.350000) {拓};
\node[Onyomi] at (-42.100000, -35.750000) {タク};
\node[Meaning] at (-42.150000, -34.100000) {cultivation};
\node[Square] at (-40.100000, -35.850000) {};
\node[Kanji] at (-40.100000, -35.350000) {搬};
\node[Onyomi] at (-40.050000, -35.750000) {ハン};
\node[Meaning] at (-40.100000, -34.100000) {transport};
\node[Square] at (-38.050000, -35.850000) {};
\node[Kanji] at (-38.050000, -35.350000) {淀};
\node[Kunyomi] at (-38.100000, -35.750000) {よど};
\node[Meaning] at (-38.050000, -34.100000) {eddy};
\node[Square] at (-36.000000, -35.850000) {};
\node[Kanji] at (-36.000000, -35.350000) {伯};
\node[Onyomi] at (-35.950000, -35.750000) {ハク};
\node[Meaning] at (-36.000000, -34.100000) {chief};
\node[Square] at (-33.950000, -35.850000) {};
\node[Kanji] at (-33.950000, -35.350000) {蓮};
\node[Onyomi] at (-33.900000, -35.750000) {レン};
\node[Kunyomi] at (-34.000000, -35.750000) {はす};
\node[Meaning] at (-33.950000, -34.100000) {lotus};
\node[Square] at (-31.900000, -35.850000) {};
\node[Kanji] at (-31.900000, -35.350000) {隼};
\node[Onyomi] at (-31.850000, -35.750000) {シュン};
\node[Kunyomi] at (-31.950000, -35.750000) {はやぶさ};
\node[Meaning] at (-31.900000, -34.100000) {falcon};
\node[Square] at (-29.850000, -35.850000) {};
\node[Kanji] at (-29.850000, -35.350000) {曙};
\node[Onyomi] at (-29.800000, -35.750000) {ショ};
\node[Kunyomi] at (-29.900000, -35.750000) {あけぼの};
\node[Meaning] at (-29.850000, -34.100000) {dawn};
\node[Square] at (-27.800000, -35.850000) {};
\node[Kanji] at (-27.800000, -35.350000) {諒};
\node[Onyomi] at (-27.750000, -35.750000) {リョウ};
\node[Kunyomi] at (-27.850000, -35.750000) {あきら.か};
\node[Meaning] at (-27.800000, -34.100000) {comprehend};
\node[Square] at (-25.750000, -35.850000) {};
\node[Kanji] at (-25.750000, -35.350000) {莉};
\node[Onyomi] at (-25.700000, -35.750000) {リ};
\node[Meaning] at (-25.750000, -34.100000) {jasmine};
\node[Square] at (-23.700000, -35.850000) {};
\node[Kanji] at (-23.700000, -35.350000) {栞};
\node[Onyomi] at (-23.650000, -35.750000) {カン};
\node[Kunyomi] at (-23.750000, -35.750000) {しおり};
\node[Meaning] at (-23.700000, -34.100000) {bookmark};
\node[Square] at (-21.650000, -35.850000) {};
\node[Kanji] at (-21.650000, -35.350000) {弊};
\node[Onyomi] at (-21.600000, -35.750000) {ヘイ};
\node[Meaning] at (-21.650000, -34.100000) {evil};
\node[Square] at (-19.600000, -35.850000) {};
\node[Kanji] at (-19.600000, -35.350000) {嘉};
\node[Onyomi] at (-19.550000, -35.750000) {カ};
\node[Kunyomi] at (-19.650000, -35.750000) {よい};
\node[Meaning] at (-19.600000, -34.100000) {esteem};
\node[Square] at (-17.550000, -35.850000) {};
\node[Kanji] at (-17.550000, -35.350000) {萌};
\node[Onyomi] at (-17.500000, -35.750000) {ホウ};
\node[Kunyomi] at (-17.600000, -35.750000) {きざ};
\node[Meaning] at (-17.550000, -34.100000) {sprout};
\node[Square] at (-15.500000, -35.850000) {};
\node[Kanji] at (-15.500000, -35.350000) {衡};
\node[Onyomi] at (-15.450000, -35.750000) {コウ};
\node[Meaning] at (-15.500000, -34.100000) {equilibrium};
\node[Square] at (-13.450000, -35.850000) {};
\node[Kanji] at (-13.450000, -35.350000) {胎};
\node[Onyomi] at (-13.400000, -35.750000) {タイ};
\node[Meaning] at (-13.450000, -34.100000) {womb};
\node[Square] at (-11.400000, -35.850000) {};
\node[Kanji] at (-11.400000, -35.350000) {坪};
\node[Onyomi] at (-11.350000, -35.750000) {ヘイ};
\node[Kunyomi] at (-11.450000, -35.750000) {つぼ};
\node[Meaning] at (-11.400000, -34.100000) {two mat area};
\node[Square] at (-9.350000, -35.850000) {};
\node[Kanji] at (-9.350000, -35.350000) {硫};
\node[Onyomi] at (-9.300000, -35.750000) {リュウ};
\node[Meaning] at (-9.350000, -34.100000) {sulfur};
\node[Square] at (-7.300000, -35.850000) {};
\node[Kanji] at (-7.300000, -35.350000) {享};
\node[Onyomi] at (-7.250000, -35.750000) {キョウ};
\node[Kunyomi] at (-7.350000, -35.750000) {う};
\node[Meaning] at (-7.300000, -34.100000) {receive};
\node[Square] at (-5.250000, -35.850000) {};
\node[Kanji] at (-5.250000, -35.350000) {禍};
\node[Onyomi] at (-5.200000, -35.750000) {カ};
\node[Kunyomi] at (-5.300000, -35.750000) {わざわい};
\node[Meaning] at (-5.250000, -34.100000) {evil};
\node[Square] at (-3.200000, -35.850000) {};
\node[Kanji] at (-3.200000, -35.350000) {拙};
\node[Onyomi] at (-3.150000, -35.750000) {セツ};
\node[Kunyomi] at (-3.250000, -35.750000) {つたな};
\node[Meaning] at (-3.200000, -34.100000) {clumsy};
\node[Square] at (-1.150000, -35.850000) {};
\node[Kanji] at (-1.150000, -35.350000) {遷};
\node[Onyomi] at (-1.100000, -35.750000) {セン};
\node[Kunyomi] at (-1.200000, -35.750000) {うつ};
\node[Meaning] at (-1.150000, -34.100000) {transition};
\node[Square] at (0.900000, -35.850000) {};
\node[Kanji] at (0.900000, -35.350000) {梓};
\node[Onyomi] at (0.950000, -35.750000) {シ};
\node[Kunyomi] at (0.850000, -35.750000) {あずさ        };
\node[Meaning] at (0.900000, -34.100000) {wood block};
\node[Square] at (2.950000, -35.850000) {};
\node[Kanji] at (2.950000, -35.350000) {酪};
\node[Onyomi] at (3.000000, -35.750000) {ラク};
\node[Meaning] at (2.950000, -34.100000) {dairy};
\node[Square] at (5.000000, -35.850000) {};
\node[Kanji] at (5.000000, -35.350000) {殉};
\node[Onyomi] at (5.050000, -35.750000) {ジュン};
\node[Meaning] at (5.000000, -34.100000) {martyr};
\node[Square] at (7.050000, -35.850000) {};
\node[Kanji] at (7.050000, -35.350000) {某};
\node[Onyomi] at (7.100000, -35.750000) {ボウ};
\node[Kunyomi] at (7.000000, -35.750000) {それがし};
\node[Meaning] at (7.050000, -34.100000) {certain};
\node[Square] at (9.100000, -35.850000) {};
\node[Kanji] at (9.100000, -35.350000) {泌};
\node[Onyomi] at (9.150000, -35.750000) {ヒ};
\node[Meaning] at (9.100000, -34.100000) {secrete};
\node[Square] at (11.150000, -35.850000) {};
\node[Kanji] at (11.150000, -35.350000) {畝};
\node[Kunyomi] at (11.100000, -35.750000) {うね};
\node[Meaning] at (11.150000, -34.100000) {furrow};
\node[Square] at (13.200000, -35.850000) {};
\node[Kanji] at (13.200000, -35.350000) {柿};
\node[Kunyomi] at (13.150000, -35.750000) {かき};
\node[Meaning] at (13.200000, -34.100000) {persimmon};
\node[Square] at (15.250000, -35.850000) {};
\node[Kanji] at (15.250000, -35.350000) {棺};
\node[Onyomi] at (15.300000, -35.750000) {カン};
\node[Meaning] at (15.250000, -34.100000) {coffin};
\node[Square] at (17.300000, -35.850000) {};
\node[Kanji] at (17.300000, -35.350000) {毀};
\node[Onyomi] at (17.350000, -35.750000) {キ};
\node[Meaning] at (17.300000, -34.100000) {destroy};
\node[Square] at (19.350000, -35.850000) {};
\node[Kanji] at (19.350000, -35.350000) {摯};
\node[Onyomi] at (19.400000, -35.750000) {シ};
\node[Meaning] at (19.350000, -34.100000) {seriousness};
\node[Square] at (21.400000, -35.850000) {};
\node[Kanji] at (21.400000, -35.350000) {綻};
\node[Onyomi] at (21.450000, -35.750000) {タン};
\node[Kunyomi] at (21.350000, -35.750000) {ほころ-びる};
\node[Meaning] at (21.400000, -34.100000) {rip};
\node[Square] at (23.450000, -35.850000) {};
\node[Kanji] at (23.450000, -35.350000) {弐};
\node[Onyomi] at (23.500000, -35.750000) {ニ};
\node[Meaning] at (23.450000, -34.100000) {II, second};
\node[Square] at (25.500000, -35.850000) {};
\node[Kanji] at (25.500000, -35.350000) {勃};
\node[Onyomi] at (25.550000, -35.750000) {ボツ};
\node[Meaning] at (25.500000, -34.100000) {rise};
\node[Square] at (27.550000, -35.850000) {};
\node[Kanji] at (27.550000, -35.350000) {楼};
\node[Onyomi] at (27.600000, -35.750000) {ロウ};
\node[Meaning] at (27.550000, -34.100000) {watchtower};
\node[Square] at (29.600000, -35.850000) {};
\node[Kanji] at (29.600000, -35.350000) {閥};
\node[Onyomi] at (29.650000, -35.750000) {バツ};
\node[Meaning] at (29.600000, -34.100000) {clique};
\node[Square] at (31.650000, -35.850000) {};
\node[Kanji] at (31.650000, -35.350000) {訟};
\node[Onyomi] at (31.700000, -35.750000) {ショウ};
\node[Meaning] at (31.650000, -34.100000) {lawsuit};
\node[Square] at (33.700000, -35.850000) {};
\node[Kanji] at (33.700000, -35.350000) {酎};
\node[Onyomi] at (33.750000, -35.750000) {チュウ};
\node[Kunyomi] at (33.650000, -35.750000) {かも.す};
\node[Meaning] at (33.700000, -34.100000) {sake};
\node[Square] at (35.750000, -35.850000) {};
\node[Kanji] at (35.750000, -35.350000) {糾};
\node[Onyomi] at (35.800000, -35.750000) {キュウ};
\node[Meaning] at (35.750000, -34.100000) {twist};
\node[Square] at (37.800000, -35.850000) {};
\node[Kanji] at (37.800000, -35.350000) {朱};
\node[Onyomi] at (37.850000, -35.750000) {シュ};
\node[Kunyomi] at (37.750000, -35.750000) {あけ};
\node[Meaning] at (37.800000, -34.100000) {vermillion};
\node[Square] at (39.850000, -35.850000) {};
\node[Kanji] at (39.850000, -35.350000) {租};
\node[Onyomi] at (39.900000, -35.750000) {ソ};
\node[Meaning] at (39.850000, -34.100000) {tariff};
\node[Square] at (41.900000, -35.850000) {};
\node[Kanji] at (41.900000, -35.350000) {班};
\node[Onyomi] at (41.950000, -35.750000) {ハン};
\node[Meaning] at (41.900000, -34.100000) {squad};
\node[Square] at (43.950000, -35.850000) {};
\node[Kanji] at (43.950000, -35.350000) {尼};
\node[Onyomi] at (44.000000, -35.750000) {ニ};
\node[Kunyomi] at (43.900000, -35.750000) {あま};
\node[Meaning] at (43.950000, -34.100000) {nun};
\node[Square] at (46.000000, -35.850000) {};
\node[Kanji] at (46.000000, -35.350000) {桑};
\node[Onyomi] at (46.050000, -35.750000) {ソウ};
\node[Kunyomi] at (45.950000, -35.750000) {くわ};
\node[Meaning] at (46.000000, -34.100000) {mulberry};
\node[Square] at (48.050000, -35.850000) {};
\node[Kanji] at (48.050000, -35.350000) {鰐};
\node[Kunyomi] at (48.000000, -35.750000) {わに};
\node[Meaning] at (48.050000, -34.100000) {alligator};
\node[Square] at (50.100000, -35.850000) {};
\node[Kanji] at (50.100000, -35.350000) {諮};
\node[Onyomi] at (50.150000, -35.750000) {シ};
\node[Kunyomi] at (50.050000, -35.750000) {はか.る};
\node[Meaning] at (50.100000, -34.100000) {consult};
\node[Square] at (52.150000, -35.850000) {};
\node[Kanji] at (52.150000, -35.350000) {惰};
\node[Onyomi] at (52.200000, -35.750000) {ダ};
\node[Meaning] at (52.150000, -34.100000) {lazy};
\node[Square] at (54.200000, -35.850000) {};
\node[Kanji] at (54.200000, -35.350000) {颯};
\node[Onyomi] at (54.250000, -35.750000) {サツ};
\node[Kunyomi] at (54.150000, -35.750000) {さっ.と};
\node[Meaning] at (54.200000, -34.100000) {quick};
\node[Square] at (56.250000, -35.850000) {};
\node[Kanji] at (56.250000, -35.350000) {帥};
\node[Onyomi] at (56.300000, -35.750000) {スイ};
\node[Meaning] at (56.250000, -34.100000) {commander};
\node[Meaning] at (-58.500000, -35.300000) {98.31\%};
\node[Square] at (-56.500000, -37.900000) {};
\node[Kanji] at (-56.500000, -37.400000) {壌};
\node[Onyomi] at (-56.450000, -37.800000) {ジョウ};
\node[Kunyomi] at (-56.550000, -37.800000) {つち};
\node[Meaning] at (-56.500000, -36.150000) {soil};
\node[Square] at (-54.450000, -37.900000) {};
\node[Kanji] at (-54.450000, -37.400000) {艇};
\node[Onyomi] at (-54.400000, -37.800000) {テイ};
\node[Meaning] at (-54.450000, -36.150000) {rowboat};
\node[Square] at (-52.400000, -37.900000) {};
\node[Kanji] at (-52.400000, -37.400000) {葵};
\node[Onyomi] at (-52.350000, -37.800000) {キ};
\node[Kunyomi] at (-52.450000, -37.800000) {あおい};
\node[Meaning] at (-52.400000, -36.150000) {hollyhock};
\node[Square] at (-50.350000, -37.900000) {};
\node[Kanji] at (-50.350000, -37.400000) {且};
\node[Onyomi] at (-50.300000, -37.800000) {ショ};
\node[Kunyomi] at (-50.400000, -37.800000) {か};
\node[Meaning] at (-50.350000, -36.150000) {also};
\node[Square] at (-48.300000, -37.900000) {};
\node[Kanji] at (-48.300000, -37.400000) {庶};
\node[Onyomi] at (-48.250000, -37.800000) {ショ};
\node[Meaning] at (-48.300000, -36.150000) {all};
\node[Square] at (-46.250000, -37.900000) {};
\node[Kanji] at (-46.250000, -37.400000) {叙};
\node[Onyomi] at (-46.200000, -37.800000) {ジョ};
\node[Kunyomi] at (-46.300000, -37.800000) {つい};
\node[Meaning] at (-46.250000, -36.150000) {describe};
\node[Square] at (-44.200000, -37.900000) {};
\node[Kanji] at (-44.200000, -37.400000) {朴};
\node[Onyomi] at (-44.150000, -37.800000) {ボク};
\node[Kunyomi] at (-44.250000, -37.800000) {えのき};
\node[Meaning] at (-44.200000, -36.150000) {simple};
\node[Square] at (-42.150000, -37.900000) {};
\node[Kanji] at (-42.150000, -37.400000) {瑛};
\node[Onyomi] at (-42.100000, -37.800000) {エイ};
\node[Meaning] at (-42.150000, -36.150000) {crystal};
\node[Square] at (-40.100000, -37.900000) {};
\node[Kanji] at (-40.100000, -37.400000) {舶};
\node[Onyomi] at (-40.050000, -37.800000) {ハク};
\node[Meaning] at (-40.100000, -36.150000) {ship};
\node[Square] at (-38.050000, -37.900000) {};
\node[Kanji] at (-38.050000, -37.400000) {尉};
\node[Onyomi] at (-38.000000, -37.800000) {イ};
\node[Meaning] at (-38.050000, -36.150000) {military officer};
\node[Square] at (-36.000000, -37.900000) {};
\node[Kanji] at (-36.000000, -37.400000) {堕};
\node[Onyomi] at (-35.950000, -37.800000) {ダ};
\node[Kunyomi] at (-36.050000, -37.800000) {お};
\node[Meaning] at (-36.000000, -36.150000) {degenerate};
\node[Square] at (-33.950000, -37.900000) {};
\node[Kanji] at (-33.950000, -37.400000) {伐};
\node[Onyomi] at (-33.900000, -37.800000) {バツ};
\node[Kunyomi] at (-34.000000, -37.800000) {う};
\node[Meaning] at (-33.950000, -36.150000) {fell};
\node[Square] at (-31.900000, -37.900000) {};
\node[Kanji] at (-31.900000, -37.400000) {俸};
\node[Onyomi] at (-31.850000, -37.800000) {ホウ};
\node[Meaning] at (-31.900000, -36.150000) {salary};
\node[Square] at (-29.850000, -37.900000) {};
\node[Kanji] at (-29.850000, -37.400000) {款};
\node[Onyomi] at (-29.800000, -37.800000) {カン};
\node[Meaning] at (-29.850000, -36.150000) {article};
\node[Square] at (-27.800000, -37.900000) {};
\node[Kanji] at (-27.800000, -37.400000) {瑠};
\node[Onyomi] at (-27.750000, -37.800000) {ル};
\node[Meaning] at (-27.800000, -36.150000) {lapis lazuli};
\node[Square] at (-25.750000, -37.900000) {};
\node[Kanji] at (-25.750000, -37.400000) {瑞};
\node[Onyomi] at (-25.700000, -37.800000) {スイ};
\node[Kunyomi] at (-25.800000, -37.800000) {みず};
\node[Meaning] at (-25.750000, -36.150000) {congratulations};
\node[Square] at (-23.700000, -37.900000) {};
\node[Kanji] at (-23.700000, -37.400000) {扶};
\node[Onyomi] at (-23.650000, -37.800000) {フ};
\node[Kunyomi] at (-23.750000, -37.800000) {たす};
\node[Meaning] at (-23.700000, -36.150000) {aid};
\node[Square] at (-21.650000, -37.900000) {};
\node[Kanji] at (-21.650000, -37.400000) {迭};
\node[Onyomi] at (-21.600000, -37.800000) {テツ};
\node[Meaning] at (-21.650000, -36.150000) {alternate};
\node[Square] at (-19.600000, -37.900000) {};
\node[Kanji] at (-19.600000, -37.400000) {嘱};
\node[Onyomi] at (-19.550000, -37.800000) {ショク};
\node[Kunyomi] at (-19.650000, -37.800000) {しょく.する};
\node[Meaning] at (-19.600000, -36.150000) {request};
\node[Square] at (-17.550000, -37.900000) {};
\node[Kanji] at (-17.550000, -37.400000) {璃};
\node[Onyomi] at (-17.500000, -37.800000) {リ};
\node[Meaning] at (-17.550000, -36.150000) {glassy};
\node[Square] at (-15.500000, -37.900000) {};
\node[Kanji] at (-15.500000, -37.400000) {叔};
\node[Onyomi] at (-15.450000, -37.800000) {シュク};
\node[Meaning] at (-15.500000, -36.150000) {uncle};
\node[Square] at (-13.450000, -37.900000) {};
\node[Kanji] at (-13.450000, -37.400000) {曹};
\node[Onyomi] at (-13.400000, -37.800000) {ソウ};
\node[Kunyomi] at (-13.500000, -37.800000) {つかさ};
\node[Meaning] at (-13.450000, -36.150000) {official};
\node[Square] at (-11.400000, -37.900000) {};
\node[Kanji] at (-11.400000, -37.400000) {睦};
\node[Onyomi] at (-11.350000, -37.800000) {ボク};
\node[Kunyomi] at (-11.450000, -37.800000) {むつ};
\node[Meaning] at (-11.400000, -36.150000) {friendly};
\node[Square] at (-9.350000, -37.900000) {};
\node[Kanji] at (-9.350000, -37.400000) {詠};
\node[Onyomi] at (-9.300000, -37.800000) {エイ};
\node[Kunyomi] at (-9.400000, -37.800000) {よ};
\node[Meaning] at (-9.350000, -36.150000) {compose};
\node[Square] at (-7.300000, -37.900000) {};
\node[Kanji] at (-7.300000, -37.400000) {漸};
\node[Onyomi] at (-7.250000, -37.800000) {ゼン};
\node[Kunyomi] at (-7.350000, -37.800000) {ようや};
\node[Meaning] at (-7.300000, -36.150000) {gradually};
\node[Square] at (-5.250000, -37.900000) {};
\node[Kanji] at (-5.250000, -37.400000) {畔};
\node[Onyomi] at (-5.200000, -37.800000) {ハン};
\node[Kunyomi] at (-5.300000, -37.800000) {あぜ};
\node[Meaning] at (-5.250000, -36.150000) {shore};
\node[Square] at (-3.200000, -37.900000) {};
\node[Kanji] at (-3.200000, -37.400000) {寡};
\node[Onyomi] at (-3.150000, -37.800000) {カ};
\node[Meaning] at (-3.200000, -36.150000) {widow};
\node[Square] at (-1.150000, -37.900000) {};
\node[Kanji] at (-1.150000, -37.400000) {劾};
\node[Onyomi] at (-1.100000, -37.800000) {ガイ};
\node[Meaning] at (-1.150000, -36.150000) {censure};
\node[Square] at (0.900000, -37.900000) {};
\node[Kanji] at (0.900000, -37.400000) {罷};
\node[Onyomi] at (0.950000, -37.800000) {ヒ};
\node[Kunyomi] at (0.850000, -37.800000) {や};
\node[Meaning] at (0.900000, -36.150000) {quit};
\node[Square] at (2.950000, -37.900000) {};
\node[Kanji] at (2.950000, -37.400000) {倹};
\node[Onyomi] at (3.000000, -37.800000) {ケン};
\node[Kunyomi] at (2.900000, -37.800000) {つづまやか};
\node[Meaning] at (2.950000, -36.150000) {thrifty};
\node[Square] at (5.000000, -37.900000) {};
\node[Kanji] at (5.000000, -37.400000) {蚕};
\node[Onyomi] at (5.050000, -37.800000) {サン};
\node[Kunyomi] at (4.950000, -37.800000) {かいこ};
\node[Meaning] at (5.000000, -36.150000) {silkworm};
\node[Square] at (7.050000, -37.900000) {};
\node[Kanji] at (7.050000, -37.400000) {淫};
\node[Onyomi] at (7.100000, -37.800000) {イン};
\node[Kunyomi] at (7.000000, -37.800000) {みだ-ら};
\node[Meaning] at (7.050000, -36.150000) {lewdness};
\node[Square] at (9.100000, -37.900000) {};
\node[Kanji] at (9.100000, -37.400000) {謁};
\node[Onyomi] at (9.150000, -37.800000) {エツ};
\node[Meaning] at (9.100000, -36.150000) {audience};
\node[Square] at (11.150000, -37.900000) {};
\node[Kanji] at (11.150000, -37.400000) {虞};
\node[Kunyomi] at (11.100000, -37.800000) {おそれ};
\node[Meaning] at (11.150000, -36.150000) {uneasiness};
\node[Square] at (13.200000, -37.900000) {};
\node[Kanji] at (13.200000, -37.400000) {楷};
\node[Onyomi] at (13.250000, -37.800000) {カイ};
\node[Meaning] at (13.200000, -36.150000) {printed style};
\node[Square] at (15.250000, -37.900000) {};
\node[Kanji] at (15.250000, -37.400000) {諧};
\node[Onyomi] at (15.300000, -37.800000) {カイ};
\node[Meaning] at (15.250000, -36.150000) {harmony};
\node[Square] at (17.300000, -37.900000) {};
\node[Kanji] at (17.300000, -37.400000) {玩};
\node[Onyomi] at (17.350000, -37.800000) {ガン};
\node[Meaning] at (17.300000, -36.150000) {trifle with};
\node[Square] at (19.350000, -37.900000) {};
\node[Kanji] at (19.350000, -37.400000) {斤};
\node[Onyomi] at (19.400000, -37.800000) {キン};
\node[Meaning] at (19.350000, -36.150000) {axe};
\node[Square] at (21.400000, -37.900000) {};
\node[Kanji] at (21.400000, -37.400000) {憬};
\node[Onyomi] at (21.450000, -37.800000) {ケイ};
\node[Meaning] at (21.400000, -36.150000) {long for};
\node[Square] at (23.450000, -37.900000) {};
\node[Kanji] at (23.450000, -37.400000) {舷};
\node[Onyomi] at (23.500000, -37.800000) {ゲン};
\node[Meaning] at (23.450000, -36.150000) {gunwale};
\node[Square] at (25.500000, -37.900000) {};
\node[Kanji] at (25.500000, -37.400000) {錮};
\node[Onyomi] at (25.550000, -37.800000) {コ};
\node[Meaning] at (25.500000, -36.150000) {tie up};
\node[Square] at (27.550000, -37.900000) {};
\node[Kanji] at (27.550000, -37.400000) {侯};
\node[Onyomi] at (27.600000, -37.800000) {コウ};
\node[Meaning] at (27.550000, -36.150000) {marquis};
\node[Square] at (29.600000, -37.900000) {};
\node[Kanji] at (29.600000, -37.400000) {梗};
\node[Onyomi] at (29.650000, -37.800000) {コウ};
\node[Meaning] at (29.600000, -36.150000) {close up};
\node[Square] at (31.650000, -37.900000) {};
\node[Kanji] at (31.650000, -37.400000) {墾};
\node[Onyomi] at (31.700000, -37.800000) {コン};
\node[Meaning] at (31.650000, -36.150000) {break ground};
\node[Square] at (33.700000, -37.900000) {};
\node[Kanji] at (33.700000, -37.400000) {恣};
\node[Onyomi] at (33.750000, -37.800000) {シ};
\node[Meaning] at (33.700000, -36.150000) {selfish};
\node[Square] at (35.750000, -37.900000) {};
\node[Kanji] at (35.750000, -37.400000) {璽};
\node[Onyomi] at (35.800000, -37.800000) {ジ};
\node[Meaning] at (35.750000, -36.150000) {emperor's seal};
\node[Square] at (37.800000, -37.900000) {};
\node[Kanji] at (37.800000, -37.400000) {儒};
\node[Onyomi] at (37.850000, -37.800000) {ジュ};
\node[Meaning] at (37.800000, -36.150000) {Confucian};
\node[Square] at (39.850000, -37.900000) {};
\node[Kanji] at (39.850000, -37.400000) {羞};
\node[Onyomi] at (39.900000, -37.800000) {シュウ};
\node[Meaning] at (39.850000, -36.150000) {feel ashamed};
\node[Square] at (41.900000, -37.900000) {};
\node[Kanji] at (41.900000, -37.400000) {抄};
\node[Onyomi] at (41.950000, -37.800000) {ショウ};
\node[Meaning] at (41.900000, -36.150000) {extract};
\node[Square] at (43.950000, -37.900000) {};
\node[Kanji] at (43.950000, -37.400000) {硝};
\node[Onyomi] at (44.000000, -37.800000) {ショウ};
\node[Meaning] at (43.950000, -36.150000) {nitrate};
\node[Square] at (46.000000, -37.900000) {};
\node[Kanji] at (46.000000, -37.400000) {詔};
\node[Onyomi] at (46.050000, -37.800000) {ショウ};
\node[Kunyomi] at (45.950000, -37.800000) {みことのり};
\node[Meaning] at (46.000000, -36.150000) {imperial edict};
\node[Square] at (48.050000, -37.900000) {};
\node[Kanji] at (48.050000, -37.400000) {斥};
\node[Onyomi] at (48.100000, -37.800000) {セキ};
\node[Meaning] at (48.050000, -36.150000) {reject};
\node[Square] at (50.100000, -37.900000) {};
\node[Kanji] at (50.100000, -37.400000) {塑};
\node[Onyomi] at (50.150000, -37.800000) {ソ};
\node[Meaning] at (50.100000, -36.150000) {model};
\node[Square] at (52.150000, -37.900000) {};
\node[Kanji] at (52.150000, -37.400000) {捉};
\node[Onyomi] at (52.200000, -37.800000) {ソク};
\node[Kunyomi] at (52.100000, -37.800000) {とら-える};
\node[Meaning] at (52.150000, -36.150000) {capture};
\node[Square] at (54.200000, -37.900000) {};
\node[Kanji] at (54.200000, -37.400000) {但};
\node[Kunyomi] at (54.150000, -37.800000) {ただ-し};
\node[Meaning] at (54.200000, -36.150000) {however};
\node[Square] at (56.250000, -37.900000) {};
\node[Kanji] at (56.250000, -37.400000) {衷};
\node[Onyomi] at (56.300000, -37.800000) {チュウ};
\node[Meaning] at (56.250000, -36.150000) {inmost};
\node[Meaning] at (-58.500000, -37.350000) {98.31\%};
\node[Square] at (-56.500000, -39.950000) {};
\node[Kanji] at (-56.500000, -39.450000) {勅};
\node[Onyomi] at (-56.450000, -39.850000) {チョク};
\node[Meaning] at (-56.500000, -38.200000) {imperial order};
\node[Square] at (-54.450000, -39.950000) {};
\node[Kanji] at (-54.450000, -39.450000) {捗};
\node[Onyomi] at (-54.400000, -39.850000) {チョク};
\node[Meaning] at (-54.450000, -38.200000) {make progress};
\node[Square] at (-52.400000, -39.950000) {};
\node[Kanji] at (-52.400000, -39.450000) {朕};
\node[Onyomi] at (-52.350000, -39.850000) {チン};
\node[Meaning] at (-52.400000, -38.200000) {majestic plural};
\node[Square] at (-50.350000, -39.950000) {};
\node[Kanji] at (-50.350000, -39.450000) {填};
\node[Onyomi] at (-50.300000, -39.850000) {テン};
\node[Meaning] at (-50.350000, -38.200000) {fill in};
\node[Square] at (-48.300000, -39.950000) {};
\node[Kanji] at (-48.300000, -39.450000) {謄};
\node[Onyomi] at (-48.250000, -39.850000) {トウ};
\node[Meaning] at (-48.300000, -38.200000) {mimeograph};
\node[Square] at (-46.250000, -39.950000) {};
\node[Kanji] at (-46.250000, -39.450000) {氾};
\node[Onyomi] at (-46.200000, -39.850000) {ハン};
\node[Meaning] at (-46.250000, -38.200000) {spread out};
\node[Square] at (-44.200000, -39.950000) {};
\node[Kanji] at (-44.200000, -39.450000) {汎};
\node[Onyomi] at (-44.150000, -39.850000) {ハン};
\node[Meaning] at (-44.200000, -38.200000) {pan-};
\node[Square] at (-42.150000, -39.950000) {};
\node[Kanji] at (-42.150000, -39.450000) {頒};
\node[Onyomi] at (-42.100000, -39.850000) {ハン};
\node[Meaning] at (-42.150000, -38.200000) {partition};
\node[Square] at (-40.100000, -39.950000) {};
\node[Kanji] at (-40.100000, -39.450000) {賦};
\node[Onyomi] at (-40.050000, -39.850000) {フ};
\node[Meaning] at (-40.100000, -38.200000) {levy};
\node[Square] at (-38.050000, -39.950000) {};
\node[Kanji] at (-38.050000, -39.450000) {丙};
\node[Onyomi] at (-38.000000, -39.850000) {ヘイ};
\node[Meaning] at (-38.050000, -38.200000) {third class};
\node[Square] at (-36.000000, -39.950000) {};
\node[Kanji] at (-36.000000, -39.450000) {冶};
\node[Onyomi] at (-35.950000, -39.850000) {ヤ};
\node[Meaning] at (-36.000000, -38.200000) {melting};
\node[Square] at (-33.950000, -39.950000) {};
\node[Kanji] at (-33.950000, -39.450000) {喩};
\node[Onyomi] at (-33.900000, -39.850000) {ユ};
\node[Meaning] at (-33.950000, -38.200000) {metaphor};
\node[Square] at (-31.900000, -39.950000) {};
\node[Kanji] at (-31.900000, -39.450000) {瘍};
\node[Onyomi] at (-31.850000, -39.850000) {ヨウ};
\node[Meaning] at (-31.900000, -38.200000) {boil (medical)};
\node[Square] at (-29.850000, -39.950000) {};
\node[Kanji] at (-29.850000, -39.450000) {窯};
\node[Onyomi] at (-29.800000, -39.850000) {ヨウ};
\node[Kunyomi] at (-29.900000, -39.850000) {かま};
\node[Meaning] at (-29.850000, -38.200000) {kiln};
\node[Square] at (-27.800000, -39.950000) {};
\node[Kanji] at (-27.800000, -39.450000) {沃};
\node[Onyomi] at (-27.750000, -39.850000) {ヨク};
\node[Meaning] at (-27.800000, -38.200000) {fertility};
\node[Square] at (-25.750000, -39.950000) {};
\node[Kanji] at (-25.750000, -39.450000) {厘};
\node[Onyomi] at (-25.700000, -39.850000) {リン};
\node[Meaning] at (-25.750000, -38.200000) {thousandth};
\node[Meaning] at (-58.500000, -39.400000) {98.31\%};
\node[Meaning] at (-58.500000, 41.150000) {1 - 56};
\node[Meaning] at (-58.500000, 39.100000) {57 - 112};
\node[Meaning] at (-58.500000, 37.050000) {113 - 168};
\node[Meaning] at (-58.500000, 35.000000) {169 - 224};
\node[Meaning] at (-58.500000, 32.950000) {225 - 280};
\node[Meaning] at (-58.500000, 30.900000) {281 - 336};
\node[Meaning] at (-58.500000, 28.850000) {337 - 392};
\node[Meaning] at (-58.500000, 26.800000) {393 - 448};
\node[Meaning] at (-58.500000, 24.750000) {449 - 504};
\node[Meaning] at (-58.500000, 22.700000) {505 - 560};
\node[Meaning] at (-58.500000, 20.650000) {561 - 616};
\node[Meaning] at (-58.500000, 18.600000) {617 - 672};
\node[Meaning] at (-58.500000, 16.550000) {673 - 728};
\node[Meaning] at (-58.500000, 14.500000) {729 - 784};
\node[Meaning] at (-58.500000, 12.450000) {785 - 840};
\node[Meaning] at (-58.500000, 10.400000) {841 - 896};
\node[Meaning] at (-58.500000, 8.350000) {897 - 952};
\node[Meaning] at (-58.500000, 6.300000) {953 - 1008};
\node[Meaning] at (-58.500000, 4.250000) {1009 - 1064};
\node[Meaning] at (-58.500000, 2.200000) {1065 - 1120};
\node[Meaning] at (-58.500000, 0.150000) {1121 - 1176};
\node[Meaning] at (-58.500000, -1.900000) {1177 - 1232};
\node[Meaning] at (-58.500000, -3.950000) {1233 - 1288};
\node[Meaning] at (-58.500000, -6.000000) {1289 - 1344};
\node[Meaning] at (-58.500000, -8.050000) {1345 - 1400};
\node[Meaning] at (-58.500000, -10.100000) {1401 - 1456};
\node[Meaning] at (-58.500000, -12.150000) {1457 - 1512};
\node[Meaning] at (-58.500000, -14.200000) {1513 - 1568};
\node[Meaning] at (-58.500000, -16.250000) {1569 - 1624};
\node[Meaning] at (-58.500000, -18.300000) {1625 - 1680};
\node[Meaning] at (-58.500000, -20.350000) {1681 - 1736};
\node[Meaning] at (-58.500000, -22.400000) {1737 - 1792};
\node[Meaning] at (-58.500000, -24.450000) {1793 - 1848};
\node[Meaning] at (-58.500000, -26.500000) {1849 - 1904};
\node[Meaning] at (-58.500000, -28.550000) {1905 - 1960};
\node[Meaning] at (-58.500000, -30.600000) {1961 - 2016};
\node[Meaning] at (-58.500000, -32.650000) {2017 - 2072};
\node[Meaning] at (-58.500000, -34.700000) {2073 - 2128};
\node[Meaning] at (-58.500000, -36.750000) {2129 - 2184};
\node[Meaning] at (-58.500000, -38.800000) {2185 - 2240};

\end{CJK}

\node [above right,outer sep=10pt,minimum width=\paperwidth,align=center] at (bottomleft) {
2200 kanji covering 98.52\% of common Japanese text.
Kanji data from \url{https://www.wanikani.com} and \url{https://en.wikipedia.org/wiki/List_of_joyo_kanji}.

};

\end{document}
