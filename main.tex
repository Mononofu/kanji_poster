\documentclass[12pt, a0paper, landscape]{tikzposter}
\usepackage[utf8]{inputenc}
\usepackage{CJKutf8}
\usepackage{anyfontsize}
\usepackage{hyperref}

\urlstyle{same}

\usetheme{Simple}

\newcommand{\Size}{2.05cm}
\tikzset{Square/.style={
    inner sep=0pt,
    text width=\Size,
    text height=\Size,
    minimum size=\Size,
    draw=lightgray,
    ultra thin,
    align=center,
    }
}

\tikzset{Kanji/.style={
    inner sep=0pt,
    text width=\Size,
    text height=\Size,
    minimum size=\Size,
    font=\fontsize{36}{43},
    align=center,
    }
}

\tikzset{Onyomi/.style={
    inner sep=0pt,
    text width=\Size,
    text height=\Size,
    minimum size=\Size,
    font=\fontsize{8}{9},
    align=left,
    }
}

\tikzset{Kunyomi/.style={
    inner sep=0pt,
    text width=\Size,
    text height=\Size,
    minimum size=\Size,
    font=\fontsize{8}{9},
    align=right,
    }
}

\tikzset{Meaning/.style={
    inner sep=0pt,
    text width=\Size,
    text height=\Size,
    minimum size=\Size,
    font=\scriptsize,
    align=center,
    text=gray,
    }
}

\begin{document}


\begin{CJK}{UTF8}{min}
\node[Square] at (-56.500000, 40.000000) {};
\node[Kanji] at (-56.500000, 40.500000) {\textcolor[HTML]{145cd5}{々}};
\node[Onyomi] at (-56.450000, 40.100000) {ノマ};
\node[Kunyomi] at (-56.550000, 40.100000) {のま};
\node[Meaning] at (-56.500000, 41.750000) {repeater};
\node[Square] at (-54.450000, 40.000000) {};
\node[Kanji] at (-54.450000, 40.500000) {\textcolor[HTML]{2570ef}{一}};
\node[Onyomi] at (-54.400000, 40.100000) {イチ};
\node[Kunyomi] at (-54.500000, 40.100000) {ひと.*};
\node[Meaning] at (-54.450000, 41.750000) {one};
\node[Square] at (-52.400000, 40.000000) {};
\node[Kanji] at (-52.400000, 40.500000) {\textcolor[HTML]{1968ed}{二}};
\node[Onyomi] at (-52.350000, 40.100000) {ニ};
\node[Kunyomi] at (-52.450000, 40.100000) {ふた.*};
\node[Meaning] at (-52.400000, 41.750000) {two};
\node[Square] at (-50.350000, 40.000000) {};
\node[Kanji] at (-50.350000, 40.500000) {\textcolor[HTML]{1461e3}{三}};
\node[Onyomi] at (-50.300000, 40.100000) {サン};
\node[Kunyomi] at (-50.400000, 40.100000) {み.*};
\node[Meaning] at (-50.350000, 41.750000) {three};
\node[Square] at (-48.300000, 40.000000) {};
\node[Kanji] at (-48.300000, 40.500000) {\textcolor[HTML]{1557c6}{四}};
\node[Onyomi] at (-48.250000, 40.100000) {シ};
\node[Kunyomi] at (-48.350000, 40.100000) {よん};
\node[Meaning] at (-48.300000, 41.750000) {four};
\node[Square] at (-46.250000, 40.000000) {};
\node[Kanji] at (-46.250000, 40.500000) {\textcolor[HTML]{1551b8}{五}};
\node[Onyomi] at (-46.200000, 40.100000) {ゴ};
\node[Kunyomi] at (-46.300000, 40.100000) {いつ.つ};
\node[Meaning] at (-46.250000, 41.750000) {five};
\node[Square] at (-44.200000, 40.000000) {};
\node[Kanji] at (-44.200000, 40.500000) {\textcolor[HTML]{154caa}{六}};
\node[Onyomi] at (-44.150000, 40.100000) {ロク};
\node[Kunyomi] at (-44.250000, 40.100000) {む.つ};
\node[Meaning] at (-44.200000, 41.750000) {six};
\node[Square] at (-42.150000, 40.000000) {};
\node[Kanji] at (-42.150000, 40.500000) {\textcolor[HTML]{154caa}{七}};
\node[Onyomi] at (-42.100000, 40.100000) {シチ};
\node[Kunyomi] at (-42.200000, 40.100000) {なな.*};
\node[Meaning] at (-42.150000, 41.750000) {seven};
\node[Square] at (-40.100000, 40.000000) {};
\node[Kanji] at (-40.100000, 40.500000) {\textcolor[HTML]{154caa}{八}};
\node[Onyomi] at (-40.050000, 40.100000) {ハチ};
\node[Kunyomi] at (-40.150000, 40.100000) {や.*};
\node[Meaning] at (-40.100000, 41.750000) {eight};
\node[Square] at (-38.050000, 40.000000) {};
\node[Kanji] at (-38.050000, 40.500000) {\textcolor[HTML]{154caa}{九}};
\node[Onyomi] at (-38.000000, 40.100000) {ク};
\node[Kunyomi] at (-38.100000, 40.100000) {ここの.*};
\node[Meaning] at (-38.050000, 41.750000) {nine};
\node[Square] at (-36.000000, 40.000000) {};
\node[Kanji] at (-36.000000, 40.500000) {\textcolor[HTML]{145cd5}{十}};
\node[Onyomi] at (-35.950000, 40.100000) {ジュウ};
\node[Kunyomi] at (-36.050000, 40.100000) {とお.*};
\node[Meaning] at (-36.000000, 41.750000) {ten};
\node[Square] at (-33.950000, 40.000000) {};
\node[Kanji] at (-33.950000, 40.500000) {\textcolor[HTML]{145cd5}{口}};
\node[Onyomi] at (-33.900000, 40.100000) {コウ};
\node[Kunyomi] at (-34.000000, 40.100000) {くち};
\node[Meaning] at (-33.950000, 41.750000) {mouth};
\node[Square] at (-31.900000, 40.000000) {};
\node[Kanji] at (-31.900000, 40.500000) {\textcolor[HTML]{2570ef}{日}};
\node[Onyomi] at (-31.850000, 40.100000) {ニチ};
\node[Kunyomi] at (-31.950000, 40.100000) {ひ};
\node[Meaning] at (-31.900000, 41.750000) {sun};
\node[Square] at (-29.850000, 40.000000) {};
\node[Kanji] at (-29.850000, 40.500000) {\textcolor[HTML]{1461e3}{月}};
\node[Onyomi] at (-29.800000, 40.100000) {ゲツ};
\node[Kunyomi] at (-29.900000, 40.100000) {つき};
\node[Meaning] at (-29.850000, 41.750000) {moon};
\node[Square] at (-27.800000, 40.000000) {};
\node[Kanji] at (-27.800000, 40.500000) {\textcolor[HTML]{1551b8}{田}};
\node[Onyomi] at (-27.750000, 40.100000) {デン};
\node[Kunyomi] at (-27.850000, 40.100000) {た};
\node[Meaning] at (-27.800000, 41.750000) {rice paddy};
\node[Square] at (-25.750000, 40.000000) {};
\node[Kanji] at (-25.750000, 40.500000) {\textcolor[HTML]{1968ed}{目}};
\node[Onyomi] at (-25.700000, 40.100000) {モク};
\node[Kunyomi] at (-25.800000, 40.100000) {め};
\node[Meaning] at (-25.750000, 41.750000) {eye};
\node[Square] at (-23.700000, 40.000000) {};
\node[Kanji] at (-23.700000, 40.500000) {\textcolor[HTML]{1551b8}{古}};
\node[Onyomi] at (-23.650000, 40.100000) {コ};
\node[Kunyomi] at (-23.750000, 40.100000) {ふる.い};
\node[Meaning] at (-23.700000, 41.750000) {old};
\node[Square] at (-21.650000, 40.000000) {};
\node[Kanji] at (-21.650000, 40.500000) {\textcolor[HTML]{0e254c}{吾}};
\node[Onyomi] at (-21.600000, 40.100000) {ゴ};
\node[Kunyomi] at (-21.700000, 40.100000) {わが};
\node[Meaning] at (-21.650000, 41.750000) {i};
\node[Square] at (-19.600000, 40.000000) {};
\node[Kanji] at (-19.600000, 40.500000) {\textcolor[HTML]{14418e}{冒}};
\node[Onyomi] at (-19.550000, 40.100000) {ボウ};
\node[Kunyomi] at (-19.650000, 40.100000) {おか.す};
\node[Meaning] at (-19.600000, 41.750000) {dare};
\node[Square] at (-17.550000, 40.000000) {};
\node[Kanji] at (-17.550000, 40.500000) {\textcolor[HTML]{1461e3}{明}};
\node[Onyomi] at (-17.500000, 40.100000) {メイ};
\node[Kunyomi] at (-17.600000, 40.100000) {あ};
\node[Meaning] at (-17.550000, 41.750000) {bright};
\node[Square] at (-15.500000, 40.000000) {};
\node[Kanji] at (-15.500000, 40.500000) {\textcolor[HTML]{14469c}{唱}};
\node[Onyomi] at (-15.450000, 40.100000) {ショウ};
\node[Kunyomi] at (-15.550000, 40.100000) {とな.える};
\node[Meaning] at (-15.500000, 41.750000) {chant};
\node[Square] at (-13.450000, 40.000000) {};
\node[Kanji] at (-13.450000, 40.500000) {\textcolor[HTML]{123673}{晶}};
\node[Onyomi] at (-13.400000, 40.100000) {ショウ};
\node[Meaning] at (-13.450000, 41.750000) {crystal};
\node[Square] at (-11.400000, 40.000000) {};
\node[Kanji] at (-11.400000, 40.500000) {\textcolor[HTML]{1551b8}{品}};
\node[Onyomi] at (-11.350000, 40.100000) {ヒン};
\node[Kunyomi] at (-11.450000, 40.100000) {しな};
\node[Meaning] at (-11.400000, 41.750000) {product};
\node[Square] at (-9.350000, 40.000000) {};
\node[Kanji] at (-9.350000, 40.500000) {\textcolor[HTML]{133c80}{呂}};
\node[Onyomi] at (-9.300000, 40.100000) {ロ};
\node[Kunyomi] at (-9.400000, 40.100000) {せぼね};
\node[Meaning] at (-9.350000, 41.750000) {bath};
\node[Square] at (-7.300000, 40.000000) {};
\node[Kanji] at (-7.300000, 40.500000) {\textcolor[HTML]{0e254c}{昌}};
\node[Onyomi] at (-7.250000, 40.100000) {ショウ};
\node[Kunyomi] at (-7.350000, 40.100000) {さかん};
\node[Meaning] at (-7.300000, 41.750000) {prosperous};
\node[Square] at (-5.250000, 40.000000) {};
\node[Kanji] at (-5.250000, 40.500000) {\textcolor[HTML]{1557c6}{早}};
\node[Onyomi] at (-5.200000, 40.100000) {ソウ};
\node[Kunyomi] at (-5.300000, 40.100000) {はや.い};
\node[Meaning] at (-5.250000, 41.750000) {early};
\node[Square] at (-3.200000, 40.000000) {};
\node[Kanji] at (-3.200000, 40.500000) {\textcolor[HTML]{1557c6}{世}};
\node[Onyomi] at (-3.150000, 40.100000) {セ};
\node[Kunyomi] at (-3.250000, 40.100000) {よ};
\node[Meaning] at (-3.200000, 41.750000) {generation};
\node[Square] at (-1.150000, 40.000000) {};
\node[Kanji] at (-1.150000, 40.500000) {\textcolor[HTML]{14418e}{胃}};
\node[Onyomi] at (-1.100000, 40.100000) {イ};
\node[Meaning] at (-1.150000, 41.750000) {stomach};
\node[Square] at (0.900000, 40.000000) {};
\node[Kanji] at (0.900000, 40.500000) {\textcolor[HTML]{123673}{旦}};
\node[Onyomi] at (0.950000, 40.100000) {タン};
\node[Kunyomi] at (0.850000, 40.100000) {あきら};
\node[Meaning] at (0.900000, 41.750000) {dawn};
\node[Square] at (2.950000, 40.000000) {};
\node[Kanji] at (2.950000, 40.500000) {\textcolor[HTML]{123673}{胆}};
\node[Onyomi] at (3.000000, 40.100000) {タン};
\node[Kunyomi] at (2.900000, 40.100000) {きも};
\node[Meaning] at (2.950000, 41.750000) {guts};
\node[Square] at (5.000000, 40.000000) {};
\node[Kanji] at (5.000000, 40.500000) {\textcolor[HTML]{113066}{凹}};
\node[Onyomi] at (5.050000, 40.100000) {オウ};
\node[Kunyomi] at (4.950000, 40.100000) {くぼ.む};
\node[Meaning] at (5.000000, 41.750000) {concave};
\node[Square] at (7.050000, 40.000000) {};
\node[Kanji] at (7.050000, 40.500000) {\textcolor[HTML]{102b59}{凸}};
\node[Onyomi] at (7.100000, 40.100000) {トツ};
\node[Kunyomi] at (7.000000, 40.100000) {でこ};
\node[Meaning] at (7.050000, 41.750000) {convex};
\node[Square] at (9.100000, 40.000000) {};
\node[Kanji] at (9.100000, 40.500000) {\textcolor[HTML]{123673}{旧}};
\node[Onyomi] at (9.150000, 40.100000) {キュウ};
\node[Meaning] at (9.100000, 41.750000) {former};
\node[Square] at (11.150000, 40.000000) {};
\node[Kanji] at (11.150000, 40.500000) {\textcolor[HTML]{1461e3}{自}};
\node[Onyomi] at (11.200000, 40.100000) {ジ};
\node[Meaning] at (11.150000, 41.750000) {self};
\node[Square] at (13.200000, 40.000000) {};
\node[Kanji] at (13.200000, 40.500000) {\textcolor[HTML]{1551b8}{白}};
\node[Onyomi] at (13.250000, 40.100000) {ハク};
\node[Kunyomi] at (13.150000, 40.100000) {しろ};
\node[Meaning] at (13.200000, 41.750000) {white};
\node[Square] at (15.250000, 40.000000) {};
\node[Kanji] at (15.250000, 40.500000) {\textcolor[HTML]{14469c}{百}};
\node[Onyomi] at (15.300000, 40.100000) {ヒャク};
\node[Meaning] at (15.250000, 41.750000) {hundred};
\node[Square] at (17.300000, 40.000000) {};
\node[Kanji] at (17.300000, 40.500000) {\textcolor[HTML]{1968ed}{中}};
\node[Onyomi] at (17.350000, 40.100000) {チュウ};
\node[Kunyomi] at (17.250000, 40.100000) {なか};
\node[Meaning] at (17.300000, 41.750000) {middle};
\node[Square] at (19.350000, 40.000000) {};
\node[Kanji] at (19.350000, 40.500000) {\textcolor[HTML]{14469c}{千}};
\node[Onyomi] at (19.400000, 40.100000) {セン};
\node[Kunyomi] at (19.300000, 40.100000) {ち};
\node[Meaning] at (19.350000, 41.750000) {thousand};
\node[Square] at (21.400000, 40.000000) {};
\node[Kanji] at (21.400000, 40.500000) {\textcolor[HTML]{133c80}{舌}};
\node[Onyomi] at (21.450000, 40.100000) {ゼツ};
\node[Kunyomi] at (21.350000, 40.100000) {した};
\node[Meaning] at (21.400000, 41.750000) {tongue};
\node[Square] at (23.450000, 40.000000) {};
\node[Kanji] at (23.450000, 40.500000) {\textcolor[HTML]{0e254c}{升}};
\node[Onyomi] at (23.500000, 40.100000) {ショウ};
\node[Kunyomi] at (23.400000, 40.100000) {ます};
\node[Meaning] at (23.450000, 41.750000) {grid};
\node[Square] at (25.500000, 40.000000) {};
\node[Kanji] at (25.500000, 40.500000) {\textcolor[HTML]{14418e}{昇}};
\node[Onyomi] at (25.550000, 40.100000) {ショウ};
\node[Kunyomi] at (25.450000, 40.100000) {のぼ.る};
\node[Meaning] at (25.500000, 41.750000) {ascend};
\node[Square] at (27.550000, 40.000000) {};
\node[Kanji] at (27.550000, 40.500000) {\textcolor[HTML]{154caa}{丸}};
\node[Onyomi] at (27.600000, 40.100000) {ガン};
\node[Kunyomi] at (27.500000, 40.100000) {まる};
\node[Meaning] at (27.550000, 41.750000) {circle};
\node[Square] at (29.600000, 40.000000) {};
\node[Kanji] at (29.600000, 40.500000) {\textcolor[HTML]{113066}{寸}};
\node[Onyomi] at (29.650000, 40.100000) {スン};
\node[Meaning] at (29.600000, 41.750000) {measurement};
\node[Square] at (31.650000, 40.000000) {};
\node[Kanji] at (31.650000, 40.500000) {\textcolor[HTML]{14469c}{専}};
\node[Onyomi] at (31.700000, 40.100000) {セン};
\node[Kunyomi] at (31.600000, 40.100000) {もっぱ.ら};
\node[Meaning] at (31.650000, 41.750000) {specialty};
\node[Square] at (33.700000, 40.000000) {};
\node[Kanji] at (33.700000, 40.500000) {\textcolor[HTML]{14418e}{博}};
\node[Onyomi] at (33.750000, 40.100000) {ハク};
\node[Meaning] at (33.700000, 41.750000) {exhibition};
\node[Square] at (35.750000, 40.000000) {};
\node[Kanji] at (35.750000, 40.500000) {\textcolor[HTML]{14418e}{占}};
\node[Onyomi] at (35.800000, 40.100000) {セン};
\node[Kunyomi] at (35.700000, 40.100000) {うらな.い};
\node[Meaning] at (35.750000, 41.750000) {fortune};
\node[Square] at (37.800000, 40.000000) {};
\node[Kanji] at (37.800000, 40.500000) {\textcolor[HTML]{2570ef}{上}};
\node[Onyomi] at (37.850000, 40.100000) {ジョウ};
\node[Kunyomi] at (37.750000, 40.100000) {うえ};
\node[Meaning] at (37.800000, 41.750000) {above};
\node[Square] at (39.850000, 40.000000) {};
\node[Kanji] at (39.850000, 40.500000) {\textcolor[HTML]{1461e3}{下}};
\node[Onyomi] at (39.900000, 40.100000) {カ};
\node[Kunyomi] at (39.800000, 40.100000) {した};
\node[Meaning] at (39.850000, 41.750000) {below};
\node[Square] at (41.900000, 40.000000) {};
\node[Kanji] at (41.900000, 40.500000) {\textcolor[HTML]{113066}{卓}};
\node[Onyomi] at (41.950000, 40.100000) {タク};
\node[Meaning] at (41.900000, 41.750000) {table};
\node[Square] at (43.950000, 40.000000) {};
\node[Kanji] at (43.950000, 40.500000) {\textcolor[HTML]{1557c6}{朝}};
\node[Onyomi] at (44.000000, 40.100000) {チョウ};
\node[Kunyomi] at (43.900000, 40.100000) {あさ};
\node[Meaning] at (43.950000, 41.750000) {morning};
\node[Square] at (46.000000, 40.000000) {};
\node[Kanji] at (46.000000, 40.500000) {\textcolor[HTML]{123673}{貝}};
\node[Kunyomi] at (45.950000, 40.100000) {かい};
\node[Meaning] at (46.000000, 41.750000) {shellfish};
\node[Square] at (48.050000, 40.000000) {};
\node[Kanji] at (48.050000, 40.500000) {\textcolor[HTML]{0e254c}{貞}};
\node[Onyomi] at (48.100000, 40.100000) {テイ};
\node[Kunyomi] at (48.000000, 40.100000) {さだ};
\node[Meaning] at (48.050000, 41.750000) {chastity};
\node[Square] at (50.100000, 40.000000) {};
\node[Kanji] at (50.100000, 40.500000) {\textcolor[HTML]{1557c6}{員}};
\node[Onyomi] at (50.150000, 40.100000) {イン};
\node[Meaning] at (50.100000, 41.750000) {member};
\node[Square] at (52.150000, 40.000000) {};
\node[Kanji] at (52.150000, 40.500000) {\textcolor[HTML]{3178f2}{見}};
\node[Onyomi] at (52.200000, 40.100000) {ケン};
\node[Kunyomi] at (52.100000, 40.100000) {み};
\node[Meaning] at (52.150000, 41.750000) {see};
\node[Square] at (54.200000, 40.000000) {};
\node[Kanji] at (54.200000, 40.500000) {\textcolor[HTML]{14418e}{児}};
\node[Onyomi] at (54.250000, 40.100000) {ジ};
\node[Kunyomi] at (54.150000, 40.100000) {こ};
\node[Meaning] at (54.200000, 41.750000) {child};
\node[Square] at (56.250000, 40.000000) {};
\node[Kanji] at (56.250000, 40.500000) {\textcolor[HTML]{1557c6}{元}};
\node[Onyomi] at (56.300000, 40.100000) {ゲン};
\node[Kunyomi] at (56.200000, 40.100000) {もと};
\node[Meaning] at (56.250000, 41.750000) {origin};
\node[Meaning] at (-58.500000, 40.550000) {10.06\%};
\node[Square] at (-56.500000, 37.950000) {};
\node[Kanji] at (-56.500000, 38.450000) {\textcolor[HTML]{154caa}{頑}};
\node[Onyomi] at (-56.450000, 38.050000) {ガン};
\node[Meaning] at (-56.500000, 39.700000) {stubborn};
\node[Square] at (-54.450000, 37.950000) {};
\node[Kanji] at (-54.450000, 38.450000) {\textcolor[HTML]{102b59}{凡}};
\node[Onyomi] at (-54.400000, 38.050000) {ボン};
\node[Kunyomi] at (-54.500000, 38.050000) {おうよ.そ};
\node[Meaning] at (-54.450000, 39.700000) {mediocre};
\node[Square] at (-52.400000, 37.950000) {};
\node[Kanji] at (-52.400000, 38.450000) {\textcolor[HTML]{154caa}{負}};
\node[Onyomi] at (-52.350000, 38.050000) {フ};
\node[Kunyomi] at (-52.450000, 38.050000) {ま.ける};
\node[Meaning] at (-52.400000, 39.700000) {lose};
\node[Square] at (-50.350000, 37.950000) {};
\node[Kanji] at (-50.350000, 38.450000) {\textcolor[HTML]{1557c6}{万}};
\node[Onyomi] at (-50.300000, 38.050000) {マン};
\node[Meaning] at (-50.350000, 39.700000) {ten thousand};
\node[Square] at (-48.300000, 37.950000) {};
\node[Kanji] at (-48.300000, 38.450000) {\textcolor[HTML]{133c80}{句}};
\node[Onyomi] at (-48.250000, 38.050000) {ク};
\node[Meaning] at (-48.300000, 39.700000) {paragraph};
\node[Square] at (-46.250000, 37.950000) {};
\node[Kanji] at (-46.250000, 38.450000) {\textcolor[HTML]{133c80}{肌}};
\node[Kunyomi] at (-46.300000, 38.050000) {はだ};
\node[Meaning] at (-46.250000, 39.700000) {skin};
\node[Square] at (-44.200000, 37.950000) {};
\node[Kanji] at (-44.200000, 38.450000) {\textcolor[HTML]{133c80}{旬}};
\node[Onyomi] at (-44.150000, 38.050000) {シュン};
\node[Meaning] at (-44.200000, 39.700000) {in season};
\node[Square] at (-42.150000, 37.950000) {};
\node[Kanji] at (-42.150000, 38.450000) {\textcolor[HTML]{1557c6}{的}};
\node[Onyomi] at (-42.100000, 38.050000) {テキ};
\node[Kunyomi] at (-42.200000, 38.050000) {まと};
\node[Meaning] at (-42.150000, 39.700000) {target};
\node[Square] at (-40.100000, 37.950000) {};
\node[Kanji] at (-40.100000, 38.450000) {\textcolor[HTML]{1557c6}{首}};
\node[Onyomi] at (-40.050000, 38.050000) {シュ};
\node[Kunyomi] at (-40.150000, 38.050000) {くび};
\node[Meaning] at (-40.100000, 39.700000) {neck};
\node[Square] at (-38.050000, 37.950000) {};
\node[Kanji] at (-38.050000, 38.450000) {\textcolor[HTML]{0e254c}{乙}};
\node[Onyomi] at (-38.000000, 38.050000) {オツ};
\node[Kunyomi] at (-38.100000, 38.050000) {おと};
\node[Meaning] at (-38.050000, 39.700000) {latter};
\node[Square] at (-36.000000, 37.950000) {};
\node[Kanji] at (-36.000000, 38.450000) {\textcolor[HTML]{14469c}{乱}};
\node[Onyomi] at (-35.950000, 38.050000) {ラン};
\node[Kunyomi] at (-36.050000, 38.050000) {みだ.す};
\node[Meaning] at (-36.000000, 39.700000) {riot};
\node[Square] at (-33.950000, 37.950000) {};
\node[Kanji] at (-33.950000, 38.450000) {\textcolor[HTML]{1551b8}{直}};
\node[Onyomi] at (-33.900000, 38.050000) {チョク};
\node[Kunyomi] at (-34.000000, 38.050000) {なお.す};
\node[Meaning] at (-33.950000, 39.700000) {fix};
\node[Square] at (-31.900000, 37.950000) {};
\node[Kanji] at (-31.900000, 38.450000) {\textcolor[HTML]{14469c}{具}};
\node[Onyomi] at (-31.850000, 38.050000) {グ};
\node[Meaning] at (-31.900000, 39.700000) {tool};
\node[Square] at (-29.850000, 37.950000) {};
\node[Kanji] at (-29.850000, 38.450000) {\textcolor[HTML]{145cd5}{真}};
\node[Onyomi] at (-29.800000, 38.050000) {シン};
\node[Kunyomi] at (-29.900000, 38.050000) {ま};
\node[Meaning] at (-29.850000, 39.700000) {reality};
\node[Square] at (-27.800000, 37.950000) {};
\node[Kanji] at (-27.800000, 38.450000) {\textcolor[HTML]{154caa}{工}};
\node[Onyomi] at (-27.750000, 38.050000) {コウ};
\node[Meaning] at (-27.800000, 39.700000) {industry};
\node[Square] at (-25.750000, 37.950000) {};
\node[Kanji] at (-25.750000, 38.450000) {\textcolor[HTML]{154caa}{左}};
\node[Onyomi] at (-25.700000, 38.050000) {サ};
\node[Kunyomi] at (-25.800000, 38.050000) {ひだり};
\node[Meaning] at (-25.750000, 39.700000) {left};
\node[Square] at (-23.700000, 37.950000) {};
\node[Kanji] at (-23.700000, 38.450000) {\textcolor[HTML]{154caa}{右}};
\node[Onyomi] at (-23.650000, 38.050000) {ウ};
\node[Kunyomi] at (-23.750000, 38.050000) {みぎ};
\node[Meaning] at (-23.700000, 39.700000) {right};
\node[Square] at (-21.650000, 37.950000) {};
\node[Kanji] at (-21.650000, 38.450000) {\textcolor[HTML]{154caa}{有}};
\node[Onyomi] at (-21.600000, 38.050000) {ユウ};
\node[Kunyomi] at (-21.700000, 38.050000) {あ.る};
\node[Meaning] at (-21.650000, 39.700000) {have};
\node[Square] at (-19.600000, 37.950000) {};
\node[Kanji] at (-19.600000, 38.450000) {\textcolor[HTML]{0e254c}{賄}};
\node[Onyomi] at (-19.550000, 38.050000) {ワイ};
\node[Kunyomi] at (-19.650000, 38.050000) {まかな.う};
\node[Meaning] at (-19.600000, 39.700000) {bribe};
\node[Square] at (-17.550000, 37.950000) {};
\node[Kanji] at (-17.550000, 38.450000) {\textcolor[HTML]{102b59}{貢}};
\node[Onyomi] at (-17.500000, 38.050000) {コウ};
\node[Kunyomi] at (-17.600000, 38.050000) {みつ.ぐ};
\node[Meaning] at (-17.550000, 39.700000) {tribute};
\node[Square] at (-15.500000, 37.950000) {};
\node[Kanji] at (-15.500000, 38.450000) {\textcolor[HTML]{113066}{項}};
\node[Onyomi] at (-15.450000, 38.050000) {コウ};
\node[Meaning] at (-15.500000, 39.700000) {paragraph};
\node[Square] at (-13.450000, 37.950000) {};
\node[Kanji] at (-13.450000, 38.450000) {\textcolor[HTML]{123673}{刀}};
\node[Onyomi] at (-13.400000, 38.050000) {トウ};
\node[Kunyomi] at (-13.500000, 38.050000) {かたな};
\node[Meaning] at (-13.450000, 39.700000) {sword};
\node[Square] at (-11.400000, 37.950000) {};
\node[Kanji] at (-11.400000, 38.450000) {\textcolor[HTML]{0e254c}{刃}};
\node[Onyomi] at (-11.350000, 38.050000) {ジン};
\node[Kunyomi] at (-11.450000, 38.050000) {は};
\node[Meaning] at (-11.400000, 39.700000) {blade};
\node[Square] at (-9.350000, 37.950000) {};
\node[Kanji] at (-9.350000, 38.450000) {\textcolor[HTML]{145cd5}{切}};
\node[Onyomi] at (-9.300000, 38.050000) {セツ};
\node[Kunyomi] at (-9.400000, 38.050000) {き.る};
\node[Meaning] at (-9.350000, 39.700000) {cut};
\node[Square] at (-7.300000, 37.950000) {};
\node[Kanji] at (-7.300000, 38.450000) {\textcolor[HTML]{133c80}{召}};
\node[Onyomi] at (-7.250000, 38.050000) {ショウ};
\node[Kunyomi] at (-7.350000, 38.050000) {め.す};
\node[Meaning] at (-7.300000, 39.700000) {call};
\node[Square] at (-5.250000, 37.950000) {};
\node[Kanji] at (-5.250000, 38.450000) {\textcolor[HTML]{133c80}{昭}};
\node[Onyomi] at (-5.200000, 38.050000) {ショウ};
\node[Meaning] at (-5.250000, 39.700000) {shining};
\node[Square] at (-3.200000, 37.950000) {};
\node[Kanji] at (-3.200000, 38.450000) {\textcolor[HTML]{154caa}{則}};
\node[Onyomi] at (-3.150000, 38.050000) {ソク};
\node[Kunyomi] at (-3.250000, 38.050000) {のっと.る};
\node[Meaning] at (-3.200000, 39.700000) {rule};
\node[Square] at (-1.150000, 37.950000) {};
\node[Kanji] at (-1.150000, 38.450000) {\textcolor[HTML]{123673}{副}};
\node[Onyomi] at (-1.100000, 38.050000) {フク};
\node[Meaning] at (-1.150000, 39.700000) {vice};
\node[Square] at (0.900000, 37.950000) {};
\node[Kanji] at (0.900000, 38.450000) {\textcolor[HTML]{1557c6}{別}};
\node[Onyomi] at (0.950000, 38.050000) {ベツ};
\node[Kunyomi] at (0.850000, 38.050000) {わか.*};
\node[Meaning] at (0.900000, 39.700000) {separate};
\node[Square] at (2.950000, 37.950000) {};
\node[Kanji] at (2.950000, 38.450000) {\textcolor[HTML]{14418e}{丁}};
\node[Onyomi] at (3.000000, 38.050000) {チョウ};
\node[Meaning] at (2.950000, 39.700000) {street};
\node[Square] at (5.000000, 37.950000) {};
\node[Kanji] at (5.000000, 38.450000) {\textcolor[HTML]{1551b8}{町}};
\node[Onyomi] at (5.050000, 38.050000) {チョウ};
\node[Kunyomi] at (4.950000, 38.050000) {まち};
\node[Meaning] at (5.000000, 39.700000) {town};
\node[Square] at (7.050000, 37.950000) {};
\node[Kanji] at (7.050000, 38.450000) {\textcolor[HTML]{1551b8}{可}};
\node[Onyomi] at (7.100000, 38.050000) {カ};
\node[Meaning] at (7.050000, 39.700000) {possible};
\node[Square] at (9.100000, 37.950000) {};
\node[Kanji] at (9.100000, 38.450000) {\textcolor[HTML]{14469c}{頂}};
\node[Onyomi] at (9.150000, 38.050000) {チョウ};
\node[Kunyomi] at (9.050000, 38.050000) {いただき};
\node[Meaning] at (9.100000, 39.700000) {summit};
\node[Square] at (11.150000, 37.950000) {};
\node[Kanji] at (11.150000, 38.450000) {\textcolor[HTML]{1968ed}{子}};
\node[Onyomi] at (11.200000, 38.050000) {シ};
\node[Kunyomi] at (11.100000, 38.050000) {こ};
\node[Meaning] at (11.150000, 39.700000) {child};
\node[Square] at (13.200000, 37.950000) {};
\node[Kanji] at (13.200000, 38.450000) {\textcolor[HTML]{113066}{孔}};
\node[Onyomi] at (13.250000, 38.050000) {コウ};
\node[Kunyomi] at (13.150000, 38.050000) {あな};
\node[Meaning] at (13.200000, 39.700000) {cavity};
\node[Square] at (15.250000, 37.950000) {};
\node[Kanji] at (15.250000, 38.450000) {\textcolor[HTML]{123673}{了}};
\node[Onyomi] at (15.300000, 38.050000) {リョウ};
\node[Meaning] at (15.250000, 39.700000) {finish};
\node[Square] at (17.300000, 37.950000) {};
\node[Kanji] at (17.300000, 38.450000) {\textcolor[HTML]{145cd5}{女}};
\node[Onyomi] at (17.350000, 38.050000) {ジョ};
\node[Kunyomi] at (17.250000, 38.050000) {おんな};
\node[Meaning] at (17.300000, 39.700000) {woman};
\node[Square] at (19.350000, 37.950000) {};
\node[Kanji] at (19.350000, 38.450000) {\textcolor[HTML]{1557c6}{好}};
\node[Onyomi] at (19.400000, 38.050000) {コウ};
\node[Kunyomi] at (19.300000, 38.050000) {す.き};
\node[Meaning] at (19.350000, 39.700000) {like};
\node[Square] at (21.400000, 37.950000) {};
\node[Kanji] at (21.400000, 38.450000) {\textcolor[HTML]{102b59}{如}};
\node[Onyomi] at (21.450000, 38.050000) {ジョ};
\node[Kunyomi] at (21.350000, 38.050000) {ごと.し};
\node[Meaning] at (21.400000, 39.700000) {likeness};
\node[Square] at (23.450000, 37.950000) {};
\node[Kanji] at (23.450000, 38.450000) {\textcolor[HTML]{1557c6}{母}};
\node[Onyomi] at (23.500000, 38.050000) {ボ};
\node[Kunyomi] at (23.400000, 38.050000) {はは};
\node[Meaning] at (23.450000, 39.700000) {mother};
\node[Square] at (25.500000, 37.950000) {};
\node[Kanji] at (25.500000, 38.450000) {\textcolor[HTML]{123673}{貫}};
\node[Onyomi] at (25.550000, 38.050000) {カン};
\node[Kunyomi] at (25.450000, 38.050000) {つらぬ};
\node[Meaning] at (25.500000, 39.700000) {pierce};
\node[Square] at (27.550000, 37.950000) {};
\node[Kanji] at (27.550000, 38.450000) {\textcolor[HTML]{154caa}{兄}};
\node[Onyomi] at (27.600000, 38.050000) {キョウ};
\node[Kunyomi] at (27.500000, 38.050000) {あに};
\node[Meaning] at (27.550000, 39.700000) {older brother};
\node[Square] at (29.600000, 37.950000) {};
\node[Kanji] at (29.600000, 38.450000) {\textcolor[HTML]{113066}{克}};
\node[Onyomi] at (29.650000, 38.050000) {コク};
\node[Meaning] at (29.600000, 39.700000) {overcome};
\node[Square] at (31.650000, 37.950000) {};
\node[Kanji] at (31.650000, 38.450000) {\textcolor[HTML]{1461e3}{小}};
\node[Onyomi] at (31.700000, 38.050000) {ショウ};
\node[Kunyomi] at (31.600000, 38.050000) {ちい.さい};
\node[Meaning] at (31.650000, 39.700000) {small};
\node[Square] at (33.700000, 37.950000) {};
\node[Kanji] at (33.700000, 38.450000) {\textcolor[HTML]{145cd5}{少}};
\node[Onyomi] at (33.750000, 38.050000) {ショウ};
\node[Kunyomi] at (33.650000, 38.050000) {すこ.し};
\node[Meaning] at (33.700000, 39.700000) {few};
\node[Square] at (35.750000, 37.950000) {};
\node[Kanji] at (35.750000, 38.450000) {\textcolor[HTML]{2570ef}{大}};
\node[Onyomi] at (35.800000, 38.050000) {タイ};
\node[Kunyomi] at (35.700000, 38.050000) {おお};
\node[Meaning] at (35.750000, 39.700000) {big};
\node[Square] at (37.800000, 37.950000) {};
\node[Kanji] at (37.800000, 38.450000) {\textcolor[HTML]{145cd5}{多}};
\node[Onyomi] at (37.850000, 38.050000) {タ};
\node[Kunyomi] at (37.750000, 38.050000) {おお.い};
\node[Meaning] at (37.800000, 39.700000) {many};
\node[Square] at (39.850000, 37.950000) {};
\node[Kanji] at (39.850000, 38.450000) {\textcolor[HTML]{154caa}{夕}};
\node[Onyomi] at (39.900000, 38.050000) {セキ};
\node[Kunyomi] at (39.800000, 38.050000) {ゆう};
\node[Meaning] at (39.850000, 39.700000) {evening};
\node[Square] at (41.900000, 37.950000) {};
\node[Kanji] at (41.900000, 38.450000) {\textcolor[HTML]{145cd5}{外}};
\node[Onyomi] at (41.950000, 38.050000) {ガイ};
\node[Kunyomi] at (41.850000, 38.050000) {そと};
\node[Meaning] at (41.900000, 39.700000) {outside};
\node[Square] at (43.950000, 37.950000) {};
\node[Kanji] at (43.950000, 38.450000) {\textcolor[HTML]{145cd5}{名}};
\node[Onyomi] at (44.000000, 38.050000) {メイ};
\node[Kunyomi] at (43.900000, 38.050000) {な};
\node[Meaning] at (43.950000, 39.700000) {name};
\node[Square] at (46.000000, 37.950000) {};
\node[Kanji] at (46.000000, 38.450000) {\textcolor[HTML]{1557c6}{石}};
\node[Onyomi] at (46.050000, 38.050000) {セキ};
\node[Kunyomi] at (45.950000, 38.050000) {いし};
\node[Meaning] at (46.000000, 39.700000) {stone};
\node[Square] at (48.050000, 37.950000) {};
\node[Kanji] at (48.050000, 38.450000) {\textcolor[HTML]{14469c}{肖}};
\node[Onyomi] at (48.100000, 38.050000) {ショウ};
\node[Kunyomi] at (48.000000, 38.050000) {あやか};
\node[Meaning] at (48.050000, 39.700000) {resemblance};
\node[Square] at (50.100000, 37.950000) {};
\node[Kanji] at (50.100000, 38.450000) {\textcolor[HTML]{0e254c}{硝}};
\node[Onyomi] at (50.150000, 38.050000) {ショウ};
\node[Meaning] at (50.100000, 39.700000) {nitrate};
\node[Square] at (52.150000, 37.950000) {};
\node[Kanji] at (52.150000, 38.450000) {\textcolor[HTML]{133c80}{砕}};
\node[Onyomi] at (52.200000, 38.050000) {サイ};
\node[Kunyomi] at (52.100000, 38.050000) {くだ.*};
\node[Meaning] at (52.150000, 39.700000) {smash};
\node[Square] at (54.200000, 37.950000) {};
\node[Kanji] at (54.200000, 38.450000) {\textcolor[HTML]{14469c}{砂}};
\node[Onyomi] at (54.250000, 38.050000) {サ};
\node[Kunyomi] at (54.150000, 38.050000) {すな};
\node[Meaning] at (54.200000, 39.700000) {sand};
\node[Square] at (56.250000, 37.950000) {};
\node[Kanji] at (56.250000, 38.450000) {\textcolor[HTML]{123673}{削}};
\node[Onyomi] at (56.300000, 38.050000) {サク};
\node[Kunyomi] at (56.200000, 38.050000) {けず.る};
\node[Meaning] at (56.250000, 39.700000) {whittle down};
\node[Meaning] at (-58.500000, 38.500000) {14.87\%};
\node[Square] at (-56.500000, 35.900000) {};
\node[Kanji] at (-56.500000, 36.400000) {\textcolor[HTML]{1557c6}{光}};
\node[Onyomi] at (-56.450000, 36.000000) {コウ};
\node[Kunyomi] at (-56.550000, 36.000000) {ひかり};
\node[Meaning] at (-56.500000, 37.650000) {sunlight};
\node[Square] at (-54.450000, 35.900000) {};
\node[Kanji] at (-54.450000, 36.400000) {\textcolor[HTML]{1551b8}{太}};
\node[Onyomi] at (-54.400000, 36.000000) {タイ};
\node[Kunyomi] at (-54.500000, 36.000000) {ふと.い};
\node[Meaning] at (-54.450000, 37.650000) {fat};
\node[Square] at (-52.400000, 35.900000) {};
\node[Kanji] at (-52.400000, 36.400000) {\textcolor[HTML]{154caa}{器}};
\node[Onyomi] at (-52.350000, 36.000000) {キ};
\node[Kunyomi] at (-52.450000, 36.000000) {うつわ};
\node[Meaning] at (-52.400000, 37.650000) {container};
\node[Square] at (-50.350000, 35.900000) {};
\node[Kanji] at (-50.350000, 36.400000) {\textcolor[HTML]{14469c}{臭}};
\node[Onyomi] at (-50.300000, 36.000000) {シュウ};
\node[Kunyomi] at (-50.400000, 36.000000) {くさ};
\node[Meaning] at (-50.350000, 37.650000) {stinking};
\node[Square] at (-48.300000, 35.900000) {};
\node[Kanji] at (-48.300000, 36.400000) {\textcolor[HTML]{14469c}{妙}};
\node[Onyomi] at (-48.250000, 36.000000) {ミョウ};
\node[Kunyomi] at (-48.350000, 36.000000) {たえ.なる};
\node[Meaning] at (-48.300000, 37.650000) {strange};
\node[Square] at (-46.250000, 35.900000) {};
\node[Kanji] at (-46.250000, 36.400000) {\textcolor[HTML]{1557c6}{省}};
\node[Onyomi] at (-46.200000, 36.000000) {ショウ};
\node[Kunyomi] at (-46.300000, 36.000000) {はぶ.く};
\node[Meaning] at (-46.250000, 37.650000) {conserve};
\node[Square] at (-44.200000, 35.900000) {};
\node[Kanji] at (-44.200000, 36.400000) {\textcolor[HTML]{14469c}{厚}};
\node[Onyomi] at (-44.150000, 36.000000) {コウ};
\node[Kunyomi] at (-44.250000, 36.000000) {あつ};
\node[Meaning] at (-44.200000, 37.650000) {thick};
\node[Square] at (-42.150000, 35.900000) {};
\node[Kanji] at (-42.150000, 36.400000) {\textcolor[HTML]{154caa}{奇}};
\node[Onyomi] at (-42.100000, 36.000000) {キ};
\node[Meaning] at (-42.150000, 37.650000) {odd};
\node[Square] at (-40.100000, 35.900000) {};
\node[Kanji] at (-40.100000, 36.400000) {\textcolor[HTML]{1551b8}{川}};
\node[Onyomi] at (-40.050000, 36.000000) {セン};
\node[Kunyomi] at (-40.150000, 36.000000) {かわ};
\node[Meaning] at (-40.100000, 37.650000) {river};
\node[Square] at (-38.050000, 35.900000) {};
\node[Kanji] at (-38.050000, 36.400000) {\textcolor[HTML]{14469c}{州}};
\node[Onyomi] at (-38.000000, 36.000000) {シュウ};
\node[Meaning] at (-38.050000, 37.650000) {state};
\node[Square] at (-36.000000, 35.900000) {};
\node[Kanji] at (-36.000000, 36.400000) {\textcolor[HTML]{133c80}{順}};
\node[Onyomi] at (-35.950000, 36.000000) {ジュン};
\node[Meaning] at (-36.000000, 37.650000) {order};
\node[Square] at (-33.950000, 35.900000) {};
\node[Kanji] at (-33.950000, 36.400000) {\textcolor[HTML]{145cd5}{水}};
\node[Onyomi] at (-33.900000, 36.000000) {スイ};
\node[Kunyomi] at (-34.000000, 36.000000) {みず};
\node[Meaning] at (-33.950000, 37.650000) {water};
\node[Square] at (-31.900000, 35.900000) {};
\node[Kanji] at (-31.900000, 36.400000) {\textcolor[HTML]{14418e}{氷}};
\node[Onyomi] at (-31.850000, 36.000000) {ヒョウ};
\node[Kunyomi] at (-31.950000, 36.000000) {こおり};
\node[Meaning] at (-31.900000, 37.650000) {ice};
\node[Square] at (-29.850000, 35.900000) {};
\node[Kanji] at (-29.850000, 36.400000) {\textcolor[HTML]{14418e}{永}};
\node[Onyomi] at (-29.800000, 36.000000) {エイ};
\node[Meaning] at (-29.850000, 37.650000) {eternity};
\node[Square] at (-27.800000, 35.900000) {};
\node[Kanji] at (-27.800000, 36.400000) {\textcolor[HTML]{133c80}{泉}};
\node[Onyomi] at (-27.750000, 36.000000) {セン};
\node[Kunyomi] at (-27.850000, 36.000000) {いずみ};
\node[Meaning] at (-27.800000, 37.650000) {spring};
\node[Square] at (-25.750000, 35.900000) {};
\node[Kanji] at (-25.750000, 36.400000) {\textcolor[HTML]{1557c6}{原}};
\node[Onyomi] at (-25.700000, 36.000000) {ゲン};
\node[Kunyomi] at (-25.800000, 36.000000) {はら};
\node[Meaning] at (-25.750000, 37.650000) {original};
\node[Square] at (-23.700000, 35.900000) {};
\node[Kanji] at (-23.700000, 36.400000) {\textcolor[HTML]{1551b8}{願}};
\node[Onyomi] at (-23.650000, 36.000000) {ガン};
\node[Kunyomi] at (-23.750000, 36.000000) {ねが};
\node[Meaning] at (-23.700000, 37.650000) {request};
\node[Square] at (-21.650000, 35.900000) {};
\node[Kanji] at (-21.650000, 36.400000) {\textcolor[HTML]{14418e}{泳}};
\node[Onyomi] at (-21.600000, 36.000000) {エイ};
\node[Kunyomi] at (-21.700000, 36.000000) {およ};
\node[Meaning] at (-21.650000, 37.650000) {swim};
\node[Square] at (-19.600000, 35.900000) {};
\node[Kanji] at (-19.600000, 36.400000) {\textcolor[HTML]{102b59}{沼}};
\node[Onyomi] at (-19.550000, 36.000000) {ショウ};
\node[Kunyomi] at (-19.650000, 36.000000) {ぬま};
\node[Meaning] at (-19.600000, 37.650000) {bog};
\node[Square] at (-17.550000, 35.900000) {};
\node[Kanji] at (-17.550000, 36.400000) {\textcolor[HTML]{154caa}{沖}};
\node[Onyomi] at (-17.500000, 36.000000) {チュウ};
\node[Kunyomi] at (-17.600000, 36.000000) {おき};
\node[Meaning] at (-17.550000, 37.650000) {open sea};
\node[Square] at (-15.500000, 35.900000) {};
\node[Kanji] at (-15.500000, 36.400000) {\textcolor[HTML]{14418e}{江}};
\node[Onyomi] at (-15.450000, 36.000000) {コウ};
\node[Kunyomi] at (-15.550000, 36.000000) {え};
\node[Meaning] at (-15.500000, 37.650000) {inlet};
\node[Square] at (-13.450000, 35.900000) {};
\node[Kanji] at (-13.450000, 36.400000) {\textcolor[HTML]{123673}{汁}};
\node[Onyomi] at (-13.400000, 36.000000) {ジュウ};
\node[Kunyomi] at (-13.500000, 36.000000) {しる};
\node[Meaning] at (-13.450000, 37.650000) {soup};
\node[Square] at (-11.400000, 35.900000) {};
\node[Kanji] at (-11.400000, 36.400000) {\textcolor[HTML]{133c80}{潮}};
\node[Onyomi] at (-11.350000, 36.000000) {チョウ};
\node[Kunyomi] at (-11.450000, 36.000000) {しお};
\node[Meaning] at (-11.400000, 37.650000) {tide};
\node[Square] at (-9.350000, 35.900000) {};
\node[Kanji] at (-9.350000, 36.400000) {\textcolor[HTML]{133c80}{源}};
\node[Onyomi] at (-9.300000, 36.000000) {ゲン};
\node[Kunyomi] at (-9.400000, 36.000000) {みなもと};
\node[Meaning] at (-9.350000, 37.650000) {origin};
\node[Square] at (-7.300000, 35.900000) {};
\node[Kanji] at (-7.300000, 36.400000) {\textcolor[HTML]{1551b8}{活}};
\node[Onyomi] at (-7.250000, 36.000000) {カツ};
\node[Meaning] at (-7.300000, 37.650000) {lively};
\node[Square] at (-5.250000, 35.900000) {};
\node[Kanji] at (-5.250000, 36.400000) {\textcolor[HTML]{1557c6}{消}};
\node[Onyomi] at (-5.200000, 36.000000) {ショウ};
\node[Kunyomi] at (-5.300000, 36.000000) {き.*};
\node[Meaning] at (-5.250000, 37.650000) {extinguish};
\node[Square] at (-3.200000, 35.900000) {};
\node[Kanji] at (-3.200000, 36.400000) {\textcolor[HTML]{14418e}{況}};
\node[Onyomi] at (-3.150000, 36.000000) {キョウ};
\node[Meaning] at (-3.200000, 37.650000) {condition};
\node[Square] at (-1.150000, 35.900000) {};
\node[Kanji] at (-1.150000, 36.400000) {\textcolor[HTML]{123673}{河}};
\node[Onyomi] at (-1.100000, 36.000000) {カ};
\node[Kunyomi] at (-1.200000, 36.000000) {かわ};
\node[Meaning] at (-1.150000, 37.650000) {river};
\node[Square] at (0.900000, 35.900000) {};
\node[Kanji] at (0.900000, 36.400000) {\textcolor[HTML]{14469c}{泊}};
\node[Onyomi] at (0.950000, 36.000000) {ハク};
\node[Kunyomi] at (0.850000, 36.000000) {と.まる};
\node[Meaning] at (0.900000, 37.650000) {overnight};
\node[Square] at (2.950000, 35.900000) {};
\node[Kanji] at (2.950000, 36.400000) {\textcolor[HTML]{154caa}{湖}};
\node[Onyomi] at (3.000000, 36.000000) {コ};
\node[Kunyomi] at (2.900000, 36.000000) {みずうみ};
\node[Meaning] at (2.950000, 37.650000) {lake};
\node[Square] at (5.000000, 35.900000) {};
\node[Kanji] at (5.000000, 36.400000) {\textcolor[HTML]{14418e}{測}};
\node[Onyomi] at (5.050000, 36.000000) {ソク};
\node[Kunyomi] at (4.950000, 36.000000) {はか.る};
\node[Meaning] at (5.000000, 37.650000) {measure};
\node[Square] at (7.050000, 35.900000) {};
\node[Kanji] at (7.050000, 36.400000) {\textcolor[HTML]{1551b8}{土}};
\node[Onyomi] at (7.100000, 36.000000) {ド};
\node[Kunyomi] at (7.000000, 36.000000) {つち};
\node[Meaning] at (7.050000, 37.650000) {soil};
\node[Square] at (9.100000, 35.900000) {};
\node[Kanji] at (9.100000, 36.400000) {\textcolor[HTML]{154caa}{吐}};
\node[Onyomi] at (9.150000, 36.000000) {ト};
\node[Kunyomi] at (9.050000, 36.000000) {は};
\node[Meaning] at (9.100000, 37.650000) {throw up};
\node[Square] at (11.150000, 35.900000) {};
\node[Kanji] at (11.150000, 36.400000) {\textcolor[HTML]{133c80}{圧}};
\node[Onyomi] at (11.200000, 36.000000) {アツ};
\node[Meaning] at (11.150000, 37.650000) {pressure};
\node[Square] at (13.200000, 35.900000) {};
\node[Kanji] at (13.200000, 36.400000) {\textcolor[HTML]{133c80}{埼}};
\node[Onyomi] at (13.250000, 36.000000) {キ};
\node[Kunyomi] at (13.150000, 36.000000) {さい};
\node[Meaning] at (13.200000, 37.650000) {cape};
\node[Square] at (15.250000, 35.900000) {};
\node[Kanji] at (15.250000, 36.400000) {\textcolor[HTML]{14418e}{垣}};
\node[Kunyomi] at (15.200000, 36.000000) {かき};
\node[Meaning] at (15.250000, 37.650000) {hedge};
\node[Square] at (17.300000, 35.900000) {};
\node[Kanji] at (17.300000, 36.400000) {\textcolor[HTML]{14418e}{封}};
\node[Onyomi] at (17.350000, 36.000000) {フウ};
\node[Meaning] at (17.300000, 37.650000) {seal};
\node[Square] at (19.350000, 35.900000) {};
\node[Kanji] at (19.350000, 36.400000) {\textcolor[HTML]{113066}{涯}};
\node[Onyomi] at (19.400000, 36.000000) {ガイ};
\node[Kunyomi] at (19.300000, 36.000000) {はて};
\node[Meaning] at (19.350000, 37.650000) {horizon};
\node[Square] at (21.400000, 35.900000) {};
\node[Kanji] at (21.400000, 36.400000) {\textcolor[HTML]{133c80}{寺}};
\node[Onyomi] at (21.450000, 36.000000) {ジ};
\node[Kunyomi] at (21.350000, 36.000000) {てら};
\node[Meaning] at (21.400000, 37.650000) {temple};
\node[Square] at (23.450000, 35.900000) {};
\node[Kanji] at (23.450000, 36.400000) {\textcolor[HTML]{1461e3}{時}};
\node[Onyomi] at (23.500000, 36.000000) {ジ};
\node[Kunyomi] at (23.400000, 36.000000) {とき};
\node[Meaning] at (23.450000, 37.650000) {time};
\node[Square] at (25.500000, 35.900000) {};
\node[Kanji] at (25.500000, 36.400000) {\textcolor[HTML]{14418e}{均}};
\node[Onyomi] at (25.550000, 36.000000) {キン};
\node[Kunyomi] at (25.450000, 36.000000) {ひと.しい};
\node[Meaning] at (25.500000, 37.650000) {equal};
\node[Square] at (27.550000, 35.900000) {};
\node[Kanji] at (27.550000, 36.400000) {\textcolor[HTML]{1557c6}{火}};
\node[Onyomi] at (27.600000, 36.000000) {カ};
\node[Kunyomi] at (27.500000, 36.000000) {ひ};
\node[Meaning] at (27.550000, 37.650000) {fire};
\node[Square] at (29.600000, 35.900000) {};
\node[Kanji] at (29.600000, 36.400000) {\textcolor[HTML]{154caa}{炎}};
\node[Onyomi] at (29.650000, 36.000000) {エン};
\node[Kunyomi] at (29.550000, 36.000000) {ほのお};
\node[Meaning] at (29.600000, 37.650000) {flame};
\node[Square] at (31.650000, 35.900000) {};
\node[Kanji] at (31.650000, 36.400000) {\textcolor[HTML]{102b59}{煩}};
\node[Onyomi] at (31.700000, 36.000000) {ハン};
\node[Kunyomi] at (31.600000, 36.000000) {うるさ};
\node[Meaning] at (31.650000, 37.650000) {annoy};
\node[Square] at (33.700000, 35.900000) {};
\node[Kanji] at (33.700000, 36.400000) {\textcolor[HTML]{133c80}{淡}};
\node[Onyomi] at (33.750000, 36.000000) {タン};
\node[Kunyomi] at (33.650000, 36.000000) {あわ.い};
\node[Meaning] at (33.700000, 37.650000) {faint};
\node[Square] at (35.750000, 35.900000) {};
\node[Kanji] at (35.750000, 36.400000) {\textcolor[HTML]{154caa}{灯}};
\node[Onyomi] at (35.800000, 36.000000) {トウ};
\node[Kunyomi] at (35.700000, 36.000000) {あかり};
\node[Meaning] at (35.750000, 37.650000) {lamp};
\node[Square] at (37.800000, 35.900000) {};
\node[Kanji] at (37.800000, 36.400000) {\textcolor[HTML]{14418e}{畑}};
\node[Kunyomi] at (37.750000, 36.000000) {はたけ};
\node[Meaning] at (37.800000, 37.650000) {field};
\node[Square] at (39.850000, 35.900000) {};
\node[Kanji] at (39.850000, 36.400000) {\textcolor[HTML]{154caa}{災}};
\node[Onyomi] at (39.900000, 36.000000) {サイ};
\node[Kunyomi] at (39.800000, 36.000000) {わざわ.い};
\node[Meaning] at (39.850000, 37.650000) {disaster};
\node[Square] at (41.900000, 35.900000) {};
\node[Kanji] at (41.900000, 36.400000) {\textcolor[HTML]{14469c}{灰}};
\node[Onyomi] at (41.950000, 36.000000) {カイ};
\node[Kunyomi] at (41.850000, 36.000000) {はい};
\node[Meaning] at (41.900000, 37.650000) {ashes};
\node[Square] at (43.950000, 35.900000) {};
\node[Kanji] at (43.950000, 36.400000) {\textcolor[HTML]{1551b8}{点}};
\node[Onyomi] at (44.000000, 36.000000) {テン};
\node[Kunyomi] at (43.900000, 36.000000) {つ.ける};
\node[Meaning] at (43.950000, 37.650000) {point};
\node[Square] at (46.000000, 35.900000) {};
\node[Kanji] at (46.000000, 36.400000) {\textcolor[HTML]{14469c}{照}};
\node[Onyomi] at (46.050000, 36.000000) {ショウ};
\node[Kunyomi] at (45.950000, 36.000000) {て.*};
\node[Meaning] at (46.000000, 37.650000) {illuminate};
\node[Square] at (48.050000, 35.900000) {};
\node[Kanji] at (48.050000, 36.400000) {\textcolor[HTML]{14469c}{魚}};
\node[Onyomi] at (48.100000, 36.000000) {ギョ};
\node[Kunyomi] at (48.000000, 36.000000) {さかな};
\node[Meaning] at (48.050000, 37.650000) {fish};
\node[Square] at (50.100000, 35.900000) {};
\node[Kanji] at (50.100000, 36.400000) {\textcolor[HTML]{14418e}{漁}};
\node[Onyomi] at (50.150000, 36.000000) {ギョ};
\node[Kunyomi] at (50.050000, 36.000000) {あさ.る};
\node[Meaning] at (50.100000, 37.650000) {fishing};
\node[Square] at (52.150000, 35.900000) {};
\node[Kanji] at (52.150000, 36.400000) {\textcolor[HTML]{123673}{里}};
\node[Onyomi] at (52.200000, 36.000000) {リ};
\node[Kunyomi] at (52.100000, 36.000000) {さと};
\node[Meaning] at (52.150000, 37.650000) {home village};
\node[Square] at (54.200000, 35.900000) {};
\node[Kanji] at (54.200000, 36.400000) {\textcolor[HTML]{1557c6}{黒}};
\node[Onyomi] at (54.250000, 36.000000) {コク};
\node[Kunyomi] at (54.150000, 36.000000) {くろ.い};
\node[Meaning] at (54.200000, 37.650000) {black};
\node[Square] at (56.250000, 35.900000) {};
\node[Kanji] at (56.250000, 36.400000) {\textcolor[HTML]{113066}{墨}};
\node[Kunyomi] at (56.200000, 36.000000) {すみ};
\node[Meaning] at (56.250000, 37.650000) {black ink};
\node[Meaning] at (-58.500000, 36.450000) {17.18\%};
\node[Square] at (-56.500000, 33.850000) {};
\node[Kanji] at (-56.500000, 34.350000) {\textcolor[HTML]{102b59}{鯉}};
\node[Onyomi] at (-56.450000, 33.950000) {リ};
\node[Kunyomi] at (-56.550000, 33.950000) {こい};
\node[Meaning] at (-56.500000, 35.600000) {carp};
\node[Square] at (-54.450000, 33.850000) {};
\node[Kanji] at (-54.450000, 34.350000) {\textcolor[HTML]{14469c}{量}};
\node[Onyomi] at (-54.400000, 33.950000) {リョウ};
\node[Kunyomi] at (-54.500000, 33.950000) {はか.る};
\node[Meaning] at (-54.450000, 35.600000) {quantity};
\node[Square] at (-52.400000, 33.850000) {};
\node[Kanji] at (-52.400000, 34.350000) {\textcolor[HTML]{0e254c}{厘}};
\node[Onyomi] at (-52.350000, 33.950000) {リン};
\node[Meaning] at (-52.400000, 35.600000) {thousandth};
\node[Square] at (-50.350000, 33.850000) {};
\node[Kanji] at (-50.350000, 34.350000) {\textcolor[HTML]{14469c}{埋}};
\node[Onyomi] at (-50.300000, 33.950000) {マイ};
\node[Kunyomi] at (-50.400000, 33.950000) {う};
\node[Meaning] at (-50.350000, 35.600000) {bury};
\node[Square] at (-48.300000, 33.850000) {};
\node[Kanji] at (-48.300000, 34.350000) {\textcolor[HTML]{145cd5}{同}};
\node[Onyomi] at (-48.250000, 33.950000) {ドウ};
\node[Kunyomi] at (-48.350000, 33.950000) {おな.じ};
\node[Meaning] at (-48.300000, 35.600000) {same};
\node[Square] at (-46.250000, 33.850000) {};
\node[Kanji] at (-46.250000, 34.350000) {\textcolor[HTML]{133c80}{洞}};
\node[Onyomi] at (-46.200000, 33.950000) {ドウ};
\node[Kunyomi] at (-46.300000, 33.950000) {ほら};
\node[Meaning] at (-46.250000, 35.600000) {cave};
\node[Square] at (-44.200000, 33.850000) {};
\node[Kanji] at (-44.200000, 34.350000) {\textcolor[HTML]{123673}{胴}};
\node[Onyomi] at (-44.150000, 33.950000) {ドウ};
\node[Meaning] at (-44.200000, 35.600000) {torso};
\node[Square] at (-42.150000, 33.850000) {};
\node[Kanji] at (-42.150000, 34.350000) {\textcolor[HTML]{1461e3}{向}};
\node[Onyomi] at (-42.100000, 33.950000) {コウ};
\node[Kunyomi] at (-42.200000, 33.950000) {む.き};
\node[Meaning] at (-42.150000, 35.600000) {yonder};
\node[Square] at (-40.100000, 33.850000) {};
\node[Kanji] at (-40.100000, 34.350000) {\textcolor[HTML]{0e254c}{尚}};
\node[Onyomi] at (-40.050000, 33.950000) {ショウ};
\node[Kunyomi] at (-40.150000, 33.950000) {なお};
\node[Meaning] at (-40.100000, 35.600000) {furthermore};
\node[Square] at (-38.050000, 33.850000) {};
\node[Kanji] at (-38.050000, 34.350000) {\textcolor[HTML]{154caa}{字}};
\node[Onyomi] at (-38.000000, 33.950000) {ジ};
\node[Meaning] at (-38.050000, 35.600000) {letter};
\node[Square] at (-36.000000, 33.850000) {};
\node[Kanji] at (-36.000000, 34.350000) {\textcolor[HTML]{1551b8}{守}};
\node[Onyomi] at (-35.950000, 33.950000) {ス};
\node[Kunyomi] at (-36.050000, 33.950000) {まも.る};
\node[Meaning] at (-36.000000, 35.600000) {protect};
\node[Square] at (-33.950000, 33.850000) {};
\node[Kanji] at (-33.950000, 34.350000) {\textcolor[HTML]{154caa}{完}};
\node[Onyomi] at (-33.900000, 33.950000) {カン};
\node[Meaning] at (-33.950000, 35.600000) {perfect};
\node[Square] at (-31.900000, 33.850000) {};
\node[Kanji] at (-31.900000, 34.350000) {\textcolor[HTML]{133c80}{宣}};
\node[Onyomi] at (-31.850000, 33.950000) {セン};
\node[Kunyomi] at (-31.950000, 33.950000) {のたま.う};
\node[Meaning] at (-31.900000, 35.600000) {proclaim};
\node[Square] at (-29.850000, 33.850000) {};
\node[Kanji] at (-29.850000, 34.350000) {\textcolor[HTML]{0e254c}{宵}};
\node[Onyomi] at (-29.800000, 33.950000) {ショウ};
\node[Kunyomi] at (-29.900000, 33.950000) {よい};
\node[Meaning] at (-29.850000, 35.600000) {wee hours};
\node[Square] at (-27.800000, 33.850000) {};
\node[Kanji] at (-27.800000, 34.350000) {\textcolor[HTML]{1557c6}{安}};
\node[Onyomi] at (-27.750000, 33.950000) {アン};
\node[Kunyomi] at (-27.850000, 33.950000) {やす.い};
\node[Meaning] at (-27.800000, 35.600000) {relax};
\node[Square] at (-25.750000, 33.850000) {};
\node[Kanji] at (-25.750000, 34.350000) {\textcolor[HTML]{123673}{宴}};
\node[Onyomi] at (-25.700000, 33.950000) {エン};
\node[Kunyomi] at (-25.800000, 33.950000) {うたげ};
\node[Meaning] at (-25.750000, 35.600000) {banquet};
\node[Square] at (-23.700000, 33.850000) {};
\node[Kanji] at (-23.700000, 34.350000) {\textcolor[HTML]{1551b8}{寄}};
\node[Onyomi] at (-23.650000, 33.950000) {キ};
\node[Kunyomi] at (-23.750000, 33.950000) {よ.る};
\node[Meaning] at (-23.700000, 35.600000) {approach};
\node[Square] at (-21.650000, 33.850000) {};
\node[Kanji] at (-21.650000, 34.350000) {\textcolor[HTML]{14469c}{富}};
\node[Onyomi] at (-21.600000, 33.950000) {フ};
\node[Kunyomi] at (-21.700000, 33.950000) {と};
\node[Meaning] at (-21.650000, 35.600000) {rich};
\node[Square] at (-19.600000, 33.850000) {};
\node[Kanji] at (-19.600000, 34.350000) {\textcolor[HTML]{133c80}{貯}};
\node[Onyomi] at (-19.550000, 33.950000) {チョ};
\node[Kunyomi] at (-19.650000, 33.950000) {たくわ.える};
\node[Meaning] at (-19.600000, 35.600000) {savings};
\node[Square] at (-17.550000, 33.850000) {};
\node[Kanji] at (-17.550000, 34.350000) {\textcolor[HTML]{1557c6}{木}};
\node[Onyomi] at (-17.500000, 33.950000) {モク};
\node[Kunyomi] at (-17.600000, 33.950000) {き};
\node[Meaning] at (-17.550000, 35.600000) {tree};
\node[Square] at (-15.500000, 33.850000) {};
\node[Kanji] at (-15.500000, 34.350000) {\textcolor[HTML]{133c80}{林}};
\node[Onyomi] at (-15.450000, 33.950000) {リン};
\node[Kunyomi] at (-15.550000, 33.950000) {はやし};
\node[Meaning] at (-15.500000, 35.600000) {forest};
\node[Square] at (-13.450000, 33.850000) {};
\node[Kanji] at (-13.450000, 34.350000) {\textcolor[HTML]{154caa}{森}};
\node[Onyomi] at (-13.400000, 33.950000) {シン};
\node[Kunyomi] at (-13.500000, 33.950000) {もり};
\node[Meaning] at (-13.450000, 35.600000) {forest};
\node[Square] at (-11.400000, 33.850000) {};
\node[Kanji] at (-11.400000, 34.350000) {\textcolor[HTML]{123673}{枠}};
\node[Kunyomi] at (-11.450000, 33.950000) {わく};
\node[Meaning] at (-11.400000, 35.600000) {frame};
\node[Square] at (-9.350000, 33.850000) {};
\node[Kanji] at (-9.350000, 34.350000) {\textcolor[HTML]{14469c}{棚}};
\node[Onyomi] at (-9.300000, 33.950000) {ホウ};
\node[Kunyomi] at (-9.400000, 33.950000) {たな};
\node[Meaning] at (-9.350000, 35.600000) {shelf};
\node[Square] at (-7.300000, 33.850000) {};
\node[Kanji] at (-7.300000, 34.350000) {\textcolor[HTML]{0e254c}{杏}};
\node[Onyomi] at (-7.250000, 33.950000) {アン};
\node[Kunyomi] at (-7.350000, 33.950000) {あんず};
\node[Meaning] at (-7.300000, 35.600000) {apricot};
\node[Square] at (-5.250000, 33.850000) {};
\node[Kanji] at (-5.250000, 34.350000) {\textcolor[HTML]{14418e}{植}};
\node[Onyomi] at (-5.200000, 33.950000) {ショク};
\node[Kunyomi] at (-5.300000, 33.950000) {う.*};
\node[Meaning] at (-5.250000, 35.600000) {plant};
\node[Square] at (-3.200000, 33.850000) {};
\node[Kanji] at (-3.200000, 34.350000) {\textcolor[HTML]{102b59}{枯}};
\node[Onyomi] at (-3.150000, 33.950000) {コ};
\node[Kunyomi] at (-3.250000, 33.950000) {か};
\node[Meaning] at (-3.200000, 35.600000) {wither};
\node[Square] at (-1.150000, 33.850000) {};
\node[Kanji] at (-1.150000, 34.350000) {\textcolor[HTML]{0e254c}{朴}};
\node[Onyomi] at (-1.100000, 33.950000) {ボク};
\node[Kunyomi] at (-1.200000, 33.950000) {えのき};
\node[Meaning] at (-1.150000, 35.600000) {simple};
\node[Square] at (0.900000, 33.850000) {};
\node[Kanji] at (0.900000, 34.350000) {\textcolor[HTML]{154caa}{村}};
\node[Onyomi] at (0.950000, 33.950000) {ソン};
\node[Kunyomi] at (0.850000, 33.950000) {むら};
\node[Meaning] at (0.900000, 35.600000) {village};
\node[Square] at (2.950000, 33.850000) {};
\node[Kanji] at (2.950000, 34.350000) {\textcolor[HTML]{1557c6}{相}};
\node[Onyomi] at (3.000000, 33.950000) {ソウ};
\node[Kunyomi] at (2.900000, 33.950000) {あい};
\node[Meaning] at (2.950000, 35.600000) {mutual};
\node[Square] at (5.000000, 33.850000) {};
\node[Kanji] at (5.000000, 34.350000) {\textcolor[HTML]{154caa}{机}};
\node[Kunyomi] at (4.950000, 33.950000) {つくえ};
\node[Meaning] at (5.000000, 35.600000) {desk};
\node[Square] at (7.050000, 33.850000) {};
\node[Kanji] at (7.050000, 34.350000) {\textcolor[HTML]{1968ed}{本}};
\node[Onyomi] at (7.100000, 33.950000) {ホン};
\node[Kunyomi] at (7.000000, 33.950000) {もと};
\node[Meaning] at (7.050000, 35.600000) {book};
\node[Square] at (9.100000, 33.850000) {};
\node[Kanji] at (9.100000, 34.350000) {\textcolor[HTML]{14418e}{札}};
\node[Onyomi] at (9.150000, 33.950000) {サツ};
\node[Kunyomi] at (9.050000, 33.950000) {ふだ};
\node[Meaning] at (9.100000, 35.600000) {bill};
\node[Square] at (11.150000, 33.850000) {};
\node[Kanji] at (11.150000, 34.350000) {\textcolor[HTML]{102b59}{暦}};
\node[Onyomi] at (11.200000, 33.950000) {レキ};
\node[Kunyomi] at (11.100000, 33.950000) {こよみ};
\node[Meaning] at (11.150000, 35.600000) {calendar};
\node[Square] at (13.200000, 33.850000) {};
\node[Kanji] at (13.200000, 34.350000) {\textcolor[HTML]{14469c}{案}};
\node[Onyomi] at (13.250000, 33.950000) {アン};
\node[Meaning] at (13.200000, 35.600000) {plan};
\node[Square] at (15.250000, 33.850000) {};
\node[Kanji] at (15.250000, 34.350000) {\textcolor[HTML]{102b59}{燥}};
\node[Onyomi] at (15.300000, 33.950000) {ソウ};
\node[Kunyomi] at (15.200000, 33.950000) {はしゃ.ぐ};
\node[Meaning] at (15.250000, 35.600000) {dry up};
\node[Square] at (17.300000, 33.850000) {};
\node[Kanji] at (17.300000, 34.350000) {\textcolor[HTML]{14469c}{未}};
\node[Onyomi] at (17.350000, 33.950000) {ミ};
\node[Kunyomi] at (17.250000, 33.950000) {ま.だ};
\node[Meaning] at (17.300000, 35.600000) {not yet};
\node[Square] at (19.350000, 33.850000) {};
\node[Kanji] at (19.350000, 34.350000) {\textcolor[HTML]{154caa}{末}};
\node[Onyomi] at (19.400000, 33.950000) {マツ};
\node[Kunyomi] at (19.300000, 33.950000) {すえ};
\node[Meaning] at (19.350000, 35.600000) {end};
\node[Square] at (21.400000, 33.850000) {};
\node[Kanji] at (21.400000, 34.350000) {\textcolor[HTML]{1557c6}{味}};
\node[Onyomi] at (21.450000, 33.950000) {ミ};
\node[Kunyomi] at (21.350000, 33.950000) {あじ};
\node[Meaning] at (21.400000, 35.600000) {flavor};
\node[Square] at (23.450000, 33.850000) {};
\node[Kanji] at (23.450000, 34.350000) {\textcolor[HTML]{154caa}{妹}};
\node[Onyomi] at (23.500000, 33.950000) {マイ};
\node[Kunyomi] at (23.400000, 33.950000) {いもうと};
\node[Meaning] at (23.450000, 35.600000) {younger sister};
\node[Square] at (25.500000, 33.850000) {};
\node[Kanji] at (25.500000, 34.350000) {\textcolor[HTML]{0e254c}{朱}};
\node[Onyomi] at (25.550000, 33.950000) {シュ};
\node[Kunyomi] at (25.450000, 33.950000) {あけ};
\node[Meaning] at (25.500000, 35.600000) {vermillion};
\node[Square] at (27.550000, 33.850000) {};
\node[Kanji] at (27.550000, 34.350000) {\textcolor[HTML]{123673}{株}};
\node[Onyomi] at (27.600000, 33.950000) {シュ};
\node[Kunyomi] at (27.500000, 33.950000) {かぶ};
\node[Meaning] at (27.550000, 35.600000) {stocks};
\node[Square] at (29.600000, 33.850000) {};
\node[Kanji] at (29.600000, 34.350000) {\textcolor[HTML]{154caa}{若}};
\node[Onyomi] at (29.650000, 33.950000) {ジャク};
\node[Kunyomi] at (29.550000, 33.950000) {わか};
\node[Meaning] at (29.600000, 35.600000) {young};
\node[Square] at (31.650000, 33.850000) {};
\node[Kanji] at (31.650000, 34.350000) {\textcolor[HTML]{14469c}{草}};
\node[Onyomi] at (31.700000, 33.950000) {ソウ};
\node[Kunyomi] at (31.600000, 33.950000) {くさ};
\node[Meaning] at (31.650000, 35.600000) {grass};
\node[Square] at (33.700000, 33.850000) {};
\node[Kanji] at (33.700000, 34.350000) {\textcolor[HTML]{154caa}{苦}};
\node[Onyomi] at (33.750000, 33.950000) {ク};
\node[Kunyomi] at (33.650000, 33.950000) {くる};
\node[Meaning] at (33.700000, 35.600000) {suffering};
\node[Square] at (35.750000, 33.850000) {};
\node[Kanji] at (35.750000, 34.350000) {\textcolor[HTML]{102b59}{寛}};
\node[Onyomi] at (35.800000, 33.950000) {カン};
\node[Kunyomi] at (35.700000, 33.950000) {くつろ.ぐ};
\node[Meaning] at (35.750000, 35.600000) {tolerance};
\node[Square] at (37.800000, 33.850000) {};
\node[Kanji] at (37.800000, 34.350000) {\textcolor[HTML]{154caa}{薄}};
\node[Onyomi] at (37.850000, 33.950000) {ハク};
\node[Kunyomi] at (37.750000, 33.950000) {うす.*};
\node[Meaning] at (37.800000, 35.600000) {dilute};
\node[Square] at (39.850000, 33.850000) {};
\node[Kanji] at (39.850000, 34.350000) {\textcolor[HTML]{1557c6}{葉}};
\node[Kunyomi] at (39.800000, 33.950000) {は};
\node[Meaning] at (39.850000, 35.600000) {leaf};
\node[Square] at (41.900000, 33.850000) {};
\node[Kanji] at (41.900000, 34.350000) {\textcolor[HTML]{14418e}{模}};
\node[Onyomi] at (41.950000, 33.950000) {モ};
\node[Meaning] at (41.900000, 35.600000) {imitation};
\node[Square] at (43.950000, 33.850000) {};
\node[Kanji] at (43.950000, 34.350000) {\textcolor[HTML]{113066}{漠}};
\node[Onyomi] at (44.000000, 33.950000) {バク};
\node[Meaning] at (43.950000, 35.600000) {desert};
\node[Square] at (46.000000, 33.850000) {};
\node[Kanji] at (46.000000, 34.350000) {\textcolor[HTML]{14469c}{墓}};
\node[Onyomi] at (46.050000, 33.950000) {ボ};
\node[Kunyomi] at (45.950000, 33.950000) {はか};
\node[Meaning] at (46.000000, 35.600000) {grave};
\node[Square] at (48.050000, 33.850000) {};
\node[Kanji] at (48.050000, 34.350000) {\textcolor[HTML]{14418e}{暮}};
\node[Onyomi] at (48.100000, 33.950000) {ボ};
\node[Kunyomi] at (48.000000, 33.950000) {く.*};
\node[Meaning] at (48.050000, 35.600000) {livelihood};
\node[Square] at (50.100000, 33.850000) {};
\node[Kanji] at (50.100000, 34.350000) {\textcolor[HTML]{113066}{膜}};
\node[Onyomi] at (50.150000, 33.950000) {マク};
\node[Meaning] at (50.100000, 35.600000) {membrane};
\node[Square] at (52.150000, 33.850000) {};
\node[Kanji] at (52.150000, 34.350000) {\textcolor[HTML]{0e254c}{苗}};
\node[Onyomi] at (52.200000, 33.950000) {ミョウ};
\node[Kunyomi] at (52.100000, 33.950000) {なえ};
\node[Meaning] at (52.150000, 35.600000) {seedling};
\node[Square] at (54.200000, 33.850000) {};
\node[Kanji] at (54.200000, 34.350000) {\textcolor[HTML]{14418e}{兆}};
\node[Onyomi] at (54.250000, 33.950000) {チョウ};
\node[Meaning] at (54.200000, 35.600000) {omen};
\node[Square] at (56.250000, 33.850000) {};
\node[Kanji] at (56.250000, 34.350000) {\textcolor[HTML]{123673}{桃}};
\node[Kunyomi] at (56.200000, 33.950000) {もも};
\node[Meaning] at (56.250000, 35.600000) {peach};
\node[Meaning] at (-58.500000, 34.400000) {19.72\%};
\node[Square] at (-56.500000, 31.800000) {};
\node[Kanji] at (-56.500000, 32.300000) {\textcolor[HTML]{14469c}{眺}};
\node[Onyomi] at (-56.450000, 31.900000) {チョウ};
\node[Kunyomi] at (-56.550000, 31.900000) {なが.める};
\node[Meaning] at (-56.500000, 33.550000) {stare};
\node[Square] at (-54.450000, 31.800000) {};
\node[Kanji] at (-54.450000, 32.300000) {\textcolor[HTML]{154caa}{犬}};
\node[Onyomi] at (-54.400000, 31.900000) {ケン};
\node[Kunyomi] at (-54.500000, 31.900000) {いぬ};
\node[Meaning] at (-54.450000, 33.550000) {dog};
\node[Square] at (-52.400000, 31.800000) {};
\node[Kanji] at (-52.400000, 32.300000) {\textcolor[HTML]{1551b8}{状}};
\node[Onyomi] at (-52.350000, 31.900000) {ジョウ};
\node[Meaning] at (-52.400000, 33.550000) {condition};
\node[Square] at (-50.350000, 31.800000) {};
\node[Kanji] at (-50.350000, 32.300000) {\textcolor[HTML]{1551b8}{黙}};
\node[Onyomi] at (-50.300000, 31.900000) {モク};
\node[Kunyomi] at (-50.400000, 31.900000) {だま.る};
\node[Meaning] at (-50.350000, 33.550000) {shut up};
\node[Square] at (-48.300000, 31.800000) {};
\node[Kanji] at (-48.300000, 32.300000) {\textcolor[HTML]{1557c6}{然}};
\node[Onyomi] at (-48.250000, 31.900000) {ゼン};
\node[Kunyomi] at (-48.350000, 31.900000) {しか};
\node[Meaning] at (-48.300000, 33.550000) {nature};
\node[Square] at (-46.250000, 31.800000) {};
\node[Kanji] at (-46.250000, 32.300000) {\textcolor[HTML]{123673}{狩}};
\node[Onyomi] at (-46.200000, 31.900000) {シュ};
\node[Kunyomi] at (-46.300000, 31.900000) {か};
\node[Meaning] at (-46.250000, 33.550000) {hunt};
\node[Square] at (-44.200000, 31.800000) {};
\node[Kanji] at (-44.200000, 32.300000) {\textcolor[HTML]{154caa}{猫}};
\node[Kunyomi] at (-44.250000, 31.900000) {ねこ};
\node[Meaning] at (-44.200000, 33.550000) {cat};
\node[Square] at (-42.150000, 31.800000) {};
\node[Kanji] at (-42.150000, 32.300000) {\textcolor[HTML]{14418e}{牛}};
\node[Onyomi] at (-42.100000, 31.900000) {ギュウ};
\node[Kunyomi] at (-42.200000, 31.900000) {うし};
\node[Meaning] at (-42.150000, 33.550000) {cow};
\node[Square] at (-40.100000, 31.800000) {};
\node[Kanji] at (-40.100000, 32.300000) {\textcolor[HTML]{1551b8}{特}};
\node[Onyomi] at (-40.050000, 31.900000) {トク};
\node[Meaning] at (-40.100000, 33.550000) {special};
\node[Square] at (-38.050000, 31.800000) {};
\node[Kanji] at (-38.050000, 32.300000) {\textcolor[HTML]{154caa}{告}};
\node[Onyomi] at (-38.000000, 31.900000) {コク};
\node[Kunyomi] at (-38.100000, 31.900000) {つ.げる};
\node[Meaning] at (-38.050000, 33.550000) {announce};
\node[Square] at (-36.000000, 31.800000) {};
\node[Kanji] at (-36.000000, 32.300000) {\textcolor[HTML]{1968ed}{先}};
\node[Onyomi] at (-35.950000, 31.900000) {セン};
\node[Kunyomi] at (-36.050000, 31.900000) {さき};
\node[Meaning] at (-36.000000, 33.550000) {previous};
\node[Square] at (-33.950000, 31.800000) {};
\node[Kanji] at (-33.950000, 32.300000) {\textcolor[HTML]{14469c}{洗}};
\node[Onyomi] at (-33.900000, 31.900000) {セン};
\node[Kunyomi] at (-34.000000, 31.900000) {あら.う};
\node[Meaning] at (-33.950000, 33.550000) {wash};
\node[Square] at (-31.900000, 31.800000) {};
\node[Kanji] at (-31.900000, 32.300000) {\textcolor[HTML]{154caa}{介}};
\node[Onyomi] at (-31.850000, 31.900000) {カイ};
\node[Meaning] at (-31.900000, 33.550000) {jammed in};
\node[Square] at (-29.850000, 31.800000) {};
\node[Kanji] at (-29.850000, 32.300000) {\textcolor[HTML]{1557c6}{界}};
\node[Onyomi] at (-29.800000, 31.900000) {カイ};
\node[Meaning] at (-29.850000, 33.550000) {world};
\node[Square] at (-27.800000, 31.800000) {};
\node[Kanji] at (-27.800000, 32.300000) {\textcolor[HTML]{154caa}{茶}};
\node[Onyomi] at (-27.750000, 31.900000) {チャ};
\node[Meaning] at (-27.800000, 33.550000) {tea};
\node[Square] at (-25.750000, 31.800000) {};
\node[Kanji] at (-25.750000, 32.300000) {\textcolor[HTML]{1461e3}{合}};
\node[Onyomi] at (-25.700000, 31.900000) {ゴウ};
\node[Kunyomi] at (-25.800000, 31.900000) {あ};
\node[Meaning] at (-25.750000, 33.550000) {suit};
\node[Square] at (-23.700000, 31.800000) {};
\node[Kanji] at (-23.700000, 32.300000) {\textcolor[HTML]{14469c}{塔}};
\node[Onyomi] at (-23.650000, 31.900000) {トウ};
\node[Meaning] at (-23.700000, 33.550000) {tower};
\node[Square] at (-21.650000, 31.800000) {};
\node[Kanji] at (-21.650000, 32.300000) {\textcolor[HTML]{154caa}{王}};
\node[Onyomi] at (-21.600000, 31.900000) {オウ};
\node[Meaning] at (-21.650000, 33.550000) {king};
\node[Square] at (-19.600000, 31.800000) {};
\node[Kanji] at (-19.600000, 32.300000) {\textcolor[HTML]{154caa}{玉}};
\node[Onyomi] at (-19.550000, 31.900000) {ギョク};
\node[Kunyomi] at (-19.650000, 31.900000) {たま};
\node[Meaning] at (-19.600000, 33.550000) {ball};
\node[Square] at (-17.550000, 31.800000) {};
\node[Kanji] at (-17.550000, 32.300000) {\textcolor[HTML]{14469c}{宝}};
\node[Onyomi] at (-17.500000, 31.900000) {ホウ};
\node[Kunyomi] at (-17.600000, 31.900000) {たから};
\node[Meaning] at (-17.550000, 33.550000) {treasure};
\node[Square] at (-15.500000, 31.800000) {};
\node[Kanji] at (-15.500000, 32.300000) {\textcolor[HTML]{113066}{珠}};
\node[Onyomi] at (-15.450000, 31.900000) {シュ};
\node[Kunyomi] at (-15.550000, 31.900000) {たましい};
\node[Meaning] at (-15.500000, 33.550000) {pearl};
\node[Square] at (-13.450000, 31.800000) {};
\node[Kanji] at (-13.450000, 32.300000) {\textcolor[HTML]{1557c6}{現}};
\node[Onyomi] at (-13.400000, 31.900000) {ゲン};
\node[Kunyomi] at (-13.500000, 31.900000) {あらわ.*};
\node[Meaning] at (-13.450000, 33.550000) {present time};
\node[Square] at (-11.400000, 31.800000) {};
\node[Kanji] at (-11.400000, 32.300000) {\textcolor[HTML]{14469c}{狂}};
\node[Onyomi] at (-11.350000, 31.900000) {キョウ};
\node[Kunyomi] at (-11.450000, 31.900000) {くる.*};
\node[Meaning] at (-11.400000, 33.550000) {lunatic};
\node[Square] at (-9.350000, 31.800000) {};
\node[Kanji] at (-9.350000, 32.300000) {\textcolor[HTML]{123673}{皇}};
\node[Onyomi] at (-9.300000, 31.900000) {コウ};
\node[Meaning] at (-9.350000, 33.550000) {emperor};
\node[Square] at (-7.300000, 31.800000) {};
\node[Kanji] at (-7.300000, 32.300000) {\textcolor[HTML]{0e254c}{呈}};
\node[Onyomi] at (-7.250000, 31.900000) {テイ};
\node[Meaning] at (-7.300000, 33.550000) {present};
\node[Square] at (-5.250000, 31.800000) {};
\node[Kanji] at (-5.250000, 32.300000) {\textcolor[HTML]{1461e3}{全}};
\node[Onyomi] at (-5.200000, 31.900000) {ゼン};
\node[Kunyomi] at (-5.300000, 31.900000) {すべ.て};
\node[Meaning] at (-5.250000, 33.550000) {all};
\node[Square] at (-3.200000, 31.800000) {};
\node[Kanji] at (-3.200000, 32.300000) {\textcolor[HTML]{113066}{栓}};
\node[Onyomi] at (-3.150000, 31.900000) {セン};
\node[Meaning] at (-3.200000, 33.550000) {cork};
\node[Square] at (-1.150000, 31.800000) {};
\node[Kanji] at (-1.150000, 32.300000) {\textcolor[HTML]{145cd5}{理}};
\node[Onyomi] at (-1.100000, 31.900000) {リ};
\node[Kunyomi] at (-1.200000, 31.900000) {ことわり};
\node[Meaning] at (-1.150000, 33.550000) {reason};
\node[Square] at (0.900000, 31.800000) {};
\node[Kanji] at (0.900000, 32.300000) {\textcolor[HTML]{1551b8}{主}};
\node[Onyomi] at (0.950000, 31.900000) {シュ};
\node[Kunyomi] at (0.850000, 31.900000) {おも};
\node[Meaning] at (0.900000, 33.550000) {master};
\node[Square] at (2.950000, 31.800000) {};
\node[Kanji] at (2.950000, 32.300000) {\textcolor[HTML]{1551b8}{注}};
\node[Onyomi] at (3.000000, 31.900000) {チュウ};
\node[Kunyomi] at (2.900000, 31.900000) {そそ.ぐ};
\node[Meaning] at (2.950000, 33.550000) {pour};
\node[Square] at (5.000000, 31.800000) {};
\node[Kanji] at (5.000000, 32.300000) {\textcolor[HTML]{14418e}{柱}};
\node[Onyomi] at (5.050000, 31.900000) {チュウ};
\node[Kunyomi] at (4.950000, 31.900000) {はしら};
\node[Meaning] at (5.000000, 33.550000) {pillar};
\node[Square] at (7.050000, 31.800000) {};
\node[Kanji] at (7.050000, 32.300000) {\textcolor[HTML]{145cd5}{金}};
\node[Onyomi] at (7.100000, 31.900000) {キン};
\node[Kunyomi] at (7.000000, 31.900000) {かね};
\node[Meaning] at (7.050000, 33.550000) {gold};
\node[Square] at (9.100000, 31.800000) {};
\node[Kanji] at (9.100000, 32.300000) {\textcolor[HTML]{123673}{鉢}};
\node[Onyomi] at (9.150000, 31.900000) {ハチ};
\node[Meaning] at (9.100000, 33.550000) {bowl};
\node[Square] at (11.150000, 31.800000) {};
\node[Kanji] at (11.150000, 32.300000) {\textcolor[HTML]{133c80}{銅}};
\node[Onyomi] at (11.200000, 31.900000) {ドウ};
\node[Kunyomi] at (11.100000, 31.900000) {あかがね};
\node[Meaning] at (11.150000, 33.550000) {copper};
\node[Square] at (13.200000, 31.800000) {};
\node[Kanji] at (13.200000, 32.300000) {\textcolor[HTML]{133c80}{釣}};
\node[Onyomi] at (13.250000, 31.900000) {チョウ};
\node[Kunyomi] at (13.150000, 31.900000) {つ};
\node[Meaning] at (13.200000, 33.550000) {fishing};
\node[Square] at (15.250000, 31.800000) {};
\node[Kanji] at (15.250000, 32.300000) {\textcolor[HTML]{133c80}{針}};
\node[Onyomi] at (15.300000, 31.900000) {シン};
\node[Kunyomi] at (15.200000, 31.900000) {はり};
\node[Meaning] at (15.250000, 33.550000) {needle};
\node[Square] at (17.300000, 31.800000) {};
\node[Kanji] at (17.300000, 32.300000) {\textcolor[HTML]{102b59}{銘}};
\node[Onyomi] at (17.350000, 31.900000) {メイ};
\node[Meaning] at (17.300000, 33.550000) {inscription};
\node[Square] at (19.350000, 31.800000) {};
\node[Kanji] at (19.350000, 32.300000) {\textcolor[HTML]{102b59}{鎮}};
\node[Onyomi] at (19.400000, 31.900000) {チン};
\node[Kunyomi] at (19.300000, 31.900000) {おさえ};
\node[Meaning] at (19.350000, 33.550000) {tranquilize};
\node[Square] at (21.400000, 31.800000) {};
\node[Kanji] at (21.400000, 32.300000) {\textcolor[HTML]{1557c6}{道}};
\node[Onyomi] at (21.450000, 31.900000) {ドウ};
\node[Kunyomi] at (21.350000, 31.900000) {みち};
\node[Meaning] at (21.400000, 33.550000) {road};
\node[Square] at (23.450000, 31.800000) {};
\node[Kanji] at (23.450000, 32.300000) {\textcolor[HTML]{14418e}{導}};
\node[Onyomi] at (23.500000, 31.900000) {ドウ};
\node[Kunyomi] at (23.400000, 31.900000) {みちび.く};
\node[Meaning] at (23.450000, 33.550000) {lead};
\node[Square] at (25.500000, 31.800000) {};
\node[Kanji] at (25.500000, 32.300000) {\textcolor[HTML]{0e254c}{迅}};
\node[Onyomi] at (25.550000, 31.900000) {ジン};
\node[Meaning] at (25.500000, 33.550000) {swift};
\node[Square] at (27.550000, 31.800000) {};
\node[Kanji] at (27.550000, 32.300000) {\textcolor[HTML]{14469c}{造}};
\node[Onyomi] at (27.600000, 31.900000) {ゾウ};
\node[Kunyomi] at (27.500000, 31.900000) {つく.る};
\node[Meaning] at (27.550000, 33.550000) {create};
\node[Square] at (29.600000, 31.800000) {};
\node[Kanji] at (29.600000, 32.300000) {\textcolor[HTML]{14418e}{迫}};
\node[Onyomi] at (29.650000, 31.900000) {ハク};
\node[Kunyomi] at (29.550000, 31.900000) {せま.る};
\node[Meaning] at (29.600000, 33.550000) {urge};
\node[Square] at (31.650000, 31.800000) {};
\node[Kanji] at (31.650000, 32.300000) {\textcolor[HTML]{1551b8}{逃}};
\node[Onyomi] at (31.700000, 31.900000) {トウ};
\node[Kunyomi] at (31.600000, 31.900000) {に.げる};
\node[Meaning] at (31.650000, 33.550000) {escape};
\node[Square] at (33.700000, 31.800000) {};
\node[Kanji] at (33.700000, 32.300000) {\textcolor[HTML]{14418e}{辺}};
\node[Onyomi] at (33.750000, 31.900000) {ヘン};
\node[Kunyomi] at (33.650000, 31.900000) {あた.り};
\node[Meaning] at (33.700000, 33.550000) {area};
\node[Square] at (35.750000, 31.800000) {};
\node[Kanji] at (35.750000, 32.300000) {\textcolor[HTML]{133c80}{巡}};
\node[Onyomi] at (35.800000, 31.900000) {ジュン};
\node[Kunyomi] at (35.700000, 31.900000) {めぐ.る};
\node[Meaning] at (35.750000, 33.550000) {patrol};
\node[Square] at (37.800000, 31.800000) {};
\node[Kanji] at (37.800000, 32.300000) {\textcolor[HTML]{145cd5}{車}};
\node[Onyomi] at (37.850000, 31.900000) {シャ};
\node[Kunyomi] at (37.750000, 31.900000) {くるま};
\node[Meaning] at (37.800000, 33.550000) {car};
\node[Square] at (39.850000, 31.800000) {};
\node[Kanji] at (39.850000, 32.300000) {\textcolor[HTML]{1557c6}{連}};
\node[Onyomi] at (39.900000, 31.900000) {レン};
\node[Kunyomi] at (39.800000, 31.900000) {つ};
\node[Meaning] at (39.850000, 33.550000) {take along};
\node[Square] at (41.900000, 31.800000) {};
\node[Kanji] at (41.900000, 32.300000) {\textcolor[HTML]{102b59}{軌}};
\node[Onyomi] at (41.950000, 31.900000) {キ};
\node[Meaning] at (41.900000, 33.550000) {rut};
\node[Square] at (43.950000, 31.800000) {};
\node[Kanji] at (43.950000, 32.300000) {\textcolor[HTML]{14469c}{輸}};
\node[Onyomi] at (44.000000, 31.900000) {ユ};
\node[Meaning] at (43.950000, 33.550000) {transport};
\node[Square] at (46.000000, 31.800000) {};
\node[Kanji] at (46.000000, 32.300000) {\textcolor[HTML]{1968ed}{前}};
\node[Onyomi] at (46.050000, 31.900000) {ゼン};
\node[Kunyomi] at (45.950000, 31.900000) {まえ};
\node[Meaning] at (46.000000, 33.550000) {front};
\node[Square] at (48.050000, 31.800000) {};
\node[Kanji] at (48.050000, 32.300000) {\textcolor[HTML]{133c80}{各}};
\node[Onyomi] at (48.100000, 31.900000) {カク};
\node[Kunyomi] at (48.000000, 31.900000) {おの};
\node[Meaning] at (48.050000, 33.550000) {each};
\node[Square] at (50.100000, 31.800000) {};
\node[Kanji] at (50.100000, 32.300000) {\textcolor[HTML]{154caa}{格}};
\node[Onyomi] at (50.150000, 31.900000) {カク};
\node[Meaning] at (50.100000, 33.550000) {status};
\node[Square] at (52.150000, 31.800000) {};
\node[Kanji] at (52.150000, 32.300000) {\textcolor[HTML]{113066}{略}};
\node[Onyomi] at (52.200000, 31.900000) {リャク};
\node[Kunyomi] at (52.100000, 31.900000) {りゃく.す};
\node[Meaning] at (52.150000, 33.550000) {abbreviation};
\node[Square] at (54.200000, 31.800000) {};
\node[Kanji] at (54.200000, 32.300000) {\textcolor[HTML]{1551b8}{客}};
\node[Onyomi] at (54.250000, 31.900000) {キャク};
\node[Meaning] at (54.200000, 33.550000) {guest};
\node[Square] at (56.250000, 31.800000) {};
\node[Kanji] at (56.250000, 32.300000) {\textcolor[HTML]{154caa}{額}};
\node[Onyomi] at (56.300000, 31.900000) {ガク};
\node[Kunyomi] at (56.200000, 31.900000) {ひたい};
\node[Meaning] at (56.250000, 33.550000) {amount};
\node[Meaning] at (-58.500000, 32.350000) {23.58\%};
\node[Square] at (-56.500000, 29.750000) {};
\node[Kanji] at (-56.500000, 30.250000) {\textcolor[HTML]{154caa}{夏}};
\node[Onyomi] at (-56.450000, 29.850000) {ゲ};
\node[Kunyomi] at (-56.550000, 29.850000) {なつ};
\node[Meaning] at (-56.500000, 31.500000) {summer};
\node[Square] at (-54.450000, 29.750000) {};
\node[Kanji] at (-54.450000, 30.250000) {\textcolor[HTML]{14469c}{処}};
\node[Onyomi] at (-54.400000, 29.850000) {ショ};
\node[Meaning] at (-54.450000, 31.500000) {deal with};
\node[Square] at (-52.400000, 29.750000) {};
\node[Kanji] at (-52.400000, 30.250000) {\textcolor[HTML]{133c80}{条}};
\node[Onyomi] at (-52.350000, 29.850000) {ジョウ};
\node[Meaning] at (-52.400000, 31.500000) {clause};
\node[Square] at (-50.350000, 29.750000) {};
\node[Kanji] at (-50.350000, 30.250000) {\textcolor[HTML]{145cd5}{落}};
\node[Onyomi] at (-50.300000, 29.850000) {ラク};
\node[Kunyomi] at (-50.400000, 29.850000) {お.ちる};
\node[Meaning] at (-50.350000, 31.500000) {fall};
\node[Square] at (-48.300000, 29.750000) {};
\node[Kanji] at (-48.300000, 30.250000) {\textcolor[HTML]{14418e}{冗}};
\node[Onyomi] at (-48.250000, 29.850000) {ジョウ};
\node[Meaning] at (-48.300000, 31.500000) {superfluous};
\node[Square] at (-46.250000, 29.750000) {};
\node[Kanji] at (-46.250000, 30.250000) {\textcolor[HTML]{14469c}{軍}};
\node[Onyomi] at (-46.200000, 29.850000) {グン};
\node[Meaning] at (-46.250000, 31.500000) {army};
\node[Square] at (-44.200000, 29.750000) {};
\node[Kanji] at (-44.200000, 30.250000) {\textcolor[HTML]{154caa}{輝}};
\node[Onyomi] at (-44.150000, 29.850000) {キ};
\node[Kunyomi] at (-44.250000, 29.850000) {かがやき};
\node[Meaning] at (-44.200000, 31.500000) {radiance};
\node[Square] at (-42.150000, 29.750000) {};
\node[Kanji] at (-42.150000, 30.250000) {\textcolor[HTML]{1557c6}{運}};
\node[Onyomi] at (-42.100000, 29.850000) {ウン};
\node[Kunyomi] at (-42.200000, 29.850000) {はこ.ぶ};
\node[Meaning] at (-42.150000, 31.500000) {carry};
\node[Square] at (-40.100000, 29.750000) {};
\node[Kanji] at (-40.100000, 30.250000) {\textcolor[HTML]{102b59}{冠}};
\node[Onyomi] at (-40.050000, 29.850000) {カン};
\node[Kunyomi] at (-40.150000, 29.850000) {かんむり};
\node[Meaning] at (-40.100000, 31.500000) {crown};
\node[Square] at (-38.050000, 29.750000) {};
\node[Kanji] at (-38.050000, 30.250000) {\textcolor[HTML]{1551b8}{夢}};
\node[Onyomi] at (-38.000000, 29.850000) {ム};
\node[Kunyomi] at (-38.100000, 29.850000) {ゆめ};
\node[Meaning] at (-38.050000, 31.500000) {dream};
\node[Square] at (-36.000000, 29.750000) {};
\node[Kanji] at (-36.000000, 30.250000) {\textcolor[HTML]{0e254c}{坑}};
\node[Onyomi] at (-35.950000, 29.850000) {コウ};
\node[Meaning] at (-36.000000, 31.500000) {pit};
\node[Square] at (-33.950000, 29.750000) {};
\node[Kanji] at (-33.950000, 30.250000) {\textcolor[HTML]{1461e3}{高}};
\node[Onyomi] at (-33.900000, 29.850000) {コウ};
\node[Kunyomi] at (-34.000000, 29.850000) {たか.い};
\node[Meaning] at (-33.950000, 31.500000) {tall};
\node[Square] at (-31.900000, 29.750000) {};
\node[Kanji] at (-31.900000, 30.250000) {\textcolor[HTML]{0e254c}{享}};
\node[Onyomi] at (-31.850000, 29.850000) {キョウ};
\node[Kunyomi] at (-31.950000, 29.850000) {う};
\node[Meaning] at (-31.900000, 31.500000) {receive};
\node[Square] at (-29.850000, 29.750000) {};
\node[Kanji] at (-29.850000, 30.250000) {\textcolor[HTML]{0e254c}{塾}};
\node[Onyomi] at (-29.800000, 29.850000) {ジュク};
\node[Meaning] at (-29.850000, 31.500000) {cram school};
\node[Square] at (-27.800000, 29.750000) {};
\node[Kanji] at (-27.800000, 30.250000) {\textcolor[HTML]{123673}{熟}};
\node[Onyomi] at (-27.750000, 29.850000) {ジュク};
\node[Kunyomi] at (-27.850000, 29.850000) {う.れる};
\node[Meaning] at (-27.800000, 31.500000) {ripen};
\node[Square] at (-25.750000, 29.750000) {};
\node[Kanji] at (-25.750000, 30.250000) {\textcolor[HTML]{102b59}{亭}};
\node[Onyomi] at (-25.700000, 29.850000) {テイ};
\node[Meaning] at (-25.750000, 31.500000) {restaurant};
\node[Square] at (-23.700000, 29.750000) {};
\node[Kanji] at (-23.700000, 30.250000) {\textcolor[HTML]{145cd5}{京}};
\node[Onyomi] at (-23.650000, 29.850000) {キョウ};
\node[Kunyomi] at (-23.750000, 29.850000) {みやこ};
\node[Meaning] at (-23.700000, 31.500000) {capital};
\node[Square] at (-21.650000, 29.750000) {};
\node[Kanji] at (-21.650000, 30.250000) {\textcolor[HTML]{123673}{涼}};
\node[Onyomi] at (-21.600000, 29.850000) {リョウ};
\node[Kunyomi] at (-21.700000, 29.850000) {すず.しい};
\node[Meaning] at (-21.650000, 31.500000) {cool};
\node[Square] at (-19.600000, 29.750000) {};
\node[Kanji] at (-19.600000, 30.250000) {\textcolor[HTML]{14469c}{景}};
\node[Onyomi] at (-19.550000, 29.850000) {ケイ};
\node[Meaning] at (-19.600000, 31.500000) {scene};
\node[Square] at (-17.550000, 29.750000) {};
\node[Kanji] at (-17.550000, 30.250000) {\textcolor[HTML]{102b59}{鯨}};
\node[Onyomi] at (-17.500000, 29.850000) {ゲイ};
\node[Kunyomi] at (-17.600000, 29.850000) {くじら};
\node[Meaning] at (-17.550000, 31.500000) {whale};
\node[Square] at (-15.500000, 29.750000) {};
\node[Kanji] at (-15.500000, 30.250000) {\textcolor[HTML]{133c80}{舎}};
\node[Onyomi] at (-15.450000, 29.850000) {シャ};
\node[Meaning] at (-15.500000, 31.500000) {cottage};
\node[Square] at (-13.450000, 29.750000) {};
\node[Kanji] at (-13.450000, 30.250000) {\textcolor[HTML]{1557c6}{周}};
\node[Onyomi] at (-13.400000, 29.850000) {シュウ};
\node[Kunyomi] at (-13.500000, 29.850000) {まわ.り};
\node[Meaning] at (-13.450000, 31.500000) {circumference};
\node[Square] at (-11.400000, 29.750000) {};
\node[Kanji] at (-11.400000, 30.250000) {\textcolor[HTML]{1557c6}{週}};
\node[Onyomi] at (-11.350000, 29.850000) {シュウ};
\node[Meaning] at (-11.400000, 31.500000) {week};
\node[Square] at (-9.350000, 29.750000) {};
\node[Kanji] at (-9.350000, 30.250000) {\textcolor[HTML]{1551b8}{士}};
\node[Onyomi] at (-9.300000, 29.850000) {シ};
\node[Kunyomi] at (-9.400000, 29.850000) {さむらい};
\node[Meaning] at (-9.350000, 31.500000) {samurai};
\node[Square] at (-7.300000, 29.750000) {};
\node[Kanji] at (-7.300000, 30.250000) {\textcolor[HTML]{14418e}{吉}};
\node[Onyomi] at (-7.250000, 29.850000) {キツ};
\node[Kunyomi] at (-7.350000, 29.850000) {よし};
\node[Meaning] at (-7.300000, 31.500000) {good luck};
\node[Square] at (-5.250000, 29.750000) {};
\node[Kanji] at (-5.250000, 30.250000) {\textcolor[HTML]{123673}{壮}};
\node[Onyomi] at (-5.200000, 29.850000) {ソウ};
\node[Meaning] at (-5.250000, 31.500000) {robust};
\node[Square] at (-3.200000, 29.750000) {};
\node[Kanji] at (-3.200000, 30.250000) {\textcolor[HTML]{0e254c}{荘}};
\node[Onyomi] at (-3.150000, 29.850000) {ソウ};
\node[Kunyomi] at (-3.250000, 29.850000) {あごそ};
\node[Meaning] at (-3.200000, 31.500000) {villa};
\node[Square] at (-1.150000, 29.750000) {};
\node[Kanji] at (-1.150000, 30.250000) {\textcolor[HTML]{1557c6}{売}};
\node[Onyomi] at (-1.100000, 29.850000) {バイ};
\node[Kunyomi] at (-1.200000, 29.850000) {う};
\node[Meaning] at (-1.150000, 31.500000) {sell};
\node[Square] at (0.900000, 29.750000) {};
\node[Kanji] at (0.900000, 30.250000) {\textcolor[HTML]{1461e3}{学}};
\node[Onyomi] at (0.950000, 29.850000) {ガク};
\node[Kunyomi] at (0.850000, 29.850000) {まな.ぶ};
\node[Meaning] at (0.900000, 31.500000) {study};
\node[Square] at (2.950000, 29.750000) {};
\node[Kanji] at (2.950000, 30.250000) {\textcolor[HTML]{1551b8}{覚}};
\node[Onyomi] at (3.000000, 29.850000) {カク};
\node[Kunyomi] at (2.900000, 29.850000) {おぼ};
\node[Meaning] at (2.950000, 31.500000) {memorize};
\node[Square] at (5.000000, 29.750000) {};
\node[Kanji] at (5.000000, 30.250000) {\textcolor[HTML]{133c80}{栄}};
\node[Onyomi] at (5.050000, 29.850000) {エイ};
\node[Kunyomi] at (4.950000, 29.850000) {さか.える};
\node[Meaning] at (5.000000, 31.500000) {prosperity};
\node[Square] at (7.050000, 29.750000) {};
\node[Kanji] at (7.050000, 30.250000) {\textcolor[HTML]{145cd5}{書}};
\node[Onyomi] at (7.100000, 29.850000) {ショ};
\node[Kunyomi] at (7.000000, 29.850000) {か.く};
\node[Meaning] at (7.050000, 31.500000) {write};
\node[Square] at (9.100000, 29.750000) {};
\node[Kanji] at (9.100000, 30.250000) {\textcolor[HTML]{154caa}{津}};
\node[Onyomi] at (9.150000, 29.850000) {シン};
\node[Kunyomi] at (9.050000, 29.850000) {つ};
\node[Meaning] at (9.100000, 31.500000) {haven};
\node[Square] at (11.150000, 29.750000) {};
\node[Kanji] at (11.150000, 30.250000) {\textcolor[HTML]{113066}{牧}};
\node[Onyomi] at (11.200000, 29.850000) {ボク};
\node[Kunyomi] at (11.100000, 29.850000) {まき};
\node[Meaning] at (11.150000, 31.500000) {pasture};
\node[Square] at (13.200000, 29.750000) {};
\node[Kanji] at (13.200000, 30.250000) {\textcolor[HTML]{14469c}{攻}};
\node[Onyomi] at (13.250000, 29.850000) {コウ};
\node[Kunyomi] at (13.150000, 29.850000) {せ.める};
\node[Meaning] at (13.200000, 31.500000) {aggression};
\node[Square] at (15.250000, 29.750000) {};
\node[Kanji] at (15.250000, 30.250000) {\textcolor[HTML]{14418e}{敗}};
\node[Onyomi] at (15.300000, 29.850000) {ハイ};
\node[Kunyomi] at (15.200000, 29.850000) {やぶ.れる};
\node[Meaning] at (15.250000, 31.500000) {failure};
\node[Square] at (17.300000, 29.750000) {};
\node[Kanji] at (17.300000, 30.250000) {\textcolor[HTML]{154caa}{枚}};
\node[Onyomi] at (17.350000, 29.850000) {マイ};
\node[Meaning] at (17.300000, 31.500000) {counter: sheets};
\node[Square] at (19.350000, 29.750000) {};
\node[Kanji] at (19.350000, 30.250000) {\textcolor[HTML]{1551b8}{故}};
\node[Onyomi] at (19.400000, 29.850000) {コ};
\node[Kunyomi] at (19.300000, 29.850000) {ゆえ};
\node[Meaning] at (19.350000, 31.500000) {circumstance};
\node[Square] at (21.400000, 29.750000) {};
\node[Kanji] at (21.400000, 30.250000) {\textcolor[HTML]{14418e}{敬}};
\node[Onyomi] at (21.450000, 29.850000) {ケイ};
\node[Kunyomi] at (21.350000, 29.850000) {うやま.う};
\node[Meaning] at (21.400000, 31.500000) {respect};
\node[Square] at (23.450000, 29.750000) {};
\node[Kanji] at (23.450000, 30.250000) {\textcolor[HTML]{3178f2}{言}};
\node[Onyomi] at (23.500000, 29.850000) {ゲン};
\node[Kunyomi] at (23.400000, 29.850000) {い.う};
\node[Meaning] at (23.450000, 31.500000) {say};
\node[Square] at (25.500000, 29.750000) {};
\node[Kanji] at (25.500000, 30.250000) {\textcolor[HTML]{1551b8}{警}};
\node[Onyomi] at (25.550000, 29.850000) {ケイ};
\node[Meaning] at (25.500000, 31.500000) {warn};
\node[Square] at (27.550000, 29.750000) {};
\node[Kanji] at (27.550000, 30.250000) {\textcolor[HTML]{1557c6}{計}};
\node[Onyomi] at (27.600000, 29.850000) {ケイ};
\node[Kunyomi] at (27.500000, 29.850000) {はか.る};
\node[Meaning] at (27.550000, 31.500000) {measure};
\node[Square] at (29.600000, 29.750000) {};
\node[Kanji] at (29.600000, 30.250000) {\textcolor[HTML]{133c80}{獄}};
\node[Onyomi] at (29.650000, 29.850000) {ゴク};
\node[Meaning] at (29.600000, 31.500000) {prison};
\node[Square] at (31.650000, 29.750000) {};
\node[Kanji] at (31.650000, 30.250000) {\textcolor[HTML]{113066}{訂}};
\node[Onyomi] at (31.700000, 29.850000) {テイ};
\node[Meaning] at (31.650000, 31.500000) {revise};
\node[Square] at (33.700000, 29.750000) {};
\node[Kanji] at (33.700000, 30.250000) {\textcolor[HTML]{113066}{討}};
\node[Onyomi] at (33.750000, 29.850000) {トウ};
\node[Meaning] at (33.700000, 31.500000) {chastise};
\node[Square] at (35.750000, 29.750000) {};
\node[Kanji] at (35.750000, 30.250000) {\textcolor[HTML]{14418e}{訓}};
\node[Onyomi] at (35.800000, 29.850000) {クン};
\node[Kunyomi] at (35.700000, 29.850000) {よ.む};
\node[Meaning] at (35.750000, 31.500000) {instruction};
\node[Square] at (37.800000, 29.750000) {};
\node[Kanji] at (37.800000, 30.250000) {\textcolor[HTML]{0e254c}{詔}};
\node[Onyomi] at (37.850000, 29.850000) {ショウ};
\node[Kunyomi] at (37.750000, 29.850000) {みことのり};
\node[Meaning] at (37.800000, 31.500000) {imperial edict};
\node[Square] at (39.850000, 29.750000) {};
\node[Kanji] at (39.850000, 30.250000) {\textcolor[HTML]{154caa}{詰}};
\node[Onyomi] at (39.900000, 29.850000) {キツ};
\node[Kunyomi] at (39.800000, 29.850000) {つ};
\node[Meaning] at (39.850000, 31.500000) {stuffed};
\node[Square] at (41.900000, 29.750000) {};
\node[Kanji] at (41.900000, 30.250000) {\textcolor[HTML]{1968ed}{話}};
\node[Onyomi] at (41.950000, 29.850000) {ワ};
\node[Kunyomi] at (41.850000, 29.850000) {はな.す};
\node[Meaning] at (41.900000, 31.500000) {talk};
\node[Square] at (43.950000, 29.750000) {};
\node[Kanji] at (43.950000, 30.250000) {\textcolor[HTML]{0e254c}{詠}};
\node[Onyomi] at (44.000000, 29.850000) {エイ};
\node[Kunyomi] at (43.900000, 29.850000) {よ};
\node[Meaning] at (43.950000, 31.500000) {compose};
\node[Square] at (46.000000, 29.750000) {};
\node[Kanji] at (46.000000, 30.250000) {\textcolor[HTML]{113066}{詩}};
\node[Onyomi] at (46.050000, 29.850000) {シ};
\node[Kunyomi] at (45.950000, 29.850000) {し};
\node[Meaning] at (46.000000, 31.500000) {poem};
\node[Square] at (48.050000, 29.750000) {};
\node[Kanji] at (48.050000, 30.250000) {\textcolor[HTML]{1551b8}{語}};
\node[Onyomi] at (48.100000, 29.850000) {ゴ};
\node[Kunyomi] at (48.000000, 29.850000) {かた.る};
\node[Meaning] at (48.050000, 31.500000) {language};
\node[Square] at (50.100000, 29.750000) {};
\node[Kanji] at (50.100000, 30.250000) {\textcolor[HTML]{1551b8}{読}};
\node[Onyomi] at (50.150000, 29.850000) {トウ};
\node[Kunyomi] at (50.050000, 29.850000) {よ};
\node[Meaning] at (50.100000, 31.500000) {read};
\node[Square] at (52.150000, 29.750000) {};
\node[Kanji] at (52.150000, 30.250000) {\textcolor[HTML]{145cd5}{調}};
\node[Onyomi] at (52.200000, 29.850000) {チョウ};
\node[Kunyomi] at (52.100000, 29.850000) {しら.べる};
\node[Meaning] at (52.150000, 31.500000) {investigate};
\node[Square] at (54.200000, 29.750000) {};
\node[Kanji] at (54.200000, 30.250000) {\textcolor[HTML]{1551b8}{談}};
\node[Onyomi] at (54.250000, 29.850000) {ダン};
\node[Meaning] at (54.200000, 31.500000) {discuss};
\node[Square] at (56.250000, 29.750000) {};
\node[Kanji] at (56.250000, 30.250000) {\textcolor[HTML]{0e254c}{諾}};
\node[Onyomi] at (56.300000, 29.850000) {ダク};
\node[Meaning] at (56.250000, 31.500000) {agreement};
\node[Meaning] at (-58.500000, 30.300000) {28.82\%};
\node[Square] at (-56.500000, 27.700000) {};
\node[Kanji] at (-56.500000, 28.200000) {\textcolor[HTML]{0e254c}{諭}};
\node[Onyomi] at (-56.450000, 27.800000) {ユ};
\node[Kunyomi] at (-56.550000, 27.800000) {さと};
\node[Meaning] at (-56.500000, 29.450000) {admonish};
\node[Square] at (-54.450000, 27.700000) {};
\node[Kanji] at (-54.450000, 28.200000) {\textcolor[HTML]{154caa}{式}};
\node[Onyomi] at (-54.400000, 27.800000) {シキ};
\node[Meaning] at (-54.450000, 29.450000) {ritual};
\node[Square] at (-52.400000, 27.700000) {};
\node[Kanji] at (-52.400000, 28.200000) {\textcolor[HTML]{1557c6}{試}};
\node[Onyomi] at (-52.350000, 27.800000) {シ};
\node[Kunyomi] at (-52.450000, 27.800000) {こころ.みる};
\node[Meaning] at (-52.400000, 29.450000) {try};
\node[Square] at (-50.350000, 27.700000) {};
\node[Kanji] at (-50.350000, 28.200000) {\textcolor[HTML]{0e254c}{弐}};
\node[Onyomi] at (-50.300000, 27.800000) {ニ};
\node[Meaning] at (-50.350000, 29.450000) {II, second};
\node[Square] at (-48.300000, 27.700000) {};
\node[Kanji] at (-48.300000, 28.200000) {\textcolor[HTML]{14418e}{域}};
\node[Onyomi] at (-48.250000, 27.800000) {イキ};
\node[Meaning] at (-48.300000, 29.450000) {region};
\node[Square] at (-46.250000, 27.700000) {};
\node[Kanji] at (-46.250000, 28.200000) {\textcolor[HTML]{0e254c}{賊}};
\node[Onyomi] at (-46.200000, 27.800000) {ゾク};
\node[Meaning] at (-46.250000, 29.450000) {robber};
\node[Square] at (-44.200000, 27.700000) {};
\node[Kanji] at (-44.200000, 28.200000) {\textcolor[HTML]{0e254c}{栽}};
\node[Onyomi] at (-44.150000, 27.800000) {サイ};
\node[Meaning] at (-44.200000, 29.450000) {planting};
\node[Square] at (-42.150000, 27.700000) {};
\node[Kanji] at (-42.150000, 28.200000) {\textcolor[HTML]{14469c}{載}};
\node[Onyomi] at (-42.100000, 27.800000) {サイ};
\node[Kunyomi] at (-42.200000, 27.800000) {の.せる};
\node[Meaning] at (-42.150000, 29.450000) {publish};
\node[Square] at (-40.100000, 27.700000) {};
\node[Kanji] at (-40.100000, 28.200000) {\textcolor[HTML]{133c80}{茂}};
\node[Onyomi] at (-40.050000, 27.800000) {モ};
\node[Kunyomi] at (-40.150000, 27.800000) {しげ.る};
\node[Meaning] at (-40.100000, 29.450000) {luxuriant};
\node[Square] at (-38.050000, 27.700000) {};
\node[Kanji] at (-38.050000, 28.200000) {\textcolor[HTML]{1551b8}{成}};
\node[Onyomi] at (-38.000000, 27.800000) {セイ};
\node[Kunyomi] at (-38.100000, 27.800000) {な.る};
\node[Meaning] at (-38.050000, 29.450000) {become};
\node[Square] at (-36.000000, 27.700000) {};
\node[Kanji] at (-36.000000, 28.200000) {\textcolor[HTML]{1551b8}{城}};
\node[Onyomi] at (-35.950000, 27.800000) {ジョウ};
\node[Kunyomi] at (-36.050000, 27.800000) {しろ};
\node[Meaning] at (-36.000000, 29.450000) {castle};
\node[Square] at (-33.950000, 27.700000) {};
\node[Kanji] at (-33.950000, 28.200000) {\textcolor[HTML]{123673}{誠}};
\node[Onyomi] at (-33.900000, 27.800000) {セイ};
\node[Kunyomi] at (-34.000000, 27.800000) {まこと};
\node[Meaning] at (-33.950000, 29.450000) {sincerity};
\node[Square] at (-31.900000, 27.700000) {};
\node[Kanji] at (-31.900000, 28.200000) {\textcolor[HTML]{133c80}{威}};
\node[Onyomi] at (-31.850000, 27.800000) {イ};
\node[Meaning] at (-31.900000, 29.450000) {majesty};
\node[Square] at (-29.850000, 27.700000) {};
\node[Kanji] at (-29.850000, 28.200000) {\textcolor[HTML]{14418e}{滅}};
\node[Onyomi] at (-29.800000, 27.800000) {メツ};
\node[Kunyomi] at (-29.900000, 27.800000) {ほろ.*};
\node[Meaning] at (-29.850000, 29.450000) {destroy};
\node[Square] at (-27.800000, 27.700000) {};
\node[Kanji] at (-27.800000, 28.200000) {\textcolor[HTML]{154caa}{減}};
\node[Onyomi] at (-27.750000, 27.800000) {ゲン};
\node[Kunyomi] at (-27.850000, 27.800000) {へ.る};
\node[Meaning] at (-27.800000, 29.450000) {decrease};
\node[Square] at (-25.750000, 27.700000) {};
\node[Kanji] at (-25.750000, 28.200000) {\textcolor[HTML]{0e254c}{桟}};
\node[Onyomi] at (-25.700000, 27.800000) {サン};
\node[Kunyomi] at (-25.800000, 27.800000) {かけはし};
\node[Meaning] at (-25.750000, 29.450000) {jetty};
\node[Square] at (-23.700000, 27.700000) {};
\node[Kanji] at (-23.700000, 28.200000) {\textcolor[HTML]{102b59}{銭}};
\node[Onyomi] at (-23.650000, 27.800000) {セン};
\node[Kunyomi] at (-23.750000, 27.800000) {ぜに};
\node[Meaning] at (-23.700000, 29.450000) {coin};
\node[Square] at (-21.650000, 27.700000) {};
\node[Kanji] at (-21.650000, 28.200000) {\textcolor[HTML]{133c80}{浅}};
\node[Onyomi] at (-21.600000, 27.800000) {セン};
\node[Kunyomi] at (-21.700000, 27.800000) {あさ};
\node[Meaning] at (-21.650000, 29.450000) {shallow};
\node[Square] at (-19.600000, 27.700000) {};
\node[Kanji] at (-19.600000, 28.200000) {\textcolor[HTML]{1557c6}{止}};
\node[Onyomi] at (-19.550000, 27.800000) {シ};
\node[Kunyomi] at (-19.650000, 27.800000) {と.まる};
\node[Meaning] at (-19.600000, 29.450000) {stop};
\node[Square] at (-17.550000, 27.700000) {};
\node[Kanji] at (-17.550000, 28.200000) {\textcolor[HTML]{145cd5}{歩}};
\node[Onyomi] at (-17.500000, 27.800000) {ホ};
\node[Kunyomi] at (-17.600000, 27.800000) {ある.く};
\node[Meaning] at (-17.550000, 29.450000) {walk};
\node[Square] at (-15.500000, 27.700000) {};
\node[Kanji] at (-15.500000, 28.200000) {\textcolor[HTML]{102b59}{渉}};
\node[Onyomi] at (-15.450000, 27.800000) {ショウ};
\node[Kunyomi] at (-15.550000, 27.800000) {わた.る};
\node[Meaning] at (-15.500000, 29.450000) {ford};
\node[Square] at (-13.450000, 27.700000) {};
\node[Kanji] at (-13.450000, 28.200000) {\textcolor[HTML]{113066}{頻}};
\node[Onyomi] at (-13.400000, 27.800000) {ヒン};
\node[Kunyomi] at (-13.500000, 27.800000) {しき.りに};
\node[Meaning] at (-13.450000, 29.450000) {frequent};
\node[Square] at (-11.400000, 27.700000) {};
\node[Kanji] at (-11.400000, 28.200000) {\textcolor[HTML]{0e254c}{肯}};
\node[Onyomi] at (-11.350000, 27.800000) {コウ};
\node[Kunyomi] at (-11.450000, 27.800000) {がえんじ.る};
\node[Meaning] at (-11.400000, 29.450000) {agreement};
\node[Square] at (-9.350000, 27.700000) {};
\node[Kanji] at (-9.350000, 28.200000) {\textcolor[HTML]{14418e}{企}};
\node[Onyomi] at (-9.300000, 27.800000) {キ};
\node[Kunyomi] at (-9.400000, 27.800000) {くわだ.てる};
\node[Meaning] at (-9.350000, 29.450000) {plan};
\node[Square] at (-7.300000, 27.700000) {};
\node[Kanji] at (-7.300000, 28.200000) {\textcolor[HTML]{14418e}{歴}};
\node[Onyomi] at (-7.250000, 27.800000) {レキ};
\node[Kunyomi] at (-7.350000, 27.800000) {へ.る};
\node[Meaning] at (-7.300000, 29.450000) {continuation};
\node[Square] at (-5.250000, 27.700000) {};
\node[Kanji] at (-5.250000, 28.200000) {\textcolor[HTML]{14469c}{武}};
\node[Onyomi] at (-5.200000, 27.800000) {ブ};
\node[Kunyomi] at (-5.300000, 27.800000) {たけ};
\node[Meaning] at (-5.250000, 29.450000) {military};
\node[Square] at (-3.200000, 27.700000) {};
\node[Kanji] at (-3.200000, 28.200000) {\textcolor[HTML]{0e254c}{賦}};
\node[Onyomi] at (-3.150000, 27.800000) {フ};
\node[Meaning] at (-3.200000, 29.450000) {levy};
\node[Square] at (-1.150000, 27.700000) {};
\node[Kanji] at (-1.150000, 28.200000) {\textcolor[HTML]{1557c6}{正}};
\node[Onyomi] at (-1.100000, 27.800000) {セイ};
\node[Kunyomi] at (-1.200000, 27.800000) {ただ.しい};
\node[Meaning] at (-1.150000, 29.450000) {correct};
\node[Square] at (0.900000, 27.700000) {};
\node[Kanji] at (0.900000, 28.200000) {\textcolor[HTML]{154caa}{証}};
\node[Onyomi] at (0.950000, 27.800000) {ショウ};
\node[Kunyomi] at (0.850000, 27.800000) {あかし};
\node[Meaning] at (0.900000, 29.450000) {evidence};
\node[Square] at (2.950000, 27.700000) {};
\node[Kanji] at (2.950000, 28.200000) {\textcolor[HTML]{1551b8}{政}};
\node[Onyomi] at (3.000000, 27.800000) {セイ};
\node[Meaning] at (2.950000, 29.450000) {politics};
\node[Square] at (5.000000, 27.700000) {};
\node[Kanji] at (5.000000, 28.200000) {\textcolor[HTML]{1557c6}{定}};
\node[Onyomi] at (5.050000, 27.800000) {テイ};
\node[Kunyomi] at (4.950000, 27.800000) {さだ};
\node[Meaning] at (5.000000, 29.450000) {determine};
\node[Square] at (7.050000, 27.700000) {};
\node[Kanji] at (7.050000, 28.200000) {\textcolor[HTML]{113066}{錠}};
\node[Onyomi] at (7.100000, 27.800000) {ジョウ};
\node[Meaning] at (7.050000, 29.450000) {lock};
\node[Square] at (9.100000, 27.700000) {};
\node[Kanji] at (9.100000, 28.200000) {\textcolor[HTML]{1557c6}{走}};
\node[Onyomi] at (9.150000, 27.800000) {ソウ};
\node[Kunyomi] at (9.050000, 27.800000) {はし.る};
\node[Meaning] at (9.100000, 29.450000) {run};
\node[Square] at (11.150000, 27.700000) {};
\node[Kanji] at (11.150000, 28.200000) {\textcolor[HTML]{14418e}{超}};
\node[Onyomi] at (11.200000, 27.800000) {チョウ};
\node[Kunyomi] at (11.100000, 27.800000) {こ.*};
\node[Meaning] at (11.150000, 29.450000) {ultra};
\node[Square] at (13.200000, 27.700000) {};
\node[Kanji] at (13.200000, 28.200000) {\textcolor[HTML]{0e254c}{赴}};
\node[Onyomi] at (13.250000, 27.800000) {フ};
\node[Kunyomi] at (13.150000, 27.800000) {おもむ};
\node[Meaning] at (13.200000, 29.450000) {proceed};
\node[Square] at (15.250000, 27.700000) {};
\node[Kanji] at (15.250000, 28.200000) {\textcolor[HTML]{154caa}{越}};
\node[Onyomi] at (15.300000, 27.800000) {エツ};
\node[Kunyomi] at (15.200000, 27.800000) {こ.*};
\node[Meaning] at (15.250000, 29.450000) {go beyond};
\node[Square] at (17.300000, 27.700000) {};
\node[Kanji] at (17.300000, 28.200000) {\textcolor[HTML]{102b59}{是}};
\node[Onyomi] at (17.350000, 27.800000) {ゼ};
\node[Meaning] at (17.300000, 29.450000) {absolutely};
\node[Square] at (19.350000, 27.700000) {};
\node[Kanji] at (19.350000, 28.200000) {\textcolor[HTML]{1557c6}{題}};
\node[Onyomi] at (19.400000, 27.800000) {ダイ};
\node[Meaning] at (19.350000, 29.450000) {topic};
\node[Square] at (21.400000, 27.700000) {};
\node[Kanji] at (21.400000, 28.200000) {\textcolor[HTML]{113066}{堤}};
\node[Onyomi] at (21.450000, 27.800000) {テイ};
\node[Kunyomi] at (21.350000, 27.800000) {つつみ};
\node[Meaning] at (21.400000, 29.450000) {embankment};
\node[Square] at (23.450000, 27.700000) {};
\node[Kanji] at (23.450000, 28.200000) {\textcolor[HTML]{1551b8}{建}};
\node[Onyomi] at (23.500000, 27.800000) {ケン};
\node[Kunyomi] at (23.400000, 27.800000) {た.*};
\node[Meaning] at (23.450000, 29.450000) {build};
\node[Square] at (25.500000, 27.700000) {};
\node[Kanji] at (25.500000, 28.200000) {\textcolor[HTML]{133c80}{延}};
\node[Onyomi] at (25.550000, 27.800000) {エン};
\node[Kunyomi] at (25.450000, 27.800000) {のば.す};
\node[Meaning] at (25.500000, 29.450000) {prolong};
\node[Square] at (27.550000, 27.700000) {};
\node[Kanji] at (27.550000, 28.200000) {\textcolor[HTML]{14418e}{誕}};
\node[Onyomi] at (27.600000, 27.800000) {タン};
\node[Meaning] at (27.550000, 29.450000) {birth};
\node[Square] at (29.600000, 27.700000) {};
\node[Kanji] at (29.600000, 28.200000) {\textcolor[HTML]{102b59}{礎}};
\node[Onyomi] at (29.650000, 27.800000) {ソ};
\node[Kunyomi] at (29.550000, 27.800000) {いしずえ};
\node[Meaning] at (29.600000, 29.450000) {foundation};
\node[Square] at (31.650000, 27.700000) {};
\node[Kanji] at (31.650000, 28.200000) {\textcolor[HTML]{0e254c}{婿}};
\node[Onyomi] at (31.700000, 27.800000) {セイ};
\node[Kunyomi] at (31.600000, 27.800000) {むこ};
\node[Meaning] at (31.650000, 29.450000) {groom};
\node[Square] at (33.700000, 27.700000) {};
\node[Kanji] at (33.700000, 28.200000) {\textcolor[HTML]{133c80}{衣}};
\node[Onyomi] at (33.750000, 27.800000) {イ};
\node[Kunyomi] at (33.650000, 27.800000) {ころも};
\node[Meaning] at (33.700000, 29.450000) {clothes};
\node[Square] at (35.750000, 27.700000) {};
\node[Kanji] at (35.750000, 28.200000) {\textcolor[HTML]{14418e}{裁}};
\node[Onyomi] at (35.800000, 27.800000) {サイ};
\node[Kunyomi] at (35.700000, 27.800000) {さば.く};
\node[Meaning] at (35.750000, 29.450000) {judge};
\node[Square] at (37.800000, 27.700000) {};
\node[Kanji] at (37.800000, 28.200000) {\textcolor[HTML]{14469c}{装}};
\node[Onyomi] at (37.850000, 27.800000) {ソウ};
\node[Kunyomi] at (37.750000, 27.800000) {よそお.う};
\node[Meaning] at (37.800000, 29.450000) {attire};
\node[Square] at (39.850000, 27.700000) {};
\node[Kanji] at (39.850000, 28.200000) {\textcolor[HTML]{154caa}{裏}};
\node[Onyomi] at (39.900000, 27.800000) {リ};
\node[Kunyomi] at (39.800000, 27.800000) {うら};
\node[Meaning] at (39.850000, 29.450000) {backside};
\node[Square] at (41.900000, 27.700000) {};
\node[Kanji] at (41.900000, 28.200000) {\textcolor[HTML]{154caa}{壊}};
\node[Onyomi] at (41.950000, 27.800000) {カイ};
\node[Kunyomi] at (41.850000, 27.800000) {こわ.*};
\node[Meaning] at (41.900000, 29.450000) {break};
\node[Square] at (43.950000, 27.700000) {};
\node[Kanji] at (43.950000, 28.200000) {\textcolor[HTML]{14418e}{哀}};
\node[Onyomi] at (44.000000, 27.800000) {アイ};
\node[Kunyomi] at (43.900000, 27.800000) {あわ.れ*};
\node[Meaning] at (43.950000, 29.450000) {pathetic};
\node[Square] at (46.000000, 27.700000) {};
\node[Kanji] at (46.000000, 28.200000) {\textcolor[HTML]{1551b8}{遠}};
\node[Onyomi] at (46.050000, 27.800000) {エン};
\node[Kunyomi] at (45.950000, 27.800000) {とお};
\node[Meaning] at (46.000000, 29.450000) {far};
\node[Square] at (48.050000, 27.700000) {};
\node[Kanji] at (48.050000, 28.200000) {\textcolor[HTML]{123673}{猿}};
\node[Onyomi] at (48.100000, 27.800000) {エン};
\node[Kunyomi] at (48.000000, 27.800000) {さる};
\node[Meaning] at (48.050000, 29.450000) {monkey};
\node[Square] at (50.100000, 27.700000) {};
\node[Kanji] at (50.100000, 28.200000) {\textcolor[HTML]{145cd5}{初}};
\node[Onyomi] at (50.150000, 27.800000) {ショ};
\node[Kunyomi] at (50.050000, 27.800000) {はじ};
\node[Meaning] at (50.100000, 29.450000) {first};
\node[Square] at (52.150000, 27.700000) {};
\node[Kanji] at (52.150000, 28.200000) {\textcolor[HTML]{14469c}{布}};
\node[Onyomi] at (52.200000, 27.800000) {フ};
\node[Kunyomi] at (52.100000, 27.800000) {ぬの};
\node[Meaning] at (52.150000, 29.450000) {cloth};
\node[Square] at (54.200000, 27.700000) {};
\node[Kanji] at (54.200000, 28.200000) {\textcolor[HTML]{0e254c}{帆}};
\node[Onyomi] at (54.250000, 27.800000) {ハン};
\node[Kunyomi] at (54.150000, 27.800000) {ほ};
\node[Meaning] at (54.200000, 29.450000) {sail};
\node[Square] at (56.250000, 27.700000) {};
\node[Kanji] at (56.250000, 28.200000) {\textcolor[HTML]{123673}{幅}};
\node[Onyomi] at (56.300000, 27.800000) {フク};
\node[Kunyomi] at (56.200000, 27.800000) {はば};
\node[Meaning] at (56.250000, 29.450000) {width};
\node[Meaning] at (-58.500000, 28.250000) {30.62\%};
\node[Square] at (-56.500000, 25.650000) {};
\node[Kanji] at (-56.500000, 26.150000) {\textcolor[HTML]{154caa}{帽}};
\node[Onyomi] at (-56.450000, 25.750000) {ボウ};
\node[Meaning] at (-56.500000, 27.400000) {hat};
\node[Square] at (-54.450000, 25.650000) {};
\node[Kanji] at (-54.450000, 26.150000) {\textcolor[HTML]{123673}{幕}};
\node[Onyomi] at (-54.400000, 25.750000) {マク};
\node[Kunyomi] at (-54.500000, 25.750000) {とばり};
\node[Meaning] at (-54.450000, 27.400000) {curtain};
\node[Square] at (-52.400000, 25.650000) {};
\node[Kanji] at (-52.400000, 26.150000) {\textcolor[HTML]{123673}{錦}};
\node[Onyomi] at (-52.350000, 25.750000) {キン};
\node[Kunyomi] at (-52.450000, 25.750000) {にしき};
\node[Meaning] at (-52.400000, 27.400000) {brocade};
\node[Square] at (-50.350000, 25.650000) {};
\node[Kanji] at (-50.350000, 26.150000) {\textcolor[HTML]{1557c6}{市}};
\node[Onyomi] at (-50.300000, 25.750000) {シ};
\node[Kunyomi] at (-50.400000, 25.750000) {いち};
\node[Meaning] at (-50.350000, 27.400000) {city};
\node[Square] at (-48.300000, 25.650000) {};
\node[Kanji] at (-48.300000, 26.150000) {\textcolor[HTML]{14418e}{姉}};
\node[Onyomi] at (-48.250000, 25.750000) {シ};
\node[Kunyomi] at (-48.350000, 25.750000) {お.ねえ};
\node[Meaning] at (-48.300000, 27.400000) {older sister};
\node[Square] at (-46.250000, 25.650000) {};
\node[Kanji] at (-46.250000, 26.150000) {\textcolor[HTML]{133c80}{肺}};
\node[Onyomi] at (-46.200000, 25.750000) {ハイ};
\node[Kunyomi] at (-46.300000, 25.750000) {はい};
\node[Meaning] at (-46.250000, 27.400000) {lung};
\node[Square] at (-44.200000, 25.650000) {};
\node[Kanji] at (-44.200000, 26.150000) {\textcolor[HTML]{154caa}{帯}};
\node[Onyomi] at (-44.150000, 25.750000) {タイ};
\node[Kunyomi] at (-44.250000, 25.750000) {おび};
\node[Meaning] at (-44.200000, 27.400000) {belt};
\node[Square] at (-42.150000, 25.650000) {};
\node[Kanji] at (-42.150000, 26.150000) {\textcolor[HTML]{133c80}{滞}};
\node[Onyomi] at (-42.100000, 25.750000) {タイ};
\node[Kunyomi] at (-42.200000, 25.750000) {とどこお.る};
\node[Meaning] at (-42.150000, 27.400000) {stagnate};
\node[Square] at (-40.100000, 25.650000) {};
\node[Kanji] at (-40.100000, 26.150000) {\textcolor[HTML]{14469c}{刺}};
\node[Onyomi] at (-40.050000, 25.750000) {シ};
\node[Kunyomi] at (-40.150000, 25.750000) {さ.*};
\node[Meaning] at (-40.100000, 27.400000) {stab};
\node[Square] at (-38.050000, 25.650000) {};
\node[Kanji] at (-38.050000, 26.150000) {\textcolor[HTML]{154caa}{制}};
\node[Onyomi] at (-38.000000, 25.750000) {セイ};
\node[Meaning] at (-38.050000, 27.400000) {control};
\node[Square] at (-36.000000, 25.650000) {};
\node[Kanji] at (-36.000000, 26.150000) {\textcolor[HTML]{154caa}{製}};
\node[Onyomi] at (-35.950000, 25.750000) {セイ};
\node[Meaning] at (-36.000000, 27.400000) {manufacture};
\node[Square] at (-33.950000, 25.650000) {};
\node[Kanji] at (-33.950000, 26.150000) {\textcolor[HTML]{1557c6}{転}};
\node[Onyomi] at (-33.900000, 25.750000) {テン};
\node[Kunyomi] at (-34.000000, 25.750000) {ころ.ぶ};
\node[Meaning] at (-33.950000, 27.400000) {revolve};
\node[Square] at (-31.900000, 25.650000) {};
\node[Kanji] at (-31.900000, 26.150000) {\textcolor[HTML]{123673}{芸}};
\node[Onyomi] at (-31.850000, 25.750000) {ゲイ};
\node[Meaning] at (-31.900000, 27.400000) {acting};
\node[Square] at (-29.850000, 25.650000) {};
\node[Kanji] at (-29.850000, 26.150000) {\textcolor[HTML]{154caa}{雨}};
\node[Onyomi] at (-29.800000, 25.750000) {ウ};
\node[Kunyomi] at (-29.900000, 25.750000) {あめ};
\node[Meaning] at (-29.850000, 27.400000) {rain};
\node[Square] at (-27.800000, 25.650000) {};
\node[Kanji] at (-27.800000, 26.150000) {\textcolor[HTML]{14418e}{雲}};
\node[Onyomi] at (-27.750000, 25.750000) {ウン};
\node[Kunyomi] at (-27.850000, 25.750000) {くも};
\node[Meaning] at (-27.800000, 27.400000) {cloud};
\node[Square] at (-25.750000, 25.650000) {};
\node[Kanji] at (-25.750000, 26.150000) {\textcolor[HTML]{133c80}{曇}};
\node[Kunyomi] at (-25.800000, 25.750000) {くも};
\node[Meaning] at (-25.750000, 27.400000) {cloudy};
\node[Square] at (-23.700000, 25.650000) {};
\node[Kanji] at (-23.700000, 26.150000) {\textcolor[HTML]{133c80}{雷}};
\node[Onyomi] at (-23.650000, 25.750000) {ライ};
\node[Kunyomi] at (-23.750000, 25.750000) {かみなり};
\node[Meaning] at (-23.700000, 27.400000) {thunder};
\node[Square] at (-21.650000, 25.650000) {};
\node[Kanji] at (-21.650000, 26.150000) {\textcolor[HTML]{102b59}{霜}};
\node[Kunyomi] at (-21.700000, 25.750000) {しも};
\node[Meaning] at (-21.650000, 27.400000) {frost};
\node[Square] at (-19.600000, 25.650000) {};
\node[Kanji] at (-19.600000, 26.150000) {\textcolor[HTML]{14418e}{冬}};
\node[Onyomi] at (-19.550000, 25.750000) {トウ};
\node[Kunyomi] at (-19.650000, 25.750000) {ふゆ};
\node[Meaning] at (-19.600000, 27.400000) {winter};
\node[Square] at (-17.550000, 25.650000) {};
\node[Kanji] at (-17.550000, 26.150000) {\textcolor[HTML]{1551b8}{天}};
\node[Onyomi] at (-17.500000, 25.750000) {テン};
\node[Kunyomi] at (-17.600000, 25.750000) {あま};
\node[Meaning] at (-17.550000, 27.400000) {heaven};
\node[Square] at (-15.500000, 25.650000) {};
\node[Kanji] at (-15.500000, 26.150000) {\textcolor[HTML]{14469c}{橋}};
\node[Onyomi] at (-15.450000, 25.750000) {キョウ};
\node[Kunyomi] at (-15.550000, 25.750000) {はし};
\node[Meaning] at (-15.500000, 27.400000) {bridge};
\node[Square] at (-13.450000, 25.650000) {};
\node[Kanji] at (-13.450000, 26.150000) {\textcolor[HTML]{1461e3}{立}};
\node[Onyomi] at (-13.400000, 25.750000) {リツ};
\node[Kunyomi] at (-13.500000, 25.750000) {た.つ};
\node[Meaning] at (-13.450000, 27.400000) {stand};
\node[Square] at (-11.400000, 25.650000) {};
\node[Kanji] at (-11.400000, 26.150000) {\textcolor[HTML]{154caa}{泣}};
\node[Onyomi] at (-11.350000, 25.750000) {キュウ};
\node[Kunyomi] at (-11.450000, 25.750000) {な};
\node[Meaning] at (-11.400000, 27.400000) {cry};
\node[Square] at (-9.350000, 25.650000) {};
\node[Kanji] at (-9.350000, 26.150000) {\textcolor[HTML]{1557c6}{章}};
\node[Onyomi] at (-9.300000, 25.750000) {ショウ};
\node[Meaning] at (-9.350000, 27.400000) {chapter};
\node[Square] at (-7.300000, 25.650000) {};
\node[Kanji] at (-7.300000, 26.150000) {\textcolor[HTML]{154caa}{競}};
\node[Onyomi] at (-7.250000, 25.750000) {キョウ};
\node[Kunyomi] at (-7.350000, 25.750000) {きそ.*};
\node[Meaning] at (-7.300000, 27.400000) {compete};
\node[Square] at (-5.250000, 25.650000) {};
\node[Kanji] at (-5.250000, 26.150000) {\textcolor[HTML]{14469c}{帝}};
\node[Onyomi] at (-5.200000, 25.750000) {テイ};
\node[Kunyomi] at (-5.300000, 25.750000) {みかど};
\node[Meaning] at (-5.250000, 27.400000) {sovereign};
\node[Square] at (-3.200000, 25.650000) {};
\node[Kanji] at (-3.200000, 26.150000) {\textcolor[HTML]{123673}{童}};
\node[Onyomi] at (-3.150000, 25.750000) {ドウ};
\node[Meaning] at (-3.200000, 27.400000) {juvenile};
\node[Square] at (-1.150000, 25.650000) {};
\node[Kanji] at (-1.150000, 26.150000) {\textcolor[HTML]{123673}{瞳}};
\node[Onyomi] at (-1.100000, 25.750000) {トウ};
\node[Kunyomi] at (-1.200000, 25.750000) {ひとみ};
\node[Meaning] at (-1.150000, 27.400000) {pupil};
\node[Square] at (0.900000, 25.650000) {};
\node[Kanji] at (0.900000, 26.150000) {\textcolor[HTML]{123673}{鐘}};
\node[Onyomi] at (0.950000, 25.750000) {ショウ};
\node[Kunyomi] at (0.850000, 25.750000) {かね};
\node[Meaning] at (0.900000, 27.400000) {bell};
\node[Square] at (2.950000, 25.650000) {};
\node[Kanji] at (2.950000, 26.150000) {\textcolor[HTML]{14418e}{商}};
\node[Onyomi] at (3.000000, 25.750000) {ショウ};
\node[Kunyomi] at (2.900000, 25.750000) {あきな.い};
\node[Meaning] at (2.950000, 27.400000) {merchandise};
\node[Square] at (5.000000, 25.650000) {};
\node[Kanji] at (5.000000, 26.150000) {\textcolor[HTML]{0e254c}{嫡}};
\node[Onyomi] at (5.050000, 25.750000) {チャク};
\node[Meaning] at (5.000000, 27.400000) {legitimate wife};
\node[Square] at (7.050000, 25.650000) {};
\node[Kanji] at (7.050000, 26.150000) {\textcolor[HTML]{133c80}{適}};
\node[Onyomi] at (7.100000, 25.750000) {テキ};
\node[Meaning] at (7.050000, 27.400000) {suitable};
\node[Square] at (9.100000, 25.650000) {};
\node[Kanji] at (9.100000, 26.150000) {\textcolor[HTML]{133c80}{滴}};
\node[Onyomi] at (9.150000, 25.750000) {テキ};
\node[Kunyomi] at (9.050000, 25.750000) {したた.る};
\node[Meaning] at (9.100000, 27.400000) {drip};
\node[Square] at (11.150000, 25.650000) {};
\node[Kanji] at (11.150000, 26.150000) {\textcolor[HTML]{14469c}{敵}};
\node[Onyomi] at (11.200000, 25.750000) {テキ};
\node[Kunyomi] at (11.100000, 25.750000) {かな.う};
\node[Meaning] at (11.150000, 27.400000) {enemy};
\node[Square] at (13.200000, 25.650000) {};
\node[Kanji] at (13.200000, 26.150000) {\textcolor[HTML]{1551b8}{北}};
\node[Onyomi] at (13.250000, 25.750000) {ホク};
\node[Kunyomi] at (13.150000, 25.750000) {きた};
\node[Meaning] at (13.200000, 27.400000) {north};
\node[Square] at (15.250000, 25.650000) {};
\node[Kanji] at (15.250000, 26.150000) {\textcolor[HTML]{1557c6}{背}};
\node[Onyomi] at (15.300000, 25.750000) {ハイ};
\node[Kunyomi] at (15.200000, 25.750000) {せ};
\node[Meaning] at (15.250000, 27.400000) {back};
\node[Square] at (17.300000, 25.650000) {};
\node[Kanji] at (17.300000, 26.150000) {\textcolor[HTML]{14469c}{比}};
\node[Onyomi] at (17.350000, 25.750000) {ヒ};
\node[Kunyomi] at (17.250000, 25.750000) {くら.べる};
\node[Meaning] at (17.300000, 27.400000) {compare};
\node[Square] at (19.350000, 25.650000) {};
\node[Kanji] at (19.350000, 26.150000) {\textcolor[HTML]{113066}{昆}};
\node[Onyomi] at (19.400000, 25.750000) {コン};
\node[Meaning] at (19.350000, 27.400000) {descendants};
\node[Square] at (21.400000, 25.650000) {};
\node[Kanji] at (21.400000, 26.150000) {\textcolor[HTML]{14418e}{皆}};
\node[Onyomi] at (21.450000, 25.750000) {カイ};
\node[Kunyomi] at (21.350000, 25.750000) {みな};
\node[Meaning] at (21.400000, 27.400000) {all};
\node[Square] at (23.450000, 25.650000) {};
\node[Kanji] at (23.450000, 26.150000) {\textcolor[HTML]{154caa}{混}};
\node[Onyomi] at (23.500000, 25.750000) {コン};
\node[Kunyomi] at (23.400000, 25.750000) {ま.*};
\node[Meaning] at (23.450000, 27.400000) {mix};
\node[Square] at (25.500000, 25.650000) {};
\node[Kanji] at (25.500000, 26.150000) {\textcolor[HTML]{102b59}{渇}};
\node[Onyomi] at (25.550000, 25.750000) {カツ};
\node[Kunyomi] at (25.450000, 25.750000) {かわ};
\node[Meaning] at (25.500000, 27.400000) {thirst};
\node[Square] at (27.550000, 25.650000) {};
\node[Kanji] at (27.550000, 26.150000) {\textcolor[HTML]{0e254c}{謁}};
\node[Onyomi] at (27.600000, 25.750000) {エツ};
\node[Meaning] at (27.550000, 27.400000) {audience};
\node[Square] at (29.600000, 25.650000) {};
\node[Kanji] at (29.600000, 26.150000) {\textcolor[HTML]{102b59}{褐}};
\node[Onyomi] at (29.650000, 25.750000) {カツ};
\node[Meaning] at (29.600000, 27.400000) {brown};
\node[Square] at (31.650000, 25.650000) {};
\node[Kanji] at (31.650000, 26.150000) {\textcolor[HTML]{113066}{喝}};
\node[Onyomi] at (31.700000, 25.750000) {カツ};
\node[Meaning] at (31.650000, 27.400000) {scold};
\node[Square] at (33.700000, 25.650000) {};
\node[Kanji] at (33.700000, 26.150000) {\textcolor[HTML]{0e254c}{旨}};
\node[Onyomi] at (33.750000, 25.750000) {シ};
\node[Kunyomi] at (33.650000, 25.750000) {うま.い};
\node[Meaning] at (33.700000, 27.400000) {point};
\node[Square] at (35.750000, 25.650000) {};
\node[Kanji] at (35.750000, 26.150000) {\textcolor[HTML]{123673}{脂}};
\node[Onyomi] at (35.800000, 25.750000) {シ};
\node[Kunyomi] at (35.700000, 25.750000) {あぶら        };
\node[Meaning] at (35.750000, 27.400000) {fat};
\node[Square] at (37.800000, 25.650000) {};
\node[Kanji] at (37.800000, 26.150000) {\textcolor[HTML]{0e254c}{壱}};
\node[Onyomi] at (37.850000, 25.750000) {イチ};
\node[Meaning] at (37.800000, 27.400000) {1 (legal)};
\node[Square] at (39.850000, 25.650000) {};
\node[Kanji] at (39.850000, 26.150000) {\textcolor[HTML]{1551b8}{毎}};
\node[Onyomi] at (39.900000, 25.750000) {マイ};
\node[Kunyomi] at (39.800000, 25.750000) {ごと};
\node[Meaning] at (39.850000, 27.400000) {every};
\node[Square] at (41.900000, 25.650000) {};
\node[Kanji] at (41.900000, 26.150000) {\textcolor[HTML]{113066}{敏}};
\node[Onyomi] at (41.950000, 25.750000) {ビン};
\node[Meaning] at (41.900000, 27.400000) {alert};
\node[Square] at (43.950000, 25.650000) {};
\node[Kanji] at (43.950000, 26.150000) {\textcolor[HTML]{123673}{梅}};
\node[Onyomi] at (44.000000, 25.750000) {バイ};
\node[Kunyomi] at (43.900000, 25.750000) {うめ};
\node[Meaning] at (43.950000, 27.400000) {ume};
\node[Square] at (46.000000, 25.650000) {};
\node[Kanji] at (46.000000, 26.150000) {\textcolor[HTML]{1557c6}{海}};
\node[Onyomi] at (46.050000, 25.750000) {カイ};
\node[Kunyomi] at (45.950000, 25.750000) {うみ};
\node[Meaning] at (46.000000, 27.400000) {sea};
\node[Square] at (48.050000, 25.650000) {};
\node[Kanji] at (48.050000, 26.150000) {\textcolor[HTML]{0e254c}{乞}};
\node[Kunyomi] at (48.000000, 25.750000) {こ-う};
\node[Meaning] at (48.050000, 27.400000) {beg};
\node[Square] at (50.100000, 25.650000) {};
\node[Kanji] at (50.100000, 26.150000) {\textcolor[HTML]{133c80}{乾}};
\node[Onyomi] at (50.150000, 25.750000) {カン};
\node[Kunyomi] at (50.050000, 25.750000) {かわ};
\node[Meaning] at (50.100000, 27.400000) {dry};
\node[Square] at (52.150000, 25.650000) {};
\node[Kanji] at (52.150000, 26.150000) {\textcolor[HTML]{154caa}{腹}};
\node[Onyomi] at (52.200000, 25.750000) {フク};
\node[Kunyomi] at (52.100000, 25.750000) {はら};
\node[Meaning] at (52.150000, 27.400000) {belly};
\node[Square] at (54.200000, 25.650000) {};
\node[Kanji] at (54.200000, 26.150000) {\textcolor[HTML]{133c80}{複}};
\node[Onyomi] at (54.250000, 25.750000) {フク};
\node[Meaning] at (54.200000, 27.400000) {duplicate};
\node[Square] at (56.250000, 25.650000) {};
\node[Kanji] at (56.250000, 26.150000) {\textcolor[HTML]{14418e}{欠}};
\node[Onyomi] at (56.300000, 25.750000) {ケツ};
\node[Kunyomi] at (56.200000, 25.750000) {か.ける};
\node[Meaning] at (56.250000, 27.400000) {lack};
\node[Meaning] at (-58.500000, 26.200000) {32.36\%};
\node[Square] at (-56.500000, 23.600000) {};
\node[Kanji] at (-56.500000, 24.100000) {\textcolor[HTML]{154caa}{吹}};
\node[Onyomi] at (-56.450000, 23.700000) {スイ};
\node[Kunyomi] at (-56.550000, 23.700000) {ふ};
\node[Meaning] at (-56.500000, 25.350000) {blow};
\node[Square] at (-54.450000, 23.600000) {};
\node[Kanji] at (-54.450000, 24.100000) {\textcolor[HTML]{0e254c}{炊}};
\node[Onyomi] at (-54.400000, 23.700000) {スイ};
\node[Kunyomi] at (-54.500000, 23.700000) {た.く};
\node[Meaning] at (-54.450000, 25.350000) {cook};
\node[Square] at (-52.400000, 23.600000) {};
\node[Kanji] at (-52.400000, 24.100000) {\textcolor[HTML]{154caa}{歌}};
\node[Onyomi] at (-52.350000, 23.700000) {カ};
\node[Kunyomi] at (-52.450000, 23.700000) {うた};
\node[Meaning] at (-52.400000, 25.350000) {song};
\node[Square] at (-50.350000, 23.600000) {};
\node[Kanji] at (-50.350000, 24.100000) {\textcolor[HTML]{113066}{軟}};
\node[Onyomi] at (-50.300000, 23.700000) {ナン};
\node[Kunyomi] at (-50.400000, 23.700000) {やわ};
\node[Meaning] at (-50.350000, 25.350000) {soft};
\node[Square] at (-48.300000, 23.600000) {};
\node[Kanji] at (-48.300000, 24.100000) {\textcolor[HTML]{1557c6}{次}};
\node[Onyomi] at (-48.250000, 23.700000) {ジ};
\node[Kunyomi] at (-48.350000, 23.700000) {つぎ};
\node[Meaning] at (-48.300000, 25.350000) {next};
\node[Square] at (-46.250000, 23.600000) {};
\node[Kanji] at (-46.250000, 24.100000) {\textcolor[HTML]{133c80}{茨}};
\node[Onyomi] at (-46.200000, 23.700000) {シ};
\node[Kunyomi] at (-46.300000, 23.700000) {いばら};
\node[Meaning] at (-46.250000, 25.350000) {briar};
\node[Square] at (-44.200000, 23.600000) {};
\node[Kanji] at (-44.200000, 24.100000) {\textcolor[HTML]{14418e}{資}};
\node[Onyomi] at (-44.150000, 23.700000) {シ};
\node[Meaning] at (-44.200000, 25.350000) {resources};
\node[Square] at (-42.150000, 23.600000) {};
\node[Kanji] at (-42.150000, 24.100000) {\textcolor[HTML]{1557c6}{姿}};
\node[Onyomi] at (-42.100000, 23.700000) {シ};
\node[Kunyomi] at (-42.200000, 23.700000) {すがた};
\node[Meaning] at (-42.150000, 25.350000) {figure};
\node[Square] at (-40.100000, 23.600000) {};
\node[Kanji] at (-40.100000, 24.100000) {\textcolor[HTML]{0e254c}{諮}};
\node[Onyomi] at (-40.050000, 23.700000) {シ};
\node[Kunyomi] at (-40.150000, 23.700000) {はか.る};
\node[Meaning] at (-40.100000, 25.350000) {consult};
\node[Square] at (-38.050000, 23.600000) {};
\node[Kanji] at (-38.050000, 24.100000) {\textcolor[HTML]{0e254c}{賠}};
\node[Onyomi] at (-38.000000, 23.700000) {バイ};
\node[Meaning] at (-38.050000, 25.350000) {compensation};
\node[Square] at (-36.000000, 23.600000) {};
\node[Kanji] at (-36.000000, 24.100000) {\textcolor[HTML]{0e254c}{培}};
\node[Onyomi] at (-35.950000, 23.700000) {バイ};
\node[Kunyomi] at (-36.050000, 23.700000) {つちか.う};
\node[Meaning] at (-36.000000, 25.350000) {cultivate};
\node[Square] at (-33.950000, 23.600000) {};
\node[Kanji] at (-33.950000, 24.100000) {\textcolor[HTML]{0e254c}{剖}};
\node[Onyomi] at (-33.900000, 23.700000) {ボウ};
\node[Meaning] at (-33.950000, 25.350000) {divide};
\node[Square] at (-31.900000, 23.600000) {};
\node[Kanji] at (-31.900000, 24.100000) {\textcolor[HTML]{145cd5}{音}};
\node[Onyomi] at (-31.850000, 23.700000) {オン};
\node[Kunyomi] at (-31.950000, 23.700000) {おと};
\node[Meaning] at (-31.900000, 25.350000) {sound};
\node[Square] at (-29.850000, 23.600000) {};
\node[Kanji] at (-29.850000, 24.100000) {\textcolor[HTML]{1557c6}{暗}};
\node[Onyomi] at (-29.800000, 23.700000) {アン};
\node[Kunyomi] at (-29.900000, 23.700000) {くら.い};
\node[Meaning] at (-29.850000, 25.350000) {dark};
\node[Square] at (-27.800000, 23.600000) {};
\node[Kanji] at (-27.800000, 24.100000) {\textcolor[HTML]{0e254c}{韻}};
\node[Onyomi] at (-27.750000, 23.700000) {イン};
\node[Meaning] at (-27.800000, 25.350000) {rhyme};
\node[Square] at (-25.750000, 23.600000) {};
\node[Kanji] at (-25.750000, 24.100000) {\textcolor[HTML]{14469c}{識}};
\node[Onyomi] at (-25.700000, 23.700000) {シキ};
\node[Meaning] at (-25.750000, 25.350000) {discerning};
\node[Square] at (-23.700000, 23.600000) {};
\node[Kanji] at (-23.700000, 24.100000) {\textcolor[HTML]{154caa}{鏡}};
\node[Onyomi] at (-23.650000, 23.700000) {キョウ};
\node[Kunyomi] at (-23.750000, 23.700000) {かがみ};
\node[Meaning] at (-23.700000, 25.350000) {mirror};
\node[Square] at (-21.650000, 23.600000) {};
\node[Kanji] at (-21.650000, 24.100000) {\textcolor[HTML]{14469c}{境}};
\node[Onyomi] at (-21.600000, 23.700000) {キョウ};
\node[Kunyomi] at (-21.700000, 23.700000) {さかい};
\node[Meaning] at (-21.650000, 25.350000) {boundary};
\node[Square] at (-19.600000, 23.600000) {};
\node[Kanji] at (-19.600000, 24.100000) {\textcolor[HTML]{1551b8}{亡}};
\node[Onyomi] at (-19.550000, 23.700000) {ボウ};
\node[Kunyomi] at (-19.650000, 23.700000) {な.く};
\node[Meaning] at (-19.600000, 25.350000) {deceased};
\node[Square] at (-17.550000, 23.600000) {};
\node[Kanji] at (-17.550000, 24.100000) {\textcolor[HTML]{102b59}{盲}};
\node[Onyomi] at (-17.500000, 23.700000) {モウ};
\node[Kunyomi] at (-17.600000, 23.700000) {めくら};
\node[Meaning] at (-17.550000, 25.350000) {blind};
\node[Square] at (-15.500000, 23.600000) {};
\node[Kanji] at (-15.500000, 24.100000) {\textcolor[HTML]{102b59}{妄}};
\node[Onyomi] at (-15.450000, 23.700000) {モウ};
\node[Kunyomi] at (-15.550000, 23.700000) {みだ};
\node[Meaning] at (-15.500000, 25.350000) {reckless};
\node[Square] at (-13.450000, 23.600000) {};
\node[Kanji] at (-13.450000, 24.100000) {\textcolor[HTML]{14469c}{荒}};
\node[Onyomi] at (-13.400000, 23.700000) {コウ};
\node[Kunyomi] at (-13.500000, 23.700000) {あ};
\node[Meaning] at (-13.450000, 25.350000) {wild};
\node[Square] at (-11.400000, 23.600000) {};
\node[Kanji] at (-11.400000, 24.100000) {\textcolor[HTML]{1551b8}{望}};
\node[Onyomi] at (-11.350000, 23.700000) {ボウ};
\node[Kunyomi] at (-11.450000, 23.700000) {のぞ.む};
\node[Meaning] at (-11.400000, 25.350000) {hope};
\node[Square] at (-9.350000, 23.600000) {};
\node[Kanji] at (-9.350000, 24.100000) {\textcolor[HTML]{1461e3}{方}};
\node[Onyomi] at (-9.300000, 23.700000) {ホウ};
\node[Kunyomi] at (-9.400000, 23.700000) {かた};
\node[Meaning] at (-9.350000, 25.350000) {direction};
\node[Square] at (-7.300000, 23.600000) {};
\node[Kanji] at (-7.300000, 24.100000) {\textcolor[HTML]{123673}{妨}};
\node[Onyomi] at (-7.250000, 23.700000) {ボウ};
\node[Kunyomi] at (-7.350000, 23.700000) {さまた.げる};
\node[Meaning] at (-7.300000, 25.350000) {obstruct};
\node[Square] at (-5.250000, 23.600000) {};
\node[Kanji] at (-5.250000, 24.100000) {\textcolor[HTML]{14469c}{坊}};
\node[Onyomi] at (-5.200000, 23.700000) {ボウ};
\node[Meaning] at (-5.250000, 25.350000) {monk};
\node[Square] at (-3.200000, 23.600000) {};
\node[Kanji] at (-3.200000, 24.100000) {\textcolor[HTML]{0e254c}{芳}};
\node[Onyomi] at (-3.150000, 23.700000) {ホウ};
\node[Kunyomi] at (-3.250000, 23.700000) {かんば};
\node[Meaning] at (-3.200000, 25.350000) {perfume};
\node[Square] at (-1.150000, 23.600000) {};
\node[Kanji] at (-1.150000, 24.100000) {\textcolor[HTML]{102b59}{肪}};
\node[Onyomi] at (-1.100000, 23.700000) {ボウ};
\node[Meaning] at (-1.150000, 25.350000) {obese};
\node[Square] at (0.900000, 23.600000) {};
\node[Kanji] at (0.900000, 24.100000) {\textcolor[HTML]{14418e}{訪}};
\node[Onyomi] at (0.950000, 23.700000) {ホウ};
\node[Kunyomi] at (0.850000, 23.700000) {たず.ねる};
\node[Meaning] at (0.900000, 25.350000) {visit};
\node[Square] at (2.950000, 23.600000) {};
\node[Kanji] at (2.950000, 24.100000) {\textcolor[HTML]{1551b8}{放}};
\node[Onyomi] at (3.000000, 23.700000) {ホウ};
\node[Kunyomi] at (2.900000, 23.700000) {はな};
\node[Meaning] at (2.950000, 25.350000) {release};
\node[Square] at (5.000000, 23.600000) {};
\node[Kanji] at (5.000000, 24.100000) {\textcolor[HTML]{1551b8}{激}};
\node[Onyomi] at (5.050000, 23.700000) {ゲキ};
\node[Kunyomi] at (4.950000, 23.700000) {はげ.しい};
\node[Meaning] at (5.000000, 25.350000) {fierce};
\node[Square] at (7.050000, 23.600000) {};
\node[Kanji] at (7.050000, 24.100000) {\textcolor[HTML]{14469c}{脱}};
\node[Onyomi] at (7.100000, 23.700000) {ダツ};
\node[Kunyomi] at (7.000000, 23.700000) {ぬ.ぐ};
\node[Meaning] at (7.050000, 25.350000) {undress};
\node[Square] at (9.100000, 23.600000) {};
\node[Kanji] at (9.100000, 24.100000) {\textcolor[HTML]{1551b8}{説}};
\node[Onyomi] at (9.150000, 23.700000) {セツ};
\node[Kunyomi] at (9.050000, 23.700000) {と.く};
\node[Meaning] at (9.100000, 25.350000) {theory};
\node[Square] at (11.150000, 23.600000) {};
\node[Kanji] at (11.150000, 24.100000) {\textcolor[HTML]{14418e}{鋭}};
\node[Onyomi] at (11.200000, 23.700000) {エイ};
\node[Kunyomi] at (11.100000, 23.700000) {するど.い};
\node[Meaning] at (11.150000, 25.350000) {sharp};
\node[Square] at (13.200000, 23.600000) {};
\node[Kanji] at (13.200000, 24.100000) {\textcolor[HTML]{0e254c}{曽}};
\node[Onyomi] at (13.250000, 23.700000) {ソウ};
\node[Meaning] at (13.200000, 25.350000) {formerly};
\node[Square] at (15.250000, 23.600000) {};
\node[Kanji] at (15.250000, 24.100000) {\textcolor[HTML]{1557c6}{増}};
\node[Onyomi] at (15.300000, 23.700000) {ゾウ};
\node[Kunyomi] at (15.200000, 23.700000) {ふ.える};
\node[Meaning] at (15.250000, 25.350000) {increase};
\node[Square] at (17.300000, 23.600000) {};
\node[Kanji] at (17.300000, 24.100000) {\textcolor[HTML]{14418e}{贈}};
\node[Onyomi] at (17.350000, 23.700000) {ゾウ};
\node[Kunyomi] at (17.250000, 23.700000) {おく.*};
\node[Meaning] at (17.300000, 25.350000) {presents};
\node[Square] at (19.350000, 23.600000) {};
\node[Kanji] at (19.350000, 24.100000) {\textcolor[HTML]{145cd5}{東}};
\node[Onyomi] at (19.400000, 23.700000) {トウ};
\node[Kunyomi] at (19.300000, 23.700000) {ひがし};
\node[Meaning] at (19.350000, 25.350000) {east};
\node[Square] at (21.400000, 23.600000) {};
\node[Kanji] at (21.400000, 24.100000) {\textcolor[HTML]{133c80}{棟}};
\node[Onyomi] at (21.450000, 23.700000) {トウ};
\node[Meaning] at (21.400000, 25.350000) {pillar};
\node[Square] at (23.450000, 23.600000) {};
\node[Kanji] at (23.450000, 24.100000) {\textcolor[HTML]{14469c}{凍}};
\node[Onyomi] at (23.500000, 23.700000) {トウ};
\node[Kunyomi] at (23.400000, 23.700000) {こお.る};
\node[Meaning] at (23.450000, 25.350000) {frozen};
\node[Square] at (25.500000, 23.600000) {};
\node[Kanji] at (25.500000, 24.100000) {\textcolor[HTML]{123673}{妊}};
\node[Onyomi] at (25.550000, 23.700000) {ニン};
\node[Meaning] at (25.500000, 25.350000) {pregnant};
\node[Square] at (27.550000, 23.600000) {};
\node[Kanji] at (27.550000, 24.100000) {\textcolor[HTML]{123673}{廷}};
\node[Onyomi] at (27.600000, 23.700000) {テイ};
\node[Meaning] at (27.550000, 25.350000) {courts};
\node[Square] at (29.600000, 23.600000) {};
\node[Kanji] at (29.600000, 24.100000) {\textcolor[HTML]{154caa}{染}};
\node[Onyomi] at (29.650000, 23.700000) {セン};
\node[Kunyomi] at (29.550000, 23.700000) {しみ};
\node[Meaning] at (29.600000, 25.350000) {dye};
\node[Square] at (31.650000, 23.600000) {};
\node[Kanji] at (31.650000, 24.100000) {\textcolor[HTML]{14469c}{燃}};
\node[Onyomi] at (31.700000, 23.700000) {ネン};
\node[Kunyomi] at (31.600000, 23.700000) {も.*};
\node[Meaning] at (31.650000, 25.350000) {burn};
\node[Square] at (33.700000, 23.600000) {};
\node[Kanji] at (33.700000, 24.100000) {\textcolor[HTML]{113066}{賓}};
\node[Onyomi] at (33.750000, 23.700000) {ヒン};
\node[Meaning] at (33.700000, 25.350000) {vip};
\node[Square] at (35.750000, 23.600000) {};
\node[Kanji] at (35.750000, 24.100000) {\textcolor[HTML]{1551b8}{歳}};
\node[Onyomi] at (35.800000, 23.700000) {サイ};
\node[Meaning] at (35.750000, 25.350000) {years old};
\node[Square] at (37.800000, 23.600000) {};
\node[Kanji] at (37.800000, 24.100000) {\textcolor[HTML]{145cd5}{県}};
\node[Onyomi] at (37.850000, 23.700000) {ケン};
\node[Meaning] at (37.800000, 25.350000) {prefecture};
\node[Square] at (39.850000, 23.600000) {};
\node[Kanji] at (39.850000, 24.100000) {\textcolor[HTML]{113066}{栃}};
\node[Kunyomi] at (39.800000, 23.700000) {とち};
\node[Meaning] at (39.850000, 25.350000) {horse chestnut};
\node[Square] at (41.900000, 23.600000) {};
\node[Kanji] at (41.900000, 24.100000) {\textcolor[HTML]{1461e3}{地}};
\node[Onyomi] at (41.950000, 23.700000) {チ};
\node[Meaning] at (41.900000, 25.350000) {earth};
\node[Square] at (43.950000, 23.600000) {};
\node[Kanji] at (43.950000, 24.100000) {\textcolor[HTML]{14418e}{池}};
\node[Onyomi] at (44.000000, 23.700000) {チ};
\node[Kunyomi] at (43.900000, 23.700000) {いけ};
\node[Meaning] at (43.950000, 25.350000) {pond};
\node[Square] at (46.000000, 23.600000) {};
\node[Kanji] at (46.000000, 24.100000) {\textcolor[HTML]{14418e}{虫}};
\node[Onyomi] at (46.050000, 23.700000) {チュウ};
\node[Kunyomi] at (45.950000, 23.700000) {むし};
\node[Meaning] at (46.000000, 25.350000) {insect};
\node[Square] at (48.050000, 23.600000) {};
\node[Kanji] at (48.050000, 24.100000) {\textcolor[HTML]{0e254c}{蛍}};
\node[Onyomi] at (48.100000, 23.700000) {ケイ};
\node[Kunyomi] at (48.000000, 23.700000) {ほたる};
\node[Meaning] at (48.050000, 25.350000) {firefly};
\node[Square] at (50.100000, 23.600000) {};
\node[Kanji] at (50.100000, 24.100000) {\textcolor[HTML]{154caa}{蛇}};
\node[Onyomi] at (50.150000, 23.700000) {ジャ};
\node[Kunyomi] at (50.050000, 23.700000) {へび};
\node[Meaning] at (50.100000, 25.350000) {snake};
\node[Square] at (52.150000, 23.600000) {};
\node[Kanji] at (52.150000, 24.100000) {\textcolor[HTML]{0e254c}{虹}};
\node[Onyomi] at (52.200000, 23.700000) {コウ};
\node[Kunyomi] at (52.100000, 23.700000) {にじ};
\node[Meaning] at (52.150000, 25.350000) {rainbow};
\node[Square] at (54.200000, 23.600000) {};
\node[Kanji] at (54.200000, 24.100000) {\textcolor[HTML]{113066}{蝶}};
\node[Onyomi] at (54.250000, 23.700000) {チョウ};
\node[Meaning] at (54.200000, 25.350000) {butterfly};
\node[Square] at (56.250000, 23.600000) {};
\node[Kanji] at (56.250000, 24.100000) {\textcolor[HTML]{14469c}{独}};
\node[Onyomi] at (56.300000, 23.700000) {ドク};
\node[Kunyomi] at (56.200000, 23.700000) {ひと.り};
\node[Meaning] at (56.250000, 25.350000) {alone};
\node[Meaning] at (-58.500000, 24.150000) {34.82\%};
\node[Square] at (-56.500000, 21.550000) {};
\node[Kanji] at (-56.500000, 22.050000) {\textcolor[HTML]{0e254c}{蚕}};
\node[Onyomi] at (-56.450000, 21.650000) {サン};
\node[Kunyomi] at (-56.550000, 21.650000) {かいこ};
\node[Meaning] at (-56.500000, 23.300000) {silkworm};
\node[Square] at (-54.450000, 21.550000) {};
\node[Kanji] at (-54.450000, 22.050000) {\textcolor[HTML]{1551b8}{風}};
\node[Onyomi] at (-54.400000, 21.650000) {フウ};
\node[Kunyomi] at (-54.500000, 21.650000) {かぜ};
\node[Meaning] at (-54.450000, 23.300000) {wind};
\node[Square] at (-52.400000, 21.550000) {};
\node[Kanji] at (-52.400000, 22.050000) {\textcolor[HTML]{123673}{己}};
\node[Onyomi] at (-52.350000, 21.650000) {コ};
\node[Kunyomi] at (-52.450000, 21.650000) {おのれ};
\node[Meaning] at (-52.400000, 23.300000) {oneself};
\node[Square] at (-50.350000, 21.550000) {};
\node[Kanji] at (-50.350000, 22.050000) {\textcolor[HTML]{145cd5}{起}};
\node[Onyomi] at (-50.300000, 21.650000) {キ};
\node[Kunyomi] at (-50.400000, 21.650000) {お};
\node[Meaning] at (-50.350000, 23.300000) {wake up};
\node[Square] at (-48.300000, 21.550000) {};
\node[Kanji] at (-48.300000, 22.050000) {\textcolor[HTML]{0e254c}{妃}};
\node[Onyomi] at (-48.250000, 21.650000) {ヒ};
\node[Meaning] at (-48.300000, 23.300000) {princess};
\node[Square] at (-46.250000, 21.550000) {};
\node[Kanji] at (-46.250000, 22.050000) {\textcolor[HTML]{133c80}{改}};
\node[Onyomi] at (-46.200000, 21.650000) {カイ};
\node[Kunyomi] at (-46.300000, 21.650000) {あらた.*};
\node[Meaning] at (-46.250000, 23.300000) {renew};
\node[Square] at (-44.200000, 21.550000) {};
\node[Kanji] at (-44.200000, 22.050000) {\textcolor[HTML]{1557c6}{記}};
\node[Onyomi] at (-44.150000, 21.650000) {キ};
\node[Kunyomi] at (-44.250000, 21.650000) {しる.す};
\node[Meaning] at (-44.200000, 23.300000) {write down};
\node[Square] at (-42.150000, 21.550000) {};
\node[Kanji] at (-42.150000, 22.050000) {\textcolor[HTML]{154caa}{包}};
\node[Onyomi] at (-42.100000, 21.650000) {ホウ};
\node[Kunyomi] at (-42.200000, 21.650000) {つつ.み};
\node[Meaning] at (-42.150000, 23.300000) {wrap};
\node[Square] at (-40.100000, 21.550000) {};
\node[Kanji] at (-40.100000, 22.050000) {\textcolor[HTML]{14418e}{胞}};
\node[Onyomi] at (-40.050000, 21.650000) {ホウ};
\node[Meaning] at (-40.100000, 23.300000) {cell};
\node[Square] at (-38.050000, 21.550000) {};
\node[Kanji] at (-38.050000, 22.050000) {\textcolor[HTML]{113066}{砲}};
\node[Onyomi] at (-38.000000, 21.650000) {ホウ};
\node[Meaning] at (-38.050000, 23.300000) {cannon};
\node[Square] at (-36.000000, 21.550000) {};
\node[Kanji] at (-36.000000, 22.050000) {\textcolor[HTML]{133c80}{泡}};
\node[Onyomi] at (-35.950000, 21.650000) {ホウ};
\node[Kunyomi] at (-36.050000, 21.650000) {あわ};
\node[Meaning] at (-36.000000, 23.300000) {bubbles};
\node[Square] at (-33.950000, 21.550000) {};
\node[Kanji] at (-33.950000, 22.050000) {\textcolor[HTML]{123673}{亀}};
\node[Kunyomi] at (-34.000000, 21.650000) {かめ};
\node[Meaning] at (-33.950000, 23.300000) {turtle};
\node[Square] at (-31.900000, 21.550000) {};
\node[Kanji] at (-31.900000, 22.050000) {\textcolor[HTML]{145cd5}{電}};
\node[Onyomi] at (-31.850000, 21.650000) {デン};
\node[Meaning] at (-31.900000, 23.300000) {electricity};
\node[Square] at (-29.850000, 21.550000) {};
\node[Kanji] at (-29.850000, 22.050000) {\textcolor[HTML]{133c80}{竜}};
\node[Onyomi] at (-29.800000, 21.650000) {リュウ};
\node[Kunyomi] at (-29.900000, 21.650000) {たつ};
\node[Meaning] at (-29.850000, 23.300000) {dragon};
\node[Square] at (-27.800000, 21.550000) {};
\node[Kanji] at (-27.800000, 22.050000) {\textcolor[HTML]{123673}{滝}};
\node[Kunyomi] at (-27.850000, 21.650000) {たき};
\node[Meaning] at (-27.800000, 23.300000) {waterfall};
\node[Square] at (-25.750000, 21.550000) {};
\node[Kanji] at (-25.750000, 22.050000) {\textcolor[HTML]{133c80}{豚}};
\node[Onyomi] at (-25.700000, 21.650000) {トン};
\node[Kunyomi] at (-25.800000, 21.650000) {ぶた};
\node[Meaning] at (-25.750000, 23.300000) {pork};
\node[Square] at (-23.700000, 21.550000) {};
\node[Kanji] at (-23.700000, 22.050000) {\textcolor[HTML]{0e254c}{逐}};
\node[Onyomi] at (-23.650000, 21.650000) {チク};
\node[Meaning] at (-23.700000, 23.300000) {pursue};
\node[Square] at (-21.650000, 21.550000) {};
\node[Kanji] at (-21.650000, 22.050000) {\textcolor[HTML]{133c80}{遂}};
\node[Onyomi] at (-21.600000, 21.650000) {スイ};
\node[Kunyomi] at (-21.700000, 21.650000) {と.げる};
\node[Meaning] at (-21.650000, 23.300000) {accomplish};
\node[Square] at (-19.600000, 21.550000) {};
\node[Kanji] at (-19.600000, 22.050000) {\textcolor[HTML]{1461e3}{家}};
\node[Onyomi] at (-19.550000, 21.650000) {カ};
\node[Kunyomi] at (-19.650000, 21.650000) {いえ};
\node[Meaning] at (-19.600000, 23.300000) {house};
\node[Square] at (-17.550000, 21.550000) {};
\node[Kanji] at (-17.550000, 22.050000) {\textcolor[HTML]{102b59}{嫁}};
\node[Onyomi] at (-17.500000, 21.650000) {カ};
\node[Kunyomi] at (-17.600000, 21.650000) {よめ};
\node[Meaning] at (-17.550000, 23.300000) {bride};
\node[Square] at (-15.500000, 21.550000) {};
\node[Kanji] at (-15.500000, 22.050000) {\textcolor[HTML]{123673}{豪}};
\node[Onyomi] at (-15.450000, 21.650000) {ゴウ};
\node[Meaning] at (-15.500000, 23.300000) {luxurious};
\node[Square] at (-13.450000, 21.550000) {};
\node[Kanji] at (-13.450000, 22.050000) {\textcolor[HTML]{113066}{腸}};
\node[Onyomi] at (-13.400000, 21.650000) {チョウ};
\node[Kunyomi] at (-13.500000, 21.650000) {はらわた};
\node[Meaning] at (-13.450000, 23.300000) {intestines};
\node[Square] at (-11.400000, 21.550000) {};
\node[Kanji] at (-11.400000, 22.050000) {\textcolor[HTML]{1461e3}{場}};
\node[Onyomi] at (-11.350000, 21.650000) {ジョウ};
\node[Kunyomi] at (-11.450000, 21.650000) {ば};
\node[Meaning] at (-11.400000, 23.300000) {location};
\node[Square] at (-9.350000, 21.550000) {};
\node[Kanji] at (-9.350000, 22.050000) {\textcolor[HTML]{14418e}{湯}};
\node[Onyomi] at (-9.300000, 21.650000) {トウ};
\node[Kunyomi] at (-9.400000, 21.650000) {ゆ};
\node[Meaning] at (-9.350000, 23.300000) {hot water};
\node[Square] at (-7.300000, 21.550000) {};
\node[Kanji] at (-7.300000, 22.050000) {\textcolor[HTML]{154caa}{羊}};
\node[Onyomi] at (-7.250000, 21.650000) {ヨウ};
\node[Kunyomi] at (-7.350000, 21.650000) {ひつじ};
\node[Meaning] at (-7.300000, 23.300000) {sheep};
\node[Square] at (-5.250000, 21.550000) {};
\node[Kanji] at (-5.250000, 22.050000) {\textcolor[HTML]{1551b8}{美}};
\node[Onyomi] at (-5.200000, 21.650000) {ビ};
\node[Kunyomi] at (-5.300000, 21.650000) {うつく.しい};
\node[Meaning] at (-5.250000, 23.300000) {beauty};
\node[Square] at (-3.200000, 21.550000) {};
\node[Kanji] at (-3.200000, 22.050000) {\textcolor[HTML]{14469c}{洋}};
\node[Onyomi] at (-3.150000, 21.650000) {ヨウ};
\node[Meaning] at (-3.200000, 23.300000) {western style};
\node[Square] at (-1.150000, 21.550000) {};
\node[Kanji] at (-1.150000, 22.050000) {\textcolor[HTML]{14418e}{詳}};
\node[Onyomi] at (-1.100000, 21.650000) {ショウ};
\node[Kunyomi] at (-1.200000, 21.650000) {くわ.しい};
\node[Meaning] at (-1.150000, 23.300000) {detailed};
\node[Square] at (0.900000, 21.550000) {};
\node[Kanji] at (0.900000, 22.050000) {\textcolor[HTML]{14418e}{鮮}};
\node[Onyomi] at (0.950000, 21.650000) {セン};
\node[Kunyomi] at (0.850000, 21.650000) {あざ.やか};
\node[Meaning] at (0.900000, 23.300000) {fresh};
\node[Square] at (2.950000, 21.550000) {};
\node[Kanji] at (2.950000, 22.050000) {\textcolor[HTML]{1557c6}{達}};
\node[Onyomi] at (3.000000, 21.650000) {タツ};
\node[Kunyomi] at (2.900000, 21.650000) {たち};
\node[Meaning] at (2.950000, 23.300000) {attain};
\node[Square] at (5.000000, 21.550000) {};
\node[Kanji] at (5.000000, 22.050000) {\textcolor[HTML]{123673}{羨}};
\node[Onyomi] at (5.050000, 21.650000) {セン};
\node[Kunyomi] at (4.950000, 21.650000) {うらや-む};
\node[Meaning] at (5.000000, 23.300000) {envy};
\node[Square] at (7.050000, 21.550000) {};
\node[Kanji] at (7.050000, 22.050000) {\textcolor[HTML]{1551b8}{差}};
\node[Onyomi] at (7.100000, 21.650000) {サ};
\node[Kunyomi] at (7.000000, 21.650000) {さ};
\node[Meaning] at (7.050000, 23.300000) {distinction};
\node[Square] at (9.100000, 21.550000) {};
\node[Kanji] at (9.100000, 22.050000) {\textcolor[HTML]{145cd5}{着}};
\node[Onyomi] at (9.150000, 21.650000) {チャク};
\node[Kunyomi] at (9.050000, 21.650000) {き};
\node[Meaning] at (9.100000, 23.300000) {wear};
\node[Square] at (11.150000, 21.550000) {};
\node[Kanji] at (11.150000, 22.050000) {\textcolor[HTML]{133c80}{唯}};
\node[Onyomi] at (11.200000, 21.650000) {ユイ};
\node[Kunyomi] at (11.100000, 21.650000) {ただ};
\node[Meaning] at (11.150000, 23.300000) {solely};
\node[Square] at (13.200000, 21.550000) {};
\node[Kanji] at (13.200000, 22.050000) {\textcolor[HTML]{14418e}{焦}};
\node[Onyomi] at (13.250000, 21.650000) {ショウ};
\node[Kunyomi] at (13.150000, 21.650000) {こ.*};
\node[Meaning] at (13.200000, 23.300000) {char};
\node[Square] at (15.250000, 21.550000) {};
\node[Kanji] at (15.250000, 22.050000) {\textcolor[HTML]{0e254c}{礁}};
\node[Onyomi] at (15.300000, 21.650000) {ショウ};
\node[Meaning] at (15.250000, 23.300000) {reef};
\node[Square] at (17.300000, 21.550000) {};
\node[Kanji] at (17.300000, 22.050000) {\textcolor[HTML]{1557c6}{集}};
\node[Onyomi] at (17.350000, 21.650000) {シュウ};
\node[Kunyomi] at (17.250000, 21.650000) {あつ.まる};
\node[Meaning] at (17.300000, 23.300000) {collect};
\node[Square] at (19.350000, 21.550000) {};
\node[Kanji] at (19.350000, 22.050000) {\textcolor[HTML]{0e254c}{准}};
\node[Onyomi] at (19.400000, 21.650000) {ジュン};
\node[Meaning] at (19.350000, 23.300000) {semi};
\node[Square] at (21.400000, 21.550000) {};
\node[Kanji] at (21.400000, 22.050000) {\textcolor[HTML]{1551b8}{進}};
\node[Onyomi] at (21.450000, 21.650000) {シン};
\node[Kunyomi] at (21.350000, 21.650000) {すす.む};
\node[Meaning] at (21.400000, 23.300000) {advance};
\node[Square] at (23.450000, 21.550000) {};
\node[Kanji] at (23.450000, 22.050000) {\textcolor[HTML]{14469c}{雑}};
\node[Onyomi] at (23.500000, 21.650000) {ザツ};
\node[Meaning] at (23.450000, 23.300000) {random};
\node[Square] at (25.500000, 21.550000) {};
\node[Kanji] at (25.500000, 22.050000) {\textcolor[HTML]{113066}{雌}};
\node[Onyomi] at (25.550000, 21.650000) {シ};
\node[Kunyomi] at (25.450000, 21.650000) {めす};
\node[Meaning] at (25.500000, 23.300000) {female};
\node[Square] at (27.550000, 21.550000) {};
\node[Kanji] at (27.550000, 22.050000) {\textcolor[HTML]{154caa}{準}};
\node[Onyomi] at (27.600000, 21.650000) {ジュン};
\node[Meaning] at (27.550000, 23.300000) {standard};
\node[Square] at (29.600000, 21.550000) {};
\node[Kanji] at (29.600000, 22.050000) {\textcolor[HTML]{154caa}{奮}};
\node[Onyomi] at (29.650000, 21.650000) {フン};
\node[Kunyomi] at (29.550000, 21.650000) {ふる.*};
\node[Meaning] at (29.600000, 23.300000) {stirred up};
\node[Square] at (31.650000, 21.550000) {};
\node[Kanji] at (31.650000, 22.050000) {\textcolor[HTML]{14418e}{奪}};
\node[Onyomi] at (31.700000, 21.650000) {ダツ};
\node[Kunyomi] at (31.600000, 21.650000) {うば};
\node[Meaning] at (31.650000, 23.300000) {rob};
\node[Square] at (33.700000, 21.550000) {};
\node[Kanji] at (33.700000, 22.050000) {\textcolor[HTML]{1551b8}{確}};
\node[Onyomi] at (33.750000, 21.650000) {カク};
\node[Kunyomi] at (33.650000, 21.650000) {たし.か};
\node[Meaning] at (33.700000, 23.300000) {certain};
\node[Square] at (35.750000, 21.550000) {};
\node[Kanji] at (35.750000, 22.050000) {\textcolor[HTML]{1551b8}{午}};
\node[Onyomi] at (35.800000, 21.650000) {ゴ};
\node[Meaning] at (35.750000, 23.300000) {noon};
\node[Square] at (37.800000, 21.550000) {};
\node[Kanji] at (37.800000, 22.050000) {\textcolor[HTML]{154caa}{許}};
\node[Onyomi] at (37.850000, 21.650000) {キョ};
\node[Kunyomi] at (37.750000, 21.650000) {ゆる.す};
\node[Meaning] at (37.800000, 23.300000) {permit};
\node[Square] at (39.850000, 21.550000) {};
\node[Kanji] at (39.850000, 22.050000) {\textcolor[HTML]{14469c}{歓}};
\node[Onyomi] at (39.900000, 21.650000) {カン};
\node[Meaning] at (39.850000, 23.300000) {delight};
\node[Square] at (41.900000, 21.550000) {};
\node[Kanji] at (41.900000, 22.050000) {\textcolor[HTML]{14469c}{権}};
\node[Onyomi] at (41.950000, 21.650000) {ケン};
\node[Meaning] at (41.900000, 23.300000) {rights};
\node[Square] at (43.950000, 21.550000) {};
\node[Kanji] at (43.950000, 22.050000) {\textcolor[HTML]{154caa}{観}};
\node[Onyomi] at (44.000000, 21.650000) {カン};
\node[Kunyomi] at (43.900000, 21.650000) {み.る};
\node[Meaning] at (43.950000, 23.300000) {view};
\node[Square] at (46.000000, 21.550000) {};
\node[Kanji] at (46.000000, 22.050000) {\textcolor[HTML]{1551b8}{羽}};
\node[Kunyomi] at (45.950000, 21.650000) {はね};
\node[Meaning] at (46.000000, 23.300000) {feather};
\node[Square] at (48.050000, 21.550000) {};
\node[Kanji] at (48.050000, 22.050000) {\textcolor[HTML]{154caa}{習}};
\node[Onyomi] at (48.100000, 21.650000) {シュウ};
\node[Kunyomi] at (48.000000, 21.650000) {なら.う};
\node[Meaning] at (48.050000, 23.300000) {learn};
\node[Square] at (50.100000, 21.550000) {};
\node[Kanji] at (50.100000, 22.050000) {\textcolor[HTML]{133c80}{翌}};
\node[Onyomi] at (50.150000, 21.650000) {ヨク};
\node[Meaning] at (50.100000, 23.300000) {the following};
\node[Square] at (52.150000, 21.550000) {};
\node[Kanji] at (52.150000, 22.050000) {\textcolor[HTML]{1551b8}{曜}};
\node[Onyomi] at (52.200000, 21.650000) {ヨウ};
\node[Meaning] at (52.150000, 23.300000) {weekday};
\node[Square] at (54.200000, 21.550000) {};
\node[Kanji] at (54.200000, 22.050000) {\textcolor[HTML]{133c80}{濯}};
\node[Onyomi] at (54.250000, 21.650000) {タク};
\node[Kunyomi] at (54.150000, 21.650000) {すす.ぐ};
\node[Meaning] at (54.200000, 23.300000) {wash};
\node[Square] at (56.250000, 21.550000) {};
\node[Kanji] at (56.250000, 22.050000) {\textcolor[HTML]{14469c}{困}};
\node[Onyomi] at (56.300000, 21.650000) {コン};
\node[Kunyomi] at (56.200000, 21.650000) {こま};
\node[Meaning] at (56.250000, 23.300000) {distressed};
\node[Meaning] at (-58.500000, 22.100000) {37.36\%};
\node[Square] at (-56.500000, 19.500000) {};
\node[Kanji] at (-56.500000, 20.000000) {\textcolor[HTML]{14469c}{固}};
\node[Onyomi] at (-56.450000, 19.600000) {コ};
\node[Kunyomi] at (-56.550000, 19.600000) {かた.い};
\node[Meaning] at (-56.500000, 21.250000) {hard};
\node[Square] at (-54.450000, 19.500000) {};
\node[Kanji] at (-54.450000, 20.000000) {\textcolor[HTML]{1461e3}{国}};
\node[Onyomi] at (-54.400000, 19.600000) {コク};
\node[Kunyomi] at (-54.500000, 19.600000) {くに};
\node[Meaning] at (-54.450000, 21.250000) {country};
\node[Square] at (-52.400000, 19.500000) {};
\node[Kanji] at (-52.400000, 20.000000) {\textcolor[HTML]{1551b8}{団}};
\node[Onyomi] at (-52.350000, 19.600000) {ダン};
\node[Meaning] at (-52.400000, 21.250000) {group};
\node[Square] at (-50.350000, 19.500000) {};
\node[Kanji] at (-50.350000, 20.000000) {\textcolor[HTML]{14469c}{因}};
\node[Onyomi] at (-50.300000, 19.600000) {イン};
\node[Kunyomi] at (-50.400000, 19.600000) {よ};
\node[Meaning] at (-50.350000, 21.250000) {cause};
\node[Square] at (-48.300000, 19.500000) {};
\node[Kanji] at (-48.300000, 20.000000) {\textcolor[HTML]{0e254c}{姻}};
\node[Onyomi] at (-48.250000, 19.600000) {イン};
\node[Meaning] at (-48.300000, 21.250000) {marry};
\node[Square] at (-46.250000, 19.500000) {};
\node[Kanji] at (-46.250000, 20.000000) {\textcolor[HTML]{154caa}{園}};
\node[Onyomi] at (-46.200000, 19.600000) {エン};
\node[Meaning] at (-46.250000, 21.250000) {garden};
\node[Square] at (-44.200000, 19.500000) {};
\node[Kanji] at (-44.200000, 20.000000) {\textcolor[HTML]{1461e3}{回}};
\node[Onyomi] at (-44.150000, 19.600000) {カイ};
\node[Kunyomi] at (-44.250000, 19.600000) {まわ.*};
\node[Meaning] at (-44.200000, 21.250000) {times};
\node[Square] at (-42.150000, 19.500000) {};
\node[Kanji] at (-42.150000, 20.000000) {\textcolor[HTML]{133c80}{壇}};
\node[Onyomi] at (-42.100000, 19.600000) {ダン};
\node[Meaning] at (-42.150000, 21.250000) {podium};
\node[Square] at (-40.100000, 19.500000) {};
\node[Kanji] at (-40.100000, 20.000000) {\textcolor[HTML]{1551b8}{店}};
\node[Onyomi] at (-40.050000, 19.600000) {テン};
\node[Kunyomi] at (-40.150000, 19.600000) {みせ};
\node[Meaning] at (-40.100000, 21.250000) {shop};
\node[Square] at (-38.050000, 19.500000) {};
\node[Kanji] at (-38.050000, 20.000000) {\textcolor[HTML]{14469c}{庫}};
\node[Onyomi] at (-38.000000, 19.600000) {コ};
\node[Kunyomi] at (-38.100000, 19.600000) {くら};
\node[Meaning] at (-38.050000, 21.250000) {storage};
\node[Square] at (-36.000000, 19.500000) {};
\node[Kanji] at (-36.000000, 20.000000) {\textcolor[HTML]{1551b8}{庭}};
\node[Onyomi] at (-35.950000, 19.600000) {テイ};
\node[Kunyomi] at (-36.050000, 19.600000) {にわ};
\node[Meaning] at (-36.000000, 21.250000) {garden};
\node[Square] at (-33.950000, 19.500000) {};
\node[Kanji] at (-33.950000, 20.000000) {\textcolor[HTML]{154caa}{庁}};
\node[Onyomi] at (-33.900000, 19.600000) {チョウ};
\node[Meaning] at (-33.950000, 21.250000) {agency};
\node[Square] at (-31.900000, 19.500000) {};
\node[Kanji] at (-31.900000, 20.000000) {\textcolor[HTML]{1551b8}{床}};
\node[Onyomi] at (-31.850000, 19.600000) {ショウ};
\node[Kunyomi] at (-31.950000, 19.600000) {ゆか};
\node[Meaning] at (-31.900000, 21.250000) {floor};
\node[Square] at (-29.850000, 19.500000) {};
\node[Kanji] at (-29.850000, 20.000000) {\textcolor[HTML]{133c80}{麻}};
\node[Onyomi] at (-29.800000, 19.600000) {マ};
\node[Kunyomi] at (-29.900000, 19.600000) {あさ};
\node[Meaning] at (-29.850000, 21.250000) {hemp};
\node[Square] at (-27.800000, 19.500000) {};
\node[Kanji] at (-27.800000, 20.000000) {\textcolor[HTML]{133c80}{磨}};
\node[Onyomi] at (-27.750000, 19.600000) {マ};
\node[Kunyomi] at (-27.850000, 19.600000) {みが};
\node[Meaning] at (-27.800000, 21.250000) {polish};
\node[Square] at (-25.750000, 19.500000) {};
\node[Kanji] at (-25.750000, 20.000000) {\textcolor[HTML]{145cd5}{心}};
\node[Onyomi] at (-25.700000, 19.600000) {シン};
\node[Kunyomi] at (-25.800000, 19.600000) {こころ};
\node[Meaning] at (-25.750000, 21.250000) {heart};
\node[Square] at (-23.700000, 19.500000) {};
\node[Kanji] at (-23.700000, 20.000000) {\textcolor[HTML]{1551b8}{忘}};
\node[Onyomi] at (-23.650000, 19.600000) {ボウ};
\node[Kunyomi] at (-23.750000, 19.600000) {わす.れる};
\node[Meaning] at (-23.700000, 21.250000) {forget};
\node[Square] at (-21.650000, 19.500000) {};
\node[Kanji] at (-21.650000, 20.000000) {\textcolor[HTML]{14469c}{忍}};
\node[Onyomi] at (-21.600000, 19.600000) {ニン};
\node[Kunyomi] at (-21.700000, 19.600000) {しの.ぶ};
\node[Meaning] at (-21.650000, 21.250000) {endure};
\node[Square] at (-19.600000, 19.500000) {};
\node[Kanji] at (-19.600000, 20.000000) {\textcolor[HTML]{154caa}{認}};
\node[Onyomi] at (-19.550000, 19.600000) {ニン};
\node[Kunyomi] at (-19.650000, 19.600000) {みと.める};
\node[Meaning] at (-19.600000, 21.250000) {recognize};
\node[Square] at (-17.550000, 19.500000) {};
\node[Kanji] at (-17.550000, 20.000000) {\textcolor[HTML]{0e254c}{忌}};
\node[Onyomi] at (-17.500000, 19.600000) {キ};
\node[Kunyomi] at (-17.600000, 19.600000) {い};
\node[Meaning] at (-17.550000, 21.250000) {mourning};
\node[Square] at (-15.500000, 19.500000) {};
\node[Kanji] at (-15.500000, 20.000000) {\textcolor[HTML]{123673}{志}};
\node[Onyomi] at (-15.450000, 19.600000) {シ};
\node[Kunyomi] at (-15.550000, 19.600000) {こころざし};
\node[Meaning] at (-15.500000, 21.250000) {intention};
\node[Square] at (-13.450000, 19.500000) {};
\node[Kanji] at (-13.450000, 20.000000) {\textcolor[HTML]{133c80}{誌}};
\node[Onyomi] at (-13.400000, 19.600000) {シ};
\node[Meaning] at (-13.450000, 21.250000) {magazine};
\node[Square] at (-11.400000, 19.500000) {};
\node[Kanji] at (-11.400000, 20.000000) {\textcolor[HTML]{14418e}{忠}};
\node[Onyomi] at (-11.350000, 19.600000) {チュウ};
\node[Meaning] at (-11.400000, 21.250000) {loyalty};
\node[Square] at (-9.350000, 19.500000) {};
\node[Kanji] at (-9.350000, 20.000000) {\textcolor[HTML]{0e254c}{串}};
\node[Kunyomi] at (-9.400000, 19.600000) {くし};
\node[Meaning] at (-9.350000, 21.250000) {skewer};
\node[Square] at (-7.300000, 19.500000) {};
\node[Kanji] at (-7.300000, 20.000000) {\textcolor[HTML]{123673}{患}};
\node[Onyomi] at (-7.250000, 19.600000) {カン};
\node[Kunyomi] at (-7.350000, 19.600000) {わずら.う};
\node[Meaning] at (-7.300000, 21.250000) {afflicted};
\node[Square] at (-5.250000, 19.500000) {};
\node[Kanji] at (-5.250000, 20.000000) {\textcolor[HTML]{1968ed}{思}};
\node[Onyomi] at (-5.200000, 19.600000) {シ};
\node[Kunyomi] at (-5.300000, 19.600000) {おも.う};
\node[Meaning] at (-5.250000, 21.250000) {think};
\node[Square] at (-3.200000, 19.500000) {};
\node[Kanji] at (-3.200000, 20.000000) {\textcolor[HTML]{102b59}{恩}};
\node[Onyomi] at (-3.150000, 19.600000) {オン};
\node[Kunyomi] at (-3.250000, 19.600000) {おん};
\node[Meaning] at (-3.200000, 21.250000) {kindness};
\node[Square] at (-1.150000, 19.500000) {};
\node[Kanji] at (-1.150000, 20.000000) {\textcolor[HTML]{154caa}{応}};
\node[Onyomi] at (-1.100000, 19.600000) {オウ};
\node[Meaning] at (-1.150000, 21.250000) {respond};
\node[Square] at (0.900000, 19.500000) {};
\node[Kanji] at (0.900000, 20.000000) {\textcolor[HTML]{145cd5}{意}};
\node[Onyomi] at (0.950000, 19.600000) {イ};
\node[Meaning] at (0.900000, 21.250000) {idea};
\node[Square] at (2.950000, 19.500000) {};
\node[Kanji] at (2.950000, 20.000000) {\textcolor[HTML]{154caa}{想}};
\node[Onyomi] at (3.000000, 19.600000) {ソウ};
\node[Meaning] at (2.950000, 21.250000) {concept};
\node[Square] at (5.000000, 19.500000) {};
\node[Kanji] at (5.000000, 20.000000) {\textcolor[HTML]{1557c6}{息}};
\node[Onyomi] at (5.050000, 19.600000) {ソク};
\node[Kunyomi] at (4.950000, 19.600000) {いき};
\node[Meaning] at (5.000000, 21.250000) {breath};
\node[Square] at (7.050000, 19.500000) {};
\node[Kanji] at (7.050000, 20.000000) {\textcolor[HTML]{113066}{憩}};
\node[Onyomi] at (7.100000, 19.600000) {ケイ};
\node[Kunyomi] at (7.000000, 19.600000) {いこ.い};
\node[Meaning] at (7.050000, 21.250000) {rest};
\node[Square] at (9.100000, 19.500000) {};
\node[Kanji] at (9.100000, 20.000000) {\textcolor[HTML]{123673}{恵}};
\node[Onyomi] at (9.150000, 19.600000) {エ};
\node[Kunyomi] at (9.050000, 19.600000) {めぐ.*};
\node[Meaning] at (9.100000, 21.250000) {favor};
\node[Square] at (11.150000, 19.500000) {};
\node[Kanji] at (11.150000, 20.000000) {\textcolor[HTML]{1557c6}{恐}};
\node[Onyomi] at (11.200000, 19.600000) {キョウ};
\node[Kunyomi] at (11.100000, 19.600000) {おそ.*};
\node[Meaning] at (11.150000, 21.250000) {fear};
\node[Square] at (13.200000, 19.500000) {};
\node[Kanji] at (13.200000, 20.000000) {\textcolor[HTML]{154caa}{惑}};
\node[Onyomi] at (13.250000, 19.600000) {ワク};
\node[Kunyomi] at (13.150000, 19.600000) {まど.う};
\node[Meaning] at (13.200000, 21.250000) {misguided};
\node[Square] at (15.250000, 19.500000) {};
\node[Kanji] at (15.250000, 20.000000) {\textcolor[HTML]{145cd5}{感}};
\node[Onyomi] at (15.300000, 19.600000) {カン};
\node[Meaning] at (15.250000, 21.250000) {feeling};
\node[Square] at (17.300000, 19.500000) {};
\node[Kanji] at (17.300000, 20.000000) {\textcolor[HTML]{14418e}{憂}};
\node[Onyomi] at (17.350000, 19.600000) {ユウ};
\node[Kunyomi] at (17.250000, 19.600000) {う};
\node[Meaning] at (17.300000, 21.250000) {grief};
\node[Square] at (19.350000, 19.500000) {};
\node[Kanji] at (19.350000, 20.000000) {\textcolor[HTML]{0e254c}{寡}};
\node[Onyomi] at (19.400000, 19.600000) {カ};
\node[Meaning] at (19.350000, 21.250000) {widow};
\node[Square] at (21.400000, 19.500000) {};
\node[Kanji] at (21.400000, 20.000000) {\textcolor[HTML]{14418e}{忙}};
\node[Onyomi] at (21.450000, 19.600000) {ボウ};
\node[Kunyomi] at (21.350000, 19.600000) {いそが};
\node[Meaning] at (21.400000, 21.250000) {busy};
\node[Square] at (23.450000, 19.500000) {};
\node[Kanji] at (23.450000, 20.000000) {\textcolor[HTML]{102b59}{悦}};
\node[Onyomi] at (23.500000, 19.600000) {エツ};
\node[Kunyomi] at (23.400000, 19.600000) {よろこ};
\node[Meaning] at (23.450000, 21.250000) {delight};
\node[Square] at (25.500000, 19.500000) {};
\node[Kanji] at (25.500000, 20.000000) {\textcolor[HTML]{113066}{恒}};
\node[Onyomi] at (25.550000, 19.600000) {コウ};
\node[Kunyomi] at (25.450000, 19.600000) {つね};
\node[Meaning] at (25.500000, 21.250000) {constant};
\node[Square] at (27.550000, 19.500000) {};
\node[Kanji] at (27.550000, 20.000000) {\textcolor[HTML]{123673}{悼}};
\node[Onyomi] at (27.600000, 19.600000) {トウ};
\node[Kunyomi] at (27.500000, 19.600000) {いた};
\node[Meaning] at (27.550000, 21.250000) {grieve};
\node[Square] at (29.600000, 19.500000) {};
\node[Kanji] at (29.600000, 20.000000) {\textcolor[HTML]{133c80}{悟}};
\node[Onyomi] at (29.650000, 19.600000) {ゴ};
\node[Kunyomi] at (29.550000, 19.600000) {さと.る};
\node[Meaning] at (29.600000, 21.250000) {comprehension};
\node[Square] at (31.650000, 19.500000) {};
\node[Kanji] at (31.650000, 20.000000) {\textcolor[HTML]{1551b8}{怖}};
\node[Onyomi] at (31.700000, 19.600000) {フ};
\node[Kunyomi] at (31.600000, 19.600000) {こわ.*};
\node[Meaning] at (31.650000, 21.250000) {scary};
\node[Square] at (33.700000, 19.500000) {};
\node[Kanji] at (33.700000, 20.000000) {\textcolor[HTML]{14469c}{慌}};
\node[Onyomi] at (33.750000, 19.600000) {コウ};
\node[Kunyomi] at (33.650000, 19.600000) {あわ-てる};
\node[Meaning] at (33.700000, 21.250000) {disconcerted};
\node[Square] at (35.750000, 19.500000) {};
\node[Kanji] at (35.750000, 20.000000) {\textcolor[HTML]{14418e}{悔}};
\node[Onyomi] at (35.800000, 19.600000) {カイ};
\node[Kunyomi] at (35.700000, 19.600000) {くや.しい};
\node[Meaning] at (35.750000, 21.250000) {regret};
\node[Square] at (37.800000, 19.500000) {};
\node[Kanji] at (37.800000, 20.000000) {\textcolor[HTML]{14418e}{憎}};
\node[Onyomi] at (37.850000, 19.600000) {ゾウ};
\node[Kunyomi] at (37.750000, 19.600000) {にく.*};
\node[Meaning] at (37.800000, 21.250000) {hate};
\node[Square] at (39.850000, 19.500000) {};
\node[Kanji] at (39.850000, 20.000000) {\textcolor[HTML]{14418e}{慣}};
\node[Onyomi] at (39.900000, 19.600000) {カン};
\node[Kunyomi] at (39.800000, 19.600000) {な.れる};
\node[Meaning] at (39.850000, 21.250000) {accustomed};
\node[Square] at (41.900000, 19.500000) {};
\node[Kanji] at (41.900000, 20.000000) {\textcolor[HTML]{123673}{愉}};
\node[Onyomi] at (41.950000, 19.600000) {ユ};
\node[Kunyomi] at (41.850000, 19.600000) {たの};
\node[Meaning] at (41.900000, 21.250000) {pleasant};
\node[Square] at (43.950000, 19.500000) {};
\node[Kanji] at (43.950000, 20.000000) {\textcolor[HTML]{0e254c}{惰}};
\node[Onyomi] at (44.000000, 19.600000) {ダ};
\node[Meaning] at (43.950000, 21.250000) {lazy};
\node[Square] at (46.000000, 19.500000) {};
\node[Kanji] at (46.000000, 20.000000) {\textcolor[HTML]{14418e}{慎}};
\node[Onyomi] at (46.050000, 19.600000) {シン};
\node[Kunyomi] at (45.950000, 19.600000) {つつし.む};
\node[Meaning] at (46.000000, 21.250000) {humility};
\node[Square] at (48.050000, 19.500000) {};
\node[Kanji] at (48.050000, 20.000000) {\textcolor[HTML]{0e254c}{憾}};
\node[Onyomi] at (48.100000, 19.600000) {カン};
\node[Kunyomi] at (48.000000, 19.600000) {うら};
\node[Meaning] at (48.050000, 21.250000) {remorse};
\node[Square] at (50.100000, 19.500000) {};
\node[Kanji] at (50.100000, 20.000000) {\textcolor[HTML]{154caa}{憶}};
\node[Onyomi] at (50.150000, 19.600000) {オク};
\node[Meaning] at (50.100000, 21.250000) {recollection};
\node[Square] at (52.150000, 19.500000) {};
\node[Kanji] at (52.150000, 20.000000) {\textcolor[HTML]{0e254c}{慕}};
\node[Onyomi] at (52.200000, 19.600000) {ボ};
\node[Kunyomi] at (52.100000, 19.600000) {した};
\node[Meaning] at (52.150000, 21.250000) {yearn for};
\node[Square] at (54.200000, 19.500000) {};
\node[Kanji] at (54.200000, 20.000000) {\textcolor[HTML]{133c80}{添}};
\node[Onyomi] at (54.250000, 19.600000) {テン};
\node[Kunyomi] at (54.150000, 19.600000) {そ.える};
\node[Meaning] at (54.200000, 21.250000) {append};
\node[Square] at (56.250000, 19.500000) {};
\node[Kanji] at (56.250000, 20.000000) {\textcolor[HTML]{1557c6}{必}};
\node[Onyomi] at (56.300000, 19.600000) {ヒツ};
\node[Kunyomi] at (56.200000, 19.600000) {かなら.ず};
\node[Meaning] at (56.250000, 21.250000) {certain};
\node[Meaning] at (-58.500000, 20.050000) {40.49\%};
\node[Square] at (-56.500000, 17.450000) {};
\node[Kanji] at (-56.500000, 17.950000) {\textcolor[HTML]{0e254c}{泌}};
\node[Onyomi] at (-56.450000, 17.550000) {ヒ};
\node[Meaning] at (-56.500000, 19.200000) {secrete};
\node[Square] at (-54.450000, 17.450000) {};
\node[Kanji] at (-54.450000, 17.950000) {\textcolor[HTML]{1968ed}{手}};
\node[Onyomi] at (-54.400000, 17.550000) {シュ};
\node[Kunyomi] at (-54.500000, 17.550000) {て};
\node[Meaning] at (-54.450000, 19.200000) {hand};
\node[Square] at (-52.400000, 17.450000) {};
\node[Kanji] at (-52.400000, 17.950000) {\textcolor[HTML]{133c80}{看}};
\node[Onyomi] at (-52.350000, 17.550000) {カン};
\node[Meaning] at (-52.400000, 19.200000) {watch over};
\node[Square] at (-50.350000, 17.450000) {};
\node[Kanji] at (-50.350000, 17.950000) {\textcolor[HTML]{0e254c}{摩}};
\node[Onyomi] at (-50.300000, 17.550000) {マ};
\node[Kunyomi] at (-50.400000, 17.550000) {さす.る};
\node[Meaning] at (-50.350000, 19.200000) {chafe};
\node[Square] at (-48.300000, 17.450000) {};
\node[Kanji] at (-48.300000, 17.950000) {\textcolor[HTML]{1557c6}{我}};
\node[Onyomi] at (-48.250000, 17.550000) {ガ};
\node[Kunyomi] at (-48.350000, 17.550000) {われ};
\node[Meaning] at (-48.300000, 19.200000) {i};
\node[Square] at (-46.250000, 17.450000) {};
\node[Kanji] at (-46.250000, 17.950000) {\textcolor[HTML]{133c80}{義}};
\node[Onyomi] at (-46.200000, 17.550000) {ギ};
\node[Meaning] at (-46.250000, 19.200000) {righteousness};
\node[Square] at (-44.200000, 17.450000) {};
\node[Kanji] at (-44.200000, 17.950000) {\textcolor[HTML]{1551b8}{議}};
\node[Onyomi] at (-44.150000, 17.550000) {ギ};
\node[Meaning] at (-44.200000, 19.200000) {deliberation};
\node[Square] at (-42.150000, 17.450000) {};
\node[Kanji] at (-42.150000, 17.950000) {\textcolor[HTML]{133c80}{犠}};
\node[Onyomi] at (-42.100000, 17.550000) {ギ};
\node[Meaning] at (-42.150000, 19.200000) {sacrifice};
\node[Square] at (-40.100000, 17.450000) {};
\node[Kanji] at (-40.100000, 17.950000) {\textcolor[HTML]{102b59}{抹}};
\node[Onyomi] at (-40.050000, 17.550000) {マツ};
\node[Meaning] at (-40.100000, 19.200000) {erase};
\node[Square] at (-38.050000, 17.450000) {};
\node[Kanji] at (-38.050000, 17.950000) {\textcolor[HTML]{154caa}{抱}};
\node[Onyomi] at (-38.000000, 17.550000) {ホウ};
\node[Kunyomi] at (-38.100000, 17.550000) {だ};
\node[Meaning] at (-38.050000, 19.200000) {hug};
\node[Square] at (-36.000000, 17.450000) {};
\node[Kanji] at (-36.000000, 17.950000) {\textcolor[HTML]{123673}{搭}};
\node[Onyomi] at (-35.950000, 17.550000) {トウ};
\node[Meaning] at (-36.000000, 19.200000) {board};
\node[Square] at (-33.950000, 17.450000) {};
\node[Kanji] at (-33.950000, 17.950000) {\textcolor[HTML]{0e254c}{抄}};
\node[Onyomi] at (-33.900000, 17.550000) {ショウ};
\node[Meaning] at (-33.950000, 19.200000) {extract};
\node[Square] at (-31.900000, 17.450000) {};
\node[Kanji] at (-31.900000, 17.950000) {\textcolor[HTML]{154caa}{抗}};
\node[Onyomi] at (-31.850000, 17.550000) {コウ};
\node[Kunyomi] at (-31.950000, 17.550000) {あらが.う};
\node[Meaning] at (-31.900000, 19.200000) {confront};
\node[Square] at (-29.850000, 17.450000) {};
\node[Kanji] at (-29.850000, 17.950000) {\textcolor[HTML]{123673}{批}};
\node[Onyomi] at (-29.800000, 17.550000) {ヒ};
\node[Meaning] at (-29.850000, 19.200000) {criticism};
\node[Square] at (-27.800000, 17.450000) {};
\node[Kanji] at (-27.800000, 17.950000) {\textcolor[HTML]{14418e}{招}};
\node[Onyomi] at (-27.750000, 17.550000) {ショウ};
\node[Kunyomi] at (-27.850000, 17.550000) {まね.く};
\node[Meaning] at (-27.800000, 19.200000) {beckon};
\node[Square] at (-25.750000, 17.450000) {};
\node[Kanji] at (-25.750000, 17.950000) {\textcolor[HTML]{0e254c}{拓}};
\node[Onyomi] at (-25.700000, 17.550000) {タク};
\node[Meaning] at (-25.750000, 19.200000) {cultivation};
\node[Square] at (-23.700000, 17.450000) {};
\node[Kanji] at (-23.700000, 17.950000) {\textcolor[HTML]{14469c}{拍}};
\node[Onyomi] at (-23.650000, 17.550000) {ハク};
\node[Meaning] at (-23.700000, 19.200000) {beat};
\node[Square] at (-21.650000, 17.450000) {};
\node[Kanji] at (-21.650000, 17.950000) {\textcolor[HTML]{1551b8}{打}};
\node[Onyomi] at (-21.600000, 17.550000) {ダ};
\node[Kunyomi] at (-21.700000, 17.550000) {う};
\node[Meaning] at (-21.650000, 19.200000) {hit};
\node[Square] at (-19.600000, 17.450000) {};
\node[Kanji] at (-19.600000, 17.950000) {\textcolor[HTML]{102b59}{拘}};
\node[Onyomi] at (-19.550000, 17.550000) {コウ};
\node[Kunyomi] at (-19.650000, 17.550000) {かか.わる};
\node[Meaning] at (-19.600000, 19.200000) {arrest};
\node[Square] at (-17.550000, 17.450000) {};
\node[Kanji] at (-17.550000, 17.950000) {\textcolor[HTML]{14469c}{捨}};
\node[Onyomi] at (-17.500000, 17.550000) {シャ};
\node[Kunyomi] at (-17.600000, 17.550000) {す};
\node[Meaning] at (-17.550000, 19.200000) {throw away};
\node[Square] at (-15.500000, 17.450000) {};
\node[Kanji] at (-15.500000, 17.950000) {\textcolor[HTML]{0e254c}{拐}};
\node[Onyomi] at (-15.450000, 17.550000) {カイ};
\node[Meaning] at (-15.500000, 19.200000) {kidnap};
\node[Square] at (-13.450000, 17.450000) {};
\node[Kanji] at (-13.450000, 17.950000) {\textcolor[HTML]{133c80}{摘}};
\node[Onyomi] at (-13.400000, 17.550000) {テキ};
\node[Kunyomi] at (-13.500000, 17.550000) {つ.む};
\node[Meaning] at (-13.450000, 19.200000) {pluck};
\node[Square] at (-11.400000, 17.450000) {};
\node[Kanji] at (-11.400000, 17.950000) {\textcolor[HTML]{133c80}{挑}};
\node[Onyomi] at (-11.350000, 17.550000) {チョウ};
\node[Kunyomi] at (-11.450000, 17.550000) {いど.む};
\node[Meaning] at (-11.400000, 19.200000) {challenge};
\node[Square] at (-9.350000, 17.450000) {};
\node[Kanji] at (-9.350000, 17.950000) {\textcolor[HTML]{1557c6}{指}};
\node[Onyomi] at (-9.300000, 17.550000) {シ};
\node[Kunyomi] at (-9.400000, 17.550000) {ゆび};
\node[Meaning] at (-9.350000, 19.200000) {finger};
\node[Square] at (-7.300000, 17.450000) {};
\node[Kanji] at (-7.300000, 17.950000) {\textcolor[HTML]{1461e3}{持}};
\node[Onyomi] at (-7.250000, 17.550000) {ジ};
\node[Kunyomi] at (-7.350000, 17.550000) {も};
\node[Meaning] at (-7.300000, 19.200000) {hold};
\node[Square] at (-5.250000, 17.450000) {};
\node[Kanji] at (-5.250000, 17.950000) {\textcolor[HTML]{113066}{括}};
\node[Onyomi] at (-5.200000, 17.550000) {カツ};
\node[Kunyomi] at (-5.300000, 17.550000) {くく.る};
\node[Meaning] at (-5.250000, 19.200000) {fasten};
\node[Square] at (-3.200000, 17.450000) {};
\node[Kanji] at (-3.200000, 17.950000) {\textcolor[HTML]{133c80}{揮}};
\node[Onyomi] at (-3.150000, 17.550000) {キ};
\node[Meaning] at (-3.200000, 19.200000) {brandish};
\node[Square] at (-1.150000, 17.450000) {};
\node[Kanji] at (-1.150000, 17.950000) {\textcolor[HTML]{133c80}{推}};
\node[Onyomi] at (-1.100000, 17.550000) {スイ};
\node[Kunyomi] at (-1.200000, 17.550000) {お.す};
\node[Meaning] at (-1.150000, 19.200000) {infer};
\node[Square] at (0.900000, 17.450000) {};
\node[Kanji] at (0.900000, 17.950000) {\textcolor[HTML]{14418e}{揚}};
\node[Onyomi] at (0.950000, 17.550000) {ヨウ};
\node[Kunyomi] at (0.850000, 17.550000) {あげ};
\node[Meaning] at (0.900000, 19.200000) {hoist};
\node[Square] at (2.950000, 17.450000) {};
\node[Kanji] at (2.950000, 17.950000) {\textcolor[HTML]{14418e}{提}};
\node[Onyomi] at (3.000000, 17.550000) {テイ};
\node[Meaning] at (2.950000, 19.200000) {present};
\node[Square] at (5.000000, 17.450000) {};
\node[Kanji] at (5.000000, 17.950000) {\textcolor[HTML]{133c80}{損}};
\node[Onyomi] at (5.050000, 17.550000) {ソン};
\node[Kunyomi] at (4.950000, 17.550000) {そこ.なう};
\node[Meaning] at (5.000000, 19.200000) {loss};
\node[Square] at (7.050000, 17.450000) {};
\node[Kanji] at (7.050000, 17.950000) {\textcolor[HTML]{14418e}{拾}};
\node[Kunyomi] at (7.000000, 17.550000) {ひろ};
\node[Meaning] at (7.050000, 19.200000) {pick up};
\node[Square] at (9.100000, 17.450000) {};
\node[Kanji] at (9.100000, 17.950000) {\textcolor[HTML]{14418e}{担}};
\node[Onyomi] at (9.150000, 17.550000) {タン};
\node[Kunyomi] at (9.050000, 17.550000) {にな.う};
\node[Meaning] at (9.100000, 19.200000) {carry};
\node[Square] at (11.150000, 17.450000) {};
\node[Kanji] at (11.150000, 17.950000) {\textcolor[HTML]{14418e}{拠}};
\node[Onyomi] at (11.200000, 17.550000) {キョ};
\node[Kunyomi] at (11.100000, 17.550000) {よ.る};
\node[Meaning] at (11.150000, 19.200000) {based on};
\node[Square] at (13.200000, 17.450000) {};
\node[Kanji] at (13.200000, 17.950000) {\textcolor[HTML]{14469c}{描}};
\node[Onyomi] at (13.250000, 17.550000) {ビョウ};
\node[Kunyomi] at (13.150000, 17.550000) {か.く};
\node[Meaning] at (13.200000, 19.200000) {draw};
\node[Square] at (15.250000, 17.450000) {};
\node[Kanji] at (15.250000, 17.950000) {\textcolor[HTML]{133c80}{操}};
\node[Onyomi] at (15.300000, 17.550000) {ソウ};
\node[Kunyomi] at (15.200000, 17.550000) {あやつ.る};
\node[Meaning] at (15.250000, 19.200000) {manipulate};
\node[Square] at (17.300000, 17.450000) {};
\node[Kanji] at (17.300000, 17.950000) {\textcolor[HTML]{14418e}{接}};
\node[Onyomi] at (17.350000, 17.550000) {セツ};
\node[Kunyomi] at (17.250000, 17.550000) {つ.ぐ};
\node[Meaning] at (17.300000, 19.200000) {adjoin};
\node[Square] at (19.350000, 17.450000) {};
\node[Kanji] at (19.350000, 17.950000) {\textcolor[HTML]{14418e}{掲}};
\node[Onyomi] at (19.400000, 17.550000) {ケイ};
\node[Kunyomi] at (19.300000, 17.550000) {かか.げる};
\node[Meaning] at (19.350000, 19.200000) {display};
\node[Square] at (21.400000, 17.450000) {};
\node[Kanji] at (21.400000, 17.950000) {\textcolor[HTML]{1557c6}{掛}};
\node[Onyomi] at (21.450000, 17.550000) {ガイ};
\node[Kunyomi] at (21.350000, 17.550000) {か};
\node[Meaning] at (21.400000, 19.200000) {hang};
\node[Square] at (23.450000, 17.450000) {};
\node[Kanji] at (23.450000, 17.950000) {\textcolor[HTML]{1551b8}{研}};
\node[Onyomi] at (23.500000, 17.550000) {ケン};
\node[Kunyomi] at (23.400000, 17.550000) {と};
\node[Meaning] at (23.450000, 19.200000) {sharpen};
\node[Square] at (25.500000, 17.450000) {};
\node[Kanji] at (25.500000, 17.950000) {\textcolor[HTML]{133c80}{戒}};
\node[Onyomi] at (25.550000, 17.550000) {カイ};
\node[Kunyomi] at (25.450000, 17.550000) {いまし.める};
\node[Meaning] at (25.500000, 19.200000) {commandment};
\node[Square] at (27.550000, 17.450000) {};
\node[Kanji] at (27.550000, 17.950000) {\textcolor[HTML]{14469c}{械}};
\node[Onyomi] at (27.600000, 17.550000) {カイ};
\node[Kunyomi] at (27.500000, 17.550000) {かせ};
\node[Meaning] at (27.550000, 19.200000) {contraption};
\node[Square] at (29.600000, 17.450000) {};
\node[Kanji] at (29.600000, 17.950000) {\textcolor[HTML]{1551b8}{鼻}};
\node[Onyomi] at (29.650000, 17.550000) {ビ};
\node[Kunyomi] at (29.550000, 17.550000) {はな};
\node[Meaning] at (29.600000, 19.200000) {nose};
\node[Square] at (31.650000, 17.450000) {};
\node[Kanji] at (31.650000, 17.950000) {\textcolor[HTML]{14418e}{刑}};
\node[Onyomi] at (31.700000, 17.550000) {ケイ};
\node[Meaning] at (31.650000, 19.200000) {punish};
\node[Square] at (33.700000, 17.450000) {};
\node[Kanji] at (33.700000, 17.950000) {\textcolor[HTML]{14469c}{型}};
\node[Onyomi] at (33.750000, 17.550000) {ケイ};
\node[Kunyomi] at (33.650000, 17.550000) {かた};
\node[Meaning] at (33.700000, 19.200000) {model};
\node[Square] at (35.750000, 17.450000) {};
\node[Kanji] at (35.750000, 17.950000) {\textcolor[HTML]{133c80}{才}};
\node[Onyomi] at (35.800000, 17.550000) {サイ};
\node[Meaning] at (35.750000, 19.200000) {genius};
\node[Square] at (37.800000, 17.450000) {};
\node[Kanji] at (37.800000, 17.950000) {\textcolor[HTML]{133c80}{財}};
\node[Onyomi] at (37.850000, 17.550000) {サイ};
\node[Meaning] at (37.800000, 19.200000) {wealth};
\node[Square] at (39.850000, 17.450000) {};
\node[Kanji] at (39.850000, 17.950000) {\textcolor[HTML]{14418e}{材}};
\node[Onyomi] at (39.900000, 17.550000) {ザイ};
\node[Meaning] at (39.850000, 19.200000) {lumber};
\node[Square] at (41.900000, 17.450000) {};
\node[Kanji] at (41.900000, 17.950000) {\textcolor[HTML]{14469c}{存}};
\node[Onyomi] at (41.950000, 17.550000) {ソン};
\node[Meaning] at (41.900000, 19.200000) {suppose};
\node[Square] at (43.950000, 17.450000) {};
\node[Kanji] at (43.950000, 17.950000) {\textcolor[HTML]{154caa}{在}};
\node[Onyomi] at (44.000000, 17.550000) {ザイ};
\node[Meaning] at (43.950000, 19.200000) {exist};
\node[Square] at (46.000000, 17.450000) {};
\node[Kanji] at (46.000000, 17.950000) {\textcolor[HTML]{0e254c}{乃}};
\node[Onyomi] at (46.050000, 17.550000) {ナイ};
\node[Kunyomi] at (45.950000, 17.550000) {すなわ};
\node[Meaning] at (46.000000, 19.200000) {from};
\node[Square] at (48.050000, 17.450000) {};
\node[Kanji] at (48.050000, 17.950000) {\textcolor[HTML]{14469c}{携}};
\node[Onyomi] at (48.100000, 17.550000) {ケイ};
\node[Kunyomi] at (48.000000, 17.550000) {たずさ.わる};
\node[Meaning] at (48.050000, 19.200000) {portable};
\node[Square] at (50.100000, 17.450000) {};
\node[Kanji] at (50.100000, 17.950000) {\textcolor[HTML]{133c80}{及}};
\node[Onyomi] at (50.150000, 17.550000) {キュウ};
\node[Kunyomi] at (50.050000, 17.550000) {およ.*};
\node[Meaning] at (50.100000, 19.200000) {reach};
\node[Square] at (52.150000, 17.450000) {};
\node[Kanji] at (52.150000, 17.950000) {\textcolor[HTML]{1551b8}{吸}};
\node[Onyomi] at (52.200000, 17.550000) {キュウ};
\node[Kunyomi] at (52.100000, 17.550000) {す.う};
\node[Meaning] at (52.150000, 19.200000) {suck};
\node[Square] at (54.200000, 17.450000) {};
\node[Kanji] at (54.200000, 17.950000) {\textcolor[HTML]{133c80}{扱}};
\node[Onyomi] at (54.250000, 17.550000) {キュウ};
\node[Kunyomi] at (54.150000, 17.550000) {あつか};
\node[Meaning] at (54.200000, 19.200000) {handle};
\node[Square] at (56.250000, 17.450000) {};
\node[Kanji] at (56.250000, 17.950000) {\textcolor[HTML]{1551b8}{丈}};
\node[Onyomi] at (56.300000, 17.550000) {ジョウ};
\node[Kunyomi] at (56.200000, 17.550000) {たけ};
\node[Meaning] at (56.250000, 19.200000) {height};
\node[Meaning] at (-58.500000, 18.000000) {42.65\%};
\node[Square] at (-56.500000, 15.400000) {};
\node[Kanji] at (-56.500000, 15.900000) {\textcolor[HTML]{14469c}{史}};
\node[Onyomi] at (-56.450000, 15.500000) {シ};
\node[Meaning] at (-56.500000, 17.150000) {history};
\node[Square] at (-54.450000, 15.400000) {};
\node[Kanji] at (-54.450000, 15.900000) {\textcolor[HTML]{0e254c}{吏}};
\node[Onyomi] at (-54.400000, 15.500000) {リ};
\node[Meaning] at (-54.450000, 17.150000) {officer};
\node[Square] at (-52.400000, 15.400000) {};
\node[Kanji] at (-52.400000, 15.900000) {\textcolor[HTML]{133c80}{更}};
\node[Onyomi] at (-52.350000, 15.500000) {コウ};
\node[Kunyomi] at (-52.450000, 15.500000) {さら};
\node[Meaning] at (-52.400000, 17.150000) {again};
\node[Square] at (-50.350000, 15.400000) {};
\node[Kanji] at (-50.350000, 15.900000) {\textcolor[HTML]{133c80}{硬}};
\node[Onyomi] at (-50.300000, 15.500000) {コウ};
\node[Kunyomi] at (-50.400000, 15.500000) {かた.い};
\node[Meaning] at (-50.350000, 17.150000) {stiff};
\node[Square] at (-48.300000, 15.400000) {};
\node[Kanji] at (-48.300000, 15.900000) {\textcolor[HTML]{0e254c}{又}};
\node[Kunyomi] at (-48.350000, 15.500000) {また};
\node[Meaning] at (-48.300000, 17.150000) {again};
\node[Square] at (-46.250000, 15.400000) {};
\node[Kanji] at (-46.250000, 15.900000) {\textcolor[HTML]{14418e}{双}};
\node[Onyomi] at (-46.200000, 15.500000) {ソウ};
\node[Kunyomi] at (-46.300000, 15.500000) {ふた};
\node[Meaning] at (-46.250000, 17.150000) {pair};
\node[Square] at (-44.200000, 15.400000) {};
\node[Kanji] at (-44.200000, 15.900000) {\textcolor[HTML]{0e254c}{桑}};
\node[Onyomi] at (-44.150000, 15.500000) {ソウ};
\node[Kunyomi] at (-44.250000, 15.500000) {くわ};
\node[Meaning] at (-44.200000, 17.150000) {mulberry};
\node[Square] at (-42.150000, 15.400000) {};
\node[Kanji] at (-42.150000, 15.900000) {\textcolor[HTML]{113066}{隻}};
\node[Onyomi] at (-42.100000, 15.500000) {セキ};
\node[Meaning] at (-42.150000, 17.150000) {ship counter};
\node[Square] at (-40.100000, 15.400000) {};
\node[Kanji] at (-40.100000, 15.900000) {\textcolor[HTML]{1551b8}{護}};
\node[Onyomi] at (-40.050000, 15.500000) {ゴ};
\node[Meaning] at (-40.100000, 17.150000) {defend};
\node[Square] at (-38.050000, 15.400000) {};
\node[Kanji] at (-38.050000, 15.900000) {\textcolor[HTML]{133c80}{獲}};
\node[Onyomi] at (-38.000000, 15.500000) {カク};
\node[Kunyomi] at (-38.100000, 15.500000) {え.る};
\node[Meaning] at (-38.050000, 17.150000) {seize};
\node[Square] at (-36.000000, 15.400000) {};
\node[Kanji] at (-36.000000, 15.900000) {\textcolor[HTML]{113066}{奴}};
\node[Onyomi] at (-35.950000, 15.500000) {ド};
\node[Kunyomi] at (-36.050000, 15.500000) {やつ};
\node[Meaning] at (-36.000000, 17.150000) {dude};
\node[Square] at (-33.950000, 15.400000) {};
\node[Kanji] at (-33.950000, 15.900000) {\textcolor[HTML]{1557c6}{怒}};
\node[Onyomi] at (-33.900000, 15.500000) {ド};
\node[Kunyomi] at (-34.000000, 15.500000) {おこ.る};
\node[Meaning] at (-33.950000, 17.150000) {angry};
\node[Square] at (-31.900000, 15.400000) {};
\node[Kanji] at (-31.900000, 15.900000) {\textcolor[HTML]{1551b8}{友}};
\node[Onyomi] at (-31.850000, 15.500000) {ユウ};
\node[Kunyomi] at (-31.950000, 15.500000) {とも};
\node[Meaning] at (-31.900000, 17.150000) {friend};
\node[Square] at (-29.850000, 15.400000) {};
\node[Kanji] at (-29.850000, 15.900000) {\textcolor[HTML]{1557c6}{抜}};
\node[Onyomi] at (-29.800000, 15.500000) {バツ};
\node[Kunyomi] at (-29.900000, 15.500000) {ぬ};
\node[Meaning] at (-29.850000, 17.150000) {extract};
\node[Square] at (-27.800000, 15.400000) {};
\node[Kanji] at (-27.800000, 15.900000) {\textcolor[HTML]{1551b8}{投}};
\node[Onyomi] at (-27.750000, 15.500000) {トウ};
\node[Kunyomi] at (-27.850000, 15.500000) {な};
\node[Meaning] at (-27.800000, 17.150000) {throw};
\node[Square] at (-25.750000, 15.400000) {};
\node[Kanji] at (-25.750000, 15.900000) {\textcolor[HTML]{133c80}{没}};
\node[Onyomi] at (-25.700000, 15.500000) {ボツ};
\node[Kunyomi] at (-25.800000, 15.500000) {おぼ};
\node[Meaning] at (-25.750000, 17.150000) {die};
\node[Square] at (-23.700000, 15.400000) {};
\node[Kanji] at (-23.700000, 15.900000) {\textcolor[HTML]{14469c}{設}};
\node[Onyomi] at (-23.650000, 15.500000) {セツ};
\node[Kunyomi] at (-23.750000, 15.500000) {もう.ける};
\node[Meaning] at (-23.700000, 17.150000) {establish};
\node[Square] at (-21.650000, 15.400000) {};
\node[Kanji] at (-21.650000, 15.900000) {\textcolor[HTML]{1551b8}{撃}};
\node[Onyomi] at (-21.600000, 15.500000) {ゲキ};
\node[Kunyomi] at (-21.700000, 15.500000) {う.つ};
\node[Meaning] at (-21.650000, 17.150000) {attack};
\node[Square] at (-19.600000, 15.400000) {};
\node[Kanji] at (-19.600000, 15.900000) {\textcolor[HTML]{123673}{殻}};
\node[Onyomi] at (-19.550000, 15.500000) {カク};
\node[Kunyomi] at (-19.650000, 15.500000) {から};
\node[Meaning] at (-19.600000, 17.150000) {husk};
\node[Square] at (-17.550000, 15.400000) {};
\node[Kanji] at (-17.550000, 15.900000) {\textcolor[HTML]{154caa}{支}};
\node[Onyomi] at (-17.500000, 15.500000) {シ};
\node[Kunyomi] at (-17.600000, 15.500000) {ささ.える};
\node[Meaning] at (-17.550000, 17.150000) {support};
\node[Square] at (-15.500000, 15.400000) {};
\node[Kanji] at (-15.500000, 15.900000) {\textcolor[HTML]{1551b8}{技}};
\node[Onyomi] at (-15.450000, 15.500000) {ギ};
\node[Kunyomi] at (-15.550000, 15.500000) {わざ};
\node[Meaning] at (-15.500000, 17.150000) {skill};
\node[Square] at (-13.450000, 15.400000) {};
\node[Kanji] at (-13.450000, 15.900000) {\textcolor[HTML]{14418e}{枝}};
\node[Onyomi] at (-13.400000, 15.500000) {シ};
\node[Kunyomi] at (-13.500000, 15.500000) {えだ};
\node[Meaning] at (-13.450000, 17.150000) {branch};
\node[Square] at (-11.400000, 15.400000) {};
\node[Kanji] at (-11.400000, 15.900000) {\textcolor[HTML]{102b59}{肢}};
\node[Onyomi] at (-11.350000, 15.500000) {シ};
\node[Meaning] at (-11.400000, 17.150000) {limb};
\node[Square] at (-9.350000, 15.400000) {};
\node[Kanji] at (-9.350000, 15.900000) {\textcolor[HTML]{113066}{茎}};
\node[Onyomi] at (-9.300000, 15.500000) {キョウ};
\node[Kunyomi] at (-9.400000, 15.500000) {くき};
\node[Meaning] at (-9.350000, 17.150000) {stem};
\node[Square] at (-7.300000, 15.400000) {};
\node[Kanji] at (-7.300000, 15.900000) {\textcolor[HTML]{154caa}{怪}};
\node[Onyomi] at (-7.250000, 15.500000) {カイ};
\node[Kunyomi] at (-7.350000, 15.500000) {あや.しい};
\node[Meaning] at (-7.300000, 17.150000) {suspicious};
\node[Square] at (-5.250000, 15.400000) {};
\node[Kanji] at (-5.250000, 15.900000) {\textcolor[HTML]{154caa}{軽}};
\node[Onyomi] at (-5.200000, 15.500000) {ケイ};
\node[Kunyomi] at (-5.300000, 15.500000) {かる};
\node[Meaning] at (-5.250000, 17.150000) {lightweight};
\node[Square] at (-3.200000, 15.400000) {};
\node[Kanji] at (-3.200000, 15.900000) {\textcolor[HTML]{0e254c}{叔}};
\node[Onyomi] at (-3.150000, 15.500000) {シュク};
\node[Meaning] at (-3.200000, 17.150000) {uncle};
\node[Square] at (-1.150000, 15.400000) {};
\node[Kanji] at (-1.150000, 15.900000) {\textcolor[HTML]{14469c}{督}};
\node[Onyomi] at (-1.100000, 15.500000) {トク};
\node[Meaning] at (-1.150000, 17.150000) {coach};
\node[Square] at (0.900000, 15.400000) {};
\node[Kanji] at (0.900000, 15.900000) {\textcolor[HTML]{14418e}{寂}};
\node[Onyomi] at (0.950000, 15.500000) {ジャク};
\node[Kunyomi] at (0.850000, 15.500000) {さび};
\node[Meaning] at (0.900000, 17.150000) {lonely};
\node[Square] at (2.950000, 15.400000) {};
\node[Kanji] at (2.950000, 15.900000) {\textcolor[HTML]{0e254c}{淑}};
\node[Onyomi] at (3.000000, 15.500000) {シュク};
\node[Kunyomi] at (2.900000, 15.500000) {しと};
\node[Meaning] at (2.950000, 17.150000) {graceful};
\node[Square] at (5.000000, 15.400000) {};
\node[Kanji] at (5.000000, 15.900000) {\textcolor[HTML]{1551b8}{反}};
\node[Onyomi] at (5.050000, 15.500000) {ハン};
\node[Meaning] at (5.000000, 17.150000) {anti};
\node[Square] at (7.050000, 15.400000) {};
\node[Kanji] at (7.050000, 15.900000) {\textcolor[HTML]{133c80}{坂}};
\node[Onyomi] at (7.100000, 15.500000) {ハン};
\node[Kunyomi] at (7.000000, 15.500000) {さか};
\node[Meaning] at (7.050000, 17.150000) {slope};
\node[Square] at (9.100000, 15.400000) {};
\node[Kanji] at (9.100000, 15.900000) {\textcolor[HTML]{14469c}{板}};
\node[Onyomi] at (9.150000, 15.500000) {ハン};
\node[Kunyomi] at (9.050000, 15.500000) {いた};
\node[Meaning] at (9.100000, 17.150000) {board};
\node[Square] at (11.150000, 15.400000) {};
\node[Kanji] at (11.150000, 15.900000) {\textcolor[HTML]{145cd5}{返}};
\node[Onyomi] at (11.200000, 15.500000) {ヘン};
\node[Kunyomi] at (11.100000, 15.500000) {かえ.る};
\node[Meaning] at (11.150000, 17.150000) {return};
\node[Square] at (13.200000, 15.400000) {};
\node[Kanji] at (13.200000, 15.900000) {\textcolor[HTML]{133c80}{販}};
\node[Onyomi] at (13.250000, 15.500000) {ハン};
\node[Meaning] at (13.200000, 17.150000) {sell};
\node[Square] at (15.250000, 15.400000) {};
\node[Kanji] at (15.250000, 15.900000) {\textcolor[HTML]{14418e}{爪}};
\node[Onyomi] at (15.300000, 15.500000) {ソウ};
\node[Kunyomi] at (15.200000, 15.500000) {つま};
\node[Meaning] at (15.250000, 17.150000) {claw};
\node[Square] at (17.300000, 15.400000) {};
\node[Kanji] at (17.300000, 15.900000) {\textcolor[HTML]{102b59}{妥}};
\node[Onyomi] at (17.350000, 15.500000) {ダ};
\node[Meaning] at (17.300000, 17.150000) {gentle};
\node[Square] at (19.350000, 15.400000) {};
\node[Kanji] at (19.350000, 15.900000) {\textcolor[HTML]{14418e}{乳}};
\node[Onyomi] at (19.400000, 15.500000) {ニュウ};
\node[Meaning] at (19.350000, 17.150000) {milk};
\node[Square] at (21.400000, 15.400000) {};
\node[Kanji] at (21.400000, 15.900000) {\textcolor[HTML]{1551b8}{浮}};
\node[Onyomi] at (21.450000, 15.500000) {フ};
\node[Kunyomi] at (21.350000, 15.500000) {う};
\node[Meaning] at (21.400000, 17.150000) {float};
\node[Square] at (23.450000, 15.400000) {};
\node[Kanji] at (23.450000, 15.900000) {\textcolor[HTML]{14418e}{将}};
\node[Onyomi] at (23.500000, 15.500000) {ショウ};
\node[Meaning] at (23.450000, 17.150000) {commander};
\node[Square] at (25.500000, 15.400000) {};
\node[Kanji] at (25.500000, 15.900000) {\textcolor[HTML]{113066}{奨}};
\node[Onyomi] at (25.550000, 15.500000) {ショウ};
\node[Meaning] at (25.500000, 17.150000) {encourage};
\node[Square] at (27.550000, 15.400000) {};
\node[Kanji] at (27.550000, 15.900000) {\textcolor[HTML]{133c80}{採}};
\node[Onyomi] at (27.600000, 15.500000) {サイ};
\node[Kunyomi] at (27.500000, 15.500000) {と.る};
\node[Meaning] at (27.550000, 17.150000) {gather};
\node[Square] at (29.600000, 15.400000) {};
\node[Kanji] at (29.600000, 15.900000) {\textcolor[HTML]{14418e}{菜}};
\node[Onyomi] at (29.650000, 15.500000) {サイ};
\node[Meaning] at (29.600000, 17.150000) {vegetable};
\node[Square] at (31.650000, 15.400000) {};
\node[Kanji] at (31.650000, 15.900000) {\textcolor[HTML]{1557c6}{受}};
\node[Onyomi] at (31.700000, 15.500000) {ジュ};
\node[Kunyomi] at (31.600000, 15.500000) {う};
\node[Meaning] at (31.650000, 17.150000) {accept};
\node[Square] at (33.700000, 15.400000) {};
\node[Kanji] at (33.700000, 15.900000) {\textcolor[HTML]{1557c6}{授}};
\node[Onyomi] at (33.750000, 15.500000) {ジュ};
\node[Kunyomi] at (33.650000, 15.500000) {さず.ける};
\node[Meaning] at (33.700000, 17.150000) {instruct};
\node[Square] at (35.750000, 15.400000) {};
\node[Kanji] at (35.750000, 15.900000) {\textcolor[HTML]{154caa}{愛}};
\node[Onyomi] at (35.800000, 15.500000) {アイ};
\node[Kunyomi] at (35.700000, 15.500000) {まな};
\node[Meaning] at (35.750000, 17.150000) {love};
\node[Square] at (37.800000, 15.400000) {};
\node[Kanji] at (37.800000, 15.900000) {\textcolor[HTML]{1551b8}{払}};
\node[Kunyomi] at (37.750000, 15.500000) {はら};
\node[Meaning] at (37.800000, 17.150000) {pay};
\node[Square] at (39.850000, 15.400000) {};
\node[Kanji] at (39.850000, 15.900000) {\textcolor[HTML]{1557c6}{広}};
\node[Onyomi] at (39.900000, 15.500000) {コウ};
\node[Kunyomi] at (39.800000, 15.500000) {ひろ};
\node[Meaning] at (39.850000, 17.150000) {wide};
\node[Square] at (41.900000, 15.400000) {};
\node[Kanji] at (41.900000, 15.900000) {\textcolor[HTML]{113066}{拡}};
\node[Onyomi] at (41.950000, 15.500000) {カク};
\node[Kunyomi] at (41.850000, 15.500000) {ひろ.がる};
\node[Meaning] at (41.900000, 17.150000) {extend};
\node[Square] at (43.950000, 15.400000) {};
\node[Kanji] at (43.950000, 15.900000) {\textcolor[HTML]{102b59}{鉱}};
\node[Onyomi] at (44.000000, 15.500000) {コウ};
\node[Kunyomi] at (43.900000, 15.500000) {あらがね};
\node[Meaning] at (43.950000, 17.150000) {mineral};
\node[Square] at (46.000000, 15.400000) {};
\node[Kanji] at (46.000000, 15.900000) {\textcolor[HTML]{133c80}{弁}};
\node[Onyomi] at (46.050000, 15.500000) {ベン};
\node[Meaning] at (46.000000, 17.150000) {dialect};
\node[Square] at (48.050000, 15.400000) {};
\node[Kanji] at (48.050000, 15.900000) {\textcolor[HTML]{14418e}{雄}};
\node[Onyomi] at (48.100000, 15.500000) {ユウ};
\node[Kunyomi] at (48.000000, 15.500000) {おす};
\node[Meaning] at (48.050000, 17.150000) {male};
\node[Square] at (50.100000, 15.400000) {};
\node[Kanji] at (50.100000, 15.900000) {\textcolor[HTML]{1551b8}{台}};
\node[Onyomi] at (50.150000, 15.500000) {ダイ};
\node[Meaning] at (50.100000, 17.150000) {machine};
\node[Square] at (52.150000, 15.400000) {};
\node[Kanji] at (52.150000, 15.900000) {\textcolor[HTML]{0e254c}{怠}};
\node[Onyomi] at (52.200000, 15.500000) {タイ};
\node[Kunyomi] at (52.100000, 15.500000) {おこた};
\node[Meaning] at (52.150000, 17.150000) {lazy};
\node[Square] at (54.200000, 15.400000) {};
\node[Kanji] at (54.200000, 15.900000) {\textcolor[HTML]{154caa}{治}};
\node[Onyomi] at (54.250000, 15.500000) {ジ};
\node[Kunyomi] at (54.150000, 15.500000) {なお.す};
\node[Meaning] at (54.200000, 17.150000) {cure};
\node[Square] at (56.250000, 15.400000) {};
\node[Kanji] at (56.250000, 15.900000) {\textcolor[HTML]{1557c6}{始}};
\node[Onyomi] at (56.300000, 15.500000) {シ};
\node[Kunyomi] at (56.200000, 15.500000) {はじ};
\node[Meaning] at (56.250000, 17.150000) {begin};
\node[Meaning] at (-58.500000, 15.950000) {44.49\%};
\node[Square] at (-56.500000, 13.350000) {};
\node[Kanji] at (-56.500000, 13.850000) {\textcolor[HTML]{0e254c}{胎}};
\node[Onyomi] at (-56.450000, 13.450000) {タイ};
\node[Meaning] at (-56.500000, 15.100000) {womb};
\node[Square] at (-54.450000, 13.350000) {};
\node[Kanji] at (-54.450000, 13.850000) {\textcolor[HTML]{1551b8}{窓}};
\node[Onyomi] at (-54.400000, 13.450000) {ソウ};
\node[Kunyomi] at (-54.500000, 13.450000) {まど};
\node[Meaning] at (-54.450000, 15.100000) {window};
\node[Square] at (-52.400000, 13.350000) {};
\node[Kanji] at (-52.400000, 13.850000) {\textcolor[HTML]{1557c6}{去}};
\node[Onyomi] at (-52.350000, 13.450000) {キョ};
\node[Kunyomi] at (-52.450000, 13.450000) {さ.る};
\node[Meaning] at (-52.400000, 15.100000) {past};
\node[Square] at (-50.350000, 13.350000) {};
\node[Kanji] at (-50.350000, 13.850000) {\textcolor[HTML]{1461e3}{法}};
\node[Onyomi] at (-50.300000, 13.450000) {ホウ};
\node[Meaning] at (-50.350000, 15.100000) {method};
\node[Square] at (-48.300000, 13.350000) {};
\node[Kanji] at (-48.300000, 13.850000) {\textcolor[HTML]{1461e3}{会}};
\node[Onyomi] at (-48.250000, 13.450000) {カイ};
\node[Kunyomi] at (-48.350000, 13.450000) {あ.う};
\node[Meaning] at (-48.300000, 15.100000) {meet};
\node[Square] at (-46.250000, 13.350000) {};
\node[Kanji] at (-46.250000, 13.850000) {\textcolor[HTML]{133c80}{至}};
\node[Onyomi] at (-46.200000, 13.450000) {シ};
\node[Kunyomi] at (-46.300000, 13.450000) {いた.る};
\node[Meaning] at (-46.250000, 15.100000) {attain};
\node[Square] at (-44.200000, 13.350000) {};
\node[Kanji] at (-44.200000, 13.850000) {\textcolor[HTML]{1557c6}{室}};
\node[Onyomi] at (-44.150000, 13.450000) {シツ};
\node[Meaning] at (-44.200000, 15.100000) {room};
\node[Square] at (-42.150000, 13.350000) {};
\node[Kanji] at (-42.150000, 13.850000) {\textcolor[HTML]{14469c}{到}};
\node[Onyomi] at (-42.100000, 13.450000) {トウ};
\node[Meaning] at (-42.150000, 15.100000) {arrival};
\node[Square] at (-40.100000, 13.350000) {};
\node[Kanji] at (-40.100000, 13.850000) {\textcolor[HTML]{14418e}{致}};
\node[Onyomi] at (-40.050000, 13.450000) {チ};
\node[Kunyomi] at (-40.150000, 13.450000) {いた.す};
\node[Meaning] at (-40.100000, 15.100000) {do};
\node[Square] at (-38.050000, 13.350000) {};
\node[Kanji] at (-38.050000, 13.850000) {\textcolor[HTML]{14469c}{互}};
\node[Onyomi] at (-38.000000, 13.450000) {ゴ};
\node[Kunyomi] at (-38.100000, 13.450000) {たが.い};
\node[Meaning] at (-38.050000, 15.100000) {mutual};
\node[Square] at (-36.000000, 13.350000) {};
\node[Kanji] at (-36.000000, 13.850000) {\textcolor[HTML]{123673}{棄}};
\node[Onyomi] at (-35.950000, 13.450000) {キ};
\node[Meaning] at (-36.000000, 15.100000) {abandon};
\node[Square] at (-33.950000, 13.350000) {};
\node[Kanji] at (-33.950000, 13.850000) {\textcolor[HTML]{1551b8}{育}};
\node[Onyomi] at (-33.900000, 13.450000) {イク};
\node[Kunyomi] at (-34.000000, 13.450000) {そだ};
\node[Meaning] at (-33.950000, 15.100000) {nurture};
\node[Square] at (-31.900000, 13.350000) {};
\node[Kanji] at (-31.900000, 13.850000) {\textcolor[HTML]{0e254c}{撤}};
\node[Onyomi] at (-31.850000, 13.450000) {テツ};
\node[Meaning] at (-31.900000, 15.100000) {withdrawal};
\node[Square] at (-29.850000, 13.350000) {};
\node[Kanji] at (-29.850000, 13.850000) {\textcolor[HTML]{123673}{充}};
\node[Onyomi] at (-29.800000, 13.450000) {ジュウ};
\node[Kunyomi] at (-29.900000, 13.450000) {あ.てる};
\node[Meaning] at (-29.850000, 15.100000) {allocate};
\node[Square] at (-27.800000, 13.350000) {};
\node[Kanji] at (-27.800000, 13.850000) {\textcolor[HTML]{133c80}{銃}};
\node[Onyomi] at (-27.750000, 13.450000) {ジュウ};
\node[Meaning] at (-27.800000, 15.100000) {gun};
\node[Square] at (-25.750000, 13.350000) {};
\node[Kanji] at (-25.750000, 13.850000) {\textcolor[HTML]{0e254c}{硫}};
\node[Onyomi] at (-25.700000, 13.450000) {リュウ};
\node[Meaning] at (-25.750000, 15.100000) {sulfur};
\node[Square] at (-23.700000, 13.350000) {};
\node[Kanji] at (-23.700000, 13.850000) {\textcolor[HTML]{1551b8}{流}};
\node[Onyomi] at (-23.650000, 13.450000) {リュウ};
\node[Kunyomi] at (-23.750000, 13.450000) {なが.*};
\node[Meaning] at (-23.700000, 15.100000) {stream};
\node[Square] at (-21.650000, 13.350000) {};
\node[Kanji] at (-21.650000, 13.850000) {\textcolor[HTML]{0e254c}{唆}};
\node[Onyomi] at (-21.600000, 13.450000) {サ};
\node[Kunyomi] at (-21.700000, 13.450000) {そそのか};
\node[Meaning] at (-21.650000, 15.100000) {instigate};
\node[Square] at (-19.600000, 13.350000) {};
\node[Kanji] at (-19.600000, 13.850000) {\textcolor[HTML]{2570ef}{出}};
\node[Onyomi] at (-19.550000, 13.450000) {シュツ};
\node[Kunyomi] at (-19.650000, 13.450000) {で.る};
\node[Meaning] at (-19.600000, 15.100000) {exit};
\node[Square] at (-17.550000, 13.350000) {};
\node[Kanji] at (-17.550000, 13.850000) {\textcolor[HTML]{1557c6}{山}};
\node[Onyomi] at (-17.500000, 13.450000) {サン};
\node[Kunyomi] at (-17.600000, 13.450000) {やま};
\node[Meaning] at (-17.550000, 15.100000) {mountain};
\node[Square] at (-15.500000, 13.350000) {};
\node[Kanji] at (-15.500000, 13.850000) {\textcolor[HTML]{0e254c}{拙}};
\node[Onyomi] at (-15.450000, 13.450000) {セツ};
\node[Kunyomi] at (-15.550000, 13.450000) {つたな};
\node[Meaning] at (-15.500000, 15.100000) {clumsy};
\node[Square] at (-13.450000, 13.350000) {};
\node[Kanji] at (-13.450000, 13.850000) {\textcolor[HTML]{154caa}{岩}};
\node[Onyomi] at (-13.400000, 13.450000) {ガン};
\node[Kunyomi] at (-13.500000, 13.450000) {いわ};
\node[Meaning] at (-13.450000, 15.100000) {boulder};
\node[Square] at (-11.400000, 13.350000) {};
\node[Kanji] at (-11.400000, 13.850000) {\textcolor[HTML]{133c80}{炭}};
\node[Onyomi] at (-11.350000, 13.450000) {タン};
\node[Kunyomi] at (-11.450000, 13.450000) {すみ};
\node[Meaning] at (-11.400000, 15.100000) {charcoal};
\node[Square] at (-9.350000, 13.350000) {};
\node[Kanji] at (-9.350000, 13.850000) {\textcolor[HTML]{113066}{岐}};
\node[Onyomi] at (-9.300000, 13.450000) {キ};
\node[Meaning] at (-9.350000, 15.100000) {branch off};
\node[Square] at (-7.300000, 13.350000) {};
\node[Kanji] at (-7.300000, 13.850000) {\textcolor[HTML]{0e254c}{峠}};
\node[Kunyomi] at (-7.350000, 13.450000) {とうげ};
\node[Meaning] at (-7.300000, 15.100000) {ridge};
\node[Square] at (-5.250000, 13.350000) {};
\node[Kanji] at (-5.250000, 13.850000) {\textcolor[HTML]{14469c}{崩}};
\node[Onyomi] at (-5.200000, 13.450000) {ホウ};
\node[Kunyomi] at (-5.300000, 13.450000) {くず.*};
\node[Meaning] at (-5.250000, 15.100000) {crumble};
\node[Square] at (-3.200000, 13.350000) {};
\node[Kanji] at (-3.200000, 13.850000) {\textcolor[HTML]{154caa}{密}};
\node[Onyomi] at (-3.150000, 13.450000) {ミツ};
\node[Kunyomi] at (-3.250000, 13.450000) {ひそ.か};
\node[Meaning] at (-3.200000, 15.100000) {secrecy};
\node[Square] at (-1.150000, 13.350000) {};
\node[Kanji] at (-1.150000, 13.850000) {\textcolor[HTML]{123673}{蜜}};
\node[Onyomi] at (-1.100000, 13.450000) {ミツ};
\node[Meaning] at (-1.150000, 15.100000) {honey};
\node[Square] at (0.900000, 13.350000) {};
\node[Kanji] at (0.900000, 13.850000) {\textcolor[HTML]{123673}{嵐}};
\node[Kunyomi] at (0.850000, 13.450000) {あらし};
\node[Meaning] at (0.900000, 15.100000) {storm};
\node[Square] at (2.950000, 13.350000) {};
\node[Kanji] at (2.950000, 13.850000) {\textcolor[HTML]{14418e}{崎}};
\node[Onyomi] at (3.000000, 13.450000) {キ};
\node[Kunyomi] at (2.900000, 13.450000) {さき};
\node[Meaning] at (2.950000, 15.100000) {cape};
\node[Square] at (5.000000, 13.350000) {};
\node[Kanji] at (5.000000, 13.850000) {\textcolor[HTML]{1968ed}{入}};
\node[Onyomi] at (5.050000, 13.450000) {ニュウ};
\node[Kunyomi] at (4.950000, 13.450000) {はい.る};
\node[Meaning] at (5.000000, 15.100000) {enter};
\node[Square] at (7.050000, 13.350000) {};
\node[Kanji] at (7.050000, 13.850000) {\textcolor[HTML]{1461e3}{込}};
\node[Kunyomi] at (7.000000, 13.450000) {こ};
\node[Meaning] at (7.050000, 15.100000) {crowded};
\node[Square] at (9.100000, 13.350000) {};
\node[Kanji] at (9.100000, 13.850000) {\textcolor[HTML]{1968ed}{分}};
\node[Onyomi] at (9.150000, 13.450000) {ブン};
\node[Kunyomi] at (9.050000, 13.450000) {わ.かる};
\node[Meaning] at (9.100000, 15.100000) {part};
\node[Square] at (11.150000, 13.350000) {};
\node[Kanji] at (11.150000, 13.850000) {\textcolor[HTML]{113066}{貧}};
\node[Onyomi] at (11.200000, 13.450000) {ビン};
\node[Kunyomi] at (11.100000, 13.450000) {まず.しい};
\node[Meaning] at (11.150000, 15.100000) {poor};
\node[Square] at (13.200000, 13.350000) {};
\node[Kanji] at (13.200000, 13.850000) {\textcolor[HTML]{0e254c}{頒}};
\node[Onyomi] at (13.250000, 13.450000) {ハン};
\node[Meaning] at (13.200000, 15.100000) {partition};
\node[Square] at (15.250000, 13.350000) {};
\node[Kanji] at (15.250000, 13.850000) {\textcolor[HTML]{154caa}{公}};
\node[Onyomi] at (15.300000, 13.450000) {コウ};
\node[Meaning] at (15.250000, 15.100000) {public};
\node[Square] at (17.300000, 13.350000) {};
\node[Kanji] at (17.300000, 13.850000) {\textcolor[HTML]{14469c}{松}};
\node[Onyomi] at (17.350000, 13.450000) {ショウ};
\node[Kunyomi] at (17.250000, 13.450000) {まつ};
\node[Meaning] at (17.300000, 15.100000) {pine};
\node[Square] at (19.350000, 13.350000) {};
\node[Kanji] at (19.350000, 13.850000) {\textcolor[HTML]{113066}{翁}};
\node[Onyomi] at (19.400000, 13.450000) {オウ};
\node[Meaning] at (19.350000, 15.100000) {old man};
\node[Square] at (21.400000, 13.350000) {};
\node[Kanji] at (21.400000, 13.850000) {\textcolor[HTML]{0e254c}{訟}};
\node[Onyomi] at (21.450000, 13.450000) {ショウ};
\node[Meaning] at (21.400000, 15.100000) {lawsuit};
\node[Square] at (23.450000, 13.350000) {};
\node[Kanji] at (23.450000, 13.850000) {\textcolor[HTML]{14469c}{谷}};
\node[Onyomi] at (23.500000, 13.450000) {コク};
\node[Kunyomi] at (23.400000, 13.450000) {たに};
\node[Meaning] at (23.450000, 15.100000) {valley};
\node[Square] at (25.500000, 13.350000) {};
\node[Kanji] at (25.500000, 13.850000) {\textcolor[HTML]{14418e}{浴}};
\node[Onyomi] at (25.550000, 13.450000) {ヨク};
\node[Kunyomi] at (25.450000, 13.450000) {あ};
\node[Meaning] at (25.500000, 15.100000) {bathe};
\node[Square] at (27.550000, 13.350000) {};
\node[Kanji] at (27.550000, 13.850000) {\textcolor[HTML]{154caa}{容}};
\node[Onyomi] at (27.600000, 13.450000) {ヨウ};
\node[Meaning] at (27.550000, 15.100000) {form};
\node[Square] at (29.600000, 13.350000) {};
\node[Kanji] at (29.600000, 13.850000) {\textcolor[HTML]{133c80}{溶}};
\node[Onyomi] at (29.650000, 13.450000) {ヨウ};
\node[Kunyomi] at (29.550000, 13.450000) {と.ける};
\node[Meaning] at (29.600000, 15.100000) {melt};
\node[Square] at (31.650000, 13.350000) {};
\node[Kanji] at (31.650000, 13.850000) {\textcolor[HTML]{14418e}{欲}};
\node[Onyomi] at (31.700000, 13.450000) {ヨク};
\node[Kunyomi] at (31.600000, 13.450000) {ほ.しい};
\node[Meaning] at (31.650000, 15.100000) {want};
\node[Square] at (33.700000, 13.350000) {};
\node[Kanji] at (33.700000, 13.850000) {\textcolor[HTML]{123673}{裕}};
\node[Onyomi] at (33.750000, 13.450000) {ユウ};
\node[Meaning] at (33.700000, 15.100000) {abundant};
\node[Square] at (35.750000, 13.350000) {};
\node[Kanji] at (35.750000, 13.850000) {\textcolor[HTML]{123673}{鉛}};
\node[Onyomi] at (35.800000, 13.450000) {エン};
\node[Kunyomi] at (35.700000, 13.450000) {なまり};
\node[Meaning] at (35.750000, 15.100000) {lead};
\node[Square] at (37.800000, 13.350000) {};
\node[Kanji] at (37.800000, 13.850000) {\textcolor[HTML]{133c80}{沿}};
\node[Onyomi] at (37.850000, 13.450000) {エン};
\node[Kunyomi] at (37.750000, 13.450000) {そ.う};
\node[Meaning] at (37.800000, 15.100000) {run alongside};
\node[Square] at (39.850000, 13.350000) {};
\node[Kanji] at (39.850000, 13.850000) {\textcolor[HTML]{154caa}{賞}};
\node[Onyomi] at (39.900000, 13.450000) {ショウ};
\node[Meaning] at (39.850000, 15.100000) {prize};
\node[Square] at (41.900000, 13.350000) {};
\node[Kanji] at (41.900000, 13.850000) {\textcolor[HTML]{14418e}{党}};
\node[Onyomi] at (41.950000, 13.450000) {トウ};
\node[Meaning] at (41.900000, 15.100000) {group};
\node[Square] at (43.950000, 13.350000) {};
\node[Kanji] at (43.950000, 13.850000) {\textcolor[HTML]{14418e}{堂}};
\node[Onyomi] at (44.000000, 13.450000) {ドウ};
\node[Meaning] at (43.950000, 15.100000) {hall};
\node[Square] at (46.000000, 13.350000) {};
\node[Kanji] at (46.000000, 13.850000) {\textcolor[HTML]{154caa}{常}};
\node[Onyomi] at (46.050000, 13.450000) {ジョウ};
\node[Kunyomi] at (45.950000, 13.450000) {つね};
\node[Meaning] at (46.000000, 15.100000) {normal};
\node[Square] at (48.050000, 13.350000) {};
\node[Kanji] at (48.050000, 13.850000) {\textcolor[HTML]{102b59}{掌}};
\node[Onyomi] at (48.100000, 13.450000) {ショウ};
\node[Kunyomi] at (48.000000, 13.450000) {てのひら};
\node[Meaning] at (48.050000, 15.100000) {manipulate};
\node[Square] at (50.100000, 13.350000) {};
\node[Kanji] at (50.100000, 13.850000) {\textcolor[HTML]{1551b8}{皮}};
\node[Onyomi] at (50.150000, 13.450000) {ヒ};
\node[Kunyomi] at (50.050000, 13.450000) {かわ};
\node[Meaning] at (50.100000, 15.100000) {skin};
\node[Square] at (52.150000, 13.350000) {};
\node[Kanji] at (52.150000, 13.850000) {\textcolor[HTML]{154caa}{波}};
\node[Onyomi] at (52.200000, 13.450000) {ハ};
\node[Kunyomi] at (52.100000, 13.450000) {なみ};
\node[Meaning] at (52.150000, 15.100000) {wave};
\node[Square] at (54.200000, 13.350000) {};
\node[Kanji] at (54.200000, 13.850000) {\textcolor[HTML]{123673}{婆}};
\node[Onyomi] at (54.250000, 13.450000) {バ};
\node[Kunyomi] at (54.150000, 13.450000) {ばあ};
\node[Meaning] at (54.200000, 15.100000) {old woman};
\node[Square] at (56.250000, 13.350000) {};
\node[Kanji] at (56.250000, 13.850000) {\textcolor[HTML]{0e254c}{披}};
\node[Onyomi] at (56.300000, 13.450000) {ヒ};
\node[Meaning] at (56.250000, 15.100000) {expose};
\node[Meaning] at (-58.500000, 13.900000) {48.90\%};
\node[Square] at (-56.500000, 11.300000) {};
\node[Kanji] at (-56.500000, 11.800000) {\textcolor[HTML]{1551b8}{破}};
\node[Onyomi] at (-56.450000, 11.400000) {ハ};
\node[Kunyomi] at (-56.550000, 11.400000) {やぶ.*};
\node[Meaning] at (-56.500000, 13.050000) {tear};
\node[Square] at (-54.450000, 11.300000) {};
\node[Kanji] at (-54.450000, 11.800000) {\textcolor[HTML]{154caa}{被}};
\node[Onyomi] at (-54.400000, 11.400000) {ヒ};
\node[Kunyomi] at (-54.500000, 11.400000) {かぶ.る};
\node[Meaning] at (-54.450000, 13.050000) {incur};
\node[Square] at (-52.400000, 11.300000) {};
\node[Kanji] at (-52.400000, 11.800000) {\textcolor[HTML]{1557c6}{残}};
\node[Onyomi] at (-52.350000, 11.400000) {ザン};
\node[Kunyomi] at (-52.450000, 11.400000) {のこ.*};
\node[Meaning] at (-52.400000, 13.050000) {remainder};
\node[Square] at (-50.350000, 11.300000) {};
\node[Kanji] at (-50.350000, 11.800000) {\textcolor[HTML]{0e254c}{殉}};
\node[Onyomi] at (-50.300000, 11.400000) {ジュン};
\node[Meaning] at (-50.350000, 13.050000) {martyr};
\node[Square] at (-48.300000, 11.300000) {};
\node[Kanji] at (-48.300000, 11.800000) {\textcolor[HTML]{102b59}{殊}};
\node[Onyomi] at (-48.250000, 11.400000) {シュ};
\node[Kunyomi] at (-48.350000, 11.400000) {こと};
\node[Meaning] at (-48.300000, 13.050000) {especially};
\node[Square] at (-46.250000, 11.300000) {};
\node[Kanji] at (-46.250000, 11.800000) {\textcolor[HTML]{123673}{殖}};
\node[Onyomi] at (-46.200000, 11.400000) {ショク};
\node[Kunyomi] at (-46.300000, 11.400000) {ふ.える};
\node[Meaning] at (-46.250000, 13.050000) {multiply};
\node[Square] at (-44.200000, 11.300000) {};
\node[Kanji] at (-44.200000, 11.800000) {\textcolor[HTML]{154caa}{列}};
\node[Onyomi] at (-44.150000, 11.400000) {レツ};
\node[Meaning] at (-44.200000, 13.050000) {row};
\node[Square] at (-42.150000, 11.300000) {};
\node[Kanji] at (-42.150000, 11.800000) {\textcolor[HTML]{14469c}{裂}};
\node[Onyomi] at (-42.100000, 11.400000) {レツ};
\node[Kunyomi] at (-42.200000, 11.400000) {さ.*};
\node[Meaning] at (-42.150000, 13.050000) {split};
\node[Square] at (-40.100000, 11.300000) {};
\node[Kanji] at (-40.100000, 11.800000) {\textcolor[HTML]{133c80}{烈}};
\node[Onyomi] at (-40.050000, 11.400000) {レツ};
\node[Kunyomi] at (-40.150000, 11.400000) {はげ.しい};
\node[Meaning] at (-40.100000, 13.050000) {violent};
\node[Square] at (-38.050000, 11.300000) {};
\node[Kanji] at (-38.050000, 11.800000) {\textcolor[HTML]{145cd5}{死}};
\node[Onyomi] at (-38.000000, 11.400000) {シ};
\node[Kunyomi] at (-38.100000, 11.400000) {し.ぬ};
\node[Meaning] at (-38.050000, 13.050000) {death};
\node[Square] at (-36.000000, 11.300000) {};
\node[Kanji] at (-36.000000, 11.800000) {\textcolor[HTML]{133c80}{葬}};
\node[Onyomi] at (-35.950000, 11.400000) {ソウ};
\node[Kunyomi] at (-36.050000, 11.400000) {ほうむ.る};
\node[Meaning] at (-36.000000, 13.050000) {burial};
\node[Square] at (-33.950000, 11.300000) {};
\node[Kanji] at (-33.950000, 11.800000) {\textcolor[HTML]{1551b8}{瞬}};
\node[Onyomi] at (-33.900000, 11.400000) {シュン};
\node[Kunyomi] at (-34.000000, 11.400000) {またた.く};
\node[Meaning] at (-33.950000, 13.050000) {blink};
\node[Square] at (-31.900000, 11.300000) {};
\node[Kanji] at (-31.900000, 11.800000) {\textcolor[HTML]{1551b8}{耳}};
\node[Onyomi] at (-31.850000, 11.400000) {ジ};
\node[Kunyomi] at (-31.950000, 11.400000) {みみ};
\node[Meaning] at (-31.900000, 13.050000) {ear};
\node[Square] at (-29.850000, 11.300000) {};
\node[Kanji] at (-29.850000, 11.800000) {\textcolor[HTML]{1461e3}{取}};
\node[Onyomi] at (-29.800000, 11.400000) {シュ};
\node[Kunyomi] at (-29.900000, 11.400000) {と};
\node[Meaning] at (-29.850000, 13.050000) {take};
\node[Square] at (-27.800000, 11.300000) {};
\node[Kanji] at (-27.800000, 11.800000) {\textcolor[HTML]{123673}{趣}};
\node[Onyomi] at (-27.750000, 11.400000) {シュ};
\node[Kunyomi] at (-27.850000, 11.400000) {おもむき};
\node[Meaning] at (-27.800000, 13.050000) {gist};
\node[Square] at (-25.750000, 11.300000) {};
\node[Kanji] at (-25.750000, 11.800000) {\textcolor[HTML]{145cd5}{最}};
\node[Onyomi] at (-25.700000, 11.400000) {サイ};
\node[Kunyomi] at (-25.800000, 11.400000) {もっと};
\node[Meaning] at (-25.750000, 13.050000) {most};
\node[Square] at (-23.700000, 11.300000) {};
\node[Kanji] at (-23.700000, 11.800000) {\textcolor[HTML]{14469c}{撮}};
\node[Onyomi] at (-23.650000, 11.400000) {サツ};
\node[Kunyomi] at (-23.750000, 11.400000) {と.る};
\node[Meaning] at (-23.700000, 13.050000) {photograph};
\node[Square] at (-21.650000, 11.300000) {};
\node[Kanji] at (-21.650000, 11.800000) {\textcolor[HTML]{14418e}{恥}};
\node[Onyomi] at (-21.600000, 11.400000) {チ};
\node[Kunyomi] at (-21.700000, 11.400000) {は};
\node[Meaning] at (-21.650000, 13.050000) {shame};
\node[Square] at (-19.600000, 11.300000) {};
\node[Kanji] at (-19.600000, 11.800000) {\textcolor[HTML]{154caa}{職}};
\node[Onyomi] at (-19.550000, 11.400000) {ショク};
\node[Meaning] at (-19.600000, 13.050000) {employment};
\node[Square] at (-17.550000, 11.300000) {};
\node[Kanji] at (-17.550000, 11.800000) {\textcolor[HTML]{14418e}{聖}};
\node[Onyomi] at (-17.500000, 11.400000) {セイ};
\node[Meaning] at (-17.550000, 13.050000) {holy};
\node[Square] at (-15.500000, 11.300000) {};
\node[Kanji] at (-15.500000, 11.800000) {\textcolor[HTML]{133c80}{敢}};
\node[Onyomi] at (-15.450000, 11.400000) {カン};
\node[Kunyomi] at (-15.550000, 11.400000) {あ};
\node[Meaning] at (-15.500000, 13.050000) {daring};
\node[Square] at (-13.450000, 11.300000) {};
\node[Kanji] at (-13.450000, 11.800000) {\textcolor[HTML]{133c80}{聴}};
\node[Onyomi] at (-13.400000, 11.400000) {チョウ};
\node[Kunyomi] at (-13.500000, 11.400000) {き.く};
\node[Meaning] at (-13.450000, 13.050000) {listen};
\node[Square] at (-11.400000, 11.300000) {};
\node[Kanji] at (-11.400000, 11.800000) {\textcolor[HTML]{133c80}{懐}};
\node[Onyomi] at (-11.350000, 11.400000) {カイ};
\node[Kunyomi] at (-11.450000, 11.400000) {なつ};
\node[Meaning] at (-11.400000, 13.050000) {nostalgia};
\node[Square] at (-9.350000, 11.300000) {};
\node[Kanji] at (-9.350000, 11.800000) {\textcolor[HTML]{14418e}{慢}};
\node[Onyomi] at (-9.300000, 11.400000) {マン};
\node[Meaning] at (-9.350000, 13.050000) {ridicule};
\node[Square] at (-7.300000, 11.300000) {};
\node[Kanji] at (-7.300000, 11.800000) {\textcolor[HTML]{123673}{漫}};
\node[Onyomi] at (-7.250000, 11.400000) {マン};
\node[Meaning] at (-7.300000, 13.050000) {manga};
\node[Square] at (-5.250000, 11.300000) {};
\node[Kanji] at (-5.250000, 11.800000) {\textcolor[HTML]{1551b8}{買}};
\node[Onyomi] at (-5.200000, 11.400000) {バイ};
\node[Kunyomi] at (-5.300000, 11.400000) {か};
\node[Meaning] at (-5.250000, 13.050000) {buy};
\node[Square] at (-3.200000, 11.300000) {};
\node[Kanji] at (-3.200000, 11.800000) {\textcolor[HTML]{1557c6}{置}};
\node[Onyomi] at (-3.150000, 11.400000) {チ};
\node[Kunyomi] at (-3.250000, 11.400000) {お.く};
\node[Meaning] at (-3.200000, 13.050000) {put};
\node[Square] at (-1.150000, 11.300000) {};
\node[Kanji] at (-1.150000, 11.800000) {\textcolor[HTML]{14469c}{罰}};
\node[Onyomi] at (-1.100000, 11.400000) {バツ};
\node[Kunyomi] at (-1.200000, 11.400000) {ばつ};
\node[Meaning] at (-1.150000, 13.050000) {penalty};
\node[Square] at (0.900000, 11.300000) {};
\node[Kanji] at (0.900000, 11.800000) {\textcolor[HTML]{133c80}{寧}};
\node[Onyomi] at (0.950000, 11.400000) {ネイ};
\node[Kunyomi] at (0.850000, 11.400000) {むし.ろ};
\node[Meaning] at (0.900000, 13.050000) {rather};
\node[Square] at (2.950000, 11.300000) {};
\node[Kanji] at (2.950000, 11.800000) {\textcolor[HTML]{113066}{濁}};
\node[Onyomi] at (3.000000, 11.400000) {ダク};
\node[Kunyomi] at (2.900000, 11.400000) {にご};
\node[Meaning] at (2.950000, 13.050000) {muddy};
\node[Square] at (5.000000, 11.300000) {};
\node[Kanji] at (5.000000, 11.800000) {\textcolor[HTML]{14469c}{環}};
\node[Onyomi] at (5.050000, 11.400000) {カン};
\node[Meaning] at (5.000000, 13.050000) {loop};
\node[Square] at (7.050000, 11.300000) {};
\node[Kanji] at (7.050000, 11.800000) {\textcolor[HTML]{123673}{還}};
\node[Onyomi] at (7.100000, 11.400000) {カン};
\node[Kunyomi] at (7.000000, 11.400000) {かえ.る};
\node[Meaning] at (7.050000, 13.050000) {send back};
\node[Square] at (9.100000, 11.300000) {};
\node[Kanji] at (9.100000, 11.800000) {\textcolor[HTML]{1557c6}{夫}};
\node[Onyomi] at (9.150000, 11.400000) {フウ};
\node[Kunyomi] at (9.050000, 11.400000) {おっと};
\node[Meaning] at (9.100000, 13.050000) {husband};
\node[Square] at (11.150000, 11.300000) {};
\node[Kanji] at (11.150000, 11.800000) {\textcolor[HTML]{0e254c}{扶}};
\node[Onyomi] at (11.200000, 11.400000) {フ};
\node[Kunyomi] at (11.100000, 11.400000) {たす};
\node[Meaning] at (11.150000, 13.050000) {aid};
\node[Square] at (13.200000, 11.300000) {};
\node[Kanji] at (13.200000, 11.800000) {\textcolor[HTML]{0e254c}{渓}};
\node[Onyomi] at (13.250000, 11.400000) {ケイ};
\node[Kunyomi] at (13.150000, 11.400000) {たに};
\node[Meaning] at (13.200000, 13.050000) {valley};
\node[Square] at (15.250000, 11.300000) {};
\node[Kanji] at (15.250000, 11.800000) {\textcolor[HTML]{14469c}{規}};
\node[Onyomi] at (15.300000, 11.400000) {キ};
\node[Meaning] at (15.250000, 13.050000) {standard};
\node[Square] at (17.300000, 11.300000) {};
\node[Kanji] at (17.300000, 11.800000) {\textcolor[HTML]{14469c}{替}};
\node[Onyomi] at (17.350000, 11.400000) {タイ};
\node[Kunyomi] at (17.250000, 11.400000) {か};
\node[Meaning] at (17.300000, 13.050000) {replace};
\node[Square] at (19.350000, 11.300000) {};
\node[Kanji] at (19.350000, 11.800000) {\textcolor[HTML]{14418e}{賛}};
\node[Onyomi] at (19.400000, 11.400000) {サン};
\node[Meaning] at (19.350000, 13.050000) {agree};
\node[Square] at (21.400000, 11.300000) {};
\node[Kanji] at (21.400000, 11.800000) {\textcolor[HTML]{14418e}{潜}};
\node[Onyomi] at (21.450000, 11.400000) {セン};
\node[Kunyomi] at (21.350000, 11.400000) {くぐ.る};
\node[Meaning] at (21.400000, 13.050000) {conceal};
\node[Square] at (23.450000, 11.300000) {};
\node[Kanji] at (23.450000, 11.800000) {\textcolor[HTML]{1551b8}{失}};
\node[Onyomi] at (23.500000, 11.400000) {シツ};
\node[Kunyomi] at (23.400000, 11.400000) {うしな.う};
\node[Meaning] at (23.450000, 13.050000) {fault};
\node[Square] at (25.500000, 11.300000) {};
\node[Kanji] at (25.500000, 11.800000) {\textcolor[HTML]{14469c}{鉄}};
\node[Onyomi] at (25.550000, 11.400000) {テツ};
\node[Meaning] at (25.500000, 13.050000) {iron};
\node[Square] at (27.550000, 11.300000) {};
\node[Kanji] at (27.550000, 11.800000) {\textcolor[HTML]{0e254c}{迭}};
\node[Onyomi] at (27.600000, 11.400000) {テツ};
\node[Meaning] at (27.550000, 13.050000) {alternate};
\node[Square] at (29.600000, 11.300000) {};
\node[Kanji] at (29.600000, 11.800000) {\textcolor[HTML]{154caa}{臣}};
\node[Onyomi] at (29.650000, 11.400000) {シン};
\node[Meaning] at (29.600000, 13.050000) {servant};
\node[Square] at (31.650000, 11.300000) {};
\node[Kanji] at (31.650000, 11.800000) {\textcolor[HTML]{113066}{姫}};
\node[Kunyomi] at (31.600000, 11.400000) {ひめ};
\node[Meaning] at (31.650000, 13.050000) {princess};
\node[Square] at (33.700000, 11.300000) {};
\node[Kanji] at (33.700000, 11.800000) {\textcolor[HTML]{133c80}{蔵}};
\node[Onyomi] at (33.750000, 11.400000) {ゾウ};
\node[Kunyomi] at (33.650000, 11.400000) {くら};
\node[Meaning] at (33.700000, 13.050000) {storehouse};
\node[Square] at (35.750000, 11.300000) {};
\node[Kanji] at (35.750000, 11.800000) {\textcolor[HTML]{14469c}{臓}};
\node[Onyomi] at (35.800000, 11.400000) {ゾウ};
\node[Meaning] at (35.750000, 13.050000) {internal organs};
\node[Square] at (37.800000, 11.300000) {};
\node[Kanji] at (37.800000, 11.800000) {\textcolor[HTML]{14418e}{賢}};
\node[Onyomi] at (37.850000, 11.400000) {ケン};
\node[Kunyomi] at (37.750000, 11.400000) {かしこ.い};
\node[Meaning] at (37.800000, 13.050000) {clever};
\node[Square] at (39.850000, 11.300000) {};
\node[Kanji] at (39.850000, 11.800000) {\textcolor[HTML]{113066}{堅}};
\node[Onyomi] at (39.900000, 11.400000) {ケン};
\node[Kunyomi] at (39.800000, 11.400000) {かた};
\node[Meaning] at (39.850000, 13.050000) {solid};
\node[Square] at (41.900000, 11.300000) {};
\node[Kanji] at (41.900000, 11.800000) {\textcolor[HTML]{113066}{臨}};
\node[Onyomi] at (41.950000, 11.400000) {リン};
\node[Kunyomi] at (41.850000, 11.400000) {のぞ.む};
\node[Meaning] at (41.900000, 13.050000) {look to};
\node[Square] at (43.950000, 11.300000) {};
\node[Kanji] at (43.950000, 11.800000) {\textcolor[HTML]{133c80}{覧}};
\node[Onyomi] at (44.000000, 11.400000) {ラン};
\node[Meaning] at (43.950000, 13.050000) {look at};
\node[Square] at (46.000000, 11.300000) {};
\node[Kanji] at (46.000000, 11.800000) {\textcolor[HTML]{1551b8}{巨}};
\node[Onyomi] at (46.050000, 11.400000) {キョ};
\node[Meaning] at (46.000000, 13.050000) {giant};
\node[Square] at (48.050000, 11.300000) {};
\node[Kanji] at (48.050000, 11.800000) {\textcolor[HTML]{123673}{拒}};
\node[Onyomi] at (48.100000, 11.400000) {キョ};
\node[Kunyomi] at (48.000000, 11.400000) {こば.む};
\node[Meaning] at (48.050000, 13.050000) {refusal};
\node[Square] at (50.100000, 11.300000) {};
\node[Kanji] at (50.100000, 11.800000) {\textcolor[HTML]{145cd5}{力}};
\node[Onyomi] at (50.150000, 11.400000) {リョク};
\node[Kunyomi] at (50.050000, 11.400000) {ちから};
\node[Meaning] at (50.100000, 13.050000) {power};
\node[Square] at (52.150000, 11.300000) {};
\node[Kanji] at (52.150000, 11.800000) {\textcolor[HTML]{145cd5}{男}};
\node[Onyomi] at (52.200000, 11.400000) {ダン};
\node[Kunyomi] at (52.100000, 11.400000) {おとこ};
\node[Meaning] at (52.150000, 13.050000) {man};
\node[Square] at (54.200000, 11.300000) {};
\node[Kanji] at (54.200000, 11.800000) {\textcolor[HTML]{14469c}{労}};
\node[Onyomi] at (54.250000, 11.400000) {ロウ};
\node[Kunyomi] at (54.150000, 11.400000) {いたわ.る};
\node[Meaning] at (54.200000, 13.050000) {labor};
\node[Square] at (56.250000, 11.300000) {};
\node[Kanji] at (56.250000, 11.800000) {\textcolor[HTML]{123673}{募}};
\node[Onyomi] at (56.300000, 11.400000) {ボ};
\node[Kunyomi] at (56.200000, 11.400000) {つの.る};
\node[Meaning] at (56.250000, 13.050000) {recruit};
\node[Meaning] at (-58.500000, 11.850000) {51.32\%};
\node[Square] at (-56.500000, 9.250000) {};
\node[Kanji] at (-56.500000, 9.750000) {\textcolor[HTML]{123673}{劣}};
\node[Onyomi] at (-56.450000, 9.350000) {レツ};
\node[Kunyomi] at (-56.550000, 9.350000) {おと.る};
\node[Meaning] at (-56.500000, 11.000000) {inferiority};
\node[Square] at (-54.450000, 9.250000) {};
\node[Kanji] at (-54.450000, 9.750000) {\textcolor[HTML]{14418e}{功}};
\node[Onyomi] at (-54.400000, 9.350000) {コウ};
\node[Meaning] at (-54.450000, 11.000000) {achievement};
\node[Square] at (-52.400000, 9.250000) {};
\node[Kanji] at (-52.400000, 9.750000) {\textcolor[HTML]{133c80}{勧}};
\node[Onyomi] at (-52.350000, 9.350000) {カン};
\node[Kunyomi] at (-52.450000, 9.350000) {すす.める};
\node[Meaning] at (-52.400000, 11.000000) {recommend};
\node[Square] at (-50.350000, 9.250000) {};
\node[Kanji] at (-50.350000, 9.750000) {\textcolor[HTML]{14418e}{努}};
\node[Onyomi] at (-50.300000, 9.350000) {ド};
\node[Kunyomi] at (-50.400000, 9.350000) {つと.める};
\node[Meaning] at (-50.350000, 11.000000) {toil};
\node[Square] at (-48.300000, 9.250000) {};
\node[Kanji] at (-48.300000, 9.750000) {\textcolor[HTML]{133c80}{励}};
\node[Onyomi] at (-48.250000, 9.350000) {レイ};
\node[Kunyomi] at (-48.350000, 9.350000) {はげ.*};
\node[Meaning] at (-48.300000, 11.000000) {encourage};
\node[Square] at (-46.250000, 9.250000) {};
\node[Kanji] at (-46.250000, 9.750000) {\textcolor[HTML]{1551b8}{加}};
\node[Onyomi] at (-46.200000, 9.350000) {カ};
\node[Kunyomi] at (-46.300000, 9.350000) {くわ.える};
\node[Meaning] at (-46.250000, 11.000000) {add};
\node[Square] at (-44.200000, 9.250000) {};
\node[Kanji] at (-44.200000, 9.750000) {\textcolor[HTML]{133c80}{賀}};
\node[Onyomi] at (-44.150000, 9.350000) {ガ};
\node[Meaning] at (-44.200000, 11.000000) {congratulate};
\node[Square] at (-42.150000, 9.250000) {};
\node[Kanji] at (-42.150000, 9.750000) {\textcolor[HTML]{113066}{架}};
\node[Onyomi] at (-42.100000, 9.350000) {カ};
\node[Kunyomi] at (-42.200000, 9.350000) {か.*};
\node[Meaning] at (-42.150000, 11.000000) {shelf};
\node[Square] at (-40.100000, 9.250000) {};
\node[Kanji] at (-40.100000, 9.750000) {\textcolor[HTML]{154caa}{脇}};
\node[Onyomi] at (-40.050000, 9.350000) {キョウ};
\node[Kunyomi] at (-40.150000, 9.350000) {わき};
\node[Meaning] at (-40.100000, 11.000000) {armpit};
\node[Square] at (-38.050000, 9.250000) {};
\node[Kanji] at (-38.050000, 9.750000) {\textcolor[HTML]{14418e}{脅}};
\node[Onyomi] at (-38.000000, 9.350000) {キョウ};
\node[Kunyomi] at (-38.100000, 9.350000) {おど};
\node[Meaning] at (-38.050000, 11.000000) {threaten};
\node[Square] at (-36.000000, 9.250000) {};
\node[Kanji] at (-36.000000, 9.750000) {\textcolor[HTML]{14469c}{協}};
\node[Onyomi] at (-35.950000, 9.350000) {キョウ};
\node[Meaning] at (-36.000000, 11.000000) {cooperation};
\node[Square] at (-33.950000, 9.250000) {};
\node[Kanji] at (-33.950000, 9.750000) {\textcolor[HTML]{1968ed}{行}};
\node[Onyomi] at (-33.900000, 9.350000) {コウ};
\node[Kunyomi] at (-34.000000, 9.350000) {い.く};
\node[Meaning] at (-33.950000, 11.000000) {go};
\node[Square] at (-31.900000, 9.250000) {};
\node[Kanji] at (-31.900000, 9.750000) {\textcolor[HTML]{14418e}{律}};
\node[Onyomi] at (-31.850000, 9.350000) {リツ};
\node[Meaning] at (-31.900000, 11.000000) {law};
\node[Square] at (-29.850000, 9.250000) {};
\node[Kanji] at (-29.850000, 9.750000) {\textcolor[HTML]{14469c}{復}};
\node[Onyomi] at (-29.800000, 9.350000) {フク};
\node[Meaning] at (-29.850000, 11.000000) {restore};
\node[Square] at (-27.800000, 9.250000) {};
\node[Kanji] at (-27.800000, 9.750000) {\textcolor[HTML]{154caa}{得}};
\node[Onyomi] at (-27.750000, 9.350000) {トク};
\node[Kunyomi] at (-27.850000, 9.350000) {え.る};
\node[Meaning] at (-27.800000, 11.000000) {acquire};
\node[Square] at (-25.750000, 9.250000) {};
\node[Kanji] at (-25.750000, 9.750000) {\textcolor[HTML]{14469c}{従}};
\node[Onyomi] at (-25.700000, 9.350000) {ジュウ};
\node[Kunyomi] at (-25.800000, 9.350000) {したが.う};
\node[Meaning] at (-25.750000, 11.000000) {obey};
\node[Square] at (-23.700000, 9.250000) {};
\node[Kanji] at (-23.700000, 9.750000) {\textcolor[HTML]{1557c6}{徒}};
\node[Onyomi] at (-23.650000, 9.350000) {ト};
\node[Meaning] at (-23.700000, 11.000000) {junior};
\node[Square] at (-21.650000, 9.250000) {};
\node[Kanji] at (-21.650000, 9.750000) {\textcolor[HTML]{1557c6}{待}};
\node[Onyomi] at (-21.600000, 9.350000) {タイ};
\node[Kunyomi] at (-21.700000, 9.350000) {ま};
\node[Meaning] at (-21.650000, 11.000000) {wait};
\node[Square] at (-19.600000, 9.250000) {};
\node[Kanji] at (-19.600000, 9.750000) {\textcolor[HTML]{14418e}{往}};
\node[Onyomi] at (-19.550000, 9.350000) {オウ};
\node[Meaning] at (-19.600000, 11.000000) {depart};
\node[Square] at (-17.550000, 9.250000) {};
\node[Kanji] at (-17.550000, 9.750000) {\textcolor[HTML]{113066}{征}};
\node[Onyomi] at (-17.500000, 9.350000) {セイ};
\node[Meaning] at (-17.550000, 11.000000) {subjugate};
\node[Square] at (-15.500000, 9.250000) {};
\node[Kanji] at (-15.500000, 9.750000) {\textcolor[HTML]{123673}{径}};
\node[Onyomi] at (-15.450000, 9.350000) {ケイ};
\node[Meaning] at (-15.500000, 11.000000) {diameter};
\node[Square] at (-13.450000, 9.250000) {};
\node[Kanji] at (-13.450000, 9.750000) {\textcolor[HTML]{1551b8}{彼}};
\node[Onyomi] at (-13.400000, 9.350000) {ヒ};
\node[Kunyomi] at (-13.500000, 9.350000) {かれ};
\node[Meaning] at (-13.450000, 11.000000) {he};
\node[Square] at (-11.400000, 9.250000) {};
\node[Kanji] at (-11.400000, 9.750000) {\textcolor[HTML]{1551b8}{役}};
\node[Onyomi] at (-11.350000, 9.350000) {ヤク};
\node[Meaning] at (-11.400000, 11.000000) {service};
\node[Square] at (-9.350000, 9.250000) {};
\node[Kanji] at (-9.350000, 9.750000) {\textcolor[HTML]{113066}{徳}};
\node[Onyomi] at (-9.300000, 9.350000) {トク};
\node[Meaning] at (-9.350000, 11.000000) {virtue};
\node[Square] at (-7.300000, 9.250000) {};
\node[Kanji] at (-7.300000, 9.750000) {\textcolor[HTML]{113066}{徹}};
\node[Onyomi] at (-7.250000, 9.350000) {テツ};
\node[Kunyomi] at (-7.350000, 9.350000) {てっ.する};
\node[Meaning] at (-7.300000, 11.000000) {penetrate};
\node[Square] at (-5.250000, 9.250000) {};
\node[Kanji] at (-5.250000, 9.750000) {\textcolor[HTML]{123673}{徴}};
\node[Onyomi] at (-5.200000, 9.350000) {チョウ};
\node[Meaning] at (-5.250000, 11.000000) {indication};
\node[Square] at (-3.200000, 9.250000) {};
\node[Kanji] at (-3.200000, 9.750000) {\textcolor[HTML]{113066}{懲}};
\node[Onyomi] at (-3.150000, 9.350000) {チョウ};
\node[Kunyomi] at (-3.250000, 9.350000) {こ.りる};
\node[Meaning] at (-3.200000, 11.000000) {chastise};
\node[Square] at (-1.150000, 9.250000) {};
\node[Kanji] at (-1.150000, 9.750000) {\textcolor[HTML]{154caa}{微}};
\node[Onyomi] at (-1.100000, 9.350000) {ビ};
\node[Kunyomi] at (-1.200000, 9.350000) {かす.か};
\node[Meaning] at (-1.150000, 11.000000) {delicate};
\node[Square] at (0.900000, 9.250000) {};
\node[Kanji] at (0.900000, 9.750000) {\textcolor[HTML]{14418e}{街}};
\node[Onyomi] at (0.950000, 9.350000) {ガイ};
\node[Kunyomi] at (0.850000, 9.350000) {まち};
\node[Meaning] at (0.900000, 11.000000) {street};
\node[Square] at (2.950000, 9.250000) {};
\node[Kanji] at (2.950000, 9.750000) {\textcolor[HTML]{0e254c}{衡}};
\node[Onyomi] at (3.000000, 9.350000) {コウ};
\node[Meaning] at (2.950000, 11.000000) {equilibrium};
\node[Square] at (5.000000, 9.250000) {};
\node[Kanji] at (5.000000, 9.750000) {\textcolor[HTML]{0e254c}{稿}};
\node[Onyomi] at (5.050000, 9.350000) {コウ};
\node[Kunyomi] at (4.950000, 9.350000) {したがき};
\node[Meaning] at (5.000000, 11.000000) {draft};
\node[Square] at (7.050000, 9.250000) {};
\node[Kanji] at (7.050000, 9.750000) {\textcolor[HTML]{123673}{稼}};
\node[Onyomi] at (7.100000, 9.350000) {カ};
\node[Kunyomi] at (7.000000, 9.350000) {かせ.ぐ};
\node[Meaning] at (7.050000, 11.000000) {earnings};
\node[Square] at (9.100000, 9.250000) {};
\node[Kanji] at (9.100000, 9.750000) {\textcolor[HTML]{133c80}{程}};
\node[Onyomi] at (9.150000, 9.350000) {テイ};
\node[Kunyomi] at (9.050000, 9.350000) {ほど};
\node[Meaning] at (9.100000, 11.000000) {extent};
\node[Square] at (11.150000, 9.250000) {};
\node[Kanji] at (11.150000, 9.750000) {\textcolor[HTML]{14418e}{税}};
\node[Onyomi] at (11.200000, 9.350000) {ゼイ};
\node[Meaning] at (11.150000, 11.000000) {tax};
\node[Square] at (13.200000, 9.250000) {};
\node[Kanji] at (13.200000, 9.750000) {\textcolor[HTML]{123673}{稚}};
\node[Onyomi] at (13.250000, 9.350000) {チ};
\node[Meaning] at (13.200000, 11.000000) {immature};
\node[Square] at (15.250000, 9.250000) {};
\node[Kanji] at (15.250000, 9.750000) {\textcolor[HTML]{154caa}{和}};
\node[Onyomi] at (15.300000, 9.350000) {ワ};
\node[Kunyomi] at (15.200000, 9.350000) {なご};
\node[Meaning] at (15.250000, 11.000000) {peace};
\node[Square] at (17.300000, 9.250000) {};
\node[Kanji] at (17.300000, 9.750000) {\textcolor[HTML]{154caa}{移}};
\node[Onyomi] at (17.350000, 9.350000) {イ};
\node[Kunyomi] at (17.250000, 9.350000) {うつ.*};
\node[Meaning] at (17.300000, 11.000000) {shift};
\node[Square] at (19.350000, 9.250000) {};
\node[Kanji] at (19.350000, 9.750000) {\textcolor[HTML]{14469c}{秒}};
\node[Onyomi] at (19.400000, 9.350000) {ビョウ};
\node[Meaning] at (19.350000, 11.000000) {second};
\node[Square] at (21.400000, 9.250000) {};
\node[Kanji] at (21.400000, 9.750000) {\textcolor[HTML]{14418e}{秋}};
\node[Kunyomi] at (21.350000, 9.350000) {あき};
\node[Meaning] at (21.400000, 11.000000) {autumn};
\node[Square] at (23.450000, 9.250000) {};
\node[Kanji] at (23.450000, 9.750000) {\textcolor[HTML]{0e254c}{愁}};
\node[Onyomi] at (23.500000, 9.350000) {シュウ};
\node[Kunyomi] at (23.400000, 9.350000) {うれ-える};
\node[Meaning] at (23.450000, 11.000000) {distress};
\node[Square] at (25.500000, 9.250000) {};
\node[Kanji] at (25.500000, 9.750000) {\textcolor[HTML]{1461e3}{私}};
\node[Onyomi] at (25.550000, 9.350000) {シ};
\node[Kunyomi] at (25.450000, 9.350000) {わたし};
\node[Meaning] at (25.500000, 11.000000) {i};
\node[Square] at (27.550000, 9.250000) {};
\node[Kanji] at (27.550000, 9.750000) {\textcolor[HTML]{102b59}{秩}};
\node[Onyomi] at (27.600000, 9.350000) {チツ};
\node[Meaning] at (27.550000, 11.000000) {order};
\node[Square] at (29.600000, 9.250000) {};
\node[Kanji] at (29.600000, 9.750000) {\textcolor[HTML]{154caa}{秘}};
\node[Onyomi] at (29.650000, 9.350000) {ヒ};
\node[Kunyomi] at (29.550000, 9.350000) {ひ.める};
\node[Meaning] at (29.600000, 11.000000) {secret};
\node[Square] at (31.650000, 9.250000) {};
\node[Kanji] at (31.650000, 9.750000) {\textcolor[HTML]{113066}{称}};
\node[Onyomi] at (31.700000, 9.350000) {ショウ};
\node[Kunyomi] at (31.600000, 9.350000) {とな.える};
\node[Meaning] at (31.650000, 11.000000) {title};
\node[Square] at (33.700000, 9.250000) {};
\node[Kanji] at (33.700000, 9.750000) {\textcolor[HTML]{1551b8}{利}};
\node[Onyomi] at (33.750000, 9.350000) {リ};
\node[Kunyomi] at (33.650000, 9.350000) {き.く};
\node[Meaning] at (33.700000, 11.000000) {profit};
\node[Square] at (35.750000, 9.250000) {};
\node[Kanji] at (35.750000, 9.750000) {\textcolor[HTML]{14418e}{梨}};
\node[Kunyomi] at (35.700000, 9.350000) {なし};
\node[Meaning] at (35.750000, 11.000000) {pear};
\node[Square] at (37.800000, 9.250000) {};
\node[Kanji] at (37.800000, 9.750000) {\textcolor[HTML]{0e254c}{穫}};
\node[Onyomi] at (37.850000, 9.350000) {カク};
\node[Meaning] at (37.800000, 11.000000) {harvest};
\node[Square] at (39.850000, 9.250000) {};
\node[Kanji] at (39.850000, 9.750000) {\textcolor[HTML]{0e254c}{穂}};
\node[Onyomi] at (39.900000, 9.350000) {スイ};
\node[Kunyomi] at (39.800000, 9.350000) {ほ};
\node[Meaning] at (39.850000, 11.000000) {head of plant};
\node[Square] at (41.900000, 9.250000) {};
\node[Kanji] at (41.900000, 9.750000) {\textcolor[HTML]{133c80}{稲}};
\node[Kunyomi] at (41.850000, 9.350000) {いね};
\node[Meaning] at (41.900000, 11.000000) {rice plant};
\node[Square] at (43.950000, 9.250000) {};
\node[Kanji] at (43.950000, 9.750000) {\textcolor[HTML]{14418e}{香}};
\node[Onyomi] at (44.000000, 9.350000) {コウ};
\node[Kunyomi] at (43.900000, 9.350000) {かお};
\node[Meaning] at (43.950000, 11.000000) {fragrance};
\node[Square] at (46.000000, 9.250000) {};
\node[Kanji] at (46.000000, 9.750000) {\textcolor[HTML]{133c80}{季}};
\node[Onyomi] at (46.050000, 9.350000) {キ};
\node[Meaning] at (46.000000, 11.000000) {seasons};
\node[Square] at (48.050000, 9.250000) {};
\node[Kanji] at (48.050000, 9.750000) {\textcolor[HTML]{14469c}{委}};
\node[Onyomi] at (48.100000, 9.350000) {イ};
\node[Meaning] at (48.050000, 11.000000) {committee};
\node[Square] at (50.100000, 9.250000) {};
\node[Kanji] at (50.100000, 9.750000) {\textcolor[HTML]{133c80}{秀}};
\node[Onyomi] at (50.150000, 9.350000) {シュウ};
\node[Kunyomi] at (50.050000, 9.350000) {ひい.でる};
\node[Meaning] at (50.100000, 11.000000) {excel};
\node[Square] at (52.150000, 9.250000) {};
\node[Kanji] at (52.150000, 9.750000) {\textcolor[HTML]{154caa}{透}};
\node[Onyomi] at (52.200000, 9.350000) {トウ};
\node[Kunyomi] at (52.100000, 9.350000) {す.ける};
\node[Meaning] at (52.150000, 11.000000) {transparent};
\node[Square] at (54.200000, 9.250000) {};
\node[Kanji] at (54.200000, 9.750000) {\textcolor[HTML]{14469c}{誘}};
\node[Onyomi] at (54.250000, 9.350000) {ユウ};
\node[Kunyomi] at (54.150000, 9.350000) {さそ.う};
\node[Meaning] at (54.200000, 11.000000) {invite};
\node[Square] at (56.250000, 9.250000) {};
\node[Kanji] at (56.250000, 9.750000) {\textcolor[HTML]{0e254c}{穀}};
\node[Onyomi] at (56.300000, 9.350000) {コク};
\node[Meaning] at (56.250000, 11.000000) {grain};
\node[Meaning] at (-58.500000, 9.800000) {53.51\%};
\node[Square] at (-56.500000, 7.200000) {};
\node[Kanji] at (-56.500000, 7.700000) {\textcolor[HTML]{123673}{菌}};
\node[Onyomi] at (-56.450000, 7.300000) {キン};
\node[Meaning] at (-56.500000, 8.950000) {bacteria};
\node[Square] at (-54.450000, 7.200000) {};
\node[Kanji] at (-54.450000, 7.700000) {\textcolor[HTML]{133c80}{米}};
\node[Onyomi] at (-54.400000, 7.300000) {ベイ};
\node[Kunyomi] at (-54.500000, 7.300000) { こめ};
\node[Meaning] at (-54.450000, 8.950000) {rice};
\node[Square] at (-52.400000, 7.200000) {};
\node[Kanji] at (-52.400000, 7.700000) {\textcolor[HTML]{14418e}{粉}};
\node[Onyomi] at (-52.350000, 7.300000) {フン};
\node[Kunyomi] at (-52.450000, 7.300000) {こな};
\node[Meaning] at (-52.400000, 8.950000) {powder};
\node[Square] at (-50.350000, 7.200000) {};
\node[Kanji] at (-50.350000, 7.700000) {\textcolor[HTML]{102b59}{粘}};
\node[Onyomi] at (-50.300000, 7.300000) {ネン};
\node[Kunyomi] at (-50.400000, 7.300000) {ねば.る};
\node[Meaning] at (-50.350000, 8.950000) {sticky};
\node[Square] at (-48.300000, 7.200000) {};
\node[Kanji] at (-48.300000, 7.700000) {\textcolor[HTML]{133c80}{粒}};
\node[Onyomi] at (-48.250000, 7.300000) {リュウ};
\node[Kunyomi] at (-48.350000, 7.300000) {つぶ};
\node[Meaning] at (-48.300000, 8.950000) {grains};
\node[Square] at (-46.250000, 7.200000) {};
\node[Kanji] at (-46.250000, 7.700000) {\textcolor[HTML]{113066}{粧}};
\node[Onyomi] at (-46.200000, 7.300000) {ショウ};
\node[Meaning] at (-46.250000, 8.950000) {cosmetics};
\node[Square] at (-44.200000, 7.200000) {};
\node[Kanji] at (-44.200000, 7.700000) {\textcolor[HTML]{14469c}{迷}};
\node[Onyomi] at (-44.150000, 7.300000) {メイ};
\node[Kunyomi] at (-44.250000, 7.300000) {まよ.う};
\node[Meaning] at (-44.200000, 8.950000) {astray};
\node[Square] at (-42.150000, 7.200000) {};
\node[Kanji] at (-42.150000, 7.700000) {\textcolor[HTML]{113066}{粋}};
\node[Onyomi] at (-42.100000, 7.300000) {スイ};
\node[Kunyomi] at (-42.200000, 7.300000) {いき};
\node[Meaning] at (-42.150000, 8.950000) {stylish};
\node[Square] at (-40.100000, 7.200000) {};
\node[Kanji] at (-40.100000, 7.700000) {\textcolor[HTML]{0e254c}{糧}};
\node[Onyomi] at (-40.050000, 7.300000) {リョウ};
\node[Kunyomi] at (-40.150000, 7.300000) {かて};
\node[Meaning] at (-40.100000, 8.950000) {provisions};
\node[Square] at (-38.050000, 7.200000) {};
\node[Kanji] at (-38.050000, 7.700000) {\textcolor[HTML]{113066}{菊}};
\node[Onyomi] at (-38.000000, 7.300000) {キク};
\node[Meaning] at (-38.050000, 8.950000) {chrysanthemum};
\node[Square] at (-36.000000, 7.200000) {};
\node[Kanji] at (-36.000000, 7.700000) {\textcolor[HTML]{154caa}{奥}};
\node[Onyomi] at (-35.950000, 7.300000) {オウ};
\node[Kunyomi] at (-36.050000, 7.300000) {おく};
\node[Meaning] at (-36.000000, 8.950000) {interior};
\node[Square] at (-33.950000, 7.200000) {};
\node[Kanji] at (-33.950000, 7.700000) {\textcolor[HTML]{1557c6}{数}};
\node[Onyomi] at (-33.900000, 7.300000) {スウ};
\node[Kunyomi] at (-34.000000, 7.300000) {かぞ.える};
\node[Meaning] at (-33.950000, 8.950000) {count};
\node[Square] at (-31.900000, 7.200000) {};
\node[Kanji] at (-31.900000, 7.700000) {\textcolor[HTML]{0e254c}{楼}};
\node[Onyomi] at (-31.850000, 7.300000) {ロウ};
\node[Meaning] at (-31.900000, 8.950000) {watchtower};
\node[Square] at (-29.850000, 7.200000) {};
\node[Kanji] at (-29.850000, 7.700000) {\textcolor[HTML]{154caa}{類}};
\node[Onyomi] at (-29.800000, 7.300000) {ルイ};
\node[Kunyomi] at (-29.900000, 7.300000) {たぐ.い};
\node[Meaning] at (-29.850000, 8.950000) {type};
\node[Square] at (-27.800000, 7.200000) {};
\node[Kanji] at (-27.800000, 7.700000) {\textcolor[HTML]{0e254c}{漆}};
\node[Onyomi] at (-27.750000, 7.300000) {シツ};
\node[Kunyomi] at (-27.850000, 7.300000) {うるし};
\node[Meaning] at (-27.800000, 8.950000) {lacquer};
\node[Square] at (-25.750000, 7.200000) {};
\node[Kanji] at (-25.750000, 7.700000) {\textcolor[HTML]{145cd5}{様}};
\node[Onyomi] at (-25.700000, 7.300000) {ヨウ};
\node[Kunyomi] at (-25.800000, 7.300000) {さま};
\node[Meaning] at (-25.750000, 8.950000) {Mr., Mrs.};
\node[Square] at (-23.700000, 7.200000) {};
\node[Kanji] at (-23.700000, 7.700000) {\textcolor[HTML]{14469c}{求}};
\node[Onyomi] at (-23.650000, 7.300000) {キュウ};
\node[Kunyomi] at (-23.750000, 7.300000) {もと.める};
\node[Meaning] at (-23.700000, 8.950000) {request};
\node[Square] at (-21.650000, 7.200000) {};
\node[Kanji] at (-21.650000, 7.700000) {\textcolor[HTML]{154caa}{球}};
\node[Onyomi] at (-21.600000, 7.300000) {キュウ};
\node[Kunyomi] at (-21.700000, 7.300000) {たま};
\node[Meaning] at (-21.650000, 8.950000) {sphere};
\node[Square] at (-19.600000, 7.200000) {};
\node[Kanji] at (-19.600000, 7.700000) {\textcolor[HTML]{14469c}{救}};
\node[Onyomi] at (-19.550000, 7.300000) {キュウ};
\node[Kunyomi] at (-19.650000, 7.300000) {すく.う};
\node[Meaning] at (-19.600000, 8.950000) {rescue};
\node[Square] at (-17.550000, 7.200000) {};
\node[Kanji] at (-17.550000, 7.700000) {\textcolor[HTML]{123673}{竹}};
\node[Kunyomi] at (-17.600000, 7.300000) {たけ};
\node[Meaning] at (-17.550000, 8.950000) {bamboo};
\node[Square] at (-15.500000, 7.200000) {};
\node[Kanji] at (-15.500000, 7.700000) {\textcolor[HTML]{145cd5}{笑}};
\node[Onyomi] at (-15.450000, 7.300000) {ショウ};
\node[Kunyomi] at (-15.550000, 7.300000) {わら};
\node[Meaning] at (-15.500000, 8.950000) {laugh};
\node[Square] at (-13.450000, 7.200000) {};
\node[Kanji] at (-13.450000, 7.700000) {\textcolor[HTML]{123673}{笠}};
\node[Kunyomi] at (-13.500000, 7.300000) {かさ};
\node[Meaning] at (-13.450000, 8.950000) {conical hat};
\node[Square] at (-11.400000, 7.200000) {};
\node[Kanji] at (-11.400000, 7.700000) {\textcolor[HTML]{14469c}{筋}};
\node[Onyomi] at (-11.350000, 7.300000) {キン};
\node[Kunyomi] at (-11.450000, 7.300000) {すじ};
\node[Meaning] at (-11.400000, 8.950000) {muscle};
\node[Square] at (-9.350000, 7.200000) {};
\node[Kanji] at (-9.350000, 7.700000) {\textcolor[HTML]{1551b8}{箱}};
\node[Kunyomi] at (-9.400000, 7.300000) {はこ};
\node[Meaning] at (-9.350000, 8.950000) {box};
\node[Square] at (-7.300000, 7.200000) {};
\node[Kanji] at (-7.300000, 7.700000) {\textcolor[HTML]{133c80}{筆}};
\node[Onyomi] at (-7.250000, 7.300000) {ヒツ};
\node[Kunyomi] at (-7.350000, 7.300000) {ふで};
\node[Meaning] at (-7.300000, 8.950000) {writing brush};
\node[Square] at (-5.250000, 7.200000) {};
\node[Kanji] at (-5.250000, 7.700000) {\textcolor[HTML]{14418e}{筒}};
\node[Onyomi] at (-5.200000, 7.300000) {トウ};
\node[Kunyomi] at (-5.300000, 7.300000) {つつ};
\node[Meaning] at (-5.250000, 8.950000) {cylinder};
\node[Square] at (-3.200000, 7.200000) {};
\node[Kanji] at (-3.200000, 7.700000) {\textcolor[HTML]{14418e}{等}};
\node[Onyomi] at (-3.150000, 7.300000) {トウ};
\node[Kunyomi] at (-3.250000, 7.300000) {ひと.しい};
\node[Meaning] at (-3.200000, 8.950000) {equal};
\node[Square] at (-1.150000, 7.200000) {};
\node[Kanji] at (-1.150000, 7.700000) {\textcolor[HTML]{14418e}{算}};
\node[Onyomi] at (-1.100000, 7.300000) {サン};
\node[Kunyomi] at (-1.200000, 7.300000) {そろ};
\node[Meaning] at (-1.150000, 8.950000) {calculate};
\node[Square] at (0.900000, 7.200000) {};
\node[Kanji] at (0.900000, 7.700000) {\textcolor[HTML]{1557c6}{答}};
\node[Onyomi] at (0.950000, 7.300000) {トウ};
\node[Kunyomi] at (0.850000, 7.300000) {こた};
\node[Meaning] at (0.900000, 8.950000) {answer};
\node[Square] at (2.950000, 7.200000) {};
\node[Kanji] at (2.950000, 7.700000) {\textcolor[HTML]{14469c}{策}};
\node[Onyomi] at (3.000000, 7.300000) {サク};
\node[Kunyomi] at (2.900000, 7.300000) {さく};
\node[Meaning] at (2.950000, 8.950000) {plan};
\node[Square] at (5.000000, 7.200000) {};
\node[Kanji] at (5.000000, 7.700000) {\textcolor[HTML]{113066}{簿}};
\node[Onyomi] at (5.050000, 7.300000) {ボ};
\node[Meaning] at (5.000000, 8.950000) {record book};
\node[Square] at (7.050000, 7.200000) {};
\node[Kanji] at (7.050000, 7.700000) {\textcolor[HTML]{133c80}{築}};
\node[Onyomi] at (7.100000, 7.300000) {チク};
\node[Kunyomi] at (7.000000, 7.300000) {きず.く};
\node[Meaning] at (7.050000, 8.950000) {construct};
\node[Square] at (9.100000, 7.200000) {};
\node[Kanji] at (9.100000, 7.700000) {\textcolor[HTML]{3178f2}{人}};
\node[Onyomi] at (9.150000, 7.300000) {ニン};
\node[Kunyomi] at (9.050000, 7.300000) {ひと};
\node[Meaning] at (9.100000, 8.950000) {person};
\node[Square] at (11.150000, 7.200000) {};
\node[Kanji] at (11.150000, 7.700000) {\textcolor[HTML]{133c80}{佐}};
\node[Onyomi] at (11.200000, 7.300000) {サ};
\node[Meaning] at (11.150000, 8.950000) {help};
\node[Square] at (13.200000, 7.200000) {};
\node[Kanji] at (13.200000, 7.700000) {\textcolor[HTML]{0e254c}{但}};
\node[Kunyomi] at (13.150000, 7.300000) {ただ-し};
\node[Meaning] at (13.200000, 8.950000) {however};
\node[Square] at (15.250000, 7.200000) {};
\node[Kanji] at (15.250000, 7.700000) {\textcolor[HTML]{1551b8}{住}};
\node[Onyomi] at (15.300000, 7.300000) {ジュウ};
\node[Kunyomi] at (15.200000, 7.300000) {す.む};
\node[Meaning] at (15.250000, 8.950000) {dwelling};
\node[Square] at (17.300000, 7.200000) {};
\node[Kanji] at (17.300000, 7.700000) {\textcolor[HTML]{14469c}{位}};
\node[Onyomi] at (17.350000, 7.300000) {イ};
\node[Kunyomi] at (17.250000, 7.300000) {くらい};
\node[Meaning] at (17.300000, 8.950000) {rank};
\node[Square] at (19.350000, 7.200000) {};
\node[Kanji] at (19.350000, 7.700000) {\textcolor[HTML]{14469c}{仲}};
\node[Onyomi] at (19.400000, 7.300000) {チュウ};
\node[Kunyomi] at (19.300000, 7.300000) {なか};
\node[Meaning] at (19.350000, 8.950000) {relationship};
\node[Square] at (21.400000, 7.200000) {};
\node[Kanji] at (21.400000, 7.700000) {\textcolor[HTML]{145cd5}{体}};
\node[Onyomi] at (21.450000, 7.300000) {タイ};
\node[Kunyomi] at (21.350000, 7.300000) {からだ};
\node[Meaning] at (21.400000, 8.950000) {body};
\node[Square] at (23.450000, 7.200000) {};
\node[Kanji] at (23.450000, 7.700000) {\textcolor[HTML]{102b59}{悠}};
\node[Onyomi] at (23.500000, 7.300000) {ユウ};
\node[Meaning] at (23.450000, 8.950000) {leisure};
\node[Square] at (25.500000, 7.200000) {};
\node[Kanji] at (25.500000, 7.700000) {\textcolor[HTML]{1551b8}{件}};
\node[Onyomi] at (25.550000, 7.300000) {ケン};
\node[Meaning] at (25.500000, 8.950000) {matter};
\node[Square] at (27.550000, 7.200000) {};
\node[Kanji] at (27.550000, 7.700000) {\textcolor[HTML]{1557c6}{仕}};
\node[Onyomi] at (27.600000, 7.300000) {シ};
\node[Kunyomi] at (27.500000, 7.300000) {つか.える};
\node[Meaning] at (27.550000, 8.950000) {doing};
\node[Square] at (29.600000, 7.200000) {};
\node[Kanji] at (29.600000, 7.700000) {\textcolor[HTML]{154caa}{他}};
\node[Onyomi] at (29.650000, 7.300000) {タ};
\node[Kunyomi] at (29.550000, 7.300000) {ほか};
\node[Meaning] at (29.600000, 8.950000) {other};
\node[Square] at (31.650000, 7.200000) {};
\node[Kanji] at (31.650000, 7.700000) {\textcolor[HTML]{14418e}{伏}};
\node[Onyomi] at (31.700000, 7.300000) {フク};
\node[Kunyomi] at (31.600000, 7.300000) {ふ};
\node[Meaning] at (31.650000, 8.950000) {bow};
\node[Square] at (33.700000, 7.200000) {};
\node[Kanji] at (33.700000, 7.700000) {\textcolor[HTML]{1557c6}{伝}};
\node[Onyomi] at (33.750000, 7.300000) {デン};
\node[Kunyomi] at (33.650000, 7.300000) {つた};
\node[Meaning] at (33.700000, 8.950000) {transmit};
\node[Square] at (35.750000, 7.200000) {};
\node[Kanji] at (35.750000, 7.700000) {\textcolor[HTML]{133c80}{仏}};
\node[Onyomi] at (35.800000, 7.300000) {ブツ};
\node[Kunyomi] at (35.700000, 7.300000) {ほとけ};
\node[Meaning] at (35.750000, 8.950000) {buddha};
\node[Square] at (37.800000, 7.200000) {};
\node[Kanji] at (37.800000, 7.700000) {\textcolor[HTML]{1551b8}{休}};
\node[Onyomi] at (37.850000, 7.300000) {キュウ};
\node[Kunyomi] at (37.750000, 7.300000) {やす.み};
\node[Meaning] at (37.800000, 8.950000) {rest};
\node[Square] at (39.850000, 7.200000) {};
\node[Kanji] at (39.850000, 7.700000) {\textcolor[HTML]{14418e}{仮}};
\node[Onyomi] at (39.900000, 7.300000) {カ};
\node[Kunyomi] at (39.800000, 7.300000) {かり};
\node[Meaning] at (39.850000, 8.950000) {temporary};
\node[Square] at (41.900000, 7.200000) {};
\node[Kanji] at (41.900000, 7.700000) {\textcolor[HTML]{0e254c}{伯}};
\node[Onyomi] at (41.950000, 7.300000) {ハク};
\node[Meaning] at (41.900000, 8.950000) {chief};
\node[Square] at (43.950000, 7.200000) {};
\node[Kanji] at (43.950000, 7.700000) {\textcolor[HTML]{102b59}{俗}};
\node[Onyomi] at (44.000000, 7.300000) {ゾク};
\node[Meaning] at (43.950000, 8.950000) {vulgar};
\node[Square] at (46.000000, 7.200000) {};
\node[Kanji] at (46.000000, 7.700000) {\textcolor[HTML]{1557c6}{信}};
\node[Onyomi] at (46.050000, 7.300000) {シン};
\node[Kunyomi] at (45.950000, 7.300000) {しん};
\node[Meaning] at (46.000000, 8.950000) {believe};
\node[Square] at (48.050000, 7.200000) {};
\node[Kanji] at (48.050000, 7.700000) {\textcolor[HTML]{0e254c}{佳}};
\node[Onyomi] at (48.100000, 7.300000) {カ};
\node[Meaning] at (48.050000, 8.950000) {excellent};
\node[Square] at (50.100000, 7.200000) {};
\node[Kanji] at (50.100000, 7.700000) {\textcolor[HTML]{113066}{依}};
\node[Onyomi] at (50.150000, 7.300000) {イ};
\node[Kunyomi] at (50.050000, 7.300000) {よ.る};
\node[Meaning] at (50.100000, 8.950000) {reliant};
\node[Square] at (52.150000, 7.200000) {};
\node[Kanji] at (52.150000, 7.700000) {\textcolor[HTML]{154caa}{例}};
\node[Onyomi] at (52.200000, 7.300000) {レイ};
\node[Kunyomi] at (52.100000, 7.300000) {たと};
\node[Meaning] at (52.150000, 8.950000) {example};
\node[Square] at (54.200000, 7.200000) {};
\node[Kanji] at (54.200000, 7.700000) {\textcolor[HTML]{154caa}{個}};
\node[Onyomi] at (54.250000, 7.300000) {コ};
\node[Meaning] at (54.200000, 8.950000) {individual};
\node[Square] at (56.250000, 7.200000) {};
\node[Kanji] at (56.250000, 7.700000) {\textcolor[HTML]{14469c}{健}};
\node[Onyomi] at (56.300000, 7.300000) {ケン};
\node[Meaning] at (56.250000, 8.950000) {healthy};
\node[Meaning] at (-58.500000, 7.750000) {57.31\%};
\node[Square] at (-56.500000, 5.150000) {};
\node[Kanji] at (-56.500000, 5.650000) {\textcolor[HTML]{1551b8}{側}};
\node[Onyomi] at (-56.450000, 5.250000) {ソク};
\node[Kunyomi] at (-56.550000, 5.250000) {がわ};
\node[Meaning] at (-56.500000, 6.900000) {side};
\node[Square] at (-54.450000, 5.150000) {};
\node[Kanji] at (-54.450000, 5.650000) {\textcolor[HTML]{0e254c}{侍}};
\node[Kunyomi] at (-54.500000, 5.250000) {さむらい};
\node[Meaning] at (-54.450000, 6.900000) {samurai};
\node[Square] at (-52.400000, 5.150000) {};
\node[Kanji] at (-52.400000, 5.650000) {\textcolor[HTML]{14418e}{停}};
\node[Onyomi] at (-52.350000, 5.250000) {テイ};
\node[Meaning] at (-52.400000, 6.900000) {halt};
\node[Square] at (-50.350000, 5.150000) {};
\node[Kanji] at (-50.350000, 5.650000) {\textcolor[HTML]{154caa}{値}};
\node[Onyomi] at (-50.300000, 5.250000) {チ};
\node[Kunyomi] at (-50.400000, 5.250000) {ね};
\node[Meaning] at (-50.350000, 6.900000) {value};
\node[Square] at (-48.300000, 5.150000) {};
\node[Kanji] at (-48.300000, 5.650000) {\textcolor[HTML]{0e254c}{倣}};
\node[Onyomi] at (-48.250000, 5.250000) {ホウ};
\node[Kunyomi] at (-48.350000, 5.250000) {なら-う};
\node[Meaning] at (-48.300000, 6.900000) {emulate};
\node[Square] at (-46.250000, 5.150000) {};
\node[Kanji] at (-46.250000, 5.650000) {\textcolor[HTML]{1551b8}{倒}};
\node[Onyomi] at (-46.200000, 5.250000) {トウ};
\node[Kunyomi] at (-46.300000, 5.250000) {たお.す};
\node[Meaning] at (-46.250000, 6.900000) {overthrow};
\node[Square] at (-44.200000, 5.150000) {};
\node[Kanji] at (-44.200000, 5.650000) {\textcolor[HTML]{102b59}{偵}};
\node[Onyomi] at (-44.150000, 5.250000) {テイ};
\node[Meaning] at (-44.200000, 6.900000) {spy};
\node[Square] at (-42.150000, 5.150000) {};
\node[Kanji] at (-42.150000, 5.650000) {\textcolor[HTML]{133c80}{僧}};
\node[Onyomi] at (-42.100000, 5.250000) {ソウ};
\node[Meaning] at (-42.150000, 6.900000) {priest};
\node[Square] at (-40.100000, 5.150000) {};
\node[Kanji] at (-40.100000, 5.650000) {\textcolor[HTML]{14469c}{億}};
\node[Onyomi] at (-40.050000, 5.250000) {オク};
\node[Meaning] at (-40.100000, 6.900000) {100 million};
\node[Square] at (-38.050000, 5.150000) {};
\node[Kanji] at (-38.050000, 5.650000) {\textcolor[HTML]{14469c}{儀}};
\node[Onyomi] at (-38.000000, 5.250000) {ギ};
\node[Meaning] at (-38.050000, 6.900000) {ceremony};
\node[Square] at (-36.000000, 5.150000) {};
\node[Kanji] at (-36.000000, 5.650000) {\textcolor[HTML]{113066}{償}};
\node[Onyomi] at (-35.950000, 5.250000) {ショウ};
\node[Kunyomi] at (-36.050000, 5.250000) {つぐな.う};
\node[Meaning] at (-36.000000, 6.900000) {reparation};
\node[Square] at (-33.950000, 5.150000) {};
\node[Kanji] at (-33.950000, 5.650000) {\textcolor[HTML]{123673}{仙}};
\node[Onyomi] at (-33.900000, 5.250000) {セン};
\node[Meaning] at (-33.950000, 6.900000) {hermit};
\node[Square] at (-31.900000, 5.150000) {};
\node[Kanji] at (-31.900000, 5.650000) {\textcolor[HTML]{123673}{催}};
\node[Onyomi] at (-31.850000, 5.250000) {サイ};
\node[Kunyomi] at (-31.950000, 5.250000) {もよお.す};
\node[Meaning] at (-31.900000, 6.900000) {sponsor};
\node[Square] at (-29.850000, 5.150000) {};
\node[Kanji] at (-29.850000, 5.650000) {\textcolor[HTML]{102b59}{仁}};
\node[Onyomi] at (-29.800000, 5.250000) {ジン};
\node[Meaning] at (-29.850000, 6.900000) {humanity};
\node[Square] at (-27.800000, 5.150000) {};
\node[Kanji] at (-27.800000, 5.650000) {\textcolor[HTML]{123673}{侮}};
\node[Onyomi] at (-27.750000, 5.250000) {ブ};
\node[Kunyomi] at (-27.850000, 5.250000) {あなず};
\node[Meaning] at (-27.800000, 6.900000) {despise};
\node[Square] at (-25.750000, 5.150000) {};
\node[Kanji] at (-25.750000, 5.650000) {\textcolor[HTML]{1461e3}{使}};
\node[Onyomi] at (-25.700000, 5.250000) {シ};
\node[Kunyomi] at (-25.800000, 5.250000) {つか.う};
\node[Meaning] at (-25.750000, 6.900000) {use};
\node[Square] at (-23.700000, 5.150000) {};
\node[Kanji] at (-23.700000, 5.650000) {\textcolor[HTML]{14469c}{便}};
\node[Onyomi] at (-23.650000, 5.250000) {ベン};
\node[Kunyomi] at (-23.750000, 5.250000) {たよ.*};
\node[Meaning] at (-23.700000, 6.900000) {convenience};
\node[Square] at (-21.650000, 5.150000) {};
\node[Kanji] at (-21.650000, 5.650000) {\textcolor[HTML]{154caa}{倍}};
\node[Onyomi] at (-21.600000, 5.250000) {バイ};
\node[Meaning] at (-21.650000, 6.900000) {double};
\node[Square] at (-19.600000, 5.150000) {};
\node[Kanji] at (-19.600000, 5.650000) {\textcolor[HTML]{1551b8}{優}};
\node[Onyomi] at (-19.550000, 5.250000) {ユウ};
\node[Kunyomi] at (-19.650000, 5.250000) {やさ.しい};
\node[Meaning] at (-19.600000, 6.900000) {superior};
\node[Square] at (-17.550000, 5.150000) {};
\node[Kanji] at (-17.550000, 5.650000) {\textcolor[HTML]{0e254c}{伐}};
\node[Onyomi] at (-17.500000, 5.250000) {バツ};
\node[Kunyomi] at (-17.600000, 5.250000) {う};
\node[Meaning] at (-17.550000, 6.900000) {fell};
\node[Square] at (-15.500000, 5.150000) {};
\node[Kanji] at (-15.500000, 5.650000) {\textcolor[HTML]{154caa}{宿}};
\node[Onyomi] at (-15.450000, 5.250000) {シュク};
\node[Kunyomi] at (-15.550000, 5.250000) {やど};
\node[Meaning] at (-15.500000, 6.900000) {lodge};
\node[Square] at (-13.450000, 5.150000) {};
\node[Kanji] at (-13.450000, 5.650000) {\textcolor[HTML]{1551b8}{傷}};
\node[Onyomi] at (-13.400000, 5.250000) {ショウ};
\node[Kunyomi] at (-13.500000, 5.250000) {きず};
\node[Meaning] at (-13.450000, 6.900000) {wound};
\node[Square] at (-11.400000, 5.150000) {};
\node[Kanji] at (-11.400000, 5.650000) {\textcolor[HTML]{154caa}{保}};
\node[Onyomi] at (-11.350000, 5.250000) {ホ};
\node[Kunyomi] at (-11.450000, 5.250000) {たも.つ};
\node[Meaning] at (-11.400000, 6.900000) {preserve};
\node[Square] at (-9.350000, 5.150000) {};
\node[Kanji] at (-9.350000, 5.650000) {\textcolor[HTML]{123673}{褒}};
\node[Onyomi] at (-9.300000, 5.250000) {ホウ};
\node[Kunyomi] at (-9.400000, 5.250000) {ほ.める};
\node[Meaning] at (-9.350000, 6.900000) {praise};
\node[Square] at (-7.300000, 5.150000) {};
\node[Kanji] at (-7.300000, 5.650000) {\textcolor[HTML]{0e254c}{傑}};
\node[Onyomi] at (-7.250000, 5.250000) {ケツ};
\node[Kunyomi] at (-7.350000, 5.250000) {すぐ};
\node[Meaning] at (-7.300000, 6.900000) {greatness};
\node[Square] at (-5.250000, 5.150000) {};
\node[Kanji] at (-5.250000, 5.650000) {\textcolor[HTML]{1557c6}{付}};
\node[Onyomi] at (-5.200000, 5.250000) {フ};
\node[Kunyomi] at (-5.300000, 5.250000) {つ};
\node[Meaning] at (-5.250000, 6.900000) {attach};
\node[Square] at (-3.200000, 5.150000) {};
\node[Kanji] at (-3.200000, 5.650000) {\textcolor[HTML]{133c80}{符}};
\node[Onyomi] at (-3.150000, 5.250000) {フ};
\node[Meaning] at (-3.200000, 6.900000) {token};
\node[Square] at (-1.150000, 5.150000) {};
\node[Kanji] at (-1.150000, 5.650000) {\textcolor[HTML]{1551b8}{府}};
\node[Onyomi] at (-1.100000, 5.250000) {フ};
\node[Meaning] at (-1.150000, 6.900000) {government};
\node[Square] at (0.900000, 5.150000) {};
\node[Kanji] at (0.900000, 5.650000) {\textcolor[HTML]{154caa}{任}};
\node[Onyomi] at (0.950000, 5.250000) {ニン};
\node[Kunyomi] at (0.850000, 5.250000) {まか.せる};
\node[Meaning] at (0.900000, 6.900000) {duty};
\node[Square] at (2.950000, 5.150000) {};
\node[Kanji] at (2.950000, 5.650000) {\textcolor[HTML]{113066}{賃}};
\node[Onyomi] at (3.000000, 5.250000) {チン};
\node[Meaning] at (2.950000, 6.900000) {rent};
\node[Square] at (5.000000, 5.150000) {};
\node[Kanji] at (5.000000, 5.650000) {\textcolor[HTML]{1557c6}{代}};
\node[Onyomi] at (5.050000, 5.250000) {ダイ};
\node[Kunyomi] at (4.950000, 5.250000) {か};
\node[Meaning] at (5.000000, 6.900000) {substitute};
\node[Square] at (7.050000, 5.150000) {};
\node[Kanji] at (7.050000, 5.650000) {\textcolor[HTML]{154caa}{袋}};
\node[Onyomi] at (7.100000, 5.250000) {タイ};
\node[Kunyomi] at (7.000000, 5.250000) {ふくろ};
\node[Meaning] at (7.050000, 6.900000) {sack};
\node[Square] at (9.100000, 5.150000) {};
\node[Kanji] at (9.100000, 5.650000) {\textcolor[HTML]{14418e}{貸}};
\node[Onyomi] at (9.150000, 5.250000) {タイ};
\node[Kunyomi] at (9.050000, 5.250000) {か};
\node[Meaning] at (9.100000, 6.900000) {lend};
\node[Square] at (11.150000, 5.150000) {};
\node[Kanji] at (11.150000, 5.650000) {\textcolor[HTML]{1551b8}{化}};
\node[Onyomi] at (11.200000, 5.250000) {カ};
\node[Kunyomi] at (11.100000, 5.250000) {ば.ける};
\node[Meaning] at (11.150000, 6.900000) {change};
\node[Square] at (13.200000, 5.150000) {};
\node[Kanji] at (13.200000, 5.650000) {\textcolor[HTML]{154caa}{花}};
\node[Onyomi] at (13.250000, 5.250000) {カ};
\node[Kunyomi] at (13.150000, 5.250000) {はな};
\node[Meaning] at (13.200000, 6.900000) {flower};
\node[Square] at (15.250000, 5.150000) {};
\node[Kanji] at (15.250000, 5.650000) {\textcolor[HTML]{14469c}{貨}};
\node[Onyomi] at (15.300000, 5.250000) {カ};
\node[Meaning] at (15.250000, 6.900000) {freight};
\node[Square] at (17.300000, 5.150000) {};
\node[Kanji] at (17.300000, 5.650000) {\textcolor[HTML]{14418e}{傾}};
\node[Onyomi] at (17.350000, 5.250000) {ケイ};
\node[Kunyomi] at (17.250000, 5.250000) {かたむ.*};
\node[Meaning] at (17.300000, 6.900000) {lean};
\node[Square] at (19.350000, 5.150000) {};
\node[Kanji] at (19.350000, 5.650000) {\textcolor[HTML]{1968ed}{何}};
\node[Onyomi] at (19.400000, 5.250000) {カ};
\node[Kunyomi] at (19.300000, 5.250000) {なに};
\node[Meaning] at (19.350000, 6.900000) {what};
\node[Square] at (21.400000, 5.150000) {};
\node[Kanji] at (21.400000, 5.650000) {\textcolor[HTML]{14469c}{荷}};
\node[Onyomi] at (21.450000, 5.250000) {カ};
\node[Kunyomi] at (21.350000, 5.250000) {に};
\node[Meaning] at (21.400000, 6.900000) {luggage};
\node[Square] at (23.450000, 5.150000) {};
\node[Kanji] at (23.450000, 5.650000) {\textcolor[HTML]{102b59}{俊}};
\node[Onyomi] at (23.500000, 5.250000) {シュン};
\node[Meaning] at (23.450000, 6.900000) {genius};
\node[Square] at (25.500000, 5.150000) {};
\node[Kanji] at (25.500000, 5.650000) {\textcolor[HTML]{123673}{傍}};
\node[Onyomi] at (25.550000, 5.250000) {ボウ};
\node[Kunyomi] at (25.450000, 5.250000) {かたわ};
\node[Meaning] at (25.500000, 6.900000) {nearby};
\node[Square] at (27.550000, 5.150000) {};
\node[Kanji] at (27.550000, 5.650000) {\textcolor[HTML]{14469c}{久}};
\node[Onyomi] at (27.600000, 5.250000) {キュウ};
\node[Kunyomi] at (27.500000, 5.250000) {ひさ};
\node[Meaning] at (27.550000, 6.900000) {long time};
\node[Square] at (29.600000, 5.150000) {};
\node[Kanji] at (29.600000, 5.650000) {\textcolor[HTML]{0e254c}{畝}};
\node[Kunyomi] at (29.550000, 5.250000) {うね};
\node[Meaning] at (29.600000, 6.900000) {furrow};
\node[Square] at (31.650000, 5.150000) {};
\node[Kanji] at (31.650000, 5.650000) {\textcolor[HTML]{123673}{囚}};
\node[Onyomi] at (31.700000, 5.250000) {シュウ};
\node[Kunyomi] at (31.600000, 5.250000) {とら};
\node[Meaning] at (31.650000, 6.900000) {criminal};
\node[Square] at (33.700000, 5.150000) {};
\node[Kanji] at (33.700000, 5.650000) {\textcolor[HTML]{1551b8}{内}};
\node[Onyomi] at (33.750000, 5.250000) {ナイ};
\node[Kunyomi] at (33.650000, 5.250000) {うち};
\node[Meaning] at (33.700000, 6.900000) {inside};
\node[Square] at (35.750000, 5.150000) {};
\node[Kanji] at (35.750000, 5.650000) {\textcolor[HTML]{0e254c}{丙}};
\node[Onyomi] at (35.800000, 5.250000) {ヘイ};
\node[Meaning] at (35.750000, 6.900000) {third class};
\node[Square] at (37.800000, 5.150000) {};
\node[Kanji] at (37.800000, 5.650000) {\textcolor[HTML]{14418e}{柄}};
\node[Onyomi] at (37.850000, 5.250000) {ヘイ};
\node[Kunyomi] at (37.750000, 5.250000) {がら};
\node[Meaning] at (37.800000, 6.900000) {pattern};
\node[Square] at (39.850000, 5.150000) {};
\node[Kanji] at (39.850000, 5.650000) {\textcolor[HTML]{154caa}{肉}};
\node[Onyomi] at (39.900000, 5.250000) {ニク};
\node[Meaning] at (39.850000, 6.900000) {meat};
\node[Square] at (41.900000, 5.150000) {};
\node[Kanji] at (41.900000, 5.650000) {\textcolor[HTML]{133c80}{腐}};
\node[Onyomi] at (41.950000, 5.250000) {フ};
\node[Kunyomi] at (41.850000, 5.250000) {くさ.る};
\node[Meaning] at (41.900000, 6.900000) {rot};
\node[Square] at (43.950000, 5.150000) {};
\node[Kanji] at (43.950000, 5.650000) {\textcolor[HTML]{1557c6}{座}};
\node[Onyomi] at (44.000000, 5.250000) {ザ};
\node[Kunyomi] at (43.900000, 5.250000) {すわ.る};
\node[Meaning] at (43.950000, 6.900000) {sit};
\node[Square] at (46.000000, 5.150000) {};
\node[Kanji] at (46.000000, 5.650000) {\textcolor[HTML]{133c80}{卒}};
\node[Onyomi] at (46.050000, 5.250000) {ソツ};
\node[Meaning] at (46.000000, 6.900000) {graduate};
\node[Square] at (48.050000, 5.150000) {};
\node[Kanji] at (48.050000, 5.650000) {\textcolor[HTML]{123673}{傘}};
\node[Onyomi] at (48.100000, 5.250000) {サン};
\node[Kunyomi] at (48.000000, 5.250000) {かさ};
\node[Meaning] at (48.050000, 6.900000) {umbrella};
\node[Square] at (50.100000, 5.150000) {};
\node[Kanji] at (50.100000, 5.650000) {\textcolor[HTML]{145cd5}{以}};
\node[Onyomi] at (50.150000, 5.250000) {イ};
\node[Meaning] at (50.100000, 6.900000) {by means of};
\node[Square] at (52.150000, 5.150000) {};
\node[Kanji] at (52.150000, 5.650000) {\textcolor[HTML]{14418e}{似}};
\node[Onyomi] at (52.200000, 5.250000) {ネ};
\node[Kunyomi] at (52.100000, 5.250000) {に.る};
\node[Meaning] at (52.150000, 6.900000) {resemble};
\node[Square] at (54.200000, 5.150000) {};
\node[Kanji] at (54.200000, 5.650000) {\textcolor[HTML]{0e254c}{併}};
\node[Onyomi] at (54.250000, 5.250000) {ヘイ};
\node[Kunyomi] at (54.150000, 5.250000) {あわ.せる};
\node[Meaning] at (54.200000, 6.900000) {join};
\node[Square] at (56.250000, 5.150000) {};
\node[Kanji] at (56.250000, 5.650000) {\textcolor[HTML]{113066}{瓦}};
\node[Onyomi] at (56.300000, 5.250000) {ガ};
\node[Kunyomi] at (56.200000, 5.250000) {かわら};
\node[Meaning] at (56.250000, 6.900000) {tile};
\node[Meaning] at (-58.500000, 5.700000) {59.97\%};
\node[Square] at (-56.500000, 3.100000) {};
\node[Kanji] at (-56.500000, 3.600000) {\textcolor[HTML]{14469c}{瓶}};
\node[Onyomi] at (-56.450000, 3.200000) {ビン};
\node[Kunyomi] at (-56.550000, 3.200000) {かめ};
\node[Meaning] at (-56.500000, 4.850000) {bottle};
\node[Square] at (-54.450000, 3.100000) {};
\node[Kanji] at (-54.450000, 3.600000) {\textcolor[HTML]{14469c}{宮}};
\node[Onyomi] at (-54.400000, 3.200000) {キュウ};
\node[Kunyomi] at (-54.500000, 3.200000) {みや};
\node[Meaning] at (-54.450000, 4.850000) {shinto shrine};
\node[Square] at (-52.400000, 3.100000) {};
\node[Kanji] at (-52.400000, 3.600000) {\textcolor[HTML]{14418e}{営}};
\node[Onyomi] at (-52.350000, 3.200000) {エイ};
\node[Kunyomi] at (-52.450000, 3.200000) {いとな.む};
\node[Meaning] at (-52.400000, 4.850000) {manage};
\node[Square] at (-50.350000, 3.100000) {};
\node[Kanji] at (-50.350000, 3.600000) {\textcolor[HTML]{133c80}{善}};
\node[Onyomi] at (-50.300000, 3.200000) {ゼン};
\node[Kunyomi] at (-50.400000, 3.200000) {ぜん};
\node[Meaning] at (-50.350000, 4.850000) {morally good};
\node[Square] at (-48.300000, 3.100000) {};
\node[Kanji] at (-48.300000, 3.600000) {\textcolor[HTML]{1968ed}{年}};
\node[Onyomi] at (-48.250000, 3.200000) {ネン};
\node[Kunyomi] at (-48.350000, 3.200000) {とし};
\node[Meaning] at (-48.300000, 4.850000) {year};
\node[Square] at (-46.250000, 3.100000) {};
\node[Kanji] at (-46.250000, 3.600000) {\textcolor[HTML]{1557c6}{夜}};
\node[Onyomi] at (-46.200000, 3.200000) {ヤ};
\node[Kunyomi] at (-46.300000, 3.200000) {よ};
\node[Meaning] at (-46.250000, 4.850000) {night};
\node[Square] at (-44.200000, 3.100000) {};
\node[Kanji] at (-44.200000, 3.600000) {\textcolor[HTML]{14418e}{液}};
\node[Onyomi] at (-44.150000, 3.200000) {エキ};
\node[Meaning] at (-44.200000, 4.850000) {fluid};
\node[Square] at (-42.150000, 3.100000) {};
\node[Kanji] at (-42.150000, 3.600000) {\textcolor[HTML]{123673}{塚}};
\node[Onyomi] at (-42.100000, 3.200000) {チョウ};
\node[Kunyomi] at (-42.200000, 3.200000) {つか};
\node[Meaning] at (-42.150000, 4.850000) {mound};
\node[Square] at (-40.100000, 3.100000) {};
\node[Kanji] at (-40.100000, 3.600000) {\textcolor[HTML]{0e254c}{幣}};
\node[Onyomi] at (-40.050000, 3.200000) {ヘイ};
\node[Meaning] at (-40.100000, 4.850000) {cash};
\node[Square] at (-38.050000, 3.100000) {};
\node[Kanji] at (-38.050000, 3.600000) {\textcolor[HTML]{0e254c}{弊}};
\node[Onyomi] at (-38.000000, 3.200000) {ヘイ};
\node[Meaning] at (-38.050000, 4.850000) {evil};
\node[Square] at (-36.000000, 3.100000) {};
\node[Kanji] at (-36.000000, 3.600000) {\textcolor[HTML]{14418e}{喚}};
\node[Onyomi] at (-35.950000, 3.200000) {カン};
\node[Kunyomi] at (-36.050000, 3.200000) {わめ};
\node[Meaning] at (-36.000000, 4.850000) {scream};
\node[Square] at (-33.950000, 3.100000) {};
\node[Kanji] at (-33.950000, 3.600000) {\textcolor[HTML]{14418e}{換}};
\node[Onyomi] at (-33.900000, 3.200000) {カン};
\node[Kunyomi] at (-34.000000, 3.200000) {か.える};
\node[Meaning] at (-33.950000, 4.850000) {exchange};
\node[Square] at (-31.900000, 3.100000) {};
\node[Kanji] at (-31.900000, 3.600000) {\textcolor[HTML]{0e254c}{融}};
\node[Onyomi] at (-31.850000, 3.200000) {ユウ};
\node[Meaning] at (-31.900000, 4.850000) {dissolve};
\node[Square] at (-29.850000, 3.100000) {};
\node[Kanji] at (-29.850000, 3.600000) {\textcolor[HTML]{14418e}{施}};
\node[Onyomi] at (-29.800000, 3.200000) {シ};
\node[Kunyomi] at (-29.900000, 3.200000) {ほどこ.す};
\node[Meaning] at (-29.850000, 4.850000) {carry out};
\node[Square] at (-27.800000, 3.100000) {};
\node[Kanji] at (-27.800000, 3.600000) {\textcolor[HTML]{14418e}{旋}};
\node[Onyomi] at (-27.750000, 3.200000) {セン};
\node[Meaning] at (-27.800000, 4.850000) {rotation};
\node[Square] at (-25.750000, 3.100000) {};
\node[Kanji] at (-25.750000, 3.600000) {\textcolor[HTML]{154caa}{遊}};
\node[Onyomi] at (-25.700000, 3.200000) {ユウ};
\node[Kunyomi] at (-25.800000, 3.200000) {あそ};
\node[Meaning] at (-25.750000, 4.850000) {play};
\node[Square] at (-23.700000, 3.100000) {};
\node[Kanji] at (-23.700000, 3.600000) {\textcolor[HTML]{154caa}{旅}};
\node[Onyomi] at (-23.650000, 3.200000) {リョ};
\node[Kunyomi] at (-23.750000, 3.200000) {たび};
\node[Meaning] at (-23.700000, 4.850000) {trip};
\node[Square] at (-21.650000, 3.100000) {};
\node[Kanji] at (-21.650000, 3.600000) {\textcolor[HTML]{1461e3}{物}};
\node[Onyomi] at (-21.600000, 3.200000) {ブツ};
\node[Kunyomi] at (-21.700000, 3.200000) {もの};
\node[Meaning] at (-21.650000, 4.850000) {thing};
\node[Square] at (-19.600000, 3.100000) {};
\node[Kanji] at (-19.600000, 3.600000) {\textcolor[HTML]{133c80}{易}};
\node[Onyomi] at (-19.550000, 3.200000) {イ};
\node[Kunyomi] at (-19.650000, 3.200000) {やさ.しい};
\node[Meaning] at (-19.600000, 4.850000) {easy};
\node[Square] at (-17.550000, 3.100000) {};
\node[Kanji] at (-17.550000, 3.600000) {\textcolor[HTML]{0e254c}{賜}};
\node[Onyomi] at (-17.500000, 3.200000) {シ};
\node[Kunyomi] at (-17.600000, 3.200000) {たまわ-る};
\node[Meaning] at (-17.550000, 4.850000) {grant};
\node[Square] at (-15.500000, 3.100000) {};
\node[Kanji] at (-15.500000, 3.600000) {\textcolor[HTML]{113066}{尿}};
\node[Onyomi] at (-15.450000, 3.200000) {ニョウ};
\node[Meaning] at (-15.500000, 4.850000) {urine};
\node[Square] at (-13.450000, 3.100000) {};
\node[Kanji] at (-13.450000, 3.600000) {\textcolor[HTML]{0e254c}{尼}};
\node[Onyomi] at (-13.400000, 3.200000) {ニ};
\node[Kunyomi] at (-13.500000, 3.200000) {あま};
\node[Meaning] at (-13.450000, 4.850000) {nun};
\node[Square] at (-11.400000, 3.100000) {};
\node[Kanji] at (-11.400000, 3.600000) {\textcolor[HTML]{14418e}{泥}};
\node[Onyomi] at (-11.350000, 3.200000) {デイ};
\node[Kunyomi] at (-11.450000, 3.200000) {どろ};
\node[Meaning] at (-11.400000, 4.850000) {mud};
\node[Square] at (-9.350000, 3.100000) {};
\node[Kanji] at (-9.350000, 3.600000) {\textcolor[HTML]{102b59}{塀}};
\node[Onyomi] at (-9.300000, 3.200000) {ヘイ};
\node[Meaning] at (-9.350000, 4.850000) {fence};
\node[Square] at (-7.300000, 3.100000) {};
\node[Kanji] at (-7.300000, 3.600000) {\textcolor[HTML]{123673}{履}};
\node[Onyomi] at (-7.250000, 3.200000) {リ};
\node[Kunyomi] at (-7.350000, 3.200000) {は.く};
\node[Meaning] at (-7.300000, 4.850000) {boots};
\node[Square] at (-5.250000, 3.100000) {};
\node[Kanji] at (-5.250000, 3.600000) {\textcolor[HTML]{1461e3}{屋}};
\node[Onyomi] at (-5.200000, 3.200000) {オク};
\node[Kunyomi] at (-5.300000, 3.200000) {や};
\node[Meaning] at (-5.250000, 4.850000) {roof};
\node[Square] at (-3.200000, 3.100000) {};
\node[Kanji] at (-3.200000, 3.600000) {\textcolor[HTML]{1551b8}{握}};
\node[Onyomi] at (-3.150000, 3.200000) {アク};
\node[Kunyomi] at (-3.250000, 3.200000) {にぎ.る};
\node[Meaning] at (-3.200000, 4.850000) {grip};
\node[Square] at (-1.150000, 3.100000) {};
\node[Kanji] at (-1.150000, 3.600000) {\textcolor[HTML]{154caa}{屈}};
\node[Onyomi] at (-1.100000, 3.200000) {クツ};
\node[Kunyomi] at (-1.200000, 3.200000) {かが};
\node[Meaning] at (-1.150000, 4.850000) {yield};
\node[Square] at (0.900000, 3.100000) {};
\node[Kanji] at (0.900000, 3.600000) {\textcolor[HTML]{133c80}{掘}};
\node[Onyomi] at (0.950000, 3.200000) {クツ};
\node[Kunyomi] at (0.850000, 3.200000) {ほ.る};
\node[Meaning] at (0.900000, 4.850000) {dig};
\node[Square] at (2.950000, 3.100000) {};
\node[Kanji] at (2.950000, 3.600000) {\textcolor[HTML]{0e254c}{堀}};
\node[Onyomi] at (3.000000, 3.200000) {クツ};
\node[Kunyomi] at (2.900000, 3.200000) {ほり};
\node[Meaning] at (2.950000, 4.850000) {ditch};
\node[Square] at (5.000000, 3.100000) {};
\node[Kanji] at (5.000000, 3.600000) {\textcolor[HTML]{154caa}{居}};
\node[Onyomi] at (5.050000, 3.200000) {キョ};
\node[Kunyomi] at (4.950000, 3.200000) {い};
\node[Meaning] at (5.000000, 4.850000) {alive};
\node[Square] at (7.050000, 3.100000) {};
\node[Kanji] at (7.050000, 3.600000) {\textcolor[HTML]{133c80}{据}};
\node[Onyomi] at (7.100000, 3.200000) {キョ};
\node[Kunyomi] at (7.000000, 3.200000) {す};
\node[Meaning] at (7.050000, 4.850000) {install};
\node[Square] at (9.100000, 3.100000) {};
\node[Kanji] at (9.100000, 3.600000) {\textcolor[HTML]{133c80}{層}};
\node[Onyomi] at (9.150000, 3.200000) {ソウ};
\node[Meaning] at (9.100000, 4.850000) {layer};
\node[Square] at (11.150000, 3.100000) {};
\node[Kanji] at (11.150000, 3.600000) {\textcolor[HTML]{154caa}{局}};
\node[Onyomi] at (11.200000, 3.200000) {キョク};
\node[Meaning] at (11.150000, 4.850000) {bureau};
\node[Square] at (13.200000, 3.100000) {};
\node[Kanji] at (13.200000, 3.600000) {\textcolor[HTML]{154caa}{遅}};
\node[Onyomi] at (13.250000, 3.200000) {チ};
\node[Kunyomi] at (13.150000, 3.200000) {おそ.い};
\node[Meaning] at (13.200000, 4.850000) {slow};
\node[Square] at (15.250000, 3.100000) {};
\node[Kanji] at (15.250000, 3.600000) {\textcolor[HTML]{14469c}{漏}};
\node[Onyomi] at (15.300000, 3.200000) {ロウ};
\node[Kunyomi] at (15.200000, 3.200000) {も.らす};
\node[Meaning] at (15.250000, 4.850000) {leak};
\node[Square] at (17.300000, 3.100000) {};
\node[Kanji] at (17.300000, 3.600000) {\textcolor[HTML]{113066}{刷}};
\node[Onyomi] at (17.350000, 3.200000) {サツ};
\node[Kunyomi] at (17.250000, 3.200000) {す.る};
\node[Meaning] at (17.300000, 4.850000) {printing};
\node[Square] at (19.350000, 3.100000) {};
\node[Kanji] at (19.350000, 3.600000) {\textcolor[HTML]{0e254c}{尺}};
\node[Onyomi] at (19.400000, 3.200000) {シャク};
\node[Meaning] at (19.350000, 4.850000) {shaku};
\node[Square] at (21.400000, 3.100000) {};
\node[Kanji] at (21.400000, 3.600000) {\textcolor[HTML]{14418e}{尽}};
\node[Onyomi] at (21.450000, 3.200000) {ジン};
\node[Kunyomi] at (21.350000, 3.200000) {つ.くす};
\node[Meaning] at (21.400000, 4.850000) {exhaust};
\node[Square] at (23.450000, 3.100000) {};
\node[Kanji] at (23.450000, 3.600000) {\textcolor[HTML]{133c80}{沢}};
\node[Onyomi] at (23.500000, 3.200000) {タク};
\node[Kunyomi] at (23.400000, 3.200000) {さわ};
\node[Meaning] at (23.450000, 4.850000) {swamp};
\node[Square] at (25.500000, 3.100000) {};
\node[Kanji] at (25.500000, 3.600000) {\textcolor[HTML]{14469c}{訳}};
\node[Onyomi] at (25.550000, 3.200000) {ヤク};
\node[Kunyomi] at (25.450000, 3.200000) {わけ};
\node[Meaning] at (25.500000, 4.850000) {translation};
\node[Square] at (27.550000, 3.100000) {};
\node[Kanji] at (27.550000, 3.600000) {\textcolor[HTML]{123673}{択}};
\node[Onyomi] at (27.600000, 3.200000) {タク};
\node[Kunyomi] at (27.500000, 3.200000) {えら.ぶ};
\node[Meaning] at (27.550000, 4.850000) {select};
\node[Square] at (29.600000, 3.100000) {};
\node[Kanji] at (29.600000, 3.600000) {\textcolor[HTML]{14469c}{昼}};
\node[Kunyomi] at (29.550000, 3.200000) {ひる};
\node[Meaning] at (29.600000, 4.850000) {noon};
\node[Square] at (31.650000, 3.100000) {};
\node[Kanji] at (31.650000, 3.600000) {\textcolor[HTML]{154caa}{戸}};
\node[Onyomi] at (31.700000, 3.200000) {コ};
\node[Kunyomi] at (31.600000, 3.200000) {と};
\node[Meaning] at (31.650000, 4.850000) {door};
\node[Square] at (33.700000, 3.100000) {};
\node[Kanji] at (33.700000, 3.600000) {\textcolor[HTML]{154caa}{肩}};
\node[Onyomi] at (33.750000, 3.200000) {ケン};
\node[Kunyomi] at (33.650000, 3.200000) {かた};
\node[Meaning] at (33.700000, 4.850000) {shoulder};
\node[Square] at (35.750000, 3.100000) {};
\node[Kanji] at (35.750000, 3.600000) {\textcolor[HTML]{14469c}{房}};
\node[Onyomi] at (35.800000, 3.200000) {ボウ};
\node[Kunyomi] at (35.700000, 3.200000) {ふさ};
\node[Meaning] at (35.750000, 4.850000) {cluster};
\node[Square] at (37.800000, 3.100000) {};
\node[Kanji] at (37.800000, 3.600000) {\textcolor[HTML]{113066}{扇}};
\node[Onyomi] at (37.850000, 3.200000) {セン};
\node[Kunyomi] at (37.750000, 3.200000) {おうぎ};
\node[Meaning] at (37.800000, 4.850000) {folding fan};
\node[Square] at (39.850000, 3.100000) {};
\node[Kanji] at (39.850000, 3.600000) {\textcolor[HTML]{154caa}{炉}};
\node[Onyomi] at (39.900000, 3.200000) {ロ};
\node[Kunyomi] at (39.800000, 3.200000) {いろり};
\node[Meaning] at (39.850000, 4.850000) {furnace};
\node[Square] at (41.900000, 3.100000) {};
\node[Kanji] at (41.900000, 3.600000) {\textcolor[HTML]{145cd5}{戻}};
\node[Onyomi] at (41.950000, 3.200000) {レイ};
\node[Kunyomi] at (41.850000, 3.200000) {もど};
\node[Meaning] at (41.900000, 4.850000) {return};
\node[Square] at (43.950000, 3.100000) {};
\node[Kanji] at (43.950000, 3.600000) {\textcolor[HTML]{154caa}{涙}};
\node[Onyomi] at (44.000000, 3.200000) {ルイ};
\node[Kunyomi] at (43.900000, 3.200000) {なみだ};
\node[Meaning] at (43.950000, 4.850000) {teardrop};
\node[Square] at (46.000000, 3.100000) {};
\node[Kanji] at (46.000000, 3.600000) {\textcolor[HTML]{123673}{雇}};
\node[Onyomi] at (46.050000, 3.200000) {コ};
\node[Kunyomi] at (45.950000, 3.200000) {やと.う};
\node[Meaning] at (46.000000, 4.850000) {employ};
\node[Square] at (48.050000, 3.100000) {};
\node[Kanji] at (48.050000, 3.600000) {\textcolor[HTML]{0e254c}{顧}};
\node[Onyomi] at (48.100000, 3.200000) {コ};
\node[Kunyomi] at (48.000000, 3.200000) {かえり.みる};
\node[Meaning] at (48.050000, 4.850000) {review};
\node[Square] at (50.100000, 3.100000) {};
\node[Kanji] at (50.100000, 3.600000) {\textcolor[HTML]{102b59}{啓}};
\node[Onyomi] at (50.150000, 3.200000) {ケイ};
\node[Kunyomi] at (50.050000, 3.200000) {さと};
\node[Meaning] at (50.100000, 4.850000) {enlighten};
\node[Square] at (52.150000, 3.100000) {};
\node[Kanji] at (52.150000, 3.600000) {\textcolor[HTML]{154caa}{示}};
\node[Onyomi] at (52.200000, 3.200000) {ジ};
\node[Kunyomi] at (52.100000, 3.200000) {しめ.す};
\node[Meaning] at (52.150000, 4.850000) {indicate};
\node[Square] at (54.200000, 3.100000) {};
\node[Kanji] at (54.200000, 3.600000) {\textcolor[HTML]{14469c}{礼}};
\node[Onyomi] at (54.250000, 3.200000) {レイ};
\node[Meaning] at (54.200000, 4.850000) {thanks};
\node[Square] at (56.250000, 3.100000) {};
\node[Kanji] at (56.250000, 3.600000) {\textcolor[HTML]{0e254c}{祥}};
\node[Onyomi] at (56.300000, 3.200000) {ショウ};
\node[Kunyomi] at (56.200000, 3.200000) {きざ};
\node[Meaning] at (56.250000, 4.850000) {auspicious};
\node[Meaning] at (-58.500000, 3.650000) {62.34\%};
\node[Square] at (-56.500000, 1.050000) {};
\node[Kanji] at (-56.500000, 1.550000) {\textcolor[HTML]{14418e}{祝}};
\node[Onyomi] at (-56.450000, 1.150000) {シュク};
\node[Kunyomi] at (-56.550000, 1.150000) {いわ.う};
\node[Meaning] at (-56.500000, 2.800000) {celebrate};
\node[Square] at (-54.450000, 1.050000) {};
\node[Kanji] at (-54.450000, 1.550000) {\textcolor[HTML]{154caa}{福}};
\node[Onyomi] at (-54.400000, 1.150000) {フク};
\node[Meaning] at (-54.450000, 2.800000) {luck};
\node[Square] at (-52.400000, 1.050000) {};
\node[Kanji] at (-52.400000, 1.550000) {\textcolor[HTML]{113066}{祉}};
\node[Onyomi] at (-52.350000, 1.150000) {シ};
\node[Meaning] at (-52.400000, 2.800000) {welfare};
\node[Square] at (-50.350000, 1.050000) {};
\node[Kanji] at (-50.350000, 1.550000) {\textcolor[HTML]{145cd5}{社}};
\node[Onyomi] at (-50.300000, 1.150000) {シャ};
\node[Kunyomi] at (-50.400000, 1.150000) {やしろ};
\node[Meaning] at (-50.350000, 2.800000) {company};
\node[Square] at (-48.300000, 1.050000) {};
\node[Kanji] at (-48.300000, 1.550000) {\textcolor[HTML]{154caa}{視}};
\node[Onyomi] at (-48.250000, 1.150000) {シ};
\node[Meaning] at (-48.300000, 2.800000) {look at};
\node[Square] at (-46.250000, 1.050000) {};
\node[Kanji] at (-46.250000, 1.550000) {\textcolor[HTML]{14418e}{奈}};
\node[Onyomi] at (-46.200000, 1.150000) {ナ};
\node[Kunyomi] at (-46.300000, 1.150000) {な};
\node[Meaning] at (-46.250000, 2.800000) {nara};
\node[Square] at (-44.200000, 1.050000) {};
\node[Kanji] at (-44.200000, 1.550000) {\textcolor[HTML]{0e254c}{尉}};
\node[Onyomi] at (-44.150000, 1.150000) {イ};
\node[Meaning] at (-44.200000, 2.800000) {military officer};
\node[Square] at (-42.150000, 1.050000) {};
\node[Kanji] at (-42.150000, 1.550000) {\textcolor[HTML]{133c80}{慰}};
\node[Onyomi] at (-42.100000, 1.150000) {イ};
\node[Kunyomi] at (-42.200000, 1.150000) {なぐさ.*};
\node[Meaning] at (-42.150000, 2.800000) {consolation};
\node[Square] at (-40.100000, 1.050000) {};
\node[Kanji] at (-40.100000, 1.550000) {\textcolor[HTML]{0e254c}{款}};
\node[Onyomi] at (-40.050000, 1.150000) {カン};
\node[Meaning] at (-40.100000, 2.800000) {article};
\node[Square] at (-38.050000, 1.050000) {};
\node[Kanji] at (-38.050000, 1.550000) {\textcolor[HTML]{154caa}{禁}};
\node[Onyomi] at (-38.000000, 1.150000) {キン};
\node[Meaning] at (-38.050000, 2.800000) {prohibition};
\node[Square] at (-36.000000, 1.050000) {};
\node[Kanji] at (-36.000000, 1.550000) {\textcolor[HTML]{123673}{襟}};
\node[Onyomi] at (-35.950000, 1.150000) {キン};
\node[Kunyomi] at (-36.050000, 1.150000) {えり};
\node[Meaning] at (-36.000000, 2.800000) {collar};
\node[Square] at (-33.950000, 1.050000) {};
\node[Kanji] at (-33.950000, 1.550000) {\textcolor[HTML]{123673}{宗}};
\node[Onyomi] at (-33.900000, 1.150000) {シュウ};
\node[Meaning] at (-33.950000, 2.800000) {religion};
\node[Square] at (-31.900000, 1.050000) {};
\node[Kanji] at (-31.900000, 1.550000) {\textcolor[HTML]{113066}{崇}};
\node[Onyomi] at (-31.850000, 1.150000) {スウ};
\node[Kunyomi] at (-31.950000, 1.150000) {あが};
\node[Meaning] at (-31.900000, 2.800000) {worship};
\node[Square] at (-29.850000, 1.050000) {};
\node[Kanji] at (-29.850000, 1.550000) {\textcolor[HTML]{14469c}{祭}};
\node[Onyomi] at (-29.800000, 1.150000) {サイ};
\node[Kunyomi] at (-29.900000, 1.150000) {まつり};
\node[Meaning] at (-29.850000, 2.800000) {festival};
\node[Square] at (-27.800000, 1.050000) {};
\node[Kanji] at (-27.800000, 1.550000) {\textcolor[HTML]{1551b8}{察}};
\node[Onyomi] at (-27.750000, 1.150000) {サツ};
\node[Kunyomi] at (-27.850000, 1.150000) {さっ.する};
\node[Meaning] at (-27.800000, 2.800000) {guess};
\node[Square] at (-25.750000, 1.050000) {};
\node[Kanji] at (-25.750000, 1.550000) {\textcolor[HTML]{133c80}{擦}};
\node[Onyomi] at (-25.700000, 1.150000) {サツ};
\node[Kunyomi] at (-25.800000, 1.150000) {こす};
\node[Meaning] at (-25.750000, 2.800000) {grate};
\node[Square] at (-23.700000, 1.050000) {};
\node[Kanji] at (-23.700000, 1.550000) {\textcolor[HTML]{1551b8}{由}};
\node[Onyomi] at (-23.650000, 1.150000) {ユウ};
\node[Kunyomi] at (-23.750000, 1.150000) {よし};
\node[Meaning] at (-23.700000, 2.800000) {reason};
\node[Square] at (-21.650000, 1.050000) {};
\node[Kanji] at (-21.650000, 1.550000) {\textcolor[HTML]{0e254c}{抽}};
\node[Onyomi] at (-21.600000, 1.150000) {チュウ};
\node[Meaning] at (-21.650000, 2.800000) {pluck};
\node[Square] at (-19.600000, 1.050000) {};
\node[Kanji] at (-19.600000, 1.550000) {\textcolor[HTML]{14469c}{油}};
\node[Onyomi] at (-19.550000, 1.150000) {ユ};
\node[Kunyomi] at (-19.650000, 1.150000) {あぶら};
\node[Meaning] at (-19.600000, 2.800000) {oil};
\node[Square] at (-17.550000, 1.050000) {};
\node[Kanji] at (-17.550000, 1.550000) {\textcolor[HTML]{133c80}{袖}};
\node[Onyomi] at (-17.500000, 1.150000) {シュウ};
\node[Kunyomi] at (-17.600000, 1.150000) {そで};
\node[Meaning] at (-17.550000, 2.800000) {sleeve};
\node[Square] at (-15.500000, 1.050000) {};
\node[Kanji] at (-15.500000, 1.550000) {\textcolor[HTML]{1551b8}{宙}};
\node[Onyomi] at (-15.450000, 1.150000) {チュウ};
\node[Meaning] at (-15.500000, 2.800000) {mid air};
\node[Square] at (-13.450000, 1.050000) {};
\node[Kanji] at (-13.450000, 1.550000) {\textcolor[HTML]{154caa}{届}};
\node[Kunyomi] at (-13.500000, 1.150000) {とど};
\node[Meaning] at (-13.450000, 2.800000) {deliver};
\node[Square] at (-11.400000, 1.050000) {};
\node[Kanji] at (-11.400000, 1.550000) {\textcolor[HTML]{123673}{笛}};
\node[Onyomi] at (-11.350000, 1.150000) {テキ};
\node[Kunyomi] at (-11.450000, 1.150000) {ふえ};
\node[Meaning] at (-11.400000, 2.800000) {flute};
\node[Square] at (-9.350000, 1.050000) {};
\node[Kanji] at (-9.350000, 1.550000) {\textcolor[HTML]{0e254c}{軸}};
\node[Onyomi] at (-9.300000, 1.150000) {ジク};
\node[Meaning] at (-9.350000, 2.800000) {axis};
\node[Square] at (-7.300000, 1.050000) {};
\node[Kanji] at (-7.300000, 1.550000) {\textcolor[HTML]{14469c}{甲}};
\node[Onyomi] at (-7.250000, 1.150000) {コウ};
\node[Kunyomi] at (-7.350000, 1.150000) {か};
\node[Meaning] at (-7.300000, 2.800000) {turtle shell};
\node[Square] at (-5.250000, 1.050000) {};
\node[Kanji] at (-5.250000, 1.550000) {\textcolor[HTML]{1557c6}{押}};
\node[Onyomi] at (-5.200000, 1.150000) {オウ};
\node[Kunyomi] at (-5.300000, 1.150000) {お};
\node[Meaning] at (-5.250000, 2.800000) {push};
\node[Square] at (-3.200000, 1.050000) {};
\node[Kanji] at (-3.200000, 1.550000) {\textcolor[HTML]{0e254c}{岬}};
\node[Onyomi] at (-3.150000, 1.150000) {コウ};
\node[Kunyomi] at (-3.250000, 1.150000) {みさき};
\node[Meaning] at (-3.200000, 2.800000) {cape};
\node[Square] at (-1.150000, 1.050000) {};
\node[Kanji] at (-1.150000, 1.550000) {\textcolor[HTML]{0e254c}{挿}};
\node[Onyomi] at (-1.100000, 1.150000) {ソウ};
\node[Kunyomi] at (-1.200000, 1.150000) {さ.す};
\node[Meaning] at (-1.150000, 2.800000) {insert};
\node[Square] at (0.900000, 1.050000) {};
\node[Kanji] at (0.900000, 1.550000) {\textcolor[HTML]{154caa}{申}};
\node[Onyomi] at (0.950000, 1.150000) {シン};
\node[Kunyomi] at (0.850000, 1.150000) {もう};
\node[Meaning] at (0.900000, 2.800000) {say humbly};
\node[Square] at (2.950000, 1.050000) {};
\node[Kanji] at (2.950000, 1.550000) {\textcolor[HTML]{1551b8}{伸}};
\node[Onyomi] at (3.000000, 1.150000) {シン};
\node[Kunyomi] at (2.900000, 1.150000) {の};
\node[Meaning] at (2.950000, 2.800000) {stretch};
\node[Square] at (5.000000, 1.050000) {};
\node[Kanji] at (5.000000, 1.550000) {\textcolor[HTML]{1551b8}{神}};
\node[Onyomi] at (5.050000, 1.150000) {シン};
\node[Kunyomi] at (4.950000, 1.150000) {かみ};
\node[Meaning] at (5.000000, 2.800000) {god};
\node[Square] at (7.050000, 1.050000) {};
\node[Kanji] at (7.050000, 1.550000) {\textcolor[HTML]{133c80}{捜}};
\node[Onyomi] at (7.100000, 1.150000) {ソウ};
\node[Kunyomi] at (7.000000, 1.150000) {さが.す};
\node[Meaning] at (7.050000, 2.800000) {search};
\node[Square] at (9.100000, 1.050000) {};
\node[Kanji] at (9.100000, 1.550000) {\textcolor[HTML]{1551b8}{果}};
\node[Onyomi] at (9.150000, 1.150000) {カ};
\node[Kunyomi] at (9.050000, 1.150000) {くだ};
\node[Meaning] at (9.100000, 2.800000) {fruit};
\node[Square] at (11.150000, 1.050000) {};
\node[Kanji] at (11.150000, 1.550000) {\textcolor[HTML]{133c80}{菓}};
\node[Onyomi] at (11.200000, 1.150000) {カ};
\node[Meaning] at (11.150000, 2.800000) {cake};
\node[Square] at (13.200000, 1.050000) {};
\node[Kanji] at (13.200000, 1.550000) {\textcolor[HTML]{14469c}{課}};
\node[Onyomi] at (13.250000, 1.150000) {カ};
\node[Meaning] at (13.200000, 2.800000) {section};
\node[Square] at (15.250000, 1.050000) {};
\node[Kanji] at (15.250000, 1.550000) {\textcolor[HTML]{123673}{裸}};
\node[Onyomi] at (15.300000, 1.150000) {ラ};
\node[Kunyomi] at (15.200000, 1.150000) {はだか};
\node[Meaning] at (15.250000, 2.800000) {naked};
\node[Square] at (17.300000, 1.050000) {};
\node[Kanji] at (17.300000, 1.550000) {\textcolor[HTML]{0e254c}{斤}};
\node[Onyomi] at (17.350000, 1.150000) {キン};
\node[Meaning] at (17.300000, 2.800000) {axe};
\node[Square] at (19.350000, 1.050000) {};
\node[Kanji] at (19.350000, 1.550000) {\textcolor[HTML]{0e254c}{析}};
\node[Onyomi] at (19.400000, 1.150000) {セキ};
\node[Meaning] at (19.350000, 2.800000) {analysis};
\node[Square] at (21.400000, 1.050000) {};
\node[Kanji] at (21.400000, 1.550000) {\textcolor[HTML]{1461e3}{所}};
\node[Onyomi] at (21.450000, 1.150000) {ショ};
\node[Kunyomi] at (21.350000, 1.150000) {ところ};
\node[Meaning] at (21.400000, 2.800000) {place};
\node[Square] at (23.450000, 1.050000) {};
\node[Kanji] at (23.450000, 1.550000) {\textcolor[HTML]{14469c}{祈}};
\node[Onyomi] at (23.500000, 1.150000) {キ};
\node[Kunyomi] at (23.400000, 1.150000) {いの.る};
\node[Meaning] at (23.450000, 2.800000) {pray};
\node[Square] at (25.500000, 1.050000) {};
\node[Kanji] at (25.500000, 1.550000) {\textcolor[HTML]{145cd5}{近}};
\node[Onyomi] at (25.550000, 1.150000) {キン};
\node[Kunyomi] at (25.450000, 1.150000) {ちか.い};
\node[Meaning] at (25.500000, 2.800000) {near};
\node[Square] at (27.550000, 1.050000) {};
\node[Kanji] at (27.550000, 1.550000) {\textcolor[HTML]{154caa}{折}};
\node[Onyomi] at (27.600000, 1.150000) {セツ};
\node[Kunyomi] at (27.500000, 1.150000) {お.る};
\node[Meaning] at (27.550000, 2.800000) {fold};
\node[Square] at (29.600000, 1.050000) {};
\node[Kanji] at (29.600000, 1.550000) {\textcolor[HTML]{102b59}{哲}};
\node[Onyomi] at (29.650000, 1.150000) {テツ};
\node[Meaning] at (29.600000, 2.800000) {philosophy};
\node[Square] at (31.650000, 1.050000) {};
\node[Kanji] at (31.650000, 1.550000) {\textcolor[HTML]{102b59}{逝}};
\node[Onyomi] at (31.700000, 1.150000) {セイ};
\node[Kunyomi] at (31.600000, 1.150000) {い.く};
\node[Meaning] at (31.650000, 2.800000) {die};
\node[Square] at (33.700000, 1.050000) {};
\node[Kanji] at (33.700000, 1.550000) {\textcolor[HTML]{133c80}{誓}};
\node[Onyomi] at (33.750000, 1.150000) {セイ};
\node[Kunyomi] at (33.650000, 1.150000) {ちか.う};
\node[Meaning] at (33.700000, 2.800000) {vow};
\node[Square] at (35.750000, 1.050000) {};
\node[Kanji] at (35.750000, 1.550000) {\textcolor[HTML]{102b59}{暫}};
\node[Onyomi] at (35.800000, 1.150000) {ザン};
\node[Kunyomi] at (35.700000, 1.150000) {しばら.く};
\node[Meaning] at (35.750000, 2.800000) {temporarily};
\node[Square] at (37.800000, 1.050000) {};
\node[Kanji] at (37.800000, 1.550000) {\textcolor[HTML]{0e254c}{漸}};
\node[Onyomi] at (37.850000, 1.150000) {ゼン};
\node[Kunyomi] at (37.750000, 1.150000) {ようや};
\node[Meaning] at (37.800000, 2.800000) {gradually};
\node[Square] at (39.850000, 1.050000) {};
\node[Kanji] at (39.850000, 1.550000) {\textcolor[HTML]{154caa}{断}};
\node[Onyomi] at (39.900000, 1.150000) {ダン};
\node[Kunyomi] at (39.800000, 1.150000) {ことわ.る};
\node[Meaning] at (39.850000, 2.800000) {cut off};
\node[Square] at (41.900000, 1.050000) {};
\node[Kanji] at (41.900000, 1.550000) {\textcolor[HTML]{1551b8}{質}};
\node[Onyomi] at (41.950000, 1.150000) {シツ};
\node[Meaning] at (41.900000, 2.800000) {quality};
\node[Square] at (43.950000, 1.050000) {};
\node[Kanji] at (43.950000, 1.550000) {\textcolor[HTML]{0e254c}{斥}};
\node[Onyomi] at (44.000000, 1.150000) {セキ};
\node[Meaning] at (43.950000, 2.800000) {reject};
\node[Square] at (46.000000, 1.050000) {};
\node[Kanji] at (46.000000, 1.550000) {\textcolor[HTML]{133c80}{訴}};
\node[Onyomi] at (46.050000, 1.150000) {ソ};
\node[Kunyomi] at (45.950000, 1.150000) {うった.える};
\node[Meaning] at (46.000000, 2.800000) {sue};
\node[Square] at (48.050000, 1.050000) {};
\node[Kanji] at (48.050000, 1.550000) {\textcolor[HTML]{154caa}{昨}};
\node[Onyomi] at (48.100000, 1.150000) {サク};
\node[Meaning] at (48.050000, 2.800000) {previous};
\node[Square] at (50.100000, 1.050000) {};
\node[Kanji] at (50.100000, 1.550000) {\textcolor[HTML]{102b59}{詐}};
\node[Onyomi] at (50.150000, 1.150000) {サ};
\node[Kunyomi] at (50.050000, 1.150000) {いつわ.る};
\node[Meaning] at (50.100000, 2.800000) {lie};
\node[Square] at (52.150000, 1.050000) {};
\node[Kanji] at (52.150000, 1.550000) {\textcolor[HTML]{145cd5}{作}};
\node[Onyomi] at (52.200000, 1.150000) {サク};
\node[Kunyomi] at (52.100000, 1.150000) {つく.る};
\node[Meaning] at (52.150000, 2.800000) {make};
\node[Square] at (54.200000, 1.050000) {};
\node[Kanji] at (54.200000, 1.550000) {\textcolor[HTML]{154caa}{雪}};
\node[Onyomi] at (54.250000, 1.150000) {セツ};
\node[Kunyomi] at (54.150000, 1.150000) {ゆき};
\node[Meaning] at (54.200000, 2.800000) {snow};
\node[Square] at (56.250000, 1.050000) {};
\node[Kanji] at (56.250000, 1.550000) {\textcolor[HTML]{14469c}{録}};
\node[Onyomi] at (56.300000, 1.150000) {ロク};
\node[Meaning] at (56.250000, 2.800000) {record};
\node[Meaning] at (-58.500000, 1.600000) {64.26\%};
\node[Square] at (-56.500000, -1.000000) {};
\node[Kanji] at (-56.500000, -0.500000) {\textcolor[HTML]{14469c}{尋}};
\node[Onyomi] at (-56.450000, -0.900000) {ジン};
\node[Kunyomi] at (-56.550000, -0.900000) {たず.ねる};
\node[Meaning] at (-56.500000, 0.750000) {inquire};
\node[Square] at (-54.450000, -1.000000) {};
\node[Kanji] at (-54.450000, -0.500000) {\textcolor[HTML]{145cd5}{急}};
\node[Onyomi] at (-54.400000, -0.900000) {キュウ};
\node[Kunyomi] at (-54.500000, -0.900000) {いそ.ぐ};
\node[Meaning] at (-54.450000, 0.750000) {hurry};
\node[Square] at (-52.400000, -1.000000) {};
\node[Kanji] at (-52.400000, -0.500000) {\textcolor[HTML]{123673}{穏}};
\node[Onyomi] at (-52.350000, -0.900000) {オン};
\node[Kunyomi] at (-52.450000, -0.900000) {おだ.やか};
\node[Meaning] at (-52.400000, 0.750000) {calm};
\node[Square] at (-50.350000, -1.000000) {};
\node[Kanji] at (-50.350000, -0.500000) {\textcolor[HTML]{133c80}{侵}};
\node[Onyomi] at (-50.300000, -0.900000) {シン};
\node[Kunyomi] at (-50.400000, -0.900000) {おか.す};
\node[Meaning] at (-50.350000, 0.750000) {invade};
\node[Square] at (-48.300000, -1.000000) {};
\node[Kanji] at (-48.300000, -0.500000) {\textcolor[HTML]{133c80}{浸}};
\node[Onyomi] at (-48.250000, -0.900000) {シン};
\node[Kunyomi] at (-48.350000, -0.900000) {ひた.*};
\node[Meaning] at (-48.300000, 0.750000) {immersed};
\node[Square] at (-46.250000, -1.000000) {};
\node[Kanji] at (-46.250000, -0.500000) {\textcolor[HTML]{1551b8}{寝}};
\node[Onyomi] at (-46.200000, -0.900000) {シン};
\node[Kunyomi] at (-46.300000, -0.900000) {ね};
\node[Meaning] at (-46.250000, 0.750000) {lie down};
\node[Square] at (-44.200000, -1.000000) {};
\node[Kanji] at (-44.200000, -0.500000) {\textcolor[HTML]{14469c}{婦}};
\node[Onyomi] at (-44.150000, -0.900000) {フ};
\node[Meaning] at (-44.200000, 0.750000) {wife};
\node[Square] at (-42.150000, -1.000000) {};
\node[Kanji] at (-42.150000, -0.500000) {\textcolor[HTML]{14418e}{掃}};
\node[Onyomi] at (-42.100000, -0.900000) {ソウ};
\node[Kunyomi] at (-42.200000, -0.900000) {は.く};
\node[Meaning] at (-42.150000, 0.750000) {sweep};
\node[Square] at (-40.100000, -1.000000) {};
\node[Kanji] at (-40.100000, -0.500000) {\textcolor[HTML]{145cd5}{当}};
\node[Onyomi] at (-40.050000, -0.900000) {トウ};
\node[Kunyomi] at (-40.150000, -0.900000) {あ.たる};
\node[Meaning] at (-40.100000, 0.750000) {right};
\node[Square] at (-38.050000, -1.000000) {};
\node[Kanji] at (-38.050000, -0.500000) {\textcolor[HTML]{154caa}{争}};
\node[Onyomi] at (-38.000000, -0.900000) {ソウ};
\node[Kunyomi] at (-38.100000, -0.900000) {あらそ.う};
\node[Meaning] at (-38.050000, 0.750000) {conflict};
\node[Square] at (-36.000000, -1.000000) {};
\node[Kanji] at (-36.000000, -0.500000) {\textcolor[HTML]{0e254c}{浄}};
\node[Onyomi] at (-35.950000, -0.900000) {ジョウ};
\node[Kunyomi] at (-36.050000, -0.900000) {きよ.い};
\node[Meaning] at (-36.000000, 0.750000) {cleanse};
\node[Square] at (-33.950000, -1.000000) {};
\node[Kanji] at (-33.950000, -0.500000) {\textcolor[HTML]{1461e3}{事}};
\node[Onyomi] at (-33.900000, -0.900000) {ジ};
\node[Kunyomi] at (-34.000000, -0.900000) {こと};
\node[Meaning] at (-33.950000, 0.750000) {action};
\node[Square] at (-31.900000, -1.000000) {};
\node[Kanji] at (-31.900000, -0.500000) {\textcolor[HTML]{133c80}{唐}};
\node[Onyomi] at (-31.850000, -0.900000) {トウ};
\node[Meaning] at (-31.900000, 0.750000) {china};
\node[Square] at (-29.850000, -1.000000) {};
\node[Kanji] at (-29.850000, -0.500000) {\textcolor[HTML]{133c80}{糖}};
\node[Onyomi] at (-29.800000, -0.900000) {トウ};
\node[Meaning] at (-29.850000, 0.750000) {sugar};
\node[Square] at (-27.800000, -1.000000) {};
\node[Kanji] at (-27.800000, -0.500000) {\textcolor[HTML]{1551b8}{康}};
\node[Onyomi] at (-27.750000, -0.900000) {コウ};
\node[Meaning] at (-27.800000, 0.750000) {health};
\node[Square] at (-25.750000, -1.000000) {};
\node[Kanji] at (-25.750000, -0.500000) {\textcolor[HTML]{14418e}{逮}};
\node[Onyomi] at (-25.700000, -0.900000) {タイ};
\node[Meaning] at (-25.750000, 0.750000) {apprehend};
\node[Square] at (-23.700000, -1.000000) {};
\node[Kanji] at (-23.700000, -0.500000) {\textcolor[HTML]{133c80}{伊}};
\node[Onyomi] at (-23.650000, -0.900000) {イ};
\node[Kunyomi] at (-23.750000, -0.900000) {だ};
\node[Meaning] at (-23.700000, 0.750000) {italy};
\node[Square] at (-21.650000, -1.000000) {};
\node[Kanji] at (-21.650000, -0.500000) {\textcolor[HTML]{1461e3}{君}};
\node[Onyomi] at (-21.600000, -0.900000) {クン};
\node[Kunyomi] at (-21.700000, -0.900000) {きみ};
\node[Meaning] at (-21.650000, 0.750000) {buddy};
\node[Square] at (-19.600000, -1.000000) {};
\node[Kanji] at (-19.600000, -0.500000) {\textcolor[HTML]{14469c}{群}};
\node[Onyomi] at (-19.550000, -0.900000) {グン};
\node[Kunyomi] at (-19.650000, -0.900000) {む};
\node[Meaning] at (-19.600000, 0.750000) {flock};
\node[Square] at (-17.550000, -1.000000) {};
\node[Kanji] at (-17.550000, -0.500000) {\textcolor[HTML]{14418e}{耐}};
\node[Onyomi] at (-17.500000, -0.900000) {タイ};
\node[Kunyomi] at (-17.600000, -0.900000) {た.える};
\node[Meaning] at (-17.550000, 0.750000) {resistant};
\node[Square] at (-15.500000, -1.000000) {};
\node[Kanji] at (-15.500000, -0.500000) {\textcolor[HTML]{0e254c}{需}};
\node[Onyomi] at (-15.450000, -0.900000) {ジュ};
\node[Meaning] at (-15.500000, 0.750000) {demand};
\node[Square] at (-13.450000, -1.000000) {};
\node[Kanji] at (-13.450000, -0.500000) {\textcolor[HTML]{0e254c}{儒}};
\node[Onyomi] at (-13.400000, -0.900000) {ジュ};
\node[Meaning] at (-13.450000, 0.750000) {Confucian};
\node[Square] at (-11.400000, -1.000000) {};
\node[Kanji] at (-11.400000, -0.500000) {\textcolor[HTML]{1551b8}{端}};
\node[Onyomi] at (-11.350000, -0.900000) {タン};
\node[Kunyomi] at (-11.450000, -0.900000) {はし};
\node[Meaning] at (-11.400000, 0.750000) {edge};
\node[Square] at (-9.350000, -1.000000) {};
\node[Kanji] at (-9.350000, -0.500000) {\textcolor[HTML]{1557c6}{両}};
\node[Onyomi] at (-9.300000, -0.900000) {リョウ};
\node[Meaning] at (-9.350000, 0.750000) {both};
\node[Square] at (-7.300000, -1.000000) {};
\node[Kanji] at (-7.300000, -0.500000) {\textcolor[HTML]{154caa}{満}};
\node[Onyomi] at (-7.250000, -0.900000) {マン};
\node[Kunyomi] at (-7.350000, -0.900000) {み};
\node[Meaning] at (-7.300000, 0.750000) {full};
\node[Square] at (-5.250000, -1.000000) {};
\node[Kanji] at (-5.250000, -0.500000) {\textcolor[HTML]{1557c6}{画}};
\node[Onyomi] at (-5.200000, -0.900000) {ガ};
\node[Meaning] at (-5.250000, 0.750000) {drawing};
\node[Square] at (-3.200000, -1.000000) {};
\node[Kanji] at (-3.200000, -0.500000) {\textcolor[HTML]{154caa}{歯}};
\node[Kunyomi] at (-3.250000, -0.900000) {は};
\node[Meaning] at (-3.200000, 0.750000) {tooth};
\node[Square] at (-1.150000, -1.000000) {};
\node[Kanji] at (-1.150000, -0.500000) {\textcolor[HTML]{154caa}{曲}};
\node[Onyomi] at (-1.100000, -0.900000) {キョク};
\node[Kunyomi] at (-1.200000, -0.900000) {ま.げる};
\node[Meaning] at (-1.150000, 0.750000) {music};
\node[Square] at (0.900000, -1.000000) {};
\node[Kanji] at (0.900000, -0.500000) {\textcolor[HTML]{0e254c}{曹}};
\node[Onyomi] at (0.950000, -0.900000) {ソウ};
\node[Kunyomi] at (0.850000, -0.900000) {つかさ};
\node[Meaning] at (0.900000, 0.750000) {official};
\node[Square] at (2.950000, -1.000000) {};
\node[Kanji] at (2.950000, -0.500000) {\textcolor[HTML]{133c80}{遭}};
\node[Onyomi] at (3.000000, -0.900000) {ソウ};
\node[Kunyomi] at (2.900000, -0.900000) {あ};
\node[Meaning] at (2.950000, 0.750000) {encounter};
\node[Square] at (5.000000, -1.000000) {};
\node[Kanji] at (5.000000, -0.500000) {\textcolor[HTML]{133c80}{槽}};
\node[Onyomi] at (5.050000, -0.900000) {ソウ};
\node[Kunyomi] at (4.950000, -0.900000) {ふね};
\node[Meaning] at (5.000000, 0.750000) {tank};
\node[Square] at (7.050000, -1.000000) {};
\node[Kanji] at (7.050000, -0.500000) {\textcolor[HTML]{0e254c}{斗}};
\node[Onyomi] at (7.100000, -0.900000) {ト};
\node[Meaning] at (7.050000, 0.750000) {ladle};
\node[Square] at (9.100000, -1.000000) {};
\node[Kanji] at (9.100000, -0.500000) {\textcolor[HTML]{1551b8}{料}};
\node[Onyomi] at (9.150000, -0.900000) {リョウ};
\node[Meaning] at (9.100000, 0.750000) {fee};
\node[Square] at (11.150000, -1.000000) {};
\node[Kanji] at (11.150000, -0.500000) {\textcolor[HTML]{1551b8}{科}};
\node[Onyomi] at (11.200000, -0.900000) {カ};
\node[Meaning] at (11.150000, 0.750000) {science};
\node[Square] at (13.200000, -1.000000) {};
\node[Kanji] at (13.200000, -0.500000) {\textcolor[HTML]{1551b8}{図}};
\node[Onyomi] at (13.250000, -0.900000) {ズ};
\node[Kunyomi] at (13.150000, -0.900000) {え};
\node[Meaning] at (13.200000, 0.750000) {diagram};
\node[Square] at (15.250000, -1.000000) {};
\node[Kanji] at (15.250000, -0.500000) {\textcolor[HTML]{1557c6}{用}};
\node[Onyomi] at (15.300000, -0.900000) {ヨウ};
\node[Kunyomi] at (15.200000, -0.900000) {もち.いる};
\node[Meaning] at (15.250000, 0.750000) {task};
\node[Square] at (17.300000, -1.000000) {};
\node[Kanji] at (17.300000, -0.500000) {\textcolor[HTML]{0e254c}{庸}};
\node[Onyomi] at (17.350000, -0.900000) {ヨウ};
\node[Meaning] at (17.300000, 0.750000) {common};
\node[Square] at (19.350000, -1.000000) {};
\node[Kanji] at (19.350000, -0.500000) {\textcolor[HTML]{154caa}{備}};
\node[Onyomi] at (19.400000, -0.900000) {ビ};
\node[Kunyomi] at (19.300000, -0.900000) {そな.える};
\node[Meaning] at (19.350000, 0.750000) {provide};
\node[Square] at (21.400000, -1.000000) {};
\node[Kanji] at (21.400000, -0.500000) {\textcolor[HTML]{154caa}{昔}};
\node[Kunyomi] at (21.350000, -0.900000) {むかし};
\node[Meaning] at (21.400000, 0.750000) {long ago};
\node[Square] at (23.450000, -1.000000) {};
\node[Kanji] at (23.450000, -0.500000) {\textcolor[HTML]{113066}{錯}};
\node[Onyomi] at (23.500000, -0.900000) {サク};
\node[Meaning] at (23.450000, 0.750000) {confused};
\node[Square] at (25.500000, -1.000000) {};
\node[Kanji] at (25.500000, -0.500000) {\textcolor[HTML]{14418e}{借}};
\node[Onyomi] at (25.550000, -0.900000) {シャク};
\node[Kunyomi] at (25.450000, -0.900000) {か.りる};
\node[Meaning] at (25.500000, 0.750000) {borrow};
\node[Square] at (27.550000, -1.000000) {};
\node[Kanji] at (27.550000, -0.500000) {\textcolor[HTML]{102b59}{惜}};
\node[Onyomi] at (27.600000, -0.900000) {セキ};
\node[Kunyomi] at (27.500000, -0.900000) {お};
\node[Meaning] at (27.550000, 0.750000) {frugal};
\node[Square] at (29.600000, -1.000000) {};
\node[Kanji] at (29.600000, -0.500000) {\textcolor[HTML]{113066}{措}};
\node[Onyomi] at (29.650000, -0.900000) {ソ};
\node[Meaning] at (29.600000, 0.750000) {set aside};
\node[Square] at (31.650000, -1.000000) {};
\node[Kanji] at (31.650000, -0.500000) {\textcolor[HTML]{154caa}{散}};
\node[Onyomi] at (31.700000, -0.900000) {サン};
\node[Kunyomi] at (31.600000, -0.900000) {ち.*};
\node[Meaning] at (31.650000, 0.750000) {scatter};
\node[Square] at (33.700000, -1.000000) {};
\node[Kanji] at (33.700000, -0.500000) {\textcolor[HTML]{0e254c}{庶}};
\node[Onyomi] at (33.750000, -0.900000) {ショ};
\node[Meaning] at (33.700000, 0.750000) {all};
\node[Square] at (35.750000, -1.000000) {};
\node[Kanji] at (35.750000, -0.500000) {\textcolor[HTML]{133c80}{遮}};
\node[Onyomi] at (35.800000, -0.900000) {シャ};
\node[Kunyomi] at (35.700000, -0.900000) {さえぎ};
\node[Meaning] at (35.750000, 0.750000) {intercept};
\node[Square] at (37.800000, -1.000000) {};
\node[Kanji] at (37.800000, -0.500000) {\textcolor[HTML]{1551b8}{席}};
\node[Onyomi] at (37.850000, -0.900000) {セキ};
\node[Meaning] at (37.800000, 0.750000) {seat};
\node[Square] at (39.850000, -1.000000) {};
\node[Kanji] at (39.850000, -0.500000) {\textcolor[HTML]{145cd5}{度}};
\node[Onyomi] at (39.900000, -0.900000) {ド};
\node[Kunyomi] at (39.800000, -0.900000) {たび};
\node[Meaning] at (39.850000, 0.750000) {degrees};
\node[Square] at (41.900000, -1.000000) {};
\node[Kanji] at (41.900000, -0.500000) {\textcolor[HTML]{1551b8}{渡}};
\node[Onyomi] at (41.950000, -0.900000) {ト};
\node[Kunyomi] at (41.850000, -0.900000) {わた};
\node[Meaning] at (41.900000, 0.750000) {transit};
\node[Square] at (43.950000, -1.000000) {};
\node[Kanji] at (43.950000, -0.500000) {\textcolor[HTML]{0e254c}{奔}};
\node[Onyomi] at (44.000000, -0.900000) {ホン};
\node[Kunyomi] at (43.900000, -0.900000) {はし.る};
\node[Meaning] at (43.950000, 0.750000) {run};
\node[Square] at (46.000000, -1.000000) {};
\node[Kanji] at (46.000000, -0.500000) {\textcolor[HTML]{14469c}{噴}};
\node[Onyomi] at (46.050000, -0.900000) {フン};
\node[Kunyomi] at (45.950000, -0.900000) {ふ};
\node[Meaning] at (46.000000, 0.750000) {erupt};
\node[Square] at (48.050000, -1.000000) {};
\node[Kanji] at (48.050000, -0.500000) {\textcolor[HTML]{102b59}{墳}};
\node[Onyomi] at (48.100000, -0.900000) {フン};
\node[Meaning] at (48.050000, 0.750000) {tomb};
\node[Square] at (50.100000, -1.000000) {};
\node[Kanji] at (50.100000, -0.500000) {\textcolor[HTML]{14418e}{憤}};
\node[Onyomi] at (50.150000, -0.900000) {フン};
\node[Kunyomi] at (50.050000, -0.900000) {いきどお};
\node[Meaning] at (50.100000, 0.750000) {resent};
\node[Square] at (52.150000, -1.000000) {};
\node[Kanji] at (52.150000, -0.500000) {\textcolor[HTML]{154caa}{焼}};
\node[Onyomi] at (52.200000, -0.900000) {ショウ};
\node[Kunyomi] at (52.100000, -0.900000) {や};
\node[Meaning] at (52.150000, 0.750000) {bake};
\node[Square] at (54.200000, -1.000000) {};
\node[Kanji] at (54.200000, -0.500000) {\textcolor[HTML]{0e254c}{暁}};
\node[Onyomi] at (54.250000, -0.900000) {キョウ};
\node[Kunyomi] at (54.150000, -0.900000) {あかつき};
\node[Meaning] at (54.200000, 0.750000) {dawn};
\node[Square] at (56.250000, -1.000000) {};
\node[Kanji] at (56.250000, -0.500000) {\textcolor[HTML]{1557c6}{半}};
\node[Onyomi] at (56.300000, -0.900000) {ハン};
\node[Kunyomi] at (56.200000, -0.900000) {なか.ば};
\node[Meaning] at (56.250000, 0.750000) {half};
\node[Meaning] at (-58.500000, -0.450000) {67.08\%};
\node[Square] at (-56.500000, -3.050000) {};
\node[Kanji] at (-56.500000, -2.550000) {\textcolor[HTML]{102b59}{伴}};
\node[Onyomi] at (-56.450000, -2.950000) {ハン};
\node[Kunyomi] at (-56.550000, -2.950000) {ともな.う};
\node[Meaning] at (-56.500000, -1.300000) {accompany};
\node[Square] at (-54.450000, -3.050000) {};
\node[Kanji] at (-54.450000, -2.550000) {\textcolor[HTML]{0e254c}{畔}};
\node[Onyomi] at (-54.400000, -2.950000) {ハン};
\node[Kunyomi] at (-54.500000, -2.950000) {あぜ};
\node[Meaning] at (-54.450000, -1.300000) {shore};
\node[Square] at (-52.400000, -3.050000) {};
\node[Kanji] at (-52.400000, -2.550000) {\textcolor[HTML]{154caa}{判}};
\node[Onyomi] at (-52.350000, -2.950000) {ハン};
\node[Meaning] at (-52.400000, -1.300000) {judge};
\node[Square] at (-50.350000, -3.050000) {};
\node[Kanji] at (-50.350000, -2.550000) {\textcolor[HTML]{113066}{券}};
\node[Onyomi] at (-50.300000, -2.950000) {ケン};
\node[Meaning] at (-50.350000, -1.300000) {ticket};
\node[Square] at (-48.300000, -3.050000) {};
\node[Kanji] at (-48.300000, -2.550000) {\textcolor[HTML]{154caa}{巻}};
\node[Onyomi] at (-48.250000, -2.950000) {カン};
\node[Kunyomi] at (-48.350000, -2.950000) {ま.く};
\node[Meaning] at (-48.300000, -1.300000) {scroll};
\node[Square] at (-46.250000, -3.050000) {};
\node[Kanji] at (-46.250000, -2.550000) {\textcolor[HTML]{113066}{圏}};
\node[Onyomi] at (-46.200000, -2.950000) {ケン};
\node[Meaning] at (-46.250000, -1.300000) {range};
\node[Square] at (-44.200000, -3.050000) {};
\node[Kanji] at (-44.200000, -2.550000) {\textcolor[HTML]{1551b8}{勝}};
\node[Onyomi] at (-44.150000, -2.950000) {ショウ};
\node[Kunyomi] at (-44.250000, -2.950000) {か.つ};
\node[Meaning] at (-44.200000, -1.300000) {win};
\node[Square] at (-42.150000, -3.050000) {};
\node[Kanji] at (-42.150000, -2.550000) {\textcolor[HTML]{14469c}{藤}};
\node[Onyomi] at (-42.100000, -2.950000) {トウ};
\node[Kunyomi] at (-42.200000, -2.950000) {ふじ};
\node[Meaning] at (-42.150000, -1.300000) {wisteria};
\node[Square] at (-40.100000, -3.050000) {};
\node[Kanji] at (-40.100000, -2.550000) {\textcolor[HTML]{0e254c}{謄}};
\node[Onyomi] at (-40.050000, -2.950000) {トウ};
\node[Meaning] at (-40.100000, -1.300000) {mimeograph};
\node[Square] at (-38.050000, -3.050000) {};
\node[Kanji] at (-38.050000, -2.550000) {\textcolor[HTML]{1551b8}{片}};
\node[Onyomi] at (-38.000000, -2.950000) {ヘン};
\node[Kunyomi] at (-38.100000, -2.950000) {かた};
\node[Meaning] at (-38.050000, -1.300000) {one sided};
\node[Square] at (-36.000000, -3.050000) {};
\node[Kanji] at (-36.000000, -2.550000) {\textcolor[HTML]{113066}{版}};
\node[Onyomi] at (-35.950000, -2.950000) {ハン};
\node[Meaning] at (-36.000000, -1.300000) {edition};
\node[Square] at (-33.950000, -3.050000) {};
\node[Kanji] at (-33.950000, -2.550000) {\textcolor[HTML]{123673}{之}};
\node[Onyomi] at (-33.900000, -2.950000) {シ};
\node[Kunyomi] at (-34.000000, -2.950000) {これ};
\node[Meaning] at (-33.950000, -1.300000) {this};
\node[Square] at (-31.900000, -3.050000) {};
\node[Kanji] at (-31.900000, -2.550000) {\textcolor[HTML]{0e254c}{乏}};
\node[Onyomi] at (-31.850000, -2.950000) {ボウ};
\node[Kunyomi] at (-31.950000, -2.950000) {とぼ.しい};
\node[Meaning] at (-31.900000, -1.300000) {scarce};
\node[Square] at (-29.850000, -3.050000) {};
\node[Kanji] at (-29.850000, -2.550000) {\textcolor[HTML]{14418e}{芝}};
\node[Onyomi] at (-29.800000, -2.950000) {シ};
\node[Kunyomi] at (-29.900000, -2.950000) {しば};
\node[Meaning] at (-29.850000, -1.300000) {lawn};
\node[Square] at (-27.800000, -3.050000) {};
\node[Kanji] at (-27.800000, -2.550000) {\textcolor[HTML]{1557c6}{不}};
\node[Onyomi] at (-27.750000, -2.950000) {フ};
\node[Meaning] at (-27.800000, -1.300000) {not};
\node[Square] at (-25.750000, -3.050000) {};
\node[Kanji] at (-25.750000, -2.550000) {\textcolor[HTML]{14418e}{否}};
\node[Onyomi] at (-25.700000, -2.950000) {ヒ};
\node[Kunyomi] at (-25.800000, -2.950000) {いな};
\node[Meaning] at (-25.750000, -1.300000) {no};
\node[Square] at (-23.700000, -3.050000) {};
\node[Kanji] at (-23.700000, -2.550000) {\textcolor[HTML]{154caa}{杯}};
\node[Onyomi] at (-23.650000, -2.950000) {ハイ};
\node[Meaning] at (-23.700000, -1.300000) {cup of liquid};
\node[Square] at (-21.650000, -3.050000) {};
\node[Kanji] at (-21.650000, -2.550000) {\textcolor[HTML]{133c80}{矢}};
\node[Onyomi] at (-21.600000, -2.950000) {シ};
\node[Kunyomi] at (-21.700000, -2.950000) {や};
\node[Meaning] at (-21.650000, -1.300000) {arrow};
\node[Square] at (-19.600000, -3.050000) {};
\node[Kanji] at (-19.600000, -2.550000) {\textcolor[HTML]{0e254c}{矯}};
\node[Onyomi] at (-19.550000, -2.950000) {キョウ};
\node[Kunyomi] at (-19.650000, -2.950000) {た};
\node[Meaning] at (-19.600000, -1.300000) {correct};
\node[Square] at (-17.550000, -3.050000) {};
\node[Kanji] at (-17.550000, -2.550000) {\textcolor[HTML]{1551b8}{族}};
\node[Onyomi] at (-17.500000, -2.950000) {ゾク};
\node[Meaning] at (-17.550000, -1.300000) {tribe};
\node[Square] at (-15.500000, -3.050000) {};
\node[Kanji] at (-15.500000, -2.550000) {\textcolor[HTML]{1461e3}{知}};
\node[Onyomi] at (-15.450000, -2.950000) {チ};
\node[Kunyomi] at (-15.550000, -2.950000) {し.る};
\node[Meaning] at (-15.500000, -1.300000) {know};
\node[Square] at (-13.450000, -3.050000) {};
\node[Kanji] at (-13.450000, -2.550000) {\textcolor[HTML]{102b59}{智}};
\node[Onyomi] at (-13.400000, -2.950000) {チ};
\node[Meaning] at (-13.450000, -1.300000) {wisdom};
\node[Square] at (-11.400000, -3.050000) {};
\node[Kanji] at (-11.400000, -2.550000) {\textcolor[HTML]{0e254c}{矛}};
\node[Onyomi] at (-11.350000, -2.950000) {ム};
\node[Kunyomi] at (-11.450000, -2.950000) {ほこ};
\node[Meaning] at (-11.400000, -1.300000) {halberd};
\node[Square] at (-9.350000, -3.050000) {};
\node[Kanji] at (-9.350000, -2.550000) {\textcolor[HTML]{133c80}{柔}};
\node[Onyomi] at (-9.300000, -2.950000) {ジュウ};
\node[Kunyomi] at (-9.400000, -2.950000) {やわ.*};
\node[Meaning] at (-9.350000, -1.300000) {gentle};
\node[Square] at (-7.300000, -3.050000) {};
\node[Kanji] at (-7.300000, -2.550000) {\textcolor[HTML]{154caa}{務}};
\node[Onyomi] at (-7.250000, -2.950000) {ム};
\node[Kunyomi] at (-7.350000, -2.950000) {つと.める};
\node[Meaning] at (-7.300000, -1.300000) {task};
\node[Square] at (-5.250000, -3.050000) {};
\node[Kanji] at (-5.250000, -2.550000) {\textcolor[HTML]{14418e}{霧}};
\node[Onyomi] at (-5.200000, -2.950000) {ム};
\node[Kunyomi] at (-5.300000, -2.950000) {きり};
\node[Meaning] at (-5.250000, -1.300000) {fog};
\node[Square] at (-3.200000, -3.050000) {};
\node[Kanji] at (-3.200000, -2.550000) {\textcolor[HTML]{0e254c}{班}};
\node[Onyomi] at (-3.150000, -2.950000) {ハン};
\node[Meaning] at (-3.200000, -1.300000) {squad};
\node[Square] at (-1.150000, -3.050000) {};
\node[Kanji] at (-1.150000, -2.550000) {\textcolor[HTML]{1557c6}{帰}};
\node[Onyomi] at (-1.100000, -2.950000) {キ};
\node[Kunyomi] at (-1.200000, -2.950000) {かえ};
\node[Meaning] at (-1.150000, -1.300000) {return};
\node[Square] at (0.900000, -3.050000) {};
\node[Kanji] at (0.900000, -2.550000) {\textcolor[HTML]{123673}{弓}};
\node[Onyomi] at (0.950000, -2.950000) {キュウ};
\node[Kunyomi] at (0.850000, -2.950000) {ゆみ};
\node[Meaning] at (0.900000, -1.300000) {bow};
\node[Square] at (2.950000, -3.050000) {};
\node[Kanji] at (2.950000, -2.550000) {\textcolor[HTML]{145cd5}{引}};
\node[Onyomi] at (3.000000, -2.950000) {イン};
\node[Kunyomi] at (2.900000, -2.950000) {ひ};
\node[Meaning] at (2.950000, -1.300000) {pull};
\node[Square] at (5.000000, -3.050000) {};
\node[Kanji] at (5.000000, -2.550000) {\textcolor[HTML]{0e254c}{弔}};
\node[Onyomi] at (5.050000, -2.950000) {チョウ};
\node[Kunyomi] at (4.950000, -2.950000) {とぶら.う};
\node[Meaning] at (5.000000, -1.300000) {condolence};
\node[Square] at (7.050000, -3.050000) {};
\node[Kanji] at (7.050000, -2.550000) {\textcolor[HTML]{1557c6}{強}};
\node[Onyomi] at (7.100000, -2.950000) {キョウ};
\node[Kunyomi] at (7.000000, -2.950000) {つよ.い};
\node[Meaning] at (7.050000, -1.300000) {strong};
\node[Square] at (9.100000, -3.050000) {};
\node[Kanji] at (9.100000, -2.550000) {\textcolor[HTML]{154caa}{弱}};
\node[Onyomi] at (9.150000, -2.950000) {ジャク};
\node[Kunyomi] at (9.050000, -2.950000) {よわ.い};
\node[Meaning] at (9.100000, -1.300000) {weak};
\node[Square] at (11.150000, -3.050000) {};
\node[Kanji] at (11.150000, -2.550000) {\textcolor[HTML]{123673}{沸}};
\node[Onyomi] at (11.200000, -2.950000) {フツ};
\node[Kunyomi] at (11.100000, -2.950000) {わ};
\node[Meaning] at (11.150000, -1.300000) {boil};
\node[Square] at (13.200000, -3.050000) {};
\node[Kanji] at (13.200000, -2.550000) {\textcolor[HTML]{14418e}{費}};
\node[Onyomi] at (13.250000, -2.950000) {ヒ};
\node[Kunyomi] at (13.150000, -2.950000) {つい.やす};
\node[Meaning] at (13.200000, -1.300000) {expense};
\node[Square] at (15.250000, -3.050000) {};
\node[Kanji] at (15.250000, -2.550000) {\textcolor[HTML]{145cd5}{第}};
\node[Onyomi] at (15.300000, -2.950000) {ダイ};
\node[Meaning] at (15.250000, -1.300000) {ordinal number};
\node[Square] at (17.300000, -3.050000) {};
\node[Kanji] at (17.300000, -2.550000) {\textcolor[HTML]{14469c}{弟}};
\node[Onyomi] at (17.350000, -2.950000) {ダイ};
\node[Kunyomi] at (17.250000, -2.950000) {おとうと};
\node[Meaning] at (17.300000, -1.300000) {little brother};
\node[Square] at (19.350000, -3.050000) {};
\node[Kanji] at (19.350000, -2.550000) {\textcolor[HTML]{102b59}{巧}};
\node[Onyomi] at (19.400000, -2.950000) {コウ};
\node[Kunyomi] at (19.300000, -2.950000) {うま.い};
\node[Meaning] at (19.350000, -1.300000) {adept};
\node[Square] at (21.400000, -3.050000) {};
\node[Kanji] at (21.400000, -2.550000) {\textcolor[HTML]{154caa}{号}};
\node[Onyomi] at (21.450000, -2.950000) {ゴウ};
\node[Meaning] at (21.400000, -1.300000) {number};
\node[Square] at (23.450000, -3.050000) {};
\node[Kanji] at (23.450000, -2.550000) {\textcolor[HTML]{0e254c}{朽}};
\node[Onyomi] at (23.500000, -2.950000) {キュウ};
\node[Kunyomi] at (23.400000, -2.950000) {く};
\node[Meaning] at (23.450000, -1.300000) {rot};
\node[Square] at (25.500000, -3.050000) {};
\node[Kanji] at (25.500000, -2.550000) {\textcolor[HTML]{14418e}{誇}};
\node[Onyomi] at (25.550000, -2.950000) {コ};
\node[Kunyomi] at (25.450000, -2.950000) {ほこ.る};
\node[Meaning] at (25.500000, -1.300000) {pride};
\node[Square] at (27.550000, -3.050000) {};
\node[Kanji] at (27.550000, -2.550000) {\textcolor[HTML]{154caa}{汚}};
\node[Onyomi] at (27.600000, -2.950000) {オ};
\node[Kunyomi] at (27.500000, -2.950000) {よご};
\node[Meaning] at (27.550000, -1.300000) {dirty};
\node[Square] at (29.600000, -3.050000) {};
\node[Kanji] at (29.600000, -2.550000) {\textcolor[HTML]{14469c}{与}};
\node[Onyomi] at (29.650000, -2.950000) {ヨ};
\node[Kunyomi] at (29.550000, -2.950000) {あた.える};
\node[Meaning] at (29.600000, -1.300000) {give};
\node[Square] at (31.650000, -3.050000) {};
\node[Kanji] at (31.650000, -2.550000) {\textcolor[HTML]{1551b8}{写}};
\node[Onyomi] at (31.700000, -2.950000) {シャ};
\node[Kunyomi] at (31.600000, -2.950000) {うつ.す};
\node[Meaning] at (31.650000, -1.300000) {copy};
\node[Square] at (33.700000, -3.050000) {};
\node[Kanji] at (33.700000, -2.550000) {\textcolor[HTML]{145cd5}{身}};
\node[Onyomi] at (33.750000, -2.950000) {シン};
\node[Kunyomi] at (33.650000, -2.950000) {み};
\node[Meaning] at (33.700000, -1.300000) {somebody};
\node[Square] at (35.750000, -3.050000) {};
\node[Kanji] at (35.750000, -2.550000) {\textcolor[HTML]{154caa}{射}};
\node[Onyomi] at (35.800000, -2.950000) {シャ};
\node[Kunyomi] at (35.700000, -2.950000) {い.る};
\node[Meaning] at (35.750000, -1.300000) {shoot};
\node[Square] at (37.800000, -3.050000) {};
\node[Kanji] at (37.800000, -2.550000) {\textcolor[HTML]{14418e}{謝}};
\node[Onyomi] at (37.850000, -2.950000) {シャ};
\node[Kunyomi] at (37.750000, -2.950000) {あやま.る};
\node[Meaning] at (37.800000, -1.300000) {apologize};
\node[Square] at (39.850000, -3.050000) {};
\node[Kanji] at (39.850000, -2.550000) {\textcolor[HTML]{14469c}{老}};
\node[Onyomi] at (39.900000, -2.950000) {ロウ};
\node[Meaning] at (39.850000, -1.300000) {elderly};
\node[Square] at (41.900000, -3.050000) {};
\node[Kanji] at (41.900000, -2.550000) {\textcolor[HTML]{1461e3}{考}};
\node[Onyomi] at (41.950000, -2.950000) {コウ};
\node[Kunyomi] at (41.850000, -2.950000) {かんが};
\node[Meaning] at (41.900000, -1.300000) {think};
\node[Square] at (43.950000, -3.050000) {};
\node[Kanji] at (43.950000, -2.550000) {\textcolor[HTML]{0e254c}{孝}};
\node[Onyomi] at (44.000000, -2.950000) {コウ};
\node[Meaning] at (43.950000, -1.300000) {filial piety};
\node[Square] at (46.000000, -3.050000) {};
\node[Kanji] at (46.000000, -2.550000) {\textcolor[HTML]{145cd5}{教}};
\node[Onyomi] at (46.050000, -2.950000) {キョウ};
\node[Kunyomi] at (45.950000, -2.950000) {おし.える};
\node[Meaning] at (46.000000, -1.300000) {teach};
\node[Square] at (48.050000, -3.050000) {};
\node[Kanji] at (48.050000, -2.550000) {\textcolor[HTML]{123673}{拷}};
\node[Onyomi] at (48.100000, -2.950000) {ゴウ};
\node[Meaning] at (48.050000, -1.300000) {torture};
\node[Square] at (50.100000, -3.050000) {};
\node[Kanji] at (50.100000, -2.550000) {\textcolor[HTML]{145cd5}{者}};
\node[Onyomi] at (50.150000, -2.950000) {シャ};
\node[Kunyomi] at (50.050000, -2.950000) {もの};
\node[Meaning] at (50.100000, -1.300000) {someone};
\node[Square] at (52.150000, -3.050000) {};
\node[Kanji] at (52.150000, -2.550000) {\textcolor[HTML]{133c80}{煮}};
\node[Onyomi] at (52.200000, -2.950000) {シャ};
\node[Kunyomi] at (52.100000, -2.950000) {に};
\node[Meaning] at (52.150000, -1.300000) {boil};
\node[Square] at (54.200000, -3.050000) {};
\node[Kanji] at (54.200000, -2.550000) {\textcolor[HTML]{133c80}{著}};
\node[Onyomi] at (54.250000, -2.950000) {チョ};
\node[Kunyomi] at (54.150000, -2.950000) {いちじる.しい};
\node[Meaning] at (54.200000, -1.300000) {author};
\node[Square] at (56.250000, -3.050000) {};
\node[Kanji] at (56.250000, -2.550000) {\textcolor[HTML]{133c80}{署}};
\node[Onyomi] at (56.300000, -2.950000) {ショ};
\node[Meaning] at (56.250000, -1.300000) {govt. office};
\node[Meaning] at (-58.500000, -2.500000) {69.83\%};
\node[Square] at (-56.500000, -5.100000) {};
\node[Kanji] at (-56.500000, -4.600000) {\textcolor[HTML]{14418e}{暑}};
\node[Onyomi] at (-56.450000, -5.000000) {ショ};
\node[Kunyomi] at (-56.550000, -5.000000) {あつ};
\node[Meaning] at (-56.500000, -3.350000) {hot};
\node[Square] at (-54.450000, -5.100000) {};
\node[Kanji] at (-54.450000, -4.600000) {\textcolor[HTML]{14418e}{諸}};
\node[Onyomi] at (-54.400000, -5.000000) {ショ};
\node[Kunyomi] at (-54.500000, -5.000000) {もろ};
\node[Meaning] at (-54.450000, -3.350000) {various};
\node[Square] at (-52.400000, -5.100000) {};
\node[Kanji] at (-52.400000, -4.600000) {\textcolor[HTML]{133c80}{賭}};
\node[Kunyomi] at (-52.450000, -5.000000) {か};
\node[Meaning] at (-52.400000, -3.350000) {gamble};
\node[Square] at (-50.350000, -5.100000) {};
\node[Kanji] at (-50.350000, -4.600000) {\textcolor[HTML]{102b59}{峡}};
\node[Onyomi] at (-50.300000, -5.000000) {キョウ};
\node[Kunyomi] at (-50.400000, -5.000000) {はざま};
\node[Meaning] at (-50.350000, -3.350000) {ravine};
\node[Square] at (-48.300000, -5.100000) {};
\node[Kanji] at (-48.300000, -4.600000) {\textcolor[HTML]{14418e}{狭}};
\node[Onyomi] at (-48.250000, -5.000000) {キョウ};
\node[Kunyomi] at (-48.350000, -5.000000) {せま};
\node[Meaning] at (-48.300000, -3.350000) {narrow};
\node[Square] at (-46.250000, -5.100000) {};
\node[Kanji] at (-46.250000, -4.600000) {\textcolor[HTML]{14418e}{挟}};
\node[Onyomi] at (-46.200000, -5.000000) {キョウ};
\node[Kunyomi] at (-46.300000, -5.000000) {はさ};
\node[Meaning] at (-46.250000, -3.350000) {between};
\node[Square] at (-44.200000, -5.100000) {};
\node[Kanji] at (-44.200000, -4.600000) {\textcolor[HTML]{1551b8}{追}};
\node[Onyomi] at (-44.150000, -5.000000) {ツイ};
\node[Kunyomi] at (-44.250000, -5.000000) {お};
\node[Meaning] at (-44.200000, -3.350000) {follow};
\node[Square] at (-42.150000, -5.100000) {};
\node[Kanji] at (-42.150000, -4.600000) {\textcolor[HTML]{14469c}{師}};
\node[Onyomi] at (-42.100000, -5.000000) {シ};
\node[Meaning] at (-42.150000, -3.350000) {teacher};
\node[Square] at (-40.100000, -5.100000) {};
\node[Kanji] at (-40.100000, -4.600000) {\textcolor[HTML]{0e254c}{帥}};
\node[Onyomi] at (-40.050000, -5.000000) {スイ};
\node[Meaning] at (-40.100000, -3.350000) {commander};
\node[Square] at (-38.050000, -5.100000) {};
\node[Kanji] at (-38.050000, -4.600000) {\textcolor[HTML]{14469c}{官}};
\node[Onyomi] at (-38.000000, -5.000000) {カン};
\node[Meaning] at (-38.050000, -3.350000) {government};
\node[Square] at (-36.000000, -5.100000) {};
\node[Kanji] at (-36.000000, -4.600000) {\textcolor[HTML]{0e254c}{棺}};
\node[Onyomi] at (-35.950000, -5.000000) {カン};
\node[Meaning] at (-36.000000, -3.350000) {coffin};
\node[Square] at (-33.950000, -5.100000) {};
\node[Kanji] at (-33.950000, -4.600000) {\textcolor[HTML]{14469c}{管}};
\node[Onyomi] at (-33.900000, -5.000000) {カン};
\node[Kunyomi] at (-34.000000, -5.000000) {くだ};
\node[Meaning] at (-33.950000, -3.350000) {pipe};
\node[Square] at (-31.900000, -5.100000) {};
\node[Kanji] at (-31.900000, -4.600000) {\textcolor[HTML]{1557c6}{父}};
\node[Onyomi] at (-31.850000, -5.000000) {フ};
\node[Kunyomi] at (-31.950000, -5.000000) {ちち};
\node[Meaning] at (-31.900000, -3.350000) {father};
\node[Square] at (-29.850000, -5.100000) {};
\node[Kanji] at (-29.850000, -4.600000) {\textcolor[HTML]{1551b8}{交}};
\node[Onyomi] at (-29.800000, -5.000000) {コウ};
\node[Kunyomi] at (-29.900000, -5.000000) {まじ};
\node[Meaning] at (-29.850000, -3.350000) {mix};
\node[Square] at (-27.800000, -5.100000) {};
\node[Kanji] at (-27.800000, -4.600000) {\textcolor[HTML]{14469c}{効}};
\node[Onyomi] at (-27.750000, -5.000000) {コウ};
\node[Kunyomi] at (-27.850000, -5.000000) {き.く};
\node[Meaning] at (-27.800000, -3.350000) {effective};
\node[Square] at (-25.750000, -5.100000) {};
\node[Kanji] at (-25.750000, -4.600000) {\textcolor[HTML]{0e254c}{較}};
\node[Onyomi] at (-25.700000, -5.000000) {カク};
\node[Meaning] at (-25.750000, -3.350000) {contrast};
\node[Square] at (-23.700000, -5.100000) {};
\node[Kanji] at (-23.700000, -4.600000) {\textcolor[HTML]{145cd5}{校}};
\node[Onyomi] at (-23.650000, -5.000000) {コウ};
\node[Meaning] at (-23.700000, -3.350000) {school};
\node[Square] at (-21.650000, -5.100000) {};
\node[Kanji] at (-21.650000, -4.600000) {\textcolor[HTML]{145cd5}{足}};
\node[Onyomi] at (-21.600000, -5.000000) {ソク};
\node[Kunyomi] at (-21.700000, -5.000000) {あし};
\node[Meaning] at (-21.650000, -3.350000) {foot};
\node[Square] at (-19.600000, -5.100000) {};
\node[Kanji] at (-19.600000, -4.600000) {\textcolor[HTML]{133c80}{促}};
\node[Onyomi] at (-19.550000, -5.000000) {ソク};
\node[Kunyomi] at (-19.650000, -5.000000) {うなが.す};
\node[Meaning] at (-19.600000, -3.350000) {urge};
\node[Square] at (-17.550000, -5.100000) {};
\node[Kanji] at (-17.550000, -4.600000) {\textcolor[HTML]{133c80}{距}};
\node[Onyomi] at (-17.500000, -5.000000) {キョ};
\node[Meaning] at (-17.550000, -3.350000) {distance};
\node[Square] at (-15.500000, -5.100000) {};
\node[Kanji] at (-15.500000, -4.600000) {\textcolor[HTML]{154caa}{路}};
\node[Onyomi] at (-15.450000, -5.000000) {ロ};
\node[Kunyomi] at (-15.550000, -5.000000) {じ};
\node[Meaning] at (-15.500000, -3.350000) {road};
\node[Square] at (-13.450000, -5.100000) {};
\node[Kanji] at (-13.450000, -4.600000) {\textcolor[HTML]{113066}{露}};
\node[Onyomi] at (-13.400000, -5.000000) {ロ};
\node[Kunyomi] at (-13.500000, -5.000000) {つゆ};
\node[Meaning] at (-13.450000, -3.350000) {expose};
\node[Square] at (-11.400000, -5.100000) {};
\node[Kanji] at (-11.400000, -4.600000) {\textcolor[HTML]{14418e}{跳}};
\node[Onyomi] at (-11.350000, -5.000000) {チョウ};
\node[Kunyomi] at (-11.450000, -5.000000) {と.ぶ};
\node[Meaning] at (-11.400000, -3.350000) {hop};
\node[Square] at (-9.350000, -5.100000) {};
\node[Kanji] at (-9.350000, -4.600000) {\textcolor[HTML]{14418e}{躍}};
\node[Onyomi] at (-9.300000, -5.000000) {ヤク};
\node[Kunyomi] at (-9.400000, -5.000000) {おど.る};
\node[Meaning] at (-9.350000, -3.350000) {leap};
\node[Square] at (-7.300000, -5.100000) {};
\node[Kanji] at (-7.300000, -4.600000) {\textcolor[HTML]{0e254c}{践}};
\node[Onyomi] at (-7.250000, -5.000000) {セン};
\node[Kunyomi] at (-7.350000, -5.000000) {ふ};
\node[Meaning] at (-7.300000, -3.350000) {practice};
\node[Square] at (-5.250000, -5.100000) {};
\node[Kanji] at (-5.250000, -4.600000) {\textcolor[HTML]{14469c}{踏}};
\node[Onyomi] at (-5.200000, -5.000000) {トウ};
\node[Kunyomi] at (-5.300000, -5.000000) {ふ.む};
\node[Meaning] at (-5.250000, -3.350000) {step};
\node[Square] at (-3.200000, -5.100000) {};
\node[Kanji] at (-3.200000, -4.600000) {\textcolor[HTML]{14469c}{骨}};
\node[Onyomi] at (-3.150000, -5.000000) {コツ};
\node[Kunyomi] at (-3.250000, -5.000000) {ほね};
\node[Meaning] at (-3.200000, -3.350000) {bone};
\node[Square] at (-1.150000, -5.100000) {};
\node[Kanji] at (-1.150000, -4.600000) {\textcolor[HTML]{154caa}{滑}};
\node[Onyomi] at (-1.100000, -5.000000) {カツ};
\node[Kunyomi] at (-1.200000, -5.000000) {すべ.る};
\node[Meaning] at (-1.150000, -3.350000) {slippery};
\node[Square] at (0.900000, -5.100000) {};
\node[Kanji] at (0.900000, -4.600000) {\textcolor[HTML]{0e254c}{髄}};
\node[Onyomi] at (0.950000, -5.000000) {ズイ};
\node[Meaning] at (0.900000, -3.350000) {marrow};
\node[Square] at (2.950000, -5.100000) {};
\node[Kanji] at (2.950000, -4.600000) {\textcolor[HTML]{0e254c}{禍}};
\node[Onyomi] at (3.000000, -5.000000) {カ};
\node[Kunyomi] at (2.900000, -5.000000) {わざわい};
\node[Meaning] at (2.950000, -3.350000) {evil};
\node[Square] at (5.000000, -5.100000) {};
\node[Kanji] at (5.000000, -4.600000) {\textcolor[HTML]{133c80}{渦}};
\node[Onyomi] at (5.050000, -5.000000) {カ};
\node[Kunyomi] at (4.950000, -5.000000) {うず};
\node[Meaning] at (5.000000, -3.350000) {whirlpool};
\node[Square] at (7.050000, -5.100000) {};
\node[Kanji] at (7.050000, -4.600000) {\textcolor[HTML]{1557c6}{過}};
\node[Onyomi] at (7.100000, -5.000000) {カ};
\node[Kunyomi] at (7.000000, -5.000000) {す.ぎ};
\node[Meaning] at (7.050000, -3.350000) {surpass};
\node[Square] at (9.100000, -5.100000) {};
\node[Kanji] at (9.100000, -4.600000) {\textcolor[HTML]{14418e}{阪}};
\node[Onyomi] at (9.150000, -5.000000) {ハン};
\node[Kunyomi] at (9.050000, -5.000000) {さか};
\node[Meaning] at (9.100000, -3.350000) {heights};
\node[Square] at (11.150000, -5.100000) {};
\node[Kanji] at (11.150000, -4.600000) {\textcolor[HTML]{102b59}{阿}};
\node[Onyomi] at (11.200000, -5.000000) {ア};
\node[Kunyomi] at (11.100000, -5.000000) {おもね};
\node[Meaning] at (11.150000, -3.350000) {flatter};
\node[Square] at (13.200000, -5.100000) {};
\node[Kanji] at (13.200000, -4.600000) {\textcolor[HTML]{1551b8}{際}};
\node[Onyomi] at (13.250000, -5.000000) {サイ};
\node[Kunyomi] at (13.150000, -5.000000) {きわ};
\node[Meaning] at (13.200000, -3.350000) {occasion};
\node[Square] at (15.250000, -5.100000) {};
\node[Kanji] at (15.250000, -4.600000) {\textcolor[HTML]{14418e}{障}};
\node[Onyomi] at (15.300000, -5.000000) {ショウ};
\node[Kunyomi] at (15.200000, -5.000000) {さわ.る};
\node[Meaning] at (15.250000, -3.350000) {hinder};
\node[Square] at (17.300000, -5.100000) {};
\node[Kanji] at (17.300000, -4.600000) {\textcolor[HTML]{0e254c}{随}};
\node[Onyomi] at (17.350000, -5.000000) {ズイ};
\node[Kunyomi] at (17.250000, -5.000000) {したが.う};
\node[Meaning] at (17.300000, -3.350000) {all};
\node[Square] at (19.350000, -5.100000) {};
\node[Kanji] at (19.350000, -4.600000) {\textcolor[HTML]{0e254c}{陪}};
\node[Onyomi] at (19.400000, -5.000000) {バイ};
\node[Meaning] at (19.350000, -3.350000) {accompany};
\node[Square] at (21.400000, -5.100000) {};
\node[Kanji] at (21.400000, -4.600000) {\textcolor[HTML]{154caa}{陽}};
\node[Onyomi] at (21.450000, -5.000000) {ヨウ};
\node[Kunyomi] at (21.350000, -5.000000) {ひ};
\node[Meaning] at (21.400000, -3.350000) {sunshine};
\node[Square] at (23.450000, -5.100000) {};
\node[Kanji] at (23.450000, -4.600000) {\textcolor[HTML]{0e254c}{陳}};
\node[Onyomi] at (23.500000, -5.000000) {チン};
\node[Kunyomi] at (23.400000, -5.000000) {ひ.ねる};
\node[Meaning] at (23.450000, -3.350000) {exhibit};
\node[Square] at (25.500000, -5.100000) {};
\node[Kanji] at (25.500000, -4.600000) {\textcolor[HTML]{1551b8}{防}};
\node[Onyomi] at (25.550000, -5.000000) {ボウ};
\node[Kunyomi] at (25.450000, -5.000000) {ふせ.ぐ};
\node[Meaning] at (25.500000, -3.350000) {prevent};
\node[Square] at (27.550000, -5.100000) {};
\node[Kanji] at (27.550000, -4.600000) {\textcolor[HTML]{0e254c}{附}};
\node[Onyomi] at (27.600000, -5.000000) {フ};
\node[Meaning] at (27.550000, -3.350000) {affixed};
\node[Square] at (29.600000, -5.100000) {};
\node[Kanji] at (29.600000, -4.600000) {\textcolor[HTML]{154caa}{院}};
\node[Onyomi] at (29.650000, -5.000000) {イン};
\node[Meaning] at (29.600000, -3.350000) {institution};
\node[Square] at (31.650000, -5.100000) {};
\node[Kanji] at (31.650000, -4.600000) {\textcolor[HTML]{123673}{陣}};
\node[Onyomi] at (31.700000, -5.000000) {ジン};
\node[Meaning] at (31.650000, -3.350000) {army base};
\node[Square] at (33.700000, -5.100000) {};
\node[Kanji] at (33.700000, -4.600000) {\textcolor[HTML]{14469c}{隊}};
\node[Onyomi] at (33.750000, -5.000000) {タイ};
\node[Meaning] at (33.700000, -3.350000) {squad};
\node[Square] at (35.750000, -5.100000) {};
\node[Kanji] at (35.750000, -4.600000) {\textcolor[HTML]{113066}{墜}};
\node[Onyomi] at (35.800000, -5.000000) {ツイ};
\node[Meaning] at (35.750000, -3.350000) {crash};
\node[Square] at (37.800000, -5.100000) {};
\node[Kanji] at (37.800000, -4.600000) {\textcolor[HTML]{1551b8}{降}};
\node[Onyomi] at (37.850000, -5.000000) {コウ};
\node[Kunyomi] at (37.750000, -5.000000) {お.りる};
\node[Meaning] at (37.800000, -3.350000) {descend};
\node[Square] at (39.850000, -5.100000) {};
\node[Kanji] at (39.850000, -4.600000) {\textcolor[HTML]{1557c6}{階}};
\node[Onyomi] at (39.900000, -5.000000) {カイ};
\node[Meaning] at (39.850000, -3.350000) {floor};
\node[Square] at (41.900000, -5.100000) {};
\node[Kanji] at (41.900000, -4.600000) {\textcolor[HTML]{0e254c}{陛}};
\node[Onyomi] at (41.950000, -5.000000) {ヘイ};
\node[Meaning] at (41.900000, -3.350000) {highness};
\node[Square] at (43.950000, -5.100000) {};
\node[Kanji] at (43.950000, -4.600000) {\textcolor[HTML]{154caa}{隣}};
\node[Onyomi] at (44.000000, -5.000000) {リン};
\node[Kunyomi] at (43.900000, -5.000000) {となり};
\node[Meaning] at (43.950000, -3.350000) {neighbor};
\node[Square] at (46.000000, -5.100000) {};
\node[Kanji] at (46.000000, -4.600000) {\textcolor[HTML]{113066}{隔}};
\node[Onyomi] at (46.050000, -5.000000) {カク};
\node[Kunyomi] at (45.950000, -5.000000) {へだ.*};
\node[Meaning] at (46.000000, -3.350000) {isolate};
\node[Square] at (48.050000, -5.100000) {};
\node[Kanji] at (48.050000, -4.600000) {\textcolor[HTML]{1551b8}{隠}};
\node[Onyomi] at (48.100000, -5.000000) {イン};
\node[Kunyomi] at (48.000000, -5.000000) {かく.*};
\node[Meaning] at (48.050000, -3.350000) {hide};
\node[Square] at (50.100000, -5.100000) {};
\node[Kanji] at (50.100000, -4.600000) {\textcolor[HTML]{0e254c}{堕}};
\node[Onyomi] at (50.150000, -5.000000) {ダ};
\node[Kunyomi] at (50.050000, -5.000000) {お};
\node[Meaning] at (50.100000, -3.350000) {degenerate};
\node[Square] at (52.150000, -5.100000) {};
\node[Kanji] at (52.150000, -4.600000) {\textcolor[HTML]{123673}{陥}};
\node[Onyomi] at (52.200000, -5.000000) {カン};
\node[Kunyomi] at (52.100000, -5.000000) {おちい};
\node[Meaning] at (52.150000, -3.350000) {cave in};
\node[Square] at (54.200000, -5.100000) {};
\node[Kanji] at (54.200000, -4.600000) {\textcolor[HTML]{1551b8}{穴}};
\node[Onyomi] at (54.250000, -5.000000) {ケツ};
\node[Kunyomi] at (54.150000, -5.000000) {あな};
\node[Meaning] at (54.200000, -3.350000) {hole};
\node[Square] at (56.250000, -5.100000) {};
\node[Kanji] at (56.250000, -4.600000) {\textcolor[HTML]{145cd5}{空}};
\node[Onyomi] at (56.300000, -5.000000) {クウ};
\node[Kunyomi] at (56.200000, -5.000000) {そら};
\node[Meaning] at (56.250000, -3.350000) {sky};
\node[Meaning] at (-58.500000, -4.550000) {71.72\%};
\node[Square] at (-56.500000, -7.150000) {};
\node[Kanji] at (-56.500000, -6.650000) {\textcolor[HTML]{113066}{控}};
\node[Onyomi] at (-56.450000, -7.050000) {コウ};
\node[Kunyomi] at (-56.550000, -7.050000) {ひか};
\node[Meaning] at (-56.500000, -5.400000) {abstain};
\node[Square] at (-54.450000, -7.150000) {};
\node[Kanji] at (-54.450000, -6.650000) {\textcolor[HTML]{1557c6}{突}};
\node[Onyomi] at (-54.400000, -7.050000) {トツ};
\node[Kunyomi] at (-54.500000, -7.050000) {つ.く};
\node[Meaning] at (-54.450000, -5.400000) {stab};
\node[Square] at (-52.400000, -7.150000) {};
\node[Kanji] at (-52.400000, -6.650000) {\textcolor[HTML]{1551b8}{究}};
\node[Onyomi] at (-52.350000, -7.050000) {キュウ};
\node[Kunyomi] at (-52.450000, -7.050000) {きわ.める};
\node[Meaning] at (-52.400000, -5.400000) {research};
\node[Square] at (-50.350000, -7.150000) {};
\node[Kanji] at (-50.350000, -6.650000) {\textcolor[HTML]{102b59}{窒}};
\node[Onyomi] at (-50.300000, -7.050000) {チツ};
\node[Meaning] at (-50.350000, -5.400000) {suffocate};
\node[Square] at (-48.300000, -7.150000) {};
\node[Kanji] at (-48.300000, -6.650000) {\textcolor[HTML]{0e254c}{窃}};
\node[Onyomi] at (-48.250000, -7.050000) {セツ};
\node[Kunyomi] at (-48.350000, -7.050000) {ぬす};
\node[Meaning] at (-48.300000, -5.400000) {steal};
\node[Square] at (-46.250000, -7.150000) {};
\node[Kanji] at (-46.250000, -6.650000) {\textcolor[HTML]{0e254c}{搾}};
\node[Onyomi] at (-46.200000, -7.050000) {サク};
\node[Kunyomi] at (-46.300000, -7.050000) {しぼ};
\node[Meaning] at (-46.250000, -5.400000) {squeeze};
\node[Square] at (-44.200000, -7.150000) {};
\node[Kanji] at (-44.200000, -6.650000) {\textcolor[HTML]{0e254c}{窯}};
\node[Onyomi] at (-44.150000, -7.050000) {ヨウ};
\node[Kunyomi] at (-44.250000, -7.050000) {かま};
\node[Meaning] at (-44.200000, -5.400000) {kiln};
\node[Square] at (-42.150000, -7.150000) {};
\node[Kanji] at (-42.150000, -6.650000) {\textcolor[HTML]{113066}{窮}};
\node[Onyomi] at (-42.100000, -7.050000) {キュウ};
\node[Kunyomi] at (-42.200000, -7.050000) {きわ};
\node[Meaning] at (-42.150000, -5.400000) {destitute};
\node[Square] at (-40.100000, -7.150000) {};
\node[Kanji] at (-40.100000, -6.650000) {\textcolor[HTML]{1551b8}{探}};
\node[Onyomi] at (-40.050000, -7.050000) {タン};
\node[Kunyomi] at (-40.150000, -7.050000) {さが.す};
\node[Meaning] at (-40.100000, -5.400000) {look for};
\node[Square] at (-38.050000, -7.150000) {};
\node[Kanji] at (-38.050000, -6.650000) {\textcolor[HTML]{1551b8}{深}};
\node[Onyomi] at (-38.000000, -7.050000) {シン};
\node[Kunyomi] at (-38.100000, -7.050000) {ふか.*};
\node[Meaning] at (-38.050000, -5.400000) {deep};
\node[Square] at (-36.000000, -7.150000) {};
\node[Kanji] at (-36.000000, -6.650000) {\textcolor[HTML]{123673}{丘}};
\node[Onyomi] at (-35.950000, -7.050000) {キュウ};
\node[Kunyomi] at (-36.050000, -7.050000) {おか};
\node[Meaning] at (-36.000000, -5.400000) {hill};
\node[Square] at (-33.950000, -7.150000) {};
\node[Kanji] at (-33.950000, -6.650000) {\textcolor[HTML]{102b59}{岳}};
\node[Onyomi] at (-33.900000, -7.050000) {ガク};
\node[Kunyomi] at (-34.000000, -7.050000) {たけ};
\node[Meaning] at (-33.950000, -5.400000) {peak};
\node[Square] at (-31.900000, -7.150000) {};
\node[Kanji] at (-31.900000, -6.650000) {\textcolor[HTML]{14418e}{兵}};
\node[Onyomi] at (-31.850000, -7.050000) {ヘイ};
\node[Meaning] at (-31.900000, -5.400000) {soldier};
\node[Square] at (-29.850000, -7.150000) {};
\node[Kanji] at (-29.850000, -6.650000) {\textcolor[HTML]{14418e}{浜}};
\node[Onyomi] at (-29.800000, -7.050000) {ヒン};
\node[Kunyomi] at (-29.900000, -7.050000) {はま};
\node[Meaning] at (-29.850000, -5.400000) {beach};
\node[Square] at (-27.800000, -7.150000) {};
\node[Kanji] at (-27.800000, -6.650000) {\textcolor[HTML]{14418e}{糸}};
\node[Onyomi] at (-27.750000, -7.050000) {シ};
\node[Kunyomi] at (-27.850000, -7.050000) {いと};
\node[Meaning] at (-27.800000, -5.400000) {thread};
\node[Square] at (-25.750000, -7.150000) {};
\node[Kanji] at (-25.750000, -6.650000) {\textcolor[HTML]{14418e}{織}};
\node[Onyomi] at (-25.700000, -7.050000) {シキ};
\node[Kunyomi] at (-25.800000, -7.050000) {お.る};
\node[Meaning] at (-25.750000, -5.400000) {weave};
\node[Square] at (-23.700000, -7.150000) {};
\node[Kanji] at (-23.700000, -6.650000) {\textcolor[HTML]{113066}{繕}};
\node[Onyomi] at (-23.650000, -7.050000) {ゼン};
\node[Kunyomi] at (-23.750000, -7.050000) {つくろ-う};
\node[Meaning] at (-23.700000, -5.400000) {darning};
\node[Square] at (-21.650000, -7.150000) {};
\node[Kanji] at (-21.650000, -6.650000) {\textcolor[HTML]{14418e}{縮}};
\node[Onyomi] at (-21.600000, -7.050000) {シュク};
\node[Kunyomi] at (-21.700000, -7.050000) {ちぢ};
\node[Meaning] at (-21.650000, -5.400000) {shrink};
\node[Square] at (-19.600000, -7.150000) {};
\node[Kanji] at (-19.600000, -6.650000) {\textcolor[HTML]{123673}{繁}};
\node[Onyomi] at (-19.550000, -7.050000) {ハン};
\node[Kunyomi] at (-19.650000, -7.050000) {しげ.*};
\node[Meaning] at (-19.600000, -5.400000) {overgrown};
\node[Square] at (-17.550000, -7.150000) {};
\node[Kanji] at (-17.550000, -6.650000) {\textcolor[HTML]{154caa}{縦}};
\node[Onyomi] at (-17.500000, -7.050000) {ジュウ};
\node[Kunyomi] at (-17.600000, -7.050000) {たて};
\node[Meaning] at (-17.550000, -5.400000) {vertical};
\node[Square] at (-15.500000, -7.150000) {};
\node[Kanji] at (-15.500000, -6.650000) {\textcolor[HTML]{1557c6}{線}};
\node[Onyomi] at (-15.450000, -7.050000) {セン};
\node[Meaning] at (-15.500000, -5.400000) {line};
\node[Square] at (-13.450000, -7.150000) {};
\node[Kanji] at (-13.450000, -6.650000) {\textcolor[HTML]{14469c}{締}};
\node[Onyomi] at (-13.400000, -7.050000) {テイ};
\node[Kunyomi] at (-13.500000, -7.050000) {し};
\node[Meaning] at (-13.450000, -5.400000) {tighten};
\node[Square] at (-11.400000, -7.150000) {};
\node[Kanji] at (-11.400000, -6.650000) {\textcolor[HTML]{0e254c}{維}};
\node[Onyomi] at (-11.350000, -7.050000) {イ};
\node[Meaning] at (-11.400000, -5.400000) {maintain};
\node[Square] at (-9.350000, -7.150000) {};
\node[Kanji] at (-9.350000, -6.650000) {\textcolor[HTML]{113066}{羅}};
\node[Onyomi] at (-9.300000, -7.050000) {ラ};
\node[Kunyomi] at (-9.400000, -7.050000) {うすもの        };
\node[Meaning] at (-9.350000, -5.400000) {spread out};
\node[Square] at (-7.300000, -7.150000) {};
\node[Kanji] at (-7.300000, -6.650000) {\textcolor[HTML]{1551b8}{練}};
\node[Onyomi] at (-7.250000, -7.050000) {レン};
\node[Kunyomi] at (-7.350000, -7.050000) {ね};
\node[Meaning] at (-7.300000, -5.400000) {practice};
\node[Square] at (-5.250000, -7.150000) {};
\node[Kanji] at (-5.250000, -6.650000) {\textcolor[HTML]{1557c6}{緒}};
\node[Onyomi] at (-5.200000, -7.050000) {ショ};
\node[Meaning] at (-5.250000, -5.400000) {together};
\node[Square] at (-3.200000, -7.150000) {};
\node[Kanji] at (-3.200000, -6.650000) {\textcolor[HTML]{145cd5}{続}};
\node[Onyomi] at (-3.150000, -7.050000) {ゾク};
\node[Kunyomi] at (-3.250000, -7.050000) {つづ.く};
\node[Meaning] at (-3.200000, -5.400000) {continue};
\node[Square] at (-1.150000, -7.150000) {};
\node[Kanji] at (-1.150000, -6.650000) {\textcolor[HTML]{154caa}{絵}};
\node[Onyomi] at (-1.100000, -7.050000) {エ};
\node[Meaning] at (-1.150000, -5.400000) {drawing};
\node[Square] at (0.900000, -7.150000) {};
\node[Kanji] at (0.900000, -6.650000) {\textcolor[HTML]{14469c}{統}};
\node[Onyomi] at (0.950000, -7.050000) {トウ};
\node[Kunyomi] at (0.850000, -7.050000) {す.べる};
\node[Meaning] at (0.900000, -5.400000) {unite};
\node[Square] at (2.950000, -7.150000) {};
\node[Kanji] at (2.950000, -6.650000) {\textcolor[HTML]{133c80}{絞}};
\node[Onyomi] at (3.000000, -7.050000) {コウ};
\node[Kunyomi] at (2.900000, -7.050000) {し};
\node[Meaning] at (2.950000, -5.400000) {strangle};
\node[Square] at (5.000000, -7.150000) {};
\node[Kanji] at (5.000000, -6.650000) {\textcolor[HTML]{14418e}{給}};
\node[Onyomi] at (5.050000, -7.050000) {キュウ};
\node[Kunyomi] at (4.950000, -7.050000) {たま.う};
\node[Meaning] at (5.000000, -5.400000) {salary};
\node[Square] at (7.050000, -7.150000) {};
\node[Kanji] at (7.050000, -6.650000) {\textcolor[HTML]{14469c}{絡}};
\node[Onyomi] at (7.100000, -7.050000) {ラク};
\node[Kunyomi] at (7.000000, -7.050000) {から.む};
\node[Meaning] at (7.050000, -5.400000) {entangle};
\node[Square] at (9.100000, -7.150000) {};
\node[Kanji] at (9.100000, -6.650000) {\textcolor[HTML]{1557c6}{結}};
\node[Onyomi] at (9.150000, -7.050000) {ケツ};
\node[Kunyomi] at (9.050000, -7.050000) {むす.ぶ};
\node[Meaning] at (9.100000, -5.400000) {bind};
\node[Square] at (11.150000, -7.150000) {};
\node[Kanji] at (11.150000, -6.650000) {\textcolor[HTML]{1557c6}{終}};
\node[Onyomi] at (11.200000, -7.050000) {シュウ};
\node[Kunyomi] at (11.100000, -7.050000) {おわ.り};
\node[Meaning] at (11.150000, -5.400000) {end};
\node[Square] at (13.200000, -7.150000) {};
\node[Kanji] at (13.200000, -6.650000) {\textcolor[HTML]{14418e}{級}};
\node[Onyomi] at (13.250000, -7.050000) {キュウ};
\node[Meaning] at (13.200000, -5.400000) {rank};
\node[Square] at (15.250000, -7.150000) {};
\node[Kanji] at (15.250000, -6.650000) {\textcolor[HTML]{133c80}{紀}};
\node[Onyomi] at (15.300000, -7.050000) {キ};
\node[Meaning] at (15.250000, -5.400000) {account};
\node[Square] at (17.300000, -7.150000) {};
\node[Kanji] at (17.300000, -6.650000) {\textcolor[HTML]{14469c}{紅}};
\node[Onyomi] at (17.350000, -7.050000) {コウ};
\node[Kunyomi] at (17.250000, -7.050000) {べに};
\node[Meaning] at (17.300000, -5.400000) {deep red};
\node[Square] at (19.350000, -7.150000) {};
\node[Kanji] at (19.350000, -6.650000) {\textcolor[HTML]{14418e}{納}};
\node[Onyomi] at (19.400000, -7.050000) {ノウ};
\node[Kunyomi] at (19.300000, -7.050000) {おさ};
\node[Meaning] at (19.350000, -5.400000) {supply};
\node[Square] at (21.400000, -7.150000) {};
\node[Kanji] at (21.400000, -6.650000) {\textcolor[HTML]{0e254c}{紡}};
\node[Onyomi] at (21.450000, -7.050000) {ボウ};
\node[Kunyomi] at (21.350000, -7.050000) {つむ};
\node[Meaning] at (21.400000, -5.400000) {spinning};
\node[Square] at (23.450000, -7.150000) {};
\node[Kanji] at (23.450000, -6.650000) {\textcolor[HTML]{123673}{紛}};
\node[Onyomi] at (23.500000, -7.050000) {フン};
\node[Kunyomi] at (23.400000, -7.050000) {まぎ.*};
\node[Meaning] at (23.450000, -5.400000) {distract};
\node[Square] at (25.500000, -7.150000) {};
\node[Kanji] at (25.500000, -6.650000) {\textcolor[HTML]{14469c}{紹}};
\node[Onyomi] at (25.550000, -7.050000) {ショウ};
\node[Meaning] at (25.500000, -5.400000) {introduce};
\node[Square] at (27.550000, -7.150000) {};
\node[Kanji] at (27.550000, -6.650000) {\textcolor[HTML]{1551b8}{経}};
\node[Onyomi] at (27.600000, -7.050000) {ケイ};
\node[Kunyomi] at (27.500000, -7.050000) {た.つ};
\node[Meaning] at (27.550000, -5.400000) {passage of time};
\node[Square] at (29.600000, -7.150000) {};
\node[Kanji] at (29.600000, -6.650000) {\textcolor[HTML]{102b59}{紳}};
\node[Onyomi] at (29.650000, -7.050000) {シン};
\node[Meaning] at (29.600000, -5.400000) {gentleman};
\node[Square] at (31.650000, -7.150000) {};
\node[Kanji] at (31.650000, -6.650000) {\textcolor[HTML]{145cd5}{約}};
\node[Onyomi] at (31.700000, -7.050000) {ヤク};
\node[Meaning] at (31.650000, -5.400000) {promise};
\node[Square] at (33.700000, -7.150000) {};
\node[Kanji] at (33.700000, -6.650000) {\textcolor[HTML]{1551b8}{細}};
\node[Onyomi] at (33.750000, -7.050000) {サイ};
\node[Kunyomi] at (33.650000, -7.050000) {ほそ};
\node[Meaning] at (33.700000, -5.400000) {thin};
\node[Square] at (35.750000, -7.150000) {};
\node[Kanji] at (35.750000, -6.650000) {\textcolor[HTML]{0e254c}{累}};
\node[Onyomi] at (35.800000, -7.050000) {ルイ};
\node[Meaning] at (35.750000, -5.400000) {accumulate};
\node[Square] at (37.800000, -7.150000) {};
\node[Kanji] at (37.800000, -6.650000) {\textcolor[HTML]{133c80}{索}};
\node[Onyomi] at (37.850000, -7.050000) {サク};
\node[Meaning] at (37.800000, -5.400000) {search};
\node[Square] at (39.850000, -7.150000) {};
\node[Kanji] at (39.850000, -6.650000) {\textcolor[HTML]{14469c}{総}};
\node[Onyomi] at (39.900000, -7.050000) {ソウ};
\node[Meaning] at (39.850000, -5.400000) {whole};
\node[Square] at (41.900000, -7.150000) {};
\node[Kanji] at (41.900000, -6.650000) {\textcolor[HTML]{113066}{綿}};
\node[Onyomi] at (41.950000, -7.050000) {メン};
\node[Kunyomi] at (41.850000, -7.050000) {わた};
\node[Meaning] at (41.900000, -5.400000) {cotton};
\node[Square] at (43.950000, -7.150000) {};
\node[Kanji] at (43.950000, -6.650000) {\textcolor[HTML]{123673}{絹}};
\node[Onyomi] at (44.000000, -7.050000) {ケン};
\node[Kunyomi] at (43.900000, -7.050000) {きぬ};
\node[Meaning] at (43.950000, -5.400000) {silk};
\node[Square] at (46.000000, -7.150000) {};
\node[Kanji] at (46.000000, -6.650000) {\textcolor[HTML]{14469c}{繰}};
\node[Onyomi] at (46.050000, -7.050000) {ソウ};
\node[Kunyomi] at (45.950000, -7.050000) {く};
\node[Meaning] at (46.000000, -5.400000) {spin};
\node[Square] at (48.050000, -7.150000) {};
\node[Kanji] at (48.050000, -6.650000) {\textcolor[HTML]{14418e}{継}};
\node[Onyomi] at (48.100000, -7.050000) {ケイ};
\node[Kunyomi] at (48.000000, -7.050000) {つ.ぐ};
\node[Meaning] at (48.050000, -5.400000) {inherit};
\node[Square] at (50.100000, -7.150000) {};
\node[Kanji] at (50.100000, -6.650000) {\textcolor[HTML]{14469c}{緑}};
\node[Onyomi] at (50.150000, -7.050000) {リョク};
\node[Kunyomi] at (50.050000, -7.050000) {みどり};
\node[Meaning] at (50.100000, -5.400000) {green};
\node[Square] at (52.150000, -7.150000) {};
\node[Kanji] at (52.150000, -6.650000) {\textcolor[HTML]{14418e}{縁}};
\node[Onyomi] at (52.200000, -7.050000) {エン};
\node[Kunyomi] at (52.100000, -7.050000) {ふち};
\node[Meaning] at (52.150000, -5.400000) {edge};
\node[Square] at (54.200000, -7.150000) {};
\node[Kanji] at (54.200000, -6.650000) {\textcolor[HTML]{133c80}{網}};
\node[Onyomi] at (54.250000, -7.050000) {モウ};
\node[Kunyomi] at (54.150000, -7.050000) {あみ};
\node[Meaning] at (54.200000, -5.400000) {netting};
\node[Square] at (56.250000, -7.150000) {};
\node[Kanji] at (56.250000, -6.650000) {\textcolor[HTML]{14469c}{緊}};
\node[Onyomi] at (56.300000, -7.050000) {キン};
\node[Meaning] at (56.250000, -5.400000) {tense};
\node[Meaning] at (-58.500000, -6.600000) {73.58\%};
\node[Square] at (-56.500000, -9.200000) {};
\node[Kanji] at (-56.500000, -8.700000) {\textcolor[HTML]{14418e}{紫}};
\node[Onyomi] at (-56.450000, -9.100000) {シ};
\node[Kunyomi] at (-56.550000, -9.100000) {むらさき};
\node[Meaning] at (-56.500000, -7.450000) {purple};
\node[Square] at (-54.450000, -9.200000) {};
\node[Kanji] at (-54.450000, -8.700000) {\textcolor[HTML]{14418e}{縛}};
\node[Onyomi] at (-54.400000, -9.100000) {バク};
\node[Kunyomi] at (-54.500000, -9.100000) {しば};
\node[Meaning] at (-54.450000, -7.450000) {bind};
\node[Square] at (-52.400000, -9.200000) {};
\node[Kanji] at (-52.400000, -8.700000) {\textcolor[HTML]{154caa}{縄}};
\node[Onyomi] at (-52.350000, -9.100000) {ジョウ};
\node[Kunyomi] at (-52.450000, -9.100000) {なわ};
\node[Meaning] at (-52.400000, -7.450000) {rope};
\node[Square] at (-50.350000, -9.200000) {};
\node[Kanji] at (-50.350000, -8.700000) {\textcolor[HTML]{133c80}{幼}};
\node[Onyomi] at (-50.300000, -9.100000) {ヨウ};
\node[Kunyomi] at (-50.400000, -9.100000) {おさな.い};
\node[Meaning] at (-50.350000, -7.450000) {infancy};
\node[Square] at (-48.300000, -9.200000) {};
\node[Kanji] at (-48.300000, -8.700000) {\textcolor[HTML]{1461e3}{後}};
\node[Onyomi] at (-48.250000, -9.100000) {ゴ};
\node[Kunyomi] at (-48.350000, -9.100000) {うし.ろ};
\node[Meaning] at (-48.300000, -7.450000) {behind};
\node[Square] at (-46.250000, -9.200000) {};
\node[Kanji] at (-46.250000, -8.700000) {\textcolor[HTML]{123673}{幽}};
\node[Onyomi] at (-46.200000, -9.100000) {ユウ};
\node[Meaning] at (-46.250000, -7.450000) {secluded};
\node[Square] at (-44.200000, -9.200000) {};
\node[Kanji] at (-44.200000, -8.700000) {\textcolor[HTML]{113066}{幾}};
\node[Onyomi] at (-44.150000, -9.100000) {キ};
\node[Kunyomi] at (-44.250000, -9.100000) {いく};
\node[Meaning] at (-44.200000, -7.450000) {how many};
\node[Square] at (-42.150000, -9.200000) {};
\node[Kanji] at (-42.150000, -8.700000) {\textcolor[HTML]{1557c6}{機}};
\node[Onyomi] at (-42.100000, -9.100000) {キ};
\node[Kunyomi] at (-42.200000, -9.100000) {はた};
\node[Meaning] at (-42.150000, -7.450000) {machine};
\node[Square] at (-40.100000, -9.200000) {};
\node[Kanji] at (-40.100000, -8.700000) {\textcolor[HTML]{154caa}{玄}};
\node[Onyomi] at (-40.050000, -9.100000) {ゲン};
\node[Kunyomi] at (-40.150000, -9.100000) {くろ};
\node[Meaning] at (-40.100000, -7.450000) {mysterious};
\node[Square] at (-38.050000, -9.200000) {};
\node[Kanji] at (-38.050000, -8.700000) {\textcolor[HTML]{0e254c}{畜}};
\node[Onyomi] at (-38.000000, -9.100000) {チク};
\node[Meaning] at (-38.050000, -7.450000) {livestock};
\node[Square] at (-36.000000, -9.200000) {};
\node[Kanji] at (-36.000000, -8.700000) {\textcolor[HTML]{102b59}{蓄}};
\node[Onyomi] at (-35.950000, -9.100000) {チク};
\node[Kunyomi] at (-36.050000, -9.100000) {たくわ.える};
\node[Meaning] at (-36.000000, -7.450000) {amass};
\node[Square] at (-33.950000, -9.200000) {};
\node[Kanji] at (-33.950000, -8.700000) {\textcolor[HTML]{0e254c}{弦}};
\node[Onyomi] at (-33.900000, -9.100000) {ゲン};
\node[Kunyomi] at (-34.000000, -9.100000) {つる};
\node[Meaning] at (-33.950000, -7.450000) {chord};
\node[Square] at (-31.900000, -9.200000) {};
\node[Kanji] at (-31.900000, -8.700000) {\textcolor[HTML]{0e254c}{擁}};
\node[Onyomi] at (-31.850000, -9.100000) {ヨウ};
\node[Meaning] at (-31.900000, -7.450000) {embrace};
\node[Square] at (-29.850000, -9.200000) {};
\node[Kanji] at (-29.850000, -8.700000) {\textcolor[HTML]{102b59}{滋}};
\node[Onyomi] at (-29.800000, -9.100000) {ジ};
\node[Meaning] at (-29.850000, -7.450000) {nourishing};
\node[Square] at (-27.800000, -9.200000) {};
\node[Kanji] at (-27.800000, -8.700000) {\textcolor[HTML]{102b59}{慈}};
\node[Onyomi] at (-27.750000, -9.100000) {ジ};
\node[Kunyomi] at (-27.850000, -9.100000) {いつく};
\node[Meaning] at (-27.800000, -7.450000) {mercy};
\node[Square] at (-25.750000, -9.200000) {};
\node[Kanji] at (-25.750000, -8.700000) {\textcolor[HTML]{102b59}{磁}};
\node[Onyomi] at (-25.700000, -9.100000) {ジ};
\node[Meaning] at (-25.750000, -7.450000) {magnet};
\node[Square] at (-23.700000, -9.200000) {};
\node[Kanji] at (-23.700000, -8.700000) {\textcolor[HTML]{14418e}{系}};
\node[Onyomi] at (-23.650000, -9.100000) {ケイ};
\node[Meaning] at (-23.700000, -7.450000) {lineage};
\node[Square] at (-21.650000, -9.200000) {};
\node[Kanji] at (-21.650000, -8.700000) {\textcolor[HTML]{154caa}{係}};
\node[Onyomi] at (-21.600000, -9.100000) {ケイ};
\node[Kunyomi] at (-21.700000, -9.100000) {かか};
\node[Meaning] at (-21.650000, -7.450000) {connection};
\node[Square] at (-19.600000, -9.200000) {};
\node[Kanji] at (-19.600000, -8.700000) {\textcolor[HTML]{123673}{孫}};
\node[Onyomi] at (-19.550000, -9.100000) {ソン};
\node[Kunyomi] at (-19.650000, -9.100000) {まご};
\node[Meaning] at (-19.600000, -7.450000) {grandchild};
\node[Square] at (-17.550000, -9.200000) {};
\node[Kanji] at (-17.550000, -8.700000) {\textcolor[HTML]{133c80}{懸}};
\node[Onyomi] at (-17.500000, -9.100000) {ケン};
\node[Kunyomi] at (-17.600000, -9.100000) {か.*};
\node[Meaning] at (-17.550000, -7.450000) {suspend};
\node[Square] at (-15.500000, -9.200000) {};
\node[Kanji] at (-15.500000, -8.700000) {\textcolor[HTML]{123673}{却}};
\node[Onyomi] at (-15.450000, -9.100000) {キャク};
\node[Kunyomi] at (-15.550000, -9.100000) {かえって};
\node[Meaning] at (-15.500000, -7.450000) {contrary};
\node[Square] at (-13.450000, -9.200000) {};
\node[Kanji] at (-13.450000, -8.700000) {\textcolor[HTML]{14469c}{脚}};
\node[Onyomi] at (-13.400000, -9.100000) {キャク};
\node[Kunyomi] at (-13.500000, -9.100000) {あし};
\node[Meaning] at (-13.450000, -7.450000) {leg};
\node[Square] at (-11.400000, -9.200000) {};
\node[Kanji] at (-11.400000, -8.700000) {\textcolor[HTML]{0e254c}{卸}};
\node[Onyomi] at (-11.350000, -9.100000) {シャ};
\node[Kunyomi] at (-11.450000, -9.100000) {おろし};
\node[Meaning] at (-11.400000, -7.450000) {wholesale};
\node[Square] at (-9.350000, -9.200000) {};
\node[Kanji] at (-9.350000, -8.700000) {\textcolor[HTML]{14418e}{御}};
\node[Onyomi] at (-9.300000, -9.100000) {ゴ};
\node[Kunyomi] at (-9.400000, -9.100000) {お};
\node[Meaning] at (-9.350000, -7.450000) {honorable};
\node[Square] at (-7.300000, -9.200000) {};
\node[Kanji] at (-7.300000, -8.700000) {\textcolor[HTML]{154caa}{服}};
\node[Onyomi] at (-7.250000, -9.100000) {フク};
\node[Meaning] at (-7.300000, -7.450000) {clothes};
\node[Square] at (-5.250000, -9.200000) {};
\node[Kanji] at (-5.250000, -8.700000) {\textcolor[HTML]{1551b8}{命}};
\node[Onyomi] at (-5.200000, -9.100000) {メイ};
\node[Kunyomi] at (-5.300000, -9.100000) {いのち};
\node[Meaning] at (-5.250000, -7.450000) {fate};
\node[Square] at (-3.200000, -9.200000) {};
\node[Kanji] at (-3.200000, -8.700000) {\textcolor[HTML]{14469c}{令}};
\node[Onyomi] at (-3.150000, -9.100000) {レイ};
\node[Meaning] at (-3.200000, -7.450000) {orders};
\node[Square] at (-1.150000, -9.200000) {};
\node[Kanji] at (-1.150000, -8.700000) {\textcolor[HTML]{102b59}{零}};
\node[Onyomi] at (-1.100000, -9.100000) {レイ};
\node[Kunyomi] at (-1.200000, -9.100000) {こぼ.す};
\node[Meaning] at (-1.150000, -7.450000) {zero};
\node[Square] at (0.900000, -9.200000) {};
\node[Kanji] at (0.900000, -8.700000) {\textcolor[HTML]{14418e}{齢}};
\node[Onyomi] at (0.950000, -9.100000) {レイ};
\node[Kunyomi] at (0.850000, -9.100000) {よわい};
\node[Meaning] at (0.900000, -7.450000) {age};
\node[Square] at (2.950000, -9.200000) {};
\node[Kanji] at (2.950000, -8.700000) {\textcolor[HTML]{1551b8}{冷}};
\node[Onyomi] at (3.000000, -9.100000) {レイ};
\node[Kunyomi] at (2.900000, -9.100000) {つめ.たい};
\node[Meaning] at (2.950000, -7.450000) {cool};
\node[Square] at (5.000000, -9.200000) {};
\node[Kanji] at (5.000000, -8.700000) {\textcolor[HTML]{14469c}{領}};
\node[Onyomi] at (5.050000, -9.100000) {リョウ};
\node[Meaning] at (5.000000, -7.450000) {territory};
\node[Square] at (7.050000, -9.200000) {};
\node[Kanji] at (7.050000, -8.700000) {\textcolor[HTML]{154caa}{鈴}};
\node[Onyomi] at (7.100000, -9.100000) {リン};
\node[Kunyomi] at (7.000000, -9.100000) {すず};
\node[Meaning] at (7.050000, -7.450000) {buzzer};
\node[Square] at (9.100000, -9.200000) {};
\node[Kanji] at (9.100000, -8.700000) {\textcolor[HTML]{14418e}{勇}};
\node[Onyomi] at (9.150000, -9.100000) {ユウ};
\node[Kunyomi] at (9.050000, -9.100000) {いさ};
\node[Meaning] at (9.100000, -7.450000) {courage};
\node[Square] at (11.150000, -9.200000) {};
\node[Kanji] at (11.150000, -8.700000) {\textcolor[HTML]{145cd5}{通}};
\node[Onyomi] at (11.200000, -9.100000) {ツウ};
\node[Kunyomi] at (11.100000, -9.100000) {とお.る};
\node[Meaning] at (11.150000, -7.450000) {pass through};
\node[Square] at (13.200000, -9.200000) {};
\node[Kanji] at (13.200000, -8.700000) {\textcolor[HTML]{14469c}{踊}};
\node[Onyomi] at (13.250000, -9.100000) {ヨウ};
\node[Kunyomi] at (13.150000, -9.100000) {おど};
\node[Meaning] at (13.200000, -7.450000) {dance};
\node[Square] at (15.250000, -9.200000) {};
\node[Kanji] at (15.250000, -8.700000) {\textcolor[HTML]{154caa}{疑}};
\node[Onyomi] at (15.300000, -9.100000) {ギ};
\node[Kunyomi] at (15.200000, -9.100000) {うたが.う};
\node[Meaning] at (15.250000, -7.450000) {doubt};
\node[Square] at (17.300000, -9.200000) {};
\node[Kanji] at (17.300000, -8.700000) {\textcolor[HTML]{0e254c}{擬}};
\node[Onyomi] at (17.350000, -9.100000) {ギ};
\node[Kunyomi] at (17.250000, -9.100000) {まが};
\node[Meaning] at (17.300000, -7.450000) {imitate};
\node[Square] at (19.350000, -9.200000) {};
\node[Kanji] at (19.350000, -8.700000) {\textcolor[HTML]{133c80}{凝}};
\node[Onyomi] at (19.400000, -9.100000) {ギョウ};
\node[Kunyomi] at (19.300000, -9.100000) {こ};
\node[Meaning] at (19.350000, -7.450000) {congeal};
\node[Square] at (21.400000, -9.200000) {};
\node[Kanji] at (21.400000, -8.700000) {\textcolor[HTML]{123673}{範}};
\node[Onyomi] at (21.450000, -9.100000) {ハン};
\node[Meaning] at (21.400000, -7.450000) {example};
\node[Square] at (23.450000, -9.200000) {};
\node[Kanji] at (23.450000, -8.700000) {\textcolor[HTML]{14469c}{犯}};
\node[Onyomi] at (23.500000, -9.100000) {ハン};
\node[Kunyomi] at (23.400000, -9.100000) {おか.す};
\node[Meaning] at (23.450000, -7.450000) {crime};
\node[Square] at (25.500000, -9.200000) {};
\node[Kanji] at (25.500000, -8.700000) {\textcolor[HTML]{123673}{厄}};
\node[Onyomi] at (25.550000, -9.100000) {ヤク};
\node[Meaning] at (25.500000, -7.450000) {unlucky};
\node[Square] at (27.550000, -9.200000) {};
\node[Kanji] at (27.550000, -8.700000) {\textcolor[HTML]{1551b8}{危}};
\node[Onyomi] at (27.600000, -9.100000) {キ};
\node[Kunyomi] at (27.500000, -9.100000) {あぶ.ない};
\node[Meaning] at (27.550000, -7.450000) {dangerous};
\node[Square] at (29.600000, -9.200000) {};
\node[Kanji] at (29.600000, -8.700000) {\textcolor[HTML]{123673}{宛}};
\node[Kunyomi] at (29.550000, -9.100000) {あ-てる};
\node[Meaning] at (29.600000, -7.450000) {allocate};
\node[Square] at (31.650000, -9.200000) {};
\node[Kanji] at (31.650000, -8.700000) {\textcolor[HTML]{1551b8}{腕}};
\node[Onyomi] at (31.700000, -9.100000) {ワン};
\node[Kunyomi] at (31.600000, -9.100000) {うで};
\node[Meaning] at (31.650000, -7.450000) {arm};
\node[Square] at (33.700000, -9.200000) {};
\node[Kanji] at (33.700000, -8.700000) {\textcolor[HTML]{0e254c}{怨}};
\node[Onyomi] at (33.750000, -9.100000) {エン};
\node[Meaning] at (33.700000, -7.450000) {grudge};
\node[Square] at (35.750000, -9.200000) {};
\node[Kanji] at (35.750000, -8.700000) {\textcolor[HTML]{133c80}{柳}};
\node[Onyomi] at (35.800000, -9.100000) {リュウ};
\node[Kunyomi] at (35.700000, -9.100000) {やなぎ};
\node[Meaning] at (35.750000, -7.450000) {willow};
\node[Square] at (37.800000, -9.200000) {};
\node[Kanji] at (37.800000, -8.700000) {\textcolor[HTML]{14469c}{卵}};
\node[Onyomi] at (37.850000, -9.100000) {ラン};
\node[Kunyomi] at (37.750000, -9.100000) {たまご};
\node[Meaning] at (37.800000, -7.450000) {egg};
\node[Square] at (39.850000, -9.200000) {};
\node[Kanji] at (39.850000, -8.700000) {\textcolor[HTML]{154caa}{留}};
\node[Onyomi] at (39.900000, -9.100000) {ル};
\node[Kunyomi] at (39.800000, -9.100000) {と};
\node[Meaning] at (39.850000, -7.450000) {detain};
\node[Square] at (41.900000, -9.200000) {};
\node[Kanji] at (41.900000, -8.700000) {\textcolor[HTML]{123673}{貿}};
\node[Onyomi] at (41.950000, -9.100000) {ボウ};
\node[Meaning] at (41.900000, -7.450000) {trade};
\node[Square] at (43.950000, -9.200000) {};
\node[Kanji] at (43.950000, -8.700000) {\textcolor[HTML]{154caa}{印}};
\node[Onyomi] at (44.000000, -9.100000) {イン};
\node[Kunyomi] at (43.900000, -9.100000) {しるし};
\node[Meaning] at (43.950000, -7.450000) {seal};
\node[Square] at (46.000000, -9.200000) {};
\node[Kanji] at (46.000000, -8.700000) {\textcolor[HTML]{1551b8}{興}};
\node[Onyomi] at (46.050000, -9.100000) {キョウ};
\node[Meaning] at (46.000000, -7.450000) {interest};
\node[Square] at (48.050000, -9.200000) {};
\node[Kanji] at (48.050000, -8.700000) {\textcolor[HTML]{14469c}{酒}};
\node[Onyomi] at (48.100000, -9.100000) {シュ};
\node[Kunyomi] at (48.000000, -9.100000) {さけ};
\node[Meaning] at (48.050000, -7.450000) {alcohol};
\node[Square] at (50.100000, -9.200000) {};
\node[Kanji] at (50.100000, -8.700000) {\textcolor[HTML]{0e254c}{酌}};
\node[Onyomi] at (50.150000, -9.100000) {シャク};
\node[Kunyomi] at (50.050000, -9.100000) {く};
\node[Meaning] at (50.100000, -7.450000) {serve};
\node[Square] at (52.150000, -9.200000) {};
\node[Kanji] at (52.150000, -8.700000) {\textcolor[HTML]{0e254c}{酵}};
\node[Onyomi] at (52.200000, -9.100000) {コウ};
\node[Meaning] at (52.150000, -7.450000) {ferment};
\node[Square] at (54.200000, -9.200000) {};
\node[Kanji] at (54.200000, -8.700000) {\textcolor[HTML]{123673}{酷}};
\node[Onyomi] at (54.250000, -9.100000) {コク};
\node[Kunyomi] at (54.150000, -9.100000) {ひど};
\node[Meaning] at (54.200000, -7.450000) {cruel};
\node[Square] at (56.250000, -9.200000) {};
\node[Kanji] at (56.250000, -8.700000) {\textcolor[HTML]{0e254c}{酬}};
\node[Onyomi] at (56.300000, -9.100000) {シュウ};
\node[Kunyomi] at (56.200000, -9.100000) {むく};
\node[Meaning] at (56.250000, -7.450000) {repay};
\node[Meaning] at (-58.500000, -8.650000) {75.22\%};
\node[Square] at (-56.500000, -11.250000) {};
\node[Kanji] at (-56.500000, -10.750000) {\textcolor[HTML]{0e254c}{酪}};
\node[Onyomi] at (-56.450000, -11.150000) {ラク};
\node[Meaning] at (-56.500000, -9.500000) {dairy};
\node[Square] at (-54.450000, -11.250000) {};
\node[Kanji] at (-54.450000, -10.750000) {\textcolor[HTML]{0e254c}{酢}};
\node[Kunyomi] at (-54.500000, -11.150000) {す};
\node[Meaning] at (-54.450000, -9.500000) {vinegar};
\node[Square] at (-52.400000, -11.250000) {};
\node[Kanji] at (-52.400000, -10.750000) {\textcolor[HTML]{133c80}{酔}};
\node[Onyomi] at (-52.350000, -11.150000) {スイ};
\node[Kunyomi] at (-52.450000, -11.150000) {よ.う};
\node[Meaning] at (-52.400000, -9.500000) {drunk};
\node[Square] at (-50.350000, -11.250000) {};
\node[Kanji] at (-50.350000, -10.750000) {\textcolor[HTML]{1557c6}{配}};
\node[Onyomi] at (-50.300000, -11.150000) {ハイ};
\node[Kunyomi] at (-50.400000, -11.150000) {くば.る};
\node[Meaning] at (-50.350000, -9.500000) {distribute};
\node[Square] at (-48.300000, -11.250000) {};
\node[Kanji] at (-48.300000, -10.750000) {\textcolor[HTML]{133c80}{酸}};
\node[Onyomi] at (-48.250000, -11.150000) {サン};
\node[Kunyomi] at (-48.350000, -11.150000) {す};
\node[Meaning] at (-48.300000, -9.500000) {acid};
\node[Square] at (-46.250000, -11.250000) {};
\node[Kanji] at (-46.250000, -10.750000) {\textcolor[HTML]{0e254c}{猶}};
\node[Onyomi] at (-46.200000, -11.150000) {ユウ};
\node[Kunyomi] at (-46.300000, -11.150000) {なお};
\node[Meaning] at (-46.250000, -9.500000) {still};
\node[Square] at (-44.200000, -11.250000) {};
\node[Kanji] at (-44.200000, -10.750000) {\textcolor[HTML]{133c80}{尊}};
\node[Onyomi] at (-44.150000, -11.150000) {ソン};
\node[Kunyomi] at (-44.250000, -11.150000) {とうと.い};
\node[Meaning] at (-44.200000, -9.500000) {revered};
\node[Square] at (-42.150000, -11.250000) {};
\node[Kanji] at (-42.150000, -10.750000) {\textcolor[HTML]{14418e}{豆}};
\node[Onyomi] at (-42.100000, -11.150000) {トウ};
\node[Kunyomi] at (-42.200000, -11.150000) {まめ};
\node[Meaning] at (-42.150000, -9.500000) {beans};
\node[Square] at (-40.100000, -11.250000) {};
\node[Kanji] at (-40.100000, -10.750000) {\textcolor[HTML]{145cd5}{頭}};
\node[Onyomi] at (-40.050000, -11.150000) {ズ};
\node[Kunyomi] at (-40.150000, -11.150000) {あたま};
\node[Meaning] at (-40.100000, -9.500000) {head};
\node[Square] at (-38.050000, -11.250000) {};
\node[Kanji] at (-38.050000, -10.750000) {\textcolor[HTML]{14469c}{短}};
\node[Onyomi] at (-38.000000, -11.150000) {タン};
\node[Kunyomi] at (-38.100000, -11.150000) {みじか.い};
\node[Meaning] at (-38.050000, -9.500000) {short};
\node[Square] at (-36.000000, -11.250000) {};
\node[Kanji] at (-36.000000, -10.750000) {\textcolor[HTML]{133c80}{豊}};
\node[Onyomi] at (-35.950000, -11.150000) {ホウ};
\node[Kunyomi] at (-36.050000, -11.150000) {ゆた.か};
\node[Meaning] at (-36.000000, -9.500000) {plentiful};
\node[Square] at (-33.950000, -11.250000) {};
\node[Kanji] at (-33.950000, -10.750000) {\textcolor[HTML]{133c80}{鼓}};
\node[Onyomi] at (-33.900000, -11.150000) {コ};
\node[Kunyomi] at (-34.000000, -11.150000) {つづみ};
\node[Meaning] at (-33.950000, -9.500000) {beat};
\node[Square] at (-31.900000, -11.250000) {};
\node[Kanji] at (-31.900000, -10.750000) {\textcolor[HTML]{154caa}{喜}};
\node[Onyomi] at (-31.850000, -11.150000) {キ};
\node[Kunyomi] at (-31.950000, -11.150000) {よろこ};
\node[Meaning] at (-31.900000, -9.500000) {rejoice};
\node[Square] at (-29.850000, -11.250000) {};
\node[Kanji] at (-29.850000, -10.750000) {\textcolor[HTML]{154caa}{樹}};
\node[Onyomi] at (-29.800000, -11.150000) {ジュ};
\node[Kunyomi] at (-29.900000, -11.150000) {き};
\node[Meaning] at (-29.850000, -9.500000) {wood};
\node[Square] at (-27.800000, -11.250000) {};
\node[Kanji] at (-27.800000, -10.750000) {\textcolor[HTML]{14469c}{皿}};
\node[Kunyomi] at (-27.850000, -11.150000) {さら};
\node[Meaning] at (-27.800000, -9.500000) {plate};
\node[Square] at (-25.750000, -11.250000) {};
\node[Kanji] at (-25.750000, -10.750000) {\textcolor[HTML]{1551b8}{血}};
\node[Onyomi] at (-25.700000, -11.150000) {ケツ};
\node[Kunyomi] at (-25.800000, -11.150000) {ち};
\node[Meaning] at (-25.750000, -9.500000) {blood};
\node[Square] at (-23.700000, -11.250000) {};
\node[Kanji] at (-23.700000, -10.750000) {\textcolor[HTML]{14418e}{盆}};
\node[Onyomi] at (-23.650000, -11.150000) {ボン};
\node[Meaning] at (-23.700000, -9.500000) {lantern festival};
\node[Square] at (-21.650000, -11.250000) {};
\node[Kanji] at (-21.650000, -10.750000) {\textcolor[HTML]{123673}{盟}};
\node[Onyomi] at (-21.600000, -11.150000) {メイ};
\node[Meaning] at (-21.650000, -9.500000) {alliance};
\node[Square] at (-19.600000, -11.250000) {};
\node[Kanji] at (-19.600000, -10.750000) {\textcolor[HTML]{154caa}{盗}};
\node[Onyomi] at (-19.550000, -11.150000) {トウ};
\node[Kunyomi] at (-19.650000, -11.150000) {ぬす.む};
\node[Meaning] at (-19.600000, -9.500000) {steal};
\node[Square] at (-17.550000, -11.250000) {};
\node[Kanji] at (-17.550000, -10.750000) {\textcolor[HTML]{154caa}{温}};
\node[Onyomi] at (-17.500000, -11.150000) {オン};
\node[Kunyomi] at (-17.600000, -11.150000) {あたた.*};
\node[Meaning] at (-17.550000, -9.500000) {warm};
\node[Square] at (-15.500000, -11.250000) {};
\node[Kanji] at (-15.500000, -10.750000) {\textcolor[HTML]{154caa}{監}};
\node[Onyomi] at (-15.450000, -11.150000) {カン};
\node[Meaning] at (-15.500000, -9.500000) {oversee};
\node[Square] at (-13.450000, -11.250000) {};
\node[Kanji] at (-13.450000, -10.750000) {\textcolor[HTML]{0e254c}{濫}};
\node[Onyomi] at (-13.400000, -11.150000) {ラン};
\node[Meaning] at (-13.450000, -9.500000) {excessive};
\node[Square] at (-11.400000, -11.250000) {};
\node[Kanji] at (-11.400000, -10.750000) {\textcolor[HTML]{0e254c}{鑑}};
\node[Onyomi] at (-11.350000, -11.150000) {カン};
\node[Meaning] at (-11.400000, -9.500000) {model};
\node[Square] at (-9.350000, -11.250000) {};
\node[Kanji] at (-9.350000, -10.750000) {\textcolor[HTML]{14418e}{猛}};
\node[Onyomi] at (-9.300000, -11.150000) {モウ};
\node[Meaning] at (-9.350000, -9.500000) {fierce};
\node[Square] at (-7.300000, -11.250000) {};
\node[Kanji] at (-7.300000, -10.750000) {\textcolor[HTML]{14418e}{盛}};
\node[Onyomi] at (-7.250000, -11.150000) {セイ};
\node[Kunyomi] at (-7.350000, -11.150000) {も.る};
\node[Meaning] at (-7.300000, -9.500000) {pile};
\node[Square] at (-5.250000, -11.250000) {};
\node[Kanji] at (-5.250000, -10.750000) {\textcolor[HTML]{133c80}{塩}};
\node[Onyomi] at (-5.200000, -11.150000) {エン};
\node[Kunyomi] at (-5.300000, -11.150000) {しお};
\node[Meaning] at (-5.250000, -9.500000) {salt};
\node[Square] at (-3.200000, -11.250000) {};
\node[Kanji] at (-3.200000, -10.750000) {\textcolor[HTML]{1551b8}{銀}};
\node[Onyomi] at (-3.150000, -11.150000) {ギン};
\node[Meaning] at (-3.200000, -9.500000) {silver};
\node[Square] at (-1.150000, -11.250000) {};
\node[Kanji] at (-1.150000, -10.750000) {\textcolor[HTML]{133c80}{恨}};
\node[Onyomi] at (-1.100000, -11.150000) {コン};
\node[Kunyomi] at (-1.200000, -11.150000) {うら.む};
\node[Meaning] at (-1.150000, -9.500000) {grudge};
\node[Square] at (0.900000, -11.250000) {};
\node[Kanji] at (0.900000, -10.750000) {\textcolor[HTML]{1551b8}{根}};
\node[Onyomi] at (0.950000, -11.150000) {コン};
\node[Kunyomi] at (0.850000, -11.150000) {ね};
\node[Meaning] at (0.900000, -9.500000) {root};
\node[Square] at (2.950000, -11.250000) {};
\node[Kanji] at (2.950000, -10.750000) {\textcolor[HTML]{133c80}{即}};
\node[Onyomi] at (3.000000, -11.150000) {ソク};
\node[Kunyomi] at (2.900000, -11.150000) {すなわ.ち};
\node[Meaning] at (2.950000, -9.500000) {instant};
\node[Square] at (5.000000, -11.250000) {};
\node[Kanji] at (5.000000, -10.750000) {\textcolor[HTML]{113066}{爵}};
\node[Onyomi] at (5.050000, -11.150000) {シャク};
\node[Meaning] at (5.000000, -9.500000) {baron};
\node[Square] at (7.050000, -11.250000) {};
\node[Kanji] at (7.050000, -10.750000) {\textcolor[HTML]{14469c}{節}};
\node[Onyomi] at (7.100000, -11.150000) {セツ};
\node[Kunyomi] at (7.000000, -11.150000) {ふし};
\node[Meaning] at (7.050000, -9.500000) {season};
\node[Square] at (9.100000, -11.250000) {};
\node[Kanji] at (9.100000, -10.750000) {\textcolor[HTML]{154caa}{退}};
\node[Onyomi] at (9.150000, -11.150000) {タイ};
\node[Kunyomi] at (9.050000, -11.150000) {しりぞ.く};
\node[Meaning] at (9.100000, -9.500000) {retreat};
\node[Square] at (11.150000, -11.250000) {};
\node[Kanji] at (11.150000, -10.750000) {\textcolor[HTML]{14418e}{限}};
\node[Onyomi] at (11.200000, -11.150000) {ゲン};
\node[Kunyomi] at (11.100000, -11.150000) {かぎ.る};
\node[Meaning] at (11.150000, -9.500000) {limit};
\node[Square] at (13.200000, -11.250000) {};
\node[Kanji] at (13.200000, -10.750000) {\textcolor[HTML]{14469c}{眼}};
\node[Onyomi] at (13.250000, -11.150000) {ガン};
\node[Kunyomi] at (13.150000, -11.150000) {め};
\node[Meaning] at (13.200000, -9.500000) {eyeball};
\node[Square] at (15.250000, -11.250000) {};
\node[Kanji] at (15.250000, -10.750000) {\textcolor[HTML]{154caa}{良}};
\node[Onyomi] at (15.300000, -11.150000) {リョウ};
\node[Kunyomi] at (15.200000, -11.150000) {よ};
\node[Meaning] at (15.250000, -9.500000) {good};
\node[Square] at (17.300000, -11.250000) {};
\node[Kanji] at (17.300000, -10.750000) {\textcolor[HTML]{123673}{朗}};
\node[Onyomi] at (17.350000, -11.150000) {ロウ};
\node[Kunyomi] at (17.250000, -11.150000) {ほが.らか};
\node[Meaning] at (17.300000, -9.500000) {bright};
\node[Square] at (19.350000, -11.250000) {};
\node[Kanji] at (19.350000, -10.750000) {\textcolor[HTML]{113066}{浪}};
\node[Onyomi] at (19.400000, -11.150000) {ロウ};
\node[Meaning] at (19.350000, -9.500000) {wander};
\node[Square] at (21.400000, -11.250000) {};
\node[Kanji] at (21.400000, -10.750000) {\textcolor[HTML]{14418e}{娘}};
\node[Kunyomi] at (21.350000, -11.150000) {むすめ};
\node[Meaning] at (21.400000, -9.500000) {daughter};
\node[Square] at (23.450000, -11.250000) {};
\node[Kanji] at (23.450000, -10.750000) {\textcolor[HTML]{1461e3}{食}};
\node[Onyomi] at (23.500000, -11.150000) {ショク};
\node[Kunyomi] at (23.400000, -11.150000) {た.べる};
\node[Meaning] at (23.450000, -9.500000) {eat};
\node[Square] at (25.500000, -11.250000) {};
\node[Kanji] at (25.500000, -10.750000) {\textcolor[HTML]{14418e}{飯}};
\node[Onyomi] at (25.550000, -11.150000) {ハン};
\node[Kunyomi] at (25.450000, -11.150000) {めし};
\node[Meaning] at (25.500000, -9.500000) {meal};
\node[Square] at (27.550000, -11.250000) {};
\node[Kanji] at (27.550000, -10.750000) {\textcolor[HTML]{1557c6}{飲}};
\node[Onyomi] at (27.600000, -11.150000) {イン};
\node[Kunyomi] at (27.500000, -11.150000) {の};
\node[Meaning] at (27.550000, -9.500000) {drink};
\node[Square] at (29.600000, -11.250000) {};
\node[Kanji] at (29.600000, -10.750000) {\textcolor[HTML]{113066}{飢}};
\node[Onyomi] at (29.650000, -11.150000) {キ};
\node[Kunyomi] at (29.550000, -11.150000) {う.える};
\node[Meaning] at (29.600000, -9.500000) {starve};
\node[Square] at (31.650000, -11.250000) {};
\node[Kanji] at (31.650000, -10.750000) {\textcolor[HTML]{0e254c}{餓}};
\node[Onyomi] at (31.700000, -11.150000) {ガ};
\node[Kunyomi] at (31.600000, -11.150000) {う.える};
\node[Meaning] at (31.650000, -9.500000) {starve};
\node[Square] at (33.700000, -11.250000) {};
\node[Kanji] at (33.700000, -10.750000) {\textcolor[HTML]{154caa}{飾}};
\node[Onyomi] at (33.750000, -11.150000) {ショク};
\node[Kunyomi] at (33.650000, -11.150000) {かざ.る};
\node[Meaning] at (33.700000, -9.500000) {decorate};
\node[Square] at (35.750000, -11.250000) {};
\node[Kanji] at (35.750000, -10.750000) {\textcolor[HTML]{1551b8}{館}};
\node[Onyomi] at (35.800000, -11.150000) {カン};
\node[Meaning] at (35.750000, -9.500000) {public building};
\node[Square] at (37.800000, -11.250000) {};
\node[Kanji] at (37.800000, -10.750000) {\textcolor[HTML]{133c80}{養}};
\node[Onyomi] at (37.850000, -11.150000) {ヨウ};
\node[Kunyomi] at (37.750000, -11.150000) {やしな.う};
\node[Meaning] at (37.800000, -9.500000) {foster};
\node[Square] at (39.850000, -11.250000) {};
\node[Kanji] at (39.850000, -10.750000) {\textcolor[HTML]{102b59}{飽}};
\node[Onyomi] at (39.900000, -11.150000) {ホウ};
\node[Kunyomi] at (39.800000, -11.150000) {あ};
\node[Meaning] at (39.850000, -9.500000) {bored};
\node[Square] at (41.900000, -11.250000) {};
\node[Kanji] at (41.900000, -10.750000) {\textcolor[HTML]{0e254c}{既}};
\node[Onyomi] at (41.950000, -11.150000) {キ};
\node[Kunyomi] at (41.850000, -11.150000) {すで};
\node[Meaning] at (41.900000, -9.500000) {previously};
\node[Square] at (43.950000, -11.250000) {};
\node[Kanji] at (43.950000, -10.750000) {\textcolor[HTML]{0e254c}{概}};
\node[Onyomi] at (44.000000, -11.150000) {ガイ};
\node[Kunyomi] at (43.900000, -11.150000) {おおむ.ね};
\node[Meaning] at (43.950000, -9.500000) {approximation};
\node[Square] at (46.000000, -11.250000) {};
\node[Kanji] at (46.000000, -10.750000) {\textcolor[HTML]{133c80}{慨}};
\node[Onyomi] at (46.050000, -11.150000) {ガイ};
\node[Meaning] at (46.000000, -9.500000) {sigh};
\node[Square] at (48.050000, -11.250000) {};
\node[Kanji] at (48.050000, -10.750000) {\textcolor[HTML]{1557c6}{平}};
\node[Onyomi] at (48.100000, -11.150000) {ヘイ};
\node[Kunyomi] at (48.000000, -11.150000) {たいら};
\node[Meaning] at (48.050000, -9.500000) {flat};
\node[Square] at (50.100000, -11.250000) {};
\node[Kanji] at (50.100000, -10.750000) {\textcolor[HTML]{1557c6}{呼}};
\node[Onyomi] at (50.150000, -11.150000) {コ};
\node[Kunyomi] at (50.050000, -11.150000) {よ};
\node[Meaning] at (50.100000, -9.500000) {call};
\node[Square] at (52.150000, -11.250000) {};
\node[Kanji] at (52.150000, -10.750000) {\textcolor[HTML]{0e254c}{坪}};
\node[Onyomi] at (52.200000, -11.150000) {ヘイ};
\node[Kunyomi] at (52.100000, -11.150000) {つぼ};
\node[Meaning] at (52.150000, -9.500000) {two mat area};
\node[Square] at (54.200000, -11.250000) {};
\node[Kanji] at (54.200000, -10.750000) {\textcolor[HTML]{133c80}{評}};
\node[Onyomi] at (54.250000, -11.150000) {ヒョウ};
\node[Meaning] at (54.200000, -9.500000) {evaluate};
\node[Square] at (56.250000, -11.250000) {};
\node[Kanji] at (56.250000, -10.750000) {\textcolor[HTML]{113066}{刈}};
\node[Kunyomi] at (56.200000, -11.150000) {か};
\node[Meaning] at (56.250000, -9.500000) {prune};
\node[Meaning] at (-58.500000, -10.700000) {76.95\%};
\node[Square] at (-56.500000, -13.300000) {};
\node[Kanji] at (-56.500000, -12.800000) {\textcolor[HTML]{14469c}{希}};
\node[Onyomi] at (-56.450000, -13.200000) {キ};
\node[Kunyomi] at (-56.550000, -13.200000) {まれ};
\node[Meaning] at (-56.500000, -11.550000) {wish};
\node[Square] at (-54.450000, -13.300000) {};
\node[Kanji] at (-54.450000, -12.800000) {\textcolor[HTML]{113066}{凶}};
\node[Onyomi] at (-54.400000, -13.200000) {キョウ};
\node[Meaning] at (-54.450000, -11.550000) {villain};
\node[Square] at (-52.400000, -13.300000) {};
\node[Kanji] at (-52.400000, -12.800000) {\textcolor[HTML]{154caa}{胸}};
\node[Onyomi] at (-52.350000, -13.200000) {キョウ};
\node[Kunyomi] at (-52.450000, -13.200000) {むね};
\node[Meaning] at (-52.400000, -11.550000) {chest};
\node[Square] at (-50.350000, -13.300000) {};
\node[Kanji] at (-50.350000, -12.800000) {\textcolor[HTML]{1557c6}{離}};
\node[Onyomi] at (-50.300000, -13.200000) {リ};
\node[Kunyomi] at (-50.400000, -13.200000) {はな.*};
\node[Meaning] at (-50.350000, -11.550000) {detach};
\node[Square] at (-48.300000, -13.300000) {};
\node[Kanji] at (-48.300000, -12.800000) {\textcolor[HTML]{1557c6}{殺}};
\node[Onyomi] at (-48.250000, -13.200000) {サツ};
\node[Kunyomi] at (-48.350000, -13.200000) {ころ.す};
\node[Meaning] at (-48.300000, -11.550000) {kill};
\node[Square] at (-46.250000, -13.300000) {};
\node[Kanji] at (-46.250000, -12.800000) {\textcolor[HTML]{14418e}{純}};
\node[Onyomi] at (-46.200000, -13.200000) {ジュン};
\node[Meaning] at (-46.250000, -11.550000) {pure};
\node[Square] at (-44.200000, -13.300000) {};
\node[Kanji] at (-44.200000, -12.800000) {\textcolor[HTML]{123673}{鈍}};
\node[Onyomi] at (-44.150000, -13.200000) {ドン};
\node[Kunyomi] at (-44.250000, -13.200000) {にぶ.い};
\node[Meaning] at (-44.200000, -11.550000) {dull};
\node[Square] at (-42.150000, -13.300000) {};
\node[Kanji] at (-42.150000, -12.800000) {\textcolor[HTML]{14469c}{辛}};
\node[Onyomi] at (-42.100000, -13.200000) {シン};
\node[Kunyomi] at (-42.200000, -13.200000) {から.い};
\node[Meaning] at (-42.150000, -11.550000) {spicy};
\node[Square] at (-40.100000, -13.300000) {};
\node[Kanji] at (-40.100000, -12.800000) {\textcolor[HTML]{14469c}{辞}};
\node[Onyomi] at (-40.050000, -13.200000) {ジ};
\node[Kunyomi] at (-40.150000, -13.200000) {や.める};
\node[Meaning] at (-40.100000, -11.550000) {quit};
\node[Square] at (-38.050000, -13.300000) {};
\node[Kanji] at (-38.050000, -12.800000) {\textcolor[HTML]{0e254c}{梓}};
\node[Onyomi] at (-38.000000, -13.200000) {シ};
\node[Kunyomi] at (-38.100000, -13.200000) {あずさ        };
\node[Meaning] at (-38.050000, -11.550000) {wood block};
\node[Square] at (-36.000000, -13.300000) {};
\node[Kanji] at (-36.000000, -12.800000) {\textcolor[HTML]{0e254c}{宰}};
\node[Onyomi] at (-35.950000, -13.200000) {サイ};
\node[Meaning] at (-36.000000, -11.550000) {manager};
\node[Square] at (-33.950000, -13.300000) {};
\node[Kanji] at (-33.950000, -12.800000) {\textcolor[HTML]{1551b8}{壁}};
\node[Onyomi] at (-33.900000, -13.200000) {ヘキ};
\node[Kunyomi] at (-34.000000, -13.200000) {かべ};
\node[Meaning] at (-33.950000, -11.550000) {wall};
\node[Square] at (-31.900000, -13.300000) {};
\node[Kanji] at (-31.900000, -12.800000) {\textcolor[HTML]{154caa}{避}};
\node[Onyomi] at (-31.850000, -13.200000) {ヒ};
\node[Kunyomi] at (-31.950000, -13.200000) {さ.ける};
\node[Meaning] at (-31.900000, -11.550000) {dodge};
\node[Square] at (-29.850000, -13.300000) {};
\node[Kanji] at (-29.850000, -12.800000) {\textcolor[HTML]{145cd5}{新}};
\node[Onyomi] at (-29.800000, -13.200000) {シン};
\node[Kunyomi] at (-29.900000, -13.200000) {あたら.しい};
\node[Meaning] at (-29.850000, -11.550000) {new};
\node[Square] at (-27.800000, -13.300000) {};
\node[Kanji] at (-27.800000, -12.800000) {\textcolor[HTML]{102b59}{薪}};
\node[Onyomi] at (-27.750000, -13.200000) {シン};
\node[Kunyomi] at (-27.850000, -13.200000) {たきぎ};
\node[Meaning] at (-27.800000, -11.550000) {fuel};
\node[Square] at (-25.750000, -13.300000) {};
\node[Kanji] at (-25.750000, -12.800000) {\textcolor[HTML]{1557c6}{親}};
\node[Onyomi] at (-25.700000, -13.200000) {シン};
\node[Kunyomi] at (-25.800000, -13.200000) {おや};
\node[Meaning] at (-25.750000, -11.550000) {parent};
\node[Square] at (-23.700000, -13.300000) {};
\node[Kanji] at (-23.700000, -12.800000) {\textcolor[HTML]{154caa}{幸}};
\node[Onyomi] at (-23.650000, -13.200000) {コウ};
\node[Kunyomi] at (-23.750000, -13.200000) {しあわ.せ};
\node[Meaning] at (-23.700000, -11.550000) {happiness};
\node[Square] at (-21.650000, -13.300000) {};
\node[Kanji] at (-21.650000, -12.800000) {\textcolor[HTML]{133c80}{執}};
\node[Onyomi] at (-21.600000, -13.200000) {シュウ};
\node[Kunyomi] at (-21.700000, -13.200000) {と.る};
\node[Meaning] at (-21.650000, -11.550000) {tenacious};
\node[Square] at (-19.600000, -13.300000) {};
\node[Kanji] at (-19.600000, -12.800000) {\textcolor[HTML]{1551b8}{報}};
\node[Onyomi] at (-19.550000, -13.200000) {ホウ};
\node[Kunyomi] at (-19.650000, -13.200000) {むく.いる};
\node[Meaning] at (-19.600000, -11.550000) {news};
\node[Square] at (-17.550000, -13.300000) {};
\node[Kanji] at (-17.550000, -12.800000) {\textcolor[HTML]{1557c6}{叫}};
\node[Onyomi] at (-17.500000, -13.200000) {キョウ};
\node[Kunyomi] at (-17.600000, -13.200000) {さけ.ぶ};
\node[Meaning] at (-17.550000, -11.550000) {shout};
\node[Square] at (-15.500000, -13.300000) {};
\node[Kanji] at (-15.500000, -12.800000) {\textcolor[HTML]{0e254c}{糾}};
\node[Onyomi] at (-15.450000, -13.200000) {キュウ};
\node[Meaning] at (-15.500000, -11.550000) {twist};
\node[Square] at (-13.450000, -13.300000) {};
\node[Kanji] at (-13.450000, -12.800000) {\textcolor[HTML]{14469c}{収}};
\node[Onyomi] at (-13.400000, -13.200000) {シュウ};
\node[Kunyomi] at (-13.500000, -13.200000) {おさ.める};
\node[Meaning] at (-13.450000, -11.550000) {obtain};
\node[Square] at (-11.400000, -13.300000) {};
\node[Kanji] at (-11.400000, -12.800000) {\textcolor[HTML]{113066}{卑}};
\node[Onyomi] at (-11.350000, -13.200000) {ヒ};
\node[Kunyomi] at (-11.450000, -13.200000) {いや};
\node[Meaning] at (-11.400000, -11.550000) {lowly};
\node[Square] at (-9.350000, -13.300000) {};
\node[Kanji] at (-9.350000, -12.800000) {\textcolor[HTML]{113066}{碑}};
\node[Onyomi] at (-9.300000, -13.200000) {ヒ};
\node[Kunyomi] at (-9.400000, -13.200000) {いしぶみ};
\node[Meaning] at (-9.350000, -11.550000) {tombstone};
\node[Square] at (-7.300000, -13.300000) {};
\node[Kanji] at (-7.300000, -12.800000) {\textcolor[HTML]{14469c}{陸}};
\node[Onyomi] at (-7.250000, -13.200000) {リク};
\node[Meaning] at (-7.300000, -11.550000) {land};
\node[Square] at (-5.250000, -13.300000) {};
\node[Kanji] at (-5.250000, -12.800000) {\textcolor[HTML]{0e254c}{睦}};
\node[Onyomi] at (-5.200000, -13.200000) {ボク};
\node[Kunyomi] at (-5.300000, -13.200000) {むつ};
\node[Meaning] at (-5.250000, -11.550000) {friendly};
\node[Square] at (-3.200000, -13.300000) {};
\node[Kanji] at (-3.200000, -12.800000) {\textcolor[HTML]{1551b8}{勢}};
\node[Onyomi] at (-3.150000, -13.200000) {セイ};
\node[Kunyomi] at (-3.250000, -13.200000) {いきお.い};
\node[Meaning] at (-3.200000, -11.550000) {force};
\node[Square] at (-1.150000, -13.300000) {};
\node[Kanji] at (-1.150000, -12.800000) {\textcolor[HTML]{1551b8}{熱}};
\node[Onyomi] at (-1.100000, -13.200000) {ネツ};
\node[Kunyomi] at (-1.200000, -13.200000) {あつ.い};
\node[Meaning] at (-1.150000, -11.550000) {heat};
\node[Square] at (0.900000, -13.300000) {};
\node[Kanji] at (0.900000, -12.800000) {\textcolor[HTML]{0e254c}{陵}};
\node[Onyomi] at (0.950000, -13.200000) {リョウ};
\node[Kunyomi] at (0.850000, -13.200000) {みささぎ};
\node[Meaning] at (0.900000, -11.550000) {mausoleum};
\node[Square] at (2.950000, -13.300000) {};
\node[Kanji] at (2.950000, -12.800000) {\textcolor[HTML]{123673}{核}};
\node[Onyomi] at (3.000000, -13.200000) {カク};
\node[Kunyomi] at (2.900000, -13.200000) {かく};
\node[Meaning] at (2.950000, -11.550000) {nucleus};
\node[Square] at (5.000000, -13.300000) {};
\node[Kanji] at (5.000000, -12.800000) {\textcolor[HTML]{154caa}{刻}};
\node[Onyomi] at (5.050000, -13.200000) {コク};
\node[Kunyomi] at (4.950000, -13.200000) {きざ.む};
\node[Meaning] at (5.000000, -11.550000) {carve};
\node[Square] at (7.050000, -13.300000) {};
\node[Kanji] at (7.050000, -12.800000) {\textcolor[HTML]{0e254c}{該}};
\node[Onyomi] at (7.100000, -13.200000) {ガイ};
\node[Meaning] at (7.050000, -11.550000) {the above};
\node[Square] at (9.100000, -13.300000) {};
\node[Kanji] at (9.100000, -12.800000) {\textcolor[HTML]{0e254c}{劾}};
\node[Onyomi] at (9.150000, -13.200000) {ガイ};
\node[Meaning] at (9.100000, -11.550000) {censure};
\node[Square] at (11.150000, -13.300000) {};
\node[Kanji] at (11.150000, -12.800000) {\textcolor[HTML]{123673}{述}};
\node[Onyomi] at (11.200000, -13.200000) {ジュツ};
\node[Kunyomi] at (11.100000, -13.200000) {の.べる};
\node[Meaning] at (11.150000, -11.550000) {mention};
\node[Square] at (13.200000, -13.300000) {};
\node[Kanji] at (13.200000, -12.800000) {\textcolor[HTML]{1557c6}{術}};
\node[Onyomi] at (13.250000, -13.200000) {ジュツ};
\node[Meaning] at (13.200000, -11.550000) {art};
\node[Square] at (15.250000, -13.300000) {};
\node[Kanji] at (15.250000, -12.800000) {\textcolor[HTML]{14418e}{寒}};
\node[Onyomi] at (15.300000, -13.200000) {カン};
\node[Kunyomi] at (15.200000, -13.200000) {さむ};
\node[Meaning] at (15.250000, -11.550000) {cold};
\node[Square] at (17.300000, -13.300000) {};
\node[Kanji] at (17.300000, -12.800000) {\textcolor[HTML]{0e254c}{醸}};
\node[Onyomi] at (17.350000, -13.200000) {ジョウ};
\node[Kunyomi] at (17.250000, -13.200000) {かも};
\node[Meaning] at (17.300000, -11.550000) {brew};
\node[Square] at (19.350000, -13.300000) {};
\node[Kanji] at (19.350000, -12.800000) {\textcolor[HTML]{113066}{譲}};
\node[Onyomi] at (19.400000, -13.200000) {ジョウ};
\node[Kunyomi] at (19.300000, -13.200000) {ゆず.る};
\node[Meaning] at (19.350000, -11.550000) {defer};
\node[Square] at (21.400000, -13.300000) {};
\node[Kanji] at (21.400000, -12.800000) {\textcolor[HTML]{0e254c}{壌}};
\node[Onyomi] at (21.450000, -13.200000) {ジョウ};
\node[Kunyomi] at (21.350000, -13.200000) {つち};
\node[Meaning] at (21.400000, -11.550000) {soil};
\node[Square] at (23.450000, -13.300000) {};
\node[Kanji] at (23.450000, -12.800000) {\textcolor[HTML]{123673}{嬢}};
\node[Onyomi] at (23.500000, -13.200000) {ジョウ};
\node[Kunyomi] at (23.400000, -13.200000) {むすめ};
\node[Meaning] at (23.450000, -11.550000) {miss};
\node[Square] at (25.500000, -13.300000) {};
\node[Kanji] at (25.500000, -12.800000) {\textcolor[HTML]{154caa}{毒}};
\node[Onyomi] at (25.550000, -13.200000) {ドク};
\node[Meaning] at (25.500000, -11.550000) {poison};
\node[Square] at (27.550000, -13.300000) {};
\node[Kanji] at (27.550000, -12.800000) {\textcolor[HTML]{154caa}{素}};
\node[Onyomi] at (27.600000, -13.200000) {ス};
\node[Meaning] at (27.550000, -11.550000) {element};
\node[Square] at (29.600000, -13.300000) {};
\node[Kanji] at (29.600000, -12.800000) {\textcolor[HTML]{123673}{麦}};
\node[Onyomi] at (29.650000, -13.200000) {バク};
\node[Kunyomi] at (29.550000, -13.200000) {むぎ};
\node[Meaning] at (29.600000, -11.550000) {wheat};
\node[Square] at (31.650000, -13.300000) {};
\node[Kanji] at (31.650000, -12.800000) {\textcolor[HTML]{154caa}{青}};
\node[Onyomi] at (31.700000, -13.200000) {ショウ};
\node[Kunyomi] at (31.600000, -13.200000) {あお};
\node[Meaning] at (31.650000, -11.550000) {blue};
\node[Square] at (33.700000, -13.300000) {};
\node[Kanji] at (33.700000, -12.800000) {\textcolor[HTML]{154caa}{精}};
\node[Onyomi] at (33.750000, -13.200000) {セイ};
\node[Meaning] at (33.700000, -11.550000) {spirit};
\node[Square] at (35.750000, -13.300000) {};
\node[Kanji] at (35.750000, -12.800000) {\textcolor[HTML]{133c80}{請}};
\node[Onyomi] at (35.800000, -13.200000) {セイ};
\node[Kunyomi] at (35.700000, -13.200000) {う.ける};
\node[Meaning] at (35.750000, -11.550000) {request};
\node[Square] at (37.800000, -13.300000) {};
\node[Kanji] at (37.800000, -12.800000) {\textcolor[HTML]{1557c6}{情}};
\node[Onyomi] at (37.850000, -13.200000) {ジョウ};
\node[Kunyomi] at (37.750000, -13.200000) {なさけ};
\node[Meaning] at (37.800000, -11.550000) {feeling};
\node[Square] at (39.850000, -13.300000) {};
\node[Kanji] at (39.850000, -12.800000) {\textcolor[HTML]{14418e}{晴}};
\node[Onyomi] at (39.900000, -13.200000) {セイ};
\node[Kunyomi] at (39.800000, -13.200000) {は};
\node[Meaning] at (39.850000, -11.550000) {clear up};
\node[Square] at (41.900000, -13.300000) {};
\node[Kanji] at (41.900000, -12.800000) {\textcolor[HTML]{133c80}{清}};
\node[Onyomi] at (41.950000, -13.200000) {セイ};
\node[Kunyomi] at (41.850000, -13.200000) {きよ.い};
\node[Meaning] at (41.900000, -11.550000) {pure};
\node[Square] at (43.950000, -13.300000) {};
\node[Kanji] at (43.950000, -12.800000) {\textcolor[HTML]{1551b8}{静}};
\node[Onyomi] at (44.000000, -13.200000) {セイ};
\node[Kunyomi] at (43.900000, -13.200000) {しず.か};
\node[Meaning] at (43.950000, -11.550000) {quiet};
\node[Square] at (46.000000, -13.300000) {};
\node[Kanji] at (46.000000, -12.800000) {\textcolor[HTML]{14469c}{責}};
\node[Onyomi] at (46.050000, -13.200000) {セキ};
\node[Kunyomi] at (45.950000, -13.200000) {せ.める};
\node[Meaning] at (46.000000, -11.550000) {blame};
\node[Square] at (48.050000, -13.300000) {};
\node[Kanji] at (48.050000, -12.800000) {\textcolor[HTML]{133c80}{績}};
\node[Onyomi] at (48.100000, -13.200000) {セキ};
\node[Meaning] at (48.050000, -11.550000) {exploits};
\node[Square] at (50.100000, -13.300000) {};
\node[Kanji] at (50.100000, -12.800000) {\textcolor[HTML]{14469c}{積}};
\node[Onyomi] at (50.150000, -13.200000) {セキ};
\node[Kunyomi] at (50.050000, -13.200000) {つ.む};
\node[Meaning] at (50.100000, -11.550000) {accumulate};
\node[Square] at (52.150000, -13.300000) {};
\node[Kanji] at (52.150000, -12.800000) {\textcolor[HTML]{0e254c}{債}};
\node[Onyomi] at (52.200000, -13.200000) {サイ};
\node[Meaning] at (52.150000, -11.550000) {debt};
\node[Square] at (54.200000, -13.300000) {};
\node[Kanji] at (54.200000, -12.800000) {\textcolor[HTML]{113066}{漬}};
\node[Onyomi] at (54.250000, -13.200000) {シ};
\node[Kunyomi] at (54.150000, -13.200000) {つ};
\node[Meaning] at (54.200000, -11.550000) {pickle};
\node[Square] at (56.250000, -13.300000) {};
\node[Kanji] at (56.250000, -12.800000) {\textcolor[HTML]{1557c6}{表}};
\node[Onyomi] at (56.300000, -13.200000) {ヒョウ};
\node[Kunyomi] at (56.200000, -13.200000) {あらわ.す};
\node[Meaning] at (56.250000, -11.550000) {express};
\node[Meaning] at (-58.500000, -12.750000) {78.90\%};
\node[Square] at (-56.500000, -15.350000) {};
\node[Kanji] at (-56.500000, -14.850000) {\textcolor[HTML]{0e254c}{俵}};
\node[Onyomi] at (-56.450000, -15.250000) {ヒョウ};
\node[Kunyomi] at (-56.550000, -15.250000) {たわら};
\node[Meaning] at (-56.500000, -13.600000) {sack};
\node[Square] at (-54.450000, -15.350000) {};
\node[Kanji] at (-54.450000, -14.850000) {\textcolor[HTML]{113066}{潔}};
\node[Onyomi] at (-54.400000, -15.250000) {ケツ};
\node[Kunyomi] at (-54.500000, -15.250000) {いさぎよ.い};
\node[Meaning] at (-54.450000, -13.600000) {pure};
\node[Square] at (-52.400000, -15.350000) {};
\node[Kanji] at (-52.400000, -14.850000) {\textcolor[HTML]{123673}{契}};
\node[Onyomi] at (-52.350000, -15.250000) {ケイ};
\node[Meaning] at (-52.400000, -13.600000) {pledge};
\node[Square] at (-50.350000, -15.350000) {};
\node[Kanji] at (-50.350000, -14.850000) {\textcolor[HTML]{113066}{喫}};
\node[Onyomi] at (-50.300000, -15.250000) {キツ};
\node[Kunyomi] at (-50.400000, -15.250000) {の.む};
\node[Meaning] at (-50.350000, -13.600000) {consume};
\node[Square] at (-48.300000, -15.350000) {};
\node[Kanji] at (-48.300000, -14.850000) {\textcolor[HTML]{1551b8}{害}};
\node[Onyomi] at (-48.250000, -15.250000) {ガイ};
\node[Meaning] at (-48.300000, -13.600000) {damage};
\node[Square] at (-46.250000, -15.350000) {};
\node[Kanji] at (-46.250000, -14.850000) {\textcolor[HTML]{0e254c}{轄}};
\node[Onyomi] at (-46.200000, -15.250000) {カツ};
\node[Kunyomi] at (-46.300000, -15.250000) {くさび};
\node[Meaning] at (-46.250000, -13.600000) {control};
\node[Square] at (-44.200000, -15.350000) {};
\node[Kanji] at (-44.200000, -14.850000) {\textcolor[HTML]{154caa}{割}};
\node[Onyomi] at (-44.150000, -15.250000) {カツ};
\node[Kunyomi] at (-44.250000, -15.250000) {わり};
\node[Meaning] at (-44.200000, -13.600000) {divide};
\node[Square] at (-42.150000, -15.350000) {};
\node[Kanji] at (-42.150000, -14.850000) {\textcolor[HTML]{113066}{憲}};
\node[Onyomi] at (-42.100000, -15.250000) {ケン};
\node[Meaning] at (-42.150000, -13.600000) {constitution};
\node[Square] at (-40.100000, -15.350000) {};
\node[Kanji] at (-40.100000, -14.850000) {\textcolor[HTML]{2570ef}{生}};
\node[Onyomi] at (-40.050000, -15.250000) {セイ};
\node[Kunyomi] at (-40.150000, -15.250000) {い.きる};
\node[Meaning] at (-40.100000, -13.600000) {life};
\node[Square] at (-38.050000, -15.350000) {};
\node[Kanji] at (-38.050000, -14.850000) {\textcolor[HTML]{1551b8}{星}};
\node[Onyomi] at (-38.000000, -15.250000) {セイ};
\node[Kunyomi] at (-38.100000, -15.250000) {ほし};
\node[Meaning] at (-38.050000, -13.600000) {star};
\node[Square] at (-36.000000, -15.350000) {};
\node[Kanji] at (-36.000000, -14.850000) {\textcolor[HTML]{102b59}{姓}};
\node[Onyomi] at (-35.950000, -15.250000) {セイ};
\node[Meaning] at (-36.000000, -13.600000) {surname};
\node[Square] at (-33.950000, -15.350000) {};
\node[Kanji] at (-33.950000, -14.850000) {\textcolor[HTML]{145cd5}{性}};
\node[Onyomi] at (-33.900000, -15.250000) {セイ};
\node[Meaning] at (-33.950000, -13.600000) {gender};
\node[Square] at (-31.900000, -15.350000) {};
\node[Kanji] at (-31.900000, -14.850000) {\textcolor[HTML]{133c80}{牲}};
\node[Onyomi] at (-31.850000, -15.250000) {セイ};
\node[Meaning] at (-31.900000, -13.600000) {offering};
\node[Square] at (-29.850000, -15.350000) {};
\node[Kanji] at (-29.850000, -14.850000) {\textcolor[HTML]{1551b8}{産}};
\node[Onyomi] at (-29.800000, -15.250000) {サン};
\node[Kunyomi] at (-29.900000, -15.250000) {う.む};
\node[Meaning] at (-29.850000, -13.600000) {give birth};
\node[Square] at (-27.800000, -15.350000) {};
\node[Kanji] at (-27.800000, -14.850000) {\textcolor[HTML]{113066}{隆}};
\node[Onyomi] at (-27.750000, -15.250000) {リュウ};
\node[Meaning] at (-27.800000, -13.600000) {prosperity};
\node[Square] at (-25.750000, -15.350000) {};
\node[Kanji] at (-25.750000, -14.850000) {\textcolor[HTML]{0e254c}{峰}};
\node[Onyomi] at (-25.700000, -15.250000) {ホウ};
\node[Kunyomi] at (-25.800000, -15.250000) {みね};
\node[Meaning] at (-25.750000, -13.600000) {summit};
\node[Square] at (-23.700000, -15.350000) {};
\node[Kanji] at (-23.700000, -14.850000) {\textcolor[HTML]{133c80}{縫}};
\node[Onyomi] at (-23.650000, -15.250000) {ホウ};
\node[Kunyomi] at (-23.750000, -15.250000) {ぬ};
\node[Meaning] at (-23.700000, -13.600000) {sew};
\node[Square] at (-21.650000, -15.350000) {};
\node[Kanji] at (-21.650000, -14.850000) {\textcolor[HTML]{133c80}{拝}};
\node[Onyomi] at (-21.600000, -15.250000) {ハイ};
\node[Kunyomi] at (-21.700000, -15.250000) {おが.む};
\node[Meaning] at (-21.650000, -13.600000) {worship};
\node[Square] at (-19.600000, -15.350000) {};
\node[Kanji] at (-19.600000, -14.850000) {\textcolor[HTML]{123673}{寿}};
\node[Onyomi] at (-19.550000, -15.250000) {ジュ};
\node[Kunyomi] at (-19.650000, -15.250000) {ことぶき};
\node[Meaning] at (-19.600000, -13.600000) {lifespan};
\node[Square] at (-17.550000, -15.350000) {};
\node[Kanji] at (-17.550000, -14.850000) {\textcolor[HTML]{0e254c}{鋳}};
\node[Onyomi] at (-17.500000, -15.250000) {チュウ};
\node[Kunyomi] at (-17.600000, -15.250000) {い};
\node[Meaning] at (-17.550000, -13.600000) {cast};
\node[Square] at (-15.500000, -15.350000) {};
\node[Kanji] at (-15.500000, -14.850000) {\textcolor[HTML]{102b59}{籍}};
\node[Onyomi] at (-15.450000, -15.250000) {セキ};
\node[Meaning] at (-15.500000, -13.600000) {enroll};
\node[Square] at (-13.450000, -15.350000) {};
\node[Kanji] at (-13.450000, -14.850000) {\textcolor[HTML]{14469c}{春}};
\node[Onyomi] at (-13.400000, -15.250000) {シュン};
\node[Kunyomi] at (-13.500000, -15.250000) {はる};
\node[Meaning] at (-13.450000, -13.600000) {spring};
\node[Square] at (-11.400000, -15.350000) {};
\node[Kanji] at (-11.400000, -14.850000) {\textcolor[HTML]{0e254c}{泰}};
\node[Onyomi] at (-11.350000, -15.250000) {タイ};
\node[Meaning] at (-11.400000, -13.600000) {peace};
\node[Square] at (-9.350000, -15.350000) {};
\node[Kanji] at (-9.350000, -14.850000) {\textcolor[HTML]{123673}{奏}};
\node[Onyomi] at (-9.300000, -15.250000) {ソウ};
\node[Kunyomi] at (-9.400000, -15.250000) {かな.でる};
\node[Meaning] at (-9.350000, -13.600000) {play music};
\node[Square] at (-7.300000, -15.350000) {};
\node[Kanji] at (-7.300000, -14.850000) {\textcolor[HTML]{1557c6}{実}};
\node[Onyomi] at (-7.250000, -15.250000) {ジツ};
\node[Kunyomi] at (-7.350000, -15.250000) {み};
\node[Meaning] at (-7.300000, -13.600000) {truth};
\node[Square] at (-5.250000, -15.350000) {};
\node[Kanji] at (-5.250000, -14.850000) {\textcolor[HTML]{0e254c}{奉}};
\node[Onyomi] at (-5.200000, -15.250000) {ホウ};
\node[Kunyomi] at (-5.300000, -15.250000) {たてまつ};
\node[Meaning] at (-5.250000, -13.600000) {dedicate};
\node[Square] at (-3.200000, -15.350000) {};
\node[Kanji] at (-3.200000, -14.850000) {\textcolor[HTML]{0e254c}{俸}};
\node[Onyomi] at (-3.150000, -15.250000) {ホウ};
\node[Meaning] at (-3.200000, -13.600000) {salary};
\node[Square] at (-1.150000, -15.350000) {};
\node[Kanji] at (-1.150000, -14.850000) {\textcolor[HTML]{14418e}{棒}};
\node[Onyomi] at (-1.100000, -15.250000) {ボウ};
\node[Kunyomi] at (-1.200000, -15.250000) {ぼう};
\node[Meaning] at (-1.150000, -13.600000) {pole};
\node[Square] at (0.900000, -15.350000) {};
\node[Kanji] at (0.900000, -14.850000) {\textcolor[HTML]{0e254c}{謹}};
\node[Onyomi] at (0.950000, -15.250000) {キン};
\node[Kunyomi] at (0.850000, -15.250000) {つつし};
\node[Meaning] at (0.900000, -13.600000) {humble};
\node[Square] at (2.950000, -15.350000) {};
\node[Kanji] at (2.950000, -14.850000) {\textcolor[HTML]{133c80}{勤}};
\node[Onyomi] at (3.000000, -15.250000) {キン};
\node[Kunyomi] at (2.900000, -15.250000) {つと.*};
\node[Meaning] at (2.950000, -13.600000) {work};
\node[Square] at (5.000000, -15.350000) {};
\node[Kanji] at (5.000000, -14.850000) {\textcolor[HTML]{123673}{漢}};
\node[Onyomi] at (5.050000, -15.250000) {カン};
\node[Meaning] at (5.000000, -13.600000) {chinese};
\node[Square] at (7.050000, -15.350000) {};
\node[Kanji] at (7.050000, -14.850000) {\textcolor[HTML]{133c80}{嘆}};
\node[Onyomi] at (7.100000, -15.250000) {タン};
\node[Kunyomi] at (7.000000, -15.250000) {なげ.く};
\node[Meaning] at (7.050000, -13.600000) {sigh};
\node[Square] at (9.100000, -15.350000) {};
\node[Kanji] at (9.100000, -14.850000) {\textcolor[HTML]{1551b8}{難}};
\node[Onyomi] at (9.150000, -15.250000) {ナン};
\node[Kunyomi] at (9.050000, -15.250000) {むずか.しい};
\node[Meaning] at (9.100000, -13.600000) {difficult};
\node[Square] at (11.150000, -15.350000) {};
\node[Kanji] at (11.150000, -14.850000) {\textcolor[HTML]{133c80}{華}};
\node[Onyomi] at (11.200000, -15.250000) {カ};
\node[Kunyomi] at (11.100000, -15.250000) {はな};
\node[Meaning] at (11.150000, -13.600000) {showy};
\node[Square] at (13.200000, -15.350000) {};
\node[Kanji] at (13.200000, -14.850000) {\textcolor[HTML]{14418e}{垂}};
\node[Onyomi] at (13.250000, -15.250000) {スイ};
\node[Kunyomi] at (13.150000, -15.250000) {た.*};
\node[Meaning] at (13.200000, -13.600000) {dangle};
\node[Square] at (15.250000, -15.350000) {};
\node[Kanji] at (15.250000, -14.850000) {\textcolor[HTML]{113066}{睡}};
\node[Onyomi] at (15.300000, -15.250000) {スイ};
\node[Meaning] at (15.250000, -13.600000) {drowsy};
\node[Square] at (17.300000, -15.350000) {};
\node[Kanji] at (17.300000, -14.850000) {\textcolor[HTML]{1557c6}{乗}};
\node[Onyomi] at (17.350000, -15.250000) {ジョウ};
\node[Kunyomi] at (17.250000, -15.250000) {の};
\node[Meaning] at (17.300000, -13.600000) {ride};
\node[Square] at (19.350000, -15.350000) {};
\node[Kanji] at (19.350000, -14.850000) {\textcolor[HTML]{0e254c}{剰}};
\node[Onyomi] at (19.400000, -15.250000) {ジョウ};
\node[Kunyomi] at (19.300000, -15.250000) {あまつさえ};
\node[Meaning] at (19.350000, -13.600000) {surplus};
\node[Square] at (21.400000, -15.350000) {};
\node[Kanji] at (21.400000, -14.850000) {\textcolor[HTML]{1461e3}{今}};
\node[Onyomi] at (21.450000, -15.250000) {コン};
\node[Kunyomi] at (21.350000, -15.250000) {いま};
\node[Meaning] at (21.400000, -13.600000) {now};
\node[Square] at (23.450000, -15.350000) {};
\node[Kanji] at (23.450000, -14.850000) {\textcolor[HTML]{133c80}{含}};
\node[Onyomi] at (23.500000, -15.250000) {ガン};
\node[Kunyomi] at (23.400000, -15.250000) {ふく.む};
\node[Meaning] at (23.450000, -13.600000) {include};
\node[Square] at (25.500000, -15.350000) {};
\node[Kanji] at (25.500000, -14.850000) {\textcolor[HTML]{113066}{吟}};
\node[Onyomi] at (25.550000, -15.250000) {ギン};
\node[Meaning] at (25.500000, -13.600000) {recital};
\node[Square] at (27.550000, -15.350000) {};
\node[Kanji] at (27.550000, -14.850000) {\textcolor[HTML]{154caa}{念}};
\node[Onyomi] at (27.600000, -15.250000) {ネン};
\node[Meaning] at (27.550000, -13.600000) {thought};
\node[Square] at (29.600000, -15.350000) {};
\node[Kanji] at (29.600000, -14.850000) {\textcolor[HTML]{102b59}{琴}};
\node[Kunyomi] at (29.550000, -15.250000) {こと};
\node[Meaning] at (29.600000, -13.600000) {harp};
\node[Square] at (31.650000, -15.350000) {};
\node[Kanji] at (31.650000, -14.850000) {\textcolor[HTML]{14469c}{陰}};
\node[Onyomi] at (31.700000, -15.250000) {イン};
\node[Kunyomi] at (31.600000, -15.250000) {かげ};
\node[Meaning] at (31.650000, -13.600000) {shade};
\node[Square] at (33.700000, -15.350000) {};
\node[Kanji] at (33.700000, -14.850000) {\textcolor[HTML]{1557c6}{予}};
\node[Onyomi] at (33.750000, -15.250000) {ヨ};
\node[Kunyomi] at (33.650000, -15.250000) {あらかじ};
\node[Meaning] at (33.700000, -13.600000) {beforehand};
\node[Square] at (35.750000, -15.350000) {};
\node[Kanji] at (35.750000, -14.850000) {\textcolor[HTML]{102b59}{序}};
\node[Onyomi] at (35.800000, -15.250000) {ジョ};
\node[Kunyomi] at (35.700000, -15.250000) {つい};
\node[Meaning] at (35.750000, -13.600000) {preface};
\node[Square] at (37.800000, -15.350000) {};
\node[Kanji] at (37.800000, -14.850000) {\textcolor[HTML]{133c80}{預}};
\node[Onyomi] at (37.850000, -15.250000) {ヨ};
\node[Kunyomi] at (37.750000, -15.250000) {あず.ける};
\node[Meaning] at (37.800000, -13.600000) {deposit};
\node[Square] at (39.850000, -15.350000) {};
\node[Kanji] at (39.850000, -14.850000) {\textcolor[HTML]{1551b8}{野}};
\node[Onyomi] at (39.900000, -15.250000) {ヤ};
\node[Kunyomi] at (39.800000, -15.250000) {の};
\node[Meaning] at (39.850000, -13.600000) {field};
\node[Square] at (41.900000, -15.350000) {};
\node[Kanji] at (41.900000, -14.850000) {\textcolor[HTML]{0e254c}{兼}};
\node[Onyomi] at (41.950000, -15.250000) {ケン};
\node[Kunyomi] at (41.850000, -15.250000) {か.ねる};
\node[Meaning] at (41.900000, -13.600000) {concurrently};
\node[Square] at (43.950000, -15.350000) {};
\node[Kanji] at (43.950000, -14.850000) {\textcolor[HTML]{154caa}{嫌}};
\node[Onyomi] at (44.000000, -15.250000) {ケン};
\node[Kunyomi] at (43.900000, -15.250000) {いや};
\node[Meaning] at (43.950000, -13.600000) {dislike};
\node[Square] at (46.000000, -15.350000) {};
\node[Kanji] at (46.000000, -14.850000) {\textcolor[HTML]{113066}{鎌}};
\node[Onyomi] at (46.050000, -15.250000) {ケン};
\node[Kunyomi] at (45.950000, -15.250000) {かま};
\node[Meaning] at (46.000000, -13.600000) {sickle};
\node[Square] at (48.050000, -15.350000) {};
\node[Kanji] at (48.050000, -14.850000) {\textcolor[HTML]{0e254c}{謙}};
\node[Onyomi] at (48.100000, -15.250000) {ケン};
\node[Meaning] at (48.050000, -13.600000) {modesty};
\node[Square] at (50.100000, -15.350000) {};
\node[Kanji] at (50.100000, -14.850000) {\textcolor[HTML]{102b59}{廉}};
\node[Onyomi] at (50.150000, -15.250000) {レン};
\node[Meaning] at (50.100000, -13.600000) {bargain};
\node[Square] at (52.150000, -15.350000) {};
\node[Kanji] at (52.150000, -14.850000) {\textcolor[HTML]{154caa}{西}};
\node[Onyomi] at (52.200000, -15.250000) {セイ};
\node[Kunyomi] at (52.100000, -15.250000) {にし};
\node[Meaning] at (52.150000, -13.600000) {west};
\node[Square] at (54.200000, -15.350000) {};
\node[Kanji] at (54.200000, -14.850000) {\textcolor[HTML]{14418e}{価}};
\node[Onyomi] at (54.250000, -15.250000) {カ};
\node[Kunyomi] at (54.150000, -15.250000) {あたい};
\node[Meaning] at (54.200000, -13.600000) {value};
\node[Square] at (56.250000, -15.350000) {};
\node[Kanji] at (56.250000, -14.850000) {\textcolor[HTML]{1557c6}{要}};
\node[Onyomi] at (56.300000, -15.250000) {ヨウ};
\node[Kunyomi] at (56.200000, -15.250000) {い};
\node[Meaning] at (56.250000, -13.600000) {need};
\node[Meaning] at (-58.500000, -14.800000) {81.43\%};
\node[Square] at (-56.500000, -17.400000) {};
\node[Kanji] at (-56.500000, -16.900000) {\textcolor[HTML]{154caa}{腰}};
\node[Onyomi] at (-56.450000, -17.300000) {ヨウ};
\node[Kunyomi] at (-56.550000, -17.300000) {こし};
\node[Meaning] at (-56.500000, -15.650000) {waist};
\node[Square] at (-54.450000, -17.400000) {};
\node[Kanji] at (-54.450000, -16.900000) {\textcolor[HTML]{14418e}{票}};
\node[Onyomi] at (-54.400000, -17.300000) {ヒョウ};
\node[Meaning] at (-54.450000, -15.650000) {ballot};
\node[Square] at (-52.400000, -17.400000) {};
\node[Kanji] at (-52.400000, -16.900000) {\textcolor[HTML]{14418e}{漂}};
\node[Onyomi] at (-52.350000, -17.300000) {ヒョウ};
\node[Kunyomi] at (-52.450000, -17.300000) {ただよ.う};
\node[Meaning] at (-52.400000, -15.650000) {drift};
\node[Square] at (-50.350000, -17.400000) {};
\node[Kanji] at (-50.350000, -16.900000) {\textcolor[HTML]{14418e}{標}};
\node[Onyomi] at (-50.300000, -17.300000) {ヒョウ};
\node[Kunyomi] at (-50.400000, -17.300000) {しるし};
\node[Meaning] at (-50.350000, -15.650000) {signpost};
\node[Square] at (-48.300000, -17.400000) {};
\node[Kanji] at (-48.300000, -16.900000) {\textcolor[HTML]{0e254c}{遷}};
\node[Onyomi] at (-48.250000, -17.300000) {セン};
\node[Kunyomi] at (-48.350000, -17.300000) {うつ};
\node[Meaning] at (-48.300000, -15.650000) {transition};
\node[Square] at (-46.250000, -17.400000) {};
\node[Kanji] at (-46.250000, -16.900000) {\textcolor[HTML]{154caa}{覆}};
\node[Onyomi] at (-46.200000, -17.300000) {フク};
\node[Kunyomi] at (-46.300000, -17.300000) {おお.う};
\node[Meaning] at (-46.250000, -15.650000) {capsize};
\node[Square] at (-44.200000, -17.400000) {};
\node[Kanji] at (-44.200000, -16.900000) {\textcolor[HTML]{14469c}{煙}};
\node[Onyomi] at (-44.150000, -17.300000) {エン};
\node[Kunyomi] at (-44.250000, -17.300000) {けむ.り};
\node[Meaning] at (-44.200000, -15.650000) {smoke};
\node[Square] at (-42.150000, -17.400000) {};
\node[Kanji] at (-42.150000, -16.900000) {\textcolor[HTML]{14469c}{南}};
\node[Onyomi] at (-42.100000, -17.300000) {ナン};
\node[Kunyomi] at (-42.200000, -17.300000) {みなみ};
\node[Meaning] at (-42.150000, -15.650000) {south};
\node[Square] at (-40.100000, -17.400000) {};
\node[Kanji] at (-40.100000, -16.900000) {\textcolor[HTML]{113066}{献}};
\node[Onyomi] at (-40.050000, -17.300000) {ケン};
\node[Kunyomi] at (-40.150000, -17.300000) {たてまつ.る};
\node[Meaning] at (-40.100000, -15.650000) {offer};
\node[Square] at (-38.050000, -17.400000) {};
\node[Kanji] at (-38.050000, -16.900000) {\textcolor[HTML]{154caa}{門}};
\node[Onyomi] at (-38.000000, -17.300000) {モン};
\node[Meaning] at (-38.050000, -15.650000) {gates};
\node[Square] at (-36.000000, -17.400000) {};
\node[Kanji] at (-36.000000, -16.900000) {\textcolor[HTML]{1557c6}{問}};
\node[Onyomi] at (-35.950000, -17.300000) {モン};
\node[Kunyomi] at (-36.050000, -17.300000) {と};
\node[Meaning] at (-36.000000, -15.650000) {problem};
\node[Square] at (-33.950000, -17.400000) {};
\node[Kanji] at (-33.950000, -16.900000) {\textcolor[HTML]{0e254c}{閲}};
\node[Onyomi] at (-33.900000, -17.300000) {エツ};
\node[Kunyomi] at (-34.000000, -17.300000) {けみ};
\node[Meaning] at (-33.950000, -15.650000) {inspection};
\node[Square] at (-31.900000, -17.400000) {};
\node[Kanji] at (-31.900000, -16.900000) {\textcolor[HTML]{0e254c}{閥}};
\node[Onyomi] at (-31.850000, -17.300000) {バツ};
\node[Meaning] at (-31.900000, -15.650000) {clique};
\node[Square] at (-29.850000, -17.400000) {};
\node[Kanji] at (-29.850000, -16.900000) {\textcolor[HTML]{1968ed}{間}};
\node[Onyomi] at (-29.800000, -17.300000) {カン};
\node[Kunyomi] at (-29.900000, -17.300000) {あいだ};
\node[Meaning] at (-29.850000, -15.650000) {interval};
\node[Square] at (-27.800000, -17.400000) {};
\node[Kanji] at (-27.800000, -16.900000) {\textcolor[HTML]{14469c}{簡}};
\node[Onyomi] at (-27.750000, -17.300000) {カン};
\node[Meaning] at (-27.800000, -15.650000) {simplicity};
\node[Square] at (-25.750000, -17.400000) {};
\node[Kanji] at (-25.750000, -16.900000) {\textcolor[HTML]{145cd5}{開}};
\node[Onyomi] at (-25.700000, -17.300000) {カイ};
\node[Kunyomi] at (-25.800000, -17.300000) {あ.ける};
\node[Meaning] at (-25.750000, -15.650000) {open};
\node[Square] at (-23.700000, -17.400000) {};
\node[Kanji] at (-23.700000, -16.900000) {\textcolor[HTML]{1551b8}{閉}};
\node[Onyomi] at (-23.650000, -17.300000) {ヘイ};
\node[Kunyomi] at (-23.750000, -17.300000) {し};
\node[Meaning] at (-23.700000, -15.650000) {closed};
\node[Square] at (-21.650000, -17.400000) {};
\node[Kanji] at (-21.650000, -16.900000) {\textcolor[HTML]{133c80}{閣}};
\node[Onyomi] at (-21.600000, -17.300000) {カク};
\node[Meaning] at (-21.650000, -15.650000) {the cabinet};
\node[Square] at (-19.600000, -17.400000) {};
\node[Kanji] at (-19.600000, -16.900000) {\textcolor[HTML]{0e254c}{閑}};
\node[Onyomi] at (-19.550000, -17.300000) {カン};
\node[Meaning] at (-19.600000, -15.650000) {leisure};
\node[Square] at (-17.550000, -17.400000) {};
\node[Kanji] at (-17.550000, -16.900000) {\textcolor[HTML]{1968ed}{聞}};
\node[Onyomi] at (-17.500000, -17.300000) {ブン};
\node[Kunyomi] at (-17.600000, -17.300000) {き.く};
\node[Meaning] at (-17.550000, -15.650000) {hear};
\node[Square] at (-15.500000, -17.400000) {};
\node[Kanji] at (-15.500000, -16.900000) {\textcolor[HTML]{113066}{潤}};
\node[Onyomi] at (-15.450000, -17.300000) {ジュン};
\node[Kunyomi] at (-15.550000, -17.300000) {うるお.*};
\node[Meaning] at (-15.500000, -15.650000) {watered};
\node[Square] at (-13.450000, -17.400000) {};
\node[Kanji] at (-13.450000, -16.900000) {\textcolor[HTML]{0e254c}{欄}};
\node[Onyomi] at (-13.400000, -17.300000) {ラン};
\node[Kunyomi] at (-13.500000, -17.300000) {てすり        };
\node[Meaning] at (-13.450000, -15.650000) {column};
\node[Square] at (-11.400000, -17.400000) {};
\node[Kanji] at (-11.400000, -16.900000) {\textcolor[HTML]{14418e}{闘}};
\node[Onyomi] at (-11.350000, -17.300000) {トウ};
\node[Kunyomi] at (-11.450000, -17.300000) {たたか.う};
\node[Meaning] at (-11.400000, -15.650000) {struggle};
\node[Square] at (-9.350000, -17.400000) {};
\node[Kanji] at (-9.350000, -16.900000) {\textcolor[HTML]{133c80}{倉}};
\node[Onyomi] at (-9.300000, -17.300000) {ソウ};
\node[Kunyomi] at (-9.400000, -17.300000) {くら};
\node[Meaning] at (-9.350000, -15.650000) {warehouse};
\node[Square] at (-7.300000, -17.400000) {};
\node[Kanji] at (-7.300000, -16.900000) {\textcolor[HTML]{133c80}{創}};
\node[Onyomi] at (-7.250000, -17.300000) {ソウ};
\node[Meaning] at (-7.300000, -15.650000) {create};
\node[Square] at (-5.250000, -17.400000) {};
\node[Kanji] at (-5.250000, -16.900000) {\textcolor[HTML]{14469c}{非}};
\node[Onyomi] at (-5.200000, -17.300000) {ヒ};
\node[Meaning] at (-5.250000, -15.650000) {injustice};
\node[Square] at (-3.200000, -17.400000) {};
\node[Kanji] at (-3.200000, -16.900000) {\textcolor[HTML]{123673}{俳}};
\node[Onyomi] at (-3.150000, -17.300000) {ハイ};
\node[Meaning] at (-3.200000, -15.650000) {haiku};
\node[Square] at (-1.150000, -17.400000) {};
\node[Kanji] at (-1.150000, -16.900000) {\textcolor[HTML]{113066}{排}};
\node[Onyomi] at (-1.100000, -17.300000) {ハイ};
\node[Meaning] at (-1.150000, -15.650000) {reject};
\node[Square] at (0.900000, -17.400000) {};
\node[Kanji] at (0.900000, -16.900000) {\textcolor[HTML]{1551b8}{悲}};
\node[Onyomi] at (0.950000, -17.300000) {ヒ};
\node[Kunyomi] at (0.850000, -17.300000) {かな};
\node[Meaning] at (0.900000, -15.650000) {sad};
\node[Square] at (2.950000, -17.400000) {};
\node[Kanji] at (2.950000, -16.900000) {\textcolor[HTML]{14469c}{罪}};
\node[Onyomi] at (3.000000, -17.300000) {ザイ};
\node[Kunyomi] at (2.900000, -17.300000) {つみ};
\node[Meaning] at (2.950000, -15.650000) {guilt};
\node[Square] at (5.000000, -17.400000) {};
\node[Kanji] at (5.000000, -16.900000) {\textcolor[HTML]{14469c}{輩}};
\node[Onyomi] at (5.050000, -17.300000) {ハイ};
\node[Meaning] at (5.000000, -15.650000) {comrade};
\node[Square] at (7.050000, -17.400000) {};
\node[Kanji] at (7.050000, -16.900000) {\textcolor[HTML]{1551b8}{扉}};
\node[Onyomi] at (7.100000, -17.300000) {ヒ};
\node[Kunyomi] at (7.000000, -17.300000) {とびら};
\node[Meaning] at (7.050000, -15.650000) {front door};
\node[Square] at (9.100000, -17.400000) {};
\node[Kanji] at (9.100000, -16.900000) {\textcolor[HTML]{0e254c}{侯}};
\node[Onyomi] at (9.150000, -17.300000) {コウ};
\node[Meaning] at (9.100000, -15.650000) {marquis};
\node[Square] at (11.150000, -17.400000) {};
\node[Kanji] at (11.150000, -16.900000) {\textcolor[HTML]{14418e}{候}};
\node[Onyomi] at (11.200000, -17.300000) {コウ};
\node[Meaning] at (11.150000, -15.650000) {climate};
\node[Square] at (13.200000, -17.400000) {};
\node[Kanji] at (13.200000, -16.900000) {\textcolor[HTML]{145cd5}{決}};
\node[Onyomi] at (13.250000, -17.300000) {ケツ};
\node[Kunyomi] at (13.150000, -17.300000) {き.める};
\node[Meaning] at (13.200000, -15.650000) {decide};
\node[Square] at (15.250000, -17.400000) {};
\node[Kanji] at (15.250000, -16.900000) {\textcolor[HTML]{14418e}{快}};
\node[Onyomi] at (15.300000, -17.300000) {カイ};
\node[Kunyomi] at (15.200000, -17.300000) {こころよ.い};
\node[Meaning] at (15.250000, -15.650000) {pleasant};
\node[Square] at (17.300000, -17.400000) {};
\node[Kanji] at (17.300000, -16.900000) {\textcolor[HTML]{133c80}{偉}};
\node[Onyomi] at (17.350000, -17.300000) {イ};
\node[Kunyomi] at (17.250000, -17.300000) {えら};
\node[Meaning] at (17.300000, -15.650000) {greatness};
\node[Square] at (19.350000, -17.400000) {};
\node[Kanji] at (19.350000, -16.900000) {\textcolor[HTML]{145cd5}{違}};
\node[Onyomi] at (19.400000, -17.300000) {イ};
\node[Kunyomi] at (19.300000, -17.300000) {ちが};
\node[Meaning] at (19.350000, -15.650000) {different};
\node[Square] at (21.400000, -17.400000) {};
\node[Kanji] at (21.400000, -16.900000) {\textcolor[HTML]{0e254c}{緯}};
\node[Onyomi] at (21.450000, -17.300000) {イ};
\node[Kunyomi] at (21.350000, -17.300000) {ぬき};
\node[Meaning] at (21.400000, -15.650000) {latitude};
\node[Square] at (23.450000, -17.400000) {};
\node[Kanji] at (23.450000, -16.900000) {\textcolor[HTML]{154caa}{衛}};
\node[Onyomi] at (23.500000, -17.300000) {エイ};
\node[Meaning] at (23.450000, -15.650000) {defense};
\node[Square] at (25.500000, -17.400000) {};
\node[Kanji] at (25.500000, -16.900000) {\textcolor[HTML]{14469c}{韓}};
\node[Onyomi] at (25.550000, -17.300000) {カン};
\node[Meaning] at (25.500000, -15.650000) {korea};
\node[Square] at (27.550000, -17.400000) {};
\node[Kanji] at (27.550000, -16.900000) {\textcolor[HTML]{133c80}{干}};
\node[Onyomi] at (27.600000, -17.300000) {カン};
\node[Kunyomi] at (27.500000, -17.300000) {ほ.す};
\node[Meaning] at (27.550000, -15.650000) {dry};
\node[Square] at (29.600000, -17.400000) {};
\node[Kanji] at (29.600000, -16.900000) {\textcolor[HTML]{133c80}{肝}};
\node[Onyomi] at (29.650000, -17.300000) {カン};
\node[Kunyomi] at (29.550000, -17.300000) {きも};
\node[Meaning] at (29.600000, -15.650000) {liver};
\node[Square] at (31.650000, -17.400000) {};
\node[Kanji] at (31.650000, -16.900000) {\textcolor[HTML]{14418e}{刊}};
\node[Onyomi] at (31.700000, -17.300000) {カン};
\node[Meaning] at (31.650000, -15.650000) {edition};
\node[Square] at (33.700000, -17.400000) {};
\node[Kanji] at (33.700000, -16.900000) {\textcolor[HTML]{14418e}{汗}};
\node[Onyomi] at (33.750000, -17.300000) {カン};
\node[Kunyomi] at (33.650000, -17.300000) {あせ};
\node[Meaning] at (33.700000, -15.650000) {sweat};
\node[Square] at (35.750000, -17.400000) {};
\node[Kanji] at (35.750000, -16.900000) {\textcolor[HTML]{123673}{軒}};
\node[Onyomi] at (35.800000, -17.300000) {ケン};
\node[Kunyomi] at (35.700000, -17.300000) {のき        };
\node[Meaning] at (35.750000, -15.650000) {house counter};
\node[Square] at (37.800000, -17.400000) {};
\node[Kanji] at (37.800000, -16.900000) {\textcolor[HTML]{14469c}{岸}};
\node[Onyomi] at (37.850000, -17.300000) {ガン};
\node[Kunyomi] at (37.750000, -17.300000) {きし};
\node[Meaning] at (37.800000, -15.650000) {coast};
\node[Square] at (39.850000, -17.400000) {};
\node[Kanji] at (39.850000, -16.900000) {\textcolor[HTML]{14469c}{幹}};
\node[Onyomi] at (39.900000, -17.300000) {カン};
\node[Kunyomi] at (39.800000, -17.300000) {みき};
\node[Meaning] at (39.850000, -15.650000) {tree trunk};
\node[Square] at (41.900000, -17.400000) {};
\node[Kanji] at (41.900000, -16.900000) {\textcolor[HTML]{0e254c}{芋}};
\node[Kunyomi] at (41.850000, -17.300000) {いも};
\node[Meaning] at (41.900000, -15.650000) {potato};
\node[Square] at (43.950000, -17.400000) {};
\node[Kanji] at (43.950000, -16.900000) {\textcolor[HTML]{154caa}{宇}};
\node[Onyomi] at (44.000000, -17.300000) {ウ};
\node[Meaning] at (43.950000, -15.650000) {outer space};
\node[Square] at (46.000000, -17.400000) {};
\node[Kanji] at (46.000000, -16.900000) {\textcolor[HTML]{14418e}{余}};
\node[Onyomi] at (46.050000, -17.300000) {ヨ};
\node[Kunyomi] at (45.950000, -17.300000) {あま.る};
\node[Meaning] at (46.000000, -15.650000) {surplus};
\node[Square] at (48.050000, -17.400000) {};
\node[Kanji] at (48.050000, -16.900000) {\textcolor[HTML]{14469c}{除}};
\node[Onyomi] at (48.100000, -17.300000) {ジョ};
\node[Kunyomi] at (48.000000, -17.300000) {のぞ.く};
\node[Meaning] at (48.050000, -15.650000) {exclude};
\node[Square] at (50.100000, -17.400000) {};
\node[Kanji] at (50.100000, -16.900000) {\textcolor[HTML]{123673}{徐}};
\node[Onyomi] at (50.150000, -17.300000) {ジョ};
\node[Kunyomi] at (50.050000, -17.300000) {おもむ};
\node[Meaning] at (50.100000, -15.650000) {gently};
\node[Square] at (52.150000, -17.400000) {};
\node[Kanji] at (52.150000, -16.900000) {\textcolor[HTML]{0e254c}{叙}};
\node[Onyomi] at (52.200000, -17.300000) {ジョ};
\node[Kunyomi] at (52.100000, -17.300000) {つい};
\node[Meaning] at (52.150000, -15.650000) {describe};
\node[Square] at (54.200000, -17.400000) {};
\node[Kanji] at (54.200000, -16.900000) {\textcolor[HTML]{154caa}{途}};
\node[Onyomi] at (54.250000, -17.300000) {ト};
\node[Meaning] at (54.200000, -15.650000) {route};
\node[Square] at (56.250000, -17.400000) {};
\node[Kanji] at (56.250000, -16.900000) {\textcolor[HTML]{133c80}{斜}};
\node[Onyomi] at (56.300000, -17.300000) {シャ};
\node[Kunyomi] at (56.200000, -17.300000) {なな.め};
\node[Meaning] at (56.250000, -15.650000) {diagonal};
\node[Meaning] at (-58.500000, -16.850000) {84.19\%};
\node[Square] at (-56.500000, -19.450000) {};
\node[Kanji] at (-56.500000, -18.950000) {\textcolor[HTML]{133c80}{塗}};
\node[Onyomi] at (-56.450000, -19.350000) {ト};
\node[Kunyomi] at (-56.550000, -19.350000) {ぬる};
\node[Meaning] at (-56.500000, -17.700000) {paint};
\node[Square] at (-54.450000, -19.450000) {};
\node[Kanji] at (-54.450000, -18.950000) {\textcolor[HTML]{14469c}{束}};
\node[Onyomi] at (-54.400000, -19.350000) {ソク};
\node[Kunyomi] at (-54.500000, -19.350000) {たば};
\node[Meaning] at (-54.450000, -17.700000) {bundle};
\node[Square] at (-52.400000, -19.450000) {};
\node[Kanji] at (-52.400000, -18.950000) {\textcolor[HTML]{154caa}{頼}};
\node[Onyomi] at (-52.350000, -19.350000) {ライ};
\node[Kunyomi] at (-52.450000, -19.350000) {たの};
\node[Meaning] at (-52.400000, -17.700000) {trust};
\node[Square] at (-50.350000, -19.450000) {};
\node[Kanji] at (-50.350000, -18.950000) {\textcolor[HTML]{123673}{瀬}};
\node[Onyomi] at (-50.300000, -19.350000) {ライ};
\node[Kunyomi] at (-50.400000, -19.350000) {せ};
\node[Meaning] at (-50.350000, -17.700000) {rapids};
\node[Square] at (-48.300000, -19.450000) {};
\node[Kanji] at (-48.300000, -18.950000) {\textcolor[HTML]{0e254c}{勅}};
\node[Onyomi] at (-48.250000, -19.350000) {チョク};
\node[Meaning] at (-48.300000, -17.700000) {imperial order};
\node[Square] at (-46.250000, -19.450000) {};
\node[Kanji] at (-46.250000, -18.950000) {\textcolor[HTML]{102b59}{疎}};
\node[Onyomi] at (-46.200000, -19.350000) {ソ};
\node[Kunyomi] at (-46.300000, -19.350000) {うと};
\node[Meaning] at (-46.250000, -17.700000) {neglect};
\node[Square] at (-44.200000, -19.450000) {};
\node[Kanji] at (-44.200000, -18.950000) {\textcolor[HTML]{1551b8}{速}};
\node[Onyomi] at (-44.150000, -19.350000) {ソク};
\node[Kunyomi] at (-44.250000, -19.350000) {はや.い};
\node[Meaning] at (-44.200000, -17.700000) {fast};
\node[Square] at (-42.150000, -19.450000) {};
\node[Kanji] at (-42.150000, -18.950000) {\textcolor[HTML]{133c80}{整}};
\node[Onyomi] at (-42.100000, -19.350000) {セイ};
\node[Kunyomi] at (-42.200000, -19.350000) {ととの.*};
\node[Meaning] at (-42.150000, -17.700000) {arrange};
\node[Square] at (-40.100000, -19.450000) {};
\node[Kanji] at (-40.100000, -18.950000) {\textcolor[HTML]{154caa}{剣}};
\node[Onyomi] at (-40.050000, -19.350000) {ケン};
\node[Kunyomi] at (-40.150000, -19.350000) {つるぎ};
\node[Meaning] at (-40.100000, -17.700000) {sword};
\node[Square] at (-38.050000, -19.450000) {};
\node[Kanji] at (-38.050000, -18.950000) {\textcolor[HTML]{1551b8}{険}};
\node[Onyomi] at (-38.000000, -19.350000) {ケン};
\node[Kunyomi] at (-38.100000, -19.350000) {けわ.しい};
\node[Meaning] at (-38.050000, -17.700000) {risky};
\node[Square] at (-36.000000, -19.450000) {};
\node[Kanji] at (-36.000000, -18.950000) {\textcolor[HTML]{14469c}{検}};
\node[Onyomi] at (-35.950000, -19.350000) {ケン};
\node[Meaning] at (-36.000000, -17.700000) {examine};
\node[Square] at (-33.950000, -19.450000) {};
\node[Kanji] at (-33.950000, -18.950000) {\textcolor[HTML]{0e254c}{倹}};
\node[Onyomi] at (-33.900000, -19.350000) {ケン};
\node[Kunyomi] at (-34.000000, -19.350000) {つづまやか};
\node[Meaning] at (-33.950000, -17.700000) {thrifty};
\node[Square] at (-31.900000, -19.450000) {};
\node[Kanji] at (-31.900000, -18.950000) {\textcolor[HTML]{1551b8}{重}};
\node[Onyomi] at (-31.850000, -19.350000) {ジュウ};
\node[Kunyomi] at (-31.950000, -19.350000) {おも.い};
\node[Meaning] at (-31.900000, -17.700000) {heavy};
\node[Square] at (-29.850000, -19.450000) {};
\node[Kanji] at (-29.850000, -18.950000) {\textcolor[HTML]{1461e3}{動}};
\node[Onyomi] at (-29.800000, -19.350000) {ドウ};
\node[Kunyomi] at (-29.900000, -19.350000) {うご.*};
\node[Meaning] at (-29.850000, -17.700000) {move};
\node[Square] at (-27.800000, -19.450000) {};
\node[Kanji] at (-27.800000, -18.950000) {\textcolor[HTML]{123673}{勲}};
\node[Onyomi] at (-27.750000, -19.350000) {クン};
\node[Kunyomi] at (-27.850000, -19.350000) {いさお};
\node[Meaning] at (-27.800000, -17.700000) {merit};
\node[Square] at (-25.750000, -19.450000) {};
\node[Kanji] at (-25.750000, -18.950000) {\textcolor[HTML]{154caa}{働}};
\node[Onyomi] at (-25.700000, -19.350000) {ドウ};
\node[Kunyomi] at (-25.800000, -19.350000) {はたら.*};
\node[Meaning] at (-25.750000, -17.700000) {work};
\node[Square] at (-23.700000, -19.450000) {};
\node[Kanji] at (-23.700000, -18.950000) {\textcolor[HTML]{154caa}{種}};
\node[Onyomi] at (-23.650000, -19.350000) {シュ};
\node[Kunyomi] at (-23.750000, -19.350000) {たね};
\node[Meaning] at (-23.700000, -17.700000) {kind};
\node[Square] at (-21.650000, -19.450000) {};
\node[Kanji] at (-21.650000, -18.950000) {\textcolor[HTML]{14469c}{衝}};
\node[Onyomi] at (-21.600000, -19.350000) {ショウ};
\node[Kunyomi] at (-21.700000, -19.350000) {つ.く};
\node[Meaning] at (-21.650000, -17.700000) {collide};
\node[Square] at (-19.600000, -19.450000) {};
\node[Kanji] at (-19.600000, -18.950000) {\textcolor[HTML]{0e254c}{薫}};
\node[Onyomi] at (-19.550000, -19.350000) {クン};
\node[Kunyomi] at (-19.650000, -19.350000) {かお-る};
\node[Meaning] at (-19.600000, -17.700000) {fragrant};
\node[Square] at (-17.550000, -19.450000) {};
\node[Kanji] at (-17.550000, -18.950000) {\textcolor[HTML]{1557c6}{病}};
\node[Onyomi] at (-17.500000, -19.350000) {ビョウ};
\node[Kunyomi] at (-17.600000, -19.350000) {や};
\node[Meaning] at (-17.550000, -17.700000) {sick};
\node[Square] at (-15.500000, -19.450000) {};
\node[Kanji] at (-15.500000, -18.950000) {\textcolor[HTML]{102b59}{痴}};
\node[Onyomi] at (-15.450000, -19.350000) {チ};
\node[Kunyomi] at (-15.550000, -19.350000) {おろか};
\node[Meaning] at (-15.500000, -17.700000) {stupid};
\node[Square] at (-13.450000, -19.450000) {};
\node[Kanji] at (-13.450000, -18.950000) {\textcolor[HTML]{102b59}{痘}};
\node[Onyomi] at (-13.400000, -19.350000) {トウ};
\node[Meaning] at (-13.450000, -17.700000) {pox};
\node[Square] at (-11.400000, -19.450000) {};
\node[Kanji] at (-11.400000, -18.950000) {\textcolor[HTML]{154caa}{症}};
\node[Onyomi] at (-11.350000, -19.350000) {ショウ};
\node[Meaning] at (-11.400000, -17.700000) {symptom};
\node[Square] at (-9.350000, -19.450000) {};
\node[Kanji] at (-9.350000, -18.950000) {\textcolor[HTML]{133c80}{疾}};
\node[Onyomi] at (-9.300000, -19.350000) {シツ};
\node[Kunyomi] at (-9.400000, -19.350000) {はや};
\node[Meaning] at (-9.350000, -17.700000) {rapid};
\node[Square] at (-7.300000, -19.450000) {};
\node[Kanji] at (-7.300000, -18.950000) {\textcolor[HTML]{102b59}{痢}};
\node[Onyomi] at (-7.250000, -19.350000) {リ};
\node[Meaning] at (-7.300000, -17.700000) {diarrhea};
\node[Square] at (-5.250000, -19.450000) {};
\node[Kanji] at (-5.250000, -18.950000) {\textcolor[HTML]{14469c}{疲}};
\node[Onyomi] at (-5.200000, -19.350000) {ヒ};
\node[Kunyomi] at (-5.300000, -19.350000) {つか.れる};
\node[Meaning] at (-5.250000, -17.700000) {exhausted};
\node[Square] at (-3.200000, -19.450000) {};
\node[Kanji] at (-3.200000, -18.950000) {\textcolor[HTML]{102b59}{疫}};
\node[Onyomi] at (-3.150000, -19.350000) {エキ};
\node[Meaning] at (-3.200000, -17.700000) {epidemic};
\node[Square] at (-1.150000, -19.450000) {};
\node[Kanji] at (-1.150000, -18.950000) {\textcolor[HTML]{1551b8}{痛}};
\node[Onyomi] at (-1.100000, -19.350000) {ツウ};
\node[Kunyomi] at (-1.200000, -19.350000) {いた.い};
\node[Meaning] at (-1.150000, -17.700000) {pain};
\node[Square] at (0.900000, -19.450000) {};
\node[Kanji] at (0.900000, -18.950000) {\textcolor[HTML]{113066}{癖}};
\node[Onyomi] at (0.950000, -19.350000) {ヘキ};
\node[Kunyomi] at (0.850000, -19.350000) {くせ};
\node[Meaning] at (0.900000, -17.700000) {habit};
\node[Square] at (2.950000, -19.450000) {};
\node[Kanji] at (2.950000, -18.950000) {\textcolor[HTML]{102b59}{匿}};
\node[Onyomi] at (3.000000, -19.350000) {トク};
\node[Kunyomi] at (2.900000, -19.350000) {かくま.う};
\node[Meaning] at (2.950000, -17.700000) {hide};
\node[Square] at (5.000000, -19.450000) {};
\node[Kanji] at (5.000000, -18.950000) {\textcolor[HTML]{0e254c}{匠}};
\node[Onyomi] at (5.050000, -19.350000) {ショウ};
\node[Kunyomi] at (4.950000, -19.350000) {たくみ};
\node[Meaning] at (5.000000, -17.700000) {artisan};
\node[Square] at (7.050000, -19.450000) {};
\node[Kanji] at (7.050000, -18.950000) {\textcolor[HTML]{14469c}{医}};
\node[Onyomi] at (7.100000, -19.350000) {イ};
\node[Meaning] at (7.050000, -17.700000) {medicine};
\node[Square] at (9.100000, -19.450000) {};
\node[Kanji] at (9.100000, -18.950000) {\textcolor[HTML]{14469c}{匹}};
\node[Kunyomi] at (9.050000, -19.350000) {ひき};
\node[Meaning] at (9.100000, -17.700000) {small animal};
\node[Square] at (11.150000, -19.450000) {};
\node[Kanji] at (11.150000, -18.950000) {\textcolor[HTML]{154caa}{区}};
\node[Onyomi] at (11.200000, -19.350000) {ク};
\node[Meaning] at (11.150000, -17.700000) {district};
\node[Square] at (13.200000, -19.450000) {};
\node[Kanji] at (13.200000, -18.950000) {\textcolor[HTML]{0e254c}{枢}};
\node[Onyomi] at (13.250000, -19.350000) {スウ};
\node[Kunyomi] at (13.150000, -19.350000) {からくり};
\node[Meaning] at (13.200000, -17.700000) {hinge};
\node[Square] at (15.250000, -19.450000) {};
\node[Kanji] at (15.250000, -18.950000) {\textcolor[HTML]{133c80}{殴}};
\node[Onyomi] at (15.300000, -19.350000) {オウ};
\node[Kunyomi] at (15.200000, -19.350000) {なぐ.る};
\node[Meaning] at (15.250000, -17.700000) {assault};
\node[Square] at (17.300000, -19.450000) {};
\node[Kanji] at (17.300000, -18.950000) {\textcolor[HTML]{102b59}{欧}};
\node[Onyomi] at (17.350000, -19.350000) {オウ};
\node[Meaning] at (17.300000, -17.700000) {europe};
\node[Square] at (19.350000, -19.450000) {};
\node[Kanji] at (19.350000, -18.950000) {\textcolor[HTML]{14418e}{抑}};
\node[Onyomi] at (19.400000, -19.350000) {ヨク};
\node[Kunyomi] at (19.300000, -19.350000) {おさ.える};
\node[Meaning] at (19.350000, -17.700000) {suppress};
\node[Square] at (21.400000, -19.450000) {};
\node[Kanji] at (21.400000, -18.950000) {\textcolor[HTML]{14418e}{仰}};
\node[Onyomi] at (21.450000, -19.350000) {ギョウ};
\node[Kunyomi] at (21.350000, -19.350000) {あお.ぐ};
\node[Meaning] at (21.400000, -17.700000) {look up to};
\node[Square] at (23.450000, -19.450000) {};
\node[Kanji] at (23.450000, -18.950000) {\textcolor[HTML]{14469c}{迎}};
\node[Onyomi] at (23.500000, -19.350000) {ゲイ};
\node[Kunyomi] at (23.400000, -19.350000) {むか.える};
\node[Meaning] at (23.450000, -17.700000) {welcome};
\node[Square] at (25.500000, -19.450000) {};
\node[Kanji] at (25.500000, -18.950000) {\textcolor[HTML]{1551b8}{登}};
\node[Onyomi] at (25.550000, -19.350000) {トウ};
\node[Kunyomi] at (25.450000, -19.350000) {のぼ.る};
\node[Meaning] at (25.500000, -17.700000) {climb};
\node[Square] at (27.550000, -19.450000) {};
\node[Kanji] at (27.550000, -18.950000) {\textcolor[HTML]{133c80}{澄}};
\node[Onyomi] at (27.600000, -19.350000) {チョウ};
\node[Kunyomi] at (27.500000, -19.350000) {す.*};
\node[Meaning] at (27.550000, -17.700000) {lucidity};
\node[Square] at (29.600000, -19.450000) {};
\node[Kanji] at (29.600000, -18.950000) {\textcolor[HTML]{145cd5}{発}};
\node[Onyomi] at (29.650000, -19.350000) {ハツ};
\node[Meaning] at (29.600000, -17.700000) {departure};
\node[Square] at (31.650000, -19.450000) {};
\node[Kanji] at (31.650000, -18.950000) {\textcolor[HTML]{123673}{廃}};
\node[Onyomi] at (31.700000, -19.350000) {ハイ};
\node[Kunyomi] at (31.600000, -19.350000) {すた};
\node[Meaning] at (31.650000, -17.700000) {obsolete};
\node[Square] at (33.700000, -19.450000) {};
\node[Kanji] at (33.700000, -18.950000) {\textcolor[HTML]{113066}{僚}};
\node[Onyomi] at (33.750000, -19.350000) {リョウ};
\node[Meaning] at (33.700000, -17.700000) {colleague};
\node[Square] at (35.750000, -19.450000) {};
\node[Kanji] at (35.750000, -18.950000) {\textcolor[HTML]{154caa}{寮}};
\node[Onyomi] at (35.800000, -19.350000) {リョウ};
\node[Meaning] at (35.750000, -17.700000) {dormitory};
\node[Square] at (37.800000, -19.450000) {};
\node[Kanji] at (37.800000, -18.950000) {\textcolor[HTML]{14418e}{療}};
\node[Onyomi] at (37.850000, -19.350000) {リョウ};
\node[Meaning] at (37.800000, -17.700000) {heal};
\node[Square] at (39.850000, -19.450000) {};
\node[Kanji] at (39.850000, -18.950000) {\textcolor[HTML]{123673}{彫}};
\node[Onyomi] at (39.900000, -19.350000) {チョウ};
\node[Kunyomi] at (39.800000, -19.350000) {ほ.る};
\node[Meaning] at (39.850000, -17.700000) {carve};
\node[Square] at (41.900000, -19.450000) {};
\node[Kanji] at (41.900000, -18.950000) {\textcolor[HTML]{1551b8}{形}};
\node[Onyomi] at (41.950000, -19.350000) {ケイ};
\node[Kunyomi] at (41.850000, -19.350000) {かた};
\node[Meaning] at (41.900000, -17.700000) {shape};
\node[Square] at (43.950000, -19.450000) {};
\node[Kanji] at (43.950000, -18.950000) {\textcolor[HTML]{154caa}{影}};
\node[Onyomi] at (44.000000, -19.350000) {エイ};
\node[Kunyomi] at (43.900000, -19.350000) {かげ};
\node[Meaning] at (43.950000, -17.700000) {shadow};
\node[Square] at (46.000000, -19.450000) {};
\node[Kanji] at (46.000000, -18.950000) {\textcolor[HTML]{113066}{杉}};
\node[Kunyomi] at (45.950000, -19.350000) {すぎ};
\node[Meaning] at (46.000000, -17.700000) {cedar};
\node[Square] at (48.050000, -19.450000) {};
\node[Kanji] at (48.050000, -18.950000) {\textcolor[HTML]{102b59}{彩}};
\node[Onyomi] at (48.100000, -19.350000) {サイ};
\node[Kunyomi] at (48.000000, -19.350000) {いろど.る};
\node[Meaning] at (48.050000, -17.700000) {coloring};
\node[Square] at (50.100000, -19.450000) {};
\node[Kanji] at (50.100000, -18.950000) {\textcolor[HTML]{102b59}{彰}};
\node[Onyomi] at (50.150000, -19.350000) {ショウ};
\node[Meaning] at (50.100000, -17.700000) {patent};
\node[Square] at (52.150000, -19.450000) {};
\node[Kanji] at (52.150000, -18.950000) {\textcolor[HTML]{1461e3}{顔}};
\node[Onyomi] at (52.200000, -19.350000) {ガン};
\node[Kunyomi] at (52.100000, -19.350000) {かお};
\node[Meaning] at (52.150000, -17.700000) {face};
\node[Square] at (54.200000, -19.450000) {};
\node[Kanji] at (54.200000, -18.950000) {\textcolor[HTML]{113066}{須}};
\node[Onyomi] at (54.250000, -19.350000) {ス};
\node[Kunyomi] at (54.150000, -19.350000) {すべから};
\node[Meaning] at (54.200000, -17.700000) {necessary};
\node[Square] at (56.250000, -19.450000) {};
\node[Kanji] at (56.250000, -18.950000) {\textcolor[HTML]{14418e}{膨}};
\node[Onyomi] at (56.300000, -19.350000) {ボウ};
\node[Kunyomi] at (56.200000, -19.350000) {ふく};
\node[Meaning] at (56.250000, -17.700000) {swell};
\node[Meaning] at (-58.500000, -18.900000) {86.12\%};
\node[Square] at (-56.500000, -21.500000) {};
\node[Kanji] at (-56.500000, -21.000000) {\textcolor[HTML]{154caa}{参}};
\node[Onyomi] at (-56.450000, -21.400000) {サン};
\node[Kunyomi] at (-56.550000, -21.400000) {まい.る};
\node[Meaning] at (-56.500000, -19.750000) {participate};
\node[Square] at (-54.450000, -21.500000) {};
\node[Kanji] at (-54.450000, -21.000000) {\textcolor[HTML]{14418e}{惨}};
\node[Onyomi] at (-54.400000, -21.400000) {サン};
\node[Kunyomi] at (-54.500000, -21.400000) {みじ};
\node[Meaning] at (-54.450000, -19.750000) {disaster};
\node[Square] at (-52.400000, -21.500000) {};
\node[Kanji] at (-52.400000, -21.000000) {\textcolor[HTML]{14418e}{修}};
\node[Onyomi] at (-52.350000, -21.400000) {シュウ};
\node[Kunyomi] at (-52.450000, -21.400000) {おさ.まる};
\node[Meaning] at (-52.400000, -19.750000) {discipline};
\node[Square] at (-50.350000, -21.500000) {};
\node[Kanji] at (-50.350000, -21.000000) {\textcolor[HTML]{133c80}{珍}};
\node[Onyomi] at (-50.300000, -21.400000) {チン};
\node[Kunyomi] at (-50.400000, -21.400000) {めずら.しい};
\node[Meaning] at (-50.350000, -19.750000) {rare};
\node[Square] at (-48.300000, -21.500000) {};
\node[Kanji] at (-48.300000, -21.000000) {\textcolor[HTML]{123673}{診}};
\node[Onyomi] at (-48.250000, -21.400000) {シン};
\node[Kunyomi] at (-48.350000, -21.400000) {み.る};
\node[Meaning] at (-48.300000, -19.750000) {diagnose};
\node[Square] at (-46.250000, -21.500000) {};
\node[Kanji] at (-46.250000, -21.000000) {\textcolor[HTML]{145cd5}{文}};
\node[Onyomi] at (-46.200000, -21.400000) {ブン};
\node[Meaning] at (-46.250000, -19.750000) {writing};
\node[Square] at (-44.200000, -21.500000) {};
\node[Kanji] at (-44.200000, -21.000000) {\textcolor[HTML]{145cd5}{対}};
\node[Onyomi] at (-44.150000, -21.400000) {タイ};
\node[Meaning] at (-44.200000, -19.750000) {versus};
\node[Square] at (-42.150000, -21.500000) {};
\node[Kanji] at (-42.150000, -21.000000) {\textcolor[HTML]{123673}{紋}};
\node[Onyomi] at (-42.100000, -21.400000) {モン};
\node[Meaning] at (-42.150000, -19.750000) {family crest};
\node[Square] at (-40.100000, -21.500000) {};
\node[Kanji] at (-40.100000, -21.000000) {\textcolor[HTML]{123673}{蚊}};
\node[Kunyomi] at (-40.150000, -21.400000) {か};
\node[Meaning] at (-40.100000, -19.750000) {mosquito};
\node[Square] at (-38.050000, -21.500000) {};
\node[Kanji] at (-38.050000, -21.000000) {\textcolor[HTML]{14418e}{斉}};
\node[Onyomi] at (-38.000000, -21.400000) {セイ};
\node[Meaning] at (-38.050000, -19.750000) {simultaneous};
\node[Square] at (-36.000000, -21.500000) {};
\node[Kanji] at (-36.000000, -21.000000) {\textcolor[HTML]{14418e}{剤}};
\node[Onyomi] at (-35.950000, -21.400000) {ザイ};
\node[Meaning] at (-36.000000, -19.750000) {dose};
\node[Square] at (-33.950000, -21.500000) {};
\node[Kanji] at (-33.950000, -21.000000) {\textcolor[HTML]{14469c}{済}};
\node[Onyomi] at (-33.900000, -21.400000) {サイ};
\node[Kunyomi] at (-34.000000, -21.400000) {す.ます};
\node[Meaning] at (-33.950000, -19.750000) {come to an end};
\node[Square] at (-31.900000, -21.500000) {};
\node[Kanji] at (-31.900000, -21.000000) {\textcolor[HTML]{0e254c}{斎}};
\node[Onyomi] at (-31.850000, -21.400000) {サイ};
\node[Kunyomi] at (-31.950000, -21.400000) {いつ.く};
\node[Meaning] at (-31.900000, -19.750000) {purification};
\node[Square] at (-29.850000, -21.500000) {};
\node[Kanji] at (-29.850000, -21.000000) {\textcolor[HTML]{102b59}{粛}};
\node[Onyomi] at (-29.800000, -21.400000) {シュク};
\node[Kunyomi] at (-29.900000, -21.400000) {つつし};
\node[Meaning] at (-29.850000, -19.750000) {solemn};
\node[Square] at (-27.800000, -21.500000) {};
\node[Kanji] at (-27.800000, -21.000000) {\textcolor[HTML]{0e254c}{塁}};
\node[Onyomi] at (-27.750000, -21.400000) {ルイ};
\node[Meaning] at (-27.800000, -19.750000) {base};
\node[Square] at (-25.750000, -21.500000) {};
\node[Kanji] at (-25.750000, -21.000000) {\textcolor[HTML]{1557c6}{楽}};
\node[Onyomi] at (-25.700000, -21.400000) {ガク};
\node[Kunyomi] at (-25.800000, -21.400000) {たの.しい};
\node[Meaning] at (-25.750000, -19.750000) {comfort};
\node[Square] at (-23.700000, -21.500000) {};
\node[Kanji] at (-23.700000, -21.000000) {\textcolor[HTML]{1551b8}{薬}};
\node[Onyomi] at (-23.650000, -21.400000) {ヤク};
\node[Kunyomi] at (-23.750000, -21.400000) {くすり};
\node[Meaning] at (-23.700000, -19.750000) {medicine};
\node[Square] at (-21.650000, -21.500000) {};
\node[Kanji] at (-21.650000, -21.000000) {\textcolor[HTML]{133c80}{率}};
\node[Onyomi] at (-21.600000, -21.400000) {リツ};
\node[Kunyomi] at (-21.700000, -21.400000) {ひき.いる};
\node[Meaning] at (-21.650000, -19.750000) {percent};
\node[Square] at (-19.600000, -21.500000) {};
\node[Kanji] at (-19.600000, -21.000000) {\textcolor[HTML]{133c80}{渋}};
\node[Onyomi] at (-19.550000, -21.400000) {ジュウ};
\node[Kunyomi] at (-19.650000, -21.400000) {しぶ.い};
\node[Meaning] at (-19.600000, -19.750000) {bitter};
\node[Square] at (-17.550000, -21.500000) {};
\node[Kanji] at (-17.550000, -21.000000) {\textcolor[HTML]{0e254c}{摂}};
\node[Onyomi] at (-17.500000, -21.400000) {セツ};
\node[Kunyomi] at (-17.600000, -21.400000) {おさ};
\node[Meaning] at (-17.550000, -19.750000) {in addition};
\node[Square] at (-15.500000, -21.500000) {};
\node[Kanji] at (-15.500000, -21.000000) {\textcolor[HTML]{14418e}{央}};
\node[Onyomi] at (-15.450000, -21.400000) {オウ};
\node[Meaning] at (-15.500000, -19.750000) {central};
\node[Square] at (-13.450000, -21.500000) {};
\node[Kanji] at (-13.450000, -21.000000) {\textcolor[HTML]{154caa}{英}};
\node[Onyomi] at (-13.400000, -21.400000) {エイ};
\node[Meaning] at (-13.450000, -19.750000) {england};
\node[Square] at (-11.400000, -21.500000) {};
\node[Kanji] at (-11.400000, -21.000000) {\textcolor[HTML]{1551b8}{映}};
\node[Onyomi] at (-11.350000, -21.400000) {エイ};
\node[Kunyomi] at (-11.450000, -21.400000) {うつ};
\node[Meaning] at (-11.400000, -19.750000) {reflect};
\node[Square] at (-9.350000, -21.500000) {};
\node[Kanji] at (-9.350000, -21.000000) {\textcolor[HTML]{1557c6}{赤}};
\node[Onyomi] at (-9.300000, -21.400000) {セキ};
\node[Kunyomi] at (-9.400000, -21.400000) {あか};
\node[Meaning] at (-9.350000, -19.750000) {red};
\node[Square] at (-7.300000, -21.500000) {};
\node[Kanji] at (-7.300000, -21.000000) {\textcolor[HTML]{113066}{赦}};
\node[Onyomi] at (-7.250000, -21.400000) {シャ};
\node[Meaning] at (-7.300000, -19.750000) {pardon};
\node[Square] at (-5.250000, -21.500000) {};
\node[Kanji] at (-5.250000, -21.000000) {\textcolor[HTML]{145cd5}{変}};
\node[Onyomi] at (-5.200000, -21.400000) {ヘン};
\node[Kunyomi] at (-5.300000, -21.400000) {か.*};
\node[Meaning] at (-5.250000, -19.750000) {change};
\node[Square] at (-3.200000, -21.500000) {};
\node[Kanji] at (-3.200000, -21.000000) {\textcolor[HTML]{14469c}{跡}};
\node[Onyomi] at (-3.150000, -21.400000) {セキ};
\node[Kunyomi] at (-3.250000, -21.400000) {あと};
\node[Meaning] at (-3.200000, -19.750000) {traces};
\node[Square] at (-1.150000, -21.500000) {};
\node[Kanji] at (-1.150000, -21.000000) {\textcolor[HTML]{102b59}{蛮}};
\node[Onyomi] at (-1.100000, -21.400000) {バン};
\node[Kunyomi] at (-1.200000, -21.400000) {えびす};
\node[Meaning] at (-1.150000, -19.750000) {barbarian};
\node[Square] at (0.900000, -21.500000) {};
\node[Kanji] at (0.900000, -21.000000) {\textcolor[HTML]{14469c}{恋}};
\node[Onyomi] at (0.950000, -21.400000) {レン};
\node[Kunyomi] at (0.850000, -21.400000) {こい};
\node[Meaning] at (0.900000, -19.750000) {romance};
\node[Square] at (2.950000, -21.500000) {};
\node[Kanji] at (2.950000, -21.000000) {\textcolor[HTML]{14418e}{湾}};
\node[Onyomi] at (3.000000, -21.400000) {ワン};
\node[Meaning] at (2.950000, -19.750000) {gulf};
\node[Square] at (5.000000, -21.500000) {};
\node[Kanji] at (5.000000, -21.000000) {\textcolor[HTML]{14469c}{黄}};
\node[Onyomi] at (5.050000, -21.400000) {オウ};
\node[Kunyomi] at (4.950000, -21.400000) {き};
\node[Meaning] at (5.000000, -19.750000) {yellow};
\node[Square] at (7.050000, -21.500000) {};
\node[Kanji] at (7.050000, -21.000000) {\textcolor[HTML]{1557c6}{横}};
\node[Onyomi] at (7.100000, -21.400000) {オウ};
\node[Kunyomi] at (7.000000, -21.400000) {よこ};
\node[Meaning] at (7.050000, -19.750000) {side};
\node[Square] at (9.100000, -21.500000) {};
\node[Kanji] at (9.100000, -21.000000) {\textcolor[HTML]{0e254c}{把}};
\node[Onyomi] at (9.150000, -21.400000) {ワ};
\node[Meaning] at (9.100000, -19.750000) {bundle};
\node[Square] at (11.150000, -21.500000) {};
\node[Kanji] at (11.150000, -21.000000) {\textcolor[HTML]{145cd5}{色}};
\node[Onyomi] at (11.200000, -21.400000) {シキ};
\node[Kunyomi] at (11.100000, -21.400000) {いろ};
\node[Meaning] at (11.150000, -19.750000) {color};
\node[Square] at (13.200000, -21.500000) {};
\node[Kanji] at (13.200000, -21.000000) {\textcolor[HTML]{1551b8}{絶}};
\node[Onyomi] at (13.250000, -21.400000) {ゼツ};
\node[Kunyomi] at (13.150000, -21.400000) {た.*};
\node[Meaning] at (13.200000, -19.750000) {extinction};
\node[Square] at (15.250000, -21.500000) {};
\node[Kanji] at (15.250000, -21.000000) {\textcolor[HTML]{113066}{艶}};
\node[Onyomi] at (15.300000, -21.400000) {エン};
\node[Kunyomi] at (15.200000, -21.400000) {つや};
\node[Meaning] at (15.250000, -19.750000) {glossy};
\node[Square] at (17.300000, -21.500000) {};
\node[Kanji] at (17.300000, -21.000000) {\textcolor[HTML]{113066}{肥}};
\node[Onyomi] at (17.350000, -21.400000) {ヒ};
\node[Kunyomi] at (17.250000, -21.400000) {こ.える};
\node[Meaning] at (17.300000, -19.750000) {obese};
\node[Square] at (19.350000, -21.500000) {};
\node[Kanji] at (19.350000, -21.000000) {\textcolor[HTML]{14418e}{甘}};
\node[Onyomi] at (19.400000, -21.400000) {カン};
\node[Kunyomi] at (19.300000, -21.400000) {あま};
\node[Meaning] at (19.350000, -19.750000) {sweet};
\node[Square] at (21.400000, -21.500000) {};
\node[Kanji] at (21.400000, -21.000000) {\textcolor[HTML]{102b59}{紺}};
\node[Onyomi] at (21.450000, -21.400000) {コン};
\node[Meaning] at (21.400000, -19.750000) {navy};
\node[Square] at (23.450000, -21.500000) {};
\node[Kanji] at (23.450000, -21.000000) {\textcolor[HTML]{0e254c}{某}};
\node[Onyomi] at (23.500000, -21.400000) {ボウ};
\node[Kunyomi] at (23.400000, -21.400000) {それがし};
\node[Meaning] at (23.450000, -19.750000) {certain};
\node[Square] at (25.500000, -21.500000) {};
\node[Kanji] at (25.500000, -21.000000) {\textcolor[HTML]{102b59}{謀}};
\node[Onyomi] at (25.550000, -21.400000) {ボウ};
\node[Kunyomi] at (25.450000, -21.400000) {はか.る};
\node[Meaning] at (25.500000, -19.750000) {conspire};
\node[Square] at (27.550000, -21.500000) {};
\node[Kanji] at (27.550000, -21.000000) {\textcolor[HTML]{0e254c}{媒}};
\node[Onyomi] at (27.600000, -21.400000) {バイ};
\node[Kunyomi] at (27.500000, -21.400000) {なこうど};
\node[Meaning] at (27.550000, -19.750000) {mediator};
\node[Square] at (29.600000, -21.500000) {};
\node[Kanji] at (29.600000, -21.000000) {\textcolor[HTML]{102b59}{欺}};
\node[Onyomi] at (29.650000, -21.400000) {ギ};
\node[Kunyomi] at (29.550000, -21.400000) {あざむ.く};
\node[Meaning] at (29.600000, -19.750000) {deceit};
\node[Square] at (31.650000, -21.500000) {};
\node[Kanji] at (31.650000, -21.000000) {\textcolor[HTML]{113066}{棋}};
\node[Onyomi] at (31.700000, -21.400000) {キ};
\node[Kunyomi] at (31.600000, -21.400000) {ご};
\node[Meaning] at (31.650000, -19.750000) {chess piece};
\node[Square] at (33.700000, -21.500000) {};
\node[Kanji] at (33.700000, -21.000000) {\textcolor[HTML]{14418e}{旗}};
\node[Onyomi] at (33.750000, -21.400000) {キ};
\node[Kunyomi] at (33.650000, -21.400000) {はた};
\node[Meaning] at (33.700000, -19.750000) {flag};
\node[Square] at (35.750000, -21.500000) {};
\node[Kanji] at (35.750000, -21.000000) {\textcolor[HTML]{1551b8}{期}};
\node[Onyomi] at (35.800000, -21.400000) {キ};
\node[Meaning] at (35.750000, -19.750000) {period of time};
\node[Square] at (37.800000, -21.500000) {};
\node[Kanji] at (37.800000, -21.000000) {\textcolor[HTML]{102b59}{碁}};
\node[Onyomi] at (37.850000, -21.400000) {ゴ};
\node[Meaning] at (37.800000, -19.750000) {go};
\node[Square] at (39.850000, -21.500000) {};
\node[Kanji] at (39.850000, -21.000000) {\textcolor[HTML]{14469c}{基}};
\node[Onyomi] at (39.900000, -21.400000) {キ};
\node[Kunyomi] at (39.800000, -21.400000) {もと};
\node[Meaning] at (39.850000, -19.750000) {foundation};
\node[Square] at (41.900000, -21.500000) {};
\node[Kanji] at (41.900000, -21.000000) {\textcolor[HTML]{0e254c}{甚}};
\node[Onyomi] at (41.950000, -21.400000) {ジン};
\node[Kunyomi] at (41.850000, -21.400000) {はなは};
\node[Meaning] at (41.900000, -19.750000) {very};
\node[Square] at (43.950000, -21.500000) {};
\node[Kanji] at (43.950000, -21.000000) {\textcolor[HTML]{133c80}{勘}};
\node[Onyomi] at (44.000000, -21.400000) {カン};
\node[Meaning] at (43.950000, -19.750000) {intuition};
\node[Square] at (46.000000, -21.500000) {};
\node[Kanji] at (46.000000, -21.000000) {\textcolor[HTML]{123673}{堪}};
\node[Onyomi] at (46.050000, -21.400000) {カン};
\node[Kunyomi] at (45.950000, -21.400000) {た};
\node[Meaning] at (46.000000, -19.750000) {endure};
\node[Square] at (48.050000, -21.500000) {};
\node[Kanji] at (48.050000, -21.000000) {\textcolor[HTML]{14418e}{貴}};
\node[Onyomi] at (48.100000, -21.400000) {キ};
\node[Kunyomi] at (48.000000, -21.400000) {とうと.い};
\node[Meaning] at (48.050000, -19.750000) {valuable};
\node[Square] at (50.100000, -21.500000) {};
\node[Kanji] at (50.100000, -21.000000) {\textcolor[HTML]{154caa}{遺}};
\node[Onyomi] at (50.150000, -21.400000) {イ};
\node[Kunyomi] at (50.050000, -21.400000) {のこ.す};
\node[Meaning] at (50.100000, -19.750000) {leave behind};
\node[Square] at (52.150000, -21.500000) {};
\node[Kanji] at (52.150000, -21.000000) {\textcolor[HTML]{133c80}{遣}};
\node[Onyomi] at (52.200000, -21.400000) {ケン};
\node[Kunyomi] at (52.100000, -21.400000) {つか.う};
\node[Meaning] at (52.150000, -19.750000) {dispatch};
\node[Square] at (54.200000, -21.500000) {};
\node[Kanji] at (54.200000, -21.000000) {\textcolor[HTML]{154caa}{舞}};
\node[Onyomi] at (54.250000, -21.400000) {ブ};
\node[Kunyomi] at (54.150000, -21.400000) {まい};
\node[Meaning] at (54.200000, -19.750000) {dance};
\node[Square] at (56.250000, -21.500000) {};
\node[Kanji] at (56.250000, -21.000000) {\textcolor[HTML]{1557c6}{無}};
\node[Onyomi] at (56.300000, -21.400000) {ム};
\node[Kunyomi] at (56.200000, -21.400000) {な.い};
\node[Meaning] at (56.250000, -19.750000) {nothing};
\node[Meaning] at (-58.500000, -20.950000) {87.99\%};
\node[Square] at (-56.500000, -23.550000) {};
\node[Kanji] at (-56.500000, -23.050000) {\textcolor[HTML]{1557c6}{組}};
\node[Onyomi] at (-56.450000, -23.450000) {ソ};
\node[Kunyomi] at (-56.550000, -23.450000) {くみ};
\node[Meaning] at (-56.500000, -21.800000) {group};
\node[Square] at (-54.450000, -23.550000) {};
\node[Kanji] at (-54.450000, -23.050000) {\textcolor[HTML]{113066}{粗}};
\node[Onyomi] at (-54.400000, -23.450000) {ソ};
\node[Kunyomi] at (-54.500000, -23.450000) {あら};
\node[Meaning] at (-54.450000, -21.800000) {coarse};
\node[Square] at (-52.400000, -23.550000) {};
\node[Kanji] at (-52.400000, -23.050000) {\textcolor[HTML]{0e254c}{租}};
\node[Onyomi] at (-52.350000, -23.450000) {ソ};
\node[Meaning] at (-52.400000, -21.800000) {tariff};
\node[Square] at (-50.350000, -23.550000) {};
\node[Kanji] at (-50.350000, -23.050000) {\textcolor[HTML]{14418e}{祖}};
\node[Onyomi] at (-50.300000, -23.450000) {ソ};
\node[Meaning] at (-50.350000, -21.800000) {ancestor};
\node[Square] at (-48.300000, -23.550000) {};
\node[Kanji] at (-48.300000, -23.050000) {\textcolor[HTML]{123673}{阻}};
\node[Onyomi] at (-48.250000, -23.450000) {ソ};
\node[Kunyomi] at (-48.350000, -23.450000) {はば.む};
\node[Meaning] at (-48.300000, -21.800000) {thwart};
\node[Square] at (-46.250000, -23.550000) {};
\node[Kanji] at (-46.250000, -23.050000) {\textcolor[HTML]{1551b8}{査}};
\node[Onyomi] at (-46.200000, -23.450000) {サ};
\node[Meaning] at (-46.250000, -21.800000) {inspect};
\node[Square] at (-44.200000, -23.550000) {};
\node[Kanji] at (-44.200000, -23.050000) {\textcolor[HTML]{1551b8}{助}};
\node[Onyomi] at (-44.150000, -23.450000) {ジョ};
\node[Kunyomi] at (-44.250000, -23.450000) {たす};
\node[Meaning] at (-44.200000, -21.800000) {help};
\node[Square] at (-42.150000, -23.550000) {};
\node[Kanji] at (-42.150000, -23.050000) {\textcolor[HTML]{102b59}{宜}};
\node[Onyomi] at (-42.100000, -23.450000) {ギ};
\node[Kunyomi] at (-42.200000, -23.450000) {よろ};
\node[Meaning] at (-42.150000, -21.800000) {best regards};
\node[Square] at (-40.100000, -23.550000) {};
\node[Kanji] at (-40.100000, -23.050000) {\textcolor[HTML]{123673}{畳}};
\node[Onyomi] at (-40.050000, -23.450000) {ジョウ};
\node[Kunyomi] at (-40.150000, -23.450000) {たたみ};
\node[Meaning] at (-40.100000, -21.800000) {tatami mat};
\node[Square] at (-38.050000, -23.550000) {};
\node[Kanji] at (-38.050000, -23.050000) {\textcolor[HTML]{154caa}{並}};
\node[Onyomi] at (-38.000000, -23.450000) {ヘイ};
\node[Kunyomi] at (-38.100000, -23.450000) {なら.*};
\node[Meaning] at (-38.050000, -21.800000) {line up};
\node[Square] at (-36.000000, -23.550000) {};
\node[Kanji] at (-36.000000, -23.050000) {\textcolor[HTML]{154caa}{普}};
\node[Onyomi] at (-35.950000, -23.450000) {フ};
\node[Meaning] at (-36.000000, -21.800000) {normal};
\node[Square] at (-33.950000, -23.550000) {};
\node[Kanji] at (-33.950000, -23.050000) {\textcolor[HTML]{0e254c}{譜}};
\node[Onyomi] at (-33.900000, -23.450000) {フ};
\node[Meaning] at (-33.950000, -21.800000) {genealogy};
\node[Square] at (-31.900000, -23.550000) {};
\node[Kanji] at (-31.900000, -23.050000) {\textcolor[HTML]{133c80}{湿}};
\node[Onyomi] at (-31.850000, -23.450000) {シツ};
\node[Kunyomi] at (-31.950000, -23.450000) {しめ.らせる};
\node[Meaning] at (-31.900000, -21.800000) {damp};
\node[Square] at (-29.850000, -23.550000) {};
\node[Kanji] at (-29.850000, -23.050000) {\textcolor[HTML]{113066}{顕}};
\node[Onyomi] at (-29.800000, -23.450000) {ケン};
\node[Kunyomi] at (-29.900000, -23.450000) {あきらか};
\node[Meaning] at (-29.850000, -21.800000) {appear};
\node[Square] at (-27.800000, -23.550000) {};
\node[Kanji] at (-27.800000, -23.050000) {\textcolor[HTML]{113066}{繊}};
\node[Onyomi] at (-27.750000, -23.450000) {セン};
\node[Meaning] at (-27.800000, -21.800000) {fiber};
\node[Square] at (-25.750000, -23.550000) {};
\node[Kanji] at (-25.750000, -23.050000) {\textcolor[HTML]{154caa}{霊}};
\node[Onyomi] at (-25.700000, -23.450000) {レイ};
\node[Meaning] at (-25.750000, -21.800000) {ghost};
\node[Square] at (-23.700000, -23.550000) {};
\node[Kanji] at (-23.700000, -23.050000) {\textcolor[HTML]{1557c6}{業}};
\node[Onyomi] at (-23.650000, -23.450000) {ギョウ};
\node[Meaning] at (-23.700000, -21.800000) {business};
\node[Square] at (-21.650000, -23.550000) {};
\node[Kanji] at (-21.650000, -23.050000) {\textcolor[HTML]{113066}{撲}};
\node[Onyomi] at (-21.600000, -23.450000) {ボク};
\node[Meaning] at (-21.650000, -21.800000) {slap};
\node[Square] at (-19.600000, -23.550000) {};
\node[Kanji] at (-19.600000, -23.050000) {\textcolor[HTML]{1968ed}{僕}};
\node[Onyomi] at (-19.550000, -23.450000) {ボク};
\node[Meaning] at (-19.600000, -21.800000) {i};
\node[Square] at (-17.550000, -23.550000) {};
\node[Kanji] at (-17.550000, -23.050000) {\textcolor[HTML]{133c80}{共}};
\node[Onyomi] at (-17.500000, -23.450000) {キョウ};
\node[Kunyomi] at (-17.600000, -23.450000) {とも};
\node[Meaning] at (-17.550000, -21.800000) {together};
\node[Square] at (-15.500000, -23.550000) {};
\node[Kanji] at (-15.500000, -23.050000) {\textcolor[HTML]{154caa}{供}};
\node[Onyomi] at (-15.450000, -23.450000) {キョウ};
\node[Kunyomi] at (-15.550000, -23.450000) {とも};
\node[Meaning] at (-15.500000, -21.800000) {servant};
\node[Square] at (-13.450000, -23.550000) {};
\node[Kanji] at (-13.450000, -23.050000) {\textcolor[HTML]{14418e}{異}};
\node[Onyomi] at (-13.400000, -23.450000) {イ};
\node[Kunyomi] at (-13.500000, -23.450000) {こと.*};
\node[Meaning] at (-13.450000, -21.800000) {differ};
\node[Square] at (-11.400000, -23.550000) {};
\node[Kanji] at (-11.400000, -23.050000) {\textcolor[HTML]{14418e}{翼}};
\node[Onyomi] at (-11.350000, -23.450000) {ヨク};
\node[Kunyomi] at (-11.450000, -23.450000) {つばさ};
\node[Meaning] at (-11.400000, -21.800000) {wing};
\node[Square] at (-9.350000, -23.550000) {};
\node[Kanji] at (-9.350000, -23.050000) {\textcolor[HTML]{0e254c}{洪}};
\node[Onyomi] at (-9.300000, -23.450000) {コウ};
\node[Meaning] at (-9.350000, -21.800000) {flood};
\node[Square] at (-7.300000, -23.550000) {};
\node[Kanji] at (-7.300000, -23.050000) {\textcolor[HTML]{154caa}{港}};
\node[Onyomi] at (-7.250000, -23.450000) {コウ};
\node[Kunyomi] at (-7.350000, -23.450000) {みなと};
\node[Meaning] at (-7.300000, -21.800000) {harbor};
\node[Square] at (-5.250000, -23.550000) {};
\node[Kanji] at (-5.250000, -23.050000) {\textcolor[HTML]{14469c}{暴}};
\node[Onyomi] at (-5.200000, -23.450000) {ボウ};
\node[Kunyomi] at (-5.300000, -23.450000) {あば.れる};
\node[Meaning] at (-5.250000, -21.800000) {violence};
\node[Square] at (-3.200000, -23.550000) {};
\node[Kanji] at (-3.200000, -23.050000) {\textcolor[HTML]{154caa}{爆}};
\node[Onyomi] at (-3.150000, -23.450000) {バク};
\node[Kunyomi] at (-3.250000, -23.450000) {は.ぜる};
\node[Meaning] at (-3.200000, -21.800000) {explode};
\node[Square] at (-1.150000, -23.550000) {};
\node[Kanji] at (-1.150000, -23.050000) {\textcolor[HTML]{102b59}{恭}};
\node[Onyomi] at (-1.100000, -23.450000) {キョウ};
\node[Kunyomi] at (-1.200000, -23.450000) {うやうや};
\node[Meaning] at (-1.150000, -21.800000) {respect};
\node[Square] at (0.900000, -23.550000) {};
\node[Kanji] at (0.900000, -23.050000) {\textcolor[HTML]{1557c6}{選}};
\node[Onyomi] at (0.950000, -23.450000) {セン};
\node[Kunyomi] at (0.850000, -23.450000) {えら.ぶ};
\node[Meaning] at (0.900000, -21.800000) {choose};
\node[Square] at (2.950000, -23.550000) {};
\node[Kanji] at (2.950000, -23.050000) {\textcolor[HTML]{133c80}{殿}};
\node[Onyomi] at (3.000000, -23.450000) {デン};
\node[Kunyomi] at (2.900000, -23.450000) {との};
\node[Meaning] at (2.950000, -21.800000) {milord};
\node[Square] at (5.000000, -23.550000) {};
\node[Kanji] at (5.000000, -23.050000) {\textcolor[HTML]{154caa}{井}};
\node[Onyomi] at (5.050000, -23.450000) {ショウ};
\node[Kunyomi] at (4.950000, -23.450000) {い};
\node[Meaning] at (5.000000, -21.800000) {well};
\node[Square] at (7.050000, -23.550000) {};
\node[Kanji] at (7.050000, -23.050000) {\textcolor[HTML]{154caa}{囲}};
\node[Onyomi] at (7.100000, -23.450000) {イ};
\node[Kunyomi] at (7.000000, -23.450000) {かこ.む};
\node[Meaning] at (7.050000, -21.800000) {surround};
\node[Square] at (9.100000, -23.550000) {};
\node[Kanji] at (9.100000, -23.050000) {\textcolor[HTML]{0e254c}{耕}};
\node[Onyomi] at (9.150000, -23.450000) {コウ};
\node[Kunyomi] at (9.050000, -23.450000) {たがや.す};
\node[Meaning] at (9.100000, -21.800000) {plow};
\node[Square] at (11.150000, -23.550000) {};
\node[Kanji] at (11.150000, -23.050000) {\textcolor[HTML]{0e254c}{亜}};
\node[Onyomi] at (11.200000, -23.450000) {ア};
\node[Kunyomi] at (11.100000, -23.450000) {つ};
\node[Meaning] at (11.150000, -21.800000) {asia};
\node[Square] at (13.200000, -23.550000) {};
\node[Kanji] at (13.200000, -23.050000) {\textcolor[HTML]{1557c6}{悪}};
\node[Onyomi] at (13.250000, -23.450000) {アク};
\node[Kunyomi] at (13.150000, -23.450000) {わる.い};
\node[Meaning] at (13.200000, -21.800000) {bad};
\node[Square] at (15.250000, -23.550000) {};
\node[Kanji] at (15.250000, -23.050000) {\textcolor[HTML]{1551b8}{円}};
\node[Onyomi] at (15.300000, -23.450000) {エン};
\node[Kunyomi] at (15.200000, -23.450000) {まる.い};
\node[Meaning] at (15.250000, -21.800000) {yen};
\node[Square] at (17.300000, -23.550000) {};
\node[Kanji] at (17.300000, -23.050000) {\textcolor[HTML]{154caa}{角}};
\node[Onyomi] at (17.350000, -23.450000) {カク};
\node[Kunyomi] at (17.250000, -23.450000) {かど};
\node[Meaning] at (17.300000, -21.800000) {angle};
\node[Square] at (19.350000, -23.550000) {};
\node[Kanji] at (19.350000, -23.050000) {\textcolor[HTML]{154caa}{触}};
\node[Onyomi] at (19.400000, -23.450000) {ショク};
\node[Kunyomi] at (19.300000, -23.450000) {さわ.る};
\node[Meaning] at (19.350000, -21.800000) {touch};
\node[Square] at (21.400000, -23.550000) {};
\node[Kanji] at (21.400000, -23.050000) {\textcolor[HTML]{1551b8}{解}};
\node[Onyomi] at (21.450000, -23.450000) {カイ};
\node[Kunyomi] at (21.350000, -23.450000) {と.く};
\node[Meaning] at (21.400000, -21.800000) {untie};
\node[Square] at (23.450000, -23.550000) {};
\node[Kanji] at (23.450000, -23.050000) {\textcolor[HTML]{1551b8}{再}};
\node[Onyomi] at (23.500000, -23.450000) {サ};
\node[Kunyomi] at (23.400000, -23.450000) {ふたた.び};
\node[Meaning] at (23.450000, -21.800000) {again};
\node[Square] at (25.500000, -23.550000) {};
\node[Kanji] at (25.500000, -23.050000) {\textcolor[HTML]{133c80}{講}};
\node[Onyomi] at (25.550000, -23.450000) {コウ};
\node[Meaning] at (25.500000, -21.800000) {lecture};
\node[Square] at (27.550000, -23.550000) {};
\node[Kanji] at (27.550000, -23.050000) {\textcolor[HTML]{102b59}{購}};
\node[Onyomi] at (27.600000, -23.450000) {コウ};
\node[Meaning] at (27.550000, -21.800000) {subscription};
\node[Square] at (29.600000, -23.550000) {};
\node[Kanji] at (29.600000, -23.050000) {\textcolor[HTML]{154caa}{構}};
\node[Onyomi] at (29.650000, -23.450000) {コウ};
\node[Kunyomi] at (29.550000, -23.450000) {かま.*};
\node[Meaning] at (29.600000, -21.800000) {set up};
\node[Square] at (31.650000, -23.550000) {};
\node[Kanji] at (31.650000, -23.050000) {\textcolor[HTML]{113066}{溝}};
\node[Onyomi] at (31.700000, -23.450000) {コウ};
\node[Kunyomi] at (31.600000, -23.450000) {みぞ};
\node[Meaning] at (31.650000, -21.800000) {gutter};
\node[Square] at (33.700000, -23.550000) {};
\node[Kanji] at (33.700000, -23.050000) {\textcolor[HTML]{14469c}{論}};
\node[Onyomi] at (33.750000, -23.450000) {ロン};
\node[Meaning] at (33.700000, -21.800000) {theory};
\node[Square] at (35.750000, -23.550000) {};
\node[Kanji] at (35.750000, -23.050000) {\textcolor[HTML]{0e254c}{倫}};
\node[Onyomi] at (35.800000, -23.450000) {リン};
\node[Meaning] at (35.750000, -21.800000) {ethics};
\node[Square] at (37.800000, -23.550000) {};
\node[Kanji] at (37.800000, -23.050000) {\textcolor[HTML]{14469c}{輪}};
\node[Onyomi] at (37.850000, -23.450000) {リン};
\node[Kunyomi] at (37.750000, -23.450000) {わ};
\node[Meaning] at (37.800000, -21.800000) {wheel};
\node[Square] at (39.850000, -23.550000) {};
\node[Kanji] at (39.850000, -23.050000) {\textcolor[HTML]{102b59}{偏}};
\node[Onyomi] at (39.900000, -23.450000) {ヘン};
\node[Kunyomi] at (39.800000, -23.450000) {かたよ};
\node[Meaning] at (39.850000, -21.800000) {biased};
\node[Square] at (41.900000, -23.550000) {};
\node[Kanji] at (41.900000, -23.050000) {\textcolor[HTML]{0e254c}{遍}};
\node[Onyomi] at (41.950000, -23.450000) {ヘン};
\node[Kunyomi] at (41.850000, -23.450000) {あまね};
\node[Meaning] at (41.900000, -21.800000) {universal};
\node[Square] at (43.950000, -23.550000) {};
\node[Kanji] at (43.950000, -23.050000) {\textcolor[HTML]{133c80}{編}};
\node[Onyomi] at (44.000000, -23.450000) {ヘン};
\node[Kunyomi] at (43.900000, -23.450000) {あ.む};
\node[Meaning] at (43.950000, -21.800000) {knit};
\node[Square] at (46.000000, -23.550000) {};
\node[Kanji] at (46.000000, -23.050000) {\textcolor[HTML]{133c80}{冊}};
\node[Onyomi] at (46.050000, -23.450000) {サツ};
\node[Meaning] at (46.000000, -21.800000) {books counter};
\node[Square] at (48.050000, -23.550000) {};
\node[Kanji] at (48.050000, -23.050000) {\textcolor[HTML]{102b59}{典}};
\node[Onyomi] at (48.100000, -23.450000) {テン};
\node[Meaning] at (48.050000, -21.800000) {rule};
\node[Square] at (50.100000, -23.550000) {};
\node[Kanji] at (50.100000, -23.050000) {\textcolor[HTML]{1551b8}{氏}};
\node[Onyomi] at (50.150000, -23.450000) {シ};
\node[Kunyomi] at (50.050000, -23.450000) {うじ};
\node[Meaning] at (50.100000, -21.800000) {family name};
\node[Square] at (52.150000, -23.550000) {};
\node[Kanji] at (52.150000, -23.050000) {\textcolor[HTML]{1557c6}{紙}};
\node[Onyomi] at (52.200000, -23.450000) {シ};
\node[Kunyomi] at (52.100000, -23.450000) {かみ};
\node[Meaning] at (52.150000, -21.800000) {paper};
\node[Square] at (54.200000, -23.550000) {};
\node[Kanji] at (54.200000, -23.050000) {\textcolor[HTML]{14469c}{婚}};
\node[Onyomi] at (54.250000, -23.450000) {コン};
\node[Meaning] at (54.200000, -21.800000) {marriage};
\node[Square] at (56.250000, -23.550000) {};
\node[Kanji] at (56.250000, -23.050000) {\textcolor[HTML]{1551b8}{低}};
\node[Onyomi] at (56.300000, -23.450000) {テイ};
\node[Kunyomi] at (56.200000, -23.450000) {ひく.い};
\node[Meaning] at (56.250000, -21.800000) {low};
\node[Meaning] at (-58.500000, -23.000000) {90.15\%};
\node[Square] at (-56.500000, -25.600000) {};
\node[Kanji] at (-56.500000, -25.100000) {\textcolor[HTML]{133c80}{抵}};
\node[Onyomi] at (-56.450000, -25.500000) {テイ};
\node[Meaning] at (-56.500000, -23.850000) {resist};
\node[Square] at (-54.450000, -25.600000) {};
\node[Kanji] at (-54.450000, -25.100000) {\textcolor[HTML]{14469c}{底}};
\node[Onyomi] at (-54.400000, -25.500000) {テイ};
\node[Kunyomi] at (-54.500000, -25.500000) {そこ};
\node[Meaning] at (-54.450000, -23.850000) {bottom};
\node[Square] at (-52.400000, -25.600000) {};
\node[Kanji] at (-52.400000, -25.100000) {\textcolor[HTML]{154caa}{民}};
\node[Onyomi] at (-52.350000, -25.500000) {ミン};
\node[Kunyomi] at (-52.450000, -25.500000) {たみ};
\node[Meaning] at (-52.400000, -23.850000) {peoples};
\node[Square] at (-50.350000, -25.600000) {};
\node[Kanji] at (-50.350000, -25.100000) {\textcolor[HTML]{1551b8}{眠}};
\node[Onyomi] at (-50.300000, -25.500000) {ミン};
\node[Kunyomi] at (-50.400000, -25.500000) {ねむ.*};
\node[Meaning] at (-50.350000, -23.850000) {sleep};
\node[Square] at (-48.300000, -25.600000) {};
\node[Kanji] at (-48.300000, -25.100000) {\textcolor[HTML]{1551b8}{捕}};
\node[Onyomi] at (-48.250000, -25.500000) {ホ};
\node[Kunyomi] at (-48.350000, -25.500000) {つか.まる};
\node[Meaning] at (-48.300000, -23.850000) {catch};
\node[Square] at (-46.250000, -25.600000) {};
\node[Kanji] at (-46.250000, -25.100000) {\textcolor[HTML]{123673}{浦}};
\node[Onyomi] at (-46.200000, -25.500000) {ホ};
\node[Kunyomi] at (-46.300000, -25.500000) {うら};
\node[Meaning] at (-46.250000, -23.850000) {bay};
\node[Square] at (-44.200000, -25.600000) {};
\node[Kanji] at (-44.200000, -25.100000) {\textcolor[HTML]{0e254c}{舗}};
\node[Onyomi] at (-44.150000, -25.500000) {ホ};
\node[Meaning] at (-44.200000, -23.850000) {shop};
\node[Square] at (-42.150000, -25.600000) {};
\node[Kanji] at (-42.150000, -25.100000) {\textcolor[HTML]{14418e}{補}};
\node[Onyomi] at (-42.100000, -25.500000) {ホ};
\node[Kunyomi] at (-42.200000, -25.500000) {おぎな.う};
\node[Meaning] at (-42.150000, -23.850000) {supplement};
\node[Square] at (-40.100000, -25.600000) {};
\node[Kanji] at (-40.100000, -25.100000) {\textcolor[HTML]{0e254c}{邸}};
\node[Onyomi] at (-40.050000, -25.500000) {テイ};
\node[Kunyomi] at (-40.150000, -25.500000) {やしき};
\node[Meaning] at (-40.100000, -23.850000) {residence};
\node[Square] at (-38.050000, -25.600000) {};
\node[Kanji] at (-38.050000, -25.100000) {\textcolor[HTML]{123673}{郭}};
\node[Onyomi] at (-38.000000, -25.500000) {カク};
\node[Kunyomi] at (-38.100000, -25.500000) {くるわ        };
\node[Meaning] at (-38.050000, -23.850000) {enclosure};
\node[Square] at (-36.000000, -25.600000) {};
\node[Kanji] at (-36.000000, -25.100000) {\textcolor[HTML]{0e254c}{郡}};
\node[Onyomi] at (-35.950000, -25.500000) {グン};
\node[Kunyomi] at (-36.050000, -25.500000) {こおり        };
\node[Meaning] at (-36.000000, -23.850000) {county};
\node[Square] at (-33.950000, -25.600000) {};
\node[Kanji] at (-33.950000, -25.100000) {\textcolor[HTML]{113066}{郊}};
\node[Onyomi] at (-33.900000, -25.500000) {コウ};
\node[Meaning] at (-33.950000, -23.850000) {suburbs};
\node[Square] at (-31.900000, -25.600000) {};
\node[Kanji] at (-31.900000, -25.100000) {\textcolor[HTML]{1461e3}{部}};
\node[Onyomi] at (-31.850000, -25.500000) {ブ};
\node[Kunyomi] at (-31.950000, -25.500000) {へ};
\node[Meaning] at (-31.900000, -23.850000) {part};
\node[Square] at (-29.850000, -25.600000) {};
\node[Kanji] at (-29.850000, -25.100000) {\textcolor[HTML]{1551b8}{都}};
\node[Onyomi] at (-29.800000, -25.500000) {ト};
\node[Kunyomi] at (-29.900000, -25.500000) {みやこ};
\node[Meaning] at (-29.850000, -23.850000) {metropolis};
\node[Square] at (-27.800000, -25.600000) {};
\node[Kanji] at (-27.800000, -25.100000) {\textcolor[HTML]{14418e}{郵}};
\node[Onyomi] at (-27.750000, -25.500000) {ユウ};
\node[Meaning] at (-27.800000, -23.850000) {mail};
\node[Square] at (-25.750000, -25.600000) {};
\node[Kanji] at (-25.750000, -25.100000) {\textcolor[HTML]{113066}{邦}};
\node[Onyomi] at (-25.700000, -25.500000) {ホウ};
\node[Kunyomi] at (-25.800000, -25.500000) {くに};
\node[Meaning] at (-25.750000, -23.850000) {home country};
\node[Square] at (-23.700000, -25.600000) {};
\node[Kanji] at (-23.700000, -25.100000) {\textcolor[HTML]{113066}{郷}};
\node[Onyomi] at (-23.650000, -25.500000) {キョウ};
\node[Meaning] at (-23.700000, -23.850000) {hometown};
\node[Square] at (-21.650000, -25.600000) {};
\node[Kanji] at (-21.650000, -25.100000) {\textcolor[HTML]{154caa}{響}};
\node[Onyomi] at (-21.600000, -25.500000) {キョウ};
\node[Kunyomi] at (-21.700000, -25.500000) {ひび.く};
\node[Meaning] at (-21.650000, -23.850000) {echo};
\node[Square] at (-19.600000, -25.600000) {};
\node[Kanji] at (-19.600000, -25.100000) {\textcolor[HTML]{14418e}{郎}};
\node[Onyomi] at (-19.550000, -25.500000) {ロウ};
\node[Meaning] at (-19.600000, -23.850000) {guy};
\node[Square] at (-17.550000, -25.600000) {};
\node[Kanji] at (-17.550000, -25.100000) {\textcolor[HTML]{154caa}{廊}};
\node[Onyomi] at (-17.500000, -25.500000) {ロウ};
\node[Meaning] at (-17.550000, -23.850000) {corridor};
\node[Square] at (-15.500000, -25.600000) {};
\node[Kanji] at (-15.500000, -25.100000) {\textcolor[HTML]{123673}{盾}};
\node[Onyomi] at (-15.450000, -25.500000) {ジュン};
\node[Kunyomi] at (-15.550000, -25.500000) {たて};
\node[Meaning] at (-15.500000, -23.850000) {shield};
\node[Square] at (-13.450000, -25.600000) {};
\node[Kanji] at (-13.450000, -25.100000) {\textcolor[HTML]{0e254c}{循}};
\node[Onyomi] at (-13.400000, -25.500000) {ジュン};
\node[Meaning] at (-13.450000, -23.850000) {circulation};
\node[Square] at (-11.400000, -25.600000) {};
\node[Kanji] at (-11.400000, -25.100000) {\textcolor[HTML]{14418e}{派}};
\node[Onyomi] at (-11.350000, -25.500000) {ハ};
\node[Meaning] at (-11.400000, -23.850000) {sect};
\node[Square] at (-9.350000, -25.600000) {};
\node[Kanji] at (-9.350000, -25.100000) {\textcolor[HTML]{133c80}{脈}};
\node[Onyomi] at (-9.300000, -25.500000) {ミャク};
\node[Meaning] at (-9.350000, -23.850000) {vein};
\node[Square] at (-7.300000, -25.600000) {};
\node[Kanji] at (-7.300000, -25.100000) {\textcolor[HTML]{14418e}{衆}};
\node[Onyomi] at (-7.250000, -25.500000) {シュウ};
\node[Meaning] at (-7.300000, -23.850000) {populace};
\node[Square] at (-5.250000, -25.600000) {};
\node[Kanji] at (-5.250000, -25.100000) {\textcolor[HTML]{0e254c}{逓}};
\node[Onyomi] at (-5.200000, -25.500000) {テイ};
\node[Meaning] at (-5.250000, -23.850000) {relay};
\node[Square] at (-3.200000, -25.600000) {};
\node[Kanji] at (-3.200000, -25.100000) {\textcolor[HTML]{1557c6}{段}};
\node[Onyomi] at (-3.150000, -25.500000) {ダン};
\node[Meaning] at (-3.200000, -23.850000) {steps};
\node[Square] at (-1.150000, -25.600000) {};
\node[Kanji] at (-1.150000, -25.100000) {\textcolor[HTML]{102b59}{鍛}};
\node[Onyomi] at (-1.100000, -25.500000) {タン};
\node[Kunyomi] at (-1.200000, -25.500000) {きた.える};
\node[Meaning] at (-1.150000, -23.850000) {forge};
\node[Square] at (0.900000, -25.600000) {};
\node[Kanji] at (0.900000, -25.100000) {\textcolor[HTML]{0e254c}{后}};
\node[Onyomi] at (0.950000, -25.500000) {コウ};
\node[Kunyomi] at (0.850000, -25.500000) {きさき};
\node[Meaning] at (0.900000, -23.850000) {empress};
\node[Square] at (2.950000, -25.600000) {};
\node[Kanji] at (2.950000, -25.100000) {\textcolor[HTML]{123673}{幻}};
\node[Onyomi] at (3.000000, -25.500000) {ゲン};
\node[Kunyomi] at (2.900000, -25.500000) {まぼろし};
\node[Meaning] at (2.950000, -23.850000) {illusion};
\node[Square] at (5.000000, -25.600000) {};
\node[Kanji] at (5.000000, -25.100000) {\textcolor[HTML]{133c80}{司}};
\node[Onyomi] at (5.050000, -25.500000) {シ};
\node[Kunyomi] at (4.950000, -25.500000) {つかさど.る};
\node[Meaning] at (5.000000, -23.850000) {director};
\node[Square] at (7.050000, -25.600000) {};
\node[Kanji] at (7.050000, -25.100000) {\textcolor[HTML]{123673}{伺}};
\node[Onyomi] at (7.100000, -25.500000) {シ};
\node[Kunyomi] at (7.000000, -25.500000) {うかが};
\node[Meaning] at (7.050000, -23.850000) {pay respects};
\node[Square] at (9.100000, -25.600000) {};
\node[Kanji] at (9.100000, -25.100000) {\textcolor[HTML]{113066}{詞}};
\node[Onyomi] at (9.150000, -25.500000) {シ};
\node[Meaning] at (9.100000, -23.850000) {part of speech};
\node[Square] at (11.150000, -25.600000) {};
\node[Kanji] at (11.150000, -25.100000) {\textcolor[HTML]{14418e}{飼}};
\node[Onyomi] at (11.200000, -25.500000) {シ};
\node[Kunyomi] at (11.100000, -25.500000) {か};
\node[Meaning] at (11.150000, -23.850000) {domesticate};
\node[Square] at (13.200000, -25.600000) {};
\node[Kanji] at (13.200000, -25.100000) {\textcolor[HTML]{0e254c}{嗣}};
\node[Onyomi] at (13.250000, -25.500000) {シ};
\node[Meaning] at (13.200000, -23.850000) {heir};
\node[Square] at (15.250000, -25.600000) {};
\node[Kanji] at (15.250000, -25.100000) {\textcolor[HTML]{123673}{舟}};
\node[Onyomi] at (15.300000, -25.500000) {シュウ};
\node[Kunyomi] at (15.200000, -25.500000) {ふね};
\node[Meaning] at (15.250000, -23.850000) {boat};
\node[Square] at (17.300000, -25.600000) {};
\node[Kanji] at (17.300000, -25.100000) {\textcolor[HTML]{0e254c}{舶}};
\node[Onyomi] at (17.350000, -25.500000) {ハク};
\node[Meaning] at (17.300000, -23.850000) {ship};
\node[Square] at (19.350000, -25.600000) {};
\node[Kanji] at (19.350000, -25.100000) {\textcolor[HTML]{14418e}{航}};
\node[Onyomi] at (19.400000, -25.500000) {コウ};
\node[Meaning] at (19.350000, -23.850000) {navigation};
\node[Square] at (21.400000, -25.600000) {};
\node[Kanji] at (21.400000, -25.100000) {\textcolor[HTML]{123673}{般}};
\node[Onyomi] at (21.450000, -25.500000) {ハン};
\node[Meaning] at (21.400000, -23.850000) {generally};
\node[Square] at (23.450000, -25.600000) {};
\node[Kanji] at (23.450000, -25.100000) {\textcolor[HTML]{133c80}{盤}};
\node[Onyomi] at (23.500000, -25.500000) {バン};
\node[Kunyomi] at (23.400000, -25.500000) {ばん};
\node[Meaning] at (23.450000, -23.850000) {tray};
\node[Square] at (25.500000, -25.600000) {};
\node[Kanji] at (25.500000, -25.100000) {\textcolor[HTML]{0e254c}{搬}};
\node[Onyomi] at (25.550000, -25.500000) {ハン};
\node[Meaning] at (25.500000, -23.850000) {transport};
\node[Square] at (27.550000, -25.600000) {};
\node[Kanji] at (27.550000, -25.100000) {\textcolor[HTML]{154caa}{船}};
\node[Onyomi] at (27.600000, -25.500000) {セン};
\node[Kunyomi] at (27.500000, -25.500000) {ふね};
\node[Meaning] at (27.550000, -23.850000) {boat};
\node[Square] at (29.600000, -25.600000) {};
\node[Kanji] at (29.600000, -25.100000) {\textcolor[HTML]{102b59}{艦}};
\node[Onyomi] at (29.650000, -25.500000) {カン};
\node[Meaning] at (29.600000, -23.850000) {warship};
\node[Square] at (31.650000, -25.600000) {};
\node[Kanji] at (31.650000, -25.100000) {\textcolor[HTML]{0e254c}{艇}};
\node[Onyomi] at (31.700000, -25.500000) {テイ};
\node[Meaning] at (31.650000, -23.850000) {rowboat};
\node[Square] at (33.700000, -25.600000) {};
\node[Kanji] at (33.700000, -25.100000) {\textcolor[HTML]{0e254c}{瓜}};
\node[Onyomi] at (33.750000, -25.500000) {カ};
\node[Kunyomi] at (33.650000, -25.500000) {うり};
\node[Meaning] at (33.700000, -23.850000) {melon};
\node[Square] at (35.750000, -25.600000) {};
\node[Kanji] at (35.750000, -25.100000) {\textcolor[HTML]{0e254c}{弧}};
\node[Onyomi] at (35.800000, -25.500000) {コ};
\node[Meaning] at (35.750000, -23.850000) {arc};
\node[Square] at (37.800000, -25.600000) {};
\node[Kanji] at (37.800000, -25.100000) {\textcolor[HTML]{133c80}{孤}};
\node[Onyomi] at (37.850000, -25.500000) {コ};
\node[Meaning] at (37.800000, -23.850000) {orphan};
\node[Square] at (39.850000, -25.600000) {};
\node[Kanji] at (39.850000, -25.100000) {\textcolor[HTML]{0e254c}{繭}};
\node[Onyomi] at (39.900000, -25.500000) {ケン};
\node[Kunyomi] at (39.800000, -25.500000) {まゆ};
\node[Meaning] at (39.850000, -23.850000) {cocoon};
\node[Square] at (41.900000, -25.600000) {};
\node[Kanji] at (41.900000, -25.100000) {\textcolor[HTML]{123673}{益}};
\node[Onyomi] at (41.950000, -25.500000) {エキ};
\node[Meaning] at (41.900000, -23.850000) {benefit};
\node[Square] at (43.950000, -25.600000) {};
\node[Kanji] at (43.950000, -25.100000) {\textcolor[HTML]{14418e}{暇}};
\node[Onyomi] at (44.000000, -25.500000) {カ};
\node[Kunyomi] at (43.900000, -25.500000) {ひま};
\node[Meaning] at (43.950000, -23.850000) {spare time};
\node[Square] at (46.000000, -25.600000) {};
\node[Kanji] at (46.000000, -25.100000) {\textcolor[HTML]{154caa}{敷}};
\node[Onyomi] at (46.050000, -25.500000) {フ};
\node[Kunyomi] at (45.950000, -25.500000) {しき};
\node[Meaning] at (46.000000, -23.850000) {spread};
\node[Square] at (48.050000, -25.600000) {};
\node[Kanji] at (48.050000, -25.100000) {\textcolor[HTML]{1461e3}{来}};
\node[Onyomi] at (48.100000, -25.500000) {ライ};
\node[Kunyomi] at (48.000000, -25.500000) {く.る};
\node[Meaning] at (48.050000, -23.850000) {come};
\node[Square] at (50.100000, -25.600000) {};
\node[Kanji] at (50.100000, -25.100000) {\textcolor[HTML]{1968ed}{気}};
\node[Onyomi] at (50.150000, -25.500000) {キ};
\node[Kunyomi] at (50.050000, -25.500000) {いき};
\node[Meaning] at (50.100000, -23.850000) {energy};
\node[Square] at (52.150000, -25.600000) {};
\node[Kanji] at (52.150000, -25.100000) {\textcolor[HTML]{14418e}{汽}};
\node[Onyomi] at (52.200000, -25.500000) {キ};
\node[Meaning] at (52.150000, -23.850000) {steam};
\node[Square] at (54.200000, -25.600000) {};
\node[Kanji] at (54.200000, -25.100000) {\textcolor[HTML]{1461e3}{飛}};
\node[Onyomi] at (54.250000, -25.500000) {ヒ};
\node[Kunyomi] at (54.150000, -25.500000) {と};
\node[Meaning] at (54.200000, -23.850000) {fly};
\node[Square] at (56.250000, -25.600000) {};
\node[Kanji] at (56.250000, -25.100000) {\textcolor[HTML]{154caa}{沈}};
\node[Onyomi] at (56.300000, -25.500000) {チン};
\node[Kunyomi] at (56.200000, -25.500000) {しず.*};
\node[Meaning] at (56.250000, -23.850000) {sink};
\node[Meaning] at (-58.500000, -25.050000) {92.75\%};
\node[Square] at (-56.500000, -27.650000) {};
\node[Kanji] at (-56.500000, -27.150000) {\textcolor[HTML]{154caa}{妻}};
\node[Onyomi] at (-56.450000, -27.550000) {サイ};
\node[Kunyomi] at (-56.550000, -27.550000) {つま};
\node[Meaning] at (-56.500000, -25.900000) {wife};
\node[Square] at (-54.450000, -27.650000) {};
\node[Kanji] at (-54.450000, -27.150000) {\textcolor[HTML]{102b59}{衰}};
\node[Onyomi] at (-54.400000, -27.550000) {スイ};
\node[Kunyomi] at (-54.500000, -27.550000) {おとろ.える};
\node[Meaning] at (-54.450000, -25.900000) {decline};
\node[Square] at (-52.400000, -27.650000) {};
\node[Kanji] at (-52.400000, -27.150000) {\textcolor[HTML]{0e254c}{衷}};
\node[Onyomi] at (-52.350000, -27.550000) {チュウ};
\node[Meaning] at (-52.400000, -25.900000) {inmost};
\node[Square] at (-50.350000, -27.650000) {};
\node[Kanji] at (-50.350000, -27.150000) {\textcolor[HTML]{1557c6}{面}};
\node[Onyomi] at (-50.300000, -27.550000) {メン};
\node[Kunyomi] at (-50.400000, -27.550000) {おも};
\node[Meaning] at (-50.350000, -25.900000) {surface};
\node[Square] at (-48.300000, -27.650000) {};
\node[Kanji] at (-48.300000, -27.150000) {\textcolor[HTML]{133c80}{革}};
\node[Onyomi] at (-48.250000, -27.550000) {カク};
\node[Kunyomi] at (-48.350000, -27.550000) {かわ};
\node[Meaning] at (-48.300000, -25.900000) {leather};
\node[Square] at (-46.250000, -27.650000) {};
\node[Kanji] at (-46.250000, -27.150000) {\textcolor[HTML]{14418e}{靴}};
\node[Onyomi] at (-46.200000, -27.550000) {カ};
\node[Kunyomi] at (-46.300000, -27.550000) {くつ};
\node[Meaning] at (-46.250000, -25.900000) {shoes};
\node[Square] at (-44.200000, -27.650000) {};
\node[Kanji] at (-44.200000, -27.150000) {\textcolor[HTML]{113066}{覇}};
\node[Onyomi] at (-44.150000, -27.550000) {ハ};
\node[Kunyomi] at (-44.250000, -27.550000) {はたがしら};
\node[Meaning] at (-44.200000, -25.900000) {leadership};
\node[Square] at (-42.150000, -27.650000) {};
\node[Kanji] at (-42.150000, -27.150000) {\textcolor[HTML]{1968ed}{声}};
\node[Onyomi] at (-42.100000, -27.550000) {セイ};
\node[Kunyomi] at (-42.200000, -27.550000) {こえ};
\node[Meaning] at (-42.150000, -25.900000) {voice};
\node[Square] at (-40.100000, -27.650000) {};
\node[Kanji] at (-40.100000, -27.150000) {\textcolor[HTML]{0e254c}{呉}};
\node[Onyomi] at (-40.050000, -27.550000) {ゴ};
\node[Kunyomi] at (-40.150000, -27.550000) {くれ};
\node[Meaning] at (-40.100000, -25.900000) {give};
\node[Square] at (-38.050000, -27.650000) {};
\node[Kanji] at (-38.050000, -27.150000) {\textcolor[HTML]{0e254c}{娯}};
\node[Onyomi] at (-38.000000, -27.550000) {ゴ};
\node[Meaning] at (-38.050000, -25.900000) {recreation};
\node[Square] at (-36.000000, -27.650000) {};
\node[Kanji] at (-36.000000, -27.150000) {\textcolor[HTML]{123673}{誤}};
\node[Onyomi] at (-35.950000, -27.550000) {ゴ};
\node[Kunyomi] at (-36.050000, -27.550000) {あやま.*};
\node[Meaning] at (-36.000000, -25.900000) {mistake};
\node[Square] at (-33.950000, -27.650000) {};
\node[Kanji] at (-33.950000, -27.150000) {\textcolor[HTML]{123673}{蒸}};
\node[Onyomi] at (-33.900000, -27.550000) {ジョウ};
\node[Kunyomi] at (-34.000000, -27.550000) {む.れる};
\node[Meaning] at (-33.950000, -25.900000) {steam};
\node[Square] at (-31.900000, -27.650000) {};
\node[Kanji] at (-31.900000, -27.150000) {\textcolor[HTML]{14418e}{承}};
\node[Onyomi] at (-31.850000, -27.550000) {ショウ};
\node[Kunyomi] at (-31.950000, -27.550000) {うけたまわ.る};
\node[Meaning] at (-31.900000, -25.900000) {consent};
\node[Square] at (-29.850000, -27.650000) {};
\node[Kanji] at (-29.850000, -27.150000) {\textcolor[HTML]{14418e}{極}};
\node[Onyomi] at (-29.800000, -27.550000) {キョク};
\node[Kunyomi] at (-29.900000, -27.550000) {きわ.める};
\node[Meaning] at (-29.850000, -25.900000) {extreme};
\node[Square] at (-27.800000, -27.650000) {};
\node[Kanji] at (-27.800000, -27.150000) {\textcolor[HTML]{133c80}{牙}};
\node[Onyomi] at (-27.750000, -27.550000) {ゲ};
\node[Kunyomi] at (-27.850000, -27.550000) {きば};
\node[Meaning] at (-27.800000, -25.900000) {fang};
\node[Square] at (-25.750000, -27.650000) {};
\node[Kanji] at (-25.750000, -27.150000) {\textcolor[HTML]{102b59}{芽}};
\node[Onyomi] at (-25.700000, -27.550000) {ガ};
\node[Kunyomi] at (-25.800000, -27.550000) {め};
\node[Meaning] at (-25.750000, -25.900000) {sprout};
\node[Square] at (-23.700000, -27.650000) {};
\node[Kanji] at (-23.700000, -27.150000) {\textcolor[HTML]{14469c}{邪}};
\node[Onyomi] at (-23.650000, -27.550000) {ジャ};
\node[Kunyomi] at (-23.750000, -27.550000) {よこし.ま};
\node[Meaning] at (-23.700000, -25.900000) {wicked};
\node[Square] at (-21.650000, -27.650000) {};
\node[Kanji] at (-21.650000, -27.150000) {\textcolor[HTML]{123673}{雅}};
\node[Onyomi] at (-21.600000, -27.550000) {ガ};
\node[Kunyomi] at (-21.700000, -27.550000) {みや.び};
\node[Meaning] at (-21.650000, -25.900000) {elegant};
\node[Square] at (-19.600000, -27.650000) {};
\node[Kanji] at (-19.600000, -27.150000) {\textcolor[HTML]{123673}{釈}};
\node[Onyomi] at (-19.550000, -27.550000) {シャク};
\node[Kunyomi] at (-19.650000, -27.550000) {す};
\node[Meaning] at (-19.600000, -25.900000) {explanation};
\node[Square] at (-17.550000, -27.650000) {};
\node[Kanji] at (-17.550000, -27.150000) {\textcolor[HTML]{1557c6}{番}};
\node[Onyomi] at (-17.500000, -27.550000) {バン};
\node[Meaning] at (-17.550000, -25.900000) {number (series)};
\node[Square] at (-15.500000, -27.650000) {};
\node[Kanji] at (-15.500000, -27.150000) {\textcolor[HTML]{14418e}{審}};
\node[Onyomi] at (-15.450000, -27.550000) {シン};
\node[Meaning] at (-15.500000, -25.900000) {judge};
\node[Square] at (-13.450000, -27.650000) {};
\node[Kanji] at (-13.450000, -27.150000) {\textcolor[HTML]{123673}{翻}};
\node[Onyomi] at (-13.400000, -27.550000) {ホン};
\node[Kunyomi] at (-13.500000, -27.550000) {ひるがえ.*};
\node[Meaning] at (-13.450000, -25.900000) {flip};
\node[Square] at (-11.400000, -27.650000) {};
\node[Kanji] at (-11.400000, -27.150000) {\textcolor[HTML]{0e254c}{藩}};
\node[Onyomi] at (-11.350000, -27.550000) {ハン};
\node[Meaning] at (-11.400000, -25.900000) {fiefdom};
\node[Square] at (-9.350000, -27.650000) {};
\node[Kanji] at (-9.350000, -27.150000) {\textcolor[HTML]{154caa}{毛}};
\node[Onyomi] at (-9.300000, -27.550000) {モウ};
\node[Kunyomi] at (-9.400000, -27.550000) {け};
\node[Meaning] at (-9.350000, -25.900000) {fur};
\node[Square] at (-7.300000, -27.650000) {};
\node[Kanji] at (-7.300000, -27.150000) {\textcolor[HTML]{0e254c}{耗}};
\node[Onyomi] at (-7.250000, -27.550000) {モウ};
\node[Meaning] at (-7.300000, -25.900000) {decrease};
\node[Square] at (-5.250000, -27.650000) {};
\node[Kanji] at (-5.250000, -27.150000) {\textcolor[HTML]{14469c}{尾}};
\node[Onyomi] at (-5.200000, -27.550000) {ビ};
\node[Kunyomi] at (-5.300000, -27.550000) {お};
\node[Meaning] at (-5.250000, -25.900000) {tail};
\node[Square] at (-3.200000, -27.650000) {};
\node[Kanji] at (-3.200000, -27.150000) {\textcolor[HTML]{14418e}{宅}};
\node[Onyomi] at (-3.150000, -27.550000) {タク};
\node[Meaning] at (-3.200000, -25.900000) {house};
\node[Square] at (-1.150000, -27.650000) {};
\node[Kanji] at (-1.150000, -27.150000) {\textcolor[HTML]{102b59}{託}};
\node[Onyomi] at (-1.100000, -27.550000) {タク};
\node[Kunyomi] at (-1.200000, -27.550000) {かこ.*};
\node[Meaning] at (-1.150000, -25.900000) {consign};
\node[Square] at (0.900000, -27.650000) {};
\node[Kanji] at (0.900000, -27.150000) {\textcolor[HTML]{133c80}{為}};
\node[Onyomi] at (0.950000, -27.550000) {イ};
\node[Kunyomi] at (0.850000, -27.550000) {ため};
\node[Meaning] at (0.900000, -25.900000) {sake};
\node[Square] at (2.950000, -27.650000) {};
\node[Kanji] at (2.950000, -27.150000) {\textcolor[HTML]{133c80}{偽}};
\node[Onyomi] at (3.000000, -27.550000) {ギ};
\node[Kunyomi] at (2.900000, -27.550000) {にせ};
\node[Meaning] at (2.950000, -25.900000) {fake};
\node[Square] at (5.000000, -27.650000) {};
\node[Kanji] at (5.000000, -27.150000) {\textcolor[HTML]{145cd5}{長}};
\node[Onyomi] at (5.050000, -27.550000) {チョウ};
\node[Kunyomi] at (4.950000, -27.550000) {なが.い};
\node[Meaning] at (5.000000, -25.900000) {long};
\node[Square] at (7.050000, -27.650000) {};
\node[Kanji] at (7.050000, -27.150000) {\textcolor[HTML]{1557c6}{張}};
\node[Onyomi] at (7.100000, -27.550000) {チョウ};
\node[Kunyomi] at (7.000000, -27.550000) {は.る};
\node[Meaning] at (7.050000, -25.900000) {stretch};
\node[Square] at (9.100000, -27.650000) {};
\node[Kanji] at (9.100000, -27.150000) {\textcolor[HTML]{123673}{帳}};
\node[Onyomi] at (9.150000, -27.550000) {チョウ};
\node[Kunyomi] at (9.050000, -27.550000) {とばり};
\node[Meaning] at (9.100000, -25.900000) {notebook};
\node[Square] at (11.150000, -27.650000) {};
\node[Kanji] at (11.150000, -27.150000) {\textcolor[HTML]{1551b8}{髪}};
\node[Onyomi] at (11.200000, -27.550000) {ハツ};
\node[Kunyomi] at (11.100000, -27.550000) {かみ};
\node[Meaning] at (11.150000, -25.900000) {hair};
\node[Square] at (13.200000, -27.650000) {};
\node[Kanji] at (13.200000, -27.150000) {\textcolor[HTML]{14418e}{展}};
\node[Onyomi] at (13.250000, -27.550000) {テン};
\node[Kunyomi] at (13.150000, -27.550000) {のぶ};
\node[Meaning] at (13.200000, -25.900000) {expand};
\node[Square] at (15.250000, -27.650000) {};
\node[Kanji] at (15.250000, -27.150000) {\textcolor[HTML]{102b59}{喪}};
\node[Onyomi] at (15.300000, -27.550000) {ソウ};
\node[Kunyomi] at (15.200000, -27.550000) {も};
\node[Meaning] at (15.250000, -25.900000) {mourning};
\node[Square] at (17.300000, -27.650000) {};
\node[Kanji] at (17.300000, -27.150000) {\textcolor[HTML]{133c80}{巣}};
\node[Onyomi] at (17.350000, -27.550000) {ソウ};
\node[Kunyomi] at (17.250000, -27.550000) {す};
\node[Meaning] at (17.300000, -25.900000) {nest};
\node[Square] at (19.350000, -27.650000) {};
\node[Kanji] at (19.350000, -27.150000) {\textcolor[HTML]{154caa}{単}};
\node[Onyomi] at (19.400000, -27.550000) {タン};
\node[Meaning] at (19.350000, -25.900000) {simple};
\node[Square] at (21.400000, -27.650000) {};
\node[Kanji] at (21.400000, -27.150000) {\textcolor[HTML]{1557c6}{戦}};
\node[Onyomi] at (21.450000, -27.550000) {セン};
\node[Kunyomi] at (21.350000, -27.550000) {たたか.*};
\node[Meaning] at (21.400000, -25.900000) {war};
\node[Square] at (23.450000, -27.650000) {};
\node[Kanji] at (23.450000, -27.150000) {\textcolor[HTML]{0e254c}{禅}};
\node[Onyomi] at (23.500000, -27.550000) {ゼン};
\node[Meaning] at (23.450000, -25.900000) {zen};
\node[Square] at (25.500000, -27.650000) {};
\node[Kanji] at (25.500000, -27.150000) {\textcolor[HTML]{14469c}{弾}};
\node[Onyomi] at (25.550000, -27.550000) {ダン};
\node[Kunyomi] at (25.450000, -27.550000) {ひ.く};
\node[Meaning] at (25.500000, -25.900000) {bullet};
\node[Square] at (27.550000, -27.650000) {};
\node[Kanji] at (27.550000, -27.150000) {\textcolor[HTML]{154caa}{桜}};
\node[Kunyomi] at (27.500000, -27.550000) {さくら};
\node[Meaning] at (27.550000, -25.900000) {sakura};
\node[Square] at (29.600000, -27.650000) {};
\node[Kanji] at (29.600000, -27.150000) {\textcolor[HTML]{14418e}{獣}};
\node[Onyomi] at (29.650000, -27.550000) {ジュウ};
\node[Kunyomi] at (29.550000, -27.550000) {けもの};
\node[Meaning] at (29.600000, -25.900000) {beast};
\node[Square] at (31.650000, -27.650000) {};
\node[Kanji] at (31.650000, -27.150000) {\textcolor[HTML]{14418e}{脳}};
\node[Onyomi] at (31.700000, -27.550000) {ノウ};
\node[Meaning] at (31.650000, -25.900000) {brain};
\node[Square] at (33.700000, -27.650000) {};
\node[Kanji] at (33.700000, -27.150000) {\textcolor[HTML]{123673}{悩}};
\node[Onyomi] at (33.750000, -27.550000) {ノウ};
\node[Kunyomi] at (33.650000, -27.550000) {なや};
\node[Meaning] at (33.700000, -25.900000) {worry};
\node[Square] at (35.750000, -27.650000) {};
\node[Kanji] at (35.750000, -27.150000) {\textcolor[HTML]{154caa}{厳}};
\node[Onyomi] at (35.800000, -27.550000) {ゲン};
\node[Kunyomi] at (35.700000, -27.550000) {きび.しい};
\node[Meaning] at (35.750000, -25.900000) {strict};
\node[Square] at (37.800000, -27.650000) {};
\node[Kanji] at (37.800000, -27.150000) {\textcolor[HTML]{14418e}{鎖}};
\node[Onyomi] at (37.850000, -27.550000) {サ};
\node[Kunyomi] at (37.750000, -27.550000) {くさり};
\node[Meaning] at (37.800000, -25.900000) {chain};
\node[Square] at (39.850000, -27.650000) {};
\node[Kanji] at (39.850000, -27.150000) {\textcolor[HTML]{14469c}{挙}};
\node[Onyomi] at (39.900000, -27.550000) {キョ};
\node[Kunyomi] at (39.800000, -27.550000) {あ.がる};
\node[Meaning] at (39.850000, -25.900000) {raise};
\node[Square] at (41.900000, -27.650000) {};
\node[Kanji] at (41.900000, -27.150000) {\textcolor[HTML]{133c80}{誉}};
\node[Onyomi] at (41.950000, -27.550000) {ヨ};
\node[Kunyomi] at (41.850000, -27.550000) {ほ.める};
\node[Meaning] at (41.900000, -25.900000) {honor};
\node[Square] at (43.950000, -27.650000) {};
\node[Kanji] at (43.950000, -27.150000) {\textcolor[HTML]{0e254c}{猟}};
\node[Onyomi] at (44.000000, -27.550000) {リョウ};
\node[Kunyomi] at (43.900000, -27.550000) {かり};
\node[Meaning] at (43.950000, -25.900000) {hunting};
\node[Square] at (46.000000, -27.650000) {};
\node[Kanji] at (46.000000, -27.150000) {\textcolor[HTML]{154caa}{鳥}};
\node[Onyomi] at (46.050000, -27.550000) {チョウ};
\node[Kunyomi] at (45.950000, -27.550000) {とり};
\node[Meaning] at (46.000000, -25.900000) {bird};
\node[Square] at (48.050000, -27.650000) {};
\node[Kanji] at (48.050000, -27.150000) {\textcolor[HTML]{1557c6}{鳴}};
\node[Onyomi] at (48.100000, -27.550000) {メイ};
\node[Kunyomi] at (48.000000, -27.550000) {な};
\node[Meaning] at (48.050000, -25.900000) {chirp};
\node[Square] at (50.100000, -27.650000) {};
\node[Kanji] at (50.100000, -27.150000) {\textcolor[HTML]{113066}{鶴}};
\node[Onyomi] at (50.150000, -27.550000) {カク};
\node[Kunyomi] at (50.050000, -27.550000) {つる};
\node[Meaning] at (50.100000, -25.900000) {crane};
\node[Square] at (52.150000, -27.650000) {};
\node[Kanji] at (52.150000, -27.150000) {\textcolor[HTML]{0e254c}{烏}};
\node[Onyomi] at (52.200000, -27.550000) {ウオ};
\node[Kunyomi] at (52.100000, -27.550000) {からす};
\node[Meaning] at (52.150000, -25.900000) {crow};
\node[Square] at (54.200000, -27.650000) {};
\node[Kanji] at (54.200000, -27.150000) {\textcolor[HTML]{113066}{鳩}};
\node[Onyomi] at (54.250000, -27.550000) {ク};
\node[Kunyomi] at (54.150000, -27.550000) {はと};
\node[Meaning] at (54.200000, -25.900000) {dove};
\node[Square] at (56.250000, -27.650000) {};
\node[Kanji] at (56.250000, -27.150000) {\textcolor[HTML]{14418e}{鶏}};
\node[Onyomi] at (56.300000, -27.550000) {ケイ};
\node[Kunyomi] at (56.200000, -27.550000) {とり};
\node[Meaning] at (56.250000, -25.900000) {chicken};
\node[Meaning] at (-58.500000, -27.100000) {94.71\%};
\node[Square] at (-56.500000, -29.700000) {};
\node[Kanji] at (-56.500000, -29.200000) {\textcolor[HTML]{1557c6}{島}};
\node[Onyomi] at (-56.450000, -29.600000) {トウ};
\node[Kunyomi] at (-56.550000, -29.600000) {しま};
\node[Meaning] at (-56.500000, -27.950000) {island};
\node[Square] at (-54.450000, -29.700000) {};
\node[Kanji] at (-54.450000, -29.200000) {\textcolor[HTML]{154caa}{暖}};
\node[Onyomi] at (-54.400000, -29.600000) {ダン};
\node[Kunyomi] at (-54.500000, -29.600000) {あたた.かい};
\node[Meaning] at (-54.450000, -27.950000) {warm};
\node[Square] at (-52.400000, -29.700000) {};
\node[Kanji] at (-52.400000, -29.200000) {\textcolor[HTML]{0e254c}{媛}};
\node[Onyomi] at (-52.350000, -29.600000) {エン};
\node[Kunyomi] at (-52.450000, -29.600000) {ひめ};
\node[Meaning] at (-52.400000, -27.950000) {princess};
\node[Square] at (-50.350000, -29.700000) {};
\node[Kanji] at (-50.350000, -29.200000) {\textcolor[HTML]{14469c}{援}};
\node[Onyomi] at (-50.300000, -29.600000) {エン};
\node[Meaning] at (-50.350000, -27.950000) {aid};
\node[Square] at (-48.300000, -29.700000) {};
\node[Kanji] at (-48.300000, -29.200000) {\textcolor[HTML]{133c80}{緩}};
\node[Onyomi] at (-48.250000, -29.600000) {カン};
\node[Kunyomi] at (-48.350000, -29.600000) {ゆる};
\node[Meaning] at (-48.300000, -27.950000) {loose};
\node[Square] at (-46.250000, -29.700000) {};
\node[Kanji] at (-46.250000, -29.200000) {\textcolor[HTML]{14418e}{属}};
\node[Onyomi] at (-46.200000, -29.600000) {ゾク};
\node[Meaning] at (-46.250000, -27.950000) {belong};
\node[Square] at (-44.200000, -29.700000) {};
\node[Kanji] at (-44.200000, -29.200000) {\textcolor[HTML]{0e254c}{嘱}};
\node[Onyomi] at (-44.150000, -29.600000) {ショク};
\node[Kunyomi] at (-44.250000, -29.600000) {しょく.する};
\node[Meaning] at (-44.200000, -27.950000) {request};
\node[Square] at (-42.150000, -29.700000) {};
\node[Kanji] at (-42.150000, -29.200000) {\textcolor[HTML]{133c80}{偶}};
\node[Onyomi] at (-42.100000, -29.600000) {グウ};
\node[Kunyomi] at (-42.200000, -29.600000) {たま};
\node[Meaning] at (-42.150000, -27.950000) {accidentally};
\node[Square] at (-40.100000, -29.700000) {};
\node[Kanji] at (-40.100000, -29.200000) {\textcolor[HTML]{113066}{遇}};
\node[Onyomi] at (-40.050000, -29.600000) {グウ};
\node[Kunyomi] at (-40.150000, -29.600000) {あ};
\node[Meaning] at (-40.100000, -27.950000) {treatment};
\node[Square] at (-38.050000, -29.700000) {};
\node[Kanji] at (-38.050000, -29.200000) {\textcolor[HTML]{14418e}{愚}};
\node[Onyomi] at (-38.000000, -29.600000) {グ};
\node[Kunyomi] at (-38.100000, -29.600000) {おろ};
\node[Meaning] at (-38.050000, -27.950000) {foolish};
\node[Square] at (-36.000000, -29.700000) {};
\node[Kanji] at (-36.000000, -29.200000) {\textcolor[HTML]{14469c}{隅}};
\node[Onyomi] at (-35.950000, -29.600000) {グウ};
\node[Kunyomi] at (-36.050000, -29.600000) {すみ};
\node[Meaning] at (-36.000000, -27.950000) {corner};
\node[Square] at (-33.950000, -29.700000) {};
\node[Kanji] at (-33.950000, -29.200000) {\textcolor[HTML]{14469c}{逆}};
\node[Onyomi] at (-33.900000, -29.600000) {ギャク};
\node[Kunyomi] at (-34.000000, -29.600000) {さか.らう};
\node[Meaning] at (-33.950000, -27.950000) {reverse};
\node[Square] at (-31.900000, -29.700000) {};
\node[Kanji] at (-31.900000, -29.200000) {\textcolor[HTML]{0e254c}{塑}};
\node[Onyomi] at (-31.850000, -29.600000) {ソ};
\node[Meaning] at (-31.900000, -27.950000) {model};
\node[Square] at (-29.850000, -29.700000) {};
\node[Kanji] at (-29.850000, -29.200000) {\textcolor[HTML]{14469c}{岡}};
\node[Kunyomi] at (-29.900000, -29.600000) {おか};
\node[Meaning] at (-29.850000, -27.950000) {hill};
\node[Square] at (-27.800000, -29.700000) {};
\node[Kanji] at (-27.800000, -29.200000) {\textcolor[HTML]{0e254c}{鋼}};
\node[Onyomi] at (-27.750000, -29.600000) {コウ};
\node[Kunyomi] at (-27.850000, -29.600000) {はがね};
\node[Meaning] at (-27.800000, -27.950000) {steel};
\node[Square] at (-25.750000, -29.700000) {};
\node[Kanji] at (-25.750000, -29.200000) {\textcolor[HTML]{123673}{綱}};
\node[Onyomi] at (-25.700000, -29.600000) {コウ};
\node[Kunyomi] at (-25.800000, -29.600000) {つな};
\node[Meaning] at (-25.750000, -27.950000) {cable};
\node[Square] at (-23.700000, -29.700000) {};
\node[Kanji] at (-23.700000, -29.200000) {\textcolor[HTML]{0e254c}{剛}};
\node[Onyomi] at (-23.650000, -29.600000) {ゴウ};
\node[Meaning] at (-23.700000, -27.950000) {sturdy};
\node[Square] at (-21.650000, -29.700000) {};
\node[Kanji] at (-21.650000, -29.200000) {\textcolor[HTML]{123673}{缶}};
\node[Onyomi] at (-21.600000, -29.600000) {カン};
\node[Meaning] at (-21.650000, -27.950000) {tin can};
\node[Square] at (-19.600000, -29.700000) {};
\node[Kanji] at (-19.600000, -29.200000) {\textcolor[HTML]{113066}{陶}};
\node[Onyomi] at (-19.550000, -29.600000) {トウ};
\node[Meaning] at (-19.600000, -27.950000) {pottery};
\node[Square] at (-17.550000, -29.700000) {};
\node[Kanji] at (-17.550000, -29.200000) {\textcolor[HTML]{154caa}{揺}};
\node[Onyomi] at (-17.500000, -29.600000) {ヨウ};
\node[Kunyomi] at (-17.600000, -29.600000) {ゆ.*};
\node[Meaning] at (-17.550000, -27.950000) {shake};
\node[Square] at (-15.500000, -29.700000) {};
\node[Kanji] at (-15.500000, -29.200000) {\textcolor[HTML]{0e254c}{謡}};
\node[Onyomi] at (-15.450000, -29.600000) {ヨウ};
\node[Kunyomi] at (-15.550000, -29.600000) {うた};
\node[Meaning] at (-15.500000, -27.950000) {noh chanting};
\node[Square] at (-13.450000, -29.700000) {};
\node[Kanji] at (-13.450000, -29.200000) {\textcolor[HTML]{133c80}{就}};
\node[Onyomi] at (-13.400000, -29.600000) {シュウ};
\node[Kunyomi] at (-13.500000, -29.600000) {つ.く};
\node[Meaning] at (-13.450000, -27.950000) {settle in};
\node[Square] at (-11.400000, -29.700000) {};
\node[Kanji] at (-11.400000, -29.200000) {\textcolor[HTML]{0e254c}{懇}};
\node[Onyomi] at (-11.350000, -29.600000) {コン};
\node[Meaning] at (-11.400000, -27.950000) {courteous};
\node[Square] at (-9.350000, -29.700000) {};
\node[Kanji] at (-9.350000, -29.200000) {\textcolor[HTML]{0e254c}{墾}};
\node[Onyomi] at (-9.300000, -29.600000) {コン};
\node[Meaning] at (-9.350000, -27.950000) {break ground};
\node[Square] at (-7.300000, -29.700000) {};
\node[Kanji] at (-7.300000, -29.200000) {\textcolor[HTML]{133c80}{免}};
\node[Onyomi] at (-7.250000, -29.600000) {メン};
\node[Kunyomi] at (-7.350000, -29.600000) {まぬか.れる};
\node[Meaning] at (-7.300000, -27.950000) {excuse};
\node[Square] at (-5.250000, -29.700000) {};
\node[Kanji] at (-5.250000, -29.200000) {\textcolor[HTML]{14418e}{逸}};
\node[Onyomi] at (-5.200000, -29.600000) {イツ};
\node[Kunyomi] at (-5.300000, -29.600000) {そ};
\node[Meaning] at (-5.250000, -27.950000) {deviate};
\node[Square] at (-3.200000, -29.700000) {};
\node[Kanji] at (-3.200000, -29.200000) {\textcolor[HTML]{14469c}{晩}};
\node[Onyomi] at (-3.150000, -29.600000) {バン};
\node[Meaning] at (-3.200000, -27.950000) {night};
\node[Square] at (-1.150000, -29.700000) {};
\node[Kanji] at (-1.150000, -29.200000) {\textcolor[HTML]{154caa}{勉}};
\node[Onyomi] at (-1.100000, -29.600000) {ベン};
\node[Meaning] at (-1.150000, -27.950000) {exertion};
\node[Square] at (0.900000, -29.700000) {};
\node[Kanji] at (0.900000, -29.200000) {\textcolor[HTML]{154caa}{象}};
\node[Onyomi] at (0.950000, -29.600000) {ショウ};
\node[Meaning] at (0.900000, -27.950000) {elephant};
\node[Square] at (2.950000, -29.700000) {};
\node[Kanji] at (2.950000, -29.200000) {\textcolor[HTML]{1551b8}{像}};
\node[Onyomi] at (3.000000, -29.600000) {ゾウ};
\node[Meaning] at (2.950000, -27.950000) {statue};
\node[Square] at (5.000000, -29.700000) {};
\node[Kanji] at (5.000000, -29.200000) {\textcolor[HTML]{154caa}{馬}};
\node[Onyomi] at (5.050000, -29.600000) {バ};
\node[Kunyomi] at (4.950000, -29.600000) {うま};
\node[Meaning] at (5.000000, -27.950000) {horse};
\node[Square] at (7.050000, -29.700000) {};
\node[Kanji] at (7.050000, -29.200000) {\textcolor[HTML]{123673}{駒}};
\node[Onyomi] at (7.100000, -29.600000) {ク};
\node[Kunyomi] at (7.000000, -29.600000) {こま};
\node[Meaning] at (7.050000, -27.950000) {chess piece};
\node[Square] at (9.100000, -29.700000) {};
\node[Kanji] at (9.100000, -29.200000) {\textcolor[HTML]{1551b8}{験}};
\node[Onyomi] at (9.150000, -29.600000) {ケン};
\node[Kunyomi] at (9.050000, -29.600000) {ため};
\node[Meaning] at (9.100000, -27.950000) {test};
\node[Square] at (11.150000, -29.700000) {};
\node[Kanji] at (11.150000, -29.200000) {\textcolor[HTML]{14469c}{騎}};
\node[Onyomi] at (11.200000, -29.600000) {キ};
\node[Meaning] at (11.150000, -27.950000) {horse};
\node[Square] at (13.200000, -29.700000) {};
\node[Kanji] at (13.200000, -29.200000) {\textcolor[HTML]{123673}{駐}};
\node[Onyomi] at (13.250000, -29.600000) {チュウ};
\node[Meaning] at (13.200000, -27.950000) {resident};
\node[Square] at (15.250000, -29.700000) {};
\node[Kanji] at (15.250000, -29.200000) {\textcolor[HTML]{154caa}{駆}};
\node[Onyomi] at (15.300000, -29.600000) {ク};
\node[Kunyomi] at (15.200000, -29.600000) {か};
\node[Meaning] at (15.250000, -27.950000) {gallop};
\node[Square] at (17.300000, -29.700000) {};
\node[Kanji] at (17.300000, -29.200000) {\textcolor[HTML]{1551b8}{駅}};
\node[Onyomi] at (17.350000, -29.600000) {エキ};
\node[Meaning] at (17.300000, -27.950000) {station};
\node[Square] at (19.350000, -29.700000) {};
\node[Kanji] at (19.350000, -29.200000) {\textcolor[HTML]{14469c}{騒}};
\node[Onyomi] at (19.400000, -29.600000) {ソウ};
\node[Kunyomi] at (19.300000, -29.600000) {さわ.ぐ};
\node[Meaning] at (19.350000, -27.950000) {boisterous};
\node[Square] at (21.400000, -29.700000) {};
\node[Kanji] at (21.400000, -29.200000) {\textcolor[HTML]{113066}{駄}};
\node[Onyomi] at (21.450000, -29.600000) {ダ};
\node[Meaning] at (21.400000, -27.950000) {burdensome};
\node[Square] at (23.450000, -29.700000) {};
\node[Kanji] at (23.450000, -29.200000) {\textcolor[HTML]{1551b8}{驚}};
\node[Onyomi] at (23.500000, -29.600000) {キョウ};
\node[Kunyomi] at (23.400000, -29.600000) {おどろ.*};
\node[Meaning] at (23.450000, -27.950000) {surprised};
\node[Square] at (25.500000, -29.700000) {};
\node[Kanji] at (25.500000, -29.200000) {\textcolor[HTML]{0e254c}{篤}};
\node[Onyomi] at (25.550000, -29.600000) {トク};
\node[Kunyomi] at (25.450000, -29.600000) {あつ};
\node[Meaning] at (25.500000, -27.950000) {deliberate};
\node[Square] at (27.550000, -29.700000) {};
\node[Kanji] at (27.550000, -29.200000) {\textcolor[HTML]{0e254c}{騰}};
\node[Onyomi] at (27.600000, -29.600000) {トウ};
\node[Kunyomi] at (27.500000, -29.600000) {あが};
\node[Meaning] at (27.550000, -27.950000) {inflation};
\node[Square] at (29.600000, -29.700000) {};
\node[Kanji] at (29.600000, -29.200000) {\textcolor[HTML]{102b59}{虎}};
\node[Onyomi] at (29.650000, -29.600000) {コ};
\node[Kunyomi] at (29.550000, -29.600000) {とら};
\node[Meaning] at (29.600000, -27.950000) {tiger};
\node[Square] at (31.650000, -29.700000) {};
\node[Kanji] at (31.650000, -29.200000) {\textcolor[HTML]{123673}{虜}};
\node[Onyomi] at (31.700000, -29.600000) {リョ};
\node[Kunyomi] at (31.600000, -29.600000) {とりく};
\node[Meaning] at (31.650000, -27.950000) {captive};
\node[Square] at (33.700000, -29.700000) {};
\node[Kanji] at (33.700000, -29.200000) {\textcolor[HTML]{133c80}{膚}};
\node[Onyomi] at (33.750000, -29.600000) {フ};
\node[Kunyomi] at (33.650000, -29.600000) {はだ};
\node[Meaning] at (33.700000, -27.950000) {skin};
\node[Square] at (35.750000, -29.700000) {};
\node[Kanji] at (35.750000, -29.200000) {\textcolor[HTML]{133c80}{虚}};
\node[Onyomi] at (35.800000, -29.600000) {キョ};
\node[Kunyomi] at (35.700000, -29.600000) {むな.しい	};
\node[Meaning] at (35.750000, -27.950000) {void};
\node[Square] at (37.800000, -29.700000) {};
\node[Kanji] at (37.800000, -29.200000) {\textcolor[HTML]{133c80}{戯}};
\node[Onyomi] at (37.850000, -29.600000) {ギ};
\node[Kunyomi] at (37.750000, -29.600000) {ざ};
\node[Meaning] at (37.800000, -27.950000) {play};
\node[Square] at (39.850000, -29.700000) {};
\node[Kanji] at (39.850000, -29.200000) {\textcolor[HTML]{0e254c}{虞}};
\node[Kunyomi] at (39.800000, -29.600000) {おそれ};
\node[Meaning] at (39.850000, -27.950000) {uneasiness};
\node[Square] at (41.900000, -29.700000) {};
\node[Kanji] at (41.900000, -29.200000) {\textcolor[HTML]{133c80}{慮}};
\node[Onyomi] at (41.950000, -29.600000) {リョ};
\node[Kunyomi] at (41.850000, -29.600000) {おもんぱく};
\node[Meaning] at (41.900000, -27.950000) {consider};
\node[Square] at (43.950000, -29.700000) {};
\node[Kanji] at (43.950000, -29.200000) {\textcolor[HTML]{133c80}{劇}};
\node[Onyomi] at (44.000000, -29.600000) {ゲキ};
\node[Meaning] at (43.950000, -27.950000) {drama};
\node[Square] at (46.000000, -29.700000) {};
\node[Kanji] at (46.000000, -29.200000) {\textcolor[HTML]{123673}{虐}};
\node[Onyomi] at (46.050000, -29.600000) {ギャク};
\node[Kunyomi] at (45.950000, -29.600000) {しいた};
\node[Meaning] at (46.000000, -27.950000) {oppress};
\node[Square] at (48.050000, -29.700000) {};
\node[Kanji] at (48.050000, -29.200000) {\textcolor[HTML]{14469c}{鹿}};
\node[Onyomi] at (48.100000, -29.600000) {ロク};
\node[Kunyomi] at (48.000000, -29.600000) {か};
\node[Meaning] at (48.050000, -27.950000) {deer};
\node[Square] at (50.100000, -29.700000) {};
\node[Kanji] at (50.100000, -29.200000) {\textcolor[HTML]{102b59}{薦}};
\node[Onyomi] at (50.150000, -29.600000) {セン};
\node[Kunyomi] at (50.050000, -29.600000) {すす.*};
\node[Meaning] at (50.100000, -27.950000) {recommend};
\node[Square] at (52.150000, -29.700000) {};
\node[Kanji] at (52.150000, -29.200000) {\textcolor[HTML]{0e254c}{慶}};
\node[Onyomi] at (52.200000, -29.600000) {ケイ};
\node[Kunyomi] at (52.100000, -29.600000) {よろこ};
\node[Meaning] at (52.150000, -27.950000) {congratulate};
\node[Square] at (54.200000, -29.700000) {};
\node[Kanji] at (54.200000, -29.200000) {\textcolor[HTML]{113066}{麗}};
\node[Onyomi] at (54.250000, -29.600000) {レイ};
\node[Kunyomi] at (54.150000, -29.600000) {うるわ.しい};
\node[Meaning] at (54.200000, -27.950000) {lovely};
\node[Square] at (56.250000, -29.700000) {};
\node[Kanji] at (56.250000, -29.200000) {\textcolor[HTML]{14418e}{熊}};
\node[Kunyomi] at (56.200000, -29.600000) {くま};
\node[Meaning] at (56.250000, -27.950000) {bear};
\node[Meaning] at (-58.500000, -29.150000) {95.61\%};
\node[Square] at (-56.500000, -31.750000) {};
\node[Kanji] at (-56.500000, -31.250000) {\textcolor[HTML]{1551b8}{能}};
\node[Onyomi] at (-56.450000, -31.650000) {ノウ};
\node[Meaning] at (-56.500000, -30.000000) {ability};
\node[Square] at (-54.450000, -31.750000) {};
\node[Kanji] at (-54.450000, -31.250000) {\textcolor[HTML]{154caa}{態}};
\node[Onyomi] at (-54.400000, -31.650000) {タイ};
\node[Kunyomi] at (-54.500000, -31.650000) {わざ};
\node[Meaning] at (-54.450000, -30.000000) {appearance};
\node[Square] at (-52.400000, -31.750000) {};
\node[Kanji] at (-52.400000, -31.250000) {\textcolor[HTML]{14469c}{演}};
\node[Onyomi] at (-52.350000, -31.650000) {エン};
\node[Meaning] at (-52.400000, -30.000000) {perform};
\node[Square] at (-50.350000, -31.750000) {};
\node[Kanji] at (-50.350000, -31.250000) {\textcolor[HTML]{133c80}{辱}};
\node[Onyomi] at (-50.300000, -31.650000) {ジョク};
\node[Kunyomi] at (-50.400000, -31.650000) {はずかし.める};
\node[Meaning] at (-50.350000, -30.000000) {humiliate};
\node[Square] at (-48.300000, -31.750000) {};
\node[Kanji] at (-48.300000, -31.250000) {\textcolor[HTML]{1557c6}{震}};
\node[Onyomi] at (-48.250000, -31.650000) {シン};
\node[Kunyomi] at (-48.350000, -31.650000) {ふる.える};
\node[Meaning] at (-48.300000, -30.000000) {earthquake};
\node[Square] at (-46.250000, -31.750000) {};
\node[Kanji] at (-46.250000, -31.250000) {\textcolor[HTML]{145cd5}{振}};
\node[Onyomi] at (-46.200000, -31.650000) {シン};
\node[Kunyomi] at (-46.300000, -31.650000) {ふ.る};
\node[Meaning] at (-46.250000, -30.000000) {shake};
\node[Square] at (-44.200000, -31.750000) {};
\node[Kanji] at (-44.200000, -31.250000) {\textcolor[HTML]{123673}{娠}};
\node[Onyomi] at (-44.150000, -31.650000) {シン};
\node[Meaning] at (-44.200000, -30.000000) {pregnant};
\node[Square] at (-42.150000, -31.750000) {};
\node[Kanji] at (-42.150000, -31.250000) {\textcolor[HTML]{14418e}{唇}};
\node[Onyomi] at (-42.100000, -31.650000) {シン};
\node[Kunyomi] at (-42.200000, -31.650000) {くちびる};
\node[Meaning] at (-42.150000, -30.000000) {lips};
\node[Square] at (-40.100000, -31.750000) {};
\node[Kanji] at (-40.100000, -31.250000) {\textcolor[HTML]{14418e}{農}};
\node[Onyomi] at (-40.050000, -31.650000) {ノウ};
\node[Meaning] at (-40.100000, -30.000000) {farming};
\node[Square] at (-38.050000, -31.750000) {};
\node[Kanji] at (-38.050000, -31.250000) {\textcolor[HTML]{14418e}{濃}};
\node[Onyomi] at (-38.000000, -31.650000) {ノウ};
\node[Kunyomi] at (-38.100000, -31.650000) {こ.い};
\node[Meaning] at (-38.050000, -30.000000) {thick};
\node[Square] at (-36.000000, -31.750000) {};
\node[Kanji] at (-36.000000, -31.250000) {\textcolor[HTML]{1551b8}{送}};
\node[Onyomi] at (-35.950000, -31.650000) {ソウ};
\node[Kunyomi] at (-36.050000, -31.650000) {おく.る};
\node[Meaning] at (-36.000000, -30.000000) {send};
\node[Square] at (-33.950000, -31.750000) {};
\node[Kanji] at (-33.950000, -31.250000) {\textcolor[HTML]{1557c6}{関}};
\node[Onyomi] at (-33.900000, -31.650000) {カン};
\node[Kunyomi] at (-34.000000, -31.650000) {かか.わる};
\node[Meaning] at (-33.950000, -30.000000) {related};
\node[Square] at (-31.900000, -31.750000) {};
\node[Kanji] at (-31.900000, -31.250000) {\textcolor[HTML]{14418e}{咲}};
\node[Onyomi] at (-31.850000, -31.650000) {ショウ};
\node[Kunyomi] at (-31.950000, -31.650000) {さ};
\node[Meaning] at (-31.900000, -30.000000) {blossom};
\node[Square] at (-29.850000, -31.750000) {};
\node[Kanji] at (-29.850000, -31.250000) {\textcolor[HTML]{1551b8}{鬼}};
\node[Onyomi] at (-29.800000, -31.650000) {キ};
\node[Kunyomi] at (-29.900000, -31.650000) {おに};
\node[Meaning] at (-29.850000, -30.000000) {demon};
\node[Square] at (-27.800000, -31.750000) {};
\node[Kanji] at (-27.800000, -31.250000) {\textcolor[HTML]{133c80}{醜}};
\node[Onyomi] at (-27.750000, -31.650000) {シュウ};
\node[Kunyomi] at (-27.850000, -31.650000) {しこ};
\node[Meaning] at (-27.800000, -30.000000) {ugly};
\node[Square] at (-25.750000, -31.750000) {};
\node[Kanji] at (-25.750000, -31.250000) {\textcolor[HTML]{1551b8}{魂}};
\node[Onyomi] at (-25.700000, -31.650000) {コン};
\node[Kunyomi] at (-25.800000, -31.650000) {たましい};
\node[Meaning] at (-25.750000, -30.000000) {soul};
\node[Square] at (-23.700000, -31.750000) {};
\node[Kanji] at (-23.700000, -31.250000) {\textcolor[HTML]{1461e3}{魔}};
\node[Onyomi] at (-23.650000, -31.650000) {マ};
\node[Meaning] at (-23.700000, -30.000000) {devil};
\node[Square] at (-21.650000, -31.750000) {};
\node[Kanji] at (-21.650000, -31.250000) {\textcolor[HTML]{123673}{魅}};
\node[Onyomi] at (-21.600000, -31.650000) {ミ};
\node[Meaning] at (-21.650000, -30.000000) {alluring};
\node[Square] at (-19.600000, -31.750000) {};
\node[Kanji] at (-19.600000, -31.250000) {\textcolor[HTML]{133c80}{塊}};
\node[Onyomi] at (-19.550000, -31.650000) {カイ};
\node[Kunyomi] at (-19.650000, -31.650000) {かたまり};
\node[Meaning] at (-19.600000, -30.000000) {lump};
\node[Square] at (-17.550000, -31.750000) {};
\node[Kanji] at (-17.550000, -31.250000) {\textcolor[HTML]{154caa}{襲}};
\node[Onyomi] at (-17.500000, -31.650000) {シュウ};
\node[Kunyomi] at (-17.600000, -31.650000) {おそ.う};
\node[Meaning] at (-17.550000, -30.000000) {attack};
\node[Square] at (-15.500000, -31.750000) {};
\node[Kanji] at (-15.500000, -31.250000) {\textcolor[HTML]{102b59}{嚇}};
\node[Onyomi] at (-15.450000, -31.650000) {カク};
\node[Meaning] at (-15.500000, -30.000000) {menacing};
\node[Square] at (-13.450000, -31.750000) {};
\node[Kanji] at (-13.450000, -31.250000) {\textcolor[HTML]{0e254c}{朕}};
\node[Onyomi] at (-13.400000, -31.650000) {チン};
\node[Meaning] at (-13.450000, -30.000000) {majestic plural};
\node[Square] at (-11.400000, -31.750000) {};
\node[Kanji] at (-11.400000, -31.250000) {\textcolor[HTML]{133c80}{雰}};
\node[Onyomi] at (-11.350000, -31.650000) {フン};
\node[Meaning] at (-11.400000, -30.000000) {atmosphere};
\node[Square] at (-9.350000, -31.750000) {};
\node[Kanji] at (-9.350000, -31.250000) {\textcolor[HTML]{102b59}{箇}};
\node[Onyomi] at (-9.300000, -31.650000) {カ};
\node[Meaning] at (-9.350000, -30.000000) {counters};
\node[Square] at (-7.300000, -31.750000) {};
\node[Kanji] at (-7.300000, -31.250000) {\textcolor[HTML]{0e254c}{錬}};
\node[Onyomi] at (-7.250000, -31.650000) {レン};
\node[Kunyomi] at (-7.350000, -31.650000) {ね};
\node[Meaning] at (-7.300000, -30.000000) {tempering};
\node[Square] at (-5.250000, -31.750000) {};
\node[Kanji] at (-5.250000, -31.250000) {\textcolor[HTML]{0e254c}{遵}};
\node[Onyomi] at (-5.200000, -31.650000) {ジュン};
\node[Meaning] at (-5.250000, -30.000000) {abide by};
\node[Square] at (-3.200000, -31.750000) {};
\node[Kanji] at (-3.200000, -31.250000) {\textcolor[HTML]{0e254c}{罷}};
\node[Onyomi] at (-3.150000, -31.650000) {ヒ};
\node[Kunyomi] at (-3.250000, -31.650000) {や};
\node[Meaning] at (-3.200000, -30.000000) {quit};
\node[Square] at (-1.150000, -31.750000) {};
\node[Kanji] at (-1.150000, -31.250000) {\textcolor[HTML]{0e254c}{屯}};
\node[Onyomi] at (-1.100000, -31.650000) {トン};
\node[Meaning] at (-1.150000, -30.000000) {barracks};
\node[Square] at (0.900000, -31.750000) {};
\node[Kanji] at (0.900000, -31.250000) {\textcolor[HTML]{0e254c}{且}};
\node[Onyomi] at (0.950000, -31.650000) {ショ};
\node[Kunyomi] at (0.850000, -31.650000) {か};
\node[Meaning] at (0.900000, -30.000000) {also};
\node[Square] at (2.950000, -31.750000) {};
\node[Kanji] at (2.950000, -31.250000) {\textcolor[HTML]{0e254c}{藻}};
\node[Onyomi] at (3.000000, -31.650000) {ソウ};
\node[Kunyomi] at (2.900000, -31.650000) {も};
\node[Meaning] at (2.950000, -30.000000) {seaweed};
\node[Square] at (5.000000, -31.750000) {};
\node[Kanji] at (5.000000, -31.250000) {\textcolor[HTML]{113066}{隷}};
\node[Onyomi] at (5.050000, -31.650000) {レイ};
\node[Meaning] at (5.000000, -30.000000) {slave};
\node[Square] at (7.050000, -31.750000) {};
\node[Kanji] at (7.050000, -31.250000) {\textcolor[HTML]{133c80}{癒}};
\node[Onyomi] at (7.100000, -31.650000) {ユ};
\node[Kunyomi] at (7.000000, -31.650000) {い};
\node[Meaning] at (7.050000, -30.000000) {healing};
\node[Square] at (9.100000, -31.750000) {};
\node[Kanji] at (9.100000, -31.250000) {\textcolor[HTML]{113066}{丹}};
\node[Onyomi] at (9.150000, -31.650000) {タン};
\node[Kunyomi] at (9.050000, -31.650000) {に};
\node[Meaning] at (9.100000, -30.000000) {rust colored};
\node[Square] at (11.150000, -31.750000) {};
\node[Kanji] at (11.150000, -31.250000) {\textcolor[HTML]{123673}{潟}};
\node[Onyomi] at (11.200000, -31.650000) {セキ};
\node[Kunyomi] at (11.100000, -31.650000) {かた};
\node[Meaning] at (11.150000, -30.000000) {lagoon};
\node[Square] at (13.200000, -31.750000) {};
\node[Kanji] at (13.200000, -31.250000) {\textcolor[HTML]{0e254c}{柴}};
\node[Onyomi] at (13.250000, -31.650000) {サイ};
\node[Kunyomi] at (13.150000, -31.650000) {しば};
\node[Meaning] at (13.200000, -30.000000) {brushwood};
\node[Square] at (15.250000, -31.750000) {};
\node[Kanji] at (15.250000, -31.250000) {\textcolor[HTML]{0e254c}{璃}};
\node[Onyomi] at (15.300000, -31.650000) {リ};
\node[Meaning] at (15.250000, -30.000000) {glassy};
\node[Square] at (17.300000, -31.750000) {};
\node[Kanji] at (17.300000, -31.250000) {\textcolor[HTML]{1557c6}{俺}};
\node[Kunyomi] at (17.250000, -31.650000) {おれ};
\node[Meaning] at (17.300000, -30.000000) {i};
\node[Square] at (19.350000, -31.750000) {};
\node[Kanji] at (19.350000, -31.250000) {\textcolor[HTML]{0e254c}{駿}};
\node[Onyomi] at (19.400000, -31.650000) {シュン};
\node[Kunyomi] at (19.300000, -31.650000) {すぐ};
\node[Meaning] at (19.350000, -30.000000) {speed};
\node[Square] at (21.400000, -31.750000) {};
\node[Kanji] at (21.400000, -31.250000) {\textcolor[HTML]{0e254c}{臼}};
\node[Onyomi] at (21.450000, -31.650000) {キュウ};
\node[Kunyomi] at (21.350000, -31.650000) {うす};
\node[Meaning] at (21.400000, -30.000000) {mortar};
\node[Square] at (23.450000, -31.750000) {};
\node[Kanji] at (23.450000, -31.250000) {\textcolor[HTML]{0e254c}{毀}};
\node[Onyomi] at (23.500000, -31.650000) {キ};
\node[Meaning] at (23.450000, -30.000000) {destroy};
\node[Square] at (25.500000, -31.750000) {};
\node[Kanji] at (25.500000, -31.250000) {\textcolor[HTML]{0e254c}{脊}};
\node[Onyomi] at (25.550000, -31.650000) {セキ};
\node[Kunyomi] at (25.450000, -31.650000) {せせい};
\node[Meaning] at (25.500000, -30.000000) {stature};
\node[Square] at (27.550000, -31.750000) {};
\node[Kanji] at (27.550000, -31.250000) {\textcolor[HTML]{0e254c}{璽}};
\node[Onyomi] at (27.600000, -31.650000) {ジ};
\node[Meaning] at (27.550000, -30.000000) {emperor's seal};
\node[Square] at (29.600000, -31.750000) {};
\node[Kanji] at (29.600000, -31.250000) {\textcolor[HTML]{154caa}{妖}};
\node[Onyomi] at (29.650000, -31.650000) {ヨウ};
\node[Kunyomi] at (29.550000, -31.650000) {あや-しい};
\node[Meaning] at (29.600000, -30.000000) {bewitching};
\node[Square] at (31.650000, -31.750000) {};
\node[Kanji] at (31.650000, -31.250000) {\textcolor[HTML]{0e254c}{沃}};
\node[Onyomi] at (31.700000, -31.650000) {ヨク};
\node[Meaning] at (31.650000, -30.000000) {fertility};
\node[Square] at (33.700000, -31.750000) {};
\node[Kanji] at (33.700000, -31.250000) {\textcolor[HTML]{113066}{稽}};
\node[Onyomi] at (33.750000, -31.650000) {ケイ};
\node[Meaning] at (33.700000, -30.000000) {consider};
\node[Square] at (35.750000, -31.750000) {};
\node[Kanji] at (35.750000, -31.250000) {\textcolor[HTML]{102b59}{采}};
\node[Onyomi] at (35.800000, -31.650000) {サイ};
\node[Meaning] at (35.750000, -30.000000) {form};
\node[Square] at (37.800000, -31.750000) {};
\node[Kanji] at (37.800000, -31.250000) {\textcolor[HTML]{0e254c}{斬}};
\node[Onyomi] at (37.850000, -31.650000) {ザン};
\node[Kunyomi] at (37.750000, -31.650000) {き.る};
\node[Meaning] at (37.800000, -30.000000) {slice};
\node[Square] at (39.850000, -31.750000) {};
\node[Kanji] at (39.850000, -31.250000) {\textcolor[HTML]{113066}{也}};
\node[Kunyomi] at (39.800000, -31.650000) {なり};
\node[Meaning] at (39.850000, -30.000000) {considerably};
\node[Square] at (41.900000, -31.750000) {};
\node[Kanji] at (41.900000, -31.250000) {\textcolor[HTML]{133c80}{巾}};
\node[Onyomi] at (41.950000, -31.650000) {キン};
\node[Meaning] at (41.900000, -30.000000) {towel};
\node[Square] at (43.950000, -31.750000) {};
\node[Kanji] at (43.950000, -31.250000) {\textcolor[HTML]{0e254c}{僅}};
\node[Onyomi] at (44.000000, -31.650000) {キン};
\node[Kunyomi] at (43.900000, -31.650000) {わず-か};
\node[Meaning] at (43.950000, -30.000000) {a wee bit};
\node[Square] at (46.000000, -31.750000) {};
\node[Kanji] at (46.000000, -31.250000) {\textcolor[HTML]{0e254c}{侶}};
\node[Onyomi] at (46.050000, -31.650000) {リョ};
\node[Meaning] at (46.000000, -30.000000) {companion};
\node[Square] at (48.050000, -31.750000) {};
\node[Kanji] at (48.050000, -31.250000) {\textcolor[HTML]{123673}{伎}};
\node[Onyomi] at (48.100000, -31.650000) {キ};
\node[Kunyomi] at (48.000000, -31.650000) {わざ};
\node[Meaning] at (48.050000, -30.000000) {deed};
\node[Square] at (50.100000, -31.750000) {};
\node[Kanji] at (50.100000, -31.250000) {\textcolor[HTML]{113066}{凄}};
\node[Onyomi] at (50.150000, -31.650000) {セイ};
\node[Meaning] at (50.100000, -30.000000) {uncanny};
\node[Square] at (52.150000, -31.750000) {};
\node[Kanji] at (52.150000, -31.250000) {\textcolor[HTML]{102b59}{凌}};
\node[Onyomi] at (52.200000, -31.650000) {リョウ};
\node[Kunyomi] at (52.100000, -31.650000) {しの};
\node[Meaning] at (52.150000, -30.000000) {endure};
\node[Square] at (54.200000, -31.750000) {};
\node[Kanji] at (54.200000, -31.250000) {\textcolor[HTML]{0e254c}{冶}};
\node[Onyomi] at (54.250000, -31.650000) {ヤ};
\node[Meaning] at (54.200000, -30.000000) {melting};
\node[Square] at (56.250000, -31.750000) {};
\node[Kanji] at (56.250000, -31.250000) {\textcolor[HTML]{0e254c}{凛}};
\node[Onyomi] at (56.300000, -31.650000) {リン};
\node[Kunyomi] at (56.200000, -31.650000) {きびし};
\node[Meaning] at (56.250000, -30.000000) {cold};
\node[Meaning] at (-58.500000, -31.200000) {97.10\%};
\node[Square] at (-56.500000, -33.800000) {};
\node[Kanji] at (-56.500000, -33.300000) {\textcolor[HTML]{0e254c}{刹}};
\node[Onyomi] at (-56.450000, -33.700000) {サツ};
\node[Meaning] at (-56.500000, -32.050000) {temple};
\node[Square] at (-54.450000, -33.800000) {};
\node[Kanji] at (-54.450000, -33.300000) {\textcolor[HTML]{133c80}{剥}};
\node[Onyomi] at (-54.400000, -33.700000) {ハク};
\node[Kunyomi] at (-54.500000, -33.700000) {は-がす};
\node[Meaning] at (-54.450000, -32.050000) {peel};
\node[Square] at (-52.400000, -33.800000) {};
\node[Kanji] at (-52.400000, -33.300000) {\textcolor[HTML]{133c80}{匂}};
\node[Kunyomi] at (-52.450000, -33.700000) {にお-う};
\node[Meaning] at (-52.400000, -32.050000) {scent};
\node[Square] at (-50.350000, -33.800000) {};
\node[Kanji] at (-50.350000, -33.300000) {\textcolor[HTML]{0e254c}{勾}};
\node[Onyomi] at (-50.300000, -33.700000) {コウ};
\node[Meaning] at (-50.350000, -32.050000) {capture};
\node[Square] at (-48.300000, -33.800000) {};
\node[Kanji] at (-48.300000, -33.300000) {\textcolor[HTML]{133c80}{嘲}};
\node[Onyomi] at (-48.250000, -33.700000) {チョウ};
\node[Kunyomi] at (-48.350000, -33.700000) {あざけ-る};
\node[Meaning] at (-48.300000, -32.050000) {ridicule};
\node[Square] at (-46.250000, -33.800000) {};
\node[Kanji] at (-46.250000, -33.300000) {\textcolor[HTML]{123673}{咽}};
\node[Onyomi] at (-46.200000, -33.700000) {イン};
\node[Meaning] at (-46.250000, -32.050000) {throat};
\node[Square] at (-44.200000, -33.800000) {};
\node[Kanji] at (-44.200000, -33.300000) {\textcolor[HTML]{14469c}{喉}};
\node[Onyomi] at (-44.150000, -33.700000) {コウ};
\node[Kunyomi] at (-44.250000, -33.700000) {のど};
\node[Meaning] at (-44.200000, -32.050000) {throat};
\node[Square] at (-42.150000, -33.800000) {};
\node[Kanji] at (-42.150000, -33.300000) {\textcolor[HTML]{133c80}{唾}};
\node[Onyomi] at (-42.100000, -33.700000) {ダ};
\node[Kunyomi] at (-42.200000, -33.700000) {つば};
\node[Meaning] at (-42.150000, -32.050000) {saliva};
\node[Square] at (-40.100000, -33.800000) {};
\node[Kanji] at (-40.100000, -33.300000) {\textcolor[HTML]{1557c6}{呪}};
\node[Kunyomi] at (-40.150000, -33.700000) {のろ};
\node[Meaning] at (-40.100000, -32.050000) {curse};
\node[Square] at (-38.050000, -33.800000) {};
\node[Kanji] at (-38.050000, -33.300000) {\textcolor[HTML]{123673}{噌}};
\node[Onyomi] at (-38.000000, -33.700000) {ソ};
\node[Meaning] at (-38.050000, -32.050000) {boisterous};
\node[Square] at (-36.000000, -33.800000) {};
\node[Kanji] at (-36.000000, -33.300000) {\textcolor[HTML]{0e254c}{唄}};
\node[Onyomi] at (-35.950000, -33.700000) {バイ};
\node[Kunyomi] at (-36.050000, -33.700000) {うた};
\node[Meaning] at (-36.000000, -32.050000) {shamisen song};
\node[Square] at (-33.950000, -33.800000) {};
\node[Kanji] at (-33.950000, -33.300000) {\textcolor[HTML]{123673}{叱}};
\node[Kunyomi] at (-34.000000, -33.700000) {しか};
\node[Meaning] at (-33.950000, -32.050000) {scold};
\node[Square] at (-31.900000, -33.800000) {};
\node[Kanji] at (-31.900000, -33.300000) {\textcolor[HTML]{14418e}{呆}};
\node[Onyomi] at (-31.850000, -33.700000) {ホウ};
\node[Kunyomi] at (-31.950000, -33.700000) {あき};
\node[Meaning] at (-31.900000, -32.050000) {shock};
\node[Square] at (-29.850000, -33.800000) {};
\node[Kanji] at (-29.850000, -33.300000) {\textcolor[HTML]{0e254c}{堆}};
\node[Onyomi] at (-29.800000, -33.700000) {タイ};
\node[Meaning] at (-29.850000, -32.050000) {piled high};
\node[Square] at (-27.800000, -33.800000) {};
\node[Kanji] at (-27.800000, -33.300000) {\textcolor[HTML]{0e254c}{填}};
\node[Onyomi] at (-27.750000, -33.700000) {テン};
\node[Meaning] at (-27.800000, -32.050000) {fill in};
\node[Square] at (-25.750000, -33.800000) {};
\node[Kanji] at (-25.750000, -33.300000) {\textcolor[HTML]{0e254c}{堰}};
\node[Onyomi] at (-25.700000, -33.700000) {セキ};
\node[Meaning] at (-25.750000, -32.050000) {dam};
\node[Square] at (-23.700000, -33.800000) {};
\node[Kanji] at (-23.700000, -33.300000) {\textcolor[HTML]{113066}{妬}};
\node[Onyomi] at (-23.650000, -33.700000) {ト};
\node[Kunyomi] at (-23.750000, -33.700000) {ねた-む};
\node[Meaning] at (-23.700000, -32.050000) {jealousy};
\node[Square] at (-21.650000, -33.800000) {};
\node[Kanji] at (-21.650000, -33.300000) {\textcolor[HTML]{102b59}{嫉}};
\node[Onyomi] at (-21.600000, -33.700000) {シツ};
\node[Meaning] at (-21.650000, -32.050000) {envy};
\node[Square] at (-19.600000, -33.800000) {};
\node[Kanji] at (-19.600000, -33.300000) {\textcolor[HTML]{133c80}{塞}};
\node[Onyomi] at (-19.550000, -33.700000) {サイ};
\node[Kunyomi] at (-19.650000, -33.700000) {ふさ-ぐ};
\node[Meaning] at (-19.600000, -32.050000) {obstruct};
\node[Square] at (-17.550000, -33.800000) {};
\node[Kanji] at (-17.550000, -33.300000) {\textcolor[HTML]{14418e}{尻}};
\node[Kunyomi] at (-17.600000, -33.700000) {しり};
\node[Meaning] at (-17.550000, -32.050000) {butt};
\node[Square] at (-15.500000, -33.800000) {};
\node[Kanji] at (-15.500000, -33.300000) {\textcolor[HTML]{123673}{崖}};
\node[Onyomi] at (-15.450000, -33.700000) {ガイ};
\node[Kunyomi] at (-15.550000, -33.700000) {がけ};
\node[Meaning] at (-15.500000, -32.050000) {cliff};
\node[Square] at (-13.450000, -33.800000) {};
\node[Kanji] at (-13.450000, -33.300000) {\textcolor[HTML]{0e254c}{庄}};
\node[Onyomi] at (-13.400000, -33.700000) {ショウ};
\node[Meaning] at (-13.450000, -32.050000) {manor};
\node[Square] at (-11.400000, -33.800000) {};
\node[Kanji] at (-11.400000, -33.300000) {\textcolor[HTML]{0e254c}{弥}};
\node[Onyomi] at (-11.350000, -33.700000) {ビ};
\node[Kunyomi] at (-11.450000, -33.700000) {や};
\node[Meaning] at (-11.400000, -32.050000) {increasing};
\node[Square] at (-9.350000, -33.800000) {};
\node[Kanji] at (-9.350000, -33.300000) {\textcolor[HTML]{14469c}{挨}};
\node[Onyomi] at (-9.300000, -33.700000) {アイ};
\node[Meaning] at (-9.350000, -32.050000) {push open};
\node[Square] at (-7.300000, -33.800000) {};
\node[Kanji] at (-7.300000, -33.300000) {\textcolor[HTML]{123673}{捻}};
\node[Onyomi] at (-7.250000, -33.700000) {ネン};
\node[Kunyomi] at (-7.350000, -33.700000) {ひね-る};
\node[Meaning] at (-7.300000, -32.050000) {twist};
\node[Square] at (-5.250000, -33.800000) {};
\node[Kanji] at (-5.250000, -33.300000) {\textcolor[HTML]{14418e}{拭}};
\node[Onyomi] at (-5.200000, -33.700000) {ショク};
\node[Kunyomi] at (-5.300000, -33.700000) {ふ-く};
\node[Meaning] at (-5.250000, -32.050000) {wipe};
\node[Square] at (-3.200000, -33.800000) {};
\node[Kanji] at (-3.200000, -33.300000) {\textcolor[HTML]{0e254c}{捉}};
\node[Onyomi] at (-3.150000, -33.700000) {ソク};
\node[Kunyomi] at (-3.250000, -33.700000) {とら-える};
\node[Meaning] at (-3.200000, -32.050000) {capture};
\node[Square] at (-1.150000, -33.800000) {};
\node[Kanji] at (-1.150000, -33.300000) {\textcolor[HTML]{14469c}{拶}};
\node[Onyomi] at (-1.100000, -33.700000) {サツ};
\node[Meaning] at (-1.150000, -32.050000) {be imminent};
\node[Square] at (0.900000, -33.800000) {};
\node[Kanji] at (0.900000, -33.300000) {\textcolor[HTML]{0e254c}{捗}};
\node[Onyomi] at (0.950000, -33.700000) {チョク};
\node[Meaning] at (0.900000, -32.050000) {make progress};
\node[Square] at (2.950000, -33.800000) {};
\node[Kanji] at (2.950000, -33.300000) {\textcolor[HTML]{102b59}{憧}};
\node[Onyomi] at (3.000000, -33.700000) {ショウ};
\node[Kunyomi] at (2.900000, -33.700000) {あこが};
\node[Meaning] at (2.950000, -32.050000) {long for};
\node[Square] at (5.000000, -33.800000) {};
\node[Kanji] at (5.000000, -33.300000) {\textcolor[HTML]{133c80}{湧}};
\node[Onyomi] at (5.050000, -33.700000) {ユウ};
\node[Kunyomi] at (4.950000, -33.700000) {わ};
\node[Meaning] at (5.000000, -32.050000) {well};
\node[Square] at (7.050000, -33.800000) {};
\node[Kanji] at (7.050000, -33.300000) {\textcolor[HTML]{113066}{沙}};
\node[Onyomi] at (7.100000, -33.700000) {サ};
\node[Kunyomi] at (7.000000, -33.700000) {すな};
\node[Meaning] at (7.050000, -32.050000) {sand};
\node[Square] at (9.100000, -33.800000) {};
\node[Kanji] at (9.100000, -33.300000) {\textcolor[HTML]{0e254c}{淫}};
\node[Onyomi] at (9.150000, -33.700000) {イン};
\node[Kunyomi] at (9.050000, -33.700000) {みだ-ら};
\node[Meaning] at (9.100000, -32.050000) {lewdness};
\node[Square] at (11.150000, -33.800000) {};
\node[Kanji] at (11.150000, -33.300000) {\textcolor[HTML]{0e254c}{氾}};
\node[Onyomi] at (11.200000, -33.700000) {ハン};
\node[Meaning] at (11.150000, -32.050000) {spread out};
\node[Square] at (13.200000, -33.800000) {};
\node[Kanji] at (13.200000, -33.300000) {\textcolor[HTML]{113066}{溺}};
\node[Onyomi] at (13.250000, -33.700000) {デキ};
\node[Kunyomi] at (13.150000, -33.700000) {おぼ-れる};
\node[Meaning] at (13.200000, -32.050000) {drown};
\node[Square] at (15.250000, -33.800000) {};
\node[Kanji] at (15.250000, -33.300000) {\textcolor[HTML]{102b59}{汰}};
\node[Onyomi] at (15.300000, -33.700000) {タ};
\node[Kunyomi] at (15.200000, -33.700000) {おご};
\node[Meaning] at (15.250000, -32.050000) {select};
\node[Square] at (17.300000, -33.800000) {};
\node[Kanji] at (17.300000, -33.300000) {\textcolor[HTML]{102b59}{潰}};
\node[Onyomi] at (17.350000, -33.700000) {カイ};
\node[Kunyomi] at (17.250000, -33.700000) {つぶ-す};
\node[Meaning] at (17.300000, -32.050000) {crush};
\node[Square] at (19.350000, -33.800000) {};
\node[Kanji] at (19.350000, -33.300000) {\textcolor[HTML]{0e254c}{汎}};
\node[Onyomi] at (19.400000, -33.700000) {ハン};
\node[Meaning] at (19.350000, -32.050000) {pan-};
\node[Square] at (21.400000, -33.800000) {};
\node[Kanji] at (21.400000, -33.300000) {\textcolor[HTML]{0e254c}{淀}};
\node[Kunyomi] at (21.350000, -33.700000) {よど};
\node[Meaning] at (21.400000, -32.050000) {eddy};
\node[Square] at (23.450000, -33.800000) {};
\node[Kanji] at (23.450000, -33.300000) {\textcolor[HTML]{102b59}{釜}};
\node[Kunyomi] at (23.400000, -33.700000) {かま};
\node[Meaning] at (23.450000, -32.050000) {kettle};
\node[Square] at (25.500000, -33.800000) {};
\node[Kanji] at (25.500000, -33.300000) {\textcolor[HTML]{102b59}{狐}};
\node[Onyomi] at (25.550000, -33.700000) {コ};
\node[Kunyomi] at (25.450000, -33.700000) {きつね};
\node[Meaning] at (25.500000, -32.050000) {fox};
\node[Square] at (27.550000, -33.800000) {};
\node[Kanji] at (27.550000, -33.300000) {\textcolor[HTML]{14418e}{狙}};
\node[Onyomi] at (27.600000, -33.700000) {ソ};
\node[Kunyomi] at (27.500000, -33.700000) {ねら.い};
\node[Meaning] at (27.550000, -32.050000) {aim};
\node[Square] at (29.600000, -33.800000) {};
\node[Kanji] at (29.600000, -33.300000) {\textcolor[HTML]{0e254c}{莉}};
\node[Onyomi] at (29.650000, -33.700000) {リ};
\node[Meaning] at (29.600000, -32.050000) {jasmine};
\node[Square] at (31.650000, -33.800000) {};
\node[Kanji] at (31.650000, -33.300000) {\textcolor[HTML]{133c80}{萎}};
\node[Onyomi] at (31.700000, -33.700000) {イ};
\node[Kunyomi] at (31.600000, -33.700000) {な-える};
\node[Meaning] at (31.650000, -32.050000) {wither};
\node[Square] at (33.700000, -33.800000) {};
\node[Kanji] at (33.700000, -33.300000) {\textcolor[HTML]{0e254c}{蔽}};
\node[Onyomi] at (33.750000, -33.700000) {ヘイ};
\node[Meaning] at (33.700000, -32.050000) {cover};
\node[Square] at (35.750000, -33.800000) {};
\node[Kanji] at (35.750000, -33.300000) {\textcolor[HTML]{0e254c}{蓮}};
\node[Onyomi] at (35.800000, -33.700000) {レン};
\node[Kunyomi] at (35.700000, -33.700000) {はす};
\node[Meaning] at (35.750000, -32.050000) {lotus};
\node[Square] at (37.800000, -33.800000) {};
\node[Kanji] at (37.800000, -33.300000) {\textcolor[HTML]{113066}{芯}};
\node[Onyomi] at (37.850000, -33.700000) {シン};
\node[Meaning] at (37.800000, -32.050000) {wick};
\node[Square] at (39.850000, -33.800000) {};
\node[Kanji] at (39.850000, -33.300000) {\textcolor[HTML]{0e254c}{藍}};
\node[Onyomi] at (39.900000, -33.700000) {ラン};
\node[Kunyomi] at (39.800000, -33.700000) {あい};
\node[Meaning] at (39.850000, -32.050000) {indigo};
\node[Square] at (41.900000, -33.800000) {};
\node[Kanji] at (41.900000, -33.300000) {\textcolor[HTML]{133c80}{苛}};
\node[Onyomi] at (41.950000, -33.700000) {カ};
\node[Meaning] at (41.900000, -32.050000) {torment};
\node[Square] at (43.950000, -33.800000) {};
\node[Kanji] at (43.950000, -33.300000) {\textcolor[HTML]{0e254c}{萌}};
\node[Onyomi] at (44.000000, -33.700000) {ホウ};
\node[Kunyomi] at (43.900000, -33.700000) {きざ};
\node[Meaning] at (43.950000, -32.050000) {sprout};
\node[Square] at (46.000000, -33.800000) {};
\node[Kanji] at (46.000000, -33.300000) {\textcolor[HTML]{0e254c}{蒙}};
\node[Onyomi] at (46.050000, -33.700000) {モウ};
\node[Kunyomi] at (45.950000, -33.700000) {おお};
\node[Meaning] at (46.000000, -32.050000) {darkness};
\node[Square] at (48.050000, -33.800000) {};
\node[Kanji] at (48.050000, -33.300000) {\textcolor[HTML]{133c80}{蓋}};
\node[Onyomi] at (48.100000, -33.700000) {ガイ};
\node[Kunyomi] at (48.000000, -33.700000) {ふた};
\node[Meaning] at (48.050000, -32.050000) {cover};
\node[Square] at (50.100000, -33.800000) {};
\node[Kanji] at (50.100000, -33.300000) {\textcolor[HTML]{123673}{蔑}};
\node[Onyomi] at (50.150000, -33.700000) {ベツ};
\node[Kunyomi] at (50.050000, -33.700000) {さげす};
\node[Meaning] at (50.100000, -32.050000) {scorn};
\node[Square] at (52.150000, -33.800000) {};
\node[Kanji] at (52.150000, -33.300000) {\textcolor[HTML]{0e254c}{葵}};
\node[Onyomi] at (52.200000, -33.700000) {キ};
\node[Kunyomi] at (52.100000, -33.700000) {あおい};
\node[Meaning] at (52.150000, -32.050000) {hollyhock};
\node[Square] at (54.200000, -33.800000) {};
\node[Kanji] at (54.200000, -33.300000) {\textcolor[HTML]{113066}{葛}};
\node[Onyomi] at (54.250000, -33.700000) {カツ};
\node[Kunyomi] at (54.150000, -33.700000) {くず};
\node[Meaning] at (54.200000, -32.050000) {arrowroot};
\node[Square] at (56.250000, -33.800000) {};
\node[Kanji] at (56.250000, -33.300000) {\textcolor[HTML]{14418e}{蒼}};
\node[Onyomi] at (56.300000, -33.700000) {ソウ};
\node[Kunyomi] at (56.200000, -33.700000) {あお};
\node[Meaning] at (56.250000, -32.050000) {pale};
\node[Meaning] at (-58.500000, -33.250000) {97.47\%};
\node[Square] at (-56.500000, -35.850000) {};
\node[Kanji] at (-56.500000, -35.350000) {\textcolor[HTML]{102b59}{茜}};
\node[Onyomi] at (-56.450000, -35.750000) {セン};
\node[Kunyomi] at (-56.550000, -35.750000) {あかね};
\node[Meaning] at (-56.500000, -34.100000) {red dye};
\node[Square] at (-54.450000, -35.850000) {};
\node[Kanji] at (-54.450000, -35.350000) {\textcolor[HTML]{0e254c}{菅}};
\node[Onyomi] at (-54.400000, -35.750000) {カン};
\node[Kunyomi] at (-54.500000, -35.750000) {すげ};
\node[Meaning] at (-54.450000, -34.100000) {sedge};
\node[Square] at (-52.400000, -35.850000) {};
\node[Kanji] at (-52.400000, -35.350000) {\textcolor[HTML]{102b59}{遥}};
\node[Onyomi] at (-52.350000, -35.750000) {ヨウ};
\node[Kunyomi] at (-52.450000, -35.750000) {はる};
\node[Meaning] at (-52.400000, -34.100000) {far off};
\node[Square] at (-50.350000, -35.850000) {};
\node[Kanji] at (-50.350000, -35.350000) {\textcolor[HTML]{0e254c}{遼}};
\node[Onyomi] at (-50.300000, -35.750000) {リョウ};
\node[Meaning] at (-50.350000, -34.100000) {distant};
\node[Square] at (-48.300000, -35.850000) {};
\node[Kanji] at (-48.300000, -35.350000) {\textcolor[HTML]{0e254c}{遜}};
\node[Onyomi] at (-48.250000, -35.750000) {ソン};
\node[Kunyomi] at (-48.350000, -35.750000) {したが.う};
\node[Meaning] at (-48.300000, -34.100000) {humble};
\node[Square] at (-46.250000, -35.850000) {};
\node[Kanji] at (-46.250000, -35.350000) {\textcolor[HTML]{14418e}{隙}};
\node[Onyomi] at (-46.200000, -35.750000) {ゲキ};
\node[Kunyomi] at (-46.300000, -35.750000) {すき};
\node[Meaning] at (-46.250000, -34.100000) {crevice};
\node[Square] at (-44.200000, -35.850000) {};
\node[Kanji] at (-44.200000, -35.350000) {\textcolor[HTML]{113066}{曖}};
\node[Onyomi] at (-44.150000, -35.750000) {アイ};
\node[Meaning] at (-44.200000, -34.100000) {not clear};
\node[Square] at (-42.150000, -35.850000) {};
\node[Kanji] at (-42.150000, -35.350000) {\textcolor[HTML]{113066}{昧}};
\node[Onyomi] at (-42.100000, -35.750000) {マイ};
\node[Meaning] at (-42.150000, -34.100000) {foolish};
\node[Square] at (-40.100000, -35.850000) {};
\node[Kanji] at (-40.100000, -35.350000) {\textcolor[HTML]{0e254c}{曙}};
\node[Onyomi] at (-40.050000, -35.750000) {ショ};
\node[Kunyomi] at (-40.150000, -35.750000) {あけぼの};
\node[Meaning] at (-40.100000, -34.100000) {dawn};
\node[Square] at (-38.050000, -35.850000) {};
\node[Kanji] at (-38.050000, -35.350000) {\textcolor[HTML]{0e254c}{旺}};
\node[Onyomi] at (-38.000000, -35.750000) {オウ};
\node[Meaning] at (-38.050000, -34.100000) {flourishing};
\node[Square] at (-36.000000, -35.850000) {};
\node[Kanji] at (-36.000000, -35.350000) {\textcolor[HTML]{0e254c}{腎}};
\node[Onyomi] at (-35.950000, -35.750000) {ジン};
\node[Meaning] at (-36.000000, -34.100000) {kidney};
\node[Square] at (-33.950000, -35.850000) {};
\node[Kanji] at (-33.950000, -35.350000) {\textcolor[HTML]{133c80}{股}};
\node[Onyomi] at (-33.900000, -35.750000) {コ};
\node[Kunyomi] at (-34.000000, -35.750000) {また};
\node[Meaning] at (-33.950000, -34.100000) {crotch};
\node[Square] at (-31.900000, -35.850000) {};
\node[Kanji] at (-31.900000, -35.350000) {\textcolor[HTML]{113066}{臆}};
\node[Onyomi] at (-31.850000, -35.750000) {オク};
\node[Meaning] at (-31.900000, -34.100000) {timidity};
\node[Square] at (-29.850000, -35.850000) {};
\node[Kanji] at (-29.850000, -35.350000) {\textcolor[HTML]{14469c}{膝}};
\node[Kunyomi] at (-29.900000, -35.750000) {ひざ};
\node[Meaning] at (-29.850000, -34.100000) {knee};
\node[Square] at (-27.800000, -35.850000) {};
\node[Kanji] at (-27.800000, -35.350000) {\textcolor[HTML]{14418e}{肘}};
\node[Kunyomi] at (-27.850000, -35.750000) {ひじ};
\node[Meaning] at (-27.800000, -34.100000) {elbow};
\node[Square] at (-25.750000, -35.850000) {};
\node[Kanji] at (-25.750000, -35.350000) {\textcolor[HTML]{102b59}{腺}};
\node[Onyomi] at (-25.700000, -35.750000) {セン};
\node[Meaning] at (-25.750000, -34.100000) {gland};
\node[Square] at (-23.700000, -35.850000) {};
\node[Kanji] at (-23.700000, -35.350000) {\textcolor[HTML]{14418e}{腫}};
\node[Onyomi] at (-23.650000, -35.750000) {シュ};
\node[Kunyomi] at (-23.750000, -35.750000) {は-れる};
\node[Meaning] at (-23.700000, -34.100000) {tumor};
\node[Square] at (-21.650000, -35.850000) {};
\node[Kanji] at (-21.650000, -35.350000) {\textcolor[HTML]{0e254c}{膳}};
\node[Onyomi] at (-21.600000, -35.750000) {ゼン};
\node[Meaning] at (-21.650000, -34.100000) {tray};
\node[Square] at (-19.600000, -35.850000) {};
\node[Kanji] at (-19.600000, -35.350000) {\textcolor[HTML]{113066}{胡}};
\node[Onyomi] at (-19.550000, -35.750000) {コ};
\node[Kunyomi] at (-19.650000, -35.750000) {なんぞ};
\node[Meaning] at (-19.600000, -34.100000) {barbarian};
\node[Square] at (-17.550000, -35.850000) {};
\node[Kanji] at (-17.550000, -35.350000) {\textcolor[HTML]{0e254c}{楓}};
\node[Onyomi] at (-17.500000, -35.750000) {フウ};
\node[Kunyomi] at (-17.600000, -35.750000) {かえで};
\node[Meaning] at (-17.550000, -34.100000) {maple};
\node[Square] at (-15.500000, -35.850000) {};
\node[Kanji] at (-15.500000, -35.350000) {\textcolor[HTML]{133c80}{枕}};
\node[Onyomi] at (-15.450000, -35.750000) {シン};
\node[Kunyomi] at (-15.550000, -35.750000) {まくら};
\node[Meaning] at (-15.500000, -34.100000) {pillow};
\node[Square] at (-13.450000, -35.850000) {};
\node[Kanji] at (-13.450000, -35.350000) {\textcolor[HTML]{1551b8}{椅}};
\node[Onyomi] at (-13.400000, -35.750000) {イ};
\node[Meaning] at (-13.450000, -34.100000) {chair};
\node[Square] at (-11.400000, -35.850000) {};
\node[Kanji] at (-11.400000, -35.350000) {\textcolor[HTML]{0e254c}{柿}};
\node[Kunyomi] at (-11.450000, -35.750000) {かき};
\node[Meaning] at (-11.400000, -34.100000) {persimmon};
\node[Square] at (-9.350000, -35.850000) {};
\node[Kanji] at (-9.350000, -35.350000) {\textcolor[HTML]{0e254c}{桁}};
\node[Kunyomi] at (-9.400000, -35.750000) {けた};
\node[Meaning] at (-9.350000, -34.100000) {beam};
\node[Square] at (-7.300000, -35.850000) {};
\node[Kanji] at (-7.300000, -35.350000) {\textcolor[HTML]{0e254c}{梗}};
\node[Onyomi] at (-7.250000, -35.750000) {コウ};
\node[Meaning] at (-7.300000, -34.100000) {close up};
\node[Square] at (-5.250000, -35.850000) {};
\node[Kanji] at (-5.250000, -35.350000) {\textcolor[HTML]{0e254c}{椎}};
\node[Onyomi] at (-5.200000, -35.750000) {ツイ};
\node[Kunyomi] at (-5.300000, -35.750000) {う};
\node[Meaning] at (-5.250000, -34.100000) {oak};
\node[Square] at (-3.200000, -35.850000) {};
\node[Kanji] at (-3.200000, -35.350000) {\textcolor[HTML]{133c80}{柵}};
\node[Onyomi] at (-3.150000, -35.750000) {サク};
\node[Meaning] at (-3.200000, -34.100000) {fence};
\node[Square] at (-1.150000, -35.850000) {};
\node[Kanji] at (-1.150000, -35.350000) {\textcolor[HTML]{0e254c}{栞}};
\node[Onyomi] at (-1.100000, -35.750000) {カン};
\node[Kunyomi] at (-1.200000, -35.750000) {しおり};
\node[Meaning] at (-1.150000, -34.100000) {bookmark};
\node[Square] at (0.900000, -35.850000) {};
\node[Kanji] at (0.900000, -35.350000) {\textcolor[HTML]{123673}{煎}};
\node[Onyomi] at (0.950000, -35.750000) {セン};
\node[Kunyomi] at (0.850000, -35.750000) {い-る};
\node[Meaning] at (0.900000, -34.100000) {broil};
\node[Square] at (2.950000, -35.850000) {};
\node[Kanji] at (2.950000, -35.350000) {\textcolor[HTML]{0e254c}{瑠}};
\node[Onyomi] at (3.000000, -35.750000) {ル};
\node[Meaning] at (2.950000, -34.100000) {lapis lazuli};
\node[Square] at (5.000000, -35.850000) {};
\node[Kanji] at (5.000000, -35.350000) {\textcolor[HTML]{102b59}{斑}};
\node[Onyomi] at (5.050000, -35.750000) {ハン};
\node[Meaning] at (5.000000, -34.100000) {blemish};
\node[Square] at (7.050000, -35.850000) {};
\node[Kanji] at (7.050000, -35.350000) {\textcolor[HTML]{0e254c}{弄}};
\node[Onyomi] at (7.100000, -35.750000) {ロウ};
\node[Kunyomi] at (7.000000, -35.750000) {もてあそ-ぶ};
\node[Meaning] at (7.050000, -34.100000) {tamper with};
\node[Square] at (9.100000, -35.850000) {};
\node[Kanji] at (9.100000, -35.350000) {\textcolor[HTML]{0e254c}{瑞}};
\node[Onyomi] at (9.150000, -35.750000) {スイ};
\node[Kunyomi] at (9.050000, -35.750000) {みず};
\node[Meaning] at (9.100000, -34.100000) {congratulations};
\node[Square] at (11.150000, -35.850000) {};
\node[Kanji] at (11.150000, -35.350000) {\textcolor[HTML]{0e254c}{瑛}};
\node[Onyomi] at (11.200000, -35.750000) {エイ};
\node[Meaning] at (11.150000, -34.100000) {crystal};
\node[Square] at (13.200000, -35.850000) {};
\node[Kanji] at (13.200000, -35.350000) {\textcolor[HTML]{0e254c}{玩}};
\node[Onyomi] at (13.250000, -35.750000) {ガン};
\node[Meaning] at (13.200000, -34.100000) {trifle with};
\node[Square] at (15.250000, -35.850000) {};
\node[Kanji] at (15.250000, -35.350000) {\textcolor[HTML]{113066}{畏}};
\node[Onyomi] at (15.300000, -35.750000) {イ};
\node[Kunyomi] at (15.200000, -35.750000) {おそ-れる};
\node[Meaning] at (15.250000, -34.100000) {fear};
\node[Square] at (17.300000, -35.850000) {};
\node[Kanji] at (17.300000, -35.350000) {\textcolor[HTML]{133c80}{痩}};
\node[Onyomi] at (17.350000, -35.750000) {ソウ};
\node[Kunyomi] at (17.250000, -35.750000) {や-せる};
\node[Meaning] at (17.300000, -34.100000) {get thin};
\node[Square] at (19.350000, -35.850000) {};
\node[Kanji] at (19.350000, -35.350000) {\textcolor[HTML]{14469c}{痕}};
\node[Onyomi] at (19.400000, -35.750000) {コン};
\node[Kunyomi] at (19.300000, -35.750000) {あと};
\node[Meaning] at (19.350000, -34.100000) {mark};
\node[Square] at (21.400000, -35.850000) {};
\node[Kanji] at (21.400000, -35.350000) {\textcolor[HTML]{0e254c}{瞭}};
\node[Onyomi] at (21.450000, -35.750000) {リョウ};
\node[Meaning] at (21.400000, -34.100000) {clear};
\node[Square] at (23.450000, -35.850000) {};
\node[Kanji] at (23.450000, -35.350000) {\textcolor[HTML]{14418e}{眉}};
\node[Onyomi] at (23.500000, -35.750000) {ビ};
\node[Kunyomi] at (23.400000, -35.750000) {まゆ};
\node[Meaning] at (23.450000, -34.100000) {eyebrow};
\node[Square] at (25.500000, -35.850000) {};
\node[Kanji] at (25.500000, -35.350000) {\textcolor[HTML]{123673}{窟}};
\node[Onyomi] at (25.550000, -35.750000) {クツ};
\node[Meaning] at (25.500000, -34.100000) {cavern};
\node[Square] at (27.550000, -35.850000) {};
\node[Kanji] at (27.550000, -35.350000) {\textcolor[HTML]{0e254c}{颯}};
\node[Onyomi] at (27.600000, -35.750000) {サツ};
\node[Kunyomi] at (27.500000, -35.750000) {さっ.と};
\node[Meaning] at (27.550000, -34.100000) {quick};
\node[Square] at (29.600000, -35.850000) {};
\node[Kanji] at (29.600000, -35.350000) {\textcolor[HTML]{0e254c}{靖}};
\node[Onyomi] at (29.650000, -35.750000) {ジョウ};
\node[Kunyomi] at (29.550000, -35.750000) {やす};
\node[Meaning] at (29.600000, -34.100000) {peaceful};
\node[Square] at (31.650000, -35.850000) {};
\node[Kanji] at (31.650000, -35.350000) {\textcolor[HTML]{123673}{裾}};
\node[Kunyomi] at (31.600000, -35.750000) {すそ};
\node[Meaning] at (31.650000, -34.100000) {cuff};
\node[Square] at (33.700000, -35.850000) {};
\node[Kanji] at (33.700000, -35.350000) {\textcolor[HTML]{113066}{箸}};
\node[Onyomi] at (33.750000, -35.750000) {チャク};
\node[Kunyomi] at (33.650000, -35.750000) {はし};
\node[Meaning] at (33.700000, -34.100000) {chopsticks};
\node[Square] at (35.750000, -35.850000) {};
\node[Kanji] at (35.750000, -35.350000) {\textcolor[HTML]{0e254c}{緋}};
\node[Onyomi] at (35.800000, -35.750000) {ヒ};
\node[Kunyomi] at (35.700000, -35.750000) {あか};
\node[Meaning] at (35.750000, -34.100000) {scarlet};
\node[Square] at (37.800000, -35.850000) {};
\node[Kanji] at (37.800000, -35.350000) {\textcolor[HTML]{0e254c}{綺}};
\node[Onyomi] at (37.850000, -35.750000) {キ};
\node[Meaning] at (37.800000, -34.100000) {beautiful};
\node[Square] at (39.850000, -35.850000) {};
\node[Kanji] at (39.850000, -35.350000) {\textcolor[HTML]{0e254c}{綾}};
\node[Onyomi] at (39.900000, -35.750000) {リン};
\node[Kunyomi] at (39.800000, -35.750000) {あや};
\node[Meaning] at (39.850000, -34.100000) {design};
\node[Square] at (41.900000, -35.850000) {};
\node[Kanji] at (41.900000, -35.350000) {\textcolor[HTML]{0e254c}{綻}};
\node[Onyomi] at (41.950000, -35.750000) {タン};
\node[Kunyomi] at (41.850000, -35.750000) {ほころ-びる};
\node[Meaning] at (41.900000, -34.100000) {rip};
\node[Square] at (43.950000, -35.850000) {};
\node[Kanji] at (43.950000, -35.350000) {\textcolor[HTML]{0e254c}{舷}};
\node[Onyomi] at (44.000000, -35.750000) {ゲン};
\node[Meaning] at (43.950000, -34.100000) {gunwale};
\node[Square] at (46.000000, -35.850000) {};
\node[Kanji] at (46.000000, -35.350000) {\textcolor[HTML]{102b59}{聡}};
\node[Onyomi] at (46.050000, -35.750000) {ソウ};
\node[Kunyomi] at (45.950000, -35.750000) {さと.い};
\node[Meaning] at (46.000000, -34.100000) {wise};
\node[Square] at (48.050000, -35.850000) {};
\node[Kanji] at (48.050000, -35.350000) {\textcolor[HTML]{0e254c}{蟹}};
\node[Kunyomi] at (48.000000, -35.750000) {かに};
\node[Meaning] at (48.050000, -34.100000) {crab};
\node[Square] at (50.100000, -35.850000) {};
\node[Kanji] at (50.100000, -35.350000) {\textcolor[HTML]{123673}{蜂}};
\node[Onyomi] at (50.150000, -35.750000) {ホウ};
\node[Kunyomi] at (50.050000, -35.750000) {はち};
\node[Meaning] at (50.100000, -34.100000) {bee};
\node[Square] at (52.150000, -35.850000) {};
\node[Kanji] at (52.150000, -35.350000) {\textcolor[HTML]{113066}{罵}};
\node[Onyomi] at (52.200000, -35.750000) {バ};
\node[Kunyomi] at (52.100000, -35.750000) {ののし-る};
\node[Meaning] at (52.150000, -34.100000) {insult};
\node[Square] at (54.200000, -35.850000) {};
\node[Kanji] at (54.200000, -35.350000) {\textcolor[HTML]{0e254c}{戴}};
\node[Onyomi] at (54.250000, -35.750000) {タイ};
\node[Kunyomi] at (54.150000, -35.750000) {いただ};
\node[Meaning] at (54.200000, -34.100000) {receive};
\node[Square] at (56.250000, -35.850000) {};
\node[Kanji] at (56.250000, -35.350000) {\textcolor[HTML]{0e254c}{哉}};
\node[Onyomi] at (56.300000, -35.750000) {サイ};
\node[Kunyomi] at (56.200000, -35.750000) {や};
\node[Meaning] at (56.250000, -34.100000) {question mark};
\node[Meaning] at (-58.500000, -35.300000) {97.72\%};
\node[Square] at (-56.500000, -37.900000) {};
\node[Kanji] at (-56.500000, -37.400000) {\textcolor[HTML]{133c80}{謎}};
\node[Kunyomi] at (-56.550000, -37.800000) {なぞ};
\node[Meaning] at (-56.500000, -36.150000) {riddle};
\node[Square] at (-54.450000, -37.900000) {};
\node[Kanji] at (-54.450000, -37.400000) {\textcolor[HTML]{0e254c}{諒}};
\node[Onyomi] at (-54.400000, -37.800000) {リョウ};
\node[Kunyomi] at (-54.500000, -37.800000) {あきら.か};
\node[Meaning] at (-54.450000, -36.150000) {comprehend};
\node[Square] at (-52.400000, -37.900000) {};
\node[Kanji] at (-52.400000, -37.400000) {\textcolor[HTML]{145cd5}{誰}};
\node[Kunyomi] at (-52.450000, -37.800000) {だれ};
\node[Meaning] at (-52.400000, -36.150000) {who};
\node[Square] at (-50.350000, -37.900000) {};
\node[Kanji] at (-50.350000, -37.400000) {\textcolor[HTML]{0e254c}{詣}};
\node[Onyomi] at (-50.300000, -37.800000) {ケイ};
\node[Kunyomi] at (-50.400000, -37.800000) {もう-でる};
\node[Meaning] at (-50.350000, -36.150000) {visit a temple};
\node[Square] at (-48.300000, -37.900000) {};
\node[Kanji] at (-48.300000, -37.400000) {\textcolor[HTML]{133c80}{諦}};
\node[Onyomi] at (-48.250000, -37.800000) {テイ};
\node[Kunyomi] at (-48.350000, -37.800000) {あきら-める};
\node[Meaning] at (-48.300000, -36.150000) {abandon};
\node[Square] at (-46.250000, -37.900000) {};
\node[Kanji] at (-46.250000, -37.400000) {\textcolor[HTML]{113066}{詮}};
\node[Onyomi] at (-46.200000, -37.800000) {セン};
\node[Meaning] at (-46.250000, -36.150000) {discussion};
\node[Square] at (-44.200000, -37.900000) {};
\node[Kanji] at (-44.200000, -37.400000) {\textcolor[HTML]{0e254c}{輔}};
\node[Onyomi] at (-44.150000, -37.800000) {フ};
\node[Kunyomi] at (-44.250000, -37.800000) {たす.ける};
\node[Meaning] at (-44.200000, -36.150000) {help};
\node[Square] at (-42.150000, -37.900000) {};
\node[Kanji] at (-42.150000, -37.400000) {\textcolor[HTML]{102b59}{貌}};
\node[Onyomi] at (-42.100000, -37.800000) {ボウ};
\node[Meaning] at (-42.150000, -36.150000) {appearance};
\node[Square] at (-40.100000, -37.900000) {};
\node[Kanji] at (-40.100000, -37.400000) {\textcolor[HTML]{14418e}{貼}};
\node[Onyomi] at (-40.050000, -37.800000) {チョウ};
\node[Kunyomi] at (-40.150000, -37.800000) {は};
\node[Meaning] at (-40.100000, -36.150000) {paste};
\node[Square] at (-38.050000, -37.900000) {};
\node[Kanji] at (-38.050000, -37.400000) {\textcolor[HTML]{0e254c}{賂}};
\node[Onyomi] at (-38.000000, -37.800000) {ロ};
\node[Meaning] at (-38.050000, -36.150000) {bribe};
\node[Square] at (-36.000000, -37.900000) {};
\node[Kanji] at (-36.000000, -37.400000) {\textcolor[HTML]{14418e}{蹴}};
\node[Onyomi] at (-35.950000, -37.800000) {シュウ};
\node[Kunyomi] at (-36.050000, -37.800000) {け-る};
\node[Meaning] at (-36.000000, -36.150000) {kick};
\node[Square] at (-33.950000, -37.900000) {};
\node[Kanji] at (-33.950000, -37.400000) {\textcolor[HTML]{0e254c}{醤}};
\node[Onyomi] at (-33.900000, -37.800000) {ショウ};
\node[Meaning] at (-33.950000, -36.150000) {soy sauce};
\node[Square] at (-31.900000, -37.900000) {};
\node[Kanji] at (-31.900000, -37.400000) {\textcolor[HTML]{0e254c}{酎}};
\node[Onyomi] at (-31.850000, -37.800000) {チュウ};
\node[Kunyomi] at (-31.950000, -37.800000) {かも.す};
\node[Meaning] at (-31.900000, -36.150000) {sake};
\node[Square] at (-29.850000, -37.900000) {};
\node[Kanji] at (-29.850000, -37.400000) {\textcolor[HTML]{0e254c}{醒}};
\node[Onyomi] at (-29.800000, -37.800000) {セイ};
\node[Meaning] at (-29.850000, -36.150000) {disillusioned};
\node[Square] at (-27.800000, -37.900000) {};
\node[Kanji] at (-27.800000, -37.400000) {\textcolor[HTML]{0e254c}{麺}};
\node[Onyomi] at (-27.750000, -37.800000) {メン};
\node[Meaning] at (-27.800000, -36.150000) {noodles};
\node[Square] at (-25.750000, -37.900000) {};
\node[Kanji] at (-25.750000, -37.400000) {\textcolor[HTML]{14469c}{鍋}};
\node[Kunyomi] at (-25.800000, -37.800000) {なべ};
\node[Meaning] at (-25.750000, -36.150000) {pot};
\node[Square] at (-23.700000, -37.900000) {};
\node[Kanji] at (-23.700000, -37.400000) {\textcolor[HTML]{14469c}{鍵}};
\node[Onyomi] at (-23.650000, -37.800000) {ケン};
\node[Kunyomi] at (-23.750000, -37.800000) {かぎ};
\node[Meaning] at (-23.700000, -36.150000) {key};
\node[Square] at (-21.650000, -37.900000) {};
\node[Kanji] at (-21.650000, -37.400000) {\textcolor[HTML]{1557c6}{闇}};
\node[Onyomi] at (-21.600000, -37.800000) {アン};
\node[Kunyomi] at (-21.700000, -37.800000) {やみ};
\node[Meaning] at (-21.650000, -36.150000) {darkness};
\node[Square] at (-19.600000, -37.900000) {};
\node[Kanji] at (-19.600000, -37.400000) {\textcolor[HTML]{102b59}{頓}};
\node[Onyomi] at (-19.550000, -37.800000) {トン};
\node[Meaning] at (-19.600000, -36.150000) {suddenly};
\node[Square] at (-17.550000, -37.900000) {};
\node[Kanji] at (-17.550000, -37.400000) {\textcolor[HTML]{14418e}{頃}};
\node[Kunyomi] at (-17.600000, -37.800000) {ころ};
\node[Meaning] at (-17.550000, -36.150000) {approximate};
\node[Square] at (-15.500000, -37.900000) {};
\node[Kanji] at (-15.500000, -37.400000) {\textcolor[HTML]{14418e}{頬}};
\node[Kunyomi] at (-15.550000, -37.800000) {ほお};
\node[Meaning] at (-15.500000, -36.150000) {cheek};
\node[Square] at (-13.450000, -37.900000) {};
\node[Kanji] at (-13.450000, -37.400000) {\textcolor[HTML]{14418e}{顎}};
\node[Onyomi] at (-13.400000, -37.800000) {ガク};
\node[Kunyomi] at (-13.500000, -37.800000) {あご};
\node[Meaning] at (-13.450000, -36.150000) {jaw};
\node[Square] at (-11.400000, -37.900000) {};
\node[Kanji] at (-11.400000, -37.400000) {\textcolor[HTML]{133c80}{餌}};
\node[Onyomi] at (-11.350000, -37.800000) {ジ};
\node[Kunyomi] at (-11.450000, -37.800000) {えさ};
\node[Meaning] at (-11.400000, -36.150000) {bait};
\node[Square] at (-9.350000, -37.900000) {};
\node[Kanji] at (-9.350000, -37.400000) {\textcolor[HTML]{113066}{餅}};
\node[Onyomi] at (-9.300000, -37.800000) {ヘイ};
\node[Kunyomi] at (-9.400000, -37.800000) {もち};
\node[Meaning] at (-9.350000, -36.150000) {mochi};
\node[Square] at (-7.300000, -37.900000) {};
\node[Kanji] at (-7.300000, -37.400000) {\textcolor[HTML]{0e254c}{鰐}};
\node[Kunyomi] at (-7.350000, -37.800000) {わに};
\node[Meaning] at (-7.300000, -36.150000) {alligator};
\node[Square] at (-5.250000, -37.900000) {};
\node[Kanji] at (-5.250000, -37.400000) {\textcolor[HTML]{0e254c}{麓}};
\node[Onyomi] at (-5.200000, -37.800000) {ロク};
\node[Kunyomi] at (-5.300000, -37.800000) {ふもと};
\node[Meaning] at (-5.250000, -36.150000) {foothills};
\node[Square] at (-3.200000, -37.900000) {};
\node[Kanji] at (-3.200000, -37.400000) {\textcolor[HTML]{113066}{冥}};
\node[Onyomi] at (-3.150000, -37.800000) {メイ};
\node[Meaning] at (-3.200000, -36.150000) {dark};
\node[Square] at (-1.150000, -37.900000) {};
\node[Kanji] at (-1.150000, -37.400000) {\textcolor[HTML]{113066}{挫}};
\node[Onyomi] at (-1.100000, -37.800000) {ザ};
\node[Meaning] at (-1.150000, -36.150000) {sprain};
\node[Square] at (0.900000, -37.900000) {};
\node[Kanji] at (0.900000, -37.400000) {\textcolor[HTML]{0e254c}{遡}};
\node[Onyomi] at (0.950000, -37.800000) {ソ};
\node[Kunyomi] at (0.850000, -37.800000) {さかのぼ-る};
\node[Meaning] at (0.900000, -36.150000) {go upstream};
\node[Square] at (2.950000, -37.900000) {};
\node[Kanji] at (2.950000, -37.400000) {\textcolor[HTML]{0e254c}{嘉}};
\node[Onyomi] at (3.000000, -37.800000) {カ};
\node[Kunyomi] at (2.900000, -37.800000) {よい};
\node[Meaning] at (2.950000, -36.150000) {esteem};
\node[Square] at (5.000000, -37.900000) {};
\node[Kanji] at (5.000000, -37.400000) {\textcolor[HTML]{102b59}{爽}};
\node[Onyomi] at (5.050000, -37.800000) {ソウ};
\node[Kunyomi] at (4.950000, -37.800000) {さわ};
\node[Meaning] at (5.000000, -36.150000) {refreshing};
\node[Square] at (7.050000, -37.900000) {};
\node[Kanji] at (7.050000, -37.400000) {\textcolor[HTML]{0e254c}{勃}};
\node[Onyomi] at (7.100000, -37.800000) {ボツ};
\node[Meaning] at (7.050000, -36.150000) {rise};
\node[Square] at (9.100000, -37.900000) {};
\node[Kanji] at (9.100000, -37.400000) {\textcolor[HTML]{14418e}{骸}};
\node[Onyomi] at (9.150000, -37.800000) {ガイ};
\node[Meaning] at (9.100000, -36.150000) {dead remains};
\node[Square] at (11.150000, -37.900000) {};
\node[Kanji] at (11.150000, -37.400000) {\textcolor[HTML]{0e254c}{隼}};
\node[Onyomi] at (11.200000, -37.800000) {シュン};
\node[Kunyomi] at (11.100000, -37.800000) {はやぶさ};
\node[Meaning] at (11.150000, -36.150000) {falcon};
\node[Square] at (13.200000, -37.900000) {};
\node[Kanji] at (13.200000, -37.400000) {\textcolor[HTML]{123673}{戚}};
\node[Onyomi] at (13.250000, -37.800000) {セキ};
\node[Meaning] at (13.200000, -36.150000) {grieve};
\node[Square] at (15.250000, -37.900000) {};
\node[Kanji] at (15.250000, -37.400000) {\textcolor[HTML]{113066}{丼}};
\node[Onyomi] at (15.300000, -37.800000) {ドン};
\node[Kunyomi] at (15.200000, -37.800000) {どんぶり};
\node[Meaning] at (15.250000, -36.150000) {rice bowl};
\node[Square] at (17.300000, -37.900000) {};
\node[Kanji] at (17.300000, -37.400000) {\textcolor[HTML]{102b59}{畿}};
\node[Onyomi] at (17.350000, -37.800000) {キ};
\node[Meaning] at (17.300000, -36.150000) {capital};
\node[Square] at (19.350000, -37.900000) {};
\node[Kanji] at (19.350000, -37.400000) {\textcolor[HTML]{0e254c}{斐}};
\node[Onyomi] at (19.400000, -37.800000) {イ};
\node[Meaning] at (19.350000, -36.150000) {patterned};
\node[Square] at (21.400000, -37.900000) {};
\node[Kanji] at (21.400000, -37.400000) {\textcolor[HTML]{14418e}{拳}};
\node[Onyomi] at (21.450000, -37.800000) {ケン};
\node[Kunyomi] at (21.350000, -37.800000) {こぶし};
\node[Meaning] at (21.400000, -36.150000) {fist};
\node[Square] at (23.450000, -37.900000) {};
\node[Kanji] at (23.450000, -37.400000) {\textcolor[HTML]{0e254c}{亮}};
\node[Onyomi] at (23.500000, -37.800000) {リョウ};
\node[Kunyomi] at (23.400000, -37.800000) {あきらか        };
\node[Meaning] at (23.450000, -36.150000) {clear};
\node[Square] at (25.500000, -37.900000) {};
\node[Kanji] at (25.500000, -37.400000) {\textcolor[HTML]{113066}{阜}};
\node[Onyomi] at (25.550000, -37.800000) {フ};
\node[Meaning] at (25.500000, -36.150000) {mound};
\node[Square] at (27.550000, -37.900000) {};
\node[Kanji] at (27.550000, -37.400000) {\textcolor[HTML]{102b59}{翔}};
\node[Onyomi] at (27.600000, -37.800000) {ショウ};
\node[Kunyomi] at (27.500000, -37.800000) {かけ};
\node[Meaning] at (27.550000, -36.150000) {fly};
\node[Square] at (29.600000, -37.900000) {};
\node[Kanji] at (29.600000, -37.400000) {\textcolor[HTML]{133c80}{那}};
\node[Onyomi] at (29.650000, -37.800000) {ナ};
\node[Kunyomi] at (29.550000, -37.800000) {いかん};
\node[Meaning] at (29.600000, -36.150000) {what};
\node[Square] at (31.650000, -37.900000) {};
\node[Kanji] at (31.650000, -37.400000) {\textcolor[HTML]{102b59}{龍}};
\node[Onyomi] at (31.700000, -37.800000) {リュウ};
\node[Kunyomi] at (31.600000, -37.800000) {たつ};
\node[Meaning] at (31.650000, -36.150000) {imperial};
\node[Square] at (33.700000, -37.900000) {};
\node[Kanji] at (33.700000, -37.400000) {\textcolor[HTML]{0e254c}{箋}};
\node[Onyomi] at (33.750000, -37.800000) {セン};
\node[Meaning] at (33.700000, -36.150000) {paper};
\node[Square] at (35.750000, -37.900000) {};
\node[Kanji] at (35.750000, -37.400000) {\textcolor[HTML]{0e254c}{彙}};
\node[Onyomi] at (35.800000, -37.800000) {イ};
\node[Meaning] at (35.750000, -36.150000) {same kind};
\node[Square] at (37.800000, -37.900000) {};
\node[Kanji] at (37.800000, -37.400000) {\textcolor[HTML]{14418e}{籠}};
\node[Onyomi] at (37.850000, -37.800000) {ロウ};
\node[Kunyomi] at (37.750000, -37.800000) {かご};
\node[Meaning] at (37.800000, -36.150000) {basket};
\node[Square] at (39.850000, -37.900000) {};
\node[Kanji] at (39.850000, -37.400000) {\textcolor[HTML]{133c80}{嗅}};
\node[Onyomi] at (39.900000, -37.800000) {キュウ};
\node[Kunyomi] at (39.800000, -37.800000) {か-ぐ};
\node[Meaning] at (39.850000, -36.150000) {smell};
\node[Square] at (41.900000, -37.900000) {};
\node[Kanji] at (41.900000, -37.400000) {\textcolor[HTML]{133c80}{璧}};
\node[Onyomi] at (41.950000, -37.800000) {ヘキ};
\node[Meaning] at (41.900000, -36.150000) {sphere};
\node[Square] at (43.950000, -37.900000) {};
\node[Kanji] at (43.950000, -37.400000) {\textcolor[HTML]{123673}{鬱}};
\node[Onyomi] at (44.000000, -37.800000) {ウツ};
\node[Meaning] at (43.950000, -36.150000) {gloom};
\node[Square] at (46.000000, -37.900000) {};
\node[Kanji] at (46.000000, -37.400000) {\textcolor[HTML]{113066}{傲}};
\node[Onyomi] at (46.050000, -37.800000) {ゴウ};
\node[Kunyomi] at (45.950000, -37.800000) {あなど};
\node[Meaning] at (46.000000, -36.150000) {proud};
\node[Square] at (48.050000, -37.900000) {};
\node[Kanji] at (48.050000, -37.400000) {\textcolor[HTML]{113066}{拉}};
\node[Onyomi] at (48.100000, -37.800000) {ラ};
\node[Meaning] at (48.050000, -36.150000) {crush};
\node[Square] at (50.100000, -37.900000) {};
\node[Kanji] at (50.100000, -37.400000) {\textcolor[HTML]{102b59}{踪}};
\node[Onyomi] at (50.150000, -37.800000) {ソウ};
\node[Meaning] at (50.100000, -36.150000) {remains};
\node[Square] at (52.150000, -37.900000) {};
\node[Kanji] at (52.150000, -37.400000) {\textcolor[HTML]{102b59}{貪}};
\node[Onyomi] at (52.200000, -37.800000) {ドン};
\node[Kunyomi] at (52.100000, -37.800000) {むさぼ-る};
\node[Meaning] at (52.150000, -36.150000) {covet};
\node[Square] at (54.200000, -37.900000) {};
\node[Kanji] at (54.200000, -37.400000) {\textcolor[HTML]{102b59}{辣}};
\node[Onyomi] at (54.250000, -37.800000) {ラツ};
\node[Meaning] at (54.200000, -36.150000) {bitter};
\node[Square] at (56.250000, -37.900000) {};
\node[Kanji] at (56.250000, -37.400000) {\textcolor[HTML]{0e254c}{慄}};
\node[Onyomi] at (56.300000, -37.800000) {リツ};
\node[Meaning] at (56.250000, -36.150000) {fear};
\node[Meaning] at (-58.500000, -37.350000) {98.30\%};
\node[Square] at (-56.500000, -39.950000) {};
\node[Kanji] at (-56.500000, -39.450000) {\textcolor[HTML]{0e254c}{惧}};
\node[Onyomi] at (-56.450000, -39.850000) {グ};
\node[Meaning] at (-56.500000, -38.200000) {dread};
\node[Square] at (-54.450000, -39.950000) {};
\node[Kanji] at (-54.450000, -39.450000) {\textcolor[HTML]{0e254c}{贅}};
\node[Onyomi] at (-54.400000, -39.850000) {ゼイ};
\node[Kunyomi] at (-54.500000, -39.850000) {いぼ};
\node[Meaning] at (-54.450000, -38.200000) {luxury};
\node[Square] at (-52.400000, -39.950000) {};
\node[Kanji] at (-52.400000, -39.450000) {\textcolor[HTML]{0e254c}{漣}};
\node[Onyomi] at (-52.350000, -39.850000) {レン};
\node[Kunyomi] at (-52.450000, -39.850000) {さざなみ};
\node[Meaning] at (-52.400000, -38.200000) {ripples};
\node[Square] at (-50.350000, -39.950000) {};
\node[Kanji] at (-50.350000, -39.450000) {\textcolor[HTML]{0e254c}{哺}};
\node[Onyomi] at (-50.300000, -39.850000) {ホ};
\node[Kunyomi] at (-50.400000, -39.850000) {ほぐく};
\node[Meaning] at (-50.350000, -38.200000) {nurse};
\node[Square] at (-48.300000, -39.950000) {};
\node[Kanji] at (-48.300000, -39.450000) {\textcolor[HTML]{0e254c}{墟}};
\node[Onyomi] at (-48.250000, -39.850000) {キョ};
\node[Meaning] at (-48.300000, -38.200000) {ruins};
\node[Square] at (-46.250000, -39.950000) {};
\node[Kanji] at (-46.250000, -39.450000) {\textcolor[HTML]{0e254c}{訃}};
\node[Onyomi] at (-46.200000, -39.850000) {フ};
\node[Meaning] at (-46.250000, -38.200000) {obituary};
\node[Square] at (-44.200000, -39.950000) {};
\node[Kanji] at (-44.200000, -39.450000) {\textcolor[HTML]{0e254c}{緻}};
\node[Onyomi] at (-44.150000, -39.850000) {チ};
\node[Meaning] at (-44.200000, -38.200000) {fine};
\node[Square] at (-42.150000, -39.950000) {};
\node[Kanji] at (-42.150000, -39.450000) {\textcolor[HTML]{0e254c}{摯}};
\node[Onyomi] at (-42.100000, -39.850000) {シ};
\node[Meaning] at (-42.150000, -38.200000) {seriousness};
\node[Square] at (-40.100000, -39.950000) {};
\node[Kanji] at (-40.100000, -39.450000) {\textcolor[HTML]{0e254c}{楷}};
\node[Onyomi] at (-40.050000, -39.850000) {カイ};
\node[Meaning] at (-40.100000, -38.200000) {printed style};
\node[Square] at (-38.050000, -39.950000) {};
\node[Kanji] at (-38.050000, -39.450000) {\textcolor[HTML]{0e254c}{諧}};
\node[Onyomi] at (-38.000000, -39.850000) {カイ};
\node[Meaning] at (-38.050000, -38.200000) {harmony};
\node[Square] at (-36.000000, -39.950000) {};
\node[Kanji] at (-36.000000, -39.450000) {\textcolor[HTML]{0e254c}{憬}};
\node[Onyomi] at (-35.950000, -39.850000) {ケイ};
\node[Meaning] at (-36.000000, -38.200000) {long for};
\node[Square] at (-33.950000, -39.950000) {};
\node[Kanji] at (-33.950000, -39.450000) {\textcolor[HTML]{0e254c}{錮}};
\node[Onyomi] at (-33.900000, -39.850000) {コ};
\node[Meaning] at (-33.950000, -38.200000) {tie up};
\node[Square] at (-31.900000, -39.950000) {};
\node[Kanji] at (-31.900000, -39.450000) {\textcolor[HTML]{0e254c}{恣}};
\node[Onyomi] at (-31.850000, -39.850000) {シ};
\node[Meaning] at (-31.900000, -38.200000) {selfish};
\node[Square] at (-29.850000, -39.950000) {};
\node[Kanji] at (-29.850000, -39.450000) {\textcolor[HTML]{0e254c}{羞}};
\node[Onyomi] at (-29.800000, -39.850000) {シュウ};
\node[Meaning] at (-29.850000, -38.200000) {feel ashamed};
\node[Square] at (-27.800000, -39.950000) {};
\node[Kanji] at (-27.800000, -39.450000) {\textcolor[HTML]{0e254c}{喩}};
\node[Onyomi] at (-27.750000, -39.850000) {ユ};
\node[Meaning] at (-27.800000, -38.200000) {metaphor};
\node[Square] at (-25.750000, -39.950000) {};
\node[Kanji] at (-25.750000, -39.450000) {\textcolor[HTML]{0e254c}{瘍}};
\node[Onyomi] at (-25.700000, -39.850000) {ヨウ};
\node[Meaning] at (-25.750000, -38.200000) {boil (medical)};
\node[Meaning] at (-58.500000, -39.400000) {98.31\%};
\node[Meaning] at (-58.500000, 41.150000) {1 - 56};
\node[Meaning] at (-58.500000, 39.100000) {57 - 112};
\node[Meaning] at (-58.500000, 37.050000) {113 - 168};
\node[Meaning] at (-58.500000, 35.000000) {169 - 224};
\node[Meaning] at (-58.500000, 32.950000) {225 - 280};
\node[Meaning] at (-58.500000, 30.900000) {281 - 336};
\node[Meaning] at (-58.500000, 28.850000) {337 - 392};
\node[Meaning] at (-58.500000, 26.800000) {393 - 448};
\node[Meaning] at (-58.500000, 24.750000) {449 - 504};
\node[Meaning] at (-58.500000, 22.700000) {505 - 560};
\node[Meaning] at (-58.500000, 20.650000) {561 - 616};
\node[Meaning] at (-58.500000, 18.600000) {617 - 672};
\node[Meaning] at (-58.500000, 16.550000) {673 - 728};
\node[Meaning] at (-58.500000, 14.500000) {729 - 784};
\node[Meaning] at (-58.500000, 12.450000) {785 - 840};
\node[Meaning] at (-58.500000, 10.400000) {841 - 896};
\node[Meaning] at (-58.500000, 8.350000) {897 - 952};
\node[Meaning] at (-58.500000, 6.300000) {953 - 1008};
\node[Meaning] at (-58.500000, 4.250000) {1009 - 1064};
\node[Meaning] at (-58.500000, 2.200000) {1065 - 1120};
\node[Meaning] at (-58.500000, 0.150000) {1121 - 1176};
\node[Meaning] at (-58.500000, -1.900000) {1177 - 1232};
\node[Meaning] at (-58.500000, -3.950000) {1233 - 1288};
\node[Meaning] at (-58.500000, -6.000000) {1289 - 1344};
\node[Meaning] at (-58.500000, -8.050000) {1345 - 1400};
\node[Meaning] at (-58.500000, -10.100000) {1401 - 1456};
\node[Meaning] at (-58.500000, -12.150000) {1457 - 1512};
\node[Meaning] at (-58.500000, -14.200000) {1513 - 1568};
\node[Meaning] at (-58.500000, -16.250000) {1569 - 1624};
\node[Meaning] at (-58.500000, -18.300000) {1625 - 1680};
\node[Meaning] at (-58.500000, -20.350000) {1681 - 1736};
\node[Meaning] at (-58.500000, -22.400000) {1737 - 1792};
\node[Meaning] at (-58.500000, -24.450000) {1793 - 1848};
\node[Meaning] at (-58.500000, -26.500000) {1849 - 1904};
\node[Meaning] at (-58.500000, -28.550000) {1905 - 1960};
\node[Meaning] at (-58.500000, -30.600000) {1961 - 2016};
\node[Meaning] at (-58.500000, -32.650000) {2017 - 2072};
\node[Meaning] at (-58.500000, -34.700000) {2073 - 2128};
\node[Meaning] at (-58.500000, -36.750000) {2129 - 2184};
\node[Meaning] at (-58.500000, -38.800000) {2185 - 2240};

\end{CJK}

\node [above right,outer sep=10pt,minimum width=\paperwidth,align=center] at (bottomleft) {
2200 kanji covering 98.52\% of common Japanese text. Data from \url{https://www.wanikani.com} and \url{https://en.wikipedia.org/wiki/List_of_joyo_kanji}. Kanji colors using colormap Balance, most to least frequent: \textcolor[HTML]{3c0912}{█}\textcolor[HTML]{830e29}{█}\textcolor[HTML]{a11d25}{█}\textcolor[HTML]{b74029}{█}\textcolor[HTML]{cd8268}{█}\textcolor[HTML]{d69f8d}{█}\textcolor[HTML]{d2a293}{█}\textcolor[HTML]{c8a59d}{█}\textcolor[HTML]{a3bac2}{█}\textcolor[HTML]{91b7c3}{█}\textcolor[HTML]{68a4bc}{█}\textcolor[HTML]{408dba}{█}\textcolor[HTML]{1059be}{█}\textcolor[HTML]{29409e}{█}\textcolor[HTML]{242e6c}{█}\textcolor[HTML]{181c43}{█}
Kanji data from \url{https://www.wanikani.com} and \url{https://en.wikipedia.org/wiki/List_of_joyo_kanji}.

};

\end{document}
